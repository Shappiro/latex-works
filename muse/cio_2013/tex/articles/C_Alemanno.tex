\setcounter{figure}{0}
\setcounter{table}{0}

\begin{adjustwidth}{-3.5cm}{0cm}
\pagestyle{CIOpage}
\authortoc{Alemanno S., Ragni B.}
\chapter*[La coturnice nel Parco Nazionale dei Sibillini]{La coturnice appenninica \textbf{\textit{Alectoris
graeca}}\textbf{ nel Parco Nazionale dei Monti Sibillini}}
\addcontentsline{toc}{chapter}{La coturnice nel Parco Nazionale dei Sibillini}

\textsc{Simone Alemanno}$^{1}$, \textsc{Bernardino Ragni}$^{1*}$ \\

\index{Alemanno Simone} \index{Ragni Bernardino}
\noindent\color{MUSEBLUE}\rule{27cm}{2pt}
\vspace{1cm}
\end{adjustwidth}
\marginnote{\raggedright $^1$Dipartimento di Chimica, Biologia e
Biotecnologie, Universit\`a degli Studi di Perugia \\
\vspace{.5cm}
{\emph{\small $^*$Autore per la corrispondenza: \href{mailto:bernardino.ragni@unipg.it}{ber\allowbreak nar\allowbreak di\allowbreak no.\allowbreak ra\allowbreak gni@\allowbreak uni\allowbreak pg.\allowbreak it}}} \\
\keywords{\textit{Alectoris graeca}, Appennino Centrale, Parco
Nazionale Monti Sibillini}
{\textit{Alectoris graeca}, Central  Appennines, Monti
Sibillini National Park}
%\index{keywords}{\textit{Alectoris graeca}} \index{keywords}{Appennino Centrale} \index{keywords}{Parco Nazionale Monti Sibillini}
}
{
\small
\noindent \textsc{\color{MUSEBLUE} Summary} / The aim of this research is to define the \textit{status} (past and
present distribution, abundance, habitat and anthropogenic critical
factors) of rock partridge \textit{Alectoris graeca} in the umbrian
part of Sibillini mountains NP (Central Apennines). The species was
ascertained throughout 1970-2010 with a population density probably
conditioned by a heavy human pressure. \\
\noindent \textsc{\color{MUSEBLUE} Riassunto} / Lo studio si prefigge lo scopo di definire lo status della coturnice
\textit{Alectoris graeca} nella porzione umbra del Parco Nazionale dei
Monti Sibillini, Appennino Umbro-marchigiano, tramite la definizione
di: distribuzione pregressa e attuale, abbondanza specifica, selezione
dell{\textquoteright}habitat, presenza di fattori critici di origine
antropica. Il galliforme risulta ininterrottamente presente
nell{\textquoteright}area di studio, almeno nel quarantennio 1970-2010;
anche se l{\textquoteright}abbondanza specifica appare condizionata da
una pesante pressione antropica, la popolazione di coturnice
dell{\textquoteright}area protetta pu\`o rappresentare una importante
{\textquotedblleft}sorgente{\textquotedblright} per il circostante
{\textquotedblleft}bacino{\textquotedblright} montano dove, assieme
agli stessi fattori critici osservati nello studio, \`e presente una
diffusa attivit\`a di bracconaggio. \\
}


\section*{Introduzione}

La coturnice pu\`o essere considerata una specie bandiera (Caro
\textit{et al}. 2004) degli ambienti montani della catena appenninica.

La sua presenza in cCentro Italia risulta discontinua e localizzata in
sub-areali disgiunti (Brichetti \& Fracasso 2004) e diversi studi,
dagli anni Novanta dello scorso secolo, si sono occupati di \textit{A.
graeca }nell{\textquoteright}Appennino Umbro-marchigiano. Le ricerche,
in genere, l{\textquoteright}hanno interessata prevalentemente nella
fascia orografica d{\textquoteright}elezione, compresa tra i 1600--2200
m s.l.m. (Petretti 1985) entro la quale, nella Penisola centrale,
insistono vaste aree protette appenniniche; meno frequentemente gli
studi si sono svolti entro intervalli altitudinali inferiori.

Il presente studio triennale \`e stato portato a termine
all{\textquoteright}interno di un{\textquoteright}area protetta
nazionale, ma a una quota inferiore (compresa tra i 1168 ed i 1883 m
s.l.m.). Dal momento che a pi\`u di quindici anni
dall{\textquoteright}istituzione del Parco Nazionale dei Monti
Sibillini, la specie poteva risultare fortemente rarefatta o
addirittura scomparsa, analogamente a quanto accaduto in gran parte
dell{\textquoteright}Appennino (Magrini \& Gambaro 1997; Velatta
\textit{et al.} 2009) nel presente lavoro si \`e voluta saggiare tale
ipotesi, ricostruendo l{\textquoteright}areale pregresso della specie e
confrontandolo con quello attuale. Si sono inoltre indagati:
l{\textquoteright}abbondanza relativa della coturnice,
l{\textquoteright}eventuale scelta rispetto a fattori ambientali
predefiniti, gli eventuali fattori avversi, la presenza e la permanenza
della specie nell{\textquoteright}area.

\begin{figure}[!h]
\centering
\includegraphics[width=.8\columnwidth]{Alemanno_fig_1.png}

\caption{Ubicazione dell{\textquoteright}area di studio (ADS) nel Parco Nazionale dei Monti Sibillini (Appennino Umbro-marchigiano)}
\label{Alemanno_fig_1}
\end{figure}

\section*{Metodi}

L{\textquoteright}area di studio (ADS) riguarda l{\textquoteright}ambito
sud-occidentale del Parco, compreso nel SIC-ZPS
{\textquotedblleft}Monti Sibillini{\textquotedblright} in territorio
umbro. Essa \`e stata suddivisa in 56 celle (Unit\`a Spaziali di
Rilevamento, USR) quadrate -- di 1 km di lato -- a partire dal reticolo
UTM (Fig. \ref{Alemanno_fig_1}). L{\textquoteright}ADS \`e stata suddivisa in tre settori:
settentrionale (SS, 19 USR) centrale (SC, 21 USR) meridionale (SM, 6
USR).

Sono stati impiegati i seguenti metodi: inchiesta per intervista diretta, metodo cinegetico (MC) e
metodo naturalistico (MN).

Inchiesta: sono stati somministrati questionari a persone interessate
alla gestione fa\allowbreak u\allowbreak nis\allowbreak ti\allowbreak co-\allowbreak ve\allowbreak na\allowbreak to\allowbreak ria nel periodo precedente
l{\textquoteright}istituzione del Parco, per la determinazione dei siti
d{\textquoteright}involo e dei carnieri sui quali ricostruire
l{\textquoteright}areale pregresso di \textit{A. graeca}.

Il MC ha visto l{\textquoteright}impiego di cani da ferma addestrati per
la cerca e l{\textquoteright}individuazione dei galliformi
all{\textquoteright}interno di aree campione localizzate in ognuno dei
settori dell{\textquoteright}ADS. Tale metodo ha consentito la stima
del deme (Zunino \& Zullini 1995) post-riproduttivo e
l{\textquoteright}elaborazione dell{\textquoteright}indice cinegetico
di abbondanza della specie indagata,
l{\textquoteright}ICA\textsubscript{S} (Tab. \ref{Alemanno_tab_1}).

Il MN ha consentito il rilevamento diretto (osservazione e/o ascolto di
individui) e indiretto (osservazione e/o raccolta di feci, piume e
penne, orme e piste,
{\textquotedblleft}spollinatoi{\textquotedblright}, uova, nidi, ecc.)
della specie, percorrendo ripetutamente a piedi undici transetti (22
giornate, 167 km). L{\textquoteright}azione \`e stata integrata, in
periodo pre-riproduttivo, con la stimolazione al canto dei maschi
territoriali eventualmente presenti, tramite 65 stazioni di
emissione-ascolto, disposte sulla rete di transetti. Sui risultati
cos\`i ottenuti si basano: carte della distribuzione specifica e dei
maschi territoriali; indici specifici: di diffusione
ID\textsubscript{S}, chilometrico di abbondanza IKA\textsubscript{S},
puntiforme di abbondanza IPA\textsubscript{S }(Tab. \ref{Alemanno_tab_1}).

Tramite la rete di transetti si sono rilevate anche la presenza,
l{\textquoteright}abbondanza, la variazione nello spazio, di fattori
ecologici, favorevoli (14 categorie tra vegetazionali,
orografico-morfologiche e idriche) o avversi (6 categorie di disturbo
antropico) ritenuti significativi per la specie oggetto di studio.
Successivamente \`e stata verificata l{\textquoteright}esistenza di
correlazione tra gli indici di presenza, rilevati
nell{\textquoteright}ADS, e la
{\textquotedblleft}offerta{\textquotedblright} qualitativa e
quantitativa di tali fattori.

Quelli avversi sono stati {\textquotedblleft}pesati{\textquotedblright}
tramite punteggio ordinale di valore decrescente, basato
sull{\textquoteright}esperienza degli autori e di studiosi della specie
nell{\textquoteright}Appennino (Tab. \ref{Alemanno_tab_2}).

I dati raccolti sul campo sono stati saggiati per mezzo dei seguenti
\textit{test }statistici: \textit{Goodness of fit},
${\chi}$\textsuperscript{2}, nelle associazioni tra variabili;
Coefficiente di correlazione di Spearman, r\textsubscript{S}, nelle
relazioni tra variabili (Fowler \& Cohen 2002; Jacobs 1974).

\begin{table}[!h]
\centering
\begin{tabular}{>{\raggedright\arraybackslash}p{.2\columnwidth}>{\raggedright\arraybackslash}p{.7\columnwidth}}
\toprule
\textbf{Indice} & \textbf{Algoritmo} \\ \midrule
Indice di Diffusione specifica (IDS) & 
IDS = USRS / USRT; dove USRS: numero di celle nelle quali \`e accertata la presenza della specie S, USRT: totale delle celle esplorate; per IDS = 1, la diffusione della specie \`e la massima possibile, essa ha “saturato” tutto lo spazio disponibile, per IDS = 0, la diffusione specifica \`e nulla, essa \`e assente dall{\textquoteright}ADS \\ \midrule
Indice Cinegetico di Abbondanza specifica (ICAS) &
ICAS = NS / SI; dove: NS = numero di contatti stabiliti con la specie indipendentemente dagli individui; SI = estensione dell{\textquoteright}area effettivamente indagata espressa in km$^2$ \\ \midrule
Indice Chilometrico di Abbondanza specifica (IKAS) &
IKAS = $\sum$IPS / LT(R); dove: $\sum$IPS: sommatoria degli indici di presenza della specie S raccolti lungo il transetto T o la rete di transetti R; LT: lunghezza in chilometri del transetto T percorso, LR: lunghezza in km della rete di transetti R percorsa; per IKAS = 0, l{\textquoteright}abbondanza della specie S nel transetto T o nella rete R \`e nulla, la specie non vi \`e stata rilevata, IKAS> 0, l{\textquoteright}abbondanza della specie S nel transetto T o nella rete R \`e tanto pi\`u elevata quanto pi\`u il parametro \`e maggiore di 0 \\ \midrule
Indice Puntiforme di Abbondanza specifica (IPAS) &
IPAS = $\sum$RP / NP; dove, $\sum$RP: sommatoria delle risposte emesse dalla specie S rilevate dalle stazioni puntiformi P; NP: numero delle stazioni puntiformi dalle quali sono stati emessi richiami; per IPAS = 0, l{\textquoteright}abbondanza della specie S nell{\textquoteright}insieme dei punti-stazione \`e nulla, la specie non vi \`e stata rilevata, IPAS> 0, l{\textquoteright}abbondanza della specie S nell{\textquoteright}area di studio \`e tanto pi\`u elevata quanto pi\`u il parametro \`e maggiore di 0 \\ \midrule
Indice Chilometrico delle Risposte specifiche (IKRS) &
IKRS = $\sum$RS / LT(R); dove $\sum$RS: sommatoria delle risposte emesse dalla specie S e rilevati lungo il transetto T o la rete di transetti R; LT: lunghezza in chilometri del transetto T percorso, LR: lunghezza in km della rete di transetti R percorsa; per IKRS = 0, l{\textquoteright}abbondanza della specie S nel transetto T o nella rete R \`e nulla, la specie non vi \`e stata rilevata, IKRS> 0, l{\textquoteright}abbondanza della specie S nel transetto T o nella rete R \`e tanto pi\`u elevata quanto pi\`u il parametro \`e maggiore di 0 \\ \bottomrule
\end{tabular}

\caption{}
\label{Alemanno_tab_1}
\end{table}


\section*{Discussione}

L{\textquoteright}areale pregresso (1970--1990) presenta un
ID\textsubscript{C} (Tab. \ref{Alemanno_tab_1}) = 0,48; quello attuale \`e risultato
essere 0,50: il grado di sovrapposizione del secondo sul primo \`e 0,74
(Fig. \ref{Alemanno_fig_2}). L{\textquoteright}IKA\textsubscript{C} medio per
l{\textquoteright}intera ADS \`e 0,49; 0,59 per SS, 0,81 per SC e 0,43
per il SM.


\begin{figure}[!h]
\centering
	\includegraphics[width=.8\columnwidth]{Alemanno_fig_2.png}

\caption{Sovrapposizione degli areali pregresso (1970) e attuale (2010) della coturnice nell{\textquoteright}ADS}	
\label{Alemanno_fig_2}
\end{figure}



La stimolazione al canto ha ottenuto 8 risposte di maschi territoriali,
con IPA\textsubscript{C} = 0,12, valore uguale a quello ottenuto in una
precedente ricerca svolta nell{\textquoteright}intero comprensorio dei
Sibillini (Renzini \& Ragni 1998). Le distanze fra i centroidi delle
USR positive alla stimolazione risultano non significativamente diverse
(p{\textless}0,05) dal valore medio, pari a  m 2583;
l{\textquoteright}equiripartizione spaziale dei galli suggerisce la
{\textquotedblleft}saturazione{\textquotedblright} dello spazio
ecologicamente idoneo nell{\textquoteright}ADS.

I valori dell{\textquoteright}indice chilometrico costruito sulle
risposte, IKR\textsubscript{C}  e dello IPA\textsubscript{C} risultano
congruenti nel loro andamento spaziale: r\textsubscript{S}= 0,98
p{\textless}0,01. L{\textquoteright}ICA\textsubscript{C} medio
nell{\textquoteright}ADS \`e pari a 0,5: nel SS il valore \`e nullo,
nel SC \`e 1,45, nel SM 0,24. Vi \`e una congruenza parziale con quelli
risultanti dal MN, in quanto le
{\textquotedblleft}attese{\textquotedblright} per il SS risultano
insoddisfatte: ci\`o \`e imputabile a fattori contingenti che hanno
causato un minore sforzo di campionamento con MC in tale Settore. 

Riguardo alla selezione dell{\textquoteright}\textit{habitat}, la specie
sceglie, significativamente: cotico erboso continuo e discontinuo;
vegetazione arboreo-arbustiva frammentata e con basso grado di
copertura, associabile a una maggiore offerta in alimento ed in rifugio
dalla predazione di \textit{Aquila chrysaetos},  \textit{A. graeca} si
associa anche ad affioramenti rocciosi, rocce emergenti e ghiaioni,
utilizzati come rifugio e/o ricovero notturno;
nell{\textquoteright}arido paesaggio appenninico anche i punti
d{\textquoteright}acqua sembrano rappresentare un fattore favorevole
alla presenza della specie. Lo studio conferma la predilezione per i
versanti a solat\`io, mentre la fascia altitudinale prescelta \`e
quella posta fra i 1601 e i 1900 m.

Il rilevamento di campo tramite MN e MC ha consentito di raccogliere 33
dati di presenza e attivit\`a antropica, classificabili entro 6
categorie di disturbo (Tab. \ref{Alemanno_tab_2}) risultando interessate 11 delle 56 USR
dell{\textquoteright}ADS. Nel SS \`e accentuata la presenza di cani da
pastore sia liberamente vaganti che associati al pascolo in atto, i
quali possono ricadere nella condizione di randagi a causa della
mancanza di controllo umano. Tale presenza ha complicato
l{\textquoteright}attuazione del MC. Nel SM sono stati spesso osservati
escursionisti, \textit{mountain biker} e cercatori di funghi fuori dai
sentieri e saltuariamente autoveicoli fuori dal tracciato stradale.

La H\textsubscript{0}: {\textquotedblleft}la presenza e
l{\textquoteright}abbondanza dei fattori di disturbo antropico non
influenzano la presenza e l{\textquoteright}abbondanza della coturnice
appenninica nell{\textquoteright}area di studio{\textquotedblright},
saggiata tramite i rispettivi indici di abbondanza, \`e stata
rigettata, ottenendo un livello altamente significativo di relazione
inversa tra le due variabili (r\textsubscript{S }=-0,745
p{\textless}0,01).


%\rowcolors{2}{MUSEBLUE!40!white}{white}
\begin{table}[!h]
\centering
\small
\begin{tabular}{>{\raggedright\arraybackslash}p{.5\columnwidth}>{\raggedleft\arraybackslash}p{.13\columnwidth}>{\raggedleft\arraybackslash}p{.13\columnwidth}>{\raggedleft\arraybackslash}p{.13\columnwidth}}
\toprule
\textbf{Categoria disturbo antropico} & \textbf{Punteggio numerico ordinale} & \textbf{Numero di eventi riscontrati} & \textbf{Peso complessivo dato dalla categoria} \\
\toprule
%\showrowcolors
Cani da guardia-difesa liberamente vaganti & 6 & 3 & 18 \\
Cani da guardia-difesa associati a pascolo in atto & 5 & 7 & 35 \\
Escursionisti, deltaplanisti, cercatori di funghi & 4 & 7 & 28 \\ 
Veicoli a motore fuori dal tracciato stradale & 3 & 1 & 3 \\
Pascolo ovino in atto & 2 & 7 & 14 \\
Strada provinciale con traffico intenso & 1 & 8 & 8 \\
\bottomrule
\hiderowcolors
\end{tabular}
\caption{}
\label{Alemanno_tab_2}
\end{table}

\section*{Conclusioni}

Nella porzione pi\`u marginale, sia dal punto di vista ecologico che
geografico, dei Monti Sibillini \textit{A. graeca} costituisce, con
certezza, parte integrante e autoctona della zoocenosi appenninica nel
quarantennio 1970-2010. L{\textquoteright}area studiata sembra
rappresentare ancora oggi una \textit{source} per il vasto \textit{sink
}circostante (Ritchie 1997) il quale, oltre a soffrire le medesime
criticit\`a riscontrate nell{\textquoteright}ADS, \`e paradigma della
non corretta gestione faunistico-venatoria, con
l{\textquoteright}aggravante del bracconaggio.

La positiva realt\`a riscontrata nell{\textquoteright}ADS andrebbe
sostenuta con analisi delle dinamiche fitosociologiche, che valutino
quanto e con quale velocit\`a esse influenzano la specie, con un
monitoraggio specie-specifico a lungo termine, con \textit{best
practice} nell{\textquoteright}uso del territorio e degli ecosistemi,
da rivolgere alle attivit\`a agro-silvo-pastorali e a quelle
ludico-ricreative.

\section*{Ringraziamenti}

Gli Autori sono grati all{\textquoteright}Ente Parco Nazionale dei Monti
Sibillini, per aver consentito la ricerca e la fruizione del materiale
cartografico. Esprimono viva riconoscenza, per i contributi offerti, a:
Loreto Capotondo, Gianfranco Cavarischia, Vincenzo Di Felice, Giuseppe
Nobili, Fulvio Rosa, Massimiliano Cervosi, Rodolfo e Stefano Alemanno;
Guido Bellini, Nando Masciotti (gi\`a Provincia di Perugia); Giovanni
Bucciarelli, Manuele Cacciatori, Roberto Perucci (CTA-CFS); Luca
Convito (Provincia di Perugia), Nicola Felicetti (Studio LEA),
Francesco Renzini (CFS), Silvio Span\`o (UniGE).

\section*{Bibliografia}
\begin{itemize}\itemsep0pt
	\item Caro T., Engilis A. Jr, Fitzherbert E. \& Gardner T., 2004 - Preliminary
assessment of the flagship species concept at a small scale.
\textit{Animal Conservation}, 7: 63-70.

	\item Brichetti P. \& Fracasso G., 2004 - \textit{Ornitologia Italiana}.
\textit{Tetraonidae-Scolapacidae}. Casa editrice Alberto Perdisa
Editore, Bologna: 412 pp.

	\item Fowler J. \& Cohen L., 2002 - \textit{Statistica per Ornitologi e
Naturalisti}. British Trust for Ornithology, Franco Muzzio Editore,
Roma: 240 pp. 

	\item Jacobs J., 1974 - Quantitative measurement of food selection.
\textit{Oecologia}, 14: 413-417.

	\item Magrini M. \& Gambaro C., 1997 - \textit{Atlante Ornitologico
dell{\textquoteright}Umbria}. Regione dell{\textquoteright}Umbria,
Petruzzi Editore, Citt\`a di Castello: 239 pp.

	\item Petretti F., 1985 - La coturnice negli Appennini. \textit{Serie Atti e
Studi},  WWF Italia, 4: 2-24.

	\item Renzini F. \& Ragni B., 1998 - La coturnice nel Parco Nazionale dei
Monti Sibillini. \textit{L{\textquoteright}uomo e
l{\textquoteright}ambiente}. Universit\`a degli Studi di Camerino, 29:
1-40.

	\item Ritchie M. E., 1997 - Populations in a Landscape Context: Sources, Sinks
and Metapopulations. In: Bissonette J. A. (ed.), \textit{Wildlife and
Landscape Ecology. Effect of Patterns and Scale}. Springer-Verlag, New
York: 160-184.

	\item Velatta F., Lombardi G., Sergiacomi U. \& Viali P., 2010 - Monitoraggio
dell{\textquoteright}Avifauna Umbra. \textit{I Quaderni
dell{\textquoteright}Osservatorio} (NS), Regione Umbria: 1-390. 

	\item Zunino M. \& Zullini A., 1995 - Biogeografia, la dimensione
spaziale dell{\textquoteright}evoluzione. Casa Editrice Ambrosiana,
Milano: 310 pp.
\end{itemize}
