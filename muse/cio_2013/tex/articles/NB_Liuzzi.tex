\setcounter{figure}{0}
\setcounter{table}{0}

\begin{adjustwidth}{-3.5cm}{0cm}
\pagestyle{CIOpage}
\authortoc{\textsc{Liuzzi C.}, \textsc{Mastropasqua F.},
\textsc{Todisco S.}}
\chapter*[La collezione ornitologica di Vincenzo de Romita]{La collezione ornitologica di Vincenzo de Romita}
\addcontentsline{toc}{chapter}{La collezione ornitologica di Vincenzo de Romita}

\textsc{Cristiano Liuzzi}$^{1*}$, \textsc{Fabio Mastropasqua}$^1$,
\textsc{Simone Todisco}$^1$ \\

\index{Liuzzi Cristiano} \index{Mastropasqua Fabio} \index{Todisco Simone}
\noindent\color{MUSEBLUE}\rule{27cm}{2pt}
\vspace{1cm}
\end{adjustwidth}



\marginnote{\raggedright $^1$Associazione Centro Studi de Romita, C. da Tavarello
n. 362/A, Monopoli (BA) \\
\vspace{.5cm}
{\emph{\small $^*$Autore per la corrispondenza: \href{mailto:c.liuzzi@wwfoasi.it}{c.\allowbreak li\allowbreak uz\allowbreak zi@\allowbreak wwf\allowbreak oa\allowbreak si.\allowbreak it}}} \\
\keywords{Bari, collezione ornitologica, de Romita}
{Bari, ornithological collection, de Romita}
%\index{keywords}{Bari} \index{keywords}{Collezione ornitologica} \index{keywords}{De Romita}
}
{\small
\noindent \textsc{\color{MUSEBLUE} Summary} / We report the results of a scientific review of V. de Romita
ornithological collection. The collection includes 994 samples
belonging to 318 different species, with an average of 3.1 specimens
per species, collected mainly between 1871 and 1893 and coming mostly
from the province of Bari (79\%). All collected data were published in
the book {\textquotedbl}Avifauna Puglia ... 130 years
later{\textquotedbl}.\\
}




\section*{Introduzione}
Si riportano i risultati del progetto {\textquotedblleft}Museo de
Romita{\textquotedblright} (finanziato dalla Regione Puglia), che ha
previsto la revisione della collezione del Prof. de Romita, custodita
presso l{\textquoteright}IISS Vivante-Pitagora nel centro di Bari. Il
Museo de Romita nasce per ospitare i reperti collezionati dal de Romita
e da lui stesso donati al Regio istituto tecnico e nautico di Bari
(oggi IISS Vivante-Pitagora). Con la supervisione scientifica
dell{\textquoteright}ISPRA e la collaborazione con
l{\textquoteright}Associazione Or. Me., \`e stato possibile programmare
un piano d{\textquoteright}analisi dei singoli reperti. 

\section*{Metodi}

La revisione \`e stata realizzata in un anno (09/2011-08/2012) da tre
operatori; i reperti sono stati controllati secondo i numeri
progressivi dell'inventario, effettuando un controllo
incrociato tra il catalogo e quanto riportato sui cartellini apposti
nel tempo dai diversi curatori della collezione, confrontando il tutto
con gli scritti del de Romita (1884, 1889, 1899, 1900). I dati
dell'inventario e i nuovi acquisiti sono stati
informatizzati. I reperti, dopo essere stati classificati, sono stati
sottoposti a un intervento di pulizia e suddivisi in base allo stato di
conservazione in modo da esporre quelli integri e interessanti, e
conservare i reperti danneggiati, nella speranza che possano un giorno
essere riparati. Gli esemplari di difficile identificazione sono stati
analizzati dettagliatamente e sottoposti in un secondo tempo anche al
parere di esperti ornitologi.

\section*{Risultati e discussione}

Il Catalogo cos\`i ottenuto consta di 994 reperti, per un totale di 472
esemplari di non Passeriformi e 522 di Passeriformi, riferibili a 318
specie; tra queste, la pi\`u rappresentata \`e \textit{Motacilla
flava}, con 115 individui, mentre la media per specie \`e di 3,1. 

La collezione comprende esemplari catturati dal 1869 al 1940. Gli anni
pi\`u prolifici per la raccolta vanno dal 1871 al 1893, periodo durante
il quale sono stati acquisiti 486 reperti (min. 27 nel 1871, max. 100
nel 1891), pi\`u del 50\% del totale. Il 7,9\% dei reperti elencati nel
catalogo non \`e stato riscontrato durante la revisione; dei restanti,
il 6,4\% \`e conservato in pelle mentre l'85,7\% \`e
montato.

La maggior parte dei reperti \`e di origine pugliese (866; circa
l{\textquoteright}87\%); tra le altre regioni italiane le pi\`u
rappresentate sono: Piemonte (17), Friuli-Venezia Giulia (10) e
Lombardia (7). Solo 4 esemplari provengono dall{\textquoteright}estero.
Infine 77 reperti sono di origine ignota (cartellini assenti,
illeggibili ecc.). La provincia con il maggior numero di reperti \`e
Bari, con 781 esemplari (79\%).

Si segnalano tra le specie non riportate in precedenza dal de Romita: 1
ind. di \textit{Charadrius pecuarius} proveniente da Bari (1908):
sottoposta alla COI (se convalidata sarebbe la prima per
l{\textquoteright}Italia); 1 ind. di \textit{Phalaropus tricolor}
proveniente da Bari (24 ottobre 1889), sottoposto al parere della COI;
1 ind. di \textit{Turdus eunomus} proveniente da Bari (1901), prima
segnalazione regionale. Inoltre sono stati sottoposti a parere della
COI: 1 ind. di \textit{Sylvia hortensis} con alcune caratteristiche
riconducibili alla ssp. \textit{crassirostris}; 1 ind. di
\textit{Ficedula hypoleuca} con alcune caratteristiche riconducibili
alla ssp. \textit{speculigera}.

Inoltre nella collezione sono presenti anche 1 ind. di \textit{Cursorius
cursor} (Gallipoli 1899), 1 ind. di \textit{Charadrius asiaticus} (Bari
1898), 1 ind. di \textit{Calidris maritima} (Bari 1904), 1 ind. di
\textit{Calandrella rufescens} (Bari 1875), 1 ind. di
\textit{Eremophila alpestris} (Bari 1877).

L{\textquoteright}elenco completo della collezione \`e parte integrante
del volume {\textquotedblleft}Avifauna pugliese{\dots} 130 anni
dopo{\textquotedblright} pubblicato al termine del progetto, ed \`e
consultabile anche sul sito
\href{http://www.museoderomita.it}{www.\allowbreak mu\allowbreak se\allowbreak o\allowbreak de\allowbreak ro\allowbreak mi\allowbreak ta.\allowbreak it}.

\section*{Bibliografia}
\begin{itemize}\itemsep0pt
	\item De Romita V., 1884 - Avifauna Pugliese, Catalogo sistematico degli
uccelli. \textit{Rist.anast. a cura di Arnaldo Forni Editore}, Bari. 

	\item De Romita V., 1889 - Aggiunte all'ornitologia pugliese.
\textit{Annuario del Regio Istituto Tecnico Nautico di Bari}. 

	\item De Romita V., 1899 - Nuove aggiunte all'ornitologia
pugliese. \textit{Annuario del Regio Istituto Tecnico Nautico di Bari.}

	\item De Romita V., 1900 - Materiali per una fauna barese. In: La Sorsa S. (a
cura di). \textit{La Terra di Bari sotto l{\textquoteright}aspetto
storico, economico e naturale, vol.III, Vecchi, Trani. pp. 245-338},
ried. Levante editori, Bari 1986. 
\end{itemize}
