\setcounter{figure}{0}
\setcounter{table}{0}

\begin{adjustwidth}{-3.5cm}{0cm}
\pagestyle{CIOpage}
\authortoc{\textsc{Fusari M.}, \textsc{Morganti N.}}
\chapter*[Migrazione dei rapaci nel Parco del Conero dal 2007 al 2013]{La migrazione primaverile dei rapaci nel Parco Regionale del
Conero: risultati delle osservazioni dal 2007 al 2013}
\addcontentsline{toc}{chapter}{Migrazione dei rapaci nel Parco del Conero dal 2007 al 2013}

\textsc{Maurizio Fusari}$^{1*}$, \textsc{Niki Morganti}$^{2**}$ \\

\index{Fusari Maurizio} \index{Morganti Niki}
\noindent\color{MUSEBLUE}\rule{27cm}{2pt}
\vspace{1cm}
\end{adjustwidth}


\marginnote{\raggedright $^1$Studio Faunistico Chiros, Macerata \\
$^2$Studio Naturalistico Diatomea, Senigallia (AN) \\
\vspace{.5cm}
{\emph{\small $^*$Autore per la corrispondenza: \href{mailto:chiros.studio@libero.it}{chiros.studio@libero.it}}} \\
{\emph{\small $^{**}$Autore per la corrispondenza: \href{mailto:info@studiodiatomea.it}{info@studiodiatomea.it}}} \\
\keywords{Conero, rapaci, migrazione}
{Conero, raptors, migration}
%\index{keywords}{Conero} \index{keywords}{Rapaci} \index{keywords}{Migrazione}
}
{\small
\noindent \textsc{\color{MUSEBLUE} Summary} / Conero mount is one of the most important site for pre-breeding migrator
raptors crossing the Adriatic sea to fly to Balkans and east-Europe.
Authors counted almost 47.000 raptors of 26 different species during
the years 2007-2013. The migration direction of 90\% of birds were
SW-NE, honey buzzard and marsh harrier were the most common spotted
species.
}



\section*{Introduzione}

Le prime osservazioni del fenomeno migratorio sul promontorio del Conero
svolte da Marco Borioni alla fine degli anni {\textquoteright}80 misero
subito in evidenza la rilevanza dell{\textquoteright}area per la
migrazione primaverile dei rapaci (Borioni 1997). Il Parco Regionale
del Conero, a partire dal 1999 ha avviato un monitoraggio
standardizzato per approfondire le conoscenze sulla migrazione
nell{\textquoteright}area protetta. Nel presente contributo si
riportano in sintesi i risultati riferiti al periodo 2007-2013. 
L{\textquoteright}area di studio ricade all{\textquoteright}interno del
Parco del Conero, situato lungo la costa al centro delle Marche. Il
monte Conero \`e il secondo promontorio roccioso dopo il Gargano che
gli uccelli incontrano lungo la loro rotta migratoria, risalendo il
litorale adriatico da sud. La morfologia dell{\textquoteright}area
presenta falesie calcaree a ridosso del mare e la zona interna con
colline e valli. Il Conero segna il confine settentrionale virtuale del
clima mediterraneo lungo l{\textquoteright}Adriatico, e la vegetazione
che lo caratterizza rispecchia questo fattore con presenza diffusa di
macchia mediterranea e boschi di leccio e roverella nei versanti
rivolti a sud e sud-est; mentre, nei versanti rivolti a nord, la
vegetazione \`e pi\`u mesofila, con ornielli e carpini. Oltre ai gi\`a
citati campi coltivati, sono presenti molti vigneti e numerosi
arbusteti dominati da ginestra; risultano ancora numerosi i boschi a
conifere (principalmente pino d{\textquoteright}Aleppo) derivanti dai
rimboschimenti degli anni {\textquoteright}30 del secolo scorso.

\section*{Metodi}

I rilevamenti sono stati svolti in localit\`a Gradina del Poggio, sito
che permette di avere un{\textquoteright}ampia visuale verso sud-ovest,
e quindi di controllare la direzione principale di provenienza dei
rapaci durante la migrazione preriproduttiva.

I campi di rilevamento sono stati svolti dal 15 aprile al 31 maggio
dalle ore 9 alle 19. Le osservazioni sono state annotate su schede
standard utili alla raccolta di dati relativi a: orario di
osservazione, specie, numero di individui, sesso, et\`a, direzioni di
provenienza e scomparsa e altezza di volo, oltre alle informazioni
relative alle condizioni meteorologiche e sulla visibilit\`a.

\section*{Risultati e discussione}

Sono state censite 26 specie in migrazione, per un totale di quasi
47.000 rapaci contati (Tab.1). Il falco pecchiaiolo \textit{Pernis
apivorus}, ad eccezione del 2007, \`e sempre stata la specie pi\`u
comune seguita dal falco di palude \textit{Circus aeruginosus}: questi
due rapaci negli anni hanno costituito tra il 68\% e il 91\% del totale
delle osservazioni. Nel periodo considerato dopo un picco registrato
negli anni 2008 e 2009, si \`e verificato un calo delle osservazioni
che hanno toccato il minimo nel 2012 (60\% in meno rispetto al 2009).
L{\textquoteright}andamento giornaliero del flusso migratorio evidenzia
il massimo passaggio nel periodo compreso tra la fine di aprile e la
prima met\`a di maggio, periodo coincidente con il picco migratorio del
falco pecchiaiolo. Nel presente lavoro sono state analizzate le
variazioni nella migrazione di 5 specie per cui l{\textquoteright}area
del Conero sembra rivestire particolare importanza; falco pecchiaiolo,
falco di palude, albanella pallida \textit{Circus macrorus}, albanella
minore \textit{Circus pygargus} e falco cuculo \textit{Falco
vespertinus}. L{\textquoteright}andamento negli anni del flusso
migratorio del falco pecchiaiolo rispecchia quanto gi\`a evidenziato
per il totale dei rapaci, con un picco tra il 2008 e il 2010 e un calo
negli anni successivi, mentre il falco di palude mostra un andamento
molto pi\`u costante. Il falco pecchiaiolo, inoltre, evidenzia un trend
nella mediana del passaggio anticipata anno dopo anno. Le altre due
specie di \textit{Circus} mostrano un andamento quasi coincidente con
un maggior numero di individui registrato negli anni 2007, 2011 e 2013.
Il falco cuculo, dopo una vera e propria
{\textquotedblleft}invasione{\textquotedblright} registrata nel 2008 e
una quasi {\textquotedblleft}scomparsa{\textquotedblright}
nell{\textquoteright}anno successivo, ha mantenuto nei successivi anni
un andamento piuttosto costante. Dall{\textquoteright}analisi oraria
delle osservazioni, si nota come la migrazione abbia due picchi, uno
nella tarda mattinata (il falco pecchiaiolo attraversa
l{\textquoteright}area prevalentemente al mattino) e uno nella seconda
met\`a del pomeriggio (il falco di palude sorvola il Conero
principalmente di pomeriggio). L{\textquoteright}analisi della
direzione di provenienza e di scomparsa dei rapaci, evidenzia come la
migrazione abbia un netto asse preferenziale SW-NE (circa il 90\% dei
rapaci).

Il Conero si conferma un{\textquoteright}area fondamentale soprattutto
per i rapaci che intendono raggiungere la Penisola balcanica e
l{\textquoteright}Est Europa e che utilizzano il promontorio come
{\textquotedblleft}trampolino di lancio{\textquotedblright} prima del
difficile attraversamento dell{\textquoteright}Adriatico.
\newcolumntype{S}{>{\raggedleft\arraybackslash}p{.074\columnwidth}}
\begin{longtable}{>{\raggedright\arraybackslash}p{.28\columnwidth}SSSSSSS}
\toprule
\textbf{Specie} & \textbf{2007} & \textbf{2008} & \textbf{2009} & \textbf{2010} & \textbf{2011} & \textbf{2012} & \textbf{2013} \\
\toprule
\endfirsthead
\multicolumn{8}{l}{\textit{\footnotesize Continua dalla pagina precedente}} \\
\toprule
\textbf{Specie} & \textbf{2007} & \textbf{2008} & \textbf{2009} & \textbf{2010} & \textbf{2011} & \textbf{2012} & \textbf{2013} \\
\toprule
\endhead
 Falco pecchiaiolo & 1765 & 5381 & 6546 & 4644 & 2734 & 1711 & 2366 \\
 Nibbio bruno & 59 & 50 & 44 & 29 & 40 & 13 & 47 \\
 Nibbio reale & 7 & 11 & 3 & 7 & 6 & 5 & 5 \\
 Biancone & 4 & 11 & 5 & 2 & 11 & 6 & 3 \\
 Falco di palude & 1896 & 1960 & 2347 & 1569 & 1716 & 1025 & 1621 \\
 Albanella reale & 8 & 6 & 2 & 6 & 4 & 5 & 15 \\
 Albanella pallida & 19 & 8 & 9 & 9 & 34 & 3 & 29 \\
 Albanella minore & 259 & 97 & 135 & 112 & 279 & 73 & 224 \\
 Albanella ind. & 27 & 28 & 10 & 8 & 9 & 23 & 53 \\
 Sparviere & 60 & 22 & 30 & 24 & 22 & 54 & 108 \\
 Sparviere levantino & 1 & 0 & 0 & 0 & 0 & 0 & 0 \\
 Poiana & 225 & 80 & 80 & 93 & 143 & 121 & 460 \\
 Poiana delle steppe & 1 & 0 & 1 & 1 & 0 & 0 & 0 \\
 Poiana coda bianca & 1 & 0 & 1 & 1 & 0 & 1 & 1 \\
 Poiana calzata & 0 & 0 & 0 & 0 & 1 & 0 & 0 \\
 Aquila anatraia minore & 0 & 1 & 0 & 0 & 0 & 1 & 0 \\
 Aquila delle steppe & 0 & 1 & 0 & 0 & 0 & 0 & 0 \\
 Aquila imperiale & 0 & 0 & 0 & 1 & 0 & 0 & 0 \\
 Aquila minore & 1 & 1 & 1 & 1 & 1 & 1 & 2 \\
 Accipitridi ind. & 26 & 7 & 8 & 0 & 0 & 56 & 9 \\
 Falco pescatore & 23 & 16 & 25 & 21 & 15 & 12 & 17 \\
 Grillaio & 6 & 2 & 0 & 1 & 0 & 2 & 3 \\
 Gheppio & 186 & 114 & 239 & 319 & 195 & 119 & 171 \\
 Gheppio/Grillaio & 65 & 84 & 63 & 52 & 38 & 52 & 138 \\
 Falco cuculo & 238 & 1471 & 61 & 325 & 267 & 232 & 365 \\
 Smeriglio & 0 & 0 & 1 & 0 & 0 & 0 & 0 \\
 Lodolaio & 141 & 112 & 115 & 157 & 180 & 137 & 219 \\
 Falco della regina & 0 & 1 & 5 & 3 & 1 & 0 & 0 \\
 Sacro & 2 & 0 & 3 & 0 & 0 & 1 & 0 \\
 Pellegrino ssp. \textit{calidus} & 0 & 0 & 1 & 4 & 1 & 0 & 0 \\
 Falconidi ind. & 8 & 43 & 13 & 26 & 0 & 29 & 20 \\
 \toprule
 \hiderowcolors
 \textbf{ Totale } & \textbf{ 5028 } & \textbf{ 9507 } & \textbf{ 9748 } & \textbf{ 7415 } & \textbf{ 5697 } & \textbf{ 3682 } & \textbf{ 5876} \\
\bottomrule
\multicolumn{8}{l}{} \\
\caption{Numero rapaci censiti nel periodo 2007-2013}
\label{Fusari_tab_1}
\end{longtable}

Proprio in considerazione della rilevanza dell{\textquoteright}area
sarebbe auspicabile riuscire a implementare le conoscenze acquisite con
i monitoraggi che permettano di comprendere la percentuale di esemplari
che dal monte Conero si dirigono verso le coste croate e
l{\textquoteright}ampliamento del periodo di indagine al fine di avere
un quadro completo dell{\textquoteright}intero fenomeno migratorio.



\section*{Ringraziamenti}

Si ringrazia l{\textquoteright}Ente Parco Regionale del Conero, Marco
Borioni, Maria Rosa Baldoni, Mina Pascucci, Vittorio e nonno Franco.


\section*{Bibliografia}
\begin{itemize}\itemsep0pt
 \item Borioni M., 1997 -- \textit{Ali in un Parco}. Printem Edizioni, Ancona,
95 pp.
\end{itemize}

