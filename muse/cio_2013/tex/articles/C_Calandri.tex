\setcounter{figure}{0}
\setcounter{table}{0}


\begin{adjustwidth}{-3.5cm}{0cm}
\pagestyle{CIOpage}
\authortoc{\textsc{Calandri S.}, \textsc{Ragni B.}, 
\textsc{Andreini F.}}
\chapter*[Colombi urbani e la riduzione della
risorsa acqua]{\textcolor{black}{Il problema dei colombi
urbani} \textbf{\textit{\textcolor{black}{Columba
livia }}}\textbf{\textcolor{black}{affrontato con un
esperimento di riduzione della risorsa acqua}}}
\addcontentsline{toc}{chapter}{Colombi urbani e la riduzione della risorsa acqua}

\textsc{Simone Calandri}$^{1*}$, \textsc{Bernardino Ragni}$^{2}$, 
\textsc{Federica Andreini}$^{1}$ \\

\index{Calandri Simone} \index{Ragni Bernardino} \index{Andreini Federica}
\noindent\color{MUSEBLUE}\rule{27cm}{2pt}
\vspace{1cm}
\end{adjustwidth}



\marginnote{\raggedright $^1$Ufficio Ambiente, Comune di Spoleto \\
$^2$Dipartimento di \textcolor{black}{Chimica Biologia
Biotecnologie}, Universit\`a degli Studi di Perugia \\
\vspace{.5cm}
{\emph{\small $^*$Autore per la corrispondenza: \href{mailto:simone.calandri@comunespoleto.gov.it}{si\allowbreak mo\allowbreak ne.\allowbreak ca\allowbreak lan\allowbreak dri@\allowbreak co\allowbreak mu\allowbreak ne\allowbreak spo\allowbreak le\allowbreak to.\allowbreak gov.\allowbreak it}}} \\
\keywords{\textit{Columba livia}, popolazione urbana,
Spoleto}
{\textit{Columba livia}, urban population, Spoleto city}
%\index{keywords}{\textit{Columba livia}} \index{keywords}{Popolazione urbana} \index{keywords}{Spoleto}
}
{\small
\noindent \textsc{\color{MUSEBLUE} Summary} / In the town of Spoleto (Umbria), urban population of \textit{Columba
livia }is estimated at least 4000 individuals, one of the highest
population densities known in the literature.
The freshwater made available by the numerous public fountains, supports
this large population; in fact, previous research of behavioral ecology
has shown that water is a key resource for the urban pigeon. In this
study, we report the results of an experiment involving the temporary
removal of this water for a period of 30 consecutive days; this time
frame can intercept significant fractions of the reproductive cycle of
\textit{Columba livia}. The demographic trend has been monitored
through indicators of the population consisting of 5 roost, by 4 counts
night, before, during and after the experiment. After good starting
results, human- and  weather-linked events prevented the project to be
successful. However, the counts have shown a significant reduction of
pigeons (25\%) up to episodes of heavy rain; we believe that under
ideal conditions the procedure could have achieved its aim.  \\
\noindent \textsc{\color{MUSEBLUE} Riassunto} / La popolazione di colombi stanziali entro la cinta urbica di Spoleto \`e
stimata in almeno 4000 individui, con una delle densit\`a pi\`u alte
note in letteratura. Un precedente studio di ecologia comportamentale
propedeutico al progetto di contenimento, ha dimostrato come la
principale causa ambientale dell{\textquoteright}elevata densit\`a di
colombi urbani risiedesse nel cruciale rapporto che la specie
stabilisce con l{\textquoteright}acqua, una risorsa molto abbondante
nel centro storico della citt\`a, offerta da numerose fontane e lavatoi
pubblici. In questo studio si riportano i risultati di un esperimento
di riduzione temporanea della risorsa acqua per un periodo di 30 giorni
consecutivi; tale durata consente di intercettare frazioni
significative del ciclo riproduttivo del colombo urbano \textit{Columba
livia}. L{\textquoteright}andamento demografico \`e stato monitorato
tramite la stima numerica di indicatori di popolazione, rappresentati
da 5 \textit{roost}, per mezzo di 4 conteggi notturni precedenti,
contemporanei e successivi all{\textquoteright}esperimento. Nonostante
alcuni problemi tecnico-amministrativi e un periodo di precipitazioni
eccezionali, le quantificazioni hanno indicato una significativa
riduzione di colombi (25\%) fino agli episodi di pioggia abbondante.
Sulla base di questi risultati, la stima statistica del tempo
necessario per raggiungere il teorico valore
{\textquotedblleft}0{\textquotedblright} degli indicatori di
popolazione \`e stata di 40.4 giorni di riduzione della risorsa acqua. \\
}



\section*{Introduzione}

Le popolazioni sinantropiche di \textit{Columba livia }possono assumere
aspetti altamente critici per la conservazione del patrimonio
storico-artistico-architettonico, per la salute umana e per il pubblico
decoro nei centri storici italiani ed europei (Baldaccini \& Giunchi
2006). Quella insediatasi entro la cinta urbica di Spoleto (km$^2$ 0,75;
Umbria) \`e stimata in almeno 4000 individui, presentando cos\`i, una
delle densit\`a pi\`u alte note in letteratura (NOMISMA, 2003; Ragni
\textit{et al.} 2008). Nell{\textquoteright}ambito del progetto di
controllo di detta popolazione, la sperimentazione di numerosi metodi
di cattura e di dissuasione dei colombi non ha portato ad alcun
risultato significativo e utile. 

Lo studio dell{\textquoteright}ecologia comportamentale, sulla base di
protocolli d{\textquoteright}osservazione e sperimentali, ha portato a
constatare la stretta dipendenza di \textit{Columba livia
}dall{\textquoteright}acqua fresca, ferma o corrente. La presenza di
numerose fontane e lavatoi pubblici funzionanti, uniformemente
distribuiti entro la cinta urbica spoletana, pu\`o potenzialmente
essere considerata il fattore ecologico basilare per tale eccezionale
densit\`a di popolazione. \textit{E.g.} la Fontana del Foro risulta
essere una delle pi\`u frequentate: applicando il
\textit{Lincoln-Petersen-Chapman Index }(Krebs 1999) a un campione
catturato-marcato-rilasciato \`e stato possibile stimare in 2192-2561 i
colombi gravitanti giornalmente su di essa. Tale fontana \`e stata
sottoposta a un programma di osservazione diretta di 6 ore al giorno
per 10 giorni: il numero di contatti colombo-acqua nei giorni sereni
\`e stato significativamente superiore a quello osservato nei giorni
piovosi (${\chi}$\textsuperscript{2} = 92,34  p{\textless}0,001;
Grafico 1). La Fontana del Mascherone \`e risultata la seconda risorsa
d{\textquoteright}acqua pi\`u importante per i colombi spoletani, posta
a 250 metri dalla precedente. La chiusura sperimentale della Fontana
del Foro ha portato i piccioni a frequentare la seconda, con un
conseguente aumento significativo dei frequentatori (senza considerare
interferenza umana: ${\chi}$\textsuperscript{2} = 4,8 
0,01{\textless}p{\textless}0,05; considerando interferenza umana:
${\chi}$\textsuperscript{2} = 37,2 p{\textless}0,001; Grafico 2).
Osservazioni comparative svolte a Foligno (Umbria) hanno mostrato
identici \textit{pattern }eco-etologici dei colombi nei confronti della
stessa risorsa, nella fattispecie rappresentata dal fiume Topino che
attraversa la citt\`a storica. 

Pertanto, l{\textquoteright}ipotesi di ricerca da sottoporre a test
sperimentale \`e quella secondo la quale: sottraendo alla
disponibilit\`a dei colombi l{\textquoteright}acqua delle fontane
pubbliche cittadine, si ottiene un allontanamento di questi dal centro
storico di Spoleto motivato dalla necessit\`a di ricercare nuove fonti
idriche. Con la chiusura delle fontane ci si attende quindi una
drastica riduzione della popolazione per ridimensionamento
dell{\textquoteright}attivit\`a riproduttiva e per emigrazione: il
colombo rimane legato al luogo di nascita dove torna a nidificare una
volta raggiunta la maturit\`a sessuale, quindi sfavorire la
nidificazione implica agevolarne lo spostamento definitivo verso luoghi
pi\`u idonei alla cura e allevamento dei \textit{pulli}.

\section*{Metodi}

Sono state censite 36 fontane e punti d{\textquoteright}acqua nel centro
storico o in prossimit\`a di questo, verificandone il funzionamento.
L{\textquoteright}\textit{optimum }sperimentale prevede un
prosciugamento di essi \textit{ad libitum}, il cui unico termine \`e
rappresentato dal raggiungimento degli scopi sperimentali: rimozione
completa o altamente significativa della popolazione presente nel
centro storico.

L{\textquoteright}impraticabilit\`a socio-politica del raggiungimento
dell{\textquoteright}\textit{optimum }sperimentale ha indotto i
ricercatori \textcolor{black}{a fissare a 30 giorni consecutivi il
periodo sperimentale di essiccamento delle fontane}; lasso di tempo che
consente di intercettare frazioni significative delle due fasi critiche
del ciclo riproduttivo di \textit{C. livia}: deposizione e cova,
schiusa e cura parentale, della durata complessiva media di 55 giorni.

In considerazione dei dati disponibili sulle serie storiche regionali
delle precipitazioni meteoriche, con particolare riferimento allo
Spoletino (regione Umbria, 1997, 2004) il mese di luglio \`e
statisticamente risultato quello pi\`u siccitoso
dell{\textquoteright}anno; quindi il migliore ai fini
dell{\textquoteright}esperimento. Anche in questo caso, considerazioni
socio-politiche hanno indotto a ubicare cronologicamente
l{\textquoteright}esperimento tra i giorni 3 giugno e 3 luglio.
L{\textquoteright}andamento della sperimentazione \`e stato monitorato
tramite la stima numerica di indicatori di popolazione, rappresentati
da 5 siti, selezionati per fenologia, ubicazione, accessibilit\`a e
stabilit\`a. Per ogni sito sono stati effettuati 4 conteggi notturni
precedenti l{\textquoteright}inizio dell{\textquoteright}esperimento, 4
durante l{\textquoteright}attuazione dell{\textquoteright}operazione e
4 successivamente alla riapertura delle fontane.

\section*{Risultati e discussione}

La chiusura dei punti d{\textquoteright}acqua ha richiesto almeno 2
giorni (Tab. \ref{Calandri_tab_1}); la prima quantificazione post-chiusura degli
indicatori (Tab. 1; Graf. 3) ha registrato una significativa riduzione
di colombi (${\chi}$\textsuperscript{2} = 5,7 
0,01{\textless}p{\textless}0,05) pari al 24,7\%. Successivamente si
sono avute gravi, inattese, interferenze nell{\textquoteright}offerta
d{\textquoteright}acqua. La prima consiste nella riapertura della
fontana del Mascherone e di quella del Fortilizio dei Molini, due siti
strategici, avvenuta almeno 2 giorni prima del secondo conteggio
post-chiusura  (Tab. \ref{Calandri_tab_1}). Inoltre, tra le 14 e le 16 del giorno in cui
\`e stato effettuato il secondo conteggio notturno, il territorio
spoletino \`e stato interessato da forti precipitazioni (Tab. \ref{Calandri_tab_1}). Da
tale giorno, il diciassettesimo dall{\textquoteright}avvio
dell{\textquoteright}operazione Fontisecche, Spoleto,
l{\textquoteright}Umbria e gran parte della Penisola, sono state
investite da precipitazioni piovose, prolungate e abbondanti, a cadenza
pressoch\'e giornaliera (Tab. \ref{Calandri_tab_1}). I successivi conteggi degli
indicatori certificano la pronta risposta dei colombi alle mutate
condizioni dell{\textquoteright}habitat: offerta
d{\textquoteright}acqua abbondante, continua e diffusa; tanto che il
loro andamento nel tempo procede in modo profondamente diverso dalle
attese (Graf. 3): tra i giorni 26 e 38 i valori sono tornati
nell{\textquoteright}intervallo pre-chiusura a causa di aperture
abusive di fontane e presenza di precipitazioni; tra i giorni 51 e 67
(riapertura delle fontane e perdurare di precipitazioni meteoriche)
\textcolor{black}{il numero dei colombi conteggiati} sale
all{\textquoteright}intervallo superiore, 105-113, per effetto
dell{\textquoteright}involo dei nuovo nati. 

Per quanto riguarda l{\textquoteright}esperimento Fontisecche,
l{\textquoteright}unico tentativo che pu\`o essere esperito, al fine di
immaginarne l{\textquoteright}esito se le condizioni sperimentali
previste fossero state rispettate, lo si pu\`o leggere
nell{\textquoteright}equazione: 

\begin{center}
 y = - 2,3x + 93 
\end{center}

della retta di regressione lineare (R\textsuperscript{2} =1,0) che lega
le uniche due situazioni libere dalle gravi condizioni di perturbazione
dell{\textquoteright}esperimento: l{\textquoteright}ultimo conteggio
pre- e il primo conteggio post- chiusura delle fontane.
L{\textquoteright}algoritmo consente di stimare quanto tempo possa
occorrere, in tali condizioni, per raggiungere il teorico valore
{\textquotedblleft}0{\textquotedblright} degli indicatori di
popolazione; tale lasso di tempo \`e stimato in 40,4\textbf{ }giorni:
un intervallo non molto distante da quello, arbitrario, proposto dai
ricercatori. Concludendo, dal punto di vista concreto,
l{\textquoteright}esperimento Fontisecche non ha raggiunto lo scopo
prefissato, quello di una drastica e durevole riduzione della
popolazione sinantropica di \textit{Columba livia }nel centro storico
di Spoleto, probabilmente a causa di gravi perturbazioni delle
condizioni sperimentali previste. Tuttavia, indicazioni parziali
suggeriscono che, in assenza delle anzidette perturbazioni,
l{\textquoteright}intervento avrebbe potuto funzionare
\textcolor{black}{e }portare a esito positivo con un tempo di
sperimentazione moderatamente pi\`u lungo di quello previsto e, aspetto
non secondario, del tutto privo di costi pubblici, anzi risparmiando
preziosa acqua potabile. I risultati del presente lavoro, seppur non
conclusivi e basati su un disegno sperimentale non pienamente
realizzato, supportano l{\textquoteright}ipotesi secondo la quale la
rimozione degli accessi all{\textquoteright}acqua nei confronti dei
colombi pu\`o essere un efficace strumento di gestione volto alla
riduzione delle popolazioni cittadine.

{\footnotesize
\begin{longtable}{>{\raggedright\arraybackslash}p{.1\columnwidth}>{\raggedright\arraybackslash}p{.18\columnwidth}>{\raggedright\arraybackslash}p{.62\columnwidth}}
\toprule
\textbf{Giorno} &  \textbf{N$^{\circ}$ colombi} & \textbf{Eventi} \\
\toprule
\endfirsthead
\multicolumn{3}{l}{\footnotesize Continua dalla pagina precedente} \\
\toprule
\textbf{Giorno} &  \textbf{N$^{\circ}$ colombi} & \textbf{Eventi} \\
\toprule
\endhead
%\showrowcolors
1 &	84&	Primo conteggio pre-chiusura (26 Maggio) \\
2 & & \\		
3& 	82 &	Secondo conteggio pre-chiusura \\
4-5 & & \\		
6 &	82 &	Terzo conteggio pre-chiusura \\
7 & & \\		
8 & 93 &	Quarto conteggio pre-chiusura \\
9	&	& Chiusura fontane \\
10 &	&	Chiusura fontane (4 Giugno) \\
11-17 & & \\		
18 & 70	& Primo conteggio post-chiusura \\
19-23 & & \\		
24	& & Apertura abusiva Fontane Mascherone e Fortilizio dei Mulini \\
25	& & \\	
26	& 88 &	Secondo conteggio post-chiusura; Nubifragio \\
27	& &	 Pioggia \\
28	& &	Pioggia \\
29	& & Temporale e pioggia \\
30	& & \\	
31	& &	Pioggia \\
32	& &	Nubifragio \\
33	& &	Temporale e pioggia \\
34	& & \\	
35	& 90 &	Terzo conteggio post-chiusura; Pioggia \\
36	& & \\	
37	& &	Nubifragio e pioggia \\
38	& 86 &	Quarto conteggio post-chiusura; Temporale e pioggia \\
39	& &	Apertura fontane (3 Luglio); Temporale e pioggia \\
40	& &	Apertura fontane; Temporale e pioggia \\
41	& &	Apertura fontane; Temporale e pioggia \\
42-50 & & \\		
51	& 106 &	Primo conteggio post-apertura \\
52-55 & & \\		
56 & 105 & Secondo conteggio post-apertura \\
57-62 & & \\		
63	& 113 &	Terzo conteggio post-apertura \\
64-66 & & \\		
67 & 109 & Quarto conteggio post-apertura (31 Luglio) \\
\bottomrule
\hiderowcolors
\caption{Esperimento “Fontisecche”}
\label{Calandri_tab_1}
\end{longtable}
}

\begin{figure}[!h]
\centering
\includegraphics[width=.95\columnwidth]{Calandri_fig_1.png}
\caption{Fontana del Foro (Spoleto, Umbria): contatti colombi - acqua in relazione alle giornate piovose (PIOGGIA) e serene (SOLE)}
\label{Calandri_fig_1}
\end{figure}

\begin{figure}[!h]
\centering
\includegraphics[width=.95\columnwidth]{Calandri_fig_2.png}
\caption{Fontana del Mascherone (Spoleto, Umbria): contatti colombi - acqua mentre la vicina Fontana del Foro porta acqua (ACQUA) e non la porta (ASCIUTTA). Tra i giorni 16 e 19 si \`e verificato un forte e continuo disturbo antropico in prossimit\`a della Fontana del Mascherone che ha spaventato i colombi}
\label{Calandri_fig_2}
\end{figure}

\begin{figure}[!h]
\centering
\includegraphics[width=.95\columnwidth]{Calandri_fig_3.jpg}
\caption{Esperimento "Fontisecche" (Spoleto, Umbria). Conteggio dei colombi (ascisse) nei 67 giorni (ordinate) di sperimentazione. Chiusura temporanea fontane pubbliche (giorni 10-38) e riapertura definitiva (dal giorno 39)}
\label{Calandri_fig_3}
\end{figure}



\newpage 
\section*{Bibliografia}
\begin{itemize}\itemsep0pt
	\item \textcolor[rgb]{0.13725491,0.12156863,0.1254902}{Baldaccini N. E.,
Giunchi D., }\textcolor[rgb]{0.13725491,0.12156863,0.1254902}{2006 - 
}\textcolor[rgb]{0.13725491,0.12156863,0.1254902}{Le popolazioni urbane
di colombo: considerazioni sulla loro genesi e sulle metodologie di
gestione. }\textit{Biologia Ambientale}, 20 (2): 125-141. 

	\item Krebs J.C., 1999 - \textit{Ecological Methodology}. Addison Wesley
Longman, Menlo Park (CA): 581 pp.

	\item NOMISMA, 2003 - \textit{Valutazione dei costi economici e sociali dei
colombi in ambiente urbano}. Rapporto finale di ricerca. Bologna: 218
pp.

	\item Ragni B., Calandri S., Andreini F. \& Manni A.C., 2008 - Progetto
colombo urbano (\textit{Columba livia} forma \textit{domestica}) nel
centro storico della citt\`a di Spoleto. Universit\`a degli Studi di
Perugia, Comune di Spoleto, Perugia: 73 pp.

	\item Regione Umbria, 1997 - \textit{Relazione sullo stato
dell{\textquoteright}ambiente in Umbria}. Perugia: 344 pp.

	\item Regione Umbria,  2004 - \textit{Relazione sullo stato
dell{\textquoteright}ambiente in Umbria}. Perugia: 448 pp.
\end{itemize}