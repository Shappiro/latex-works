\setcounter{figure}{0}
\setcounter{table}{0}

\begin{adjustwidth}{-3.5cm}{0cm}
\pagestyle{CIOpage}
\authortoc{\textsc{De Giacomo U.}, \textsc{Guerrieri G.}}
\chapter*[Il nibbio bruno nella discarica di Malagrotta (Roma)]{Osservazioni sulla presenza di giovani di nibbio bruno \textbf{\textit{Milvus migrans}}\textbf{ nella discarica di Malagrotta
(Roma)}}
\addcontentsline{toc}{chapter}{Il nibbio bruno nella discarica di Malagrotta (Roma)}

\textsc{Umberto De Giacomo}$^{1*}$, \textsc{Gaspare Guerrieri}$^{1,2}$ \\

\index{De Giacomo Umberto} \index{Guerrieri Gaspare}
\noindent\color{MUSEBLUE}\rule{27cm}{2pt}
\vspace{1cm}
\end{adjustwidth}

\marginnote{\raggedright $^1$ALTURA - Associazione per la tutela degli uccelli
rapaci e dei loro ambienti \\
$^2$GAROL - Gruppo Attivit\`a Ricerche Ornitologiche del
Litorale \\
\vspace{.5cm}
{\emph{\small $^*$Autore per la corrispondenza: \href{mailto:udegiacomo@libero.it}{udegiacomo@libero.it}}} \\
\keywords{\textit{Milvus migrans}, giovani, discariche di rifiuti,
Italia centrale}
{\textit{Milvus migrans}, young, landfills, central
Italy}
%\index{keywords}{\textit{Milvus migrans}} \index{keywords}{Giovani} \index{keywords}{Discariche di rifiuti} \index{keywords}{Italia centrale}
}
{\small
\noindent \textsc{\color{MUSEBLUE} Summary} / Young black kites \textit{Milvus migrans}\textbf{ }were observed after
the fledglings, in Malagrotta landfill, main trophic area for suburban
Rome population (about 50 breeding pairs), in 2012 and 2013 years.
Their presence grows until the second week of August, while that of
other individuals, progressively decreases from the maximum value
recorded at the end of July. It seems evident, therefore, that many
young reside for a limited amount of time near the dump before
migrating, stopping and feeding at this site before migration.
}


\section*{Introduzione}

Presente da marzo ad agosto, il nibbio bruno, utilizza le discariche di
rifiuti quali aree di foraggiamento. Nel Lazio, la popolazione
riproduttiva pi\`u numerosa (39-54 coppie) \`e quella presente nel
circondario della discarica di Roma (Guerrieri \& De Giacomo 2012).

Scopo del lavoro \`e stato quello di analizzare la frequentazione della
discarica da parte dei giovani dell{\textquoteright}anno di nibbio
bruno, nel periodo successivo agli involi. Infatti, differenze nel
calendario fenologico tra le loro partenze migratorie e quelle degli
adulti, porrebbero l{\textquoteright}accento
sull{\textquoteright}importanza di questi impianti quali aree trofiche
e di sosta pre-migratoria soprattutto per i giovani.

\section*{Metodi}

La discarica di Malagrotta (41$^\circ$ 51'
21'' N - 12$^\circ$
20'  24''  E) si
estende per 200 ha nella periferia occidentale di Roma, citt\`a della
quale ospita i rifiuti (circa 4000 t/giorno) dal
1975. L{\textquoteright}area, posta a 4,5 km dal fiume Tevere, \`e
inserita in un contesto rurale in cui sono presenti anche attivit\`a
estrattive e industriali legate ai rifiuti. Tre aree boscate protette,
poste a breve distanza (1-9 km), costituiscono altrettanti nuclei per
le coppie di nibbio bruno nidificanti.

I rilievi, sono stati condotti a intervalli di 10 giorni, nel periodo
post-riproduttivo (dalla terza decade di giugno alla fine di agosto)
del 2012 e del 2013. Durante ogni se\allowbreak \allowbreak ssione sono stati effettuati
dei conteggi a tempo, uno ogni 6', sui nibbi in
alimentazione nell{\textquoteright}area dei rifiuti, censendo
separatamente i giovani dell{\textquoteright}anno e gli altri individui
(accorpando adulti e immaturi dell{\textquoteright}anno precedente). La
durata di ogni sessione \`e stata di 4 ore (12:00-15:00; N = 40) nel
2012, e di 10 ore nel 2013 (08:00-17:00, N = 100). Inoltre, nel 2013,
sono stati censiti, ogni 30', anche gli individui
posati (in riposo) in un raggio di 100 m all{\textquoteright}esterno
dell{\textquoteright}area di lavorazione dei rifiuti.

I confronti statistici, sono stati effettuati tra le medie dei rilievi a
carico dei due gruppi (adulti e giovani) per sessione (sia tra quelli
in attivit\`a trofica che tra i posati del 2013) e tra i due anni
(nell{\textquoteright}intervallo 12-15), e valutati tramite test z
(confronti duplici) e ANOVA (confronti multipli). 

\section*{Risultati e discussione}

Durante l{\textquoteright}estate del 2013, sono stati registrati
complessivamente 5134 contatti riguardanti individui in attivit\`a
trofica nell{\textquoteright}area dei rifiuti, di cui 623 (8,2 \%)
erano giovani dell{\textquoteright}anno. Questi ultimi, rilevati dalla
fine giugno (0,3 {\textpm} 1,0 DS; N = 100), sono aumentati
progressivamente fino a un massimo registrato nella seconda decade di
agosto (3,3 {\textpm} 1,0 DS; N = 100; ANOVA: F = 51,7; P {\textless}
0,05), quando rappresentavano il 63,9 \% degli individui in attivit\`a
trofica nella discarica. I restanti individui (adulti e immaturi nati
l{\textquoteright}anno precedente), partendo dal valore pi\`u elevato
rilevato all{\textquoteright}inizio dello studio (11,7 {\textpm} 7,9
DS, N = 100), hanno raggiunto quello pi\`u basso nella terza decade di
agosto (0,8 {\textpm} 1,1 DS; N = 100; ANOVA: F = 23.1, P {\textless}
0,05). Anche tra gli individui a riposo (n = 887), la presenza dei
giovani \`e aumentata dalla fine di luglio (0,8 {\textpm} 1,7 DS; N =
20) fino a raggiungere il valore pi\`u alto nella seconda decade di
agosto (10,2 {\textpm} 8,1 DS; N = 20. Fig. \ref{DeGiacomo_fig_1}), costituendo il 59,3 \%
dei nibbi censiti. Le differenze tra le decadi sono risultate
significative (ANOVA: F = 67.8, P {\textless} 0,001). Nel 2013, la
presenza dei giovani rilevati nelle ore centrali della giornata (pari a
1,0 {\textpm} 1,7 DS; N = 20) non \`e risultata diversa da quella
osservata nel 2012 (pari a 0.6 {\textpm} 0,8 DS, N = 20; z = 0,2; g.l.
= 558; n.s.).

Nei rapaci \`e nota la migrazione ritardata da parte dei giovani
(Kjell\'en 1992). In particolare nel nibbio bruno, si considera che gli
adulti lascino i quartieri riproduttivi 3-4 settimane prima dei giovani
(Newton 1979) e che questi costituiscano in agosto il 60 \% dei
migranti (Panuccio 2005).

Inoltre, prima della migrazione, aumenta la tendenza al gregarismo
creando una maggiore concentrazione di individui nelle zone di
{\textquotedblleft}\textit{stopover}{\textquotedblright} della
migrazione, tra le quali figurano per questa specie le discariche di
rifiuti (Panuccio \textit{et al.} 2005). Nel caso di Malagrotta, le
osservazioni effettuate all{\textquoteright}interno della discarica
indicano che i giovani dell{\textquoteright}anno tendono a raggiungere
precocemente quest{\textquoteright}area trofica dopo
l{\textquoteright}involo e che aumentano la loro consistenza
progressivamente sino alla met\`a del mese di agosto, quando
rappresentano la maggior parte della popolazione di nibbi presente nel
sito.

\section*{Conclusioni}

I risultati ottenuti confermano quindi che, presso questa discarica, i
giovani nibbi tendano a trattenersi nell{\textquoteright}area e a
partire in ritardo rispetto alla gran parte degli adulti, probabilmente
per accumulare le riserve di energia necessarie a permettere loro di
superare la migrazione, al pari di quanto accade per altre specie
(Baker 1978). L{\textquoteright}applicazione della 1999/31/CE, che
prevede la chiusura delle discariche di rifiuti solidi urbani non
trattati in Europa, avr\`a ripercussioni negative sulla specie
(Peronace \textit{et al.} 2012) e in particolare su questa fascia
d{\textquoteright}et\`a, che \`e anche quella maggiormente soggetta
alla pressione selettiva (Sergio \textit{et al.} 2011).

\begin{figure}[!h]
\centering
\includegraphics[width=.8\columnwidth]{DeGiacomo_fig_1.png}
\caption{Numero medio $\pm$ E.S. (errore standard) di giovani dell{\textquoteright}anno vs. altre classi di et\`a (adulti e subadulti) del nibbio bruno in attivit\`a trofica e in riposo, rilevato per decadi dal 20 luglio al 30 agosto 2013 nella discarica di Malagrotta (RM). Juv/at = giovani in attivit\`a trofica; A+S/at = tutti gli altri individui (adulti e subadulti) in attivit\`a trofica; Juv/rp = giovani in riposo; A+S/rp = altri individui (adulti e subadulti) in riposo}
\label{DeGiacomo_fig_1}
\end{figure}

\section*{Bibliografia}
\begin{itemize}\itemsep0pt
	\item Baker R.R., 1978 - \textit{The Evolutionary Ecology of Animal
Migration}. Holmes \& Meier Publishers, New York, 1012 pp.

	\item Guerrieri G. \& De Giacomo U., 2012 - Nibbio bruno \textit{Milvus
migrans}. In: Aradis A., Sarrocco S. \& Brunelli M. Analisi dello
status e della distribuzione dei rapaci diurni nidificanti nel Lazio.
\textit{Quaderni Natura e Biodiversit\`a} 2/2012 ISPRA: 23-29.

	\item Kjell\'en N.,1992 - Differential timing of autumn migration between sex
and age groups in raptors at Faltserbo, Sweden. \textit{Ornis
scandinavica,} 23: 420-434.

	\item Newton J., 1979 - \textit{Population ecology of Raptors}. T.\& A.D.
Poyser, London, 399 pp.

	\item Panuccio M., 2005 - Dati sulla presenza del Nibbio bruno \textit{Milvus
migrans} in due discariche di rifiuti urbani. \textit{Alula,} 12/1-2:
189-192.

	\item Peronace V., Cecere J.C., Gustin M. \& Rondinini C., 2011 - Lista Rossa
2011 degli Uccelli Nidificanti in Italia. \textit{Avocetta,} 36: 11-58.

	\item Sergio F., Tavecchia G., Blas G., L\'opez L., Tanferna A. \& Hiraldo F.,
2011 - Variation in age-structured vital rates of a long-lived raptor:
Implications for population growth. \textit{Basic and Applied Ecology},
12: 107-115.
\end{itemize}
