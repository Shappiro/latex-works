\setcounter{figure}{0}
\setcounter{table}{0}

\begin{adjustwidth}{-3.5cm}{0cm}
\pagestyle{CIOpage}
\authortoc{\textsc{Trotta M.}}
\chapter*[Successo alimentare del chiurlo maggiore in periodo invernale]{Ulteriori dati sul successo alimentare del chiurlo maggiore \textbf{\textit{Numenius arquata}}\textbf{ in periodo invernale nel Parco Nazionale del Circeo (Lazio, Italia Centrale)}}
\addcontentsline{toc}{chapter}{Successo alimentare del chiurlo maggiore in periodo invernale}

\textsc{Marco Trotta}$^{1*}$ \\

\index{Trotta Marco}
\noindent\color{MUSEBLUE}\rule{27cm}{2pt}
\vspace{1cm}
\end{adjustwidth}


\marginnote{\raggedright $^1$SROPU -- Stazione Romana Osservazione e Protezione Uccelli, Via Britannia 36, 00183 Roma, Italia \\
\vspace{.5cm}
{\emph{\small $^*$Autore per la corrispondenza: \href{mailto:marcotrot@gmail.com}{marcotrot@gmail.com}}} \\
\keywords{Parco Nazionale del Circeo, Lazio, \textit{Numenius arquata}, successo alimentare}{Circeo National Park, Latium, \textit{Numenius arquata}}
}
{\small
\noindent \textsc{\color{MUSEBLUE} Summary} / The foraging success of the Eurasian curlew \textit{Numenius
arquata} in winter was studied in a coastal region of southern Latium. Comparing the data collected in the period 1997-98/2000-01 with the results of this study, an increase of the foraging attempts in contrast
with a decrease in the feeding success was found. This result is probably caused by a decrease of the density of earthworms in the feeding areas.\\
}
\vspace{1cm}

Nel periodo 1/12/12-28/2/13 \`e stata condotta
un{\textquoteright}indagine sul successo di foraggiamento del chiurlo
maggiore \textit{Numenius arquata} in un sito di svernamento del Lazio
meridionale. I rilievi sono stati effettuati in alcune aree adibite al
pascolo situate all{\textquoteright}interno del P.N. del Circeo ed
utilizzate dalla specie per alimentarsi. Sono stati eseguiti, su
soggetti in attivit\`a di foraggiamento, 66 campionamenti della durata
di 3 minuti ciascuno. I dati raccolti sono stati confrontati con i
risultati registrati nelle medesime aree di alimentazione in quattro
stagioni invernali, nel periodo 1997-98/2000-01 (Trotta 2008). Il
chiurlo maggiore ha effettuato 9.2 tentativi/minuto, il successo di
foraggiamento \`e stato di 1.14 prede/minuto. Su 227 prede, il 93\% \`e
rappresentato da Artropodi e il 7\% da lombrichi.
{Sulle coste tedesche del mar Baltico, in zone esposte
a marea, Rippe \& Dierschke (1997) registrano un successo di cattura
fino a tre volte maggiore,}\textcolor{red}{ }{sebbene
le specie predate siano differenti;}\textcolor{red}{
}{questo risultato \`e determinato dalla facilit\`a
con cui i chiurli riescono a individuare il polichete
}\textit{{Nereis diversicolor}}{
nelle velme }{coperte da acqua a profondit\`a compresa
tra 3 e 15 cm.}{ }Nel presente studio, il successo di
foraggiamento \`e pi\`u basso di quello rilevato da Berg (1993) in
ambienti prativi della Svezia centrale e risulta inferiore anche alla
media registrata nel P.N. del Circeo nel quadriennio 1997-98/2000-01
(Tab. \ref{Trotta_tab_1}), sebbene la frequenza di tentativi sia maggiore. Una chiave di
lettura potrebbe essere rappresentata dal marcato decremento di
lombrichi catturati, solo il 7\% in questa indagine rispetto al 18\%
registrato nelle stesse aree di foraggiamento nel 1997-98/2000-01. Il
decremento \`e ancora pi\`u evidente se si confrontano i dati raccolti
nel 1998-99 in una indagine sui fattori di disturbo nelle aree di
foraggiamento; su un campione di 70 prede i lombrichi rappresentano
infatti il 42,7\% del totale (Trotta 2003). Nella dieta del chiurlo
maggiore, gli oligocheti della famiglia \textit{Lumbricidae} occupano
una porzione rilevante in termini di biomassa e rappresentano una
risorsa di elevato valore nutrizionale (Camus \textit{et al.} 2001).
L{\textquoteright}impiego di alcuni fertilizzanti e il ricorso a
profonde lavorazioni annuali dei terreni sono tra gli interventi che
possono causare una forte riduzione dei lombrichi (Chiarini \& Conte
2010). Densit\`a basse di questa preda potrebbero non consentire ai
chiurli di soddisfare il fabbisogno energetico giornaliero, con gravi
conseguenze sulla sopravvivenza degli individui; nei mesi invernali,
infatti, \`e di fondamentale importanza per gli uccelli avere
un{\textquoteright}adeguata disponibilit\`a trofica che permetta di
accumulare le provviste energetiche per fronteggiare le condizioni
meteorologiche avverse (Owen 1980; Van Gils \textit{et al.} 2006). Con
future ricerche sarebbe interessante indagare il comportamento del
chiurlo maggiore nelle ore successive al tramonto, in modo da
comprendere se il modesto successo alimentare registrato nel P.N. del
Circeo sia compensato da un{\textquoteright}intensa attivit\`a trofica
notturna.

\begin{table}[!h]
\centering
\small
\begin{tabular}{>{\raggedright\arraybackslash}p{.2\columnwidth}>{\raggedright\arraybackslash}p{.15\columnwidth}>{\raggedright\arraybackslash}p{.15\columnwidth}>{\raggedright\arraybackslash}p{.15\columnwidth}>{\raggedright\arraybackslash}p{.15\columnwidth}}
\toprule
& \textbf{Successo/\allowbreak minuto} & \textbf{Tentativi/\allowbreak minuto} & \textbf{Lombrichi/\allowbreak minuto} & \textbf{Minuti di campionamento} \\
\toprule
2012-13	& 1.14 & 9.2 & 0.08 & 198 \\
\midrule
1997-98/2000-01	& 1.41 & 7.3 & 0.26 & 298 \\
\bottomrule
\end{tabular}
\caption{Tentativi di cattura e successo alimentare del chiurlo maggiore nel P.N. del Circeo; sono riportati in tabella, per comparazione, anche i dati raccolti nel quadriennio 1997-98/2000-01}
\label{Trotta_tab_1}
\end{table}

\section*{Bibliografia}
\begin{itemize}\itemsep0pt
	\item Berg \r{A}., 1993 - Food resources and foraging success of Curlews
\textit{Numenius arquata} in different farmland habitats. \textit{Ornis
Fennica}, 70: 22-31.
	\item Camus A., Bernard J.L. \& Granval P., 2001 - Bordi dei campi e
lombrichi. A cura di ANUU Migratoristi, ZENECA Sopra: 14-15.
	\item Chiarini F. \& Conte L., 2010 - Avvicendamenti, consociazioni e
fertilit\`a del suolo in agricoltura biologica. Veneto Agricoltura, 56
pp.
	\item Owen M., 1980 - Wild geese of the world: their life history and
ecology. Batsford, London, 236 pp. 
	\item Rippe H. \& Dierschke V., 1997 - Picking out the plum jobs: feeding
ecology of curlews Numenius arquata in a Baltic Sea wind flat.
\textit{Mar. Ecol. Prog. Ser.}, 159: 239-247.
	\item Trotta M., 2003 - Alimentazione del Chiurlo maggiore \textit{Numenius
arquata} in periodo invernale e analisi dei fattori di  disturbo nelle
aree di foraggiamento. \textit{Avocetta}, 27: 23.
	\item Trotta M., 2008 - Strategie di foraggiamento del Chiurlo maggiore
\textit{Numenius arquata} e differenze di successo alimentare tra sessi
in un sito di svernamento dell{\textquoteright}Italia centrale.
\textit{Avocetta}, 32: 41-46. 
	\item Van Gils J.A., Spaans B., Dekinga A. \& Piersma T.,
2006 - Foraging in a tidally structured environment
by Red Knots (Calidris canutus): ideal, but not free. \textit{{Ecol.}}{, 87 (5):
1189-202.}
\end{itemize}