\setcounter{figure}{0}
\setcounter{table}{0}

\begin{adjustwidth}{-3.5cm}{0cm}
\pagestyle{CIOpage}
\authortoc{\textsc{Basile M.}}
\chapter*[Dieta del gufo comune in habitat semi-urbano (Arzano,
Campania)]{Analisi della dieta del gufo comune \textbf{\textit{Asio
otus}}\textbf{ durante lo svernamento in habitat semi-urbano (Arzano,
Campania)}}
\addcontentsline{toc}{chapter}{Dieta del gufo comune in habitat semi-urbano (Arzano,
Campania)}


\textsc{Marco Basile}$^{1*}$ \\

\index{Basile Marco}
\noindent\color{MUSEBLUE}\rule{27cm}{2pt}
\vspace{1cm}
\end{adjustwidth}



\marginnote{\raggedright $^1$Associazione per la Ricerca, la Divulgazione e
l{\textquotesingle}Educazione Ambientale -- ARDEA \\
\vspace{.5cm}
{\emph{\small $^*$Autore per la corrispondenza: \href{mailto:marcob.nat@gmail.com}{marcob.nat@gmail.com}}} \\
\keywords{\textit{Asio otus}, Campania, dieta, borre,
habitat semi-urbano}
{\textit{Asio otus}, Campania, diet, pellets, suburban habitat}
%\index{keywords}{\textit{Asio otus}} \index{keywords}{Campania} \index{keywords}{Dieta} \index{keywords}{Borre}
%\index{keywords}{Habitat semi-urbano}
}
{\small
\noindent \textsc{\color{MUSEBLUE} Summary} / A long-eared owl \textit{Asio otus}\textbf{ }suburban roost was found in
2012 at Arzano (Campania, Southern Italy). The roost was occupied from
January till March. The analysis of pellets allowed to identify 6
different micromammals species plus indistinct bird remains.
Savi{\textquoteright}s pine vole is the most frequent prey (76\%).
Despite the low number of black rat and brown rat, amounting for 5\%
altogether, they form 23\% of the total biomass.
}
\section*{Introduzione}
Il gufo comune \textit{Asio otus} \`e uno strigiforme parzialmente
sedentario, nonch\'e migratore regolare e svernante regolare in Italia
(Galeotti 2003). Come tutti gli Strigiformi, questa specie produce
borre che si mantengono integre a lungo; ci\`o permette di effettuare
studi approfonditi sull{\textquoteright}alimentazione (Nappi 2011).
Indagini pregresse hanno messo in risalto la relazione tra le
popolazioni di arvicole e la composizione della dieta, evidenziando un
comportamento predatorio piuttosto specialista (Korpim\"aki 1992; Tome
2009). A differenza di quanto accade in Europa, tuttavia, in Italia il
gufo comune sembra essere meno specialista, avendo a disposizione uno
spettro di prede pi\`u ampio (Bertolino \textit{et al.} 2001). In
particolare, un comportamento alimentare generalista risulta pi\`u
evidente in habitat semiurbani, dove sono disponibili anche prede di
grosse dimensioni quali \textit{Rattus }sp. (Pirovano \textit{et al.}
2000).

In questo studio, vengono riportate informazioni sulla dieta
giornaliera, ottenute attraverso l{\textquoteright}analisi delle borre
raccolte presso un \textit{roost} sito ad Arzano (NA), a 79 m s.l.m. 

\section*{Metodi}

Il \textit{roost} era localizzato in un giardino privato, su un pino
marittimo \textit{Pinus pinaster}, in ambiente semi-urbano, con campi
coltivati o incolti e tessuto urbano. Le borre sono state raccolte ogni
giorno, conservate per almeno 24 h in un congelatore a -18
{\textdegree}C e dissezionate a secco. Per
l{\textquoteright}identificazione dei micro-Mammiferi si \`e fatto
riferimento a Nappi (2001), mentre gli Uccelli, come in altri studi,
sono stati considerati come un{\textquoteright}unica entit\`a (Galeotti
\& Canova 1994; Pirovano \textit{et al.} 2000). Per valutare se il
numero di specie rinvenute potesse essere considerato esaustivo dello
spettro alimentare \`e stato paragonato allo stimatore non parametrico
di ricchezza in specie Chao1-bc (Chao 2005). Questo stimatore stima la
ricchezza di specie teorica osservabile, calcolata dalle abbondanze di
ogni specie. Una sua comparazione con la ricchezza di specie reale
pu\`o dare un{\textquoteright}idea della rappresentativit\`a del
campione. La biomassa delle prede \`e stata calcolata utilizzando le
tabelle fornite da Lovari et al. (1976), ad eccezione che per le specie
appartenenti al genere \textit{Rattus}, per le quali sono state usate
le equazioni fornite da Di Palma e Massa (1981). Per gli Uccelli \`e
stata utilizzata, seguendo il medesimo autore,
un{\textquoteright}approssimazione di 10 g per individuo.

Il \textit{roost} \`e stato frequentato dal 6 gennaio
all{\textquoteright}8 marzo, per un totale di 53 giorni di presenza. Il
numero medio di individui \`e stato di tre (range 1-4). Sono state
raccolte 87 borre, che fanno riscontrare una produzione di 0.55 borre
al giorno per gufo. Sono state identificate 198 prede, tra cui \`e
stato possibile discernere 7 specie diverse di micro-Mammiferi, per un
valore totale di biomassa di 3788,7 g (Tab. \ref{Basile_tab_1}). 

\section*{Risultati e discussione}

La biodiversit\`a riscontrata \`e stata ritenuta esaustiva dello spettro
alimentare, come risultato dallo stimatore non parametrico Chao1 (mean
{\textpm} s.e. = 8.0 {\textpm} 0.5). Nel 57\% delle borre si sono
trovate una o due prede, mentre il pasto medio \`e risultato di 19-38
g. Il 76\% delle prede determinate appartiene alla specie
\textit{Microtus savii}, mentre solo il 5\% \`e stato attribuito a
\textit{Rattus rattus} o \textit{Rattus norvegicus}. Di contro, il 55\%
della biomassa \`e costituita da \textit{M. savii}, mentre il 23\% da
\textit{Rattus} sp. (pesi medi: \textit{R. rattus} = 58 g; \textit{R.
norvegicus} = 148 g). Gli uccelli sono un{\textquoteright}altra
importante fonte di alimentazione, costituendo il 9\% degli individui e
il 10\% della biomassa.

I ratti, nonostante costituiscano una fonte
d{\textquoteright}alimentazione potenzialmente non trascurabile,
sembrano essere predati in maniera occasionale, mentre
l{\textquoteright}arvicola di Savi risulta essere la preda principale.
Tali risultati appaiono in linea con quanto gi\`a noto per il nord
Italia (Pirovano \textit{et al.} 2000). Differentemente da altri studi,
lo spettro alimentare riscontrato risulta notevolmente ridotto
(Galeotti \& Canova 1994; Cecere \& Vicini 2000). Le cause di ci\`o
potrebbero essere ricercate in una scarsa qualit\`a
dell{\textquoteright}ambiente circostante, talvolta oggetto di
derattizzazioni. 

\begin{table}[!h]
\centering

\begin{tabular}{>{\raggedright\arraybackslash}p{.32\columnwidth}>{\raggedright\arraybackslash}p{.18\columnwidth}>{\centering\arraybackslash}p{.07\columnwidth}>{\centering\arraybackslash}p{.07\columnwidth}>{\centering\arraybackslash}p{.07\columnwidth}>{\centering\arraybackslash}p{.07\columnwidth}}
\toprule
\textbf{Specie} & \textbf{Nome comune} & \multicolumn{2}{c}{\textbf{Frequenza}} & \multicolumn{2}{c}{\textbf{Biomassa}} \\
& & \multicolumn{1}{c}{n} & \multicolumn{1}{c}{\%} & \multicolumn{1}{c}{g} & \multicolumn{1}{c}{\%} \\
\toprule
\multicolumn{4}{l}{\textit{Rodentia - Cricetidae}} \\
\toprule
%\showrowcolors
\textit{Microtus savii} & Arvicola di Savi & 151 & 76.26 & 2076.25 & 54.80 \\
\toprule
\hiderowcolors
\multicolumn{4}{l}{\textit{Rodentia - Muridae}} \\
\toprule
\textit{Apodemus sp.} & & 10 & 5.05 & 265 & 6.99 \\
\textit{Apodemus sylvaticus} & Topo selvatico & 4 & 2.02 & 106 & 2.80 \\
\textit{Mus domesticus} & Topo comune & 5 & 2.53 & 95 & 2.51 \\
\textit{Rattus rattus} & Ratto nero & 5 & 2.53 & 291.33 & 7.69 \\
\textit{Rattus norvegicus} & Ratto norvegese & 4 & 2.02 & 591.66 & 15.62 \\
\toprule
\hiderowcolors
\multicolumn{4}{l}{\textit{Soricomorpha - Soricidae}} \\
\toprule
%\showrowcolors
\textit{Crocidura suaveolens} & Crocidura minore & 1 & 0.51 & 3.5 & 0.09 \\
\toprule
\hiderowcolors
\textit{Aves - Passeriformes} & Uccelli & 18 & 9.09 & 360 & 9.50 \\
\toprule
\textbf{Totale} & & 198 & 100 & 3788.74 & 100 \\
\bottomrule
\end{tabular}
\caption{Specie rinvenute nelle borre con frequenza e biomassa assoluta e relative. La biomassa e la frequenza assolute sono espresse come numero di individui (n) e grammi (g). La biomassa e frequenza relative sono espresse in percentuali.}
\label{Basile_tab_1}
\end{table}

\section*{Bibliografia}
\begin{itemize}\itemsep0pt
	\item Read C.B. \& Vidakovic B. (eds), 2006 - \textit{Encyclopedia of
Statistical Sciences, 2nd Edition, Vol. 12}. Wiley, New York.

	\item Di Palma M.G. \& Massa B., 1981 - Contributo metodologico per lo studio
dell{\textquoteright}alimentazione dei rapaci. In: Farina A. (eds),
Atti I Convegno Italiano di Ornitologia Aulla (MS). \textit{Volume
monografico }(1982). 

	\item Galeotti P. \& Canova L., 1994 - Winter diet of Long-eared Owls
(\textit{Asio otus}) in the Po plain (northern Italy). \textit{J.
Raptor Research }28 (4): 265 - 268.

	\item Galeotti P., 2003 - Gufo comune. In Spagnesi M., Serra L. (eds).
\textit{Quad. Cons. Natura 16. }Min. Ambiente - Ist. Naz. Fauna
Selvatica.

	\item Korpim\"aki E., 1992 - Diet composition, prey choice, and breeding
success of Long-eared Owls: effects of multiannual fluctuations in food
abundance. \textit{Can. J. Zool}. 70: 2373 - 2381.

	\item Lovari S., Renzoni A. \& Fondi R., 1976 - The predatory habits of the
Barn Owl (\textit{Tyto alba }Scopoli) in relation to the vegetation
cover. \textit{Boll. Zool. }43: 173 - 191.

	\item Nappi A., 2001 - \textit{Micromammiferi d{\textquoteright}Italia.
}Edizioni Simone.

	\item Nappi A., 2011 - L{\textquoteright}analisi delle borre degli uccelli:
metodiche, applicazioni e informazioni. Un lavoro monografico.
\textit{Picus }37 (72): 106 - 120. 

	\item Pirovano A., Rubolini D., Brambilla S. \& Ferrari N., 2000 - Winter diet
of urban roosting Long-eared Owls \textit{Asio otus} in northern Italy:
the importance of the Brown Rat \textit{Rattus norvegicus}.
\textit{Bird Study }47: 242 - 244.

	\item Tome G., 2009 - Changes in the diet of Long-eared Owl \textit{Asio
otus}: seasonal patterns of dependence on vole abundance.
\textit{Ardeola }56 (1): 49 - 56. 
\end{itemize}

