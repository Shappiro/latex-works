\setcounter{figure}{0}
\setcounter{table}{0}

\begin{adjustwidth}{-3.5cm}{0cm}
\pagestyle{CIOpage}
\authortoc{\textsc{Sighele M.}, \textsc{Lerco R.}}
\chapter*[Zigolo delle nevi nel Parco della Lessinia e innevamento
al suolo]{Presenza invernale dello zigolo delle nevi
\textbf{\textit{Plectrophenax nivalis}}\textbf{ nel Parco della
Lessinia (Verona, Veneto): relazione con l{\textquoteright}innevamento
al suolo}}
\addcontentsline{toc}{chapter}{Zigolo delle nevi nel Parco della Lessinia e innevamento
al suolo}

\textsc{Maurizio Sighele}$^{1*}$, \textsc{Roberto Lerco}$^1$  \\

\index{Sighele Maurizio} \index{Lerco Roberto}
\noindent\color{MUSEBLUE}\rule{27cm}{2pt}
\vspace{1cm}
\end{adjustwidth}


\marginnote{\raggedright $^1$Associazione Verona Birdwatching, Via Lungolor\`i,
5/A, 37127 Verona \\
\vspace{.5cm}
{\emph{\small $^*$Autore per la corrispondenza: \href{mailto:info@veronabirdwatching.org}{in\allowbreak fo@ve\allowbreak ro\allowbreak na\allowbreak bir\allowbreak dwatch\allowbreak ing.\allowbreak org}}} \\
\keywords{\textit{Plectrophenax nivalis}, Lessinia, Verona, Veneto, svernamento, neve}
{\textit{Plectrophenax nivalis}, Lessinia, Verona,
Veneto, wintering}
}
{\small
\noindent \textsc{\color{MUSEBLUE} Summary} / Snow bunting \textit{Plectrophenax
nivalis}\textcolor[rgb]{0.14901961,0.14901961,0.14901961}{ has only
been a regular visitor to the Natural Regional Park of Lessinia
(Verona, Veneto) for the last eight winters}. Authors compared data of
bunting sightings to the presence of snow on the ground, suggesting
that snow buntings have never been observed in absence of snow. No
relation was found between number of buntings and amount of snow on the
ground.\\
}
\vspace{1cm}



Nel XX secolo lo zigolo delle nevi (\textit{Plectrophenax nivalis}) era
segnalato assai raramente nel Parco Naturale Regionale della Lessinia
(Verona, Veneto), con poche osservazioni note tra gli anni
'60 e '80 per le quali era
considerato migratore irregolare o, in modo pi\`u ottimistico, di
presenza abituale (De Franceschi 1993; Sauro 1980). In realt\`a fino al
2001 si possono contare rare e sporadiche segnalazioni e non sono noti
periodi di svernamento. A partire dall{\textquoteright}inverno
2003/2004, invece, questo passeriforme artico \`e stato segnalato ogni
anno nel territorio del Parco, anche se le osservazioni di un numero
significativo di individui che si sono ripetute per alcune settimane
sono state riscontrate solo negli ultimi 8 inverni, cio\`e dal
2005/2006 (Sighele \& Parricelli 2007, 2009, 2010, 2011, 2012, 2013).
La ragione del maggiore numero di segnalazioni dello zigolo delle nevi
potrebbe essere correlata alla maggiore diffusione delle informazioni,
cartacee o telematiche, e all{\textquoteright}individuazione di una
particolare localit\`a, Bocca di Selva, dove \`e pi\`u facile
incontrare la specie negli inverni pi\`u rigidi, con conseguente
aumento di visitatori e di \textit{feedback}. In questi anni \`e
sembrato che la presenza degli zigoli delle nevi in Lessinia fosse
legata all{\textquoteright}innevamento al suolo, ma questa ipotesi,
avanzata anche 150 anni orsono (Perini 1858; De Betta 1863), fino ad
ora non era mai stata suffragata da dati certi. Sono stati pertanto
analizzati e confrontati i dati relativi alla presenza dello zigolo
delle nevi e dell{\textquoteright}innevamento in Lessinia nei 6 inverni
tra il 2007/2008 e il 2012/2013. In questo periodo lo zigolo delle nevi
\`e stato osservato ogni inverno, prevalentemente tra la prima decade
di dicembre e la seconda decade di marzo; la data pi\`u precoce
rilevata \`e il 3 dicembre (2012) e quella pi\`u tardiva il 12 aprile
(2013). La media degli individui contati in questi 6 inverni \`e
superiore a 20 indd. (20,3), con un conteggio massimo di 38 indd. nel
marzo 2010. Confrontando le registrazioni dei periodi di osservazione
di questa specie con la presenza di innevamento al suolo si osserva che
gli zigoli non sono mai stati segnalati in caso di mancanza di neve
(Fig. 1). Nell{\textquoteright}unico inverno con poche precipitazioni
considerato da questa ricerca (2011/2012), le osservazioni della specie
sono coincise col periodo di presenza di neve al suolo. Quando il
periodo di innevamento si \`e prolungato per tutto
l{\textquoteright}inverno, da dicembre a marzo, la durata delle
osservazioni \`e solitamente risultata limitata a 6-10 settimane, con
l{\textquoteright}eccezione del 2012/2013, quando gli zigoli sono stati
presenti tra dicembre e aprile. Non si sono invece evidenziati aspetti
significativi legati all{\textquoteright}entit\`a
dell{\textquoteright}innevamento. I risultati di questa analisi,
pertanto, indicano che la presenza dello zigolo delle nevi nel Parco
della Lessinia \`e strettamente legata alla presenza di neve al suolo,
seppur non alla sua entit\`a; vi \`e quindi una maggior difficolt\`a
nell{\textquoteright}osservare questa specie quando le precipitazioni
nevose sono assenti.

\begin{figure}[!h]
\centering
\includegraphics[width=.8\columnwidth]{Sighele_fig_1.png}
\caption{Zigoli delle nevi e innevamento nel Parco della Lessinia}
\label{Sighele_fig_1}
\end{figure}

\section*{Ringraziamenti}
Si ringrazia il Corpo Forestale dello Stato, Comando di Verona, per la
concessione dei dati relativi all{\textquoteright}innevamento al suolo.

\section*{Bibliografia}

\begin{itemize}\itemsep0pt
	\item Arrigoni Degli Oddi E., 1899 - Note ornitologiche della provincia di
Verona. Atti Soc. ital. Sc. Nat., 38 (1/2): 75-190.

	\item De Betta E., 1863 - Materiali per una fauna veronese. Tip.
Vicentini e Franchini, 144 pp.

	\item {De Franceschi P., 1993 - Parco Naturale Regionale della Lessinia - Piano ambientale: Zoologia
-- Vertebrati. {Regione del Veneto e Comunit\`a Montana della Lessinia.}}

	\item Perini G., 1858 - Uccelli Veronesi. Tip. Vicentini, 320 pp.

	\item Sauro E., 1980 - \textcolor{black}{Alcune variazioni della flora e della
fauna lessinica dal 1950 ad oggi. }La Lessinia - Ieri oggi
domani - Quaderno culturale: 25-26.

	\item Sighele M. \& Parricelli P., 2007 - Resoconti ornitologici del Parco
della Lessinia. Anno 2006. Parco Naturale della Lessinia \&
Verona Birdwatching, 24 pp.

	\item Sighele M. \& Parricelli P., 2009 - Resoconti ornitologici del Parco
della Lessinia. Anno 2008. Parco Naturale della Lessinia \&
Verona Birdwatching, 32 pp.

	\item Sighele M. \& Parricelli P., 2010 - Resoconti ornitologici del Parco
della Lessinia. Anno 2009. Parco Naturale della Lessinia \&
Verona Birdwatching, 32 pp.

	\item Sighele M. \& Parricelli P., 2011 - Resoconti ornitologici del Parco
della Lessinia. Anno 2010. Parco Naturale della Lessinia \&
Verona Birdwatching, 32 pp.

	\item Sighele M. \& Parricelli P., 2012 - Resoconti ornitologici del Parco
della Lessinia. Anno 2011. Parco Naturale della Lessinia \&
Verona Birdwatching, 32 pp.

	\item Sighele M. \& Parricelli P., 2013 - Resoconti ornitologici del Parco
della Lessinia. Anno 2012. Parco Naturale della Lessinia \&
Verona Birdwatching, 32 pp.
\end{itemize}
