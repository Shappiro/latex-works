\setcounter{figure}{0}
\setcounter{table}{0}

\begin{adjustwidth}{-3.5cm}{0cm}
\pagestyle{CIOpage}
\authortoc{\textsc{Sarrocco S.}, \textsc{Capizzi D.}, \textsc{Pizzol I.}, 
\textsc{Scalisi M.}}
\chapter*[Il monitoraggio delle specie della Direttiva Uccelli nel
Lazio]{\textbf{Il monitoraggio delle specie della Direttiva Uccelli
(2009/147/CE) nel Lazio: stima dei parametri di alcuni
}\textbf{\textit{taxa}}\textbf{ di interesse conservazionistico}}
\addcontentsline{toc}{chapter}{Il monitoraggio delle specie della Direttiva Uccelli nel
Lazio}

\textsc{Stefano Sarrocco}$^{1*}$, \textsc{Dario Capizzi}$^1$, \textsc{Ivana Pizzol}$^{1,2}$, 
\textsc{Marco Scalisi}$^1$  \\

\index{Sarrocco Stefano} \index{Capizzi Dario} \index{Pizzol Ivana} \index{Scalisi Marco}
\noindent\color{MUSEBLUE}\rule{27cm}{2pt}
\vspace{1cm}
\end{adjustwidth}


\marginnote{\raggedright $^1$Agenzia Regionale Parchi, Area Biodiversit\`a e
Geodiversit\`a \\
$^2$Universit\`a della Tuscia di Viterbo, DEB \\
\vspace{.5cm}
{\emph{\small $^*$Autore per la corrispondenza: \href{mailto:ssarrocco@regione.lazio.it}{s\allowbreak sar\allowbreak roc\allowbreak co@\allowbreak re\allowbreak gio\allowbreak ne.\allowbreak la\allowbreak zio.\allowbreak it}}} \\
\keywords{Lazio, Monitoraggio, Direttiva Uccelli, Modelli}
{Latium, monitoring, Birds, modeling}
%\index{keywords}{Lazio} \index{keywords}{Monitoraggio} \index{keywords}{Modelli} \index{keywords}{Direttiva Uccelli}
}
{\small
\noindent \textsc{\color{MUSEBLUE} Summary} / We modeled the regional (Latium, central Italy) suitability distribution
of 14 breeding birds of European conservation concern by Maxent
software. All the AUC values of the ROC curve obtained by Maxent are
higher than 0,8. Some species showed a wide suitability distribution
(e.g. honey buzzard, red-backed shrike), others less than 1000
km\textsuperscript{2} (e.g. rock partridge, white-backed woodpecker).\\
}

\section*{Introduzione}

Le direttive europee {\textquotedblleft}Habitat{\textquotedblright} e
{\textquotedblleft}Uccelli{\textquotedblright} prevedono il
monitoraggio dello stato di conservazione degli habitat e delle specie,
con l{\textquoteright}obiettivo del loro mantenimento in uno stato
favorevole. I principali parametri da misurare sono: \textit{range
}della specie e dell{\textquoteright}habitat, consistenza della
popolazione e variazioni nel tempo, ampiezza e qualit\`a
dell{\textquoteright}habitat. La Regione Lazio per rispondere a questi
obblighi normativi si \`e dotata di una Rete regionale di Monitoraggio
della Biodiversit\`a, organizzata in un centro regionale (\textit{Focal
Point}), alcuni centri tematici e una rete capillare di laboratori
territoriali (presso le aree naturali protette). In questo contributo
sono presentati i modelli predittivi, ottenuti con
l{\textquoteright}uso del software \textit{Maxent} (Phillips \textit{et
al}. 2006), su 14 specie di interesse comunitario nidificanti nel
Lazio: coturnice \textit{Alectoris graeca}, falco pecchiaiolo
\textit{Pernis apivorus}, nibbio bruno \textit{Milvus migrans},
biancone \textit{Circaetus gallicus}, aquila reale \textit{Aquila
chrysaetos}, succiacapre \textit{Caprimulgus europaeus}, picchio
dalmatino \textit{Dendrocopos leucotos}, calandra \textit{Melanocorypha
calandra}, calandrella \textit{Calandrella brachydactyla}, tottavilla
\textit{Lullula arborea}, calandro \textit{Anthus campestris}, balia
dal collare \textit{Ficedula albicollis}, averla piccola \textit{Lanius
collurio} e ortolano \textit{Emberiza hortulana}. Obiettivi
dell{\textquoteright}uso di questi modelli sono la definizione
dell{\textquoteright}area di distribuzione potenziale e la stima delle
consistenze delle popolazioni regionali.

\section*{Area di studio}

L{\textquoteright}area di studio coincide con il territorio della
Regione Lazio, esteso 17.207 km\textsuperscript{2}. I dati ambientali
(\textit{environmental data}) utilizzati sono costituiti da 13 strati
informativi derivanti dall{\textquoteright}uso del suolo (CUS della
Regione Lazio, 2004), oltre a 9 strati relativi ad aspetti morfologici
(DTM, pendenze, morfologia), idrografici e demografici (centri abitati
e densit\`a demografica); tutti gli strati sono stati trasformati in
distanze minime di ogni punto dalla categoria considerata, ad eccezione
degli aspetti morfologici che hanno mantenuto il valore originario. 

\section*{Metodi}

I dati di presenza delle specie (\textit{samples}) provengono dal nuovo
atlante regionale degli uccelli nidificanti (Brunelli \textit{et al}.
2011), oltre a ulteriori dati raccolti nel 2011-2013 che hanno permesso
di ottenere le densit\`a di alcune specie lungo percorsi campione
(\textit{line transect method}: Jarvinen \& Vaisanen 1976). Per un
confronto sullo stato di conservazione favorevole delle specie
analizzate si \`e fatto riferimento ai valori di FRV
(\textit{Favourable Reference Value}) riportati da Gustin \textit{et
al}. (2009). I modelli sono stati prodotti in un formato \textit{Raster
ArcGis} con celle quadrate di 100 metri di lato; questi sono stati
successivamente riclassificati in tre classi di idoneit\`a attraverso
la modalit\`a del \textit{natural break} (algoritmo di Jenks). Per il
calcolo dell{\textquoteright}area idonea di ogni specie non sono state
considerate le celle ricadenti nella classe di idoneit\`a minore. 

\section*{Risultati e discussione}

In tabella 1 sono indicati i valori dell{\textquoteright}area sotto la
curva ROC (AUC) ottenuti nell{\textquoteright}elaborazione dei modelli
di idoneit\`a: la soglia di efficienza dei modelli si attesta su valori
superiori a 0,8 che dimostra una buona capacit\`a dei modelli di
predire la distribuzione della specie. L{\textquoteright}estensione
dell{\textquoteright}area a media e alta idoneit\`a evidenzia un gruppo
di specie a distribuzione ristretta (\textit{Aquila chrysaetos},
\textit{Ficedula albicollis}, \textit{Alectoris graeca}  e
\textit{Dendrocopos leucotos}) con valori di idoneit\`a minori di 1000
km\textsuperscript{2 }(va tenuto presente che le specie a distribuzione
ristretta assumono valori di AUC maggiori rispetto a quelle a
distribuzione ampia), un secondo gruppo costituito da specie ad ampia
distribuzione. Nella tabella sono evidenziate le densit\`a e le
consistenze di alcune specie, registrate nel corso dei rilievi e
riprese dalla bibliografia pi\`u recente (Sorace \textit{et al}. 2011,
cfr. Brunelli \textit{et al}. 2011, Aradis \textit{et al}. 2012),
confrontate con i valori di FRV o le Indicazioni di Conservazione (Ind.
Cons.) proposti da Gustin \textit{et al}. (2009). In alcuni casi le
densit\`a medie regionali sono simili o superiori a quelle soglia
(\textit{Lanius collurio} e \textit{Dendrocopos leucotos}), in altri
casi inferiori (\textit{Melanocorypha calandra}, \textit{Calandrella
brachydactyla} e \textit{Lullula arborea}). Per alcuni valori di FVR la
soglia \`e riferita all{\textquoteright}intera popolazione peninsulare
o appenninica, come nel caso della Balia dal collare; per questa
specie, in questo lavoro, \`e stata effettuata una stima per
estrapolazione della consistenza della popolazione: considerando la
sola area ad alta idoneit\`a e la densit\`a minima registrata nella
regione si ottiene una stima di oltre 1700 coppie, ben oltre il 50\% di
quella indicata per l{\textquoteright}intera Italia peninsulare.


\begin{table}[!h]
\centering
%\rowcolors{2}{white!60!MUSEBLUE}{white}
\newcolumntype{S}{>{\raggedleft\arraybackslash}p{.1\columnwidth}}
\scalebox{.7}{
\begin{tabular}{>{\raggedright\arraybackslash}p{.2\columnwidth}SSSSSSSSS}
\toprule
\textbf{Specie} & \textbf{N} & \textbf{AUC} & \textbf{Media idoneit\`a [km$^2$]} & \textbf{ Alta idoneit\`a [km$^2$]} & \textbf{ Tot. Area idoneit\`a [km$^2$]} & \textbf{Densit\`a media cp/10 [ha]} & \textbf{ Popol. Lazio coppie} & \textbf{FVR Cp. o cp/10 [ha]} & \textbf{Ind. Cons. cp/10 [ha]} \\
\toprule
\textit{Aquila chrysaetos} &  20 & 0,993 & 452,72 & 188,13 & 640,85 & - & 11 & 170$^{*}$ & - \\
\textit{Lanius collurio} &  546 & 0,805 & 8901,61 & 5735,64 & 14637,25 & 0,65 & - & 0,5 & 0,5 \\
\textit{Ficedula albicollis} &  44 & 0,99 & 463,62 & 394,74 & 858,36 & 2,46 & 1718,3 & 3000$^{*}$ & 2,5 \\
\textit{Circaetus gallicus} &  144 & 0,91 & 5421,56 & 2586,8 & 8008,36 & - & 54-82 & - & 1,0 $^{**}$ \\
\textit{Melanocorypha calandra} &  53 & 0,966 & 1359,69 & 858,02 & 2217,71 & 3,15 & - & 6 & 3,5 \\
\textit{Calandrella brachydactyla} &  45 & 0,949 & 2502,38 & 1213,15 & 3715,53 & 2,82 & - & 10 & 3,5 \\
\textit{Anthus campestris} &  172 & 0,931 & 3593,46 & 1462 & 5055,46 & 1,32 & - & 2,5 & 1,5 \\
\textit{Alectoris graeca} &  38 & 0,991 & 269,93 & 157,48 & 427,41 & 0,16 & 171-342 & - & - \\
\textit{Dendrocopos leucotos} &  269 & 0,87 & 315,46 & 236,19 & 551,65 & 0,5 & 160-210 & - & 5-6 \\
\textit{Milvus migrans} &  326 & 0,916 & 4275,44 & 1600,03 & 5875,47 & - & 77-117 & 700$^{*}$ & - \\
\textit{Emberiza hortulana} &  32 & 0,979 & 2646,45 & 1012,16 & 3658,61 & - & - & - & 8 \\
\textit{Pernis apivorus} &  24 & 0,995 & 6644,65 & 3951,19 & 10595,84 & 2,00 $^{**}$ & - & - & 0,2 $^{**}$ \\
\textit{ Caprimulgus europaeus } &  116 & 0,94 & 5353,04 & 1582,33 & 6935,37 & - & - & 5,0 $^{**}$ & - \\
\textit{ Lullula arborea } &  199 & 0,91 & 4542,39 & 1755,57 & 6297,96 & 0,58 & - & 3 & 1 \\
\bottomrule
\hiderowcolors
\multicolumn{10}{l}{}\\
\multicolumn{10}{l}{ $^{**}$ =  valori densit\`a espressi in coppia/100km$^2$}\\
\multicolumn{10}{l}{$^*$ = popolazione Italia peninsulare} \\
\end{tabular}
}
\caption{Numero di dati, valori di AUC, dimensione delle aree a media e alta idoneit\`a, densit\`a media e consistenza della popolazione delle specie analizzate nel Lazio. Sono riportati i valori di FVR e le Indicazioni di Conservazione proposti da Gustin \textit{et al}. (2009)}
\label{Sarrocco_tab_1}
\end{table}

\newpage
\section*{Bibliografia}
\begin{itemize}\itemsep0pt
 \item Aradis A., Sarrocco S. \& Brunelli M. 2012 - Analisi dello
status e della distribuzione dei rapaci diurni nidificanti nel Lazio.
Quaderni Natura e Biodiversit\`a 2/2012 ISPRA, ARP Lazio, 139 pp.

 \item Brunelli M., Sarrocco S., Corbi F., Sorace A., Boano A., De Felici S.,
Guerrieri G., Meschini A. \& Roma S. (a cura di), 2011 - Nuovo
Atlante degli Uccelli Nidificanti nel Lazio. \textit{Edizioni ARP (Agenzia
Regionale Parchi)}, Roma, 464 pp.

 \item Gustin M., Brambilla M. \& Celada C. (a cura di), 2009 - Valutazione
dello Stato di Conservazione dell{\textquoteright}avifauna italiana.
Ministero dell{\textquoteright}Ambiente e della Tutela del Territorio e
del Mare, Lega Italiana Protezione Uccelli (LIPU), 1153 pp.

 \item Jarvinen O. \& Vaisanen R.A., 1976 - Finnish Line Transect Censuses.
\textit{Ornis Fennica}, 53: 115-118.

 \item Phillips S. J., Anderson R. P. \& Schapire R. E., 2006 - Maximum entropy
modeling of species geographic distributions. \textit{Ecological
Modelling}, Vol 190 (3-4): 231-259.

 \item Sorace A., Properzi S., Guglielmi S., Riga F., Trocchi V. \& Scalisi M.,
2011 - La Coturnice nel Lazio: status e piano
d{\textquoteright}azione. \textit{Edizione ARP}, Roma; 80 pp.
\end{itemize}
