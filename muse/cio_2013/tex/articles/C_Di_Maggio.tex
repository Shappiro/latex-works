\setcounter{figure}{0}
\setcounter{table}{0}

\begin{adjustwidth}{-3.5cm}{0cm}
\pagestyle{CIOpage}
\authortoc{\textsc{Di Maggio R.}, \textsc{Campobello D.}, 
\textsc{Mascara R.}, \textsc{Sar\`a M.}}
\chapter*[Microclimate affects lesser kestrel breeding success in
Sicily]{Nest characteristics affect microclimate and breeding success of
lesser kestrel \textbf{\textit{Falco naumanni}}\textbf{ in the Gela
plain}}
\addcontentsline{toc}{chapter}{Microclimate affects lesser kestrel breeding success in
Sicily}

\textsc{Rosanna Di Maggio}$^{1*}$, \textsc{Daniela Campobello}$^{1}$, 
\textsc{Rosario Mascara}$^{2}$, \textsc{Maurizio Sar\`a}$^{1}$ \\

\index{Di Maggio Rosanna} \index{Campobello Daniela} \index{Mascara Rosario} \index{Sar\`a Maurizio}
\noindent\color{MUSEBLUE}\rule{27cm}{2pt}
\vspace{1cm}
\end{adjustwidth}



\marginnote{\raggedright $^1$STEBICEF Department (Biological, Chemical and
Pharmaceutical Sciences and Technologies), University of Palermo, Italy \\
$^2$Fondo Siciliano per la Natura, via Popolo, 6, 93015,
Niscemi (Caltanissetta, Italy) \\
\vspace{.5cm}
{\emph{\small $^*$Autore per la corrispondenza: \href{mailto:rosannadimaggio@gmail.com}{ro\allowbreak san\allowbreak na\allowbreak di\allowbreak mag\allowbreak gio@\allowbreak g\allowbreak ma\allowbreak il.\allowbreak com}}} \\
\keywords{\textit{Falco naumanni}, Sicilia, colonialit\`a
facoltativa, temperatura nel nido, umidit\`a}
{\textit{Falco naumanni}, Sicily,
facultative coloniality, nest temperature, humidity}
%\index{keywords}{\textit{Falco naumanni}} \index{keywords}{Sicilia} \index{keywords}{Colonialit\`a
%facoltativa} \index{keywords}{Temperatura nel nido} \index{keywords}{Umidit\`a}
}
{\small
\noindent \textsc{\color{MUSEBLUE} Summary} / Microclimate is an important factor for nest site selection and it
influences breeding success. Secondary-cavity nesters are compelled by
nest site availability to select existing nest structures to optimize
microclimate conditions. The lesser kestrel is a colonial raptor
breeding in pseudo-steppe habitats. It does not build a nest but breeds
under roof tiles or wall crevices of rural buildings. We studied 45
nest sites in five lesser kestrel \textit{Falco naumanni}\textbf{
}colonies located in the Gela Plain (Sicily). We measured microclimate
by placing thermo loggers inside nests so that we could record
temperature and relative humidity every hour, from laying to fledging
period. Our results revealed a significant effect of nest orientation
and nest type on the relative humidity, with the highest values in
nests under tiles placed in N-NW sides. Temperature was higher in hole
nests than in those under tiles. Nest orientation and nest type created
a specific microclimate that affected the reproductive outcome. In
particular, a nest under tiles in a S-SE side had a higher hatching
success while a nest in a hole in a S-SE side had a higher chance to
produce more fledglings.  \\
\noindent \textsc{\color{MUSEBLUE} Riassunto} / Il microclima \`e un importante aspetto della selezione del sito di
nidificazione e influenza il successo riproduttivo. Specie che non
costruiscono il nido ma usano strutture preesistenti
(\textit{secondary-cavity nesters}) sono particolarmente limitate dalla
disponibilit\`a del sito riproduttivo poich\'e obbligate a selezionare
strutture gi\`a esistenti al fine di ottimizzare le condizioni
microclimatiche. Il grillaio \`e un rapace coloniale che nidifica in
habitat pseudo-steppici. Nidifica sotto le tegole dei tetti o nei buchi
di edifici rurali abbandonati. In questo lavoro abbiamo studiato 45
nidi in cinque colonie di grillaio situate nella Piana di Gela. Il
microclima del nido \`e stato rilevato inserendo dei sensori termici,
in modo da registrare temperatura e umidit\`a relativa ogni ora del
giorno, dalla deposizione delle uova all{\textquoteright}involo dei
pulcini. I risultati rivelano un effetto
dell{\textquoteright}esposizione e della tipologia di nido
sull{\textquoteright}umidit\`a relativa con i valori pi\`u alti nei
nidi sotto le tegole esposte a N-NO mentre la temperatura risulta pi\`u
alta nei buchi rispetto alle tegole. L{\textquotesingle}esposizione e
la tipologia di nido creano uno specifico microclima che sembra
influenzare il successo riproduttivo. In particolare, un nido sotto le
tegole esposte a S-SE presenta un maggiore tasso di schiusa mentre un
nido in un buco esposto a S-SE invola pi\`u pulcini. \\
}


 
\section*{Introduction}

Specific microclimate within the nest (i.e. temperature and humidity) is
an important factor driving nest site selection and might influence
breeding success (Martin 1998; Lloyd \& Martin 2004). Some birds seem
to actively choose certain cavities but not all birds build their own
nests (Robert \textit{et al. }2010)\textit{.} Secondary-cavity nesters
are compelled by nest site availability to select existing nest
structures that minimize predation pressures while protecting eggs and
chicks against climatic variations (Sar\`a \textit{et al.} 2012). Such
species may optimize microclimate by selecting specific nest sites
characterized by specific exposures associated with suitable thermal
characteristics. Several studies have in fact shown the importance of
nest location and orientation with regard to solar radiation (e.g.
Lloyd \& Martin 2004). Nest microclimate can have important influences
on parental reproduction in secondary cavity nesting birds. The lesser
kestrel \textit{Falco naumanni} is a small raptor breeding in colonies
of variable size (2-60 pairs, Catry \textit{et al.} 2011; Sar\`a 2010)
in pseudo-steppe habitats, choosing its hole-nest in cliffs, under roof
tiles or wall crevices of rural buildings (Di Maggio \textit{et al.}
2013). It does not build a nest but lays its eggs directly on the
cavity floor after scraping the substrate. This species, former
considered {vulnerable}, has recently changed its
conservation status to {\textquotedblleft}least
concern{\textquotedblright}, due to conservation actions in part of its
range (I\~nigo \& Barov 2011). Also, the Gela plain lesser kestrel
population, the largest in Sicily, has grown consistently over the last
decade (Sar\`a 2010). In this species little is known about the effects
of nest thermal characteristics on reproductive success, so the aims of
this study are: 1) to determine the relationship between microclimate
and nest site characteristics; 2) to evaluate whether and how nest site
characteristics affects lesser kestrel breeding success.

\section*{Methods}

The Gela plain, located in south-eastern Sicily (Italy, 378070N,
148190E), is one of the largest plains in Italy (about 474
km\textsuperscript{2}). Due to limited precipitation (350 mm/year), the
agricultural landscape is composed of a mosaic of pseudo-steppes
dominated by artichoke \textit{Cynara} spp. fields and non-irrigated
crops. Across the plain, numerous rural buildings, often partially
destroyed or abandoned, host lesser kestrel nests (Di Maggio \textit{et
al.} 2014). Each colony is usually composed of a single building, or
two or more small houses very close, but with different sides.

The study was conducted in 2010, from April to July. We studied 45 nests
in five lesser kestrel colonies of different size. Colonies were
located within two different areas of the plain: the core area, where
the surface of all cropland land uses was ${\geq}$50 \% within a radius
of 500 m around the colony and an altitude between 0 ${\leq}$ 100 m
a.s.l and the edge, where the cropland area was {\textless} 50\% and
the altitude {\textgreater} 100 m a.s.l. We measured nest microclimate
by setting thermo loggers, so that temperature and relative humidity of
the nest were recorded each day every hour, from the laying (April-May)
to the fledging period (June-July). Each nest was checked at least 3-4
times in order to record the number of eggs and nestlings and then to
measure hatching and fledging rates. 

We used a generalized linear mixed model (GLMM, McCullagh \& Searle
2000) with a normal distribution and an identity link function to
describe the relationship between microclimate and the following
features: 1) colony location, 2) nest-type and 3) nest orientation.
Moreover, we used a second GLMM to test the relationship between
hatching and fledging rate and nest characteristics. In both analyses
we included colony and nest identities as random effects.

\section*{Discussion}

Our results revealed a significant effect of nest type (tiles or wall
holes) on the temperature inside nests with higher temperatures in
holes than under tiles (Fig. \ref{DiMaggio_fig_1}, Tab. \ref{DiMaggio_tab_1}). Furthermore, we recorded an
effect of nest orientation and nest type on the relative humidity with
the highest values in nests under tiles placed in N-NW sides (Figg. \ref{DiMaggio_fig_2} -
\ref{DiMaggio_fig_3}, Tab. \ref{DiMaggio_tab_2}). We did not find any effect of the colony location on both
temperature and relative humidity (Tab. \ref{DiMaggio_tab_1} - \ref{DiMaggio_tab_2}), and this result
suggested that differences in habitat and topography (i.e. altitude)
did not reflect a relevant change in microclimate inside nests.

Nest orientation and type determined when and for how long a nest is
exposed to direct solar radiation and wind.  Additionally, nest
orientation largely determined temperature and humidity within the
nest. Other factors (e.g., parent presence or number of chicks) are
warranted of further investigation for their potential effect on
microclimate. Nest orientation and nest type resulted in a specific
microclimate, affecting the reproductive success. In particular, a nest
under tiles in a S-SE building side would have a higher hatching
success (GLMM, F\textsubscript{side} = 3.114, P {\textless} 0.001,
F\textsubscript{type  } = 63.87, P {\textless} 0.001); additionally a
nest in a hole in a S-SE side had a higher chance to produce more
fledglings (GLMM, F\textsubscript{side} = 9.569, P {\textless} 0.001,
F\textsubscript{type  } = 5.634, P {\textless} 0.001; Figg. \ref{DiMaggio_fig_4} - \ref{DiMaggio_fig_5}).
Nest type and nest exposure compensate humidity, as a drier nest type
(hole) in a wetter side (N-NW) should have the same relative humidity
than the reverse combination (tile nest in dry S-SE side). Nest site
characteristics (i.e. type and orientation) have in turn a strong
impact on reproductive success through their effects on microclimate.
This latter could in conclusion minimize thermal requirements of eggs
and nestlings and improve reproductive success. 


\newpage
\begin{figure}[!h]
\centering
\begin{minipage}{0.49\textwidth}
\centering
	\includegraphics[width=.97\columnwidth]{DiMaggio_fig_1.png}
	\caption{Temperature inside lesser kestrel nests as a function of nest type (GLMM, N = 45)}
	\label{DiMaggio_fig_1}
\end{minipage}\hfill
\begin{minipage}{0.49\textwidth}
\centering
	\includegraphics[width=.97\columnwidth]{DiMaggio_fig_2.png}
	\caption{Relative humidity inside lesser kestrel nests as a function of nest type (GLMM, N = 45)}
	\label{DiMaggio_fig_2}
\end{minipage}
\end{figure}

\begin{figure}[!h]
\centering
	\includegraphics[width=.6\columnwidth]{DiMaggio_fig_3.png}
	\caption{Relative humidity inside lesser kestrel nests as a function of nest orientation (GLMM, N = 45)}
	\label{DiMaggio_fig_3}
\end{figure}

\begin{figure}[!h]
\centering
	\includegraphics[width=.95\columnwidth]{DiMaggio_fig_4.png}
	\caption{Hatching rate of lesser kestrel as a function of nest type and nest orientation (GLMM, N = 45)}
	\label{DiMaggio_fig_4}
\end{figure}

\begin{figure}[!h]
\centering
	\includegraphics[width=.95\columnwidth]{DiMaggio_fig_5.png}
	\caption{Fledging rate of lesser kestrel as a function of nest type and nest orientation (GLMM, N = 45)}
	\label{DiMaggio_fig_5}
\end{figure}

\vspace{1cm}\null

\begin{table}[!h]
\centering
\begin{tabular}{>{\raggedright\arraybackslash}p{.2\columnwidth}>{\raggedleft\arraybackslash}p{.2\columnwidth}>{\raggedleft\arraybackslash}p{.2\columnwidth}>{\raggedleft\arraybackslash}p{.2\columnwidth}}
\toprule
\textbf{Explanatory variable} & \textbf{Parameter estimate} & \textbf{F} & \textbf{P} \\
\toprule
%\showrowcolors
Intercept & 20.588 & & 1.000  \\
Colony location & 0.483 & 1.628 & 0.202 \\
Nest type & 1.242 & 612.208 & \textbf{< 0.001} \\
Nest orientation & 0.073 & 3.614 & 0.570 \\
\bottomrule
\hiderowcolors
\end{tabular}
\caption{Effects of colony location, nest type and nest orientation on temperature ($^\circ$C) measured inside lesser kestrel nests (GLMM, N = 45). In bold variables that significantly predicted nest temperatures}
\label{DiMaggio_tab_1}
\end{table}

\vspace{1cm}\null

\begin{table}[!h]
\centering
\begin{tabular}{>{\raggedright\arraybackslash}p{.2\columnwidth}>{\raggedleft\arraybackslash}p{.2\columnwidth}>{\raggedleft\arraybackslash}p{.2\columnwidth}>{\raggedleft\arraybackslash}p{.2\columnwidth}}
\toprule
\textbf{Explanatory variable} & \textbf{Parameter estimate} & \textbf{F} & \textbf{P} \\
\toprule
%\showrowcolors
Intercept & 51.922 & & 0.999 \\
Colony location & -0.892 & 0.445 & 0.505 \\
Nest type & -2.557 & 293.127 & \textbf{< 0.001} \\
Nest orientation & 6.366 & 30.905 & \textbf{< 0.001} \\
\hiderowcolors
\bottomrule
\end{tabular}
\caption{Effects of colony location, nest type and nest orientation on relative humidity (\%) measured inside lesser kestrel nests (GLMM, N = 45). In bold: variables that significantly predicted relative humidity}
\label{DiMaggio_tab_2}
\end{table}

\newpage
\section*{Acknowledgements}

We greatly thank Stefano Triolo, Jo\"elle Tysseire and Laura Zanca, for
assistance during field work. This project was supported by the Italian
Ministry of Education, University and Research (PRIN 2010/2011, 20108
TZKHC).


\section*{Bibliography}
\begin{itemize}\itemsep0pt
	\item Catry I., Franco A. M. A. \& Sutherland W. J., 2011 - Adapting
conservation efforts to face climate change: Modifying nest-site
provisioning for Lesser Kestrels.\textit{ Biol. Cons.}, 144 (3):
1111-1119.

	\item Di Maggio R., Campobello D. \& Sar\`a M., 2013 - Nest aggregation and
reproductive synchrony promote Lesser Kestrel \textit{falco naumanni}
seasonal fitness.\textit{ J. of Ornithol.,} 154: 901--910.

\item Di Maggio R., Mengoni C., Mucci N., Campobello D., Randi E. \& Sar\`a
M., 2014 - Do not disturb the family: roles of colony size and human
disturbance in the genetic structure of Lesser Kestrel. \textit{J.
Zool., }In press.

	\item I\~nigo A. \& Barov B., 2011 - \textit{Action plan for the Lesser
Kestrel Falco naumanni in the European Union}. SEO-BirdLife \& BirdLife
International for the European Commission.

	\item Lloyd J. D. \& Martin T. E., 2004 - Nest-site preference and maternal
effects on offspring growth.\textit{ Behav. Ecol., 15 }(5): 816-823.

	\item {Martin T. E., 1998 - Are microhabitat preferences of
coexisting species under selection and adaptive?
}\textit{{Ecology}}{, 79: 656--670.}

	\item McCullagh P. \& Searle S. R., 2000 - \textit{Generalized Linear and
Mixed Models.} Wiley-Interscience, New York.

	\item Robert M., Vaillancourt M. A. \& Drapeau P., 2010 - Characteristics of
nest cavities of Barrow{\textquoteright}s Goldeneyes in eastern Canada.
\textit{J. Field Ornithol.,} 81: 287--293.

	\item Sar\`a M., 2010 - Climate and land-use changes as determinants of Lesser
Kestrel \textit{falco naumanni} abundance in mediterranean cereal
steppes (Sicily).\textit{ Ardeola, }57(SPEC. DECEMBER): 3-22.

	\item Sar\`a M., Campobello D. \& Zanca L., 2012 - Effects of nest and colony
features on Lesser Kestrel (\textit{falco naumanni}) reproductive
success.\textit{ Avian Biol. Res., }5 (4): 209-217.
\end{itemize}
