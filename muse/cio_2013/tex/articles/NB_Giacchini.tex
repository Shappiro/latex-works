\setcounter{figure}{0}
\setcounter{table}{0}

\begin{adjustwidth}{-3.5cm}{0cm}
\pagestyle{CIOpage}
\authortoc{\textsc{Giacchini P.}}
\chapter*[Diversit\`a ornitica basso-montana di Fiordimonte (MC -
Marche)]{\bfseries
L{\textquoteright}importanza degli ambienti aperti basso-montani per la
biodiversit\`a ornitica nelle marche: il caso di Fiordimonte (MC)}
\addcontentsline{toc}{chapter}{Diversit\`a ornitica basso-montana di Fiordimonte (MC - Marche)}

\textsc{Paolo Giacchini}$^{1*}$ \\

\index{Giacchini Paolo}
\noindent\color{MUSEBLUE}\rule{27cm}{2pt}
\vspace{1cm}
\end{adjustwidth}



\marginnote{\raggedright $^1$Hystrix srl - via Castelfidardo 7 -- 61032 Fano (PU)  \\
\vspace{.5cm}
{\emph{\small $^*$Autore per la corrispondenza: \href{mailto:paolo.giacchini@hystrix.it}{pao\allowbreak lo.\allowbreak giac\allowbreak chi\allowbreak ni@\allowbreak hys\allowbreak trix.\allowbreak it}}} \\
\keywords{Marche, biodiversit\`a ornitica, comunit\`a nidificante}
{Marche region, bird population biodiversity, breeding
population}
}
{\small
\noindent \textsc{\color{MUSEBLUE} Summary} / This study represents an in-depth analysis of the bird population
biodiversity nesting in a low mountainous area of the Marche region.
The breeding population of \textit{Lanius collurio} showing a maximum
density of 0,65 territorial males/10 ha; \textit{Coturnix coturnix,
Lullula arborea, Alauda arvensis, Emberiza citrinella, Miliaria
calandra} have been considered of particular interest.
}



\section*{Introduzione}

Il presente lavoro rappresenta un approfondimento
dell{\textquoteright}analisi della biodiversit\`a ornitica nidificante
in un{\textquoteright}area basso-montana della regione Marche.

L{\textquoteright}area di studio riguarda un territorio di circa 460 ha
nei comuni di Fiastra e Fiordimonte (MC), al confine con il Parco
Nazionale dei monti Sibillini e con il SIC {\textquotedblleft}val di
Fibbia -- valle dell{\textquoteright}Acquasanta{\textquotedblright}. Si
tratta di una dorsale che sale da 850 m s.l.m. fino a circa 1400 m
s.l.m., caratterizzata da coltivi (orzo, erba medica, erbai polifiti)
nella parte basale, macchie boscate nei fossi e nelle pendenze meno
coltivabili, pascoli pi\`u o meno cespugliati nella parte intermedia
fino a quelli sommitali. Presenti ma in modo ridotto, gli affioramenti
rocciosi.
L{\textquoteright}area \`e compresa in parte in
un{\textquoteright}azienda agri-turistico-venatoria (Fiordimonte), con
all{\textquoteright}interno una zona addestramento cani (ZAC) di tipo
C. 

\section*{Metodi}

Le osservazioni sono state condotte nel periodo maggio-luglio 2007,
ripetute a maggio-giugno 2012 e 2013, con sessioni di ascolto al canto
e osservazione diretta da punti fissi e su transetti lineari. 

\section*{Risultati e discussione}

I rilevamenti hanno individuato, in particolare, la seguente comunit\`a
ornitica nidificante (Tab. \ref{Giacchini_tab_1}).

Tra gli elementi di maggior interesse spicca la popolazione di averla
piccola, ancora diffusa come nidificante nelle Marche, pur avendo
mostrato locali, evidenti, contrazioni numeriche. 

La consistenza rilevata nel 2007, gi\`a elevata, \`e aumentata nel 2012
giungendo a 30 maschi territoriali in aree aperte cespugliate con
\textit{Rosa canina, Juniperus communis, Prunus spinosa}, che usano in
prevalenza per la riproduzione. La densit\`a rilevata \`e di 0,50
maschi territoriali /10 ha nel 2007 e di 0,65 maschi/10 ha nel 2012;
considerando singole aree a maggior densit\`a, si arriva fino a valori
di 2,07 maschi/10 ha (19 maschi territoriali in un{\textquoteright}area
di 92 ha nel 2012). 

La distribuzione altitudinale dei territori va da 870 a 1330 m s.l.m. A
differenza di quanto rilevato in altre aree marchigiane (Morelli \&
Pandolfi 2011), non sembra esservi un rapporto privilegiato dei siti
riproduttivi con il reticolo stradale, n\'e con abitazioni rurali. 

Importante \`e anche la presenza di altre specie non comuni in ambito
regionale, tra cui lo zigolo giallo, con almeno 9 siti riproduttivi, e
le concentrazioni di quaglia, allodola, tottavilla e strillozzo. Per
quaglia e allodola si \`e riscontrata una certa contrazione nel 2012,
pur mostrando ancora una presenza diffusa.

Lo studio ha evidenziato la ricchezza ornitica di
quest{\textquoteright}area; nonostante la presenza di un istituto di
attivit\`a venatoria, ma anche per la vicinanza degli istituti di
protezione (in particolare il Parco Nazionale dei monti Sibillini), le
consistenze delle popolazioni di specie legate ad ambienti aperti
appaiono di estremo interesse, essendo non del tutto comuni ad altre
aree marchigiane simili, esterne ad aree protette. 

\begin{table}[!h]
\centering
\begin{tabular}{>{\raggedright\arraybackslash}p{.4\columnwidth}>{\raggedright\arraybackslash}p{.2\columnwidth}>{\raggedright\arraybackslash}p{.2\columnwidth}}
\toprule
\textbf{Specie} & \textbf{N. coppie 2007} & \textbf{N. coppie 2012} \\
\toprule
%\showrowcolors
Quaglia comune \textit{Coturnix coturnix} & circa 30 & circa 20 \\
Gheppio \textit{Falco tinnunculus} & 1 & ? \\
Tottavilla \textit{Lullula arborea} & >30 & circa 20 \\
Allodola \textit{Alauda arvensis} & >120 & circa 40 \\
Calandro \textit{Anthus campestris} & 4 & 2 \\
Codirossone \textit{Monticola saxatilis} & 1 & ? \\
Sterpazzola \textit{Sylvia communis} & 4 & 4 \\
Culbianco \textit{Oenanthe oenanthe} & 5 & 1 \\
Averla piccola \textit{Lanius collurio} & 23 & 30 \\
Zigolo giallo \textit{Emberiza citrinella} & 9 & 3 \\
Strillozzo \textit{Miliaria calandra} & 45-60 & 30-40 \\
\bottomrule
\hiderowcolors
\end{tabular}
\caption{Comunit\'a ornitica di maggior interesse nidificante nell{\textquoteright}area di studio}
\label{Giacchini_tab_1}
\end{table}

\section*{Ringraziamenti}
Si ringrazia Pietro Spadoni per la partecipazione ai rilevamenti.

\section*{Bibliografia}
\begin{itemize}\itemsep0pt
	\item Morelli F. \& Pandolfi M., 2011 - Breeding habitat and nesting site of
the red-backed shrike \textit{Lanius collurio} in farmland of the
Marche region, Italy. \textit{Avocetta}, 35: 43-49.
\end{itemize}
