\setcounter{figure}{0}
\setcounter{table}{0}

\begin{adjustwidth}{-3.5cm}{0cm}
\pagestyle{CIOpage}
\addtocontents{toc}{\protect\newpage}
\authortoc{\textsc{Massari S.}, \textsc{Grattini N.},
\textsc{Fiozzi A.}, \textsc{Ferando E.},
\textsc{Cavaletti E.}}
\chapter*[Analisi delle presenze presso il CRAS Parcobaleno]{\bfseries
Analisi delle presenze ornitiche del Centro recupero (CRAS) di
Parcobaleno (Mantova) - periodo 2003/2012}
\addcontentsline{toc}{chapter}{Analisi delle presenze presso il CRAS Parcobaleno}

\textsc{Simone Massari}$^{1*}$, \textsc{Nunzio Grattini}$^{2}$,
\textsc{Andrea Fiozzi}$^{1}$, \textsc{Eleonora Ferando}$^{1}$,
\textsc{Enrico Cavaletti}$^{1}$\\

\index{Massari Simone} \index{Grattini Nunzio} \index{Fiozzi Andrea} \index{Ferando Eleonora} \index{Cavaletti Enrico}
\noindent\color{MUSEBLUE}\rule{27cm}{2pt}
\vspace{1cm}
\end{adjustwidth}



\marginnote{\raggedright $^1$Centro recupero fauna selvatica WWF Parcobaleno -
Parcobaleno, via Guerra 4/B, 46100 Mantova \\
$^2$Gruppo Ricerche
Avifauna Mantovano (GRAM) - Centro visite del Parco San Lorenzo, Strada
Falconiera, 46020 Pegognaga~(MN) \\
\vspace{.5cm}
{\emph{\small $^*$Autore per la corrispondenza: \href{mailto:simomassari@inwind.it}{simomassari@inwind.it}}} \\
\keywords{Parcobaleno, Mantova, Lombardia,
rapaci, Centro recupero}
{Parcobaleno, Mantua, Lombardy, birds of prey,
recovery}
%\index{keywords}{Parcobaleno} \index{keywords}{Mantova} \index{keywords}{Lombardia} \index{keywords}{Rapaci} \index{keywords}{Centro recupero}
}
{\small
\noindent \textsc{\color{MUSEBLUE} Summary} / The wildlife recovery centre Parcobaleno of Mantova (Lombardy), located
in an abandoned municipal area, in the decade 2003-2012 hosted 865
individuals of 74 different species. Data underline a significant
increase in the number of birds from 2003 to 2012, with a positive peak
in the 2010. The annual average of recovered animals is of 86.5
individuals, with significant fluctuations. Standard deviation over
years is higher for the number of individuals (35.75) than for the
number of species (8,19). The most represented species are:
\textit{Athene noctua }(255 individuals), \textit{Falco tinnunculus}
(160 individuals), \textit{Asio otus} (68 individuals) and
\textit{Strix aluco} (32 individuals), an interesting sample to be used
for ornithological surveys. Crossing provenience data with soil use
maps, it has been discovered a correlation between arrival frequency of
birds from a particular area and the possible abundance of habitat
suitable for their breeding and survival. The majority of little owl
individuals come from the municipalities of Mantova and Roncoferraro
where it is plenty of large old historical and rural buildings. The
kestrel, instead, appears to be more uniformly distributed (Grattini
2008), even if it prefers rural landscapes (Grattini \& Longhi 2010). \\
}
\section*{Introduzione}
Negli anni '90 il WWF locale di Mantova ha elaborato
una proposta di gestione di un{\textquoteright}area comunale
abbandonata per garantire, tra le altre attivit\`a, la costituzione di
un Centro di Recupero della Fauna Selvatica in difficolt\`a. Per
attuare tale proposta ha costituito l'Associazione
Anticitt\`a, che ne \`e a tutt{\textquoteright}oggi il gestore. Nasce
cos\`i il centro di educazione ambientale di PARCOBALENO, situato in
prossimit\`a del Parco storico di Bosco Virgiliano e al confine con la
riserva Naturale della {\textquotedblleft}Vallazza{\textquotedblright}.
Un'area che si presta al reinserimento in natura degli
animali riabilitati.

\section*{Metodi}
In questo lavoro vengono presentati i dati delle presenze ornitiche
ospitate al CRAS (Centro Recupero Animali Selvatici) di Parcobaleno,
nel periodo 2003-12, analizzando l{\textquoteright}andamento de: il
numero di uccelli ricoverati nei 10 anni; il numero di specie totale;
il numero di specie suddiviso per anno nonch\'e una breve analisi
territoriale per le specie pi\`u ricorrenti. 

\section*{Risultati e discussione}
In totale sono stati ricoverati 865 individui, appartenenti a 74 specie.
I dati evidenziano l{\textquoteright}aumento significativo del numero
di uccelli ospitati dall{\textquoteright}anno di apertura (2003) al
2012, con un numero massimo registrato nel 2010.\textcolor{red}{ }

La media annua delle presenze \`e di 86,5 individui, anche se nel
periodo analizzato, si sono manifestate delle oscillazioni
significative. La d.s. misurata tra gli anni \`e maggiore per il valore
del numero di individui (35,75) che per il valore del numero di specie
(8,19). 

Fra le specie pi\`u numerose si ricordano: civetta comune \textit{Athene
noctua}, gheppio \textit{Falco tinnunculus}, gufo comune \textit{Asio
otus} e allocco \textit{Strix aluco}, un interessante campione
potenzialmente utilizzabile per eventuali studi ornitologici. 

Nel corso dei 10 anni, sono giunti al Centro 255 civette, 160 gheppi, 68
gufi comuni e 32 allocchi. Incrociando i dati di provenienza con carte
dell{\textquoteright}uso del suolo si sono individuate delle
correlazioni tra la frequenza dei ricoveri provenienti da una
determinata area e la presenza di habitat adatti alla sopravvivenza e
alla riproduzione della specie. 

A titolo di esempio, dalle analisi dei ritrovamenti della civetta
comune, il maggior numero di individui o nidiate della specie
provengono maggiormente da Mantova e Roncoferraro (sono esclusi
dall{\textquoteright}elaborazione dei dati i soggetti provenienti dai
comuni veronesi ed emiliani di cui non si dispongono dati
sull'utilizzo del suolo). Ipotizziamo che questi
territori forniscano numerosi siti adatti alla nidificazione per la
presenza di centri storici di una certa dimensione (come a Mantova) o
edifici rurali diffusi, come nel caso di Roncoferraro e nei Comuni di
Suzzara, Bagnolo S. Vito e S. Benedetto Po; sembra invece che
l{\textquoteright}area morenica sia meno adatta alla presenza della
civetta comune. Dai dati raccolti emerge tuttavia che la specie \`e
presente in tutto il territorio provinciale. 

Il gheppio \textit{Falco tinnunculus,} rispetto alla civetta comune,
risulterebbe omogeneamente distribuito (sono esclusi
dall{\textquoteright}elaborazione i comuni veronesi ed emiliani),
probabilmente per la sua attitudine a nidificare in nidi abbandonati di
\textit{Corvidae}. Non si individua infatti un{\textquoteright}area
d{\textquoteright}elezione, a conferma di quanto riscontrato
recentemente nel mantovano (Grattini 2008). Anche per il gheppio
tuttavia, sembrano essere privilegiati i contesti rurali (comuni come
Marcaria, Suzzara, S. Benedetto Po, Marmirolo) che probabilmente
offrono sia siti di nidificazione, sia importanti zone di alimentazione
(Grattini \& Longhi  2010).  

Questa preliminare analisi, evidenzia come i CRAS possano contribuire,
nel corso degli anni, alla costituzione di una buona base di dati
ornitologici, che sono utilizzabili per studi sulle dinamiche di
popolazione di alcune specie ornitiche di interesse conservazionistico
(rapaci diurni e notturni \textit{in primis}). 

\section*{Bibliografia}
\begin{itemize}
	\item Grattini N.,\textsc{ 2008 - }Distribuzione, consistenza ed espansione
territoriale del Gheppio \textit{Falco} \textit{tinnunculus}
nidificante in Provincia di Mantova.
\textbf{\textmd{\textit{Alula}}}\textbf{\textmd{, Volume XV (1/2):
189-194.}}

	\item Grattini N. \& Longhi D\textsc{., 2010. - }Avifauna del mantovano
(Lombardia, Italia settentrionale).\textit{ Nat. Bresc.}, 37: 143-181.
\end{itemize}
