\setcounter{figure}{0}
\setcounter{table}{0}

\begin{adjustwidth}{-3.5cm}{0cm}
\pagestyle{CIOpage}
\authortoc{\textsc{Basile M.}, \textsc{Balestrieri R.},
\textsc{Buoninconti F.}, \textsc{Capobianco G.},
 \textsc{Altea T.}, \textsc{Matteucci G.},
\textsc{Posillico M.}}
\chapter*[Catturabilit\`a del rampichino comune]{
\textbf{Studio preliminare sulla catturabilit\`a del rampichino comune
}\textbf{\textit{Certhia brachydactyla}}}
\addcontentsline{toc}{chapter}{Catturabilit\`a del rampichino comune}


\textsc{Marco Basile}$^{1}$, \textsc{Rosario Balestrieri}$^{1*}$,
\textsc{Francesca Buoninconti}$^{1}$, \textsc{Giovanni Capobianco}$^{1}$,
 \textsc{Tiziana Altea}$^{2}$, \textsc{Giorgio Matteucci}$^{1}$,
\textsc{Mario Posillico}$^{1,2}$ \\

\index{Basile Marco} \index{Balestrieri Rosario} \index{Buoninconti Francesca} \index{Capobianco Giovanna} \index{Altea Tiziana} \index{Matteucci Giorgio} \index{Posillico Mario}
\noindent\color{MUSEBLUE}\rule{27cm}{2pt}
\vspace{1cm}
\end{adjustwidth}



\marginnote{\raggedright $^1$Istituto di Biologia Agroambientale e
Forestale del CNR Via Salaria km 29.3, 00015 Monterotondo Scalo RM \\
$^2$Corpo Forestale dello Stato, Ufficio
Territoriale Biodiversit\`a di Castel di Sangro, Via Sangro, 45-67031
Castel di Sangro (AQ) \\
{\textquotedblleft}L{\textquoteright}Assiolo{\textquotedblright}, via
Donizetti Loc. Ronchi, 54100 Marina di Massa (MS), Italy \\
\vspace{.5cm}
{\emph{\small $^*$Autore per la corrispondenza: \href{mailto:rosario.balestrieri@ibaf.cnr.it}{ro\allowbreak sa\allowbreak rio.\allowbreak ba\allowbreak les\allowbreak trie\allowbreak ri@\allowbreak i\allowbreak baf.\allowbreak cnr.\allowbreak it}}} \\
\keywords{Foresta demaniale regionale
{\textquotedblleft}Feudozzo{\textquotedblright}, \textit{Certhia brachydactyla}, catturabilit\`a}
{Feudozzo regional forest, Certhia brachydactyla, catchability}
%\index{keywords}{Foresta demaniale regionale {\textquotedblleft}Feudozzo{\textquotedblright}} \index{keywords}{\textit{Certhia brachydactyla}} \index{keywords}{Catturabilit\`a}
}
{\small
\noindent \textsc{\color{MUSEBLUE} Summary} / The short-toed treecreeper \textit{Certhia brachydactyla} is a bird
species strictly linked to old-growth forests. Nest and foraging sites,
indeed, are due to cavity, dead snag, ivy and other typical old-growth
forest characteristics. Thus, it could be an eligible species for
forest management research, as it is within the project LIFE ManFor
CBD. In order to do this kind of research in the near future, we need
to test the catchability of the short-toed treecreeper. This study was
carried out in the {\textquotedblleft}Feudozzo{\textquotedblright}
regional Forest, an old-growth forest of oak \textit{Quercus Cerris},
with relevant presence of beech \textit{Fagus sylvatica}. Catching
protocol consisted in the use of three 6-meters mist-nets, forming a
triangle around a tree, to which the base is located a playback-speaker
playing the treecreeper song, for a total of three triangles. In the
two-days session five individuals were captured plus one more
individual recaptured. Among these, four were born in the current
season. Capture frequency resulted in 0.4 individual/hour of capture.
}


\section*{Introduzione}
Il rampichino comune \textit{Certhia brachydactyla} \`e una specie
ornitica prettamente forestale, la cui presenza \`e strettamente
correlata a quella di alberi vetusti (Brichetti \& Fracasso 2011).
L{\textquoteright}area di nidificazione, in particolare, \`e legata a
quei caratteri tipici delle foreste miste, quali l{\textquoteright}alta
densit\`a di cavit\`a negli alberi, gli intricati fasci
d{\textquoteright}edera, nonch\'e (a una diversa scala spaziale) la
fitta copertura arborea (Cramp \& Perrins 1994). Inoltre il territorio
di nidificazione risulta particolarmente ristretto, con i casi estremi
registrati di: 0.4 ha in Germania orientale e 7.9 ha sulle Alpi
Marittime (Cramp \& Perrins 1994). Infine la specie \`e sedentaria e in
genere effettua limitati spostamenti, salvo il \textit{dispersal
}giovanile, comunque limitato (Cramp \& Perrins 1994). Date queste
peculiarit\`a, il rampichino comune \`e un buon soggetto di studio nel
contesto dell{\textquoteright}ecologia e gestione forestale. I
parametri demografici e la stima di popolazione potrebbero fornire
interessanti informazioni sugli effetti a breve e medio termine degli
interventi forestali. Nell{\textquoteright}ottica di effettuare studi
in tal senso, nell{\textquoteright}ambito del LIFE ManFor CBD, \`e
stato svolto nella Foresta demaniale regionale
{\textquotedblleft}Feudozzo{\textquotedblright} (Castel di Sangro, AQ)
uno studio preliminare sulla catturabilit\`a del rampichino comune, al
fine di valutare l{\textquoteright}efficacia delle trappole e lo sforzo
di campo.

\section*{Metodi}
La Foresta demaniale regionale
{\textquotedblleft}Feudozzo{\textquotedblright} \`e una foresta mista
di caducifoglie. La comunit\`a arborea \`e riconducibile alla cerreta
\textit{Quercus cerris}, con importanti presenze di faggio
\textit{Fagus sylvatica }nelle aree sommitali. Le trappole sono
costituite da tre \textit{mist-net }(maglia 16 mm, lunghezza 6 m,
altezza 2.4 m), montate a formare un triangolo intorno a un albero.
Alla base dell{\textquoteright}albero \`e stato posizionato un richiamo
elettroacustico per attrarre i rampichini aumentando le probabilit\`a
di cattura. Il richiamo elettroacustico proveniva da un lettore MP3
(Olympus DM-550), che emetteva, in modalit\`a ripetizione, un file
audio contenente il canto della specie. Le tre trappole sono state
posizionate lungo un transetto di circa 425 m (TR1 -- TR2 = 208 m; TR1
-- TR3 = 245 m; TR2 -- TR3 = 425 m), in maniera opportunistica, in
un{\textquoteright}area in cui era stata precedentemente rilevata la
specie. Le trappole sono state attive dal 9 al 10 agosto 2012, per un
totale di 15 ore. Le trappole sono state attive
dall{\textquoteright}alba alle 11, e dalle 16:30 al tramonto e
controllate ogni ora. Ogni rampichino catturato \`e stato marcato con
un anello ISPRA. Inoltre sono state prese le seguenti misure
biometriche: lunghezza della terza remigante primaria, corda massima
dell{\textquoteright}ala, lunghezza del becco, lunghezza del tarso,
peso; e le seguenti misure visive: et\`a, sesso (quando possibile),
accumulo di grasso, sviluppo dei muscoli pettorali. Tutte le misure
sono state prese, da un inanellatore abilitato, seguendo le direttive
ISPRA e con la strumentazione ufficiale. Lo sforzo di campo, invece,
\`e stato sostenuto da quattro persone, per un totale di 16:30 ore
(4.10 ore cadauno, compresa installazione e disinstallazione della
trappola). Le frequenze orarie di cattura sono state calcolate come
numero di catture/ore di funzionamento trappola, da cui poi \`e stata
ricavata la media aritmetica tra le trappole. La dimensione della
popolazione \`e stata stimata in maniera esplorativa secondo il modello
a variabilit\`a temporale Mt (EE), implementato dal software CARE-2
(Chao \& Yang 2003).

\section*{Risultati e discussione}
Sono stati marcati 5 individui, di cui uno ricatturato in una successiva
sessione, per un totale di 6 catture. La frequenza oraria media delle
catture \`e 0.4 catture/ora. La frequenza oraria media di cattura non
\`e risultata costante poich\'e in tutte le trappole i rampichini sono
stati catturati entro 2 ore dalla prima emissione del richiamo. Inoltre
l{\textquoteright}unica cattura pomeridiana \`e la ricattura di un
individuo inanellato 11 ore prima, nella stessa trappola. Infatti,
considerando le due trappole attive per due giorni (TR1-TR2), 4 catture
su 5 sono state effettuate il primo giorno. Dei 5 individui marcati, 4
sono risultati essere giovani nati nell{\textquoteright}anno in corso
(et\`a euring: 3), mentre 1 risulta essere adulto (et\`a euring: 4).
Con questi dati la popolazione \`e stata stimata in 9.8 individui
(IC\textsubscript{95\%} = 5 -- 15.6).~Tuttavia, trattandosi di uno
studio preliminare, volto a testare l{\textquoteright}efficacia del
metodo di cattura, l{\textquoteright}area effettivamente coperta dalle
trappole non \`e stimabile.

Nonostante il breve arco temporale indagato, sono stati ottenuti
risultati interessanti, innanzitutto l{\textquoteright}alta rispondenza
(risposta al richiamo e cattura) degli individui nelle prime ore del
mattino, dovuta alla spiccata territorialit\`a (Brichetti \& Fracasso
2011).  Sulla base delle esperienze effettuate, si suggerisce
l{\textquoteright}emissione in \textit{playback} di varie tipologie di
canto, riprodotte in maniera casuale per la realizzazione di ulteriori
sessioni di cattura sulla specie. Considerando, quindi, che
nell{\textquoteright}area indagata sono state ottenute 5 catture in tre
trappole diverse nelle prime 5 ore di attivit\`a, si pu\`o affermare
che, aumentando lo sforzo di campo, si potrebbe catturare buona parte
della popolazione nidificante entro una determinata area boscata.

\section*{Bibliografia}
\begin{itemize}\itemsep0pt
\item {
Brichetti P. \& Fracasso G., 2011- \textit{Ornitologia Italiana vol. 7 -
Paridae-Corvidae}. Oasi Alberto Perdisa Editore, Bologna, 493 pp.}

	\item {
Cramp S. \& Perrins C.M., 1994 - \textit{The Birds of the Western
Palearctic} Volume VIII. Oxford University Press, Oxford, New York, 899
pp. }

	\item {
Chao A. \& Yang H.C., 2003 - Program CARE-2 (for Capture-Recapture Part.
2). Program and user{\textquotesingle}s guide published at
\url{http://chao.stat.nthu.edu.tw}}
\end{itemize}
