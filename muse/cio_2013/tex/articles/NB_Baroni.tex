\setcounter{figure}{0}
\setcounter{table}{0}

\begin{adjustwidth}{-3.5cm}{0cm}
\pagestyle{CIOpage}
\authortoc{\textsc{Baroni D.}, \textsc{Rapetti C.}}
\chapter*[L{\textquoteright}avifauna delle acque costiere]{L{\textquoteright}avifauna delle acque costiere del Mar Ligure centro-occidentale}
\addcontentsline{toc}{chapter}{L{\textquoteright}avifauna delle acque costiere}

\textsc{Daniele Baroni}$^{1*}$, \textsc{Carla Rapetti}$^{2}$\\

\index{Baroni Daniele} \index{Rapetti Carla}
\noindent\color{MUSEBLUE}\rule{27cm}{2pt}
\vspace{1cm}
\end{adjustwidth}



\marginnote{\raggedright $^1$Via Gaspare Buffa 4, 16158 -- Genova (GE) \\
$^2$Via Trento 14, 16011 -- Arenzano (GE) \\
\vspace{.5cm}
{\emph{\small $^*$Autore per la corrispondenza: \href{mailto:dbaroni12@gmail.com}{dbaroni12@gmail.com}}} \\
\keywords{Mar Ligure, uccelli acquatici, selezione
dell{\textquoteright}habitat}
{Ligurian Sea, diving birds, habitat selection}
%\index{keywords}{Mar Ligure} \index{keywords}{Uccelli acquatici} \index{keywords}{Selezione dell{\textquoteright}habitat}
}
{\small
\noindent \textsc{\color{MUSEBLUE} Summary} / Phenology and habitat selection of water birds in marine waters has been
surveyed in winter along a coastal transect of 62 km. During foraging
activity, harbors were preferred mostly by great crested grebe;
red-breasted merganser consistently selected reefs; arctic loon
selected waters along the shores.
}
\vspace{1cm}
\section*{Introduzione}
La selezione dell{\textquoteright}habitat e la fenologia di alcune
specie di uccelli acquatici sono state indagate in un settore del Mar
Ligure centro-occidentale; a livello regionale questi aspetti non sono
noti nel dettaglio e informazioni pregresse si trovano ad esempio in
Gorlier (1975); Andreotti \textit{et al.} (1991);  Borgo \textit{et
al.} (1991); Span\`o \textit{et al.} (1998); Ballardini \textit{et al.}
(2005). I dati sono stati raccolti percorrendo transetti costieri: tra
Genova Pra{\textquoteright} (GE) e Finale Ligure (SV), per un totale di
62 km, con frequenza mensile e tra Genova Pra{\textquoteright} (GE) e
Vado Ligure (SV), per un totale di 42 km, ogni decade.
L{\textquoteright}area di studio \`e caratterizzata da una forte
componente antropica e include due ampie aree portuali (Genova Voltri e
Savona), tratti di costa rocciosa e litorali ciottolosi o ghiaiosi. Il
periodo d{\textquoteright}indagine comprende il semestre ottobre-marzo,
dal 2008 al 2011 e la durata di ogni uscita ha interessato quasi tutte
le ore di luce, al fine di consentire il maggior rilevamento possibile
delle presenze. I transetti sono stati percorsi in macchina,
effettuando soste distribuite lungo il percorso in modo da monitorare
l{\textquoteright}intera area di studio mediante
l{\textquoteright}utilizzo di strumenti ottici (cannocchiale e
binocolo). L{\textquoteright}ordine progressivo con il quale sono stati
censiti i diversi settori dell{\textquoteright}area \`e stato
diversificato nell{\textquoteright}orario delle osservazioni,  al fine
di compensare l{\textquoteright}effetto delle variazioni di attivit\`a
degli uccelli durante il giorno. La selezione
dell{\textquoteright}habitat \`e stata quindi indagata mediante test
del $\tilde\chi^2$ al fine di verificare se le frequenze osservate e
attese in ogni habitat differissero significativamente. In caso
affermativo \`e stata utilizzata una statistica \textit{z} di
Bonferroni al fine di individuare quali tipologie fossero selezionate
positivamente o negativamente (Neu \textit{et al.} 1974).

\section*{Risultati e discussione}

Su 1826 osservazioni relative a 58 specie di uccelli acquatici, sono
state analizzate 4 specie meglio rappresentate e pi\`u significative a
livello di distribuzione, appartenenti alla \textit{guild} trofica dei
{\textquotedblleft}tuffatori{\textquotedblright}.
L{\textquoteright}elevata diversit\`a di habitat marini cartografati
nell{\textquoteright}area (Diviacco \& Coppo 2006) \`e stata ricondotta
a tre sole tipologie (Tab. \ref{Baroni_tab_1}). In tutti i casi le osservazioni hanno
evidenziato un utilizzo dell{\textquoteright}habitat che si discosta
dall{\textquoteright}atteso calcolato sulla disponibilit\`a ambientale
in termini di estensione (P{\textless}0.0001; Tab. \ref{Baroni_tab_1}). Lo smergo minore
\textit{Mergus serrator} evidenzia un massimo del numero di individui
nei mesi di gennaio e di dicembre. Analizzando le sole localizzazioni
relative a soggetti in attivit\`a trofica (N = 25) la costa rocciosa a
ridosso degli scogli \`e risultata essere la tipologia ambientale
fortemente selezionata, mentre l{\textquoteright}interno di strutture
portuali, lungo le banchine, \`e stato frequentato in minor misura e
non risulta selezionato attivamente. La maggior parte delle
segnalazioni (43\%) di strolaga mezzana \textit{Gavia arctica} \`e
relativa al mese di gennaio e le acque antistanti tratti di spiaggia
costituiscono l{\textquoteright}unica tipologia ambientale selezionata
dagli individui in alimentazione (N = 61). Lo svasso maggiore
\textit{Podiceps cristatus} evidenzia un picco delle segnalazioni in
febbraio (40\%), mentre per ci\`o che concerne gli individui in
attivit\`a trofica (N = 91) si evidenzia una preferenza per porti e
porticcioli. Infine, l{\textquoteright}analisi quadriennale dei dati
raccolti mensilmente sul cormorano \textit{Phalacrocorax carbo} sui 62
km di costa permette un{\textquoteright}analisi fenologica che
evidenzia maggiore variabilit\`a nei quantitativi osservati in
corrispondenza dei periodi migratori (ottobre e marzo) e un massimo in
gennaio ( ${x}$ = 46.7{\textpm}15.6).

La presenza massima di acquatici
{\textquotedblleft}tuffatori{\textquotedblright} nei mesi di gennaio e
febbraio \`e in linea con la fenologia nazionale delle specie
(Brichetti \& Fracasso 2003). Per ci\`o che concerne la selezione
dell{\textquoteright}habitat a fini trofici si evidenzia come la
predilezione di alcune specie per aree antropizzate (es. porticcioli)
sia probabilmente in funzione dell{\textquoteright}assenza, nel
contesto ligure, degli ambienti naturali d{\textquoteright}elezione per
le specie considerate, quali ad esempio le lagune costiere.



\newcolumntype{S}{>{\centering\arraybackslash}p{.1\columnwidth}}
\newcolumntype{s}{>{\centering\arraybackslash}p{.04\columnwidth}}
\newcolumntype{B}{>{\centering\arraybackslash}p{.12\columnwidth}}

\begin{adjustwidth}{1cm}{1cm}
\begin{table}[!h]
\centering
\footnotesize
\scalebox{.8}{
\begin{tabular}{>{\raggedright\arraybackslash}p{.15\columnwidth}ssBS|ssBS|ssBS}
\hiderowcolors
\toprule
& \multicolumn{4}{c}{\textbf{\textit{Podiceps cristatus}}} & \multicolumn{4}{c}{\textbf{\textit{Mergus serrator}}} & \multicolumn{4}{c}{\textbf{\textit{Gavia arctica}}} \\
\toprule
\textbf{Tipologie ambientali} & \textbf{pi att.} & \textbf{pi oss.} & \textbf{I.C. 95\% corr. Bonferroni}
& \textbf{Selezione} & \textbf{pi att.} & \textbf{pi oss.} & \textbf{I.C. 95\% corr. Bonferroni}
& \textbf{Selezione} & \textbf{pi att.} & \textbf{pi oss.} & \textbf{I.C. 95\% corr. Bonferroni}
& \textbf{Selezione} \\
\toprule
Ambiti portuali & 0,20 & 0,65 & 0,53-0,77 & selez. & 0,20 & 0,28 & 0,06-0,50 & non selez. & 0,20 & 0,05 & 0-0,12 & evitato \\
Acque antistanti spiagge & 0,47 & 0,25 & 0,14-0,36 & evitato & 0,47 & 0,00 & - & evitato & 0,47 & 0,84 & 0,73-0,95 & selez. \\
Falesie & 0,33 & 0,10 & 0,03-0,17 & evitato & 0,33 & 0,72 & 0,51-0,94 & selez. & 0,33 & 0,11 & 0,01-0,21 & evitato \\
\toprule
& \multicolumn{4}{c}{$\chi^{2}$ = 117,4; P < 0,0001} & \multicolumn{4}{c}{$\chi^{2}$ = 25,3; P < 0,0001} & \multicolumn{4}{c}{$\chi^{2}$ = 31,9; P < 0,0001} \\
\toprule
\end{tabular}
}
\caption{Frequenze attese e osservate di uso dell{\textquoteright}habitat e selezione delle tre tipologie}
\label{Baroni_tab_1}
\end{table}
\end{adjustwidth}

\section*{Bibliografia}
\begin{itemize}\itemsep0pt
	\item Andreotti A., Borgo E. \& Truffi G., 1991 -- Presenze di Strolaghe
(Gavia spp.) in Liguria. In: SROPU (red.). Atti V Conv. Ital. Orn.
\textit{Suppl. Ric. Biol. }\textit{Selvaggina}, 17: 449-451.

	\item Ballardini M., Calvini M., Nani B. \& Toffoli R., 2005 -- Osservazioni
su presenza e distribuzione di pulcinella di mare \textit{Fratercula
arctica} e gazza marina \textit{Alca torda} nel mar Ligure occidentale.
Atti XIII Conv. Ital. Orn. \textit{Avocetta}, 29: 167.

	\item Borgo E., Span\`o S. \& Truffi G., 1991 -- Eccezionale presenza di
edredoni in Liguria: dati quantitativi. In: Fasola M. (red.). atti II
Semin. Ital. Censim. Faunistici dei Vertebrati. \textit{Suppl. Ric.
Biol. Selvaggina}, 16: 297-300.

	\item Brichetti P. \& Fracasso G., 2003 -- \textit{Ornitologia italiana. Vol.
1 -- Gaviidae-Falconidae.} Alberto Perdisa Editore, Bologna, 463 pp.

	\item Diviacco G. \& Coppo S., 2006 -- \textit{Atlante degli habitat marini
della Liguria. Descrizione e cartografia delle praterie di }Posidonia
oceanica\textit{ e dei principali popolamenti marini costieri.} Regione
Liguria, Genova, 205 pp.

	\item Gorlier G., 1975 -- Osservazioni ornitologiche del litorale e della zona
di mare compresa tra Vado Ligure (SV) e Finale Ligure (SV).
\textit{Riv. Ital. Orn.,} 45: 61-67.

	\item Neu C.W., Byers C.R. \& Peek J.M., 1974 -- A technique for analysis of
utilization-availability data. \textit{J. Wildl. Manage.}, 38: 541-545.

	\item Span\`o S., Truffi G. \& Burlando B. (a cura di), 1998 --
\textit{Atlante degli Uccelli svernanti in Liguria.} Regione Liguria,
Genova, 253 pp.
\end{itemize}
