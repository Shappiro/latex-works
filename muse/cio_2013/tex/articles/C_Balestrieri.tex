\setcounter{figure}{0}
\setcounter{table}{0}

\begin{adjustwidth}{-3.5cm}{0cm}
\pagestyle{CIOpage}
\authortoc{\textsc{Balestrieri R.}, \textsc{Posillico M.}, 
\textsc{Basile M.}, \textsc{Altea T.}, \textsc{Matteucci G.}}
\chapter*[Punti d{\textquoteright}ascolto: intensit\`a di
campionamento e contattabilit\`a]{
\textbf{Applicazione della tecnica dei punti d{\textquoteright}ascolto
in ambiente forestale: influenza dell{\textquoteright}intensit\`a di
campionamento e della contattabilit\`a delle singole specie sulla
caratterizzazione della comunit\`a ornitica}}
\addcontentsline{toc}{chapter}{Punti d{\textquoteright}ascolto: intensit\`a di
campionamento e contattabilit\`a}


\textsc{Rosario Balestrieri}$^{1*}$, \textsc{Mario Posillico}$^{1,2}$, 
\textsc{Marco Basile}$^{1}$, \textsc{Tiziana Altea}$^{2}$, 
 \textsc{Giorgio Matteucci}$^{1}$\\
 
 \index{Balestrieri Rosario} \index{Posillico Mario} \index{Basile Marco} \index{Altea Tiziana} \index{Matteucci Giorgio} 
\noindent\color{MUSEBLUE}\rule{27cm}{2pt}
\vspace{1cm}
\end{adjustwidth}



\marginnote{\raggedright $^1$Istituto di Biologia Agroambientale e
Forestale del CNR Via Salaria km 29.3, 00015 Monterotondo Scalo RM \\
$^2$Corpo Forestale dello Stato, Ufficio
Territoriale Biodiversit\`a di Castel di Sangro, Via Sangro, 45-67031
Castel di Sangro (AQ) \\
\vspace{.5cm}
{\emph{\small $^*$Autore per la corrispondenza: \href{mailto:rosario.balestrieri@ibaf.cnr.it}{ro\allowbreak sa\allowbreak rio.\allowbreak ba\allowbreak les\allowbreak trie\allowbreak ri@\allowbreak i\allowbreak baf.\allowbreak cnr.\allowbreak it}}} \\
\keywords{Alpi, Appennino, gestione forestale, punti
d{\textquoteright}ascolto}
{Alps, Apeninnes, forest management, point counts}
%\index{keywords}{Alpi} \index{keywords}{Appennino} \index{keywords}{Gestione forestale} \index{keywords}{Punti
%d{\textquoteright}ascolto}
}
{\small
\noindent \textsc{\color{MUSEBLUE} Summary} / Investigating the relationship between birds and forests is a key
argument for sustainable forestry and biodiversity conservation, as
birds are often the most numerous group of forest vertebrates, both as
number of species and individuals. We carried out our study within the
project LIFE ManFor CBD (www.manfor.eu), which aims to test and verify
in the field the effectiveness of forest management options in meeting
multiple objectives (timber production, environment protection and
biodiversity conservation, etc.), providing data, guidance and
indications of best-practice. Therefore, we search for an optimal
sampling scheme that account for the detectability of every species.
Data were collected in 5 forest stands in 5 areas from northern to
southern Italy, each one extended for about 30 ha. We randomly selected
19 -- 24 points in each area, surveyed four-five times each,
investigated through aural-visual survey. Detectability for every
species resulted in very different values, ranging from siskin (0.05),
to robin (1). Our research suggests that the majority of the species
among the forest bird community could be detected in the first and the
second survey.   \\
\noindent \textsc{\color{MUSEBLUE} Riassunto} / La relazione tra avifauna e habitat forestale \`e un argomento chiave
per la selvicoltura sostenibile e la conservazione della
biodiversit\`a, essendo gli Uccelli spesso il \textit{taxon} pi\`u
abbondante in tale habitat, sia per numero di specie che per numero di
individui. Il presente studio \`e stato portato avanti
nell{\textquoteright}ambito del progetto LIFE+ ManFor C.BD
(www.manfor.eu), che ha l{\textquoteright}obiettivo di verificare
l{\textquoteright}efficacia di diverse opzioni di gestione forestale
nel ottenere obiettivi multipli, tra cui la produzione di legname e la
conservazione della biodiversit\`a. Pertanto, \`e stato testato uno
schema di campionamento dell{\textquoteright}avifauna forestale, che
tenga conto della probabilit\`a di contatto di ogni specie. I dati sono
stati raccolti in 5 foreste di altrettante aree italiane, ognuna ampia
circa 30 ha. Sono stati selezionati casualmente 19 -- 24 punti in ogni
area, campionati 4 -- 5 volte tramite rilievi visivi e al canto. La
probabilit\`a di contatto di ogni specie \`e risultata molto varia, da
0.05 per il  lucherino a 1 per il pettirosso. Tale ricerca suggerisce
che la maggior parte delle specie incluse nella comunit\`a ornitica
forestale pu\`o essere rilevata nei primi due eventi di campionamento. \\
}
\section*{Introduzione}
{{La valutazione delle relazioni ecologiche
tra comunit\`a ornitica e ambiente forestale \`e di rilevante
importanza per una gestione forestale sostenibile, attenta anche alla
conservazione della biodiversit\`a essendo, spesso, gli Uccelli il
}\textit{{taxon}}{
di vertebrati forestali pi\`u numeroso sia in termini di individui che
in termini di specie (DeGraaf }\textit{{et
al. }}{1996). L{\textquoteright}alterazione
e la variazione dei parametri degli ambienti forestali (es. et\`a media
degli alberi o copertura fogliare) dovuta alle pratiche selvicolturali,
pu\`o, infatti, ripercuotersi sulla comunit\`a ornitica (cfr. ad es.
Caprio }\textit{{et
al.}}{ 2008, Gil-Tena
}\textit{{et
al.}}{ 2008, White
}\textit{{et
al.}}{ 2013) e, d{\textquoteright}altro
canto, la composizione della comunit\`a ornitica pu\`o rappresentare un
valido strumento nel predire i potenziali impatti negativi per tutta la
comunit\`a. Utile per l{\textquoteright}analisi della comunit\`a
ornitica forestale \`e la realizzazione di una
}\textit{{check-list}}{
delle specie presenti in un{\textquoteright}area. Le tecniche che
permettono di raggiungere tale obiettivo con risultati robusti (e.g.
inanellamento, rilievi a vista), in ambienti forestali, risultano
difficilmente applicabili sia per i costi che per le caratteristiche
fisiche degli habitat forestali. La tecnica dei punti
d{\textquoteright}ascolto \`e comunemente usata negli ambienti boscati
e rappresenta un buon compromesso tra adeguata copertura territoriale e
superamento delle problematiche menzionate (Fuller
}\textit{{et
al.}}{ 1984, Toms 2006). Questo metodo non
\`e tuttavia scevro da problemi (legati ad es. alla contattabilit\`a) e
la sua applicazione nel progetto LIFE+ ManFor CBD (\url{www.manfor.eu}) ne
rende necessaria una disamina per valutare le eventuali fonti di
}\textit{{bias}}. 
Lo scopo di questo lavoro \`e di valutare l{\textquoteright}applicazione del metodo dei punti
d{\textquoteright}ascolto valutando la capacit\`a di rilevare le varie
specie, intesa come probabilit\`a di contatto per ogni singola
specie.}

\section*{Metodi}

{
{La ricerca \`e stata condotta in 5
particelle forestali, ognuna delle quali estesa per circa 30 ha: una
pecceta sulle Alpi Giulie (\textasciitilde30 ha); una foresta mista di peccio
(}\textit{{Picea
abies}}{) e abete bianco
(}\textit{{Abies
alba}}{) sulle Dolomiti del Cadore (\textasciitilde25
ha); e 3 boschi di faggio
(}\textit{{Fagus sylvatica}}{)
sulle Prealpi Venete (\textasciitilde 33 ha), sull{\textquoteright}Appennino
centrale (\textasciitilde 30 ha) e sull{\textquoteright}Appennino meridionale
(\textasciitilde 30 ha). Ogni area era circondata da un{\textquoteright}ulteriore
area controllo (buffer), anch{\textquoteright}essa di circa 30 ha. In
ogni particella e relativo buffer sono stati individuati 19-24 punti
casuali, la cui distanza media \`e 155,8 m {\textpm} 15,2 DS), per un
totale di 111 punti. In ogni punto il rilievo \`e stato ripetuto da 3 a
5 volte, per un totale di 526 rilievi. Il singolo rilievo \`e
consistito nel riconoscimento di tutti gli individui visti o sentiti,
durante 5 minuti, dall{\textquoteright}alba alle ore 11. Per
massimizzare lo sforzo di campo e ottenere dati confrontabili, i
rilievi sono stati svolti entro un breve arco temporale (Maggio --
Giugno 2012), coincidente con il periodo di pi\`u frequenti
vocalizzazioni. L{\textquoteright}efficacia del campionamento, intesa
come specie rilevate rispetto al numero di specie attese, \`e stata
valutata con lo stimatore non parametrico di ricchezza specifica
Chao2-bc (Chao 2005). Le analisi sono state effettuate solo sulle
specie per le quali le informazioni di bibliografia indicassero
}\textit{{home range
}}{adeguati alla scala spaziale del
campionamento, per limitare problemi di autocorrelazione spaziale
(Brichetti e Fracasso 2008, 2010, 2011, 2013). Il modello utilizzato
per calcolare la probabilit\`a di contatto delle singole specie assume
che non ci siano differenze tra le aree, ed \`e implementato nel
software Presence (vers. 5.8).}~}

\section*{Risultati e discussione}

{
{L{\textquoteright}efficacia del
campionamento \`e risultata alta per tutte le aree. In totale sono
state rilevate 23 specie (max = 19; min = 13). Una probabilit\`a di
contatto inferiore a 0.5 \`e stata verificata per il 17\% delle specie,
mentre una probabilit\`a superiore a 0.8 \`e stata registrata per il
44\% delle specie. Cincia mora
}\textit{{Periparus ater}}{, fringuello
}\textit{{Fringilla coelebs}}{ e pettirosso
}\textit{{Erithacus rubecola}}{ sono risultate le specie pi\`u
rilevabili (p= 1), mentre lucherino
}\textit{{Carduelis spinus}}{, codibugnolo
}\textit{{Aegithalos caudatus}}{ e fiorrancino
}\textit{{Regulus ignicapilla}}{ quelle meno rilevabili (p =
0.05, 0.29, 0.31, rispettivamente).}}

{
{Dai risultati si evince in generale una
buona probabilit\`a di contatto delle specie (Tab. \ref{Balestrieri_tab_1}), anche per quelle
che iniziano a vocalizzare verso la fine di marzo. Inoltre lo studio
suggerisce che la maggior parte delle specie prettamente forestali
possa essere rilevata facilmente in una o due rilievi. Ci\`o conferma
che la struttura complessiva del disegno di campionamento e il rilievo
svolto attraverso la tecnica dei punti d{\textquoteright}ascolto sono
un metodo efficiente e utile per valutare alcuni parametri di
comunit\`a nei siti di studio. Ulteriori analisi sono in corso per
valutare la capacit\`a di questo approccio nel discriminare differenze
dei parametri di comunit\`a tra aree di dimensione inferiore ai 30 ha
soggette a diversi trattamenti selvicolturali.}}
\newpage
%\rowcolors{2}{MUSEBLUE!60!white}{white}
\begin{table}[!h]
\centering
\small
\begin{tabular}{>{\raggedright\arraybackslash}p{.4\columnwidth}>{\raggedright\arraybackslash}p{.25\columnwidth}>{\raggedright\arraybackslash}p{.25\columnwidth}}
\toprule
\textbf{Specie} & \textbf{Nome commune} & \textbf{Probablit\'a di contatto} \\
\toprule
%\showrowcolors
\textit{Aegithalos caudatos} & Codibugnolo & 0.2917 (0.0928) \\
\textit{Carduelis spinus} & Lucherino & 0.05 (0.0487) \\
\textit{Certhia brachydactyla} & Rampichino comune & 0.8333 (0.0761) \\
\textit{Certhia familiaris} & Rampichino alpestre & 0.3555 (0.1763) \\
\textit{Coccothraustes coccothraustes} & Frosone & 0.7997 (0.1795) \\
\textit{Cyanistes caeruleus} & Cinciarella & 0.875 (0.0675) \\
\textit{Erithacus rubecola} & Pettirosso & 1 \\
\textit{Ficedula albicollis} & Balia dal collare & 0.799 (0.181) \\
\textit{Fringilla coelebs} & Fringuello & 1 \\
\textit{Lophophanes cristatus} & Cincia dal ciuffo & 0.9 (0.0949) \\
\textit{Parus major} & Cinciallegra & 0.9583 (0.0408) \\
\textit{Periparus ater} & Cincia bigia & 1 \\
\textit{Phoenicurus phoenicurus} & Codirosso comune & 0.6972 (0.148) \\
\textit{Phylloscopus collybita} & Lu\'i piccolo & 0.8947 (0.0704) \\
\textit{Phylloscopus sibilatrix} & Lu\'i verde & 0.7358 (0.102) \\
\textit{Poecile palustris} & Cincia mora & 0.6667 (0.0962) \\
\textit{Pyrrhula pyrrhula} & Ciuffolotto & 0.8999 (0.0672) \\
\textit{Regulus ignicapilla} & Fiorrancino & 0.3069 (0.1291) \\
\textit{Regulus regulus} & Regolo & 0.5333 (0.1288) \\
\textit{Sitta europaea} & Picchio muratore & 0.9583 (0.0408) \\
\textit{Sylvia atricapilla} & Capinera & 0.7917 (0.0829) \\
\textit{Sylvia borin} & Beccafico & 0.7997 (0.1795) \\
\textit{Troglodytes troglodytes} & Scricciolo & 0.7917 (0.0829) \\
\bottomrule
\hiderowcolors
\end{tabular}
\caption{Probabilit\`a di contattare ogni specie, stimata secondo il modello implementato nel software PRESENCE vers. 5.8, che assume uguale probabilit\`a tra le aree (Hines 2006). I numeri in parentesi rappresentano gli errori standard}
\label{Balestrieri_tab_1}
\end{table}
\section*{Bibliografia}
\begin{itemize}\itemsep0pt
	\item Brichetti P. \& Fracasso G., 2008 - \textit{Ornitologia Italiana} Vol
5.
	\item Brichetti P. \& Fracasso G., 2010 - \textit{Ornitologia Italiana} Vol
6.
	\item Brichetti P. \& Fracasso G., 2011 - \textit{Ornitologia Italiana }Vol
7.
	\item Brichetti P. \& Fracasso G., 2013 - \textit{Ornitologia Italiana} Vol
8.
	\item Caprio E., Ellena I. \& Rolando A., 2008 - Assessing habitat/landscape
predictors of bird diversity in managed deciduous forests: a seasonal
and guild-based approach. \textit{Biodivers. Conserv., }18: 1287 --
1303.
	\item Chao A., 2005 - \textit{Species estimation and applications}
in~\textit{Encyclopedia of Statistical
Sciences,~2}\textit{\textsuperscript{nd}}\textit{ Edition, Vol. 12,
7907-7916}. Balakrishnan N, Read CB, Vidakovic B, eds. Wiley, New
York.
	\item DeGraaf R. \& Miller R.I., 1996 - \textit{Conservation of faunal
diversity in forested landscapes. }Chapman \& Hall, London.
	\item Fuller R.J. \& Langslow D.R., 1984 - Estimating numbers of birds by
point counts: how long should counts last? \textit{Bird Study,} 31:
195-202
	\item Gil-Tena A., Torras O. \& Saura S., 2008 - Relationships between forest
landscape structure and avian species richness in NE Spain.
\textit{Ardeola, }55 (1): 27 -- 40.
	\item Hines J.E., 2006 - PRESENCE -- Software to estimate patch occupancy and
related parameters USGS-PWRC. \texttt{http://www.mbr-pwrc.usgs.gov/software/presence.html}
	\item Toms J.D.,	 Schmiegelow F.K.A., Hannon S.J. \& Villard M-A., 2006 - Are
point counts of boreal songbirds reliable proxies for more intensive
abundance estimators? \textit{The Auk, }123 (2): 438- 454.
	\item White A.M., Zipkin E.F., Manley P.N. \& Schlesinger M.D., 2013 -
Conservation of avian diversity in the Sierra Nevada: moving beyond a
single-species management focus. \textit{PLoS ONE, }8 (5): e63088.
doi:10.1371/journal.pone.0063088.
\end{itemize}
