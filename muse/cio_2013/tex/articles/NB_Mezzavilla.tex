\setcounter{figure}{0}
\setcounter{table}{0}

\begin{adjustwidth}{-3.5cm}{0cm}
\pagestyle{CIOpage}
\authortoc{\textsc{Mezzavilla F.}, \textsc{Martignago G.},
\textsc{Silveri G.}}
\chapter*[Il re di quaglie nelle Prealpi venete orientali]{Monitoraggio del re di quaglie \textbf{\textit{Crex crex}}\textbf{ nelle Prealpi venete orientali}}
\addcontentsline{toc}{chapter}{Il re di quaglie nelle Prealpi venete orientali}

\textsc{Francesco Mezzavilla}$^{1*}$, \textsc{Gianfranco Martignago}$^{1}$,
\textsc{Giancarlo Silveri}$^{2}$\\

\index{Mezzavilla Francesco} \index{Martignago Gianfranco} \index{Silveri Giancarlo}
\noindent\color{MUSEBLUE}\rule{27cm}{2pt}
\vspace{1cm}
\end{adjustwidth}



\marginnote{\raggedright $^1$Associazione Faunisti Veneti, Museo di Storia
Naturale, S. Croce 1730, I-30135 Venezia \\
$^2$LIPU Pedemontana Trevigiana \\
\vspace{.5cm}
{\emph{\small $^*$Autore per la corrispondenza: \href{mailto:f.mezza@libero.it}{f.mezza@libero.it}}} \\
\keywords{\textit{Crex crex}, Prealpi Trevigiane, monitoraggio
popolazione}
{\textit{Crex crex}, Prealpi Trevigiane, population
monitoring}
%\index{keywords}{\textit{Crex crex}} \index{keywords}{Prealpi Trevigiane} \index{keywords}{Monitoraggio
%popolazione}
}
{\small
\noindent \textsc{\color{MUSEBLUE} Summary} / The corn crake \textit{Crex crex} has been surveyed in eastern venetian Prealps since 1995. 615 singing males were detected within four macro-areas in a surface of ​​120.8 km$^2$. The data analysis performed by the program TRIM 3.5, allowed us to detect a trend in strong decline of - 8.1\% year.\\
}
\section*{Introduzione}
Il re di quaglie \textit{Crex crex }in Veneto \`e stato poco indagato
fino all{\textquoteright}inizio degli anni Novanta del secolo scorso.
Solo a partire dal 1995 ha preso avvio una campagna di monitoraggio che
ha interessato  il massiccio del monte Grappa (106
km\textsuperscript{2}) e successivamente nel 1997
l{\textquoteright}altopiano del Cansiglio (7,9 km\textsuperscript{2}),
nel 1999 i versanti meridionali del monte Cesen (3,5
km\textsuperscript{2}) e nel 2000 il col Visentin (3,4
km\textsuperscript{2}) (Basso \textit{et al}. 1999). Il massiccio del
monte Grappa e l{\textquoteright}altopiano del Cansiglio sono stati
censiti tutti gli anni, mentre nel col Visentin i censimenti non sono
stati svolti nel 2002, 2003 e nel monte Cesen nel 2007, 2008 e 2011. 

\section*{Metodi}

Il monitoraggio \`e stato effettuato tutti gli anni nel mese di giugno e
solo in pochi casi  nella prima decade di luglio. Il censimento si \`e
basato sulla ricerca dei maschi in canto spontaneo e in mancanza di
alcun richiamo, dopo circa 10 minuti si \`e fatto ricorso al metodo del
\textit{play back}, con successivo ascolto delle eventuali risposte.
Negli anni {\textquoteright}90 sono stati effettuati duplici conteggi
nei mesi di maggio e di giugno, ma avendo verificato che il massimo
dell{\textquoteright}attivit\`a canora dei maschi si aveva soprattutto
in giugno, dal 2000 si \`e optato per una sola sessione da svolgere in
questo secondo mese. Complessivamente i punti di ascolto sono stati 59,
cos\`i suddivisi: monte Grappa 37, monte Cesen 4, monte Visentin 5,
altopiano del Cansiglio 13. Nel corso del monitoraggio sono stati
censiti complessivamente 615 maschi cantori. 

\section*{Risultati e discussione}

Il massiccio del monte Grappa con una media annuale di 23,6 maschi ( N =
18 ; SD = 8,9) ha ospitato la popolazione pi\`u numerosa, contrapposta 
al monte Cesen che ha evidenziato la media annua pi\`u bassa di soli
2,2 ( N = 11; SD = 0,6) maschi cantori. Il monte Visentin con una media
di 9,6 maschi (N = 12 ; SD = 3,5) e il Cansiglio con 3,6 ( N = 16 ;  SD
= 2,9) hanno evidenziato situazioni intermedie.  Per valutare il trend
delle presenze negli anni indagati \`e stato  impiegato il programma
TRIM  3.53 (TRends \& Indices for Monitoring data).
Dall{\textquoteright}analisi di ogni area monitorata,  il suo status 
\`e risultato  piuttosto vario, con diminuzioni comprese tra il -7,1 \%
anno del monte Visentin e -1,5 \% anno del Cansiglio. In
quest{\textquoteright}ultima localit\`a per\`o, tra il 2009 e il 2012,
si \`e rilevato un forte calo delle presenze con un solo maschio in
canto. Il trend complessivo nelle Prealpi venete, analizzato sul totale
dei maschi cantori rilevati a partire dal 2000 nelle quattro macroaree,
ha confermato un  andamento negativo ({\textquotedblleft}forte
declino{\textquotedblright}) con valori  pari a -8,1 \%  anno (Fig. \ref{Mezzavilla_fig_1})
, simile alle tendenze rilevate in provincia di Trento (Pedrini
\textit{et al}. 2015) e in altri stati dell{\textquoteright}Europa
occidentale (Green \textit{et al}. 1997).

I motivi della progressiva diminuzione del re di quaglie nelle Prealpi
venete non sono del tutto chiari. In alcune aree fra quelle indagate,
come la piana del Cansiglio, le modalit\`a di sfalcio primaverile dei
prati hanno progressivamente modificato il suo habitat, cos\`i come la
presenza massiccia di ungulati selvatici e domestici (caprini, ovini)
al pascolo, potrebbe aver inciso sul grado di disturbo (Mezzavilla \&
Lombardo 2011). Considerato il forte livello di filopatria,
caratteristico di questa specie (Schaffer 1999), in molti casi si \`e
verificato che bastano 2-3 anni successivi di sfalcio precoce dei
prati, perch\'e in seguito non si insedi pi\`u nello stesso sito. Anche
la progressiva espansione del bosco e l{\textquoteright}abbandono dei
prati e dei pascoli, ha diffusamente ridotto il suo spazio vitale. In
alcuni siti del monte Grappa, in anni diversi \`e stato verificato il
suo insediamento presso le stalle, dove l{\textquoteright}abbondanza di
liquami favoriva uno sviluppo molto esteso di \textit{Urtica dioica}.
La scomparsa di questa specie a causa di interventi
dell{\textquoteright}uomo, cos\`i come l{\textquoteright}eliminazione
di aree dominate da \textit{Epilobium angustifolium}, collegate al
pascolo ovino, caprino e bovino hanno completamente eliminato 
specifici habitat occupati in fase riproduttiva. Tutto ci\`o a fronte
di incentivi comunitari erogati dal Piano di sviluppo rurale  per il
mantenimento degli habitat, che per\`o non sono mai stati utilizzati in
Veneto.

\begin{figure}[!h]
\centering
\includegraphics[width=.95\columnwidth]{Mezzavilla_fig_1.png}
\caption{Trend del re di quaglie rilevato negli anni 2000-2012 nell{\textquoteright}area di studio}
\label{Mezzavilla_fig_1}
\end{figure}

\section*{Bibliografia}

\begin{itemize}\itemsep0pt
	\item Basso E., Martignago G., Silveri G. \& Mezzavilla F., 1999 - Censimenti
del Re di quaglie \textit{Crex crex} nelle Prealpi Venete Orientali.
\textit{Avocetta}, 23: 115.

	\item Green R., Rocamora G. \& Schaffer N., 1997 - Populations, ecology and
threats to the Corncrake \textit{Crex crex} in Europe.
\textit{Vogelwelt}, 118:117-134.

	\item Mezzavilla F. \& Lombardo S., 2011 - Censimenti del Re di quaglie
\textit{Cex crex}, in Cansiglio (1997-2008). Atti 2$^\circ$
Convegno Aspetti naturalistici della provincia di Belluno. Tipografia
Piave. Pp. 165-170.

	\item Schaffer N., 1999 - Habitatwahl und Partnerschaftssystem von Tupfelralle
\textit{Porzana porzana} und Wachtelkonig \textit{Crex crex}.
\textit{Okologie der Vogel} (\textit{Ecol. Birds}), 21:1-267.
\end{itemize}