\setcounter{figure}{0}
\setcounter{table}{0}

\begin{adjustwidth}{-3.5cm}{0cm}
\pagestyle{CIOpage}
\authortoc{\textsc{Biondi M.}, \textsc{Dragonetti M.},
\textsc{Giovacchini P.}, \textsc{Pietrelli L.}}
\chapter*[Preferenze ambientali dell{\textquoteright}occhione
nell{\textquoteright}Italia centrale]{Preferenze ambientali dell{\textquoteright}occhione \textbf{\textit{Burhinus oedicnemus}}\textbf{ in periodo riproduttivo nell{\textquoteright}Italia centrale}}
\addcontentsline{toc}{chapter}{Preferenze ambientali dell{\textquoteright}occhione
nell{\textquoteright}Italia centrale}

\textsc{Massimo Biondi}$^{1,3*}$, \textsc{Marco Dragonetti}$^{2}$,
\textsc{Pietro Giovacchini}$^{2}$, \textsc{Loris Pietrelli}$^{3}$ \\

\index{Biondi Massimo} \index{Dragonetti Marco} \index{Giovacchini Pietro} \index{Pietrelli Loris}
\noindent\color{MUSEBLUE}\rule{27cm}{2pt}
\vspace{1cm}
\end{adjustwidth}



\marginnote{\raggedright $^1$SROPU - Stazione Romana Osservazione e Protezione Uccelli, Via Britannia 36, 00183 Roma, Italia\\
$^2$GOM - Gruppo Ornitologico Maremmano\\
$^3$GAROL - Gruppo Attivit\`a Ricerche Ornitologiche del Litorale\\
\vspace{.5cm}
{\emph{\small $^*$Autore per la corrispondenza: \href{mailto:massimo.biondi54@gmail.com}{mas\allowbreak si\allowbreak mo.\allowbreak bion\allowbreak di\allowbreak 54@\allowbreak g\allowbreak ma\allowbreak il.\allowbreak com}}} \\
\keywords{\textit{Burhinus oedicnemus, }preferenze
ambientali, Lazio, Toscana}
{\textit{Burhinus oedicnemus}, habitat preference,
Latium, Tuscany}
%\index[keywords]{\textit{Burhinus oedicnemus}} \index[keywords]{Preferenze ambientali} \index[keywords]{Lazio} \index[keywords]{Toscana}
}
{\small
\noindent \textsc{\color{MUSEBLUE} Summary} / We analyzed the breeding habitat preferences of stone curlew
\textit{Burhinus oedicnemus} in central Italy. Data were collected from
194 breeding territories (111 in Tuscany and 83 in Latium). The species
selected 14 different biotopes (following Corine categories): 13 in Latium and 10 in Tuscany.\\
}
\vspace{1cm}

L{\textquoteright}occhione \textit{Burhinus oedicnemus} nidifica in
Toscana con una stima di 150-200 coppie {(Tinarelli
}\textit{{et al.}}{ 2009)} e nel
Lazio con 50-70 coppie {(Meschini in Brunelli
}\textit{{et al.}}{ 2011)}. I dati
sulle preferenze ambientali della specie sono sempre stati scarsi e
frammentari per la Toscana e parziali per il Lazio, eccetto che per la
popolazione viterbese (Meschini 2010).

Nel presente studio, per analizzare le preferenze riproduttive
dell{\textquoteright}occhione nell{\textquoteright}Italia centrale
abbiamo utilizzato le categorie fitosociologiche proposte dal progetto
{\textquotedblleft}CORINE Biotopes{\textquotedblright} (Devillers
\textit{et al}. 1991) modificate ad uso ornitologico (Boano 1997).
Unitamente alla tipologia ambientale \`e stata registrata la fascia
altimetrica utilizzata da ciascuna coppia. I dati raccolti hanno
interessato le provincie di Grosseto, Roma e Latina. Il periodo di
studio ha analizzato i dati inediti sia per la Toscana (1995-2011) sia
per il Lazio (2008-2012) raccolti nell{\textquoteright}arco della
stagione riproduttiva (1{\textdegree} marzo -- 30 settembre). Abbiamo
considerato occhioni nidificanti laddove sono stati registrati
individui con evidente comportamento territoriale protratto per pi\`u
di 15 giorni, oppure laddove siano state rinvenute uova e/o pulli. Sono
stati analizzati complessivamente 194 territori di cui 111 in Toscana e
83 nel Lazio. La specie ha selezionato 14 diverse tipologie ambientali
(13 nel Lazio; 10 in Toscana) e tra queste, cinque sono apparse
predominanti: praterie e steppe calcaree (37.6\%), gariga su suoli
calcarei (11.8), coltivazioni estensive e tradizionali (10.8), steppe
cerealicole (8.2\%) e greti fluviali (7.2\%) (Fig. \ref{Biondi_fig_1}). In Toscana la
specie ha selezionato praterie e steppe calcaree (45\%), caratterizzate
da pascoli permanenti, e coltivazioni estensive tradizionali (18\%), in
gran parte rappresentate da oliveti maturi a conduzione tradizionale e
da colture foraggere estensive collegate
all{\textquotesingle}allevamento ovino. Nel Lazio
l{\textquoteright}occhione si insedia per il 55.4\% in due ambienti
naturali con affioramenti calcarei pascolati (ovini, bovini, equini) di
cui uno, gariga su suoli calcarei, assente in Toscana e coincidente, in
gran parte, con l{\textquoteright}area della ZPS monti della Tolfa. La
specie dimostra inoltre di poter colonizzare ambienti succedanei a
quelli steppici naturali o semi-naturali insediandosi anche in habitat
fortemente antropizzati come i siti industriali attivi o abbandonati
(6.1\%) ove da alcuni anni risulta anche svernante (Lazio) (Biondi
\textit{et al.} 2011). Dal punto di vista altitudinale
l{\textquoteright}occhione sembra prediligere due fasce altimetriche:
0-50 m (29.8\%) e 101-200 m (27.8\%) con un limite altitudinale posto a
500-600 m (1.5\%). 

\begin{adjustwidth}{-3.5cm}{-1cm}
\begin{figure}[!h]
\centering
\includegraphics[width=1.2\columnwidth]{Biondi_fig_1.png}
\caption{Preferenze ambientali dell'occhione \textit{Burhinus oedicnemus} in Italia centrale}
\label{Biondi_fig_1}
\end{figure}
\end{adjustwidth}

\newpage
\section*{Bibliografia}
\begin{itemize}\itemsep0pt
	\item Biondi M., Pietrelli L., Scrocca R. \&  Meschini A., 2011 - New
Stone-Curlew \textit{Burhinus oedicnemus }wintering site in central
Italy. \textit{Wader Study Group Bulletin} 118 (1):63-64.

	\item Boano G., 1997 - Proposta di una classificazione degli habitat ad uso
ornitologico. In: Brichetti P. \& Gariboldi A. 1997. \textit{Manuale
pratico di ornitologia. }Edagricole, Bologna: 153-165.

	\item {Devillers P., Devillers-Terschuren J.P. \& CORINE
BIOTOPES EXPERTS GROUP, 1991 - }\textit{{CORINE
biotopes manual}}{. Part 2. Habitats of the European
Community. Commission of the European Community.
}{Brussels.}

	\item {Meschini A., 2010 -
}\textit{{L{\textquoteright}Occhione tra fiumi e
pietre.}}{ Edizioni Belvedere, Latina: pp.174.}

	\item {Meschini A., 2011 - L{\textquoteright}occhione
}\textit{{Burhinus oedicnemus}}{.
In.} Brunelli M., Sarrocco S., Corbi F., Sorace A., De Felici, Boano
A., Guerrieri G., Meschini A. \& Roma S. (a cura di) 2011 -
\textit{Nuovo atlante degli Uccelli Nidificanti nel Lazio}. Edizioni
ARP-Agenzia Regionale Parchi, Roma: 146-147.

	\item {Tinarelli R., Alessandria G., Giovacchini P., Gola L.,
Ientile R. \& Meschini A., 2009 - Consistenza e distribuzione
dell{\textquoteright}occhione in Italia: aggiornamento al 2008. In:
Giunchi D., Pollonara E. \& E. Baldaccini  (a cura di) 2009 -
L{\textquoteright}occhione (}\textit{{Burhinus
oedicnemus}}{): Biologia e conservazione di una specie
di interesse comunitario - Indicazioni per la gestione del territorio
e delle aree protette. Conservazione e gestione della natura.
}\textit{{Quaderni di
documentazione,}}{ 7: 45-50.}

\end{itemize}
