\setcounter{figure}{0}
\setcounter{table}{0}

\begin{adjustwidth}{-3.5cm}{0cm}
\pagestyle{CIOpage}
\authortoc{\textsc{Trotti P.}, \textsc{Bassi E.}, 
\textsc{Bionda R.}, \textsc{Ferloni M.}, 
\textsc{Rubolini D.}}
\chapter*[Ecologia del gufo reale in due aree lombarde]{Ecologia e produttivit\`a del gufo reale \textbf{\textit{Bubo
bubo}}\textbf{ in due aree di studio della Lombardia}}
\addcontentsline{toc}{chapter}{Ecologia del gufo reale in due aree lombarde}

\textsc{Paolo Trotti}$^{1*}$, \textsc{Enrico Bassi}$^{2}$, 
\textsc{Radames Bionda}$^{3}$, \textsc{Maria Ferloni}$^{4}$, 
\textsc{Diego Rubolini}$^{1}$\\

\index{Trotti Paolo} \index{Bassi Enrico} \index{Bionda Radames} \index{Ferloni Maria} \index{Rubolini Diego}
\noindent\color{MUSEBLUE}\rule{27cm}{2pt}
\vspace{1cm}
\end{adjustwidth}


\marginnote{\raggedright $^1$Dipartimento di Bioscienze, Universit\`a degli Studi
di Milano, via Celoria 26, 20133, Milano \\
$^2$Consorzio del Parco Nazionale dello Stelvio, via De
Simoni 42, 23032, Bormio (SO) \\
$^3$Parco Naturale Alpe Veglia e Devero - Alta Valle
Antrona, viale Pieri 27, 28868 Varzo (VB) \\
$^4$Provincia di Sondrio Uff. Faunistico, via XXV Aprile,
23100, Sondrio \\
\vspace{.5cm}
{\emph{\small $^*$Autore per la corrispondenza: \href{mailto:paolotrotti6@libero.it}{paolotrotti6@libero.it}}} \\
\keywords{Alpi e Prealpi lombarde, \textit{Bubo bubo}, densit\`a,
successo riproduttivo, selezione habitat}
{Central Alps and Prealps, \textit{Bubo bubo},
density, reproductive success, habitat selection}
%\index{keywords}{Alpi e Prealpi lombarde} \index{keywords}{\textit{Bubo bubo}} \index{keywords}{Densit\`a}
%\index{keywords}{Successo riproduttivo} \index{keywords}{Selezione habitat}
}
{\small

\noindent \textsc{\color{MUSEBLUE} Summary} / CWe studied 31 pairs of eagle owl \textit{Bubo bubo} in two different
areas in northern Italy from 2010 to 2012: 6 pairs in area 1 in
Valtellina (Sondrio province) characterized by a \textit{Nearest
Neighbour Distance }(NND) of 4,280 {\textpm} 700 m with a density value
of 2.4 pairs 100 km\textsuperscript{{}2} and 25 pairs in area 2 in
Camonica Valley-Iseo lake surroundings (Bergamo and Brescia provinces)
with a NND of 2,971 {\textpm} 1,349 m and a density value of 5.2 pairs
100 km\textsuperscript{{}2}. In order to determine the habitat
preferences of the species we compared the landscape features within a
radius of 1000 m around each occupied (31) and unoccupied (33) cliffs.
Eagle owl preferred the most extensive and favourably orientated
cliffs, greater extension of open areas and woody crops that in our
study area were mainly represented by olive groves and vineyards. The
reproductive success was 44.4\% and the mean number of fledglings per
pair was 0.71. This parameter was positively influenced by the water
bodies and active quarries. \\
\noindent \textsc{\color{MUSEBLUE} Riassunto} / La ricerca ha indagato 31 coppie di gufo reale in due aree della
Lombardia dal 2010 al 2012: 6 coppie nell{\textquoteright}area 1 in
bassa Valtellina (provincia di Sondrio), caratterizzata da una NND di
4.280 {\textpm} 700 m e una densit\`a di 2,4 coppie per 100
km\textsuperscript{{}2}, e 25 coppie nell{\textquoteright}area 2 in
valle Camonica e nei dintorni del lago d{\textquoteright}Iseo (province
di Bergamo e Brescia) con NND di 2.971 {\textpm} 1.349 m e un valore di
densit\`a pari a 5,2 coppie per 100 km\textsuperscript{{}2}. Per
valutare quali variabili ambientali discriminassero i siti di presenza
da quelli di assenza, \`e stata eseguita un{\textquoteright}analisi
della selezione dell{\textquoteright}habitat
all{\textquoteright}interno di un buffer con raggio di 1000 m, tra i 31
siti di presenza e i 33 siti di assenza. La specie ha mostrato una
preferenza per le pareti rocciose pi\`u estese e meglio esposte al
sole, per maggiori estensioni di aree aperte e di colture legnose
rappresentate soprattutto da vigneti e oliveti. Il successo
riproduttivo \`e pari a 44,4\% e il numero medio di giovani involati
per coppia controllata \`e paria a 0,71. Tale parametro \`e influenzato
positivamente dalla maggiore estensione dei corpi idrici e delle cave
di versante.
}



\section*{Introduzione}

La ricerca si \`e posta l{\textquoteright}obiettivo di indagare la
presenza del gufo reale \textit{Bubo bubo,} in due aree della
Lombardia, al fine di valutarne la densit\`a, il successo riproduttivo,
le caratteristiche ambientali dei territori di nidificazione e
l{\textquoteright}influenza dell{\textquoteright}ambiente sul successo
riproduttivo. Inoltre \`e stato possibile confrontare il successo
riproduttivo di alcune coppie indagate (N= 11) con la serie storica
della produttivit\`a raccolta per le stesse coppie nel periodo
1999-2001 (Bassi 2001).

\section*{Metodi}

La ricerca si \`e concentrata in due aree lombarde: la prima, inclusa in
provincia di Sondrio (area 1: bassa Valtellina), ha estensione pari a
115 km\textsuperscript{2} e quota tra i 200 e i 1.150 m s.l.m.
L{\textquoteright}area 2 \`e invece situata tra le province di Bergamo
e Brescia e comprende le due sponde del lago d{\textquoteright}Iseo, la
val Cavallina, parte della val Borlezza e della valle Camonica, con
un{\textquoteright}estensione di 520 km\textsuperscript{2} e quota tra
i 185 e i 1200 m s.l.m. Nel triennio 2010-2012 sono state indagate
l{\textquoteright}area 1 e una porzione dell{\textquoteright}area 2
posta in provincia di Bergamo, gi\`a indagata intensivamente in passato
(Bassi 2001) mentre la porzione restante dell{\textquoteright}area 2,
posta in provincia di Brescia, \`e stata indagata nella sola stagione
riproduttiva 2012. I metodi utilizzati sono stati:
l{\textquoteright}ascolto sistematico del canto spontaneo degli adulti
territoriali (dicembre-marzo) e dei giovani (maggio-luglio), la
stimolazione con richiamo registrato (\textit{playback}) e la ricerca
diurna dei nidi e delle tracce di presenza. La distanza tra i territori
\`e stata calcolata con il metodo della \textit{Nearest Neighbour
Distance }(NND) mentre per il calcolo della densit\`a \`e stato
utilizzato il metodo dei \textit{buffer} che si basa sul valore medio
della NND utilizzato come raggio di un \textit{buffer }circolare
attorno al centro di ogni territorio; l{\textquoteright}area cos\`i
individuata costituisce l{\textquoteright}area di studio (Bionda 2002).
Per il calcolo della dispersione dei nidi \`e stato utilizzato il Test
G; valori superiori a 0,65 indicano una distribuzione uniforme dei
territori (Brown \& Rothery 1978). I principali parametri riproduttivi
calcolati sono stati il successo riproduttivo (percentuale delle coppie
riprodottesi con successo sulle coppie totali), il numero di giovani
involati su coppie totali e il numero di giovani involati su coppie di
successo.

Per le analisi ambientali nelle province considerate \`e stata
utilizzata la carta di uso del suolo D.U.S.A.F. 2.1 (Destinazione
d{\textquoteright}Uso dei Suoli Agricoli e Forestali) del 2007.

Le caratteristiche ambientali sono state calcolate
all{\textquoteright}interno di \textit{buffer }circolari di 1 km di
raggio attorno alle pareti di presenza e assenza. Le variabili
ambientali presenti all{\textquoteright}interno del \textit{buffer
}sono state raggruppate in 7 gruppi (aree aperte, aree boscate, colture
legnose, cespuglieti, corpi idrici, cave e aree urbanizzate). Le
caratteristiche della parete rocciosa sono state calcolate secondo la
metodologia adottata da Brambilla \textit{et al. }(2010) mentre, per
l{\textquoteright}esposizione della parete, \`e stato utilizzato un
sistema di punteggio che attribuisce valori maggiori alle pareti pi\`u
esposte al sole (N-NW=1; W=2; NE=3; E=4; SW=5; S-SE=6).

\`E stato effettuato un confronto descrittivo delle variabili ambientali
tra pareti occupate e pareti non occupate attraverso il test \textit{t
}di Student. L{\textquoteright}analisi ha fornito informazioni su quali
variabili potessero influenzare la presenza della specie. Per valutare
le variabili ambientali influenzanti la selezione
dell{\textquoteright}habitat \`e stata eseguita
un{\textquoteright}analisi di regressione logistica mentre, per
l{\textquoteright}analisi dell{\textquoteright}influenza
dell{\textquoteright}ambiente sul numero di giovani involati per coppia
\`e stato utilizzato un modello misto assumendo una distribuzione
poissoniana dell{\textquoteright}errore. 

\section*{Risultati e discussione}

Sono stati individuati 31 territori di gufo reale (6
nell{\textquoteright}area 1 e 25 nell{\textquoteright}area 2). Nel
settore bergamasco di quest{\textquoteright}ultima area sono stati
confermati, a distanza di 10 anni, 11 territori indagati nel periodo
1999-2001 (Bassi 2001) e scoperti 2 nuovi territori. Nel settore
bresciano di quest{\textquoteright}area sono stati riconfermati 4
territori noti (Bertoli \& Leo \textit{\textcolor{black}{ined.}}) e
scoperti 8 nuovi territori. La media NND per l{\textquoteright}area 1
\`e di 4280 {\textpm} 700 m (\textit{range} 3720-5330 m) (Bassi
\textit{et al. }2011) e per l{\textquoteright}area 2 di 2971 {\textpm}
1349 m (\textit{range} 847-5.435). 

I valori di densit\`a variano da 2,4 cp 100 km\textsuperscript{{}2},
per l{\textquoteright}area 1, a 5,2 cp 100 km\textsuperscript{{}2} per
l{\textquoteright}area 2. 

La densit\`a registrata nell{\textquoteright}area 1 rientra nella media
riportata per le Alpi (Casanova \& Galli 1998; Marchesi \textit{et al.
}1999; Bionda 2002; Bassi \textit{et al}. 2003) mentre, quella
dell{\textquoteright}area 2, costituisce uno dei valori pi\`u alti tra
quelli riportati a livello nazionale che sottolinea la particolare
vocazione di questo territorio caratterizzato
dall{\textquoteright}elevata disponibilit\`a di pareti rocciose a
ridosso di laghi e ampi fondovalle non eccessivamente urbanizzati.

Il test G ha evidenziato una distribuzione regolare e uniforme dei siti
con un valore di 0,94 per l{\textquoteright}area 1 e di 0,67 per
l{\textquoteright}area 2. Il successo riproduttivo \`e pari al 44,4\%,
in linea con alcuni studi alpini (Marchesi \textit{et al. }2002; Bionda
2002). Il numero medio di giovani involati per coppia controllata \`e
di 0,71 mentre il numero medio di giovani involati per coppia di
successo \`e di 1,61. Nelle due aree per cui sono disponibili serie
triennali di dati si \`e riprodotto con successo il 27,8\% (N= 18) e il
54,5\% (N= 33) delle coppie presenti. Il confronto del successo
riproduttivo nel periodo 1999-2001 per gli 11 territori della porzione
bergamasca dell{\textquoteright}area 2 (Bassi \textit{et al. }2003) con
quello per gli stessi nidi negli anni 2010-2012, ha evidenziato un
marcato declino (-40,7\%) cos\`i come il numero medio di giovani
involati sul totale delle coppie controllate \`e diminuito da 1,19 a
0,79. Tale risultato potrebbe dipendere dal progressivo consumo di
territorio sul fondovalle e dal sempre pi\`u marcato processo di
rimboschimento dei versanti che portano alla diminuzione di specie
preda importanti, legate alle aree pi\`u aperte.

Nel triennio 2010-2012, soltanto 5 delle 17 coppie seguite (29,4\%) si
sono riprodotte con successo ogni anno (min 1, max 3 giovani/coppia)
suggerendo l{\textquoteright}ipotesi che esistano coppie con una
maggiore \textit{fitness} riproduttiva, detentrici di territori
migliori da un punto di vista trofico e presentanti minori fattori di
mortalit\`a (ad es. elettrocuzione).

Le analisi statistiche hanno evidenziato una preferenza della specie per
le pareti rocciose pi\`u ampie e meglio esposte al sole, che possono
conferire una maggior protezione da eventuali disturbi e migliori
condizioni climatiche, una maggiore estensione di colture legnose
(vigneti e oliveti sui versanti pi\`u termofili che possono attrarre un
maggior numero di specie preda) e di aree aperte, fondamentali per la
caccia (Penteriani \textit{et al. }2001). Il numero di giovani involati
per coppia \`e influenzato positivamente
dall{\textquoteright}estensione dei corpi idrici e delle cave di
versante. La prima variabile, oltre a incrementare il potenziale
spettro trofico del predatore, \`e probabilmente una stima indiretta
della densit\`a della specie preda preferita (\textit{Rattus
norvegicus}) che, per il nord dell{\textquoteright}Italia, risulta
particolarmente consistente presso i corpi idrici (Sergio \textit{et
al}. 2004). Le cave, invece, influenzano positivamente il successo
riproduttivo sia perch\'e il divieto di accesso conferisce una maggior
protezione (al loro interno sono infatti interdette le attivit\`a di
caccia e arrampicata sportiva) sia perch\'e determinano un maggior
grado di biodiversit\`a che deriva dalla presenza di ecotoni e stagni
di cava (Bassi 2003). 

\section*{Ringraziamenti}

Si ringraziano sentitamente gli ornitologi Bertoli Roberto e Leo Rocco
per la generosa condivisione di alcuni dati e gli Agenti della Polizia
provinciale di Sondrio Mozzetti Ettore, Bernardara Enos, Ronconi
Antonio, Luciani Fausto, Pasini Massimiliano per
l{\textquoteright}attivit\`a di campo.

\section*{Bibliografia}
\begin{itemize}\itemsep0pt
	\item Bassi E., 2001 - Scelta del sito di nidificazione del Gufo reale
(\textit{Bubo bubo}, Strigiformes, Aves) nel Settore orientale delle
Prealpi Bergamasche. Tesi di Laurea, Universit\`a degli Studi di Pavia.

	\item Bassi E., 2003 - Importanza degli ambienti di cava per
l{\textquotesingle}insediamento del gufo reale \textit{Bubo bubo}.
\textit{Avocetta}, 27: 127.

	\item Bassi E., Bonvicini P. \& Galeotti P., 2003 - Successo riproduttivo e
selezione del territorio di nidificazione del Gufo reale \textit{Bubo
bubo} nelle Prealpi bergamasche\textit{. Avocetta}, 27: 97.

	\item Bassi E., Bionda R., Trotti P., Folatti M.G. \& Ferloni M., 2011 -
Mitigazione dell{\textquoteright}impatto delle linee elettriche per la
conservazione del gufo reale \textit{Bubo bubo} in provincia di
Sondrio. Atti XV Convegno Nazionale di Ornitologia, Cervia (RA), 22-25
settembre 2011 (in stampa).

	\item Bionda R., 2002 - Censimento di Gufo reale \textit{Bubo bubo} nella
provincia del Verbano Cusio Ossola. I Convegno Italiano Rapaci diurni e
notturni. Preganziol (TV) 9-10 marzo.

	\item Brambilla M., Bassi E., Ceci C. \& Rubolini D., 2010 - Environmental
factors affecting patterns of distribution and co-occurrence of two
competing raptor species. \textit{Ibis}, 152 (2): 310-322.

	\item Brown D. \& Rothery P., 1978 - Randomness and local regularity of points
in a plane. \textit{Biometrica}, 65: 115-122.

	\item Casanova M. \& Galli L., 1998 - Primi dati sulla biologia del Gufo
reale, \textit{Bubo bubo}, nel Finalese (Liguria occidentale).
\textit{Riv. ital. Orn.}, 68 (2): 167-174.

	\item Marchesi L., Pedrini P. \& Galeotti P., 1999 - Densit\`a e dispersione
territoriale del Gufo reale \textit{Bubo bubo} in provincia di Trento
(Alpi centro-orientali). \textit{Avocetta}, 23: 19-23.

	\item Marchesi L., Sergio F. \& Pedrini P., 2002 - Costs and benefits of
breeding in human-altered landscapes for the eagle owl \textit{Bubo
bubo}. \textit{Ibis}, 144, E164--E177.

	\item Penteriani V., Gallardo M., Roche P. \& Cazassus H., 2001 - Effects of
landscape spatial structure and composition on the settlement of the
eagle owl \textit{Bubo bubo }in a mediterranean habitat.
\textit{Ardea}, 89 (2): 331-340.

	\item Sergio F., Marchesi L. \& Pedrini P., 2004 - Integrating individual
habitat choice and regional distribution of a biodiversity indicator
and top predator. \textit{J Biogeogr.,} 31: 619--628.
\end{itemize}
