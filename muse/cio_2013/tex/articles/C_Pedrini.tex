\setcounter{figure}{0}
\setcounter{table}{0}

\begin{adjustwidth}{-3.5cm}{-1cm}
\pagestyle{CIOpage}
\authortoc{\textsc{Pedrini P.}, \textsc{Brambilla M.}, 
\textsc{Florit F.}, \textsc{Martignago G.}, 
\textsc{Mezzavilla F.}, \textsc{Rassati G.}, 
\textsc{Silveri G.}}
\chapter*[Il re di quaglie in Italia nord-orientale]{Andamento demografico del re di quaglie \textbf{\textit{Crex crex}}\textbf{ nell{\textquoteright}Italia nord-orientale}}
\addcontentsline{toc}{chapter}{Il re di quaglie in Italia nord-orientale}

\textsc{Paolo Pedrini}$^{1*}$, \textsc{Mattia Brambilla}$^{1}$, 
\textsc{Fabrizio Florit}$^{2}$, \textsc{Gianfranco Martignago}$^{3}$, 
\textsc{Francesco Mezzavilla}$^{3}$, \textsc{Gianluca Rassati}$^{4,5}$, 
\textsc{Giancarlo Silveri}$^{3}$\\

\index{Pedrini Paolo} \index{Brambilla Mattia} \index{Florit Fabrizio} \index{Martignago Gianfranco} \index{Mezzavilla Francesco} \index{Rassati Gianluca} \index{Silveri Giancarlo}
\noindent\color{MUSEBLUE}\rule{27cm}{2pt}
\vspace{1cm}
\end{adjustwidth}



\marginnote{\raggedright $^1$Muse - Museo delle Scienze, Sezione di Zoologia dei
Vertebrati, Corso del Lavoro e della Scienza 3 - 38123 Trento, Italy \\
$^2$Regione autonoma Friuli Venezia Giulia, Direzione
centrale risorse rurali, agroalimentari e forestali, Servizio caccia,
risorse ittiche e biodiversit\`a, Ufficio studi faunistici, via
Sabbadini 31, 33100 Udine \\
$^3$Associazione Faunisti Veneti, C/o Museo Civico di
Storia Naturale S. Croce 1730 - 30135 Venezia \\
$^4$ Via Udine 9 - 33028 Tolmezzo (UD)\\
$^5$ Regione Autonoma Friuli Venezia Giulia, Ispettorato
agricoltura e foreste di Tolmezzo, Via San Giovanni Bosco 8 - 33028
Tolmezzo (UD)\\
\vspace{.5cm}
{\emph{\small $^*$Autore per la corrispondenza: \href{mailto:paolo.pedrini@muse.it}{paolo.pedrini@muse.it}}} \\
\keywords{\textit{Crex crex}, agricoltura, Alpi orientali, prati}
{\textit{Crex crex}, Eastern Alps,
agriculture, grassland}
%\index{keywords}{\textit{Crex crex}} \index{keywords}{Agricoltura} \index{keywords}{Alpi orientali} \index{keywords}{Prati}
}
{\small

\noindent \textsc{\color{MUSEBLUE} Summary} / We counted by means of nocturnal censuses the number of corncrake
\textit{Crex crex} calling males in selected sample areas in
north-eastern Italy (Trentino, Veneto, Friuli Venezia Giulia), in an
area considered to be the species{\textquotesingle} stronghold in
Italy. Counts were done in June-July, in the period 2000-2012.
Population trend was modelled under a Generalised Estimating Equation
(GEE) approach. Despite some differences in population trend across
regions (with the trend being more dramatic in Trentino and Veneto),
the species showed a significant decline during 2000-2012, a period
that is considered as favourable to the species at the European level.
This contrast with the general trend of the species suggested that
local factors are involved in driving the species decline; in
particular, unsustainable mowing management and agricultural
intensification in general are likely the most impacting factors. Also
land abandonment and, at a more local scale, urbanization, disturbance
and overgrazing may be associated with the species decline. \\
\noindent \textsc{\color{MUSEBLUE} Riassunto} / Abbiamo censito in orario notturno il numero di maschi cantori di re di
quaglie \textit{Crex crex} in aree campione selezionate in Italia
nord-orientale (Trentino, Veneto, Friuli Venezia Giulia), in
un{\textquotesingle}area ritenuta la roccaforte della specie a livello
nazionale. I censimenti sono stati svolti in giugno-luglio, nel periodo
2000-2012. Il trend della popolazione \`e stato modellizzato
utilizzando un approccio di tipo \textit{Generalised Estimating
Equation }(GEE). Nonostante alcune differenze nel trend della specie
tra le regioni (con un declino pi\`u evidente in Trentino e Veneto), la
specie ha mostrato un decremento significativo nel periodo 2000-2012,
che generalmente \`e considerato un periodo favorevole alla specie a
scala europea, con trend positivo nella maggior parte dei casi. Questo
contrasto con il trend generale della specie suggerisce la presenza di
fattori locali responsabili del declino della specie; in particolare,
pratiche di sfalcio non sostenibili e intensificazione
dell{\textquotesingle}agricoltura in generale rappresentano i fattori
pi\`u probabilmente importanti nel determinare il trend negativo della
specie. Anche l{\textquotesingle}abbandono delle aree agricole
marginali e, a una scala pi\`u locale, urbanizzazione, disturbo e
pascolo eccessivo sono probabilmente tra le cause concomitanti del
declino.
}



\section*{Introduzione}

Il re di quaglie \textit{Crex crex} \`e una specie di grande interesse
conservazionistico (SPEC 1; BirdLife International 2004), legata ad
ambienti prativi in larga parte dell{\textquotesingle}Eurasia (Cramp \&
Simmons 1980), dove \`e un visitatore estivo diffuso ma scarso alle
medie latitudini. La popolazione europea della specie ha subito un
drammatico declino durante i secoli XIX e XX (Green \textit{et al}.,
1997a; BirdLife International, 2004), le cui cause sono probabilmente
soprattutto l{\textquotesingle}intensificazione
dell{\textquotesingle}agricoltura e la sua meccanizzazione, lo sfalcio
precoce, i cambiamenti nelle pratiche agricole (Cramp \& Simmons 1980;
Broyer 1987; Kei\v{s}s 2003; Rassati \& Rodaro 2007; Moga \textit{et
al.} 2010). Dagli anni Novanta, diverse popolazioni hanno mostrato
segni di ripresa e incremento (O{\textquoteright}Brien \emph{et al}.
2006; Kei\v{s}s 2003), mentre altre hanno proseguito il declino.

In Europa centrale e occidentale, dove il re di quaglie appare legato
agli ambienti prativi da sfalcio (prati umidi, pascoli),
nell{\textquotesingle}ultimo ventennio la specie ha mostrato segnali di
recupero in buona parte del suo areale; per l{\textquoteright}Italia
mancano per\`o dati complessivi aggiornati a scala nazionale, anche se
dalla met\`a degli anni Novanta sono stati promossi sia monitoraggi a
scala nazionale che studi locali, i quali hanno permesso di
quantificare la popolazione, definire gli habitat di nidificazione e la
distribuzione, di fatto quasi esclusivamente limitata alla regione
alpina centro orientale (vedi Gustin \textit{et al.} 2009 e riferimenti
ivi citati). Con questo contributo abbiamo analizzato il trend
demografico della specie nella fascia prealpina e alpina di Trentino,
Veneto e Friuli Venezia Giulia, dove si rinviene la gran parte della
popolazione italiana. 

\section*{Metodi}

Abbiamo utilizzato i dati raccolti durante censimenti notturni dei
maschi cantori, condotti a partire dall{\textquoteright}anno 2000 in
alcuni siti campione ritenuti rappresentativi della distribuzione
regionale della specie. Stante l{\textquotesingle}esistenza di
variazioni nel corso della stagione riproduttiva di abbondanza locale e
distribuzione (Rassati 2001, 2004, 2009; Brambilla \& Pedrini 2011;
Pedrini \textit{et al.} 2012), abbiamo considerato dati il pi\`u
possibile omogenei dal punto di vista del periodo di censimento,
scegliendo i censimenti {\textquotedblleft}tardivi{\textquotedblright}
(giugno-luglio), in quanto gli unici disponibili per quasi tutti gli
anni indagati per tutte e tre le regioni considerate dallo studio. 

Le analisi sono state condotte attraverso un approccio di tipo
Generalised Estimating Equations, utilizzando il software TRIM (TRends
\& Indices for Monitoring data) 3.54. L{\textquotesingle}approccio
adottato da TRIM \`e quello delle
\textit{{generalized
estimating
equations}}{, che
consentono di stimare valori per i dati mancanti, tenendo conto di
}\textit{overdispersion} e autocorrelazione seriale (Pannekoek \& Van
Strien 2001; Soldaat
\textit{{et
al}}{.
}{2007; Ludwig
}\textit{{et
al.}}{
}{2008). Abbiamo
fissato pari a 1 il valore dell{\textquotesingle}indice di popolazione
per il 2000, primo anno
considerato}{.} Il
modello utilizzato \`e stato \textit{linear trend }con \textit{stepwise
selection }dei \textit{changepoints}, utilizzando i valori di
probabilit\`a proposti di default dal programma, dal momento che tale
modello di trend rappresentava in generale quello statisticamente
migliore (v. anche Pedrini \textit{et al}. 2012; Brambilla \& Pedrini
2013), e inserendo la regione come covariata, per verificare
l{\textquotesingle}esistenza di diversit\`a
nell{\textquotesingle}andamento demografico tra le tre aree, e tenerne
conto nell{\textquotesingle}elaborazione del trend complessivo.

\section*{Risultati}

Il trend complessivo della specie nel suo principale areale italiano nel
periodo 2000-2012 appare negativo, con una significativa tendenza al
declino e alcune fluttuazioni (Fig. 1). In generale, si assiste a un
calo con un accenno di ripresa nel 2007-2008, seguito da nuovo declino.


Tuttavia, emergono forti discrepanze a livello regionale (Tab. \ref{Pedrini_tab_1}),
confermate dall{\textquoteright}effetto significativo del fattore
regione nel modello complessivo (p = 0.005). In particolare, si
evidenzia un marcato declino in Trentino e in Veneto, mentre la
situazione, per quanto riguarda il Friuli, risulta sostanzialmente
stabile, sebbene i dati dei censimenti precoci suggeriscano un calo
anche in questa regione (vedi sotto). 

\section*{Discussione}

I risultati mostrano chiaramente come il re di quaglie sia ancora in
calo a livello nazionale. Confrontando la situazione italiana con
quella degli altri paesi europei, si nota come sia improbabile che il
calo mostrato in Italia possa dipendere da fattori di dinamica globale
delle popolazioni, come le condizioni riscontrate durante lo
svernamento o la migrazione. Nello stesso periodo, infatti, negli altri
stati europei il re di quaglie ha mostrato trend di popolazione
sostanzialmente positivi (Kei\v{s}s 2003, 2004; O{\textquoteright}Brien
\textit{et al.} 2006; BirdLife International 2013), seppur con marcate
fluttuazioni, come tipico per la specie (Rassati \& Tout 2002).
L{\textquoteright}esistenza di un effetto dovuto a fattori locali,
legati alla gestione e conservazione dei suoi habitat di nidificazione,
pare trovare conferma nella discrepanza tra le tendenze demografiche
mostrate nelle regioni da noi considerate, nonostante si tratti di aree
tra loro confinanti. In particolare, la situazione rilevata in Friuli
Venezia Giulia appare meno critica rispetto a quella trentina e veneta,
sebbene i conteggi per la prima parte della stagione suggeriscano un
calo di popolazione anche in questa regione.

Questi risultati confermano la necessit\`a di proseguire nel
monitoraggio di questa specie, il cui stato di conservazione in Italia
risulta {\textquotedblleft}cattivo{\textquotedblright} (Gustin
\textit{et al.} 2009), e il cui andamento demografico lascia presagire
una forte riduzione e contrazione di areale nei prossimi anni, se non
si riuscir\`a a fermare il declino delle popolazioni. In questo senso,
appare fondamentale promuovere forme di gestione degli ambienti prativi
che tengano conto delle esigenze della specie, soprattutto
nell{\textquotesingle}ambito del Piano di Sviluppo Rurale nelle regioni
alpine e in particolare nelle aree dove la densit\`a della specie
rimane relativamente alta (Pedrini \textit{et al}. 2012).

I fattori di minaccia principali per la specie sono essenzialmente
riconducibili a due opposte dinamiche: da un lato,
l{\textquotesingle}intensificazione delle pratiche agricole, con una
gestione dei prati non compatibile con le esigenze della specie
(sfalcio meccanizzato su ampie superfici, rimozione di elementi
marginali, fertilizzazione eccessiva), dall{\textquotesingle}altro,
l{\textquotesingle}abbandono di molte aree
{\textquotedblleft}periferiche{\textquotedblright}, con conseguente
ritorno del bosco e perdita di habitat per il re di quaglie e le altre
specie legate agli ambienti aperti. Localmente, altri fattori come
urbanizzazione, disturbo antropico e da animali domestici e il pascolo
eccessivo, possono avere un impatto negativo sulla specie.

\begin{figure}[!h]
\centering
\includegraphics[width=.8\columnwidth]{Pedrini_fig_1.png}
\caption{Trend di popolazione ($\pm$ errore standard) complessivo del re di quaglie nell'area di studio nel periodo 2000-2012}
\label{Pedrini_fig_1}
\end{figure}

\begin{table}[!h]
\centering
\small
\begin{tabular}{>{\raggedright\arraybackslash}p{.2\columnwidth}>{\raggedright\arraybackslash}p{.18\columnwidth}>{\raggedright\arraybackslash}p{.2\columnwidth}>{\raggedright\arraybackslash}p{.3\columnwidth}}
\toprule
\textbf{Area} & \textbf{Additivo $\pm$ ES} & \textbf{Moltiplicativo} $\pm$ ES & \textbf{Classificazione del trend} \\
\toprule
%\showrowcolors
Totale & -0.0625 $\pm$ 0.0083 & 0.9394 $\pm$ 0.0078 & Declino moderato (p < 0.01) \\
Trentino & -0.0813 $\pm$ 0.0216 & 0.9219 $\pm$ 0.0199 & Declino moderato (p < 0.01) \\
Veneto & -0.0849 $\pm$ 0.0174 & 0.9186 $\pm$ 0.0160 & Declino forte (p < 0.01) \\
Friuli Ven. Giulia & 0.0167 $\pm$ 0.0147 & 1.0168 $\pm$ 0.0149 & Stabile \\
\bottomrule
\hiderowcolors
\end{tabular}
\caption{Trend di popolazione del re di quaglie ($\pm$ errore standard) nell'area di studio nel periodo 2000-2012}
\label{Pedrini_tab_1}
\end{table}

\section*{Ringraziamenti}

Gli autori desiderano ringraziare tutti quanti hanno collaborato alla
ricerca e in particolare: S. Lombardo, Stefano Noselli, Franco
Rizzolli, Francesca Rossi, Karol Tabarelli de Fatis, Gilberto Volcan. 
Ricerca parzialmente finanziata da Progetto BIODIVERSIT\`A
(PAT-2001-05), Accordo di Programma per la Ricerca PAT, 2009-13. 

\section*{Bibliografia}
\begin{itemize}\itemsep0pt
	\item BirdLife International, 2004 - Birds in the European Union: a status
assessment. Wageningen: 

	\item BirdLife International, 2013 - Species factsheet: \textit{Crex crex}.
Available at \href{http://www.birdlife.org}{http:\allowbreak //www.\allowbreak bird\allowbreak li\allowbreak fe.\allowbreak org} (accessed 6 February 2013).

	\item Brambilla M. \& Pedrini P., 2013 - The introduction of subsidies for
grassland conservation in the Italian Alps coincided with population
decline in a threatened grassland species, the Corncrake \textit{Crex
crex}. \textit{Bird Study} 60: 404-408.

	\item Brambilla M. \& Pedrini P., 2011 - Intra-seasonal changes in local
pattern of corncrake Crex crex occurrence require adaptive conservation
strategies in Alpine meadows. \textit{Bird Conserv. Int}. 21:
388{}--393.

	\item Broyer J., 1987 - The habitat of the corncrake \textit{Crex crex} in
France. \textit{Alauda} (55): 161--186 (in French with English
summary). 

	\item Green R.E., Rocamora G. \& Sch\"affer N., 1997 - Populations, ecology
and threats to the Corncrake \textit{Crex crex }in Europe.
\textit{Vogelwelt, }118: 117{}--134.

	\item Gustin M., Brambilla M. \& Celada C., 2009 - Valutazione dello stato di
conservazione dell{\textquoteright}avifauna italiana. Roma: Ministero
dell{\textquoteright}Ambiente, della Tutela del Territorio e del Mare
\& LIPU/BirdLife Italia.

	\item Kei\v{s}s O., 2003 - Recent increases in numbers and the future of
Corncrake \textit{Crex crex} in Latvia. \textit{Ornis Hungarica,}
12{}--13: 151{}--156.

	\item Kei\v{s}s O., 2004 - Results of a survey of Corncrake \textit{Crex crex}
in Latvia, 1989{}--1995. \textit{Bird Census News,} 13: 73{}--76.

	\item Ludwig T., Storch I., \& W\"ubbenhorst J., 2008 - How the black grouse
was lost: Historic reconstruction of its status and distribution in
Lower Saxony (Germany). \textit{J. Ornithol}., (149): 587{}--596.

	\item Moga C. I., Hartel T., \& \"Ollerer K. 2010 - Status, microhabitat use
and distribution of the corncrake \textit{Crex crex }in a Southern
Transylvanian rural landscape, Romania. \textit{North-Western Journal
of Zoology}, 6 (1): 63-70.

	\item O{\textquoteright}Brien M., Green R.E. \& Wilson J. 2006 - Partial
recovery of the population of Corncrakes \textit{Crex crex }in Britain,
1993{}--2004. \textit{Bird Study,} 53: 213{}--224.

	\item Pannekoek J. \& Van Strien A.J., 2001 - TRIM (Trends and Indices for
Monitoring Data). Statistics Netherlands, Voorburg. 

	\item Pedrini P., Rizzolli F., Rossi F. \& Brambilla M., 2012 - Population
trend and breeding density of corncrake \textit{Crex crex} (Aves:
Rallidae) in the Alps: Monitoring and conservation implications of a
15-years survey in Trentino, Italy\textit{. }\textit{Ital. J. Zool}.,
79: 377{}--384.

	\item {Rassati G. 2001 - Il Re
di quaglie
}\textit{{Crex
crex}}{ durante
l{\textquoteright}anno 2000 in due aree campione in Carnia (Alpi
Orientali, Friuli-Venezia Giulia).
}\textit{{Avocetta,}}{
25 (1): 239.}

	\item {Rassati G., 2004 -
Evoluzione faunistica nelle aree rurali abbandonate. La presenza del Re
di quaglie
(}\textit{{Crex
crex}}{) e della Lepre
comune (}\textit{{Lepus
europaeus}}{).
}\textit{{Agribusiness
Paesaggio \&
Ambiente}}{ VII (1):
41-48.}

	\item {Rassati G.,
}{2009 -
}{The spring and summer
censuses of Corncrake
}\textit{{Crex
crex}}{ in three sample
areas of Carnia (Eastern Alps, Friuli-Venezia Giulia, North-eastern
Italy) (Years 2000-2005).
}{Gli Uccelli
d{\textquoteright}Italia XXXIV: 50-57.}

	\item {Rassati G. \& Rodaro
P., }{2007 - Habitat,
vegetation and land management of Corncrake
}\textit{{Crex
crex}}{ breeding sites
in Carnia (Friuli-Venezia Giulia, NE Italy).
}\textit{{Acrocephalus,}}{
28 (133): 61-68.}

	\item {Rassati, G. \& Tout
C.P., 2002 - }{The
Corncrake
(}\textit{{Crex
crex}}{) in
Friuli-Venezia Giulia (North-eastern Italy).
}\textit{{Avocetta,}}{
26 (1): 3-6.}

	\item {Soldaat L., Visser H.,
Roomen M. \& Strien A.,
}{2007 -
}{Smoothing and }trend
detection in waterbird monitoring data using structural time-series
{analysis and the
Kalman filter.
}\textit{{J.
Ornithol}}{.,
}{148:
}{S351}{{}--}{S357.}
\end{itemize}
