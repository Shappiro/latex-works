\setcounter{figure}{0}
\setcounter{table}{0}

\begin{adjustwidth}{-3.5cm}{0cm}
\pagestyle{CIOpage}
\authortoc{\textsc{Assandri G.}, \textsc{Marotto P.}}
\chapter*[Il gabbiano reale nordico e il gabbiano reale pontico in
Piemonte. Una revisione critica]{Il gabbiano reale nordico \textbf{\textit{Larus
argentatus}}\textbf{ (Pontoppidan, 1763) e il gabbiano reale pontico
}\textbf{\textit{Larus cachinnans}}\textbf{ (Pallas, 1811) in Piemonte:
una revisione critica}}

\addcontentsline{toc}{chapter}{Il gabbiano reale nordico e il gabbiano reale pontico in
Piemonte}
\textsc{Giacomo Assandri}$^{1,2*}$, \textsc{Paolo Marotto}$^{1,2}$\\

\index{Assandri Giacomo} \index{Marotto Paolo}
\noindent\color{MUSEBLUE}\rule{27cm}{2pt}
\vspace{1cm}
\end{adjustwidth}



\marginnote{\raggedright $^1$Gruppo Piemontese Studi Ornitologici
{\textquotedblleft}F.A. Bonelli{\textquotedblright} ONLUS, Museo di
Storia Naturale, via San Francesco di Sales 188 - 10022 Carmagnola (TO) \\
$^2$Torino Birdwatching - Associazione EBN Italia, Via
Peyron 10 - 10143 Torino \\
\vspace{.5cm}
{\emph{\small $^*$Autore per la corrispondenza: \href{mailto:giacomo.assandri@gmail.com}{gia\allowbreak co\allowbreak mo.\allowbreak as\allowbreak san\allowbreak dri@\allowbreak g\allowbreak ma\allowbreak il.\allowbreak co\allowbreak m}}} \\
\keywords{\textit{Larus argentatus}, \textit{Larus cachinnans}, Piemonte}
{\textit{Larus argentatus}, \textit{Larus cachinnans}, Pedmont}
%\index{keywords}{\textit{Larus argentatus}} \index{keywords}{\textit{Larus cachinnans}} \index{keywords}{Piemonte}
} 
{\small
\noindent \textsc{\color{MUSEBLUE} Summary} / We present a revision of the status of Caspian and herring gulls
\textit{Larus cachinnans}\textit{ and Larus argentatus} in Piedmont (NW
Italy). Both species must be still considered vagrant in this region on
the basis of confirmed records.
}


\section*{Introduzione}

Sino al 2008, il gabbiano reale nordico era considerato, in Piemonte,
migratore irregolare e svernante occasionale sulla base di tre dati
storici certi e alcuni recenti in gran parte privi di documentazione.
Il gabbiano reale pontico era invece considerato accidentale sulla base
di quattro osservazioni adeguatamente documentate (Pavia \& Boano
2009). Come nel resto d{\textquoteright}Italia (Brichetti \& Fracasso
2006), anche in questa regione si sta recentemente assistendo a un
sensibile aumento delle segnalazioni di entrambe le specie; tuttavia
l{\textquoteright}identificazione di questi Laridi non \`e mai agevole
(Gibbins \textit{et al.} 2010) e, al fine di meglio definirne lo status
(secondo criteri AERC TAC 2003), si \`e resa necessaria la presente
revisione critica di tutti i dati disponibili. 

\section*{Metodi}

Sono state raccolte e analizzate tutte le segnalazioni note al 2012 e
quelle documentate fotograficamente sono state sottoposte alla
validazione di due esperti della COI, che hanno permesso di
classificare le segnalazioni in: confermate, non confermate, non
confermabili sulla base della documentazione disponibile o per
disaccordo dei validatori. Per entrambe le specie, basandosi sui
risultati della validazione, \`e stato inoltre calcolato un tasso di
errore sull{\textquoteright}identificazione:

\begin{center}
\textit{
(n{\textdegree} di foto erroneamente
identificate/n{\textdegree} di foto sottoposte a validazione)*100}
\end{center}

\section*{Risultati e discussione}

Per il gabbiano reale nordico sono stati analizzati 43 dati di presenza relativi
a 64 individui, successivi ai tre storici e riferiti al periodo
1987-2012. Di questi solo 5 sono corredati da foto (6 ind.), che hanno
permesso di confermare l{\textquoteright}identificazione
dell{\textquoteright}osservatore in 3 casi (3 ind.), mentre in 1 caso
(2 ind.) l{\textquoteright}identificazione \`e risultata errata (tasso
d{\textquoteright}errore=20\%) e in 1 caso (1 ind.) i validatori non
erano concordi. Il gabbiano reale pontico \`e stato segnalato per la prima volta
in Piemonte nel 2002 e al 2012 esistono 53 dati di presenza riferiti a
78 individui. Di questi 17 (20 ind.) sono corredati da foto, che hanno
permesso di confermare l{\textquoteright}identificazione
dell{\textquoteright}osservatore in 8 casi (9 ind.), mentre in 2 casi
(2 ind.) l{\textquoteright}identificazione \`e risultata errata (tasso
d{\textquoteright}errore=12\%), in 6 (7 ind.) non confermabile a causa
della scarsa qualit\`a della documentazione e in 1 caso (2 ind.) i
validatori non erano concordi. Il gabbiano reale nordico \`e stato contattato in
tutti i mesi compresi tra settembre e aprile, mentre il gabbiano reale pontico
tra novembre e aprile.

La presente revisione ha permesso di delineare il seguente quadro: I) Il
gabbiano reale nordico appare, diversamente da quanto noto finora, meno
frequente del gabbiano reale pontico in Piemonte, sebbene
quest{\textquoteright}ultimo sia stato contattato regolarmente solo a
partire 2006. II) Sulla base delle osservazioni documentate e
confermate, e quindi utilizzando un metodo conservativo di attribuzione
dello status, il gabbiano reale nordico \`e da considerarsi accidentale in
Piemonte con 6 segnalazioni, mentre il gabbiano reale pontico accidentale con 8.
Se la tendenza all{\textquoteright}incremento delle osservazioni si
manterr\`a tale, \`e prevedibile una rapida evoluzione di questo
status, che potrebbe coincidere con quello attuale, attribuito sulla
base di tutti i dati disponibili, anche non documentati (esclusi quelli
non validati): gabbiano reale nordico migratore regolare, svernante irregolare;
gabbiano reale pontico migratore irregolare, svernante irregolare. 

Sull{\textquoteright}origine dei soggetti piemontesi si \`e a conoscenza
di un gabbiano reale nordico ripreso nel marzo del 1958 sul Toce a Domodossola
(VB), inanellato nel luglio 1955 a Mellum, Germania (Moltoni 1973) e di
un gabbiano reale pontico inanellato da pullus a Kreminciuk (Ucraina) nel giugno
2010 e riosservato nel febbraio 2011 e 2012 presso
l{\textquoteright}Invaso del Meisino (TO) e pochi giorni dopo (nel
2012) anche sul lago di Varese. Tenendo conto delle difficolt\`a
oggettive di identificazione, confermate dai tassi di errore calcolati
nel presente contributo, \`e comunque ancora auspicabile da parte degli
osservatori un attivo sforzo di documentazione di queste due specie in
Piemonte.

\section*{Ringraziamenti}

Desideriamo ringraziare tutti gli osservatori che hanno reso disponibili
le loro segnalazioni e fotografie; Ottavio Janni e Michele Vigan\`o
(COI) per la validazione delle fotografie, Adriano Talamelli per la
comunicazione delle riletture.

\section*{Bibliografia}
\begin{itemize}\itemsep0pt
	\item AERC TAC, 2003 - AERC TAC{\textquoteright}s Taxonomic Recommendations. \\
	\url{http://www.aerc.eu/DOCS/AERCTAC.pdf}. 

	\item Brichetti P. \& Fracasso G., 2006 - \textit{Ornitologia Italiana} Vol. 3
-- \textit{Stercorariidae} - \textit{Caprimulgidae}. Alberto Perdisa Editore.

	\item Gibbins C., Small B. J. \& Sweeney J., 2010 - Identification of Caspian
Gull. \textit{British Birds}, 103: 142-183. 

	\item Moltoni E., 1973 - Elenco di parecchie centinaia di uccelli inanellati
all{\textquotesingle}estero e ripresi in Italia ed in Libia.
\textit{Rivista Italiana di Ornitologia, }43 (Suppl.): 1-182. 

	\item Pavia M. \& Boano G., 2009 - Check-list degli uccelli del Piemonte e
della Valle d{\textquoteright}Aosta aggiornata al dicembre 2008.
\textit{Rivista Italiana di Ornitologia}, 79: 23-47. 
\end{itemize}
