\setcounter{figure}{0}
\setcounter{table}{0}

\begin{adjustwidth}{-3.5cm}{0cm}
\pagestyle{CIOpage}
\authortoc{\textsc{La Mantia T.}, \textsc{Massa B.},
\textsc{Pipitone S.}, \textsc{R\"uhl J.}}
\chapter*[Gli uccelli come vettori di dispersione dei semi]{\bfseries
Il ruolo degli uccelli come vettori di dispersione durante le
successioni secondarie}
\addcontentsline{toc}{chapter}{Gli uccelli come vettori di dispersione dei semi}

\textsc{Tommaso La Mantia}$^{1*}$, \textsc{Bruno Massa}$^1$,
\textsc{Sergio Pipitone}$^1$, \textsc{Juliane R\"uhl}$^1$ \\

\index{La Mantia Tommaso} \index{Massa Bruno} \index{Pipitone Sergio} \index{R\"uhl Juliane}
\noindent\color{MUSEBLUE}\rule{27cm}{2pt}
\vspace{1cm}
\end{adjustwidth}



\marginnote{\raggedright $^1$Dipartimento SAF - Viale delle Scienze, Ed. 4, Ingresso H, 90128
Palermo \\
\vspace{.5cm}
{\emph{\small $^*$Autore per la corrispondenza: \href{mailto:tommaso.lamantia@unipa.it}{tom\allowbreak ma\allowbreak so.\allowbreak la\allowbreak man\allowbreak ti\allowbreak a@\allowbreak uni\allowbreak pa.\allowbreak it}}} \\
\keywords{Pantelleria, Uccelli, semi, specie arbustive}
{Pantelleria, Birds, secondary
succession, seed dispersion}
%\index{keywords}{Pantelleria} \index{keywords}{Semi} \index{keywords}{Specie arbustive}
}
{\small
\noindent \textsc{\color{MUSEBLUE} Summary} / Birds play an important role in the spread of seeds of shrubs and trees.
The abandonment processes that characterized the European countries and
in particular the countries of the Mediterranean have led to rapid
processes of colonization by vegetation. The speed with which these
processes occur depends in an essential way from the success of woody
plants to colonize the abandoned areas and is influenced in an
important way by birds. There are few quantitative data on the role
played by birds. We started a study on the role played by birds in
secondary successions in the island of Pantelleria, island affected by
heavy phenomena of abandonment. The test was conducted by placing
perches structured so you can collect the feces of birds. Preliminary
results indicate that: 1) the amount of seeds of woody species
dispersed by birds decreases with increasing distance of the mother
plants; 2) dispersion of seeds by birds is the most important type for
the colonization of abandonment field: 3) rats and rabbits also play an
important role in the dispersion of the woody species, but only in the
first few tens of meters from the mother plants.\\
}

\section*{Introduzione}


Nel XX secolo, l{\textquoteright}Europa \`e stata caratterizzata da
forti processi di abbandono dell{\textquoteright}agricoltura. Nelle
aree agricole abbandonate non soggette a disturbi (pascolo, incendio),
le dinamiche della successione secondaria (cio\`e il processo di
ricostituzione della vegetazione dopo che la copertura vegetale
preesistente \`e stata distrutta da un disturbo) hanno portato alla
formazione di comunit\`a vegetali pre-forestali e forestali. La
velocit\`a con cui avvengono questi processi dipende in maniera
essenziale dal successo delle piante legnose a colonizzare le aree
abbandonate e viene influenzato in maniera importante dagli uccelli
(R\"uhl \& Schnittler 2011). Per valutare il ruolo degli uccelli \`e
stato avviato uno studio in Sicilia, a Pantelleria, isola interessata
da forti fenomeni di abbandono (La Mantia et al., 2008), attraverso la
collocazione di posatoi per testare due ipotesi: 1) se la quantit\`a di
semi delle specie legnose della macchia dispersi dagli uccelli
diminuisce con l{\textquoteright}aumento della distanza della macchia;
2) se la dispersione ornitocora \`e pi\`u importante della dispersione
non-ornitocora per la colonizzazione degli ex-coltivi. 

\section*{Metodi}

Nel febbraio 2011 sono state posizionate 7 repliche di coppie
{\textquotedblleft}posatoi{\textquotedblright} e
{\textquotedblleft}vaschetta a terra{\textquotedblright} in giovani
ex-coltivi (vigneti-cappereti). In ogni replica, una coppia di posatoi
e vaschette (distanza tra di loro 10 m) sono stati messi,
rispettivamente, a 30, 60 e 120 m dal limite tra
l{\textquoteright}ex-coltivo e un{\textquoteright}area di macchia
mediterranea. Tra novembre 2011 e marzo 2012 sono stati effettuati
quattro controlli nei posatoi e nelle vaschette e quindi raccolti i
semi contenuti al loro interno che sono stati suddivisi in tre grandi
gruppi: 1) arbusti della macchia (= \textit{Pistacia lentiscus},
\textit{Arbutus unedo}, \textit{Phillyrea latifolia}, \textit{Myrtus
communis}, \textit{Daphne gnidium}, \textit{Teline monspessulana}); 2)
piante un tempo coltivate (\textit{Vitis vinifera}, \textit{Capparis}
\textit{spinosa}); 3) altre piante legnose (\textit{Rubus ulmifolius},
\textit{Rubia peregrina}, \textit{Prasium majus}). Sono stati raccolti
in totale 2272 semi di specie legnose, la maggior parte tra novembre e
dicembre. Nelle vaschette sono stati trovati feci di ratti e conigli, e
all{\textquoteright}interno di queste erano presenti semi. Con
riferimento agli arbusti della macchia, sono stati trovati molto pi\`u
semi nei posatoi che nelle vaschette (ca. il doppio nelle distanze di
30 e 90 m, e ca. 10 volte di pi\`u a 60 m). Inoltre, solo nel caso dei
posatoi vi \`e una diminuzione continua del numero dei semi con
distanza crescente dalla macchia, mentre nel caso delle vaschette il
numero dei semi a 30 m era uguale a quello a 90 m, e nel caso delle
feci trovate dentro le vaschette, il numero dei semi era pi\`u alto a
30 m, basso a 60 m ed intermedio a 90 m. Per le piante un tempo
coltivate, non sono state trovate delle differenze tra il numero di
semi nei posatoi e nelle vaschette a 30 e 60 m, mentre a 90 m sono
stati trovati molto pi\`u semi nelle vaschette che nei posatoi.
Inoltre, sono stati quasi assenti i semi di vite e cappero nelle feci
trovate all{\textquoteright}interno delle vaschette. I semi delle
{\textquotedblleft}altre legnose{\textquotedblright} nelle vaschette
aumentano in quantit\`a con l{\textquoteright}incremento della distanza
della macchia, mentre quelli nelle feci nelle vaschette diminuiscono
lungo il \textit{transect}. In termini assoluti, solo a 60 m sono stati
trovati pi\`u semi nei posatoi che nelle vaschette. 

\section*{Conclusioni}

I risultati del primo anno di raccolta indicano che: 1) la quantit\`a di
semi delle specie legnose della macchia dispersi dagli uccelli negli
ex-coltivi diminuisce con l{\textquoteright}aumento della distanza
delle piante madri; 2) la dispersione ornitocora \`e la tipologia pi\`u
importante per la colonizzazione degli ex-coltivi da parte delle specie
della macchia; 3) anche i ratti e i conigli svolgono un ruolo
importante di dispersione delle specie legnose della macchia, ma solo
nelle prime decine di metri dalla macchia.

\section*{Ringraziamenti}

Ricerca condotta nell{\textquoteright}ambito del progetto MIUR-PRIN
{\textquotedblleft}Strategie nazionali per la mitigazione dei
cambiamenti climatici in sistemi arborei agrari e forestali
(CARBOTREES){\textquotedblright}.

\section*{Bibliografia}
\begin{itemize}\itemsep0pt
	\item La Mantia T., R\"uhl J., Pasta S., Campisi D. \& Terrazzino G., 2008 -
Structural analysis of woody species in Mediterranean old fields.
\textit{Plant Biosystems}, 142 (3): 462-471.

	\item R\"uhl J. \& Schnittler M., 2011 - An empirical test of neighbourhood
effect and safe-site effect in abandoned mediterranean vineyards.
\textit{Acta Oecol}., 37: 71-78.
\end{itemize}
