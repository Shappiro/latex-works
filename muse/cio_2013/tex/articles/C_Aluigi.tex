\setcounter{figure}{0}
\setcounter{table}{0}

\begin{adjustwidth}{-3.5cm}{0cm}
\pagestyle{CIOpage}
\authortoc{Aluigi A., Fasano G. S., Baghino L., Campora M., Cottalasso R., Toffoli R.}
\chapter*[Rete Natura 2000 in Liguria]{Importanza conservazionistica della Rete Natura 2000 in
Liguria}
\addcontentsline{toc}{chapter}{Rete Natura 2000 in Liguria}

\textsc{Antonio Aluigi}$^{1}$, \textsc{Sergio G. Fasano}$^{1*}$,
\textsc{Luca Baghino}$^{1}$, \textsc{Massimo Campora}$^{1}$,
\textsc{Renato Cottalasso}$^{1}$, \\\textsc{Roberto Toffoli}$^{1}$ \\

\index{Aluigi Antonio} \index{Fasano G. Sergio} \index{Cottalasso Renato} \index{Toffoli Roberto}
\noindent\color{MUSEBLUE}\rule{27cm}{2pt}
\vspace{1cm}
\end{adjustwidth}



\marginnote{\raggedright $^1$Ente Parco del Beigua - Via Marconi 165, 16011
Arenzano GE - biodiv@parcobeigua.it \\
\vspace{.5cm}
{\emph{\small $^*$Autore per la corrispondenza: \href{mailto:fasanosg@gmail.com}{fa\allowbreak sa\allowbreak no\allowbreak sg@\allowbreak g\allowbreak ma\allowbreak il.\allowbreak com}}} \\
\keywords{Liguria, Parco del Beigua, monitoraggio comunit\`a}
{Liguria Region, Beigua Natural Park, monitoring, birds}
%\index{keywords}{Liguria} \index{keywords}{Parco del Beigua} \index{keywords}{Monitoraggio comunit\`a}
}
\noindent \textsc{\color{MUSEBLUE} Summary} / During 2006-2012,the Liguria Region developed a project to monitor
birds as part of an integrated system of surveys~(breeding bird
communitites in all sites, and a focus on target species in sample
areas).~ The monitoring took place in several Natura 2000 sites~as well
as~other areas.

We compared breeding birds communities detected in 5274 point counts,
confirming the conservation importance, both qualitative and
quantitative, of the sites belonging to the Natura 2000 network
compared with other areas. In addition, the likely trends of the
species exhibit significant differences within and outside Natura 2000
sites, with trends generally positive (or less negative) within the
Natura 2000 areas.\\
\noindent \textsc{\color{MUSEBLUE} Riassunto} / Tra il 2008 ed il 2012 nella Regione Liguria \`e tato condotto, mediante
l{\textquoteright}attuazione di un sistema integrato di censimenti, un
progetto di monitoraggio dell{\textquotesingle}avifauna nei siti della
Rete Natura 2000 e in altre aree di elevato interesse.

Il confronto delle comunit\`a ornitiche nidificanti rilevate in 5274
punti di ascolto conferma l{\textquotesingle}importanza
conservazionistica, sia qualitativa che quantitativa, delle aree
appartenenti alla Rete Natura 2000 rispetto alle altre aree. Inoltre, i
probabili andamenti delle specie evidenziano differenze significative
all{\textquotesingle}interno ed all{\textquotesingle}esterno dei siti
della Rete Natura 2000, con tendenze generalmente positive (o meno
negative) all{\textquotesingle}interno dei siti Natura 2000. 



\section*{Introduzione}

Tra il 2008 ed il 2012 \`e stato condotto un progetto di
{\textquotedblleft}monitoraggio della comunit\`a ornitica nelle ZPS e
nelle aree liguri a maggiore vocazionalit\`a avifaunistica ed
agricola{\textquotedblright} (Nicosia \textit{et al. }2009a, 2009b;
Fasano \textit{et al}. 2012, 2013). Il progetto,  promosso e finanziato
dalla Regione Liguria, e attuato dal Parco Naturale Regionale del
Beigua, rientra in un pi\`u vasto monitoraggio delle specie di
interesse conservazionistico, avviato dalla Regione nel 2007 in
adempimento alle Direttive 92/43/CEE
({\textquotedblleft}Habitat{\textquotedblright}) e 2009/147/CE
({\textquotedblleft}Uccelli{\textquotedblright}). Con il presente
contributo si \`e proceduto a testare eventuali differenze qualitative
e quantitative tra le aree afferenti alla Rete Natura 2000 e il resto
del territorio regionale.

\section*{Metodi}
Il progetto \`e basato su uno sforzo di campionamento maggiormente
approfondito in alcune aree della Rete Natura 2000 (sette ZPS e quattro
SIC), dove \`e incentrato sul monitoraggio di specie \textit{target}, e
sul censimento annuale della comunit\`a ornitica nidificante, sia nei
Siti Natura 2000 precedentemente individuati che in altri aggiuntivi e
in un numero variabile di  particelle UTM  (10 x 10 km di lato) in
parte ripetute e in parte scelte di anno in anno in modo da indagare
l{\textquoteright}intero territorio regionale nel corso dei cinque anni
previsti.

Per la caratterizzazione e il monitoraggio
dell{\textquoteright}ornitocenosi nidificante la tecnica di rilevamento
prescelta \`e stata quella dei punti di ascolto senza limiti di
distanza (Blondel \textit{et al}. 1981; Nicosia \textit{et al.} 2009b).
Considerando anche dati pregressi, per il periodo 2000-2012 \`e
disponibile un campione di 5274 stazioni di rilevamento (distribuite in
88 particelle UTM sul totale delle 90 afferenti alla Regione Liguria).
Per un pi\`u dettagliato confronto tra i siti liguri della Rete Natura
2000 rispetto alle altre aree, si \`e poi utilizzato un campione di
4142 punti d{\textquoteright}ascolto relativo agli anni 2008-2012; di
questi 2118 ricadono all{\textquoteright}interno di Siti Natura 2000
(7 ZPS e 49 SIC) e interessano 24 Aree Protette.

La valutazione degli andamenti delle specie comuni (anni 2000-2012;
Fasano \textit{et al}. 2012) \`e stata effettuata utilizzando il
software TRIM (TRends \& Indices for Monitoring Data), come indicato da
Gregory \textit{et al}. (2005) adottando una metodologia analoga a
quella applicata a livello nazionale per
l{\textquoteright}identificazione delle specie legate agli
agroecosistemi e ambienti boschivi (Fornasari \textit{et al.} 2002;
Tellini \textit{et al.} 2005; LIPU 2011); si \`e inoltre definita la
vocazione ambientale delle principali specie nidificanti. Di ciascun
gruppo, calcolando la media geometrica degli indici di popolazione
delle specie ad esso appartenenti (Gregory \textit{et al}. 2005), \`e
stato poi elaborato un indicatore\textbf{ }di stato di conservazione
complessivo.

Sono state quindi svolte ulteriori analisi di tendenza demografica,
verificando eventuali differenze esistenti all{\textquotesingle}interno
e all{\textquotesingle}esterno dei Siti della Rete Natura 2000. Per
fare ci\`o una particella \`e stata considerata afferente alla Rete
Natura 2000 quando almeno il 50\% dei suoi punti risultavano ricompresi
all{\textquotesingle}interno di uno o pi\`u Siti della rete;
successivamente l{\textquotesingle}appartenenza della particella alla
Rete Natura 2000 \`e stata utilizzata direttamente come covariata
categoriale nell{\textquotesingle}analisi di TRIM. Nel complesso 2603
punti d{\textquoteright}ascolto (il 49,4\% del campione relativo al
periodo 2000-2012) ricadono in Siti della rete e otto particelle, sulle
21 selezionate per il calcolo degli andamenti, rispondono ai criteri
sopra esposti.

\section*{Risultati e discussione}

Analizzando i valori medi per punto d{\textquoteright}ascolto di alcuni
parametri calcolati per le aree liguri della Rete Natura 2000
({\textquotedblleft}N2000{\textquotedblright}) rispetto alle altre aree
({\textquotedblleft}aa{\textquotedblright}), possiamo riscontrare come
le prime siano in maniera statisticamente significativa pi\`u
importanti per ci\`o che riguarda la diversit\`a, espressa attraverso
l{\textquoteright}indice di Shannon-Weaver (N2000: 1,86 {\textpm} 0,36;
aa: 1,80 {\textpm} 0,36;  t = 5,10, P {\textless} 0,001; MacArthur
1965), il valore nazionale corretto dall{\textquoteright}abbondanza
specifica (N2000: 33,5 {\textpm} 3,21; aa: 31,4 {\textpm} 3,21;  t =
21,16, P {\textless} 0,001; Brichetti \& Gariboldi 1992) e il numero di
specie incluse nell{\textquoteright}All. 1 della Direttiva
{\textquoteleft}Uccelli{\textquoteright} (N2000: 0,24 {\textpm} 0,494;
aa: 0,09 {\textpm} 0,326;  t = 11,35, P {\textless} 0,001), mentre non
si evidenziano differenze per quanto concerne la ricchezza specifica
(N2000: 7,75 {\textpm} 2,619; aa: 7,69 {\textpm} 2,523;  t = 0,83, P =
0,408). 

Considerando i parametri pi\`u strettamente legati al valore
conservazionistico dell{\textquoteright}ornitocenosi presente,
osserviamo poi come esistano differenze significative in relazione al
tipo di protezione cui \`e assoggettata l{\textquoteright}area nella
quale ricade il punto, e cio\`e se questo \`e al di fuori di aree
protette, in Aree Protette ({\textquoteleft}AP{\textquoteright}), in
Siti della Rete Natura 2000 oppure in territori che appartengano
contemporaneamente sia ad Aree Protette che a Siti della Rete Natura
2000 ({\textquoteleft}APeN2000{\textquoteright}). Il valore nazionale
corretto dall{\textquoteright}abbondanza specifica presenta
significative differenze fra le quattro categorie (APeN2000: 33,9
{\textpm} 3,28; N2000: 32,9 {\textpm} 3,01; AP: 32,2 {\textpm} 3,29;
aa: 31,3 {\textpm} 3,20; F\textsubscript{3,4135} = 172,55, P
{\textless} 0,001; test di Tukey, P {\textless} 0,05); il numero di
specie incluse nell{\textquoteright}All. 1 della Direttiva
{\textquotedblleft}Uccelli{\textquotedblright}, presenta
anch{\textquoteright}esso significative differenze fra le categorie
(APeN2000: 0,24 {\textpm} 0,487; N2000: 0,23 {\textpm} 0,506; AP: 0,08
{\textpm} 0,256; aa: 0,09 {\textpm} 0,331; F\textsubscript{3,4135} =
42,48, P {\textless} 0,001), che si raggruppano per\`o in due
sottoinsiemi omogenei (1: APeN2000 e N2000, 2: AP e aa; test di Tukey,
P {\textless} 0,05).  

Per quanto riguarda le tendenze demografiche (Fasano \textit{et al}.
2012), nel periodo 2000-2012 si \`e riscontrata una relativa
stabilit\`a delle popolazioni, ma con molte specie in diminuzione.
Sulle 54 specie considerate, il 39\% risultano tendenti
all{\textquoteright}aumento o stabili, il 37\% tendono alla
diminuzione, mentre per il 24\% la tendenza non \`e ancora
statisticamente definita. La tendenza alla diminuzione \`e pi\`u
marcata per le specie che nidificano in ambienti agrari e antropizzati
e per quelle che svernano nell{\textquoteright}Africa sub-sahariana.
Invece le specie migratrici intra-paleartiche e quelle legate alle
praterie presentano andamenti che tendono alla stabilit\`a, e le specie
che preferiscono ambienti boscati mostrano incremento moderato. 

L{\textquoteright}andamento dell{\textquoteright}indicatore complessivo,
calcolato per tutte le 54 specie, \`e simile e coerente sia
all{\textquoteright}interno che all{\textquoteright}esterno dei Siti
Natura 2000, ma con una tendenza meno negativa per questi ultimi.
L{\textquoteright}andamento \`e invece nettamente differenziato per le
specie forestali e di prateria, con tendenze positive entro i Siti
Natura 2000 e negative all{\textquoteright}esterno; mentre per
l{\textquoteright}indice relativo alle specie degli agroecosistemi la
tendenza, pur essendo simile, si inverte; ma ci\`o \`e probabilmente
riconducibile, nei settori  interessati dal campione analizzato, alla
scarsa disponibilit\`a di questi ambienti, dovuta
all{\textquoteright}abbandono delle attivit\`a agricole nelle aree
svantaggiate come quelle appenniniche.

Riscontriamo poi differenze significative all{\textquoteright}interno ed
all{\textquoteright}esterno dei Siti della Rete Natura 2000 per 13
specie (P {\textless} 0,05), delle quali quattro incluse
nell{\textquoteright}All. 1 della Direttiva
{\textquotedblleft}Uccelli{\textquotedblright}. Di queste ultime, in
tre casi osserviamo come l{\textquoteright}andamento degli indici
diverga, con tendenze positive (biancone \textit{Circaetus gallicus} e
magnanina comune \textit{Sylvia undata}) o di stabilit\`a (averla
piccola \textit{Lanius collurio}) all{\textquoteright}interno dei Siti
Natura 2000, e negative all{\textquoteright}esterno. La tottavilla
\textit{Lullula arborea}, che complessivamente presenta una diminuzione
moderata, mostra andamento simile nei due ambiti, ma con una tendenza
decisamente meno negativa all{\textquoteright}interno dei Siti Natura
2000.

Questi risultati, che confermano ulteriormente
l{\textquoteright}importanza conservazionistica della Rete Natura 2000,
sono probabilmente riconducibili non solo alle eventuali modalit\`a di
gestione attiva dei siti, ma anche al fatto che queste zone risultano,
con poche eccezioni, meno interessate (o meglio mitigate) da quei
processi che, come l{\textquotesingle}aumento incontrollato delle
superfici edificate, ha determinato drammatici cambiamenti nel
paesaggio e che, come evidenziato da Rete Rurale Nazionale \& LIPU
(2012, 2013), in certe condizioni \`e ad oggi probabilmente una delle
cause principali, se non la pi\`u importante, del declino degli uccelli
degli ambienti antropizzati e agrari.

\section*{Bibliografia}
\begin{itemize}\itemsep0pt
	\item Blondel J., Ferry C. \& Frochot B., 1981 - Point Counts with Unlimited
distance. In: Estimating Numbers of terrestrial birds. \textit{Studies
in Avian Ecologies,} 6: 414-420. 

	\item Brichetti P. \& Gariboldi A., 1992 - Un
{\guillemotleft}valore{\guillemotright} per le specie ornitiche
nidificanti in Italia. \textit{Riv.ital.Orn.,} 62: 73-87. 

	\item Fasano S. \& Aluigi A., 2007 - Dati preliminari sulla densit\`a
riproduttiva di Calandro \textit{Anthus campestris }e Magnanina comune
\textit{Sylvia undata}\textit{ }nel Parco del Beigua e nella ZPS
{\textquotedblleft}Beigua-Turchino{\textquotedblright} (GE-SV).
Abstract del XIV Convegno Italiano di Ornitologia. Trieste 26-30
settembre 2007: 47. 

	\item Fasano S., Baghino L. \& Aluigi A., 2009 - La
{\textquotedblleft}Canellona{\textquotedblright}: un \textit{hot-spot}
per l{\textquoteright}Averla piccola. (SIC IT1331402). Atti del XV
Convegno Italiano di Ornitologia. Parco Nazionale del Circeo, Sabaudia
(Latina) 14-18 ottobre 2009. \textit{Alula,} XVI (1-2): 544-546.

	\item Fasano S. G., Aluigi A., Baghino L., Campora M., Cottalasso R. \&
Toffoli R., 2012 - Monitoraggio della comunit\`a ornitica nelle
ZPS e nelle aree liguri di maggiore vocazionalit\`a avifaunistica e/o
agricola. Anno 2012. Regione Liguria -- Parco del Beigua, 235 pp.

	\item Fasano S.G., Cottalasso R., Campora M., Baghino L., Toffoli R. \& Aluigi
A. (a cura di), 2013 - Ambienti e Specie del Parco del Beigua e
dei Siti della Rete Natura 2000 funzionalmente connessi. Ente Parco
del Beigua, 100 pp. 

	\item Fornasari L., De Carli E., Brambilla S., Buvoli L., Maritan E. \&
Mingozzi T., 2002 - Distribuzione dell{\textquoteright}avifauna
nidificante in Italia: primo bollettino del progetto di
monitoraggio MITO 2000. \textit{Avocetta,} 26 (2): 59-115.

	\item Gregory R.D., van Strien A., Vorisek P., Gmelig Meyling A.W., Noble D.,
Foppen R. \& Gibbons D.W., 2005 - Developing indicators for European
birds. \textit{Phil. Trans. R. Soc. B.}, 360: 269-288.

	\item Nicosia E., Aluigi A., Fasano S. \& Toffoli R., 2009 - La Rete Natura
2000 in Liguria: caratterizzazione e con\allowbreak fronto di alcune realt\`a.
Atti del XV Convegno Italiano di Ornitologia. Parco Nazionale del
Circeo, Sabaudia (Latina) 14-18 ottobre 2009. \textit{Alula,} XVI
(1-2): 558-560.

	\item Nicosia E.,  Aluigi A., Fasano S., Baghino L., Campora M., Cottalasso
R., Toffoli R. \& Ballerini M., 2009b - Il monitoraggio della Rete
Natura 2000 in Liguria. Atti del XV Convegno Italiano di Ornitologia.
Parco Nazionale del Circeo, Sabaudia (Latina) 14-18 ottobre 2009.
\textit{Alula} XVI (1-2): 519-524. 

	\item MacArthur R.H., 1965 - Patterns of species diversity. \textit{Biol.
}\textit{Rev., }40:510-533. 

	\item LIPU, 2011 - Censimento dell{\textquoteright}avifauna per la
definizione del Farmland Bird Index a livello na\allowbreak zionale e
regionale in Italia. \textit{Farmland Bird Index e Woodland Bird Index
-- 2000-2010}. Rete Rurale Nazionale, 2007-2013. 

	\item Rete Rurale Nazionale \& LIPU, 2012 - Censimento
dell{\textquoteright}avifauna per la defnizione del Farmland Bird
Index a livello na\allowbreak zionale e regionale in Italia. \textit{Farmland
Bird Index e Woodland Bird Index -- 2000-2011}. Rete Rurale Nazionale,
2007-2013.

	\item Rete Rurale Nazionale \& LIPU, 2013 - \textit{Uccelli comuni in Italia.
Aggiornamento degli andamenti di popolazione al 2012}. Rete Rurale
Nazionale \& LIPU.

	\item Tellini Florenzano G., Buvoli L., Caliendo M.F., Rizzolli F. \&
Fornasari L., 2005 - Definizione dell{\textquoteright}ecologia
degli uccelli italiani mediante indici nazionali di selezione
d{\textquoteright}habitat. \textit{Avocetta}, 29 (n.s.): 148.
\end{itemize}
