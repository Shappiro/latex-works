\setcounter{figure}{0}
\setcounter{table}{0}

\begin{adjustwidth}{-3.5cm}{0cm}
\pagestyle{CIOpage}
\authortoc{\textsc{Bassi E.}, \textsc{Diana F.}, 
\textsc{Sartirana F.}, \textsc{Trotti P.}, 
 \textsc{Galli L.}, \textsc{Pedrotti L.}}
\chapter*[Successo riproduttivo dell'aquila reale in relazione al ritorno del gipeto]{Analisi del successo riproduttivo dell{\textquoteright}aquila
reale \textbf{\textit{Aquila chrysaetos}}\textbf{ nel Parco Nazionale
dello Stelvio in relazione al ritorno del gipeto
}\textbf{\textit{Gypaetus barbatus}}\textbf{ sulle Alpi}}
\addcontentsline{toc}{chapter}{Successo riproduttivo dell'aquila reale in relazione al ritorno del gipeto}

\end{adjustwidth}
\begin{adjustwidth}{-3.5cm}{-1cm}
\textsc{Enrico Bassi}$^{1*}$, \textsc{Francesca Diana}$^{1}$, 
\textsc{Fabiano Sartirana}$^{1}$, \textsc{Paolo Trotti}$^{1}$, 
 \textsc{Loris Galli}$^{2}$, \textsc{Luca Pedrotti}$^{1}$\\
 
 \index{Bassi Enrico} \index{Diana Francesca} \index{Sartirana Fabiano} \index{Trotti Paolo} \index{Galli Loris} \index{Pedrotti Luca}
\noindent\color{MUSEBLUE}\rule{27cm}{2pt}
\vspace{1cm}
\end{adjustwidth}



\marginnote{\raggedright $^1$Consorzio del Parco Nazionale dello Stelvio Via De
Simoni 42, 23032 Bormio (SO) \\
$^2$Universit\`a degli Studi di Genova Corso Europa 26,
16132 Genova \\
\vspace{.5cm}
{\emph{\small $^*$Autore per la corrispondenza: \href{mailto:enrico.bassi76@gmail.com}{en\allowbreak ri\allowbreak co.\allowbreak bas\allowbreak si76@\allowbreak g\allowbreak ma\allowbreak il.\allowbreak com}}} \\
\keywords{\textit{Gypaetus barbatus}, \textit{Aquila chrysaetos},
aquila reale, influenza interspecifica, \textit{no fly-zones}, Alpi
centrali}
{\textit{Gypaetus barbatus}, \textit{Aquila
chrysaetos}, interspecific influence, no fly-zones, central
Alps}
%\index{keywords}{\textit{Gypaetus barbatus}} \index{keywords}{\textit{Aquila chrysaetos}}
% \index{keywords}{Influenza interspecifica} \index{keywords}{\textit{No fly-zones}} \index{keywords}{Alpi centrali}
}
{\small
\noindent \textsc{\color{MUSEBLUE} Summary} / After the bearded vulture \textit{Gypaetus barbatus} extinction and the
subsequent reintroduction over the Alps, the first italian reproductive
nucleus settled in the Stelvio National Park (northern Italy). In the
period 1998-2013 it was characterized by high productivity rate (0.77
fledged juveniles/controlled pairs). In this area a density of 15.75
golden eagle \textit{Aquila chrysaetos} pairs/1000
km\textsuperscript{2} has been recorded, showing a productivity of 0.39
fledged juveniles/controlled pairs from 2005 to 2013. Furthermore in
this period many interspecific observations of aggressive interactions
have been recorded; starting from these data a research started to
assess whether the return of the vulture may have affected the
productivity of the golden eagle. In order to test this influence we
studied the breeding populations of golden eagle and bearded vulture
through focal sampling at nests from 2008 to 2011.

Statistical analysis showed that the productivity of the golden eagle
was positively correlated with the distance of the conspecific and
bearded vulture nearest pairs and negatively correlated with the height
of the snow in winter. The number of golden eagle pairs that laid eggs
was significantly higher in sites distant more than 5 km from bearded
vulture nests. Regarding the interactions observed, the golden eagle
was the most aggressive species. The 71.2\% of these interactions were
observed in the most recent territories defended by immature and
subadult golden eagle and especially within the areas called
{\textquoteleft}no fly zones{\textquoteright}, characterized by the
presence of recent and old nests.   \\
\noindent \textsc{\color{MUSEBLUE} Riassunto} / Dopo l{\textquoteright}estinzione del gipeto \textit{Gypaetus barbatus}
e la sua successiva reintroduzione sull{\textquoteright}arco alpino,
nel Parco Nazionale dello Stelvio (Alpi centrali) si \`e insediato il
primo nucleo riproduttivo italiano caratterizzato, nel periodo
1998-2013, da un{\textquoteright}elevata produttivit\`a pari a 0,77
giovani involati/coppie controllate. Tale area ospita anche una
popolazione di aquila reale \textit{Aquila chrysaetos} distribuita con
densit\`a pari a 15,75 coppie/1000 km\textsuperscript{2 }che, nel
periodo 2005-2013, ha mostrato una produttivit\`a di 0,39
giovani/coppie controllate. 
Inoltre in questo periodo sono state osservate tra le due specie
numerose interazioni aggressive; a partire da questi dati \`e stato
avviato uno studio a pi\`u livelli per valutare se il ritorno del
gipeto possa aver influenzato la produttivit\`a
dell{\textquoteright}aquila reale. Per testare questa influenza le
coppie nidificanti di entrambe le specie sono state oggetto dal 2008 al
2011 di studio intensivo tramite \textit{focal sampling} al nido.
Le analisi hanno mostrato che la produttivit\`a
dell{\textquoteright}aquila reale \`e correlata positivamente con la
distanza della coppia conspecifica e di gipeto pi\`u vicine e
negativamente con l{\textquoteright}altezza della neve in inverno.
Inoltre le coppie di aquila reale pi\`u lontane dai siti di gipeto
(5 km) sono quelle che depongono in maniera
significativamente maggiore rispetto ai siti vicini. Per quanto
riguarda le interazioni osservate, \`e l{\textquoteright}aquila reale
che ha mostrato una maggiore aggressivit\`a.  Il 71,2\% di tali
aggressioni \`e stato osservato nei territori di aquila reale di
neoformazione difesi da soggetti di et\`a immatura e subadulta e
soprattutto all{\textquoteright}interno delle aree di rispetto
denominate {\textquoteleft}\textit{no fly zones}{\textquoteright}
caratterizzate dalla presenza dei nidi (storici e attivi).
}



\section*{Introduzione}

Dopo l{\textquoteright}estinzione del gipeto sulle Alpi ai primi del
Novecento, nel 1986 sono stati avviati i primi rilasci
nell{\textquoteright}ambito del progetto internazionale di
reintroduzione per ricostituire metapopolazioni vitali in grado di auto
mantenersi. Dal 1998, nell{\textquoteright}intorno del Parco Nazionale
dello Stelvio (PNS) si \`e insediato il primo nucleo riproduttivo
italiano di 4 coppie (Bassi 2010) in simpatria con una consistente
popolazione di aquila reale (15,75 coppie/1000 km$^2$. Le
coppie di gipeto sono tutte incluse nell{\textquoteright}intorno del
settore lombardo del PNS (Sondrio e Brescia), a eccezione di alcuni
nidi ricadenti in Svizzera in corrispondenza del confine italiano. Nel
periodo 1998-2013 questo nucleo ha mostrato la maggiore produttivit\`a
a livello alpino, con un valore medio pari a 0,77 giovani involati per
coppia controllata (N= 54). Pi\`u modesta \`e stata invece la
produttivit\`a dell{\textquoteright}aquila reale, per il periodo
2005-2013 (pari a 0,39 giovani su 129 nidificazioni controllate). Le
date medie di deposizione e involo sono state rispettivamente il 17
gennaio e il 15 luglio nel gipeto e il 27 marzo e il 25 luglio
nell{\textquoteright}aquila reale.

Nel Parco Nazionale dello Stelvio sono stati individuati 
14 nidi di gipeto (quota media 2225.4 m
s.l.m., range 2028-2440 m; SD 102.5 m) e 124 nidi di aquila reale
(quota media 2035 m s.l.m., range 1317-2496 m; DS 230.5 m).

A partire da considerazioni relative al diverso successo riproduttivo
delle due specie negli ultimi anni, dalle frequenti osservazioni
occasionali di interazioni interspecifiche per lo pi\`u aggressive
(anche letali), dall{\textquoteright}anticipato ciclo riproduttivo del
gipeto rispetto all{\textquoteright}aquila reale e
dall{\textquoteright}evidenza che il 35,7\% dei 14 nidi usati dal
gipeto sono stati usurpati all{\textquoteright}aquila reale, si \`e
intrapreso uno studio a pi\`u livelli per valutare se il progressivo
ritorno del gipeto possa aver influenzato la produttivit\`a
dell{\textquoteright}aquila reale. 

Sono state pertanto formulate le seguenti ipotesi: 1) il gipeto,
dall{\textquoteright}epoca del suo insediamento, ha influenzato
negativamente la produttivit\`a dell{\textquoteright}aquila reale sul
breve periodo; 2) la quantit\`a di tempo speso
dall{\textquoteright}aquila reale nella difesa territoriale (in genere
rivolta verso conspecifici) pu\`o aumentare in maniera significativa
nei siti di compresenza del gipeto, tanto da indurla a
un{\textquoteright}ulteriore riduzione del tempo investito nelle cure
parentali e a un conseguente calo della sua produttivit\`a; 3) assunto
che i soggetti territoriali di entrambe le specie sono sedentari,
caratterizzati da una lunga aspettativa di vita (il monitoraggio
genetico ha infatti evidenziato la presenza di gipeti adulti tra i 10 e
i 24 anni di vita e aquile adulte di oltre 10 anni) e hanno una
conoscenza consolidata circa la localizzazione dei nidi delle coppie
vicine, esistono delle corrispondenti aree di rispetto
({\textquotedblleft}\textit{no-fly zones{\textquotedblright}}, al cui
interno gli individui territoriali non tollerano potenziali azioni e
incursioni quali soste prolungate, sorvoli, ecc.), identificabili coi
settori che includono i nidi, ben note alle coppie territoriali
confinanti, ma non ai soggetti erratici e alle coppie di pi\`u recente
insediamento.

\section*{Metodi}

Per testare la prima ipotesi, il successo riproduttivo
dell{\textquoteright}aquila reale (la variabile dicotomica
{\textquotedblleft}produttivit\`a Aquila 2004-10{\textquotedblright})
registrato per 14 coppie negli anni 2004-2010, coincidente col periodo
di insediamento di due nuove coppie di gipeto e di tre nuove coppie di
aquila, \`e stato analizzato mediante regressione logistica in
relazione a un set di variabili indipendenti: inclusione del territorio
nell{\textquoteright}area protetta, et\`a dei soggetti territoriali,
esperienza delle coppie, distanza dalla coppia di aquila pi\`u vicina,
distanza dalla coppia riproduttiva di aquila pi\`u vicina, altezza
invernale del manto nevoso (\textit{proxy} della mortalit\`a invernale
di ungulati selvatici e conseguente disponibilit\`a di carcasse),
distanza dalla coppia di gipeto pi\`u vicina e indice medio annuale di
densit\`a di aquile \textit{floaters} presenti nel mese di marzo.
Questo indice \`e stato desunto dai risultati dei conteggi simultanei
effettuati su un{\textquoteright}area di ampiezza media di 1055
km\textsuperscript{2}, tra il 2004 e il 2013, per  quantificare il
numero di adulti e \textit{floaters} nell{\textquoteright}arco di una
finestra spazio-temporale definita (Bassi 2014). La significativit\`a
delle variabili indipendenti \`e stata preliminarmente verificata
mediante analisi esplorative univariate. Successivamente \`e stato
applicato un modello di regressione logistica multivariata che
comprendeva, quali variabili indipendenti, quelle risultate
significative a livello univariato. La selezione del modello migliore
\`e stata effettuata tenendo conto delle differenze nei valori di
massima verosimiglianza dei modelli stessi (Hosmer \& Lemeshow 2000).

Per testare la seconda ipotesi invece \`e stato applicato il metodo
delle osservazioni mediante \textit{focal sampling} dei nidi occupati,
in cui sono stati calcolati i tempi dedicati alla cova e alle cure
parentali da parte degli adulti impegnati nella nidificazione
nell{\textquoteright}arco di 289 giornate di osservazione (media
giornaliera 7,8 h) distribuite su 4 anni (2008-11). I dati raccolti
sono stati confrontati tra coppie nidificanti vicine ({\textless} 5 km)
e lontane ({\textgreater} 5 km) da territori stabili di gipeto per
valutare se la vicinanza dell{\textquoteright}avvoltoio determinasse un
effettivo disturbo. Inoltre, nel corso del \textit{focal sampling,}
sono state registrate tutte le attivit\`a svolte al di fuori del nido
ed \`e stata calcolata la frequenza delle interazioni agonistiche
intra/interspecifiche. 

Per testare la terza ipotesi, lo spazio visibile dal punto di
osservazione \`e stato diviso in superfici unitarie dette
\textit{patches} (media 2.7 km\textsuperscript{2}) adattando il metodo
utilizzato da Haller (1996) per lo studio del comportamento spazio
temporale di coppie territoriali di aquila reale in Svizzera. 

Questa suddivisione \`e risultata funzionale
all{\textquoteright}assegnazione spazio temporale delle attivit\`a
delle coppie impegnate in azioni territoriali, di volo e di
sorveglianza al di fuori del nido. 

Ai \textit{patches} che includevano il nido attivo \`e stato assegnato
il codice 1, il codice 3 ai settori includenti altri nidi difesi dalla
coppia territoriale ma non occupati nell{\textquoteright}anno di
indagine e codice 5 a tutti gli altri settori in cui non erano presenti
nidi noti.

Complessivamente sono stati analizzati i dati provenienti dai
\textit{focal sampling} effettuati su 16 eventi di nidificazione di
aquila reale e 11 di gipeto. 

\section*{Risultati e discussione}

Una prima analisi esplorativa univariata indica come la variabile
dipendente {\textquotedblleft}produttivit\`a Aquila
2004-10{\textquotedblright} sia risultata significativamente correlata
a un set di variabili indipendenti quali la distanza dalla coppia di
aquila pi\`u vicina, {\textquotedblleft}Aquila NND{\textquotedblright}
(p{\textless}0,01), la media annuale dell{\textquoteright}altezza
nevosa in inverno, {\textquotedblleft}AN{\textquotedblright}
(p{\textless}0,01), la distanza dalla coppia di gipeto pi\`u vicina,
{\textquotedblleft}Gypaetus NND{\textquotedblright} (p{\textless}0,05),
e l{\textquoteright}indice di presenza invernale dei \textit{floaters}
di aquila reale censiti nel corso dei conteggi contemporanei effettuati
nel mese di marzo per gli anni 2004-13,
{\textquotedblleft}FLOAQ{\textquotedblright} (p{\textless}0,01). Il
modello migliore selezionato in base all{\textquoteright}analisi di
regressione logistica \`e risultato costituito dalle seguenti
variabili: Aquila NND (distanza dalla coppia di aquila pi\`u vicina) +
AN (altezza media del manto nevoso) + Gypaetus NND (distanza dalla
coppia di gipeti pi\`u vicina) + l{\textquoteright}interazione tra le
ultime due (Tab. \ref{Bassi_tab_1}), e classifica correttamente il 73\% dei casi. 

La produttivit\`a dell{\textquoteright}aquila reale nel periodo 2004-10
\`e risultata correlata negativamente con l{\textquoteright}altezza
della neve in inverno. Una maggiore nevosit\`a invernale causa tassi di
mortalit\`a pi\`u elevati negli ungulati in ambito alpino e quindi una
maggiore disponibilit\`a trofica di carcasse nella seconda parte della
stagione invernale stessa. Tuttavia l{\textquoteright}aumentata
disponibilit\`a alimentare pu\`o attrarre i \textit{floaters} che
costringono i soggetti riproduttivi a intensificare
l{\textquoteright}attivit\`a di difesa territoriale nel periodo di
inizio nidificazione (marzo), con esiti potenzialmente negativi sulla
produttivit\`a, poich\'e \`e ipotizzabile che venga modificato il tempo
dedicato alla cova e alle cure parentali (Jenny 1992). 

Il successo riproduttivo \`e invece positivamente correlato
all{\textquoteright}aumento della distanza dalla coppia territoriale
pi\`u vicina sia di aquila reale, sia di gipeto, lasciando presupporre
che elevate densit\`a di entrambe le specie possano determinare un
effetto congiunto di regolazione sulla dinamica della popolazione di
aquila reale, vicina al raggiungimento della capacit\`a portante (Tab.
\ref{Bassi_tab_2}). \`E possibile quindi affermare che negli anni di neo insediamento
di due nuove coppie di gipeto (2004-10), anche la distanza dalla coppia
territoriale di gipeto pi\`u vicina abbia influenzato la produttivit\`a
dell{\textquoteright}aquila reale. Pertanto si \`e ipotizzato che le
coppie di aquila nidificanti presso i territori di gipeto, oltre a
dover competere con i conspecifici, siano pi\`u frequentemente
sollecitate ad azioni di difesa contro individui di gipeto, rispetto ad
altre coppie di aquila poste a maggior distanza dal gipeto stesso, e
dunque siano sottoposte a un maggior grado di stress, a un aumentato
rischio di non deposizione e a una riduzione dei tempi di cova e di
cure parentali, in analogia con quanto dimostrato da Jenny (1992).

L{\textquotesingle}interazione Gypaetus NND * AN migliora in modo
significativo il modello e mette in evidenza che
l{\textquotesingle}effetto del gipeto sulla probabilit\`a di
riproduzione dell{\textquotesingle}aquila reale non \`e costante ma
dipende dalle condizioni dell{\textquotesingle}inverno. In inverni con
nevosit\`a al di sotto della media, la vicinanza del gipeto aumenta in
modo significativo la probabilit\`a di fallimento
dell{\textquotesingle}aquila reale; questo significa che le coppie di
aquila che si riproducono hanno una distanza media pi\`u alta dalla
coppia di gipeto pi\`u vicina. Si ipotizza che in inverni poco nevosi
la diminuzione di carcasse di ungulati sul territorio determini una
maggiore competizione trofica tra le due specie che probabilmente
tendono a interagire per lo sfruttamento delle medesime fonti
alimentari, numericamente pi\`u scarse e localizzate
nell{\textquoteright}area difesa da ciascuna coppia territoriale.

Al contrario, negli inverni con nevosit\`a al di sopra della media,
l{\textquotesingle}effetto
{\textquotedblleft}gipeto{\textquotedblright} non \`e visibile poich\'e
la distanza media dalla coppia di gipeto pi\`u vicina \`e analoga nelle
coppie che si riproducono e in quelle che falliscono. Queste condizioni
di nevosit\`a, garantendo una maggior disponibilit\`a di carcasse
distribuite su spazi pi\`u ampi, potrebbero diminuire sia il grado di
interazioni aggressive tra le due specie sia la loro parziale
competizione trofica. Inoltre, in inverni particolarmente nevosi,
l{\textquoteright}effetto di disturbo da parte di aquile
\textit{floaters }\`e verosimilmente maggiore rispetto al disturbo
operato dai pochi adulti territoriali di gipeto presenti
nell{\textquoteright}area.

Le osservazioni mediante \textit{focal sampling} sui nidi occupati non
hanno per\`o evidenziato alcuna differenza significativa della
percentuale di tempo investito nelle cure parentali tra coppie di
successo che nidificano vicine e lontane da nidi di gipeto, mentre \`e
emerso che il numero di coppie di aquila che intraprende la deposizione
\`e significativamente maggiore nei siti oltre 5 km da nidi di gipeto
(Chi-quadro= 5,57, 1 gl, p= 0,018), lasciando presupporre che la
presenza di coppie vicine di gipeto aumenti la frequenza delle coppie
di aquila reale che non iniziano la cova.  

Dall{\textquoteright}analisi di 132 interazioni, 91 sono avvenute tra
aquile e gipeti territoriali. L{\textquoteright}aquila reale ha
mostrato un{\textquoteright}aggressivit\`a significativamente maggiore
rispetto al gipeto attaccandolo nel 74\% dei casi (N= 91; Chi quadro=
20.3, p{\textless}0.01). Assumendo che le coppie di vecchio
insediamento (sedentarie e caratterizzate da una spiccata longevit\`a)
abbiano acquisito una conoscenza consolidata delle abitudini delle
coppie confinanti e delle loro {\textquotedblleft}\textit{no-fly
zones}{\textquotedblright} e, quindi, ne rispettino i confini, \`e
atteso che, laddove vi siano coppie di neoformazione e che nidificano
tra loro vicine, la frequenza delle interazioni aggressive sia
maggiore. Il 71,2\% delle aggressioni totali (inter e intra specifiche)
\`e stato infatti registrato in 2 soli territori in cui negli anni di
\textit{focal sampling} si sono insediate tre coppie di aquila reale di
neoformazione che hanno condotto la maggior parte degli attacchi.
Dall{\textquoteright}analisi spazio temporale la percentuale delle
interazioni aggressive tra le due specie \`e stata significativamente
maggiore (67\%) all{\textquoteright}interno dei \textit{patches} 1 e 3
e minore (33\%) nei \textit{patches} 5 che sono probabilmente meno
strategici in periodo riproduttivo (Chi quadro 12.03; 1 gl;
p{\textless}0.01). I \textit{patches} 1 e 3, aree assunte come
{\textquotedblleft}\textit{no-fly zones{\textquotedblright}},
identificano spazialmente quei settori che includono i nidi e in cui
gli individui territoriali non tollerano potenziali azioni e incursioni
(ad es. soste prolungate e sorvoli) da parte di intrusi e per questo
vengono suscitati a compiere azioni di attacco.
L{\textquoteright}ipotesi che le coppie neo insediate abbiano minor
conoscenza delle abitudini dell{\textquoteright}altra specie o ne
invadano intenzionalmente le aree di rispetto sembra confermata. 

Le interazioni avvengono con frequenza significativamente maggiore nei
\textit{patches} che includono i nidi (siti ad alto valore biologico
potenzialmente usurpabili e strategicamente importanti per il successo
riproduttivo della coppia nel medio periodo) ma che, per le voluminose
dimensioni che ne facilitano l{\textquoteright}individuazione da parte
dei rapaci, potrebbero svolgere anche un{\textquoteright}importante
funzione territoriale. In alcuni grandi rapaci, infatti, le specie
pi\`u forti, occupando molti dei nidi presenti in
un{\textquoteright}area, riducono la densit\`a riproduttiva della
specie pi\`u debole oppure la relegano verso nidi di minore qualit\`a
(White \& Cade 1971; Newton 1979). In quest{\textquoteright}ottica
l{\textquoteright}elevata frequenza osservata di interazioni aggressive
tra coppie di aquila di neoinsediamento e di gipeti territoriali pu\`o
spiegare, almeno in parte, il calo del successo riproduttivo
dell{\textquoteright}aquila. Quest{\textquoteright}ultima potrebbe
essere, infatti, sottoposta a un maggiore stress derivante dal rischio
di possibili interazioni/usurpazioni dei nidi da parte del gipeto che,
nidificando pi\`u precocemente rispetto all{\textquoteright}aquila,
potrebbe risultare favorito nel periodo di selezione dei siti e delle
pareti di nidificazione. Questa forma di competizione interspecifica
sembra attenuarsi fino a scomparire nelle coppie di aquila e gipeto che
nidificano da pi\`u anni a stretto contatto e che quindi avrebbero
verosimilmente imparato a conoscere e a rispettare i confini
territoriali delle coppie confinanti, riducendo in tal modo il numero
di interazioni aggressive.

\begin{table}[!h]
\centering
\begin{tabular}{>{\raggedright\arraybackslash}p{.05\columnwidth}>{\raggedright\arraybackslash}p{.45\columnwidth}>{\raggedright\arraybackslash}p{.1\columnwidth}>{\raggedright\arraybackslash}p{.1\columnwidth}>{\raggedright\arraybackslash}p{.1\columnwidth}}
\toprule
\textbf{N} & \textbf{Modello} & \textbf{LRT} & \textbf{$\Delta$LRT} & \textbf{P} \\
\toprule
%\showrowcolors
1 & \textit{Aquila} NND & 99.52 & & \\
2 & \textit{Aquila} NND + \textit{Aquila} NND repr & 98.68 & 0.84 & 0.36 \\
3 & \textit{Aquila} NND + FLOAQ & 98.29 & 1.23 & 0.27 \\
4 & \textit{Aquila} NND + AN & 90.94 & 8.58 & 0.001 \\
5 & \textit{Aquila} NND + AN + \textit{Gypaetus} NND & 86.86 & 4.08 & 0.04 \\
6 & \textit{Aquila} NND + AN + \textit{Gypaetus} NND + AN* \textit{Gypaetus} NND & 82.50 & 4.36 & 0.04 \\
\bottomrule
\hiderowcolors
\end{tabular}
\caption{Selezione del modello migliore che mette in relazione la produttivit\`a della popolazione di aquila reale del settore lombardo del Parco Nazionale dello Stelvio, per il periodo 2004-2010, con le seguenti variabili indipendenti: \textit{Aquila} NND, distanza dalla coppia di aquila reale pi\`u vicina, espressa mediante i valori di \textit{nearest neighbour distance}; \textit{Aquila} NND repr, distanza dalla coppia di aquila reale pi\`u vicina che si \'e riprodotta nello stesso anno; FLOAQ, densit\`a di \textit{floaters} di aquila reale rilevati durante il censimento contemporaneo effettuato in periodo tardo invernale; AN, altezza media del manto nevoso invernale; \textit{Gypaetus} NND, distanza dalla coppia di gipeto pi\`u vicina; LRT valore del \textit{likelihood ratio test}; $\Delta$LRT differenza tra il valore del LRT del modello testato rispetto al precedente}
\label{Bassi_tab_1}
\end{table}

\begin{table}[!h]
\centering
\begin{tabular}{>{\raggedright\arraybackslash}p{.05\columnwidth}>{\raggedright\arraybackslash}p{.25\columnwidth}>{\raggedright\arraybackslash}p{.12\columnwidth}>{\raggedright\arraybackslash}p{.12\columnwidth}>{\raggedright\arraybackslash}p{.12\columnwidth}>{\raggedright\arraybackslash}p{.12\columnwidth}}
\toprule
\textbf{N} & \textbf{Variabile} & \textbf{B} & \textbf{ES} & \textbf{P} & $\mathbf{e^{B}}$ \\
\toprule
%\showrowcolors
1 & \textit{Aquila} NND (+) & 0.359 & 0.154 & 0.019 & 1.42 \\
2 & \textit{Gypaetus} NND (+) & 0.217 & 0.090 & 0.016 & 1.24 \\
3 & AN (-) & -0.001 & 0.013 & 0.962 & 0.99 \\
4 & \textit{Gypaetus} NND * AN & -0.002 & 0.001 & 0.056 & 0.99 \\
5 & Costante & -2.275 & 1.425 & 0.110 & 0.10 \\
\bottomrule
\hiderowcolors
\end{tabular}
\caption{Modello di regressione logistica che mette in relazione la produttivit\`a della popolazione di aquila reale del settore lombardo del Parco Nazionale dello Stelvio, per il periodo 2004-2010, con la distanza tra coppie di aquila (“\textit{Aquila} NND”, espressa mediante i valori del \textit{nearest neighbour distance}) e con la distanza dalla coppia pi\`u vicina di gipeto, (\textit{Gypaetus} NND” – entrambi i fattori hanno una correlazione positiva con la produttivit\`a) e con l’altezza media del manto nevoso invernale, “AN”, che mostra al contrario una correlazione negativa}
\label{Bassi_tab_2}
\end{table}

\section*{Ringraziamenti}

Per suggerimenti nell{\textquoteright}impostazione della ricerca:
Heinrich Haller, Giuseppe Bogliani e David Jenny.
L{\textquoteright}Ufficio Fauna del PN Stelvio, gli oltre 200
volontari, gli Agenti forestali del CTA di Bormio, gli Agenti forestali
trentini dell{\textquoteright}Uff. distrettuale di Mal\'e e le Guardie
della Polizia Provinciale di Sondrio, Lecco e Brescia
impegnati nei censimenti contemporanei. Maurizio Bagnasco ed Enrico
Pregliasco (Tersia - Savona) per il finanziamento di 3 borse di 
studio.

\section*{Bibliografia}
\begin{itemize}\itemsep0pt
	\item Bassi E., 2010 -\textbf{ }Il Gipeto \textit{Gypaetus barbatus} sulle
Alpi: aggiornamento dei risultati del progetto internazionale di
reintroduzione. \textit{Ficedula} N.~44: 14-18; 

	\item Bassi E., 2014 - Sintesi dei risultati del {\textquotedblleft}XX
Censimento contemporaneo di Aquila reale e Gipeto nel Parco Nazionale
dello Stelvio e in aree limitrofe{\textquotedblright}, Parco Nazionale
dello Stelvio, Report interno. In collaborazione con: Bragalanti N.,
Buffa A. and Trotti P., 28 pp.

Haller H., 1996 \textsc{{}- }Der steinadler in der Graubunden.
Langfristige Unteruchungen zur Populationsokologie von \textit{Aquila
chrysaetos }im Zentrum der Alpen. \textit{Der Ornithologische
Beobachter, }9: 1-167. 

	\item Hosmer D.W. \& S. Lemeshow, 2000 - \textit{Applied logistic regression}
-- second edition. Wiley-Interscience, 373 pp. 

	\item Jenny D.,\textsc{ 1992 - }Bruterfolg und Bestandsregulation einer
alpinen Population des Steinadlers \textit{Aquila chrysaetos.}
\textit{Der Ornithologische Beobachter}\textsc{, 89: 1-43. }

	\item Newton I., 1979 \textit{{}- }\textit{Population Ecology of Raptors}.
Berkhamsted, UK. T \& AD Poyser.

	\item White C.M. \& Cade T.J., 1971 - Cliff-nesting raptors and Ravens along
the Colville River in arctic Alaska. \textit{Living Bird} 10: 107-50.
\end{itemize}
