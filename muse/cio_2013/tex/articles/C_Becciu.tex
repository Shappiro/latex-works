\setcounter{figure}{0}
\setcounter{table}{0}

\begin{adjustwidth}{-3.5cm}{0cm}
\pagestyle{CIOpage}
\authortoc{\textsc{Becciu P.}, \textsc{Stanzione V.}, 
\textsc{Massa B.}, \textsc{Dell'Omo G.}}
\chapter*[Riconoscimento parentale nella berta maggiore]{Riconoscimento parentale nella berta maggiore
\textbf{\textit{Calonectris diomedea}}\textbf{: un test con
l{\textquoteright}adozione incrociata dei pulcini}}
\addcontentsline{toc}{chapter}{Riconoscimento parentale nella berta maggiore}

\textsc{Paolo Becciu}$^{1*}$, \textsc{Viviana Stanzione}$^{1}$, 
\textsc{Bruno Massa}$^{2}$, \textsc{Giacomo Dell'Omo}$^{1}$ \\

\index{Becciu Paolo} \index{Stanzione Viviana} \index{Massa Bruno} \index{Dell'Omo Giacomo}
\noindent\color{MUSEBLUE}\rule{27cm}{2pt}
\vspace{1cm}
\end{adjustwidth}



\marginnote{\raggedright $^1$Ornis italica, Piazza Crati 15, 00199, Roma, IT \\
$^2$Dipartimento di Scienze Agrarie e Forestali,
Universit\`a di Palermo, Viale delle Scienze 13, 90128 Palermo, IT \\
\vspace{.5cm}
{\emph{\small $^*$Autore per la corrispondenza: \href{mailto:pablob989@gmail.com}{pablob989@gmail.com}}} \\
\keywords{\textit{Calonectris diomedea}, Linosa, adozione
incrociata, riconoscimento parentale}
{\textit{Calonectris diomedea}, Linosa island, cross
fostering, chicks recognition}
%\index{keywords}{\textit{Calonectris diomedea}} \index{keywords}{Linosa} \index{keywords}{Adozione incrociata} \index{keywords}{Riconoscimento parentale}
}
{\small
\noindent \textsc{\color{MUSEBLUE} Summary} / Scopoli{\textquoteright}s shearwater \textit{Calonectris diomedea} is a
pelagic seabird that breeds on small Mediterranean islands. During the
breeding season, adults return at night to the colony to feed their
chick while feeding themselves during the day. Most Procellariiformes
species, as shearwaters, feed their chick with highly nutrient stomach
oil, allowing chick{\textquoteright}s survival for several days and
toleration of irregular feeding attendance. It has been shown, by
T-maze tests, that parents are able to recognize their own chick by
smell. But how will they behave in the presence of a foreign chick? 
They might respond by feeding the adoptive chick as their own, or on
the contrary provide insufficient or no food to it. To test these
possibilities we cross-fostered 14 chicks and we measured their weight
and bill length once every three days for a nine-days period. Another
14 chicks, remained in their nests, were measured with the same
schedule to serve as control group. There was no significant difference
between the control and the experimental group, as both showed a normal
weight increase and similar bill growth. Hence, we showed that the
exchanged chicks received the same care as control chicks. Obviously,
these results do not prove nor exclude the lack of individual
recognition, but confirm that adult birds adopt any chick in their
nest. Further studies are required to test the importance of the
olfactory stimuli and other cues on chicks recognition.  \\
\noindent \textsc{\color{MUSEBLUE} Riassunto} / La berta maggiore \textit{Calonectris diomedea} \`e un uccello marino pelagico che nidifica in piccole isole del Mediterraneo. Durante il
periodo riproduttivo torna nella colonia durante la notte e resta in
mare aperto durante il giorno per nutrirsi. Le berte alimentano il
proprio pulcino con un olio prodotto nello stomaco, che permette al
pulcino di tollerare una frequenza irregolare di nutrizione. Il
genitore sembra riconoscere il proprio piccolo
dall{\textquoteright}olfatto e riesce a distinguerlo dai pulcini dei
nidi vicini. Questa informazione, ottenuta in passato con dei test di
scelta, non \`e stata verificata sul campo con la sostituzione dei
pulcini. Cosa succederebbe se un genitore trovasse nel nido un pulcino
diverso dal proprio? Le ipotesi in gioco sono che potrebbe alimentarlo
meno o allo stesso modo del proprio.
Per verificare queste due ipotesi abbiamo scambiato di nido 14 pulcini
provenienti da 7 nidi e monitorato ogni tre giorni per nove giorni
l{\textquoteright}andamento del peso corporeo.  Il gruppo di controllo
era costituito da 13 pulcini rimasti nei rispettivi nidi. Non sono
emerse differenze tra i pulcini di controllo e quelli scambiati che
hanno mostrato un normale incremento ponderale. Ci\`o ovviamente non
dimostra n\'e esclude la mancanza di un riconoscimento individuale, ma
suggerisce che stimoli di altro tipo garantiscono che
l{\textquoteright}adulto alimenti il pulcino nel nido a cui fa ritorno.
In test futuri si potranno controllare stimoli olfattivi. \\
}


\section*{Introduzione}


La berta maggiore mediterranea \textit{Calonectris diomedea} \`e un
Procellariforme coloniale caratterizzato da un elevato investimento
parentale, da un lungo periodo di incubazione (54 giorni) e allevamento
dei piccoli (90 giorni). Questa specie depone un solo uovo ed entrambi
i genitori nutrono il pulcino e partecipano al suo sviluppo fino
all{\textquoteright}involo. Con la crescita del pulcino diminuisce la
frequenza di imbeccata (Warham 1990) da parte dei genitori. I giovani
pulcini vengono nutriti con un particolare olio altamente energetico
secreto dallo stomaco dei genitori, che consente loro di sopportare
alcuni giorni di digiuno, cos\`i che il peso pu\`o subire variazioni a
causa della frequenza di nutrizione irregolare, mentre la crescita
corporea continua regolarmente.

Gli esemplari adulti riescono a trovare il proprio nido nel buio della
notte con l{\textquoteright}aiuto dell{\textquoteright}olfatto e della
vista, al loro arrivo nutrono il pulcino che li incita con il suo
pigolio insistente (\textit{begging}).

In questo studio, attraverso un esperimento di scambi incrociati tra
pulcini, abbiamo cercato di verificare l{\textquoteright}esistenza di
un riconoscimento genitore-figlio e la disponibilit\`a
all{\textquoteright}adozione di un pulcino non proprio. La tecnica
usata pone i genitori adottivi di fronte ad una serie di possibilit\`a
circa il nutrimento del pulcino adottivo: 1) nutrire il pulcino
estraneo come se fosse il proprio; 2) nutrirlo in modo anomalo rispetto
al comportamento standard; 3) non nutrirlo affatto. Da alcuni
esperimenti effettuati su altri Procellariformi nel passato con
tecniche di scelta, come il \textit{T-maze test}, \`e risultato
evidente che i genitori riconoscono i piccoli e il proprio nido
(Minguez 1997; Bonadonna \textit{et al.} 2004). Queste evidenze
potrebbero supportare la possibilit\`a del rifiuto del pulcino da parte
del genitore adottivo, o almeno una modalit\`a di nutrizione diversa a
causa del mancato riconoscimento del pulcino come proprio. 

\section*{Area di studio}

Il lavoro sul campo \`e stato svolto tra la fine di luglio e
l{\textquoteright}inizio di agosto del 2012 nell{\textquoteright}isola
di Linosa (Agrigento, isole Pelagie), durante il periodo di sviluppo
dei pulcini. Questa isola ospita la pi\`u grande colonia italiana di
berta maggiore mediterranea, stimata intorno alle 10.000 coppie (Massa
\& Lo Valvo 1986). 

\section*{Metodi}

I pulcini sperimentali (S; n=14) e quelli di controllo (C; n=14) sono
stati scelti casualmente tra i nidi gi\`a monitorati dal gruppo di
lavoro. Per lo studio sono stati usati solo nidi isolati, escludendo
quindi quelle situazioni in cui pi\`u coppie nidificano in una
cavit\`a. I pulcini sperimentali sono stati scambiati di nido a due
settimane di vita e, successivamente, sono stati misurati peso e
lunghezza del becco ad intervalli di tre giorni per quattro volte. Al
termine dell{\textquoteright}esperimento i pulcini scambiati sono stati
riposti nel loro nido e il loro sviluppo \`e proseguito fino
all{\textquoteright}involo (osservato nell{\textquoteright}ottobre
dello stesso anno). 

Per quanto riguarda la statistica si \`e scelta
un{\textquoteright}analisi della varianza a misure ripetute (GLM)
effettuata mediante SPSS 20.0 (SPSS Inc., Chicago, Illinois, U.S.A.). 

\section*{Risultati e discussione}

Le analisi non hanno evidenziato differenze tra il gruppo sperimentale e
il gruppo di controllo sia per quanto riguarda
l{\textquoteright}andamento del peso che
l{\textquoteright}accrescimento del becco (Fig. \ref{Becciu_fig_1}).

L{\textquoteright}esperimento \`e stato condotto al fine di valutare se
i genitori di berta maggiore alimentassero in maniera regolare un
pulcino non proprio, o se riconoscendolo non proprio, riducessero
l{\textquoteright}investimento parentale. Non essendo risultata alcuna
differenza di crescita nei parametri misurati tra i pulcini adottati e
quelli appartenenti al gruppo di controllo, possiamo asserire che i
pulcini adottati siano stati nutriti regolarmente dai genitori
adottivi. L{\textquoteright}elevata sensibilit\`a olfattiva di questi
uccelli, cos\`i come quella delle altre specie di Procellariformi
(Bonadonna \textit{et al.} 2004), aiuta loro a ritrovare nelle ore
pi\`u oscure della notte il proprio nido tra i molti presenti nella
colonia. Una spiegazione ai nostri risultati potrebbe quindi essere che
l{\textquoteright}odore molto forte del nido finisca con il mascherare
quello del pulcino, cos\`i da indurre l{\textquoteright}adulto a
nutrirlo come fosse il proprio. I pulcini emettevano comunque
vocalizzazioni e queste non hanno avuto effetti sulle modalit\`a di
alimentazione da parte dei genitori. Una seconda spiegazione, che non
esclude la precedente, consiste nella mancata selezione dei meccanismi
atti a un vero e proprio riconoscimento genitore-figlio, alla luce
della particolare modalit\`a di riproduzione di questi animali che ha
basse probabilit\`a di scambi tra nidiacei; ci\`o induce il genitore a
nutrire il pulcino che trova nel nido indipendentemente dal fatto che
sia il proprio. Bisogna inoltre considerare che
l{\textquoteright}investimento parentale in questa specie \`e
estremamente elevato e che il successo riproduttivo della coppia \`e
rappresentato dalle possibilit\`a offerte all{\textquoteright}unico
pulcino. Dopo aver investito energie nella deposizione
dell{\textquoteright}uovo e nella lunghissima incubazione, i genitori
intraprendono l{\textquoteright}ulteriore lunga fase di allevamento.
E{\textquoteright} probabile quindi che l{\textquoteright}identit\`a
del pulcino sia da loro posta in secondo ordine. \`E da precisare che
questo studio \`e stato condotto nella terza e quarta settimana di vita
dei pulcini, per cui non \`e escluso che il riconoscimento
genitore-figlio possa verificarsi solo durante una fase pi\`u avanzata,
come suggerito da uno studio di Storey \textit{et al. }(1992) sul
gabbiano tridattilo \textit{Rissa tridactyla}, in cui \`e stato
dimostrato che gli adulti iniziano a riconoscere il proprio pulcino
solo circa 15 giorni prima dell{\textquoteright}involo. \`E altres\`i
ipotizzabile che lo spostamento del materiale nei nidi in cui avviene
lo scambio incrociato sia un elemento chiave per la comprensione del
fenomeno dell{\textquoteright}adozione e del riconoscimento del nido,
come osservato negli uccelli delle tempeste \textit{Hydrobates
pelagicus} da Minguez (1997).


\begin{figure}[!h]
\centering
\includegraphics[width=.6\columnwidth]{Becciu_fig_1.png}
\caption{Andamento del peso e della lunghezza del becco (Media $\pm$ DS) dei pulcini adottati e dei pulcini di controllo durante il periodo di sperimentazione}
\label{Becciu_fig_1}
\end{figure}

\section*{Ringraziamenti}

Il lavoro da campo \`e stato sostenuto da \textit{Ornis italica} e dal
Dipartimento di Scienze Agrarie e Forestali
dell{\textquoteright}Universit\`a di Palermo
nell{\textquoteright}ambito del progetto LIFE+ Nat/It 00093
{\textquotedblleft}Pelagic Birds{\textquotedblright}.

\section*{Bibliografia}
\begin{itemize}\itemsep0pt
	\item Bonadonna F., Villafane M., Bajzak C. \& Jouventin P., 2004 -
Recognition of burrow{\textquoteright}s olfactory signature in blue
petrels, \textit{Halobaena caerulea}: an efficient discrimination
mechanism in the dark. \textit{Anim. Behav}, 67: 893-898.

	\item Massa B. \& Lo Valvo M., 1986 - Biometrical and biological
considerations on the Cory{\textquoteright}s Shearwater
\textit{Calonectris diomedea}. In: Medmaravis \& Monbailliu (eds),
\textit{Mediterranean Marine Avifauna}. Springer-Verlag, Berlin:
293-313.

	\item Minguez E., 1997 - Olfactory nest recognition by British storm-petrel
chicks. \textit{Anim. Behav.}, 53: 701-707. 

	\item Storey A.E., Anderson R.E., Porter J.M. \& McCharles A.M., 1992 -
Absence of parent--young recognition in kittiwakes: a re-examination.
\textit{Behaviour}, 120: 302--323.

	\item Warham J., 1990 - The petrels: Their ecology and breeding
systems. Academic Press, London, 452 pp.
\end{itemize}
