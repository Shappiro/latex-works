\begin{otherlanguage}{english}
\setcounter{figure}{0}
\setcounter{table}{0}

\begin{adjustwidth}{-3.5cm}{0cm}
\pagestyle{CIOpage}
\authortoc{\textsc{Campioni L.}, \textsc{Del Mar Delgado M.},
\textsc{Bettega C.}, \textsc{Penteriani V.}}
\chapter*[Repeatability of movement behaviour]{Repeatability of movement parameters inside home range
boundaries in a long-lived species: the eagle owl \textbf{\textit{Bubo
bubo}}}
\addcontentsline{toc}{chapter}{Repeatability of movement behaviour}


\textsc{Letizia Campioni}$^{1*}$, \textsc{Maria Del Mar Delgado}$^{2}$,
\textsc{Chiara Bettega}$^{3}$, \textsc{Vincenzo Penteriani}$^{3,4}$ \\

\index{Campioni Letizia} \index{Del Mar Delgado Maria} \index{Bettega Chiara} \index{Penteriani Vincenzo}
\noindent\color{MUSEBLUE}\rule{27cm}{2pt}
\vspace{1cm}
\end{adjustwidth}



\marginnote{\raggedright $^1$MARE -- Marine and Environmental Sciences Centre,
ISPA - Instituto Universit\'ario, Lisbon, Portugal \\
$^2$Department of Biosciences, University of Helsinki,
Helsinki, Finland \\
$^3$Department of Conservation Biology, Estaci\'on
Biol\'ogica de Do\~nana, C.S.I.C., Seville, Spain \\
$^4$Research Unit of Biodiversity (UMIB, UO-CSIC-PA),
Oviedo University - Campus Mieres, Mieres, Spain \\
\vspace{.5cm}
{\emph{\small $^*$Autore per la corrispondenza: \href{mailto:letiziacampioni@hotmail.com}{le\allowbreak ti\allowbreak zia\allowbreak cam\allowbreak pio\allowbreak ni@\allowbreak hot\allowbreak ma\allowbreak il.\allowbreak com}}} \\
\keywords{Sierra Norte Spagna, ripetibilit\`a, comportamento di
movimento, variazione inter-individuale, \textit{Bubo bubo}}
{Sierra Norte Spain, repeatability, animal movement, individual
consistency, between-individual variation, \textit{Bubo bubo}}
%\index{keywords}{Sierra Norte} \index{keywords}{Spagna} \index{keywords}{Ripetibilit\`a} \index{keywords}{Comportamento di movimento} \index{keywords}{Variazione inter-individuale} \index{keywords}{\textit{Bubo bubo}}
}
{\small
\noindent \textsc{\color{MUSEBLUE} Summary} / Territorial species as the eagle owl \textit{Bubo bubo }that repeatedly
move within fixed home ranges are expected to have an extensive
knowledge of their surroundings. As a consequence, owls can be expected
to be highly repeatable in their movement parameters. We found that the
repeatability of speed, time step, total distance and step length of 26
breeding owls ranged between 15-25\% revealing a considerable
individual consistency.\\
}

\vspace{1cm}
Observed movement patterns are the response of the interaction between
environmental variables and individual state (B\"orger \textit{et al.}
2008). Surprisingly, even individuals of the same species experiencing
similar environmental condition can exhibit different behavioural
responses, being these responses highly repeatable within individuals
(Biro \& Adriaenssens 2013). Here, the variation we focus on is how
individuals of a long-lived, territorial species, the nocturnal eagle
owls \textit{Bubo bubo,} move during their daily activity over multiple
years. Eagle owls that repeatedly move within fixed home ranges are
expected to have an extensive knowledge of their surroundings. As a
consequence, they are expected to show to some extent a systematic
movement strategy based on available \textit{a priori} information
and/or based on an individual behavioural consistency. If it is so, we
can expect that movement patterns varied much less between repeated
daily trajectories than between different individuals. This study was
conducted in a hilly area of the Sierra Norte of Seville located in
south-western Spain. From 2004 to 2010, 26 breeding individuals (19
males and 7 females) from 19 nests were trapped (Campioni \textit{et
al}. 2013; Penteriani \textit{et al}. 2010) and fitted with a 30-g
radio-transmitter using a Teflon ribbon backpack harness (Biotrack,
UK). We radio tracked territory holders individually throughout the
night (from 1 h before sunset to 1 h after sunrise) during 290
continuous radio tracking sessions. Individual nightly movement
behaviour was characterised by four variables: 1) total distance, as
the sum of the distance between successive steps of the nightly
displacements; 2) step length, as the distance between successive
locations; 3) speed, as the step length divided by the time interval
between successive locations; and 4) time step, as the time elapsed
between successive moves. Then, movement variables were grouped and
analyzed at a daily temporal scale, where for each individual we
constructed log10-transformed movement parameter frequency
distributions (i.e., MPFD). Accordingly, we selected four statistics
able to comprehensively describe MPFDs: 1) minimum value, 2) maximum
value, 3) median, 4) geometric mean, and 5) coefficient of variation
(CV) and then we estimated their repeatability (R). The repeatability
analysis of MPFDs of breeding owls showed a considerable individual
consistency in all movement parameters ($\Delta $R: 15-25\%; Fig. \ref{Campioni_fig_1}).
Total distance was the parameter with the highest repeatability (mean
{\textpm} SE and 95\% CI; 0.29 {\textpm} 0.08; [0.127-0.44]). Moreover,
male was the sex showing higher repeatability of movement parameters (R
mean: 0.19, $\Delta $R = 0.04-0.30), though females seemed to be more
consistent than males with respect to time step parameter (R mean:
0.18, $\Delta $R = 0.01-0.36). Accordingly, 95\% CI of repeatability
estimates for most of the statistics were well above zero
(statistically significant at $\alpha $ = 0.05). These results
suggested that owls move following a consistent movement strategy,
i.e., with similar movement parameters every night while maintaining
some degree of variation across nights. Lastly, repeatability estimates
of different owls that have been owners of the same territory or mate
of the same pair were substantially smaller ($\Delta $R = 0.0-0.08).
Namely, between-territory variability was higher than within-territory
variability. Individuals behaving in similar environmental condition
seemed to show a substantial behavioural flexibility. We suggest
homogeneity of habitat in our study area and small home range size
(mean HR: {\textless} 250 ha) to be responsible for the moderate
between-individual variation in movement patterns. 

\begin{figure}[!h]
\centering
\includegraphics[width=.8\columnwidth]{Campioni_fig_1.png}
\caption{Boxplot of the four statistics used to describe the MPFD of speed movement parameter of 26 breeding owls with the relative values of repeatability}
\label{Campioni_fig_1}
\end{figure}

\section*{Acknowledgements}

The work was funded by two research projects of the Spanish Ministry of
Science and Innovation, the Ministry of Education and Science ---
C.S.I.C., the Junta of Andaluc{\i}\'a and LICOR43. During this work,
L.C. was supported by the post-doctoral grant (SFRH/\allowbreak BPD/\allowbreak 89904/\allowbreak 2012)
from FCT (Funda\c{c}ao Ciencia e Tecnologia, Portugal).

\section*{Bibliography}
\begin{itemize}\itemsep0pt
	\item Biro P.A., \& Adriaenssens B., 2013 - Predictability as a personality
trait: consistent differences in intraindividual behavioral variation.
\textit{American Naturalist.}

	\item B\"orger L., Danziel B.D. \& Fryxell J.M., 2008 - Are there general
mechanisms of animal home range behaviour? A review and prospects for
future research. \textit{Ecol Lett }11: 637--650.

	\item Campioni L., Delgado M.M., Louren\c{c}o R., Bastianelli G., Fern\'andez
N. \& Penteriani V., 2013 - Individual and spatio-temporal variations
in the home range behaviour of a long-lived, territorial species.
\textit{Oecologia} 172: 371--385.

	\item Penteriani V., Delgado M.M., Campioni L. \& Lourenco R. (2010).
Moonlight makes owls more chatty. \textit{PLoS One} 5 (1), e8696.
\end{itemize}
\end{otherlanguage}