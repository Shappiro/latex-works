\setcounter{figure}{0}
\setcounter{table}{0}

\begin{adjustwidth}{-3.5cm}{0cm}
\pagestyle{CIOpage}
\authortoc{\textsc{Verza E.}}
\chapter*[La pernice di mare e la sterna zampenere nel Delta del
Po]{Popolazione e scelta dell{\textquoteright}habitat riproduttivo
di pernice di mare \textbf{\textit{Glareola pratincola }}\textbf{e
sterna zampenere }\textbf{\textit{Gelochelidon nilotica}}\textbf{ nella
parte veneta del Delta del Po (Rovigo); analisi del periodo 2001-2012}}
\addcontentsline{toc}{chapter}{La pernice di mare e la sterna zampenere nel Delta del
Po}

\textsc{Emiliano Verza}$^{1*}$\\

\index{Verza Emiliano}
\noindent\color{MUSEBLUE}\rule{27cm}{2pt}
\vspace{1cm}
\end{adjustwidth}


\marginnote{\raggedright $^1$Ass. Sagittaria, via Badaloni 9, 45100 Rovigo \\
\vspace{.5cm}
{\emph{\small $^*$Autore per la corrispondenza: \href{mailto:sagittaria.at@libero.it}{sagittaria.at@libero.it}}} \\
\keywords{Delta del Po, \textit{Glareola pratincola},
\textit{Gelochelidon nilotica}, nidificazione}
{Po Delta, \textit{Glareola pratincola}, \textit{Gelochelidon nilotic}, nesting}
%\index{keywords}{Delta del Po} \index{keywords}{\textit{Glareola pratincola}}
%\index{keywords}{\textit{Gelochelidon nilotic}} \index{keywords}{Nidificazione}
}
{\small
\noindent \textsc{\color{MUSEBLUE} Summary} / Collared pratincole \textit{Glareola pratincola} and gull-billed tern
\textit{Gelochelidon nilotica}, breed in bracksih marsehs (valli) of
the Po river Delta (Rovigo district). The surveys started form their
first settlement: 2004 for \textit{Glareola pratincola}, 2001 for
\textit{Gelochelidon nilotica}.

In this area they nest only on islands or dried beds of marshes.
\textit{Gelochelidon nilotica} bred with 40 -- 246 pairs (mean 157)
(2001-12); \textit{Glareola pratincola} with 1-22 pairs (mean 10,4).

The most important limiting factors are spring downpours and artificial
rising of water levels. \\
\noindent \textsc{\color{MUSEBLUE} Riassunto} / La pernice di mare \textit{Glareola pratincola }e la sterna zampenere
\textit{Gelochelidon nilotica}\textbf{ }sono specie che nidificano
all{\textquoteright}interno dei complessi vallivi della parte veneta
del Delta del Po (provincia di Rovigo). Sono state monitorate con
apposite campagne di censimento a partire dal loro insediamento,
avvenuto nel 2004 per la pernice di mare e nel 2001 per la sterna
zampenere.
}

La riproduzione avviene esclusivamente su barene o fondali di valle,
temporaneamente prosciugati. Per il periodo 2001-2012 la sterna
zampenere \`e presente con una popolazione nidificante compresa tra 40
e 246 coppie (media 157); la pernice di mare, invece, con 1-22 coppie
(media 10,4).

I principali fattori limitanti sono rappresentati dagli acquazzoni
primaverili e da improvvisi innalzamenti del livello idrico indotti
artificialmente per motivi di produzione ittica.





\section*{Introduzione}

La pernice di mare e la sterna zampenere sono specie, in Veneto,
tipicamente costiere, concentrate per la maggior parte nel Delta del Po
(provincia di Rovigo). Dato il loro status a livello nazionale, vengono
seguite con apposite campagne di monitoraggio da oltre un decennio: i
dati raccolti e solo in parte gi\`a pubblicati (Fracasso \textit{et
al}. 2003; Verza \& Trombin 2012) vengono qui sintetizzati. 
L{\textquoteright}area di indagine riguarda tutta la parte venete del
Delta del Po, ricadente in provincia di Rovigo.

\section*{Metodi}

Il periodo considerato va dal primo insediamento delle specie, quindi
dal 2001 fino alla stagione riproduttiva 2012. I dati sono stati
raccolti con campagne di censimento standardizzate annuali; per ogni
annata \`e stata svolta almeno un{\textquoteright}uscita di censimento,
ma per molti anni sono stati svolti censimenti completi due volte al
mese, da marzo a giugno.

\section*{Risultati e discussione}

\textbf{Sterna zampenere} - La sterna zampenere ha iniziato a nidificare con certezza in Veneto nel
2001, all{\textquoteright}intero in una valle della provincia di Rovigo
(Fracasso \textit{et al}. 2003). Da allora la popolazione \`e andata
rapidamente consolidandosi, con progressiva espansione a molte valli
del Delta: dapprima il complesso delle valli
Sacchetta-Canocchione-Moraro, poi nella vicina valle
Ca{\textquoteright} Pasta, rimasta per alcuni anni unico sito di
nidificazione, successivamente all{\textquoteright}interno delle valli
di Porto Tolle e Rosolina. La media delle coppie nidificanti per il
periodo 2001 -- 2012 \`e di 157 coppie, escludendo le annate 2004, 2005
e 2006 per le quali la qualit\`a dei dati non \`e sufficiente. Tale
popolazione risulta essere 1/3 del popolamento italiano, stimato in
circa 550 coppie nel 2002 (Brichetti  \& Fracasso 2004).

La nidificazione avviene spesso in consociazione con altri Caradriformi,
tra cui sterna comune e fratino. Si riproduce esclusivamente
all{\textquoteright}interno delle valli, su barene e isolotti privi di
vegetazione o con vegetazione a salicornie e \textit{Phragmites
australis} rada. Il periodo di nidificazione pu\`o protrarsi fino al
mese di luglio. Gli adulti riproduttivi si alimentano quasi
esclusivamente nelle acque dolci dei limitrofi rami del Po e dei canali
di bonifica in ambiente agrario.

La sua riproduzione \`e influenzata sostanzialmente
dall{\textquoteright}andamento climatico stagionale e della gestione
antropica valliva. Forti piogge in maggio e giugno possono determinare
il fallimento della nidificazione o lo spostamento delle colonie in
altri siti. La gestione delle barene e dei livelli idrici, effettuata a
scopi produttivi, determina la creazione o la distruzione dei siti
idonei alla nidificazione

%\rowcolors{2}{white!60!MUSEBLUE}{white}
\begin{adjustwidth}{-3.5cm}{-1cm}
\begin{table}[!h]
\centering
\footnotesize
\scalebox{.8}{
\newcolumntype{S}{>{\centering\arraybackslash}p{.08\columnwidth}}
\begin{tabular}{>{\raggedright\arraybackslash}p{.2\columnwidth}SSSSSSSSSSSS}
\toprule
\textbf{Luogo}&\textbf{2001}&\textbf{2002}&\textbf{2003}&\textbf{2004}&\textbf{2005}&\textbf{2006}&\textbf{2007}&\textbf{2008}&\textbf{2009}&\textbf{2010}&\textbf{2011}&\textbf{2012} \\
\toprule
Valle Morosina (Rosolina)&&&&&&&&&&10-20&7-8&90 \\
\midrule
Valle Ca' Pasta (Porto Viro)&&&&&?&10-?&100-120&110&100-130&&13& \\
\midrule
Valle Canocchione - Moraro (Porto Viro)&&&&10-100&&&&&&105-115&177-191&90 \\
\midrule
Valle Sacchetta (Porto Viro)&40&230&220-240&10-100&&&&&&&& \\
\midrule
Valle Ca' Zuliani (Porto Tolle)&&&?&&4-6&&&&&&0-5&66 \\
\midrule
Valle Ripiego (Porto Tolle)&&&&&&&&&&5&& \\
\midrule
\textbf{Totale}&40&230&220-240&10-100&?&?&100-120&110&90-130&120-140&197-217&246 \\
\bottomrule
\end{tabular}
}
\caption{Numero di coppie nidificanti di sterna zampenere nel delta del Po veneto}
\label{Verza_tab_1}
\end{table}
\end{adjustwidth}

\textbf{Pernice di mare} - La nidificazione, in Veneto, \`e fatto piuttosto recente. Un primo
insediamento \`e avvenuto nel 2002 nella zona veneziana di Valle
Vecchia di Caorle, non pi\`u accertato gli anni successivi (AsFaVe,
2003). Dal 2004 la Pernice di mare nidifica regolarmente ogni anno
nell{\textquotesingle}area del Delta del Po (valli di Porto Viro e
Porto Tolle), con una popolazione che appare consolidata e costituita
per il periodo 2004-12 da una media di 10,4 coppie. Questa popolazione
rappresenta circa il 10\% di quella nazionale, stimata nel 2000-01 in
100-150 coppie (Brichetti \& Fracasso 2004).

Le aree tipiche della Pernice di mare sono quelle alofile costiere. Per
la riproduzione sceglie superfici fangose o anche argilloso-sabbiose
prive di vegetazione, con predilezione per i fondali di laghi
temporaneamente prosciugati e, secondariamente, per le barene nude
recentemente rimaneggiate. Tali barene possono anche presentare rada
vegetazione alofila (es: generi \textit{Salsola, Aster }e
\textit{Limonium}) e devono trovarsi sopra il livello idrico massimo
estivo. La nidificazione sino ad ora \`e avvenuta esclusivamente
all{\textquotesingle}interno delle valli, in particolare Scanarello, S.
Carlo e Ca{\textquoteright} Zuliani; la specie si \`e riprodotta anche
all{\textquoteright}interno delle valli Sacchetta, S. Leonardo,
Ca{\textquoteright} Pasta, Chiusa e Ripiego; vi sono inoltre sospetti
che abbia potuto riprodursi anche in Bagliona e Ca{\textquoteright}
Pisani. La specie si riproduce sia con coppie singole, sia a gruppi di
2 o 3 coppie, sia a gruppi pi\`u consistenti sino a 11 insieme in
un{\textquoteright}unica area valliva. Si riproduce anche in
consociazione con fratino e fraticello.

\`E evidente, quindi, come tutto il complesso vallivo sia potenzialmente
idoneo alla nidificazione della specie. Data la sua biologia
riproduttiva osservata in altre aree italiane (ad es. Ferrara) \`e
possibile che le piane di bonifica possano ospitare la nidificazione
della specie, o comunque altre aree esterne alle valli da caccia e da
pesca.

L{\textquoteright}insediamento riproduttivo avviene da maggio, con pulli
osservati sino alla prima met\`a di luglio. Per
l{\textquotesingle}alimentazione utilizza, invece, le aree agrarie e di
argine adiacenti le valli (ad es. i siti Punta Paltanara, medicai di
Ca{\textquotesingle} Zuliani, penisola di S. Margherita). Fattore
determinante \`e la presenza di zone prative (prati arginali, incolti,
medicai) in grado di fornire una sufficiente quantit\`a di insetti. Si
osserva quindi un continuo pendolarismo tra le valli e le zone agrarie
di bonifica.

La grande potenzialit\`a offerta dalle zone vallive venete rappresenta
un possibile fattore di crescita della popolazione nazionale. Il
principale fattore limitante \`e rappresentato proprio dalle pratiche
di gestione attiva esercitate in valle, che da un lato sono in grado di
creare di anno in anno idonei siti di nidificazione (creazione di
barene, rimodellamento, messa in asciutta di laghi),
dall{\textquotesingle}altro mettono spesso a repentaglio le nidiate
(improvviso innalzamento dei livelli idrici).

Un grave fattore limitante \`e inoltre rappresentato dagli eventi
atmosferici negativi estivi. Nel 2010 si \`e assistito alla distruzione
della colonia di Valle Ca{\textquoteright} Zuliani a causa delle forti
piogge del mese di giugno, con ricostituzione della stessa in valle
Sacchetta nel mese di luglio.

Al fine di favorire la nidificazione della specie \`e auspicabile la
realizzazione di apposite barene vallive, livellate, prive di
vegetazione e preferibilmente costituite da materiale fangoso.

\begin{table}[!h]
\centering
%\rowcolors{2}{white!40!MUSEBLUE}{white}
\newcolumntype{S}{>{\centering\arraybackslash}p{.035\columnwidth}}
\scalebox{.7}{
\begin{tabular}{>{\raggedright\arraybackslash}p{.2\columnwidth}SSSSSSSSSSSSSSSSSS}
\toprule
\textbf{Luogo}&\multicolumn{2}{c}{\textbf{2004}}&\multicolumn{2}{c}{\textbf{2005}}&\multicolumn{2}{c}{\textbf{2006}}&\multicolumn{2}{c}{\textbf{2007}}&\multicolumn{2}{c}{\textbf{2008}}&\multicolumn{2}{c}{\textbf{2009}}&\multicolumn{2}{c}{\textbf{2010}}&\multicolumn{2}{c}{\textbf{2011}}&\multicolumn{2}{c}{\textbf{2012}} \\
\toprule
Valli di Porto Viro&6&8&1&1& & & & & 1&3&10&13&3&7&1&1&0&1 \\
Valli di Porto Tolle&7&8&7&15&4&4&15&15&0&1&6&9&11&14&2&5&4&5 \\
\midrule
\textbf{Totale}&13&16&8&16&4&4&15&15&1&4&16&22&14&21&3&6&4&6 \\
\bottomrule
\end{tabular}
}
\caption{Numero di coppie nidificanti di pernice di mare nel delta del Po veneto}
\label{Verza_tab_1}
\end{table}

\section*{Bibliografia}
\begin{itemize}\itemsep0pt
 \item Associazione Faunisti Veneti, 2003. (Redattori: Bon M., Sighele M.,
Verza E.). Rapporto ornitologico per la Regione Veneto. Anno 2002.
\textit{Boll. Mus. civ. St. nat. Venezia}., 54 (2003): 123-160.

 \item Brichetti P., Fracasso G., 2004 - \textit{Ornitologia Italiana}. Vol. 2
Tetraonidae - Scolopacidae. Alberto Perdisa Editore, Bologna.

 \item Fracasso G., Verza E., Boschetti E. (a cura di), 2003.  \textit{Atlante
degli uccelli nidificanti in provincia di Rovigo}. Provincia di Rovigo
- Associazione Faunisti Veneti -- Gruppo di Studi Naturalistici
{\textquotedblleft}Nisoria{\textquotedblright}.

 \item Verza E., Trombin D. (a cura di), 2012. \textit{Le valli del Delta del
Po}. Ente Parco Regionale Veneto del Delta del Po. Apogeo Editore.
\end{itemize}
