\setcounter{figure}{0}
\setcounter{table}{0}

\begin{adjustwidth}{-3.5cm}{0cm}
\pagestyle{CIOpage}
\authortoc{\textsc{Partel P.}, \textsc{Rotelli L.}}
\chapter*[Progetto radiotelemetrico sul Gallo cedrone]{Il progetto radiotelemetrico sul
Gallo cedrone \textit{Tetrao urogallus} nel Parco Naturale
Paneveggio - Pale di San Martino}
\addcontentsline{toc}{chapter}{Progetto radiotelemetrico sul Gallo cedrone}

\textsc{Piergiovanni Partel}$^{1*}$, \textsc{Luca Rotelli}$^{2}$  \\

\index{Partel Piergiovanni} \index{Rotelli Luca}
\noindent\color{MUSEBLUE}\rule{27cm}{2pt}
\vspace{1cm}
\end{adjustwidth}


\marginnote{\raggedright $^1$Parco Naturale Paneveggio Pale di San Martino \\
$^2$Universit\`a di Friburgo \\
\vspace{.5cm}
{\emph{\small $^*$Autore per la corrispondenza: \href{mailto:piergiovanni.partel@parcopan.org}{pier\allowbreak gio\allowbreak van\allowbreak ni.\allowbreak par\allowbreak tel@\allowbreak par\allowbreak co\allowbreak pan.\allowbreak org}}} \\
\keywords{Trentino, \textit{Tetrao urogallus}, radiotelemetria, successo
riproduttivo}
{Trentino, \textit{Tetrao urogallus}, radiotelemetry, breeding success}
%\index{keywords}{Trentino} \index{keywords}{\textit{Tetrao urogallus}} \index{keywords}{Radiotelemetria} \index{keywords}{Successo riproduttivo}
}
{\small
\noindent \textsc{\color{MUSEBLUE} Summary} / The Paneveggio - Pale di San Martino Provincial Park has been carrying
out a radiotelemetry project for four years. During this period 22
cocks and 6 hens were captured, tagged and intensively tracked. In the
meanwhile counts were carried out both in spring, on the leeks, to
determine the number of displaying cocks and thus the population size
and structure and in summer, by means of pointing dogs, with the aim to
determine the breeding success.\\
}


\section*{Introduzione}

{Negli ultimi decenni le popolazioni alpine
di gallo cedrone }\textit{{Tetrao
urogallus}}{ hanno subito
un{\textquoteright}importante riduzione delle loro consistenze sulle
Alpi italiane. Tra le principali cause di declino vengono indicate la
perdita, il degrado e la frammentazione degli habitat, alcune pratiche
selvicolturali, ma anche un aumento della pressione predatoria, i
cambiamenti climatici e i disturbi causati da alcune attivit\`a
antropiche come gli sport invernali (Storch 2000).}

{Nelle aree centrali e meridionali delle
Alpi, le azioni di conservazione a favore della specie fino ad ora
}si\textcolor{red}{ }{sono basate
prevalentemente su indicazioni generiche piuttosto che su solide
conoscenze maturate nel corso di studi mirati. Mancano, infatti,
conoscenze approfondite inerenti le relazioni esistenti tra il gallo
cedrone e il suo habitat e quindi delle effettive cause che minacciano
le popolazioni di questo Tetraonide.}

{Per questo motivo dal 2009 il Parco
Naturale Paneveggio-Pale di San Martino ha avviato un progetto di
ricerca sulla biologia ed ecologia della specie, condotto con
l{\textquoteright}ausilio della radiotelemetria.}

{La ricerca \`e stata realizzata in
collaborazione con l{\textquoteright}Universit\`a di Friburgo e il
Servizio Foreste e fauna della Provincia autonoma di Trento.}

\section*{Metodi}

{Dal 2009 al 2012, 22 maschi e 6 femmine
sono stati catturati, marcati con radiocollari VHF del peso di 20 gr. 
e intensamente localizzati, per mezzo di
}\textit{{homing
in}}{ e triangolazione. Nello stesso
periodo sono stati eseguiti censimenti primaverili sulle arene di
canto, per conoscere la consistenza e la struttura della popolazione.
Inoltre sono stati realizzati censimenti estivi, con
l{\textquoteright}ausilio di cani da ferma, al fine di determinare il
successo riproduttivo della specie.}

\section*{Risultati e discussione}

{La popolazione di gallo cedrone
nell{\textquoteright}area del Parco e nei territori limitrofi risulta
essere vitale, con la presenza di arene di canto spesso ben
strutturate, caratterizzate da un numero medio di maschi per arena di
canto superiore a 2, con
}\textit{{lek}}{
frequentati anche fino da 10 maschi. Tra i dati ottenuti risalta il
basso successo riproduttivo (0,63 pulli per femmina), controbilanciato
da un tasso di sopravvivenza piuttosto elevato degli adulti (68\%). In
media solo il 35\% delle femmine in estate \`e accompagnata da nidiata,
mentre il numero medio di pulli per covata \`e di 1,85. Il 65\% dei
nidi viene perso durante la cova. Di questi l'80\%
risulta essere predato, mentre nel rimanente 20\% dei casi i nidi
vengono abbandonati dalle femmine a causa del disturbo antropico. Tra
le cause di mortalit\`a degli adulti, importante risulta essere
soprattutto la predazione, la cui incidenza sulla popolazione pu\`o
variare notevolmente di anno in anno, oltre a quella da collisione
contro i cavi degli impianti di risalita (Rotelli 2012). }

{Questi primi risultati, anche se non ancora
definitivi, tuttavia sono importanti per poter implementare misure di
conservazione in grado di ridurre i fattori che influenzano in modo
negativo il trend delle popolazioni di gallo cedrone
nell{\textquoteright}area meridionale delle Alpi.}

\section*{Bibliografia}
\begin{itemize}\itemsep0pt
	\item Storch I., 2000 - Grouse Status Survey and Conservation Action
Plan 2000-2004. WPA/BirdLife/SSC Grouse Specialist Group. IUCN, Gland.
Switzerland and Cambridge, UK and the World Pheasant Association,
Reading, UK. X+, 112 pp.

	\item Rotelli L., 2012 - Risultati dell{\textquoteright}attivit\`a
svolta nell{\textquoteright}ambito del progetto sul Gallo cedrone nel
Parco Naturale Paneveggio -- Pale di San Martino nel periodo marzo
2011 -- febbraio 2012 (con integrazioni fino al 31 marzo 2012).
Dattiloscritto a cura del Dipartimento di Ecologia e Gestione della
Fauna Selvatica dell{\textquoteright}Universit\`a di Freiburg
(Germania).
\end{itemize}

