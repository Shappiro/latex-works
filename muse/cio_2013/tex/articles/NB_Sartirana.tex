\setcounter{figure}{0}
\setcounter{table}{0}

\begin{adjustwidth}{-3.5cm}{0cm}
\pagestyle{CIOpage}
\authortoc{\textsc{Sartirana F.}, \textsc{Valfiorito R.}}
\chapter*[\textit{Status} e distribuzione
dell{\textquoteright}aquila reale in provincia di Imperia]{Status e distribuzione dell{\textquoteright}aquila reale \textbf{\textit{Aquila chrysaetos}}\textbf{ }\textbf{\textit{
}}\textbf{in provincia di Imperia}}
\addcontentsline{toc}{chapter}{\textit{Status} e distribuzione
dell{\textquoteright}aquila reale in provincia di Imperia}

\textsc{Fabiano Sartirana}$^{1*}$, \textsc{Rudy Valfiorito}$^2$  \\

\index{Sartirana Fabiano} \index{Valfiorito Rudy}
\noindent\color{MUSEBLUE}\rule{27cm}{2pt}
\vspace{1cm}
\end{adjustwidth}


\marginnote{\raggedright $^1$Via Don Minzoni 14/19, 18100 Imperia\\
$^2$Corso Verbone 199, 18036 Soldano (IM) \\
\vspace{.5cm}
{\emph{\small $^*$Autore per la corrispondenza: \href{mailto:fabianosartirana@libero.it}{fa\allowbreak bia\allowbreak no\allowbreak sar\allowbreak ti\allowbreak ra\allowbreak na@\allowbreak li\allowbreak be\allowbreak ro.\allowbreak it}}} \\
\keywords{\textit{Aquila chrysaetos}, coppie territoriali,
densit\`a, sito riproduttivo, provincia di Imperia}
{\textit{Aquila chrysaetos}, territorial pairs,
density, breeding site, province of Imperia.}
}
%\index{keywords}{Densit\`a} \index{keywords}{Sito riproduttivo} \index{keywords}{Provincia di Imperia}
{\small
\noindent \textsc{\color{MUSEBLUE} Summary} / The most southern alpine 
\textit{Aquila chrysaetos} population, located in the province of Imperia (Liguria, Italy), includes 8 territorial pairs (4.6 pairs/1000 km\textsuperscript{2}). Between the 12 known
eyries, the lowest is located at 380 m a.s.l., probably the minimum
altitude recorded across the italian Alps. \\
}
\vspace{1cm}

Il presente studio si pone come obiettivo
l{\textquoteright}aggiornamento dello status e della distribuzione
dell{\textquoteright}aquila reale (\textit{Aquila chrysaetos}) nella
provincia di Imperia (1156 km\textsuperscript{2}), che ospita la
popolazione pi\`u meridionale delle Alpi. L{\textquoteright}area
indagata \`e caratterizzata dalla presenza di un{\textquoteright}estesa
copertura boscosa e di ambienti tipicamente mediterranei, come macchia
e gariga e da ridotte praterie alpine frammiste a estese zone rocciose
calcaree. Di conseguenza i territori di caccia comprendono una
variet\`a di habitat maggiore rispetto alle tipiche zone alpine, alla
quale \`e verosimilmente associata un{\textquoteright}alta diversit\`a
specifica di prede. Ci\`o \`e attestato anche
dall{\textquoteright}osservazione di numerosi atti di predazione, in
particolare nei confronti di camoscio, capra domestica, cinghiale,
marmotta, volpe, gatto domestico, lepre comune, fagiano comune e
starna.
Partendo da osservazioni di campagna e da dati pregressi di
nidificazione, sono state effettuate delle uscite mirate a stabilire la
distribuzione delle coppie territoriali e ad individuare i siti di
nidificazione.
La specie fino al 2013 era presente con 8 coppie territoriali (densit\`a
4.6/1000 km\textsuperscript{2}, calcolata sull{\textquoteright}intera
area occupata dai territori presunti delle coppie). In seguito a
ripetute osservazioni sul campo si \`e ipotizzata una nona coppia tra i
territori di Imperia e Sanremo (per una densit\`a a 5.1 coppie/1000
km\textsuperscript{2}). Questi valori risultano essere inferiori a
quelli delle principali aree alpine (per es. nel Parco Naturale
Dolomiti Friulane: 16.8/1000 km\textsuperscript{2}
n=10,\textsuperscript{ } Borgo\textsc{ 2009}) e superiori a quelli
appenninici (per es. nell{\textquoteright}Appennino Umbro-marchigiano:
2.9/1000 km\textsuperscript{2} n=13, Magrini \textit{et al. }2001).
Sono stati individuati 18 nidi (2.2 per coppia, Min. 1 Max. 6), situati
ad un{\textquoteright}altitudine media di 1005 m (Min. 380 m, Max. 1720
m., N=14). Un sito riproduttivo - utilizzato sia nel 2012 che nel 2013
- \`e stato localizzato ad un{\textquoteright}altitudine di 380 m, che
rappresenta probabilmente la quota pi\`u bassa per le Alpi, nonch\'e
una delle minori per l{\textquoteright}Italia.
Dei 18 nidi noti, 15 (83.3\%) ricadono in aree della Rete Natura 2000
(le ZPS sono pressoch\'e sovrapposte ai SIC), 12 (66.7\%) in aree a
divieto di caccia e 8 (44,4\%) si trovano all{\textquoteright}interno
del Parco Naturale Regionale delle Alpi Liguri. Uno solo dei nidi
considerati si trova al di fuori dei confini provinciali.

\section*{Ringraziamenti}
Si ringraziano sia per il materiale fotografico che per le osservazioni
sul campo le seguenti persone: Gian Pietro Pittaluga, Franco Bianchi e
Daniele Chianea.

\section*{Bibliografia}
\begin{itemize}\itemsep0pt
	\item Borgo A., 2009 - L{\textquoteright}Aquila reale ecologia,
biologia e curiosit\`a sulla regina del Parco Naturale delle
\textit{Dolomiti Friulane.} Parco Naturale delle Dolomiti Friulane, 192
pp.

	\item Magrini M., Perna P., Angelini J. \& Armetano L., 2001\textsc{ -
}Tendenza delle popolazioni di Aquila reale \textit{Aquila chrysaetos},
il Lanario \textit{Falco }\textit{biarmicus} e il Pellegrino\textit{
Falco peregrinus }nelle Marche e in Umbria. \textit{Avocetta, }25: 57.
\end{itemize}
