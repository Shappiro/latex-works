\setcounter{figure}{0}
\setcounter{table}{0}

\begin{adjustwidth}{-3.5cm}{-1cm}
\pagestyle{CIOpage}
\authortoc{\textsc{Morici F.}, \textsc{Mencarelli M.},
\textsc{Sebastianelli C.}, \textsc{Morganti N.}}
\chapter*[Disturbo alla nidificazione del fratino e misure di protezione]{Studio del disturbo alla nidificazione del fratino \textbf{\textit{Charadrius alexandrinus}} e misure di
protezione dei nidi lungo i litorali di Senigallia e Montemarciano (AN)
}
\addcontentsline{toc}{chapter}{Disturbo alla nidificazione del fratino e misure di protezione}
 
\end{adjustwidth}
\begin{adjustwidth}{-3.5cm}{0cm}
\textsc{Francesca Morici}$^{1*}$, \textsc{Mauro Mencarelli}$^{1*}$,
\textsc{Claudio Sebastianelli}$^{2**}$, \textsc{Niki Morganti}$^{1*}$  \\

\index{Morici Francesca} \index{Mencarelli Mauro} \index{Sebastianelli Claudio} \index{Morganti Niki}
\noindent\color{MUSEBLUE}\rule{27cm}{2pt}
\vspace{1cm}
\end{adjustwidth}


\marginnote{\raggedright $^1$Studio Naturalistico Diatomea, Senigallia
\textit{info@studiodiatomea.it} \\
$^2$A.R.C.A., Senigallia \\
\vspace{.5cm}
{\emph{\small $^*$Autore per la corrispondenza: \href{mailto:info@studiodiatomea.it}{in\allowbreak fo@\allowbreak stu\allowbreak dio\allowbreak dia\allowbreak to\allowbreak mea.\allowbreak it}}} \\
{\emph{\small $^{**}$Autore per la corrispondenza: \href{mailto:info@associazionearca.eu}{in\allowbreak fo@\allowbreak as\allowbreak so\allowbreak cia\allowbreak zio\allowbreak ne\allowbreak ar\allowbreak ca.\allowbreak eu}}} \\
\vspace{.5cm}
\keywords{Senigallia e Montemarciano, \textit{Charadrius
alexandrinus}, disturbo dei nidi, misure di conservazione, gabbiette di
esclusione, perimetrazione dei nidi}
{Senigallia and Montemarciano, \textit{Charadrius
alexandrinus}, nest disturbance, conservation measures,
exclosure box, nest fencing}
%\index{keywords}{Senigallia e Montemarciano} \index{keywords}{\textit{Charadrius
%alexandrinus}} \index{keywords}{Disturbo dei nidi} \index{keywords}{Misure di conservazione} \index{keywords}{Gabbiette di esclusione} \index{keywords}{Perimetrazione dei nidi}
}
{\small
\noindent \textsc{\color{MUSEBLUE} Summary} / We investigated the types of disturbance
and their effects on kentish plover \textit{Charadrius
alexandrinus} nests, also analyzing
the effectiveness of some protection measures. Individuals
responded differently to disturbance returning to the nest on average after 10 minutes if
disturbed by humans, after 60 minutes or abandoning the nest if
disturbed by motor-vehicles. Nest protection operated by exclusion
boxes against predators (dogs, crows) was not satisfying as 50\% to
100\% of the nests were abandoned because of disturbance determined by
humans attracted by the presence of the box. Fencing operated by red
and white strings around a small triangular area largely improved nest
survival in tourist beaches.\\
}
\vspace{1cm}




Nella stagione riproduttiva 2013 abbiamo studiato i diversi tipi di
disturbo che si osservano sui nidi di fratino e valutato i loro effetti
sulla nidificazione analizzando anche l'efficacia
delle azioni di tutela intraprese al fine di attuare nei prossimi anni
misure di protezione mirate pi\`u efficaci. Attualmente la specie
nidifica con regolarit\`a nelle Marche lungo i litorali di Senigallia,
Montemarciano e Fermo (Morganti \textit{et al}. 2009) (Mencarelli
\textit{et al. }2013). La ricerca ha interessato il litorale del Comune
di Senigallia (AN) e Montemarciano (AN). Sono state individuate quattro
aree distinte: Cesanella e Cesano (litorale Nord) caratterizzate da
spiaggia sabbiosa e dune embrionali; Marzocca (litorale Sud) e
Montemarciano caratterizzati da spiaggia ghiaiosa con vegetazione
dunale rada. Per lo studio del disturbo sono stati monitorati 5 nidi
con osservazioni dirette della durata variabile dai 15 ai 120 minuti,
condotte tra le ore 10:00 e le ore 13:00. Sono state annotate le
seguenti informazioni: tipologia di disturbo e distanza dal nido,
risposta del fratino al disturbo e periodo in cui il nido rimane
incustodito. Per la protezione dei nidi sono state adottate due
strategie: 1. perimetrazione triangolare con nastro plastificato bianco
e rosso per evitare il calpestio; 2. gabbiette di esclusione anti
predazione con rete a maglia larga chiuse superiormente. Nel 2013 sono
state adottate misure di protezione in 14 nidi con esiti differenti a
seconda della zona in cui sono state utilizzate.
L{\textquoteright}utilizzo delle gabbiette ha visto per il litorale
Nord il 50\% dei nidi con esito positivo, per il litorale Sud il 100\%
dei nidi con esito negativo e per Montemarciano il 50\% dei nidi con
esito positivo. L{\textquoteright}utilizzo della recinzione ha visto
per il litorale Sud l{\textquoteright}80\% dei nidi con esito positivo
e per Montemarciano il 100\% dei nidi con esito positivo. Infine,
l{\textquoteright}utilizzo della recinzione con la gabbietta, adottata
solo per un nido nella zona del litorale Sud, ha avuto esito negativo.
L'analisi del disturbo ai nidi ha permesso di
classificarne le tipologie e individuarne le pi\`u frequenti,
determinare i tempi di allontanamento dal nido e identificare quelle
che ne determinano l'abbandono. La tipologia di
disturbo pi\`u ricorrente \`e stata la presenza antropica, con un
disturbo medio di 10 minuti e un ritorno alla cova tra i 4 e i 20
minuti dopo la scomparsa del disturbo. I mezzi meccanici hanno creato
un disturbo medio di oltre 60 minuti e un ritorno al nido variabile da
60 minuti al non ritorno. Il prolungarsi di questi due disturbi ha
causato l'abbandono di almeno 3 nidi. Le misure di
protezione adottate non sono state utilizzate in maniera standard lungo
il litorale. Mentre le gabbiette sono state efficaci
nell'area del litorale Nord, al litorale Sud e
Montemarciano la presenza delle gabbiette ha incuriosito i bagnanti
che, avvicinandosi troppo al nido, ne hanno causato
l'abbandono. In queste zone sono state utilizzate con
maggiore successo le recinzioni che, delimitando pi\`u genericamente
l'area di nidificazione, attirano meno
l{\textquoteright}attenzione dei bagnanti. L{\textquoteright}utilizzo
contemporaneo di recinzione e gabbietta lungo il litorale Sud ha
comportato gli stessi problemi osservati con l{\textquoteright}utilizzo
della sola gabbietta. Dal 2014 verranno diversificate le tipologie di
recinzioni in base alla zona del litorale in cui si trovano i nidi. I
dati raccolti durante il monitoraggio del disturbo hanno fornito
elementi utili alle future azioni di conservazione. Nonostante alcuni
casi di predazione, non c{\textquoteright}\`e stato abbandono della
covata per presenza di cani o cornacchie: i fratini hanno risposto con
la fuga, per poi fare ritorno al nido. In un caso \`e stata osservata
un'interazione con un conspecifico:
l'individuo in cova ha abbandonato per pochi minuti le
uova per allontanare l'intruso. Preoccupa
particolarmente il disturbo arrecato dai mezzi meccanici. Sarebbe
necessario adottare misure di protezione condivise tra amministrazioni
locali, operatori e coloro che si occupano della tutela della specie.

\section*{Ringraziamenti}
Gli autori ringraziano Claudia Latini e Caterina Abbrugiati, tirocinanti
della Facolt\`a di Scienze dell{\textquoteright}UNIVPM di Ancona, che
hanno partecipato al monitoraggio.

\section*{Bibliografia}
\begin{itemize}\itemsep0pt
	\item Mencarelli M., Morici F., Sebastianelli C. \& Morganti N., 2013 -
\textcolor{black}{Il Fratino (}\textit{\textcolor{black}{Charadrius
alexandrinus}}\textcolor{black}{) nidificante sul litorale di
Senigallia (AN) e Montemarciano (AN): distribuzione, problematiche e
strategie di conservazione. Stagioni 2009-2012. }U.D.I. XXXVIII: 67-76
(2013).

	\item Morganti N., Fusari M., Mencarelli M., Morici F., Pascucci M. \& Marini
G., 2009 - Aspetti ecologici della nidificazione di \textit{Charadrius
alexandrinus }lungo il litorale marchigiano. In: Brunelli M., Battisti
C., Bulgarini F., Cecere J.G., Fraticelli F., Giustin M., Sarrocco S.
\& Sorace A. (A cura di). Atti del XV Convegno Italiano di Ornitologia.
Sabaudia, 14-18 ottobre 2009\textit{. Alula}, XVI (1-2): 252-254.
\end{itemize}
