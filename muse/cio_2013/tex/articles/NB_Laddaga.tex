\setcounter{figure}{0}
\setcounter{table}{0}

\begin{adjustwidth}{-3.5cm}{0cm}
\pagestyle{CIOpage}
\authortoc{\textsc{Laddaga L.}, \textsc{Luoni F.},
\textsc{Martinoli A.}, \textsc{Soldato G.}}
\chapter*[Monitoraggio ornitologico del lago di Varese e della
palude Brabbia]{Lago di Varese e palude Brabbia: sintesi dei primi anni di monitoraggio del Gruppo locale di conservazione LIPU}
\addcontentsline{toc}{chapter}{Monitoraggio ornitologico del lago di Varese e della palude Brabbia}

\textsc{Lorenzo Laddaga}$^{1*}$, \textsc{Federica Luoni}$^{1**}$,
\textsc{Alessio Martinoli}$^{1}$, \textsc{Giovanni Soldato}$^{1}$ \\

\index{Laddaga Lorenzo} \index{Luoni Federica} \index{Martinoli Alessio} \index{Soldato Giovanni}
\noindent\color{MUSEBLUE}\rule{27cm}{2pt}
\vspace{1cm}
\end{adjustwidth}



\marginnote{\raggedright $^1$LIPU BirdLife Italia -- Via Udine 3/A, 43122 Parma  \\
\vspace{.5cm}
{\emph{\small $^*$Autore per la corrispondenza: \href{mailto:l.laddaga@libero.it}{l.\allowbreak lad\allowbreak da\allowbreak ga@\allowbreak li\allowbreak be\allowbreak ro.\allowbreak it}}} \\
{\emph{\small $^**$Autore per la corrispondenza: \href{mailto:federica.luoni@lipu.it}{fe\allowbreak de\allowbreak ri\allowbreak ca.\allowbreak lu\allowbreak oni@\allowbreak li\allowbreak pu.\allowbreak it}}} \\
\keywords{GLC, palude Brabbia, lago di Varese, censimenti avifauna}
{LCG, Brabbia Marsh, Varese lake, bird census}
%\index{keywords}{GLC} \index{keywords}{Palude Brabbia} \index{keywords}{Lago di Varese} \index{keywords}{Censimenti avifauna}
}
{\small
\noindent \textsc{\color{MUSEBLUE} Summary} / Since 2009 a group of LIPU volunteers (LCG) undertakes bird surveys in
the Brabbia marsh Natural Reserve and Varese lake. In Brabbia marsh
were contacted 162 species; in particular, 20 species of wintering
waterbirds, and 19 nesting were found. On lake Varese the number of
aquatic species surveyed is 29 in the winter and 19 in spring.
}



\section*{Introduzione}

Dal 2009 la LIPU ha dato via al progetto GLC (Gruppi Locali di
Conservazione), al fine di monitorare sia il livello di conservazione,
che l{\textquoteright}andamento delle popolazioni di uccelli
nidificanti e svernanti nelle IBA o nei siti Rete Natura 2000, tramite
l{\textquoteright}attivit\`a di gruppi coordinati di volontari. Presso
la Riserva Naturale Oasi LIPU palude Brabbia \`e operante un GLC che
svolge monitoraggi ornitologici nell{\textquoteright}area della riserva
e sull{\textquoteright}adiacente lago di Varese, anche grazie al
contributo dell{\textquoteright}Unione Europea
nell{\textquoteright}ambito del progetto LIFE10 NAT IT241 TIB
{\textquotedblleft}Trans Insubria Bionet{\textquotedblright}. 

\section*{Metodi}

I metodi si differenziano in funzione della stagione e
dell{\textquoteright}area: sul lago viene svolto mensilmente, da marzo
a ottobre, un transetto in barca, mentre, nei mesi invernali i
monitoraggi vengono effettuati da punti fissi da riva. In palude
Brabbia vengono svolti transetti lineari per il censimento delle specie
legate agli ambienti boschivi e osservazioni da terra per la fauna
acquatica, oltre a due censimenti da barca nei mesi di aprile e maggio.
Per alcune specie di particolare interesse conservazionistico vengono
svolti censimenti \textit{ad hoc} durante la stagione riproduttiva. Per
ognuna delle due aree viene poi stilato un elenco mensile complessivo
aggiungendo anche osservazioni non standardizzate. 

\section*{Risultati e discussione}

In totale in palude Brabbia sono state contattate 162 specie; in
particolare sono state rilevate 20 specie di acquatici svernanti, e 19
nidificanti. La specie pi\`u abbondante, a parte il cormorano
\textit{Phalacrocorax carbo} presente con un \textit{roost} invernale
di oltre 1000 individui, \`e risultata essere
l{\textquoteright}alzavola \textit{Anas crecca} che conta contingenti
di quasi 100 individui nel periodo invernale, seguita dal germano reale
\textit{Anas platyrhynchos} che raggiunge anch{\textquoteright}esso la
massima concentrazione in inverno e che nidifica con circa una decina
di coppie. 

Per il lago di Varese il numero di specie acquatiche contattate \`e di
29 nel periodo invernale, tra le quali risulta essere dominante il
gabbiano comune \textit{Chroicocephalus ridibundus} presente con un
contingente di circa 2500 individui. Sono 19, invece, le specie
acquatiche nidificanti, tra cui svasso maggiore \textit{Podiceps
cristatus} e folaga \textit{Fulica atra} risultano essere le pi\`u
comuni.

Nell{\textquoteright}area della Riserva \`e presente anche una garzaia
di circa 100 nidi occupati al 70\% da airone cenerino \textit{Ardea
cinerea} e per la restante parte da nitticora \textit{Nycticorax
nycticorax}. L{\textquoteright}airone rosso \`e presente, invece, con 5
coppie nell{\textquoteright}area della palude Brabbia, mentre sulla
sponda nord del lago di Varese dal 2011 si \`e, inoltre, insediata una
garzaia di airone rosso e cenerino composta rispettivamente da 5 e 7
coppie. La popolazione nidificante di tarabusino \textit{Ixobrychus
minutus} \`e risultata costante durante gli anni di censimento (8
coppie sul lago di Varese e 1-2 coppie in palude Brabbia), ma in netto
calo rispetto a quanto riportato dall{\textquoteright}atlante dei
nidificanti della provincia di Varese che stima circa 20 coppie
nidificanti in palude Brabbia fino alla fine degli anni
{\textquoteright}90 (Gagliardi \textit{et al}. 2007). 

\`E stata accertata la nidificazione di una coppia di moretta tabaccata
\textit{Aythya nyroca} nella Riserva per tutti gli anni di osservazione
e, nel corso della stagione 2012, di 4 coppie sul lago di Varese, dove
\`e presente un piccolo nucleo stanziale di 8 individui che sono stati
contattati negli anni anche durante i censimenti invernali. 

Il falco di palude \textit{Circus aeruginosus} \`e presente sia
nell{\textquoteright}area delle Riserva che sul lago di Varese nel
periodo primaverile ed estivo durante tutti gli anni di monitoraggio
con un numero di individui da 3 a 9 (aprile 2011), in palude Brabbia la
specie \`e stata osservata anche durante i mesi invernali. Durante i
mesi estivi \`e stato osservato dal 2010, almeno un giovane, il che fa
supporre una probabile nidificazione della specie
all{\textquoteright}interno dell{\textquoteright}IBA, anche se non \`e
stato ritrovato il sito di nidificazione. Questa risulterebbe la prima
nidificazione della specie in provincia di Varese, essa infatti non \`e
presente negli atlanti dei nidificanti 2003-2005 (Gagliardi \textit{et
al}. 2007) e 1983-1987 (Guenzani \& Saporetti 1988).

\section*{Ringraziamenti}

Gli autori ringraziano in primo luogo i volontari del GLC lago di Varese
e palude Brabbia.

Un ringraziamento anche a tutti coloro che hanno permesso la
realizzazione di questi monitoraggi e supervisionato i lavori svolti.
Grazie quindi a: Barbara Ravasio, Claudio Celada, Giorgia Gaibani,
Massimo Soldarini. 

Un grazie anche al personale di provincia di Varese e ai partner del
progetto LIFE TIB Fondazione Cariplo e Regione Lombardia.

\section*{Bibliografia}
\begin{itemize}\itemsep0pt
	\item Gagliardi A., Guenzani W., Pratoni D.G., Saporetti F. \& Tosi G. (a cura
di), 2007 {}-- \textit{Atlante Ornitologico Georeferenziato della
provincia di Varese. Uccelli nidificanti 2003-2005}. Provincia di
Varese; Civico Museo Insubrico di Storia Naturale di Induno Olona;
Universit\`a degli Studi dell{\textquoteright}Insubria, sede di Varese,
295 pp.

	\item Guenzani W. \& Saporetti F., 1988 {}- \textit{Atlante degli uccelli
nidificanti in provincia di Varese}. Edizioni Lativia.
\end{itemize}

