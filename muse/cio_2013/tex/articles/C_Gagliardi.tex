\begin{otherlanguage}{english}
\setcounter{figure}{0}
\setcounter{table}{0}

\begin{adjustwidth}{-3.5cm}{0cm}
\pagestyle{CIOpage}
\authortoc{\textsc{Gagliardi A.}, \textsc{Casola D.}, 
\textsc{Preantoni D.}, \textsc{Wauters L.}, 
\textsc{Martinoli A.}, \textsc{Fasola M.}}
\chapter*[Breeding interactions between great cormorants and grey
herons]{Interference between great cormorants \textbf{\textit{Phalacrocorax carbo}}\textbf{ and}\textbf{ herons
}\textbf{\textit{Ardea cinerea}}\textbf{ breeding in syntopy
}\textbf{in NW Italy}}
\addcontentsline{toc}{chapter}{Breeding interactions between great cormorants and grey
herons}

\textsc{Alessandra Gagliardi}$^{1*}$, \textsc{Daniela Casola}$^{2}$, 
\textsc{Damiano Preantoni}$^{2}$, \textsc{Lucas Wauters}$^{2}$, 
\textsc{Adriano Martinoli}$^{2}$, \textsc{Mauro Fasola}$^{3}$\\

\index{Gagliardi Alessandra} \index{Casola Daniela} \index{Preantoni Damiano} \index{Wauters Lucas} \index{Martinoli Adriano} \index{Fasola Mauro}
\noindent\color{MUSEBLUE}\rule{27cm}{2pt}
\vspace{1cm}
\end{adjustwidth}



\marginnote{\raggedright $^1$Istituto-Oikos, via Crescenzago 1 - 20134
Milano \\
$^2$Dipartimento di Scienze Teoriche e Applicate,
Universit\`a degli Studi dell{\textquotesingle}Insubria, via Dunant 3 -
21100 Varese \\
$^3$Dipartimento Scienze Terra Ambiente - Universit\`a degli Studi di
Pavia, via Ferrata 1 -- 27100 Pavia \\
\vspace{.5cm}
{\emph{\small $^*$Autore per la corrispondenza: \href{mailto:alessandra.gagliardi@uninsubria.it}{a\allowbreak les\allowbreak san\allowbreak dra.\allowbreak ga\allowbreak gliar\allowbreak di@\allowbreak u\allowbreak nin\allowbreak su\allowbreak bria.\allowbreak it}}} \\
\keywords{\textit{Phalacrocorax carbo}, \textit{Ardea cinerea},
siti di nidificazione, interazioni fra specie, andamento di
popolazione, Italia nord-occidentale}
{\textit{Phalacrocorax carbo}, \textit{Ardea cinerea}, nesting
site, interactions between species, population trend, north-west Italy}
%\index{keywords}{\textit{Phalacrocorax carbo}} \index{keywords}{\textit{Ardea cinerea}}
%\index{keywords}{Siti di nidificazione} \index{keywords}{Interazioni fra specie} \index{keywords}{Andamento di
%popolazione} \index{keywords}{Italia nord-occidentale}
}
{\small
\noindent \textsc{\color{MUSEBLUE} Summary} / Most of the colonies of great cormorant in Italy are located in sites
already used by breeding grey herons and other colonial
\textit{Ardeidae}. We analyzed data from long term monitoring programs
of breeding colonial herons and great cormorants (performed by the
research group {\textquotedblleft}Garzaie Italia{\textquotedblright},
and the Cormorant Colony Count Group) and field data collected in a
sample of colonies where the two species breed in synthopy, in order to
check any potential pattern of interference. In detail, we investigated
the long-term trend in the number of grey heron nests in the colonies
with and without cormorants, the spatial overlap of the nest
distribution among species and the vertical distribution of the nests.
The arrival of the great cormorant seems to have two effects on
co-breeding grey herons: a gradual diminution of the number of grey
heron nests and a gradual spatial and vertical segregation between the
two species inside the same colony.  \\
\noindent \textsc{\color{MUSEBLUE} Riassunto} / Molte delle colonie di cormorano \textit{Phalacrocorax carbo} in Italia
sono insediate in aree occupate anche da airone cenerino \textit{Ardea
cinerea}. In questo lavoro analizziamo in modo congiunto i dati
raccolti nell{\textquoteright}ambito del programma di monitoraggio
delle colonie di \textit{Ardeidae} coloniali in Italia nord-occidentale
e i dati rilevati in colonie ove le due specie sono compresenti per
verificare se la presenza del cormorano influenza
l{\textquotesingle}andamento delle popolazioni di airone cenerino e la
distribuzione spaziale dei nidi di questa specie
all{\textquoteright}interno delle colonie. Abbiamo inoltre rilevato il
posizionamento dei nidi di cormorano e airone cenerino come
sovrapposizione topografica e come tipo di alberi e altezze, in sei
colonie. L{\textquoteright}arrivo del cormorano sembra aver causato una
graduale diminuzione dell{\textquoteright}airone cenerino nelle colonie
miste, e una graduale segregazione orizzontale e verticale nella
localizzazione dei nidi delle due specie. Queste interferenze si sono
manifestate nonostante l{\textquoteright}apparente diponibilit\`a di
ampie superfici di ambiente idoneo alla nidificazione, e non occupate.
Abbiamo in programma di estendere i rilevamenti a tutte le colonie di
cormorano di Piemonte e Lombardia, al fine di confermate queste
interferenze tra specie coloniali. \\
}



\section*{Introduction}
The mainland breeding population of great cormorant in Italy has
increased rapidly since early 1990s, and the breeding population for
whole Italy in 2013 was estimated at around 4000 occupied nests in 48
colonies (Volponi \& CorMoNet.it, 2013), mainly located in the north
and particularly in Lombardy, where 10 colonies had 1100 breeding pairs
in 2013. Most colonies in Northwestern Italy are located in protected
areas (natural reserves, regional or natural parks, sites of the Natura
2000 network), and frequently in sites that had been already used for a
long time by breeding grey herons \textit{Ardea cinerea} and by other
colonial \textit{Ardeidae}. In NW Italy (Lombardy and Piedmont), 21
cormorant colonies out of 25 are located in heronries, of which 5 with
grey herons only and 16 plurispecific with 2 to 6 species of herons and
egrets, in a landscape strongly urbanized and exploited for
agricultural activities.

Since the availability of sites suitable for colonially breeding water
birds, i.e. sites wetlands safe from human disturbance, is limited
(Fasola \& Alieri, 1992), we predicted possible negative interactions
between great cormorants and grey herons, since these species tend to
use the same forest strata for breeding. In this study we investigate
whether the recent establishment of great cormorant exhibits any
pattern of interference, that could affect the settlement of the herons
and the cormorants in their colony sites. In detail, we aimed to check:
(i) the long-term trend in the number of grey heron nests in the
colonies without vs. those with cormorants; (ii) the spatial overlap of
the nest distribution among species; (iii) the vertical distribution of
the nests.

\section*{Methods}
Data from a long term monitoring program (from 1972 to 2012) of the
breeding colonial \textit{Ardeidae} throughout Northwestern Italy
(performed by the research group {\textquotedblleft}Garzaie
Italia{\textquotedblright}, see Fasola et al., 2011), and from nest
counts of great cormorant in Italy (Volponi \& the Cormorant Colony
Count Group, 2011) were used for time series and cross correlation
analysis, using the R environment (R Core Team, 2013). The time series
analysis was performed on the number of nests in the heronries within
the breeding range of the cormorant in NW Italy (160 sites with grey
herons, and 22 colonies with cormorants, from 1972 to 2012). The time
series data of colonies with both cormorants and grey herons were
aligned before cross correlation analysis, (for each colony the year of
great cormorant establishment was set as year zero).

Between early April and late July 2013, we explored the potential
effects of great cormorant establishment on the spatial distribution of
the herons breeding in six colonies that differed in the year of
settlement of cormorants (from recent establishment to long-time
presence of the species). All trees with nests were located and mapped
using GPS. Each occupied nest on each tree was assigned to the
cormorant, the grey heron or to other \textit{Ardeidae} species, by
identifying the occupants (nestlings or adults), or when they were not
in sight by nest shape and size. The overlap between cormorants and
grey herons in each colony was estimated by producing a 5 m buffer
around each tree occupied by the two species, and by calculating the
surface area of the intersection*s between these buffers for each
species, weighted by the total surface area of the colony. The vertical
distribution of the nests of the two species was investigated, using a
random sample of at least 30 trees in each of 6 colonies. Nest height
and tree height were measured using a hypsometer. Data were analyzed by
ANOVA, testing the effects of species, site and the species/site
interaction.

\section*{Results and discussion}
The time series analysis revealed a marked increase in the total number
of great cormorant nests since first breeding, as well as a continuing
positive trend of the grey heron in the colonies without cormorants. In
contrast to this positive trend, the number of grey herons nesting in
colonies with cormorants showed a sharp decreasing trend (Fig. \ref{Gagliardi_fig_1}). The
shift in the grey heron trend was concomitant with the increase of the
number of cormorant breeding pairs. This is confirmed by the results of
cross-correlation analysis of time series data: the presence of
cormorants affected grey herons (as shown by the cross correlation
coefficient exceeding the significant threshold r = 0.4) four years
after cormorant settled in the same colonies. The decreasing trend of
the grey herons, noticeable since about 2005 even in the colonies
without great cormorants, reflects a recent generalized decline of the
species throughout the entire NW Italy (unpublished results of the
{\textquotedblleft}Garzaie-Italia{\textquotedblright} research group).

The analysis of the spatial overlap in the areas used by cormorants and
by grey herons (Tab. \ref{Gagliardi_tab_1}) showed high overlap in the colonies of recent
cormorant nesting settlement (e.g. the colony
{\textquotedblleft}Brescia centro{\textquotedblright} where cormorants
first bred in 2011, overlap proportion = 61\%,), while lower overlap
was registered in the colonies where cormorants had been present since
long (e.g. {\textquotedblleft}Zerbaglia{\textquotedblright} where
cormorants bred since 2005, overlap proportion = 20\%). A particular
case is represented by the
{\textquotedblleft}Brabbia{\textquotedblright} colony, where there was
no overlap between the two species in 2013, due to the gradual
dislocation of the heronry that started in 2004, after the first
settlement of cormorant in the area. In all the surveyed colonies, the
longer was the time elapsed since the breeding settlement by
cormorants; the lower was the overlap between the nesting areas of the
two species. The relationship between overlap proportion and years
since the initial cormorant breeding settlement can be modeled by a
hyperbolic regression (Overlap = S/Year, t\textsubscript{(5)} = 4.12; p
= 0.0092), with the S parameter (representing the rate of overlap)
estimated as S = 0,330 {\textless} 0,878 {\textless} 1,426. 

The analysis of nest height of the two species, a two-way ANOVA with
{\textquotedblleft}species{\textquotedblright} and
{\textquotedblleft}site{\textquotedblright} as fixed factors and
{\textquotedblleft}nest tree species{\textquotedblright} as random
factor, did not show any random effect of the tree species. Thus, a
fixed factor ANOVA was carried out, which showed that great cormorant
nests were placed significantly higher (average 13.21 {\textpm} 5.60 m,
N = 439) than those of the grey heron (average 10.78 {\textpm} 6.76 m,
N = 346; F\textsubscript{(4,864) }= 121.27, p {\textless} 0.0001).

The {\textquotedblleft}site{\textquotedblright} variable also has an
effect, i.e. nests of the two species were placed at different heights
on a per-site basis (F\textsubscript{(6,864) }= 460.42, p {\textless}
0.0001). Time elapsed since breeding cormorant arrival seems again to
affect the vertical distribution of the two species (species/site
interaction F\textsubscript{(13,864) }= 15.53, p {\textless} 0.0001),
as resulted for the spatial overlap. In colonies with recent breeding
cormorant settlement, the great cormorant nests were higher than those
of the grey heron, and the differences in nest height between the two
species seem to decrease in the colonies where the two species have
coexisted for a longer period (Fig. \ref{Gagliardi_fig_2}).

\begin{figure}[!h]
\centering
\includegraphics[width=.8\columnwidth]{Gagliardi_fig_1.png}
\caption{Trends 1992-2012 in the total number of nests of great cormorant (squares) and of grey heron in the heronries without cormorants (triangles) and in those with cormorants (dots) within NW Italy. Continuous line: LOESS smoothing (5 year moving window local regression)}
\label{Gagliardi_fig_1}
\end{figure}

\begin{figure}[!h]
\centering
\includegraphics[width=.98\columnwidth]{Gagliardi_fig_2.png}
\caption{Box plots of nest height of Grey Heron and Cormorant, arranged from left to right in order of recent (colonies to the left of the graph) or longer (colonies to the right) colonization by the Great Cormorant}
\label{Gagliardi_fig_2}
\end{figure}

\begin{table}[!h]
\centering
\begin{tabular}{>{\raggedright\arraybackslash}p{.3\columnwidth}>{\raggedright\arraybackslash}p{.25\columnwidth}>{\raggedright\arraybackslash}p{.15\columnwidth}>{\raggedright\arraybackslash}p{.15\columnwidth}}
\toprule
\textbf{Colony name} & \textbf{Coordinates} & \textbf{Proportion of spatial overlap} & \textbf{Year of settlement of the great cormorant} \\
\toprule 
%\showrowcolors
Brescia centro (BS) & 45.507 N, 10.238 E & 61 & 2011 \\
Carpiano (MI) & 45.326 N, 9.243 E & 20 & 2010 \\
Villanterio (PV) & 45.213 N, 9.354 E & 0 & 2009 \\
Zerbaglia (LO) & 45.271 N, 9.642 E & 20 & 2006 \\
Zelata (PV) & 45.242 N, 9.003 E & 11 & 2005 \\
Brabbia (VA) & 45.776 N, 8.713 E & 0 & 2004 \\
\hiderowcolors
\bottomrule
\end{tabular}
\caption{Spatial overlap between the great cormorant and the grey heron, in the surface area occupied by the two species in mixed colonies}
\label{Gagliardi_tab_1}
\end{table}

\section*{Conclusions}


In conclusion, in the six sampled colonies, the arrival of the great
cormorant seems to have two effects on co-breeding grey herons: a
gradual diminution of the number of grey heron nests and a gradual
spatial and altitudinal segregation between the two species inside the
same colony. The Italian breeding population of the great cormorant is
still now only a small fraction of the overall European one (less than
1\%), well under the natural carrying capacity, and if its population
will further increase, we can expect stronger effects on the
distribution and number of breeding grey herons. The interference
between these two species occurs despite an apparent abundance of the
available breeding sites, since the surface area of the colony is only
a small fraction of the apparently suitable woodland, at least in some
colonies e.g. {\textquotedblleft} the
{\textquotedblleft}Zelata{\textquotedblright} colony. We plan to extend
our survey to all the heronries with cormorant in the study area, in
order to confirm these interference patterns.



\section*{Acknowledgements}
The authors would like to thank all the collaborators to the long term
monitoring programs, the {\textquotedblleft}Cormorant Colony
Count{\textquotedblright} and the {\textquotedblleft}Garzaie
Italia{\textquotedblright} research groups.

\section*{Bibliography}
\begin{itemize}\itemsep0pt

	\item Fasola M. \& Alieri R., 1992 - Conservation of heronry sites in North
Italian agricultural landscapes. \textit{Biological Conservation,} 62:
219-228.

	\item Fasola M., Merli E., Boncompagni E. \& Rampa A., 2011 - Monitoring heron
populations in Italy, 1972-2010. \textit{Journal of Heron Biology and
Conservation,} 1 (8): 1-10. \url{www.heronhonservation.org/vol1/art8}

	\item R Core Team, 2013 - R: A language and environment for statistical
computing. R Foundation for Statistical Computing, Vienna, Austria. \\ \url{http://www.R-project.org}

	\item Volponi S. \& CorMoNet.it, 2013 - Status of the breeding population of
Great Cormorants in Italy in 2012. -- In: Bregnballe T., Lynch J.,
Parz-Gollner R., Marion L., Volponi S., Paquet J-Y. \& van Eerden M.R.
(eds.) 2013. \textit{National reports from the 2012 breeding census of
Great Cormorants Phalacrocorax carbo in parts of the Western
Palearctic}. IUCN-Wetlands International Cormorant Research Group
Report. Technical Report from DCE -- Danish Centre for Environment and
Energy, Aarhus University. No. 22: 59-64.
\end{itemize}
\end{otherlanguage}