\setcounter{figure}{0}
\setcounter{table}{0}

\begin{adjustwidth}{-3.5cm}{0cm}
\pagestyle{CIOpage}
\authortoc{\textsc{Aluigi A.}, \textsc{Fasano G. S.}}
\chapter*[Valore ornitologico del Parco del Beigua e della ZPS Beigua-Turchino]{Valore ornitologico delle principali tipologie ambientali nel Parco del Beigua e nella ZPS Beigua-Turchino (GE-SV)}
\addcontentsline{toc}{chapter}{Valore ornitologico del Parco del Beigua e della ZPS Beigua-Turchino}

\textsc{Antonio Aluigi}$^{1}$, \textsc{Sergio G. Fasano}$^{1*}$ \\

\index{Aluigi Antonio} \index{Fasano G. Sergio}
\noindent\color{MUSEBLUE}\rule{27cm}{2pt}
\vspace{1cm}
\end{adjustwidth}


\marginnote{\raggedright $^1$Ente Parco del Beigua {}- Via Marconi 165, 16011
Arenzano GE - E-mail: biodiv@parcobeigua.it \\
\vspace{.5cm}
{\emph{\small $^*$Autore per la corrispondenza: \href{mailto:fasanosg@gmail.com}{fasanosg@gmail.com}}} \\
\keywords
{Parco del Beigua, diversit\`a ornitica, conservazione}
{Beigua Natural Park, birds diversity, conservation}
%\index{keywords}{Parco del Beigua} \index{keywords}{Diversit\`a ornitica} \index{keywords}{Conservazione}
}
{\small
\noindent \textsc{\color{MUSEBLUE} Summary } / A bird monitoring project~has been carried out~in the Beigua Natural 
Park between 2006-2012 as part of an integrated system of surveys. We
compared breeding bird communities detected in 1170 point counts~~to
identify the most important habitats evaluating specie richness, specie
diversity, ornithological value (Brichetti \& Gariboldi 1992), number
of Annex 1 species of  {\textquotedblleft}Birds
Directive{\textquotedblright}, and number of SPEC species. Our results
allowed us to identify pastures, grasslands, mediterranean bush, bare
rock and heterogeneous agricultural areas as the most important
habitats from a conservational perspective.
}
\vspace{1cm}

A partire dal 2006 l{\textquoteright}Ente Parco del Beigua ha attuato,
nell{\textquoteright}area protetta e nella connessa ZPS IT1331578
Beigua -- Turchino (che complessivamente occupano una superficie di
circa 145 chilometri quadrati), un dettagliato piano di monitoraggio
dell{\textquoteright}avifauna (Fasano \& Aluigi 2007, 2011; Fasano
\textit{et al. }2009, 2013), condotto applicando la metodica dei punti
d{\textquoteright}ascolto (Blondel \textit{et al}. 1981). 

Durante le attivit\`a di campo, al fine di poter mettere in relazione la
presenza e l{\textquoteright}abbondanza delle specie con le
caratteristiche ambientali, e quindi definire settori prioritari dal
punto di vista conservazionistico e gestionale, \`e stata stimata nel
raggio di 100 metri dal punto di rilevamento la copertura percentuale
delle categorie CORINE Land Cover di terzo livello. Nel corso dei
rilevamenti,  in 1170 punti d{\textquoteright}ascolto effettuati negli
anni 2006-2012, sono state rilevate 26 variabili ambientali, che
successive analisi di agglomerazione (matrice di somiglianza ricavata
mediante il calcolo della distanza euclidea; fusione delle entit\`a
secondo il metodo del legame medio tra gruppi) hanno associato in otto
cluster distinti, riconducibili ad ambienti dominati da: mosaici
agrari, boschi di latifoglie, boschi di conifere, boschi misti,
praterie, brughiere e cespuglieti, macchia mediterranea ed aree
rocciose. 

Tra queste otto tipologie ambientali sono state rilevate differenze
significative nel numero medio di specie per punto
d{\textquoteright}ascolto (F\textsubscript{7,1145} = 21,047, P
{\textless} 0,001) e nell{\textquoteright}indice di diversit\`a di
Shannon (F\textsubscript{7,1145} = 20,782, P {\textless} 0,001;
MacArthur 1965), i cui valori massimi sono stati osservati nei mosaici
agrari. In tale classe si riscontra anche il massimo valore
dell{\textquoteright}indice di equiripartizione. Prendendo in
considerazione il valore ornitologico nazionale delle specie
nidificanti calcolato da Brichetti \& Gariboldi (1992), e calcolando il
valore medio per punto d{\textquoteright}ascolto, abbiamo riscontrato,
tra le variabili ambientali esaminate, differenze statisticamente
significative per quanto riguarda il valore nazionale complessivo
(F\textsubscript{7,1145} = 16,211, P {\textless} 0,001), il valore
nazionale medio (F\textsubscript{7,1145} = 55,593, P {\textless} 0,001)
e il valore nazionale corretto dall{\textquoteright}abbondanza
specifica (F\textsubscript{7,1145} = 58,540, P {\textless} 0,001). I
valori nazionali medi e corretti dall{\textquoteright}abbondanza
specifica sono risultati nettamente pi\`u elevati per la macchia
mediterranea e gli ambienti rocciosi. Considerando infine il numero di
specie incluse nell{\textquoteright}All. 1 della Direttiva
{\textquotedblleft}Uccelli{\textquotedblright} (2009/147/CE) o la cui
conservazione risulti di particolare importanza per
l{\textquoteright}Europa (SPEC 2 e 3 secondo BirdLife International,
2004), si \`e evidenziato come la maggior rilevanza sia da attribuire
alle praterie e, a seguire, a brughiere e cespuglieti e boschi misti.
Tra i diversi ambienti \`e stata rilevata una differenza significativa
della presenza media per punto di ascolto sia di specie di All. 1 della
Direttiva {\textquotedblleft}Uccelli{\textquotedblright}
(F\textsubscript{7,1145} = 34,942, P {\textless} 0,001), sia di specie
di importanza europea (SPEC 2: F\textsubscript{7,1145} = 26,233, P
{\textless} 0,001; SPEC 3: F\textsubscript{7,1145} = 70,565, P
{\textless} 0,001). In particolare nella macchia mediterranea e nelle
praterie la presenza e l{\textquoteright}abbondanza di specie di All. 1
\`e risultata significativamente maggiore (test di Tukey, P {\textless}
0,05).

\section*{Bibliografia}
\begin{itemize}\itemsep0pt
	\item BirdLife International, 2004 - Birds in Europe: population
estimates, trends and conservation status. \textit{Cambridge UK}:
BirdLife International. BirdLife Conservation Series No. 12. 

	\item Blondel J., Ferry C. \& Frochot B., 1981 - Point Counts with Unlimited
distance. In: Estimating Numbers of terrestrial birds. \textit{Studies
in Avian Ecologies}, 6: 414-420. 

	\item Brichetti P. \& Gariboldi A., 1992 - Un
{\guillemotleft}valore{\guillemotright} per le specie ornitiche
nidificanti in Italia. \textit{Riv. ital. Orn.}, 62:73-87. 

	\item Fasano S. \& Aluigi A., 2007 - Dati preliminari sulla densit\`a
riproduttiva di Calandro \textit{Anthus campestris }e Magnanina comune
\textit{Sylvia undata} nel Parco del Beigua e nella ZPS
{\textquotedblleft}Beigua-Turchino{\textquotedblright} (GE-SV).
Abstract del XIV Convegno Italiano di Ornitologia. Trieste 26-30
settembre 2007: 47. 

	\item Fasano S.G. \& Aluigi A., 2011 - Variazioni interannuali ed
interstagionali nella densit\`a della magnanina comune \textit{Sylvia
undata} nel Parco del Beigua e nella ZPS Beigua-Turchino
(GE-SV) - Abstract del XVI Convegno Italiano di Ornitologia. Cervia -- Milano Marittima (Ravenna) 21-25 settembre 2011: 81-82. 

	\item Fasano S., Baghino L. \& Aluigi A., 2009 - La
{\textquotedblleft}Canellona{\textquotedblright}: un \textit{hot-spot}
per l{\textquoteright}Averla piccola. (SIC IT1331402). Atti del XV
Convegno Italiano di Ornitologia. Parco Nazionale del Circeo, Sabaudia
(Latina) 14-18 ottobre 2009. \textit{Alula} XVI (1-2): 544-546.

	\item Fasano S.G., Cottalasso R., Campora M., Baghino L., Toffoli R. \& Aluigi
A. (a cura di), 2013 - \textit{Ambienti e Specie del Parco del Beigua e
dei Siti della Rete Natura 2000 funzionalmente connessi.} Ente Parco
del Beigua, 100 pp. 

	\item MacArthur R.H., 1965 - Patterns of species diversity. \textit{Biol. Rev.} 40:510-533. 
\end{itemize}
