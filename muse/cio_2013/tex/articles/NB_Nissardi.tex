\setcounter{figure}{0}
\setcounter{table}{0}

\begin{adjustwidth}{-3.5cm}{0cm}
\pagestyle{CIOpage}
\authortoc{\textsc{Nissardi S.}, \textsc{Zucca C.},
\textsc{Cherchi F.}, \textsc{Atzori J.}}
\chapter*[Density and home range of purple gallinule]{
Densit\`a e \textit{home range} di pollo sultano \textit{Porphyrio porphyrio} in un{\textquoteright}area protetta della Sardegna meridionale (Parco Naturale Regionale Molentargius Saline)}
\addcontentsline{toc}{chapter}{Density and home range of purple gallinule}

\textsc{Sergio Nissardi}$^{1*}$, \textsc{Carla Zucca}$^{1*}$,
\textsc{Fabio Cherchi}$^{2}$, \textsc{Jessica Atzori}$^{2}$  \\

\index{Nissardi Sergio} \index{Rotelli Luca} \index{Fabio Cherchi} \index{Jessica Atzori}
\noindent\color{MUSEBLUE}\rule{27cm}{2pt}
\vspace{1cm}
\end{adjustwidth}


\marginnote{\raggedright $^1$Anthus snc, Via Luigi Canepa 3, 09129 Cagliari
(Italy), Fax 070496956 \\
$^2$Via del Trifoglio n. 10, 09045 Quartu S.E. (Italy) \\
\vspace{.5cm}
{\emph{\small $^*$Autore per la corrispondenza: \href{mailto:anthus@anthus.info}{anthus@anthus.info}}} \\
\keywords{Parco naturale regionale Molentargius Saline, Sardegna,
\textit{Porphyrio porphyrio}, densit\`a, \textit{home range}}
{Molentargius Saline natural regional Park, Sardinia, \textit{Porphyrio porphyrio}, density, home range}
%\index{keywords}{\textit{Porphyrio porphyrio}} \index{keywords}{Densit\'a} \index{keywords}{\textit{Home range}}
}
{\small
\noindent \textsc{\color{MUSEBLUE} Summary} / The aim of the study was to evaluate the density, abundance and home
ranges of purple gallinule \textit{Porphyrio porphyrio} in Molentargius
natural regional Park.
The overall abundance was estimated at 55-75 (through line transects and
point counts), of which 30-40 in Ecosistema Filtro (a portion of 37
ha), where the density was evaluated at 0.8-1.1 pairs / ha.
Through banding with color rings, research of the banded birds by visual
census and radio-tracking we evaluated that the home range can be less
than one hectare, suggesting a strong sedentariness, at least for
adults, and confirming the density values obtained from the censuses.\\
}
\vspace{1cm}

{
Lo studio, svolto per conto del Parco naturale regionale Molentargius
Saline (Sardegna), \`e stato finalizzato ad approfondire le conoscenze
su densit\`a e comportamento territoriale del pollo sultano
\textit{Porphyrio porphyrio}, all{\textquoteright}interno di questa
zona umida che comprende 150 ha di bacini ad acque dolci, fra cui un
impianto di fitodepurazione di 37 ha (Ecosistema Filtro) nel quale si
sono svolte le attivit\`a di cattura e marcatura degli animali. Data la
particolare etologia di questa specie (si riproduce praticamente per
tutto l{\textquoteright}anno, con picchi di deposizioni a
dicembre-gennaio e marzo-maggio
(\textcolor[rgb]{0.0,0.0,0.039215688}{Schenk 1993; Grussu 1999)}, lo
studio ha interessato un{\textquoteright}intera annualit\`a (da ottobre
2010 a settembre 2011), al fine di valutare eventuali variazioni
intra-annuali nella densit\`a (o nella contattabilit\`a) o nel
comportamento territoriale. Lo studio si \`e svolto grazie a: 1)
censimenti periodici mediante transetti lineari e punti di
osservazione/ascolto; 2) cattura e inanellamento (marcando gli animali
anche con anello colorato per consentirne il riconoscimento mediante
osservazione a distanza); 3) campagne di lettura per localizzare gli
individui marcati; 4) radio-\textit{tracking}.}

{
I censimenti hanno evidenziato una stima prudenziale di 55-75
coppie/nuclei familiari in tutto il sistema di Molentargius, di cui
30-40 nel solo Ecosistema Filtro (densit\`a pari a 0,8-1,1 coppie
ha\textsuperscript{{}-1} di superficie, inclusi gli specchi
d{\textquoteright}acqua). Il valore cos\`i ottenuto ammonta a oltre il
10\% del totale di 450-600 coppie riportato da Andreotti (2001) per la
popolazione sarda. \`E stato catturato e marcato un campione di 7
individui, di cui 6 equipaggiati con trasmittente. La ricerca di
individui inanellati ha avuto modesti risultati in rapporto allo sforzo
di rilevamento, dato che nel periodo di studio sono stati identificati
2 soli individui, di cui uno era stato inanellato precedentemente, il
6/09/2007, ed \`e stato osservato il 3/05/2011, quindi con un
intervallo fra la marcatura e la rilettura di 1335 giorni. Il secondo,
catturato e inanellato il 20 febbraio 2011, \`e stato osservato il
21/05/2011, con un intervallo fra la marcatura e la rilettura di 90
giorni. Un terzo individuo, inanellato il 12 settembre 2009, \`e stato
osservato, dopo la conclusione del monitoraggio, il 9 giugno 2013 (1366
giorni dopo la cattura).}

{
Per quanto concerne lo studio degli spostamenti individuali, dei 6
individui a cui \`e stata applicata la radio, solo 3 hanno
l{\textquoteright}hanno tenuta per un tempo superiore ai 3 giorni:
l{\textquoteright}individuo marcato con anello AC, catturato una prima
volta il 30 dicembre 2010, ha perso la radio dopo 2 giorni; ripreso il
primo ottobre 2011 \`e stato monitorato per 85 giorni, quindi fino a
359 giorni dalla prima cattura, sempre all{\textquoteright}interno di
una superficie valutabile in 0,17-0,71 ha (intervallo ottenuto
considerando un possibile errore di circa 15 m nella localizzazione
mediante radio-\textit{tracking}); l{\textquoteright}individuo AD
marcato il 9 gennaio \`e stato seguito per 35 giorni con spostamenti in
un{\textquoteright}area di 0,24-0,81 ha; l{\textquoteright}individuo
AF, marcato il 20 febbraio, \`e stato seguito per 30 giorni fino alla
perdita della radio e riosservato dopo 90 giorni dalla prima cattura,
si \`e mosso in un{\textquoteright}area di 0,29-0,69 ha. \`E da
rilevare inoltre che gli individui AC e AD occupano territori in gran
parte sovrapposti, la cui superficie complessiva \`e valutabile in
circa un ettaro. Questo dato potrebbe indicare che si tratti di una
coppia territoriale, essendo AC certamente maschio e AD probabilmente
femmina, in quanto decisamente pi\`u piccolo, anche se con misure non
discriminanti. Questi risultati sembrano confermare i valori di
densit\`a ottenuti attraverso i censimenti ed evidenziano un
comportamento sostanzialmente sedentario, almeno degli adulti (tutti
gli individui monitorati per un periodo significativo erano adulti) che
si muovono entro territori ristretti, valutabili in meno di un ettaro,
che possono essere frequentati dai medesimi individui durante
l{\textquoteright}intero ciclo annuale.}


\section*{Bibliografia}

\begin{itemize}\itemsep0pt
	\item Andreotti A. (a cura di), 2001 - Piano d{\textquoteright}azione
nazionale per il Pollo sultano (\textit{Porphyrio porphyrio}).
Quad. Cons. Natura, 8, Min. Ambiente - Ist. Naz. Fauna
Selvatica.

	\item Grussu M., 1999 - Status and breeding ecology of the Purple Swamp-hen in
Italy. British Birds 4 (92): 183-192.

	\item Schenk H., 1993 - Pollo sultano. In: Meschini E., Frugis S.(eds),
Atlante degli uccelli nidificanti in Italia. Suppl. Ric. Biol.
Selvaggina, XX: 107.

\end{itemize}
