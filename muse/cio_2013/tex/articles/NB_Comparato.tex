\setcounter{figure}{0}
\setcounter{table}{0}

\begin{adjustwidth}{-3.5cm}{0cm}
\pagestyle{CIOpage}
\authortoc{\textsc{Comparato L.}, \textsc{Caprio E.},
\textsc{Boano G.}, \textsc{Rolando A.}}
\chapter*[Vocazionalit\`a ambientale della starna in provincia di
Asti]{Analisi di vocazionalit\`a ambientale della starna
\textbf{\textit{Perdix perdix}}\textbf{ nella Z.R.C.
{\textquotedbl}Casalino{\textquotedbl} in provincia di Asti}}
\addcontentsline{toc}{chapter}{Vocazionalit\`a ambientale della starna in provincia di
Asti}

\textsc{Laura Comparato}$^{1*}$, \textsc{Enrico Caprio}$^{1,2}$,
\textsc{Giovanni Boano}$^{3}$, \textsc{Antonio Rolando}$^{1}$ \\

\index{Comparato Laura} \index{Caprio Enrico} \index{Boano Giovanni} \index{Rolando Antonio}
\noindent\color{MUSEBLUE}\rule{27cm}{2pt}
\vspace{1cm}
\end{adjustwidth}



\marginnote{\raggedright $^1$Dipartimento di Scienze della Vita e Biologia dei
Sistemi, Universit\`a di Torino \\
$^2$Scuola di Biodiversit\`a di Villa Paolina, c/o
Consorzio Asti Studi Superiori - Piazzale F. De
Andre{\textquoteright}, 14100 Asti, Italy \\
$^3$Museo Civico di Storia Naturale di Carmagnola (TO) \\
\vspace{.5cm}
{\emph{\small $^*$Autore per la corrispondenza: \href{mailto:laura.comparato@tiscali.it}{lau\allowbreak ra.\allowbreak com\allowbreak pa\allowbreak ra\allowbreak to@\allowbreak tis\allowbreak ca\allowbreak li.\allowbreak it}}} \\
\keywords{\textit{Perdix perdix},provincia di Asti,
vocazionalit\`a ambientale, agroecosistemi}
{Asti province, \textit{Perdix perdix}, habitat suitability, agro-ecosystems}
%\index{keywords}{\textit{Perdix perdix}} \index{keywords}{Provincia di Asti} \index{keywords}{Vocazionalit\`a ambientale} \index{keywords}{Agroecosistemi}
}
{\small
\noindent \textsc{\color{MUSEBLUE} Summary} / This is a small-scale ecological study that evaluates the habitat
suitability for the grey partridge \textit{Perdix perdix} of an area of
historical presence. The results don't show a
reduction of suitable areas for the species. Therefore the causes of
this absence are due to rainy and snowy precipitations and/or
predation.
}



\section*{Introduzione}

La starna \textit{Perdix perdix} \`e una specie tipica di ambienti
coltivati che ha subito, in tutta Europa, un netto declino nella sua
distribuzione negli anni 1970-1990, imputabile a un crollo della
diversit\`a degli agro-ecososistemi, principalmente cerealicoli,
parallelamente a una mancata gestione degli ambienti.

Le modificazioni
dell{\textquoteright}agricoltura hanno determinato severi impatti sulla
biodiversit\`a, sopratutto sulle specie particolarmente legate agli
agro-ecosistemi e pertanto ritenute dei buoni indicatori (Donald
\textit{\textcolor[rgb]{0.0,0.0,0.039215688}{et
al}}\textcolor[rgb]{0.0,0.0,0.039215688}{. 2001).}

Scopo di questo studio \`e valutare modelli di vocazionalit\`a
ambientale per la starna a scala locale, in un'area di
presenza storica della specie, e identificare possibili minacce per la
sua conservazione. 

\section*{Metodi}
L'area di studio \`e compresa
all'interno della Zona di Ripopolamento e Cattura
(ZRC) {\textquotedbl}Casalino{\textquotedbl} (856 ha), territorio
collinare poco acclive, 140-261 m s.l.m., prevalentemente agricolo e
con bassa densit\`a di popolazione. La starna \`e una specie protetta
nelle province di Asti e Alessandria, interdetta alla caccia da met\`a
degli anni '90. 
{La presenza della starna \`e stata valutata tramite
conteggi pre-riproduttivi con l'Indice Chilometrico di
Abbondanza e uno post-riproduttivo, con l'ausilio di
cani da ferma, per l'anno 2012. Inoltre sono state
monitorate, con la tecnica del mappaggio, anche altre quattro specie
indicatrici di agro-ecosistemi (}\textit{{Emberiza
hortulana}}{, }\textit{{Miliaria
calandra}}{, }\textit{{Streptopelia
turtur }}{e }\textit{{Coturnix
coturnix}}{). I modelli di idoneit\`a ambientale sono
stati formulati utilizzando il software MaxEnt 3.1 (Phillips
}\textit{{et al.}}{ 2006) integrando
ai dati di presenza i monitoraggi della confinante Z.R.C.
{\textquotedbl}Val Cerrina{\textquotedbl} (AL), di cui si} ha a
disposizione una serie storica di monitoraggi effettuati tra il 2002 e
il 2009 dall'Osservatorio Faunistico della Provincia
di Alessandria{. }

L'approccio della massima entropia permette di
utilizzare le informazioni ambientali derivanti dalle interazioni con i
punti di presenza certa della specie, per poi generalizzarle
all'intero territorio indagato (Phillips \textit{et}
\textit{al}. 2006). 

Sono stati cos\`i identificati i fattori ambientali pi\`u importanti nel
determinare la presenza della specie, \`e stata generata una mappa di
distribuzione potenziale e idoneit\`a ambientale, per
l'intero territorio.

{Le variabili ambientali considerate sono state: uso
del suolo, ottenuto dai piani forestali territoriali della Regione
Piemonte (Regione Piemonte 2009), modello digitale del terreno,
pendenza ed esposizione dell'area di studio. Le mappe
di uso del suolo sono state aggiornate nel 2012 utilizzando foto aeree
recenti per mappare elementi lineari quali siepi o piccoli boschi. Le
categorie di uso del suolo sono state: boschi di latifoglie,
seminativi, siepi, prato-pascoli, torrenti e zone umide, boscaglie di
invasione, vigneti e urbanizzato, il tutto rasterizzato con immagini
con pixel 10x10m.  }

Il modello \`e stato fatto con i dati del monitoraggio per
l'anno 2004 in cui si ha la densit\`a massima della
starna e proiettato con le mappe di uso del suolo aggiornate al 2012.

I modelli sono stati selezionati sulla base
dell'analisi: i) delle curve di ROC che permettono di
determinare il limite soglia, valore al disopra del quale si ha la
massima idoneit\`a ambientale per la specie e ii)
dell'AUC che definisce la capacit\`a predittiva del
modello: il modello pu\`o considerarsi efficiente se il valore di AUC
supera lo 0,8 (Menel \textit{et al}. 2001).

\section*{Risultati e discussione}

Dai monitoraggi effettuati, possiamo concludere che la starna non \`e
pi\`u presente nell'area di studio, poich\'e non \`e
stato osservato alcun individuo n\'e trovato alcun segno di presenza
per l'anno 2012. 

I modelli ottenuti con MaxEnt per il 2004 e la proiezione degli stessi
sul 2012 mostrano un buon livello di efficienza con un AUC
{\textgreater} 0,8.

Punto di forza del presente lavoro \`e aver utilizzato i dati di
presenza delle quattro specie indicatrici di agro-ecosistemi (che
condividono l{\textquoteright}habitat della starna), monitorate nel
2012, per confrontare il modello di idoneit\`a
dell{\textquoteright}area di studio con quello derivante da specie
indicatrici di agroecosistemi, potenzialmente estendibile anche alla
starna, al fine di verificare lo stato attuale del territorio al fine
di verificare l{\textquoteright}idoneit\`a del territorio a ospitare la
specie; il valore AUC anche in questo caso \`e risultato maggiore di
{\textgreater} 0,8. Le variabili ambientali che hanno contribuito
positivamente alla costruzione di tutti i modelli sono risultate quelle
proprie di ambienti agricoli (seminativi, frutteti - vigneti ed
elementi lineari), le aree boscate sono invece risultate non idonee.
Tale risultato trova conferma in uno studio di Meriggi \textit{et al}.
(1991) che descrive come la starna selezioni gli elementi lineari
durante tutte le stagioni, ma con percentuale di utilizzo differente
secondo le stagioni, maggiore in inverno; le aree boscate sono invece
evitate tutto l'anno. Confrontando il modello del 2012
e quello del 2004 la superficie di territorio idoneo a ospitare la
specie non varia (314,69 ha per il 2004 e 314,31 ha per il 2012).  La
causa della riduzione della specie non sembra dunque imputabile a
un'alterazione nella gestione degli ambienti agricoli
poich\'e il territorio non ha subito una modificazione di utilizzo
negli ultimi trenta anni ed \`e sempre riconducibile a un ecosistema
rurale di tipo cerealicolo (fonte ISTAT).  

Le cause responsabili dell'assenza della starna in
quest{\textquoteright}area sono dunque imputabili ad altri fattori
quali l'andamento delle precipitazioni piovose
estivo-primaverili e nevose che possono influenzare negativamente le
dinamiche di popolazioni. L'andamento delle piogge
negli ultimi anni, tra giugno e luglio, potrebbe aver ridotto il
successo della covata e la disponibilit\`a di insetti nel periodo
estivo, fondamentali per le prime fasi vitali dei pulli.
L'inverno 2012 si \`e inoltre caratterizzato per
un'abbondante nevicata con temperature medie minime in
febbraio al disotto dei -10{\textdegree}C (picco -18.7 {\textdegree}C)
e coltre nevosa superiore ai 40 cm (valore neve cumulata 78 cm).
L'ostacolo non \`e la nevicata di per s\'e bens\`i il
gelo dei giorni successivi che pu\`o causare la morte per inedia data
l'incapacit\`a di raggiungere le risorse trofiche
(Cocchi \textit{et al.} 1993). Dal momento che il prelievo venatorio
per la specie non \`e previsto nelle due province limitrofe, tra i
fattori pi\`u importanti di limitazione vi pu\`o essere inoltre la
forte pressione predatoria esercitata in particolar modo dalla volpe
\textit{Vulpes vulpes} e dal gatto domestico.  

\section*{Ringraziamenti}

Ringraziamo l{\textquoteright}area Agricoltura della provincia di Asti
in particolare l{\textquoteright}Ufficio caccia, pesca e tartufi. Si
ringraziano inoltre l{\textquoteright}Osservatorio faunistico della
provincia di Alessandria per aver fornito i dati e tutte le persone che
hanno aiutato durante l'attivit\`a di campo

\section*{Bibliografia}
\begin{itemize}
	\item Cocchi R., Govoni M. \& Toso S., 1993 - La starna. \textit{Istituto
Nazionale per la Fauna Selvatica, Documenti Tecnici}, 14.

	\item Donald P. F., Green R. E. \& Heath M. F., 2001 - Agricultural
intensification and the collapse of Europe{\textquoteright}s farmland
bird populations. \textit{Proc. Roy. Soc}., Lond. B (268): 25--29.

	\item Manel S., Williams H.C. \& Ormerod S.J., 2001 - \textit{Evaluating
presence-absence models in ecology: the need to account for
prevalence}. \textit{Journal of Applied Ecology,} (38): 921-931.

	\item Meriggi A., Montagna D. \& Zacchetti D., 1991 - Habitat use by
partridges (\textit{Perdix perdix} and \textit{Alectoris rufa}) in an
area of northern Apennines, Italy. \textit{Bolletino di zoologia,}
(58): 85-90.

	\item Phillips S.J., Anderson R.P. \& Schapire R.E., 2006 - Maximum entropy
modeling of species geographic distributions. \textit{Ecological
Modeling,} (190): 231-259.
\end{itemize}
