\setcounter{figure}{0}
\setcounter{table}{0}

\begin{adjustwidth}{-3.5cm}{0cm}
\pagestyle{CIOpage}
\authortoc{\textsc{Termine R.}, \textsc{Massa B.}}
\chapter*[Nidificazione di svasso piccolo]{Nidificazione di svasso piccolo \textbf{\textit{Podiceps
nigricollis}}\textbf{ C. L. Brehm, 1831 al lago di Pergusa (Enna)}}
\addcontentsline{toc}{chapter}{Nidificazione di svasso piccolo}

\textsc{Rosa Termine}$^{1*}$, \textsc{Bruno Massa}$^{2}$ \\

\index{Termine Rosa} \index{Massa Bruno}
\noindent\color{MUSEBLUE}\rule{27cm}{2pt}
\vspace{1cm}
\end{adjustwidth}



\marginnote{\raggedright $^1$Laboratorio di Ingegneria Sanitaria Ambientale,
Universit\`a di Enna {\textquotedblleft}Kore{\textquotedblright},
Cittadella Universitaria, 94100 Enna, Italia  \\
$^2$Dipartimento di Scienze agrarie e forestali,
Universit\`a di Palermo, V.le delle Scienze, 90128 Palermo, Italia \\
\vspace{.5cm}
{\emph{\small $^*$Autore per la corrispondenza: \href{mailto:rosa.termine@unikore.it}{rosa.termine@unikore.it}}} \\
\keywords{\textit{Podiceps nigricollis}, nidificazione, lago
Pergusa, Sicilia, Italia}
{\textit{Podiceps nigricollis}, nesting, Pergusa lake,
Sicily, Italy}
%\index{keywords}{\textit{Podiceps nigricollis}} \index{keywords}{Nidificazione} \index{keywords}{Lago
%Pergusa} \index{keywords}{Sicilia}
}
{\small
\noindent \textsc{\textcolor{MUSEBLUE}{Summary}} / The black-necked grebe bred again in the Pergusa lake.
In Italy, the black- necked grebe is an irregular breeding bird;
nesting documented cases after 1950 number around twenty, mostly
restricted to a few pairs. In Sicily, apart from the nesting in 1957 in
the Pergusa lake and in 1966 in Scanzano lake, for thirty-four years
this species have been sighted occasionally and irregularly as summer
visitor; from 2000 to 2013 new cases of breeding have been recorded: in
2000 and 2005 in the province of Caltanissetta, in 2004, in 2006 and
2011 in the province of Siracusa, in 2010, 2012 and 2013 in Pergusa
lake (Enna). This lake has so far counted the largest number of
breeding pairs for Sicily and Italy; in 2013 a population of 120 adults
and 108 young has been censused. Over the last three years of breeding
(2010, 2012 and 2013) in Pergusa, a large presence of phanerophytes
\textit{Ruppia} and \textit{Potamogeton} with the
 formation of mats has been observed; it would be
interesting to research whether and how the presence of these mats is a
causal factor of the settling of nesting individuals of this species. \\
\noindent \textsc{\color{MUSEBLUE} Riassunto} / Lo svasso piccolo \textit{Podiceps nigricollis} \`e tornato a nidificare
al lago di Pergusa. In Italia, questa specie \`e nidificante
irregolare; i casi di riproduzione documentati dopo il 1950 sono circa
una ventina, perlopi\`u relativi a poche coppie. In Sicilia, a parte le
nidificazioni del 1957 al lago di Pergusa e del 1966 al lago di
Scanzano, per trentaquattro anni la specie \`e stata avvistata
occasionalmente e in modo irregolare come estivante; dal 2000 al 2013
sono stati documentati nuovi casi di nidificazioni: nel 2000 e nel 2005
in Provincia di Caltanissetta; nel 2004, nel 2006 e nel 2011 in
Provincia di Siracusa; nel 2010, nel 2012 e nel 2013 al lago di Pergusa
in Provincia di Enna. {Il lago di Pergusa }ha finora
contato il maggior numero di coppie nidificanti della specie per la
Sicilia e per l{\textquoteright}Italia; nell{\textquoteright}estate
2013 la popolazione \`e stata di 120 adulti, che hanno prodotto 108
giovani. Durante questi ultimi tre anni di nidificazioni (2010, 2012 e
2013) a Pergusa \`e stata rilevata una cospicua presenza delle
fanerofite \textit{Ruppia} e \textit{Potamogeton} con formazione di
\textit{mats}; sarebbe interessante indagare se e in che modo, la
presenza di tali \textit{mats} sia stato un fattore causale
dell{\textquoteright}insediamento dei soggetti nidificanti di questa
specie.
}




\section*{Introduzione}
La Sicilia, fino al 1800, era ricca di aree umide la cui distruzione
risale all{\textquoteright}ultimo secolo, passando da circa 100.000
ettari nel 1865 ai 47.174 dell{\textquoteright}inizio degli anni
{\textquoteleft}30 (Rallo \& Pandolfi 1988) fino ai soli 5000 ettari
attuali (Lo Valvo \textit{et al}. 1993); ci\`o ha causato
l{\textquoteright}estinzione locale di diverse specie di uccelli
acquatici. 

Il lago di Pergusa \`e uno dei pochi laghi naturali della Sicilia; esso
rappresenta una delle pi\`u importanti aree siciliane per la tutela
degli uccelli stanziali, svernanti e migratori, oltre a
rivestire{ un ruolo importante per} la nidificazione
di alcuni di loro.

In Italia, lo svasso piccolo \`e nidificante irregolare; i casi di
riproduzione documentati dopo il 1950, escludendo quelli possibili o
probabili, sono circa una ventina, perlopi\`u relativi a poche coppie
(Brichetti \& Fracasso 2013). In Sicilia tale specie era considerata
sedentaria e nidificante regolare da Benoit (1840) e Doderlein (1873).
Nel 1958 Krampitz segnal\`o 20-25 coppie nidificanti al lago di
Pergusa. Massa e Schenk (1983) invece lo ritenevano estinto in Sicilia
dal 1965. Successivamente Iapichino e Massa (1989) lo considerarono
nidificante occasionale, dopo la segnalazione del 1966 di una coppia
con 2 giovani al lago di Scanzano (PA). {Per
trentaquattro anni la specie \`e stata} avvistata
nell{\textquoteright}Isola solo con presenze occasionali e in modo
irregolare come estivante. Dal 2000 al 2012 sono stati documentati
alcuni nuovi casi di nidificazioni: nel 2000 sono state avvistate 1-2
coppie presso l{\textquoteright}invaso di Comunelli (CL, Mascara
2007); nel 2004 \`e stata avvistata una coppia con 2 pulli
presso il Pantano Cuba (SR) e una coppia con 4 \textit{juv.} al Pantano
Longarini (SR) (Corso 2005); nel 2005 \`e stata avvistata una coppia al
Comunelli e una coppia nell{\textquoteright}invaso di Cimia (CL, Mascara 2007); nel 2006 \`e stata accertata la nidificazione presso il
Pantano Baronello (SR) e avvistate 2-3 coppie al Pantano Cuba (Corso,
2007); nel 2011 \`e stata osservata una coppia con 2 pulli
presso il Pantano della Riserva Naturale Saline di Priolo (SR) (Di
Blasi, \textit{com. pers. }2011); in altri casi (Pantani di Pachino,
SR) la nidificazione \`e stata ritenuta possibile, ma non accertata. 

\section*{Area di studio}

Il lago di Pergusa, localizzato al centro della Sicilia tra i monti
Erei, ha una quota di 667 metri s.l.m. \`E l{\textquoteright}unico lago
endoreico siciliano;{ di origine tettonica, occupa la
parte pi\`u depressa di una struttura sinclinale pliocenica. La sua
fonte principale di alimentazione \`e rappresentata dalle
precipitazioni e dalle falde freatiche; esso, a causa
dell{\textquoteright}evaporazione estiva, \`e caratterizzato da acque
salmastre.}

{Vari interventi antropici, iniziati negli anni
{\textquoteright}30 con opere di bonifica e accentuati negli anni
{\textquoteright}60 e {\textquoteright}70 con
l{\textquoteright}emungimento di acqua dalle falde, hanno messo a serio
rischio la sua esistenza tanto da determinarne la quasi totale
riduzione dello specchio lacustre nell{\textquoteright}estate 2002.
Oggi, in seguito alla riduzione quasi totale
dell{\textquoteright}emungimento e grazie anche ad alcune stagioni
particolarmente piovose, il lago ha avuto una ripresa notevole,
occupando un{\textquoteright}area} di circa 141 ettari, di cui circa 35
nella cintura esterna sono ricoperti da \textit{Phragmites australis
}(Cav.) Trin.

{
Alla fine degli anni {\textquoteright}50 \`e stato realizzato un
autodromo, che cinge completamente il Lago a stretto contatto delle
sponde; la sua realizzazione ha determinato, oltre che consumo di
territorio, l{\textquoteright}isolamento della fascia riparia dalle
circostanti colline, con il depauperamento della zona ecotonale a causa
della consistente barriera lineare determinata dalla pista e dalle
strutture connesse.}

Il Lago \`e una Riserva Naturale Speciale, istituita dalla Legge
Regionale 71 del 1995 e gestita dalla Provincia Regionale di Enna;
{l{\textquoteright}area protetta ha
un{\textquoteright}estensione totale di 402,5 ha.} Fa anche parte della
Rete Natura 2000 (SIC-ZPS ITA060002 {\textquotedbl}Lago di
Pergusa{\textquotedbl}) e c{ome geosito del
{\textquotedblleft}Rocca di Cerere Geopark{\textquotedblright} rientra
nelle Reti dei Geoparchi Europea (EGN) e Globale (GGN).}
{I vincoli di protezione e la conseguente
regolamentazione delle attivit\`a motoristiche hanno migliorato le
condizioni del delicato ecosistema lacustre, diminuendone notevolmente
lo stress ambientale.}

Pur essendo di limitata estensione, Pergusa ospita una ricca fauna;
negli anni sono state censite 299 specie (Vertebrati e Invertebrati) di
cui 177 specie di uccelli tra nidificanti, svernanti e migratrici, tra
le quali diverse di importanza conservazionistica come
\textit{Porphyrio porphyrio}, \textit{Aythya nyroca},
\textit{Ixobrychus minutus}, \textit{Netta rufina}, \textit{Circus
aeruginosus}, incluse nell{\textquoteright}Allegato I della Dir.
2009/147/CE e nella Lista Rossa Italiana (Termine \textit{et al.},
2008). Numerose sono state le nidificazioni accertate al lago di
Pergusa; tra queste quella di\textit{ P. porphyrio}, di cui nel 2013
sono state censite 29 coppie, e quella di \textit{Himantopus
himantopus}{, }del quale nel 2013 \`e stato
controllato un nido con 4 uova e la nascita di 4 \textit{pulli}.

\section*{Metodi}

Il monitoraggio dello svasso piccolo \`e stato eseguito con
l{\textquotesingle}osservazione diretta sul campo, con
l{\textquoteright}ausilio di binocolo 10x42 e cannocchiale
25-50x80 con cadenza quindicinale e, a volte, anche pi\`u
frequentemente, percorrendo l{\textquoteright}intero perimetro del Lago
con un mezzo natante con motore elettrico per accedere a punti
altrimenti difficilmente osservabili dalle rive.

Nella stagione calda, i rilevamenti generalmente sono stati effettuati
nelle ore del mattino o del tardo pomeriggio, quando maggiore \`e
l{\textquoteright}attivit\`a trofica degli uccelli acquatici. I dati di
campo sono stati poi inseriti in un database.

\newcolumntype{s}{>{\centering\columncolor{white!80!MUSEBLUE}\arraybackslash}p{.06\columnwidth}}
\newcolumntype{M}{>{\centering\columncolor{white!55!MUSEBLUE}\arraybackslash}p{.06\columnwidth}}
\newcolumntype{E}{>{\centering\columncolor{white!30!MUSEBLUE}\arraybackslash}p{.06\columnwidth}}
\newcommand{\gw}{\cellcolor{white}}
\newcommand{\gr}{\cellcolor{white!5!MUSEBLUE}}
\newcommand{\gp}{\cellcolor{white!95!MUSEBLUE}}
\newcommand\crule[3][black]{\textcolor{#1}{\rule{#2}{#3}}}

%\newcolumntype{R}{>{\centering\columncolor{Gray}\arraybackslash}p{.08\columnwidth}}
%\newcolumntype{P}{>{\centering\columncolor{Gray}\arraybackslash}p{.08\columnwidth}}
\begin{table}[!h]
\centering
\scalebox{.8}{
\begin{tabular}{>{\raggedright\arraybackslash}p{.1\columnwidth}ssMMEEEEMMMs}
\hiderowcolors
\toprule
\textbf{Anno} & \gw \rotatebox{90}{\textbf{GEN}} & \gw \rotatebox{90}{\textbf{FEB}} & \gw \rotatebox{90}{\textbf{MAR}} & \gw \rotatebox{90}{\textbf{APR}} & \gw \rotatebox{90}{\textbf{MAG}} & \gw \rotatebox{90}{\textbf{GIU}} & \gw \rotatebox{90}{\textbf{LUG}} & \gw \rotatebox{90}{\textbf{AGO}} & \gw \rotatebox{90}{\textbf{SET}} & \gw \rotatebox{90}{\textbf{OTT}} & \gw \rotatebox{90}{\textbf{NOV}} & \gw \rotatebox{90}{\textbf{DIC}} \\
\toprule
 2004 & 6 & 5 & 10 &  &  &  &  &  &  &  &  &  \\
 \midrule
 2005 &  &  &  &  & 1 & 1 &  & 1 & 1 &  &  &  \\
 \midrule
 2006 &  & 2 &  & 1 & 4 & 2 &  & 3 & 2 & 1 &  &  \\
 \midrule
 2007 & 1 &  &  & 5 & 5 & 10 & 5 &  & 3 &  &  &  \\
 \midrule
 2008 & 4 & 2 & 2 & 2 & 6 &  & 4 & 3 &  &  &  & 4 \\
 \midrule
 2009 & 2 &  &  &  &  &  &  & 6 & 12 & 12 & 8 & 5 \\
 \midrule
 2010 & 4 & 4 &  & \gr & \gr 2 & \gr 6 & \gr 14 & \gp 16 & \gp 15 & \gp 15 & 6 & 2 \\
 \midrule
 2011 & 2 & 4 & 6 & 8 & 8 & 8 & 4 & 2 & 2 &  & 5 & 3 \\
 \midrule
 2012 & 8 & 8 & 15 & \gr 20 & \gr 40 & \gr 50 & \gr 50 & \gp 45 & \gp 46 & \gp 45 & 12 & 4 \\
 \midrule
 2013 & 5 & 5 & 10 & \gr 18 & \gr 41 & \gr 81 & \gr 128 & \gp 124 & \gp 120 & \gw &  \gw &  \gw \\
\bottomrule
\multicolumn{13}{c}{} \\
\multicolumn{13}{c}{\crule[white!80!MUSEBLUE]{.3cm}{.3cm} Svernamento \crule[white!55!MUSEBLUE]{.3cm}{.3cm} Migrazione \crule[white!30!MUSEBLUE]{.3cm}{.3cm} Estivazione \crule[white!5!MUSEBLUE]{.3cm}{.3cm} Riproduzione \crule[white!95!MUSEBLUE]{.3cm}{.3cm} Post-riproduzione} \\
\end{tabular}
}
\caption{Numero max di adulti di svasso piccolo osservati, in relazione alla fenologia annuale, dal 2004 al 2013
}
\label{Termine_tab_1}
\end{table}


\begin{table}[!h]
\small
\scalebox{.75}{
\begin{tabular}{>{\raggedleft\arraybackslash}p{.15\columnwidth}>{\raggedleft\arraybackslash}p{.13\columnwidth}>{\raggedleft\arraybackslash}p{.1\columnwidth}>{\raggedleft\arraybackslash}p{.13\columnwidth}>{\raggedleft\arraybackslash}p{.13\columnwidth}>{\raggedleft\arraybackslash}p{.13\columnwidth}>{\raggedleft\arraybackslash}p{.13\columnwidth}>{\raggedleft\arraybackslash}p{.15\columnwidth}}
\hiderowcolors
\toprule
\textbf{Data} & \textbf{Singoli} & \textbf{Coppie} & \textbf{Tot. Adulti} & \textbf{Pulli} & \textbf{Juv.} & \textbf{Tot. Pulli + Juv.} & \textbf{Os\allowbreak ser\allowbreak va\allowbreak to\allowbreak re$^{*}$} \\
\toprule
%\showrowcolors
\textbf{31.V.2012} &   & \textbf{20} & \textbf{40}&  29 &  & \textbf{29} & RT \\
\textbf{02.VI.2012} & & \textbf{25} & \textbf{50}&  42 &  & \textbf{42} & RT \\
\textbf{06.VII.2012} & & \textbf{25} & \textbf{50}&   & 49 & \textbf{49} & RT \\
\textbf{05.VIII.2012} & \textbf{5} & 20 & \textbf{45}&   & 48 & \textbf{48} & RT, NC \\
\textbf{12.IX.2012} & \textbf{46} &  & \textbf{46}&   & 40 & \textbf{40} & RT \\
\textbf{20.IV.2013} & \textbf{18} &  & \textbf{18}&   &  & \textbf{0} & RT \\
\textbf{14.V.2013} & \textbf{20} &  & \textbf{20}&   &  & \textbf{0} & RT \\
\textbf{24.V.2013} & \textbf{41} &  & \textbf{41}&   &  & \textbf{0} & RT, GC \\
\textbf{09.VI.2013} & \textbf{58} &  & \textbf{58}&   &  & \textbf{0} & RT \\
\textbf{22.VI.2013} & \textbf{39} & 21 & \textbf{81}&  50 &  & \textbf{50} & RT \\
\textbf{02.VII.2013} & \textbf{52} & 28 & \textbf{108}&  93 & 2 & \textbf{95} & RT \\
\textbf{13.VII.2013} & \textbf{50} & 38 & \textbf{126}&  102 & 2 & \textbf{104} & RT \\
\textbf{21.VII.2013} & \textbf{90} & 19 & \textbf{128}&  42 & 63 & \textbf{105} & RT \\
\textbf{05.VIII.2013} & \textbf{120} & 2 & \textbf{124}&  5 & 96 & \textbf{101} & RT \\
\textbf{11.VIII.2013} & \textbf{85}& 19 & \textbf{123}&  13 & 99 & \textbf{112} & RT \\
\textbf{03.IX.2013} & \textbf{120} &  & \textbf{120}&   & 108 & \textbf{108} & RT \\
\hiderowcolors
\multicolumn{8}{l}{}\\
\multicolumn{8}{l}{$^{*}$ GC: Giovanni Cumbo, NC: Natalino Cuti, RT: Rosa Termine}\\
\bottomrule
\end{tabular}
}
\caption{Osservazioni di svasso piccolo durante le nidificazioni del 2012 e del 2013}
\label{Termine_tab_2}
\end{table}

\begin{figure}[!h]
\centering
\includegraphics[width=.98\columnwidth]{Termine_fig_1.jpg}
\caption{Andamento del numero max di adulti di svasso piccolo osservati tra XI.2008 e IX.2013, in relazione alla fenologia annuale}
\label{Termine_fig_1}
\end{figure}

\section*{Risultati e discussione}

Le osservazioni di svasso piccolo, dal 2004 al settembre 2013, sono
riassunte in tabella 1, nella quale si riporta il numero massimo di
individui adulti osservati per mese, suddivise a seconda del periodo
fenologico annuale.

L{\textquoteright}andamento mensile, riferito a individui adulti censiti
da novembre del 2008 al settembre del 2013, sono riportate in figura 1.

Una seconda tabella (Tab. \ref{Termine_tab_2}), invece, riporta solo le osservazioni dello
svasso piccolo nel periodo tra maggio 2012 e settembre 2013,
differenziando tra adulti (singoli e coppie) e nuovi nati
(\textit{pulli} e \textit{juvenes}).

{Tra il 2003 e il 2004 tale specie \`e stata osservata
a Pergusa solo durante il periodo migratorio e durante il periodo di
svernamento} (Termine \textit{et al}. 2008){, mentre
presenze estive sono state registrate tra il 2005 e il 2009 (}Ientile
\textit{et al}. 2010; {Termine
}\textit{{et al.}}{ 2011). Le
osservazioni da novembre del 2008 a settembre del 2013 (Fig. \ref{Termine_fig_1}) hanno
permesso di rilevare una costante presenza estiva dello svasso piccolo
nel lago dal 2009 al 2013 e di accertarne la nidificazione negli anni
2010, 2012 (Fig. \ref{Termine_fig_2}) e 2013 (Fig. \ref{Termine_fig_3}).}

Con questa nota si documenta, quindi, il ritorno come nidificante dello
svasso piccolo \textit{Podiceps nigricollis} C. L. Brehm, 1831 al lago
di Pergusa, e la rilevanza di questa zona umida che ne ospita il
maggior numero di coppie nidificanti per la Sicilia e per
l{\textquoteright}intera Italia (Verducci \& Sighele 2013).

{Infatti, dopo 52 anni dalla storica nidificazione
(Krampitz 1958), nel 2010 al lago di Pergusa \`e stata accertata la}
presenza di 6 coppie nidificanti (Ientile \textit{et al}. 2010; Termine
\textit{et al}. 2011).

Nel 2012 presso la Riserva pergusina, oltre alle consuete presenze
invernali, a partire da fine maggio sono state osservate 20 coppie di
svasso piccolo; i primi \textit{pulli}, in numero di 29, sono stati
osservati il 31 maggio; durante il censimento del 2 giugno sono state
contate 25 coppie e 42 \textit{pulli}; a luglio sono state censite 25
coppie e 49 \textit{juv.} (Tab. \ref{Termine_tab_1}).

All{\textquoteright}inizio della primavera del 2013 sono stati osservati
presso la Riserva del lago di Pergusa 18 individui di svasso piccolo in
abito nuziale; le presenze della specie sono diventate notevolmente
pi\`u consistenti dopo il 24 maggio con 41 individui. A partire da
questa data, sono stati eseguiti ulteriori censimenti di tale specie a
cadenza massima quindicinale; i primi \textit{pulli}, in numero di 50,
sono stati osservati il 22 giugno; poi si \`e avuto un graduale
incremento fino al 21 luglio, arrivando a contare 128 adulti e 105
nuovi nati; fino all{\textquoteright}11 agosto sono state registrate
ulteriori nascite, raggiungendo il numero massimo di 112 tra
\textit{pulli} e \textit{juv.} (Tab. \ref{Termine_tab_2}).

Diversi sono i fattori che possono avere influito sulla nidificazione
dello svasso piccolo; tra questi la disponibilit\`a trofica, la
tipologia di vegetazione ripariale, la presenza di piante acquatiche
{emergenti, il rischio di predazione, la naturale
espansione della specie, la pressione antropica, etc. Probabilmente} al
lago di Pergusa la disponibilit\`a di risorse trofiche, tra cui la
notevole presenza di piccoli pesci, anfibi e invertebrati vari, ha
determinato una forte attrazione per gli svassi piccoli negli ultimi
anni, soprattutto nel periodo estivo.

Un altro fattore influente potrebbe essere stato
l{\textquoteright}espansione delle fanerofite \textit{Ruppia} sp. L. e
\textit{Potamogeton} \textit{pectinatus} L. con formazione di grandi
isole galleggianti (\textit{mats}) che gi\`a nel 2010 occupavano circa
1/3 dello specchio lacustre, mentre nel 2013 ne hanno occupato circa
3/4. La vegetazione galleggiante offre allo svasso piccolo un buon
substrato dove costruire il nido galleggiante, in grado di adeguarsi
alle variazioni del livello dell{\textquoteright}acqua.

\section*{Conclusioni}

{Il ripristino di condizioni ottimali del lago, sia in
termini di tutela, che ambientali, ha certamente favorito la
nidificazione dello svasso piccolo. Le nidificazioni della specie
avvenute a Pergusa nel 2010, 2012 e 2013 hanno coinciso con la presenza
di fanerofite galleggianti, situazione che non si \`e verificata nel
2011; infatti, tale vegetazione acquatica era pressoch\'e assente per
l{\textquoteright}eccessivo numero di
}\textit{{Cyprinus carpio}}{
(successivamente andati incontro a massiva moria
}nell{\textquoteright}autunno 2011 per la fioritura di
\textit{Prymnesium parvum }Carter, alga ittiotossica) il cui spettro
alimentare comprende anche alghe e macrofite acquatiche.
L{\textquoteright}assenza di tale vegetazione ha probabilmente
scoraggiato nel 2011 la riproduzione dello svasso piccolo nel Lago.

Pertanto sarebbe interessante comprendere se e in che modo la presenza
di tali isole galleggianti abbia rappresentato un fattore favorevole
all{\textquoteright}insediamento dei soggetti nidificanti di questa
specie. {Il lago di Pergusa ha una grande
vulnerabilit\`a; per la sua natura endoreica e la limitata estensione e
profondit\`a, \`e, infatti, molto sensibile ai cambiamenti climatici in
corso che portano ad una graduale diminuzione della piovosit\`a nella
regione. Inoltre, essendo strettamente a contatto con territori
urbanizzati, la sua sopravvivenza dipende anche dalle scelte da parte
dell{\textquoteright}uomo}; ci\`o dovrebbe far comprendere quanto sia
fondamentale effettuare una pianificazione urbanistica che preveda la
realizzazione di specifiche zone tampone a confine con gli ambienti
naturali, evitando la brusca interruzione di questi ambienti; cos\`i si
potrebbe ottenere il giusto compromesso tra le esigenze
socio-economiche e la conservazione di questa area protetta e quindi
anche della specie in argomento.

\begin{figure}[!h]
\centering
\includegraphics[width=.6\columnwidth]{Termine_fig_2.jpg}
\caption{Adulto con \textit{juvenes}, a Pergusa nel 2012 (Foto di R. Termine)}
\label{Termine_fig_2}
\end{figure}

\begin{figure}[!h]
\centering
\includegraphics[width=.7\columnwidth]{Termine_fig_3.jpg}
\caption{Svasso piccolo: adulto con \textit{pullus} sul dorso, a Pergusa nel 2013 (Foto di R. Termine)}
\label{Termine_fig_3}
\end{figure}

\section*{Ringraziamenti}
Rivolgiamo un doveroso ringraziamento alla Provincia Regionale di Enna,
ente gestore della R.N.S. lago di Pergusa, che ha promosso queste
ricerche, a Natalino Cuti e a Giovanni Cumbo per la generosa
disponibilit\`a.

\section*{Bibliografia}
\begin{itemize}\itemsep0pt
 \item {Benoit L., 1840 -
}\textit{{Ornitologia Siciliana}}{.
Stamperia G. Fiumara, Messina.}

 \item {Brichetti P.} \&{ Fracasso G., 2013 -
}\textit{{Ornitologia Italiana. Vol. 1.
Gaviidae-Falconidae}}{.
}\textit{{Edizione elettronica riveduta e
aggiornata}}{. Alberto Perdisa Ed., Bologna.}

 \item {Corso A., 2005 - }\textit{{Avifauna
di Sicilia}}{. L{\textquoteright}Epos Societ\`a
Editrice, Palermo.}

 \item {Corso A. in Ruggieri L. \& Sighele M. (red.), 2007 -
}\textit{{Annuario 2006}}{. EBN
Italia, Verona, 10 pp.}

 \item {Di Blasi F., 2011 - Comunicazione
personale}\textit{{. }}{LIPU Saline d
Priolo.}

 \item {Doderlein P., 1873 - Avifauna del Modenese e della
Sicilia. }\textit{{Giorn. Sci. Nat.
}}\textit{{Econom.,}}{ 5: 265-328.
Iapichino C. \& Massa B., 1989 - The Birds of Sicily.
}\textit{{British Ornithologist{\textquoteright}Union,
}}{Check-list, 11: 1-170.}

 \item {Ientile R., Termine R. \& Siracusa A. M., 2010 -
Nidificazione di Svasso piccolo }\textit{{Podiceps
nigricollis}}{ C. L. Brehm, 1831 (Aves
Podicipediformes) nella Riserva Naturale Speciale
Lago}\textit{{ }}{di Pergusa (Enna).
}\textit{{Naturalista  sicil}}{., S.
IV, XXXIV (3-4): 543-544.}

 \item {Krampitz H. E., 1958 - Weiteres uber die Brutvogel
Siziliens. }\textit{{J. Orn.}}{, 99:
39-58.}

 \item {Lo Valvo M., Massa B. \& Sar\`a M., 1993 - Uccelli e
paesaggio in Sicilia alle soglie del terzo millennio.
}\textit{{Naturalista sicil}}{., vol.
XVII, suppl.: 1- 371.}

 \item {Mascara R., 2007 - L{\textquoteright}avifauna degli
invasi artificiali di Cimia, Comunelli e Disueri (Caltanissetta,
Sicilia). Aggiornamento 1993-2006. }\textit{{Uccelli
d{\textquoteleft}Italia}}{, XXXII: 9-20.}

 \item {Massa B. \& Schenk H., 1983 - Similarit\`a tra le
avifaune della Sicilia, Sardegna e Corsica.
}\textit{{Lav. Soc. It.
Biogeografia}}{, 8 (1980): 757-799.}

 \item {Rallo G. \& Pandolfi M., 1988 -
}\textit{{Le zone umide del
Veneto}}{. Muzzio ed., Padova.}

 \item {Termine R., Canale E. D., Ientile R., Cuti N., Di
Grande C. S. \& Massa B., 2008 - Vertebrati della Riserva Naturale
Speciale e Sito d{\textquotesingle}Importanza Comunitaria Lago di
Pergusa. }\textit{{Naturalista
sicil.}}{, 32: 105-186.}

 \item {Termine R., Ientile R. \& Siracusa M. A., 2011 -
Nidificazione di Svasso piccolo nella Riserva Naturale Speciale del
Lago di Pergusa. }\textit{{Biologi
Italiani}}{, XLI, n{\textdegree} 2: 42-46.}

 \item {Verducci D. \& Sighele M., 2013 - La nidificazione
dello Svasso piccolo }\textit{{Podiceps
nigricollis}}{ in Italia.
}\textit{{U.D.I.}}{, XXXVIII: 39-48.}
\end{itemize}
