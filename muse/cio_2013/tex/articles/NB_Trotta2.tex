\setcounter{figure}{0}
\setcounter{table}{0}

\begin{adjustwidth}{-3.5cm}{0cm}
\pagestyle{CIOpage}
\authortoc{\textsc{Trotta M.}, \textsc{Panuccio M.}, \textsc{Dell{\textquoteright}Omo G.}}
\chapter*[La dieta del gheppio in un paesaggio agricolo
dell'Italia centrale]{La dieta del gheppio \textbf{\textit{Falco
tinnunculus}}\textbf{ nidificante in un paesaggio agricolo
dell'Italia centrale}}
\addcontentsline{toc}{chapter}{La dieta del gheppio in un paesaggio agricolo
dell'Italia centrale}

\textsc{Marco Trotta}$^{1*}$, \textsc{Michele Panuccio}$^{2,3}$, \textsc{Giacomo
dell{\textquoteright}Omo}$^{3}$ \\

\index{Trotta Marco} \index{Panuccio Michele} \index{Dell'Omo Giacomo}
\noindent\color{MUSEBLUE}\rule{27cm}{2pt}
\vspace{1cm}
\end{adjustwidth}


\marginnote{\raggedright $^1$SROPU -- Stazione Romana Osservazione e Protezione
Uccelli, Via Britannia 36, 00183 Roma, Italia \\
$^2$Ente Regionale Roma Natura \\
$^3$\textit{Ornis italica}, Roma \\
\vspace{.5cm}
{\emph{\small $^*$Autore per la corrispondenza: \href{mailto:marcotrot@gmail.com}{marcotrot@gmail.com}}} \\
\keywords{\textit{Falco tinnunculus}, Italia centrale, dieta, paesaggio agricolo}
{\textit{Falco tinnunculus}, Central Italy, diet, farmland}
}
{
\small
\noindent \textsc{\color{MUSEBLUE} Summary} / In this study, we analysed the diet of breeding kestrels \textit{Falco tinnunculus} in a Mediterranean area. The 46,3\% of the biomass of
preys is represented by birds and the rest is distributed among
mammals, reptiles and insects. The results suggest a wider diet
composition of the kestrels breeding in the study area, despite of
other European populations show a predator-prey relationships with
voles and other rodents. \\
}
\vspace{1cm}

Nella stagione riproduttiva 2012 abbiamo analizzato, attraverso la
raccolta di borre e resti di prede, il regime alimentare del gheppio
\textit{Falco tinnunculus} in un contesto tipico della campagna
agricola romana, quello della Riserva Naturale di Decima-Malafede
(Lazio). La raccolta \`e stata effettuata ispezionando 20 cassette nido
posizionate su tralicci dell{\textquoteright}alta tensione. Sono state
esaminate 91 borre integre, oltre a numerosi resti alimentari, per un
totale di {228 prede. Il 46,3\% della biomassa predata
\`e rappresentato da uccelli, la restante parte \`e distribuita tra
mammiferi (27,1\%), rettili (19,1\%) e insetti (7,6\%, Tab.1).}
{L{\textquoteright}}elevata frequenza di specie
ornitiche associate {ad ambienti agricoli e la
presenza in tarda primavera di individui appena involati, quindi pi\`u
inesperti, }hanno {probabilmente determinato, nel
contesto ambientale indagato, una maggiore predazione di uccelli. Le
preferenze alimentari sono indirizzate principalmente verso i
Passeriformi di piccole dimensioni,}

{tra cui il genere }\textit{{Passer}}
\`e il pi\`u rappresentato (26,8\% degli uccelli predati).
{Tra gli insetti catturati la specie dominante \`e il
coleottero }\textit{{Pentodon
bidens}} { (60,0\%), un ruolo importante \`e rivestito
anche dagli ordini degli Odonati (8,8\%) e degli Ortotteri (8,0\%). }I
mammiferi sono rappresentati da \textit{Microtus savii} e
sporadicamente da alcuni individui di \textit{Apodemus }spp.;
l{\textquoteright}apporto trofico dei Soricomorfi \`e irrilevante
(0,4\%). {I Lacertidi costituiscono
l{\textquoteright}81,4\% dei rettili, le cui catture sono ripartite tra
}\textit{{Podarcis }}{spp. (44,4\%) e
}\textit{{Lacerta bilineata}} {
(37,0\%); tra le prede compaiono anche
}\textit{{Tarentola mauritanica
}}{(1,3\%) e }\textit{{Chalcides
chalcides }}{(0,9\%). Precedenti studi hanno
evidenziato come la dinamica delle popolazioni europee di gheppio sia
influenzata dalla densit\`a dei micro-Mammiferi (Korpim\"aki 1985;
Village 1990; Korpim\"aki \& Norrdahl 1991; Fargallo 
}\textit{{et al.}}{ 2009), che
rappresentano quindi una componente dominante della dieta. Lo spettro
alimentare rilevato dalla nostra indagine appare invece pi\`u
diversificato. Si ritiene che tale risultato sia espressione della
}notevole plasticit\`a ecologica del gheppio, in grado di sfruttare le
risorse trofiche pi\`u abbondanti all'interno del
proprio territorio di caccia (Ferguson-Lees 2001; Costantini \textit{et
al.} 2005).

\begin{table}[!h]
\centering
\begin{tabular}{>{\raggedright\arraybackslash}p{.3\columnwidth}>{\raggedleft\arraybackslash}p{.2\columnwidth}>{\raggedleft\arraybackslash}p{.2\columnwidth}}
\toprule
\textbf{Taxa} & \textbf{Tot. prede} & \textbf{Biomassa \%} \\
\toprule
\textit{Crocidura spp.}	& 1&	0.3 \\
\toprule
\hiderowcolors
\textbf{\textit{Soricomorpha}} & 1&	0.3 \\
\toprule
%\showrowcolors
\textit{Apodemus spp.}	&2	&2.2 \\
\textit{Microtus savii}&	6	&4.6 \\
\textit{Rodentia ind.}	&26&	20.0 \\
\toprule
\hiderowcolors
\textit{\textbf{Rodentia}}	&34	&26.8 \\
\toprule
\textit{\textbf{Mammalia}}	&35&	27.1 \\
\toprule
%\showrowcolors
\textit{Motacilla alba}	&2	&1.5 \\
\textit{Troglodytes troglod.}&	1&	0.3 \\
\textit{Turdus merula}	&3	&9.9 \\
\textit{Luscinia megarhynchos}&	1	&0.8 \\
\textit{Sylvia atricapilla}&	2&	1.4 \\
\textit{Sturnus vulgaris}	&3&	8.8 \\
\textit{Passer italiae}&	3&	3.3 \\
\textit{Passer spp.}	&8	&7.8 \\
\textit{Carduelis carduelis}&	5	&2.9 \\
\textit{Passeriformi ind.}& 	13&	9.5 \\
\toprule
\hiderowcolors
\textit{\textbf{Aves}}	&41	&46.3 \\
\toprule
%\showrowcolors
\textit{Lacerta bilineata}&	10	&12.8 \\
\textit{Podarcis siculus}	&3	&1.4 \\
\textit{Podarcis muralis}	&5&	1.6 \\
\textit{Podarcis spp.}&	4	&1.5 \\
\textit{Tarentola mauritanica}&	3	&0.9 \\
\textit{Chalcides chalcides}&	2	&0.9 \\
\toprule
\hiderowcolors
\textit{\textbf{Reptilia}}	&27&	19.1 \\
\toprule
%\showrowcolors
\textit{Pentodon bidens}&	75&	5.5 \\
\textit{Coccinellidae ind.}	&3	&0.1 \\
\textit{Anacridium aegyptium}&	8&	0.6 \\
\textit{Gryllotalpa gryllotalpa}&	1&	0.1 \\
\textit{Orthoptera ind.}	&1&	0.0 \\
\textit{Odonata ind.}	&11	&0.4 \\
\textit{Lepidoptera ind.}	&2&	0.0 \\
\textit{Insecta ind.}	&24&	0.9 \\
\toprule
\hiderowcolors
\textit{\textbf{Insecta}}	&125&7.6 \\
\toprule
\textbf{Totale}	&\textbf{228}&	\textbf{100.0} \\
\bottomrule
\end{tabular}
\caption{Dieta della popolazione di gheppio nidificante nella R.N.R. di Decima-Malafede}
\label{Trotta2_tab_1}
\end{table}

\section*{Ringraziamenti}

Si ringrazia Giulio Fancello e l'Ente Regionale
RomaNatura nonch\'e TERNA per l{\textquoteright}installazione e il
mantenimento delle cassette nido sui tralicci della rete elettrica e
per il continuo supporto alle attivit\`a di monitoraggio durante la
nidificazione dei gheppi.

\section*{Bibliografia}

\begin{itemize}\itemsep0pt
	\item Costantini D., Casagrande S., Di Lieto G., Fanfani, A. \&
Dell{\textquoteright}Omo G., 2005 - Consistent differences in feeding
habits between neighbouring breeding kestrels. \textit{Behaviour,} 142:
1409-1421.

	\item Fargallo J.A., Martinez-Padilla J., Vinuela J., Blanco G., Torre I,
Vergara P. \& De Neve L., 2009 -  Kestrel-Prey Dynamic in a
Mediterranean Region: The Effect of Generalist Predation and Climatic
Factors. \textit{PLoS one} 4 (2): e4311. 

	\item Ferguson-Lees J. \& Christie D.A., 2001 - \textit{Raptors of the world}.
London: Christopher Helm, 320 pp.

	\item Korpim\"aki E., 1985 - Diet of the Kestrel \textit{Falco tinnunculus} in
the breeding season. \textit{Ornis Fenn.,} 62: 130-137.

	\item Korpim\"aki E. \& Norrdahl K., 1991 - Numerical and functional responses
of kestrels, short-eared owls, and long-eared owls to vole densities.
\textit{Ecology,} 72: 814-826.

	\item Village A., 1990 - \textit{The Kestrel}. T \& AD Poyser, Londra, 352 pp.
\end{itemize}
