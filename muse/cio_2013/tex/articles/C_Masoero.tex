\setcounter{figure}{0}
\setcounter{table}{0}

\begin{adjustwidth}{-3.5cm}{0cm}
\pagestyle{CIOpage}
\authortoc{\textsc{Masoero G.}, \textsc{Tamietti A.}, 
\textsc{Caprio E.}}
\chapter*[Trend della popolazione di topino nel Parco del Po
torinese]{Trend della popolazione di topino \textbf{\textit{Riparia
riparia }}\textbf{nel Parco del Po e della Collina Torinese}}
\addcontentsline{toc}{chapter}{Trend della popolazione di topino nel Parco del Po
torinese}

\textsc{Giulia Masoero}$^{1*}$, \textsc{Alberto Tamietti}$^{2}$, 
\textsc{Enrico Caprio}$^{1}$\\

\index{Masoero Giulia} \index{Tamietti Alberto} \index{Caprio Enrico}
\noindent\color{MUSEBLUE}\rule{27cm}{2pt}
\vspace{1cm}
\end{adjustwidth}



\marginnote{\raggedright $^1$Dipartimento di Scienze della Vita e Biologia dei
Sistemi, Universit\`a di Torino \\
$^2$Parco del Po e della collina Torinese \\
\vspace{.5cm}
{\emph{\small $^*$Autore per la corrispondenza: \href{mailto:giulia.masoero@gmail.com}{giu\allowbreak lia.\allowbreak ma\allowbreak so\allowbreak e\allowbreak ro@\allowbreak g\allowbreak ma\allowbreak il.\allowbreak com}}} \\
\keywords{\textit{Riparia riparia}, Parco del Po e della Collina
torinese}
{\textit{Riparia riparia}, Po Park, Collina torinese Park}
%\index{keywords}{\textit{Riparia riparia}} \index{keywords}{Parco del Po e della Collina
%torinese}
}
{\small
\noindent \textsc{\color{MUSEBLUE} Summary} / The sand martin \textit{Riparia riparia} is a trans-saharian migrant
passerine. This species is classified as
{\textquotedblleft}vulnerable{\textquotedblright} in the 2011 Red List
of the Birds breeding in Italy and its population are declining in all
Europe. The present study shows the results of a ten years monitoring
on a sand martins breeding population along the southern Po river in
Turin. This research was promoted by the Po and Collina Torinese
Natural Park. The birds were captured and ringed at breeding sites
within the Park and the capture-mark-recapture data collected from 2002
to 2012 were analysed with the software MARK v 6.1 to obtain survival
rates. The annual survival rates were related to two meteoclimatic
indices, the Sahel rainfall index and the North Atlantic Oscillation
(NAO) index. The results showed an average survival rate of 33.8\%. The
annual survival rate resulted related to the two climatic indices
during some of the most important periods of sand martins life; Sahel
rainfall index during wintering and NAO index during breeding and
autumn migration. \\
\noindent \textsc{\color{MUSEBLUE} Riassunto} / Il topino \textit{Riparia riparia} \`e un passeriforme migratore
trans-sahariano classificato come
{\textquotedblleft}vulnerabile{\textquotedblright} nella Lista Rossa
2011 degli Uccelli nidificanti in Italia con popolazioni in declino in
tutta Europa. Il presente lavoro mostra i risultati di uno studio
decennale sulla nidificazione del topino lungo il Po a sud di Torino,
condotto dal Parco del Po e della Collina torinese attraverso
l{\textquoteright}inanellamento a scopo scientifico nelle colonie di
nidificazione. I dati di cattura-marcatura-ricattura raccolti dal 2002
al 2012 sono stati analizzati con il software MARK v. 6.1 al fine di
ottenere i dati relativi al tasso di sopravvivenza. Il tasso di
sopravvivenza annuale ottenuto \`e stato poi messo in relazione con gli
indici meteoclimatici di Oscillazione Nord Atlantica (NAO) e di
piovosit\`a del Sahel. I risultati ottenuti mostrano complessivamente
un tasso di sopravvivenza medio pari al 33.8\%. Il tasso di
sopravvivenza annuale \`e risultato dipendere in maniera significativa
dalla piovosit\`a del Sahel durante lo svernamento e
dall{\textquoteright}indice NAO durante il periodo riproduttivo e la
migrazione autunnale. 
}


\section*{Introduzione}

 Il topino \textit{Riparia riparia} \`e un migratore trans-sahariano
comune in Italia durante la stagione riproduttiva. Studi recenti hanno
dimostrato che il clima dei quartieri di svernamento africani \`e un
fattore di importanza cruciale nel ciclo annuale di un gran numero di
uccelli migratori su lunga distanza. Uno dei fattori pi\`u importanti
che appare collegato con la loro sopravvivenza \`e la piovosit\`a della
regione del Sahel. 

Durante il nostro studio abbiamo analizzato la relazione tra clima e
sopravvivenza durante tutto il ciclo vitale di adulti di topino, con
particolare attenzione ai quattro eventi fondamentali: lo svernamento,
la nidificazione e le migrazioni.

\section*{Area di studio}

L{\textquotesingle}area studiata \`e situata a sud di Torino, compresa
nei comuni di Carignano, Carmagnola, Lombriasco, La Loggia e
Moncalieri. Il territorio ricade nei confini del Parco del Po e della
Collina torinese ed \`e tutelato come parco regionale dal 1990 (L.R.
28/90 e successive modificazioni). 

\`E stato preso in esame un tratto di fiume Po e gli impianti di
estrazione di sabbia e ghiaia (cave) presenti lungo il corso del fiume
stesso. Il tratto in esame presenta una lunghezza lineare totale di 13
km e sono presenti 14 cave, 9 delle quali hanno presentato
nidificazioni di topino nel corso dello studio.

L{\textquoteright}ambiente fluviale \`e caratterizzato da
un{\textquoteright}area di bosco ripariale (saliceto) lungo le sponde
del fiume. Quest{\textquoteright}area, di dimensioni ridotte, separa il
fiume stesso dalla pianura coltivata a mais, cereali vernini, soia e
pioppeti.

L{\textquoteright}ambiente di cava attiva \`e invece costituito da un
lago artificiale le cui sponde sono oggetto di ripristino ambientale
finalizzato a ottenere boschi planiziali, ripariali o fasce arbustive.
Questo occupa la maggior parte area della cava. Una parte di estensione
minore \`e dedicata agli impianti produttivi, con cumuli di sabbia e
pietrisco, nastri trasportatori e piazzali e strade per
l{\textquoteright}accumulo o il trasporto degli inerti.

\section*{Metodi}

Tra il 2002 e il 2012 nei mesi di maggio, giugno e luglio si sono svolte
le operazioni di cattura e inanellamento in 9 cave attive di sabbia e
ghiaia adiacenti al fiume Po. Per l{\textquotesingle}analisi dei dati
di cattura-marcatura-ricattura \`e stato utilizzato il programma MARK,
utilizzando il modello \textit{Cormack-Jolly-Seber} (Cormack 1964;
Jolly 1965; Seber 1965). Una volta marcati gli animali vengono
rilasciati e quindi catturati nuovamente, riconosciuti, e rilasciati.
Ogni incontro \`e caratterizzato da una probabilit\`a di sopravvivenza
fra quello precedente e quello successivo. Questo parametro \`e detto  
$\Phi_{i}\Phi_{i}$. Il parametro $pp$ rappresenta invece la probabilit\`a di ricatturare
l{\textquoteright}animale in quell{\textquoteright}occasione (Cooch \&
White 2012).

Il tasso di sopravvivenza annuale \`e stato messo in relazione con
l{\textquotesingle}indice di piovosit\`a del Sahel e
l{\textquotesingle}indice di oscillazione Nord-Atlantico (NAO)
utilizzando modelli lineari generalizzati univariati (GLM). Come dati
di riferimento per l{\textquoteright}indice NAO sono stati utilizzati
gli indici mensili e stagionali di Hurrell.

\section*{Risultati}

Sono stati catturati 2254 individui adulti, per un totale di 314
individui ricontrollati negli anni successivi. Questi dati sono stati
analizzati con il programma MARK utilizzando come modello di partenza,
un modello in cui tasso di sopravvivenza (\textit{$\Phi $}) e di
ricattura (\textit{p}) variano con il tempo (t) (Cormack 1964; Jolly
1965; Seber 1965). Il test di \textit{Goodness of Fit} (  $\chi_{21}^{2}=$
  12.1793; P = 0.9347) indica che il modello descrive i dati in maniera
efficace. 

I modelli possibili per le variabili di sopravvivenza e di probabilit\`a
sono i quattro modelli per cui esse variano, o meno, in dipendenza dal
tempo (Tab. \ref{Masoero_tab_1}).

Il modello pi\`u probabile \`e quello con il minor valore di AICc e
quindi quello in cui la sopravvivenza rimane costante di anno in anno
mentre cambia la probabilit\`a di cattura. 

Il modello $\Phi\Phi(.)p(t)$ ci consente quindi di fornire una stima della probabilit\`a
di sopravvivenza non dipendente dal tempo. I modelli $\Phi\Phi(t)p(.)$ e
$\Phi\Phi(t)p(t)$ hanno valore di AICc molto simili e quindi confrontabili. Per
il confronto con i dati del clima si \`e scelto il modello$\Phi\Phi(t)p(t)$.
$\Phi\Phi(t)p(t)$ dipendente dal tempo rappresenta la probabilit\`a di sopravvivenza
di anno in anno. 

Il risultato dei GLM effettuati per valutare la relazione tra tasso di
sopravvivenza annuale e indice di piovosit\`a del Sahel di ogni mese ha
individuato una relazione significativa con il mese di ottobre (Beta =
0.00127, SE = 0.0005, p {\textless} 0.05).

I GLM effettuati per valutare la relazione tra tasso di sopravvivenza
annuale e indice NAO stagionale hanno individuato come modello migliore
quello con i mesi di giugno, luglio e agosto (NAO\_JJA) (Beta =
0.13465, SE = 0.04358, p {\textless} 0.05). I GLM tra il tasso di
sopravvivenza annuale e gli indici NAO mensili hanno evidenziato una
relazione significativa con l{\textquoteright}indice NAO del mese di
settembre (Beta = -0.13950, SE = 0.01933, p {\textless} 0.001).

\section*{Discussione}

Dall{\textquoteright}analisi dei dati di inanellamento si ottiene una
probabilit\`a di sopravvivenza del 33.8\%, risultato simile a quanto
rilevato da Cowley e Siriwardena (2005) ( 31\% per i
maschi e del 29\% per le femmine) oppure in quello di Norman e Peach
(2013) 38\% per i maschi e 31\% per le femmine. Il risultato da noi
ottenuto \`e conforme alle stime pubblicate derivanti dai ritrovamenti
di uccelli morti non sessati in Gran Bretagna (35,4\%; Dobson, 1990),
che suggeriscono che il nostro studio non ha eccessivamente
sottostimato la vera sopravvivenza (Norman \& Peach 2013).

Il tasso di sopravvivenza annuale fornisce invece valori che oscillano
di anno in anno, passando da una sopravvivenza minima
del\textcolor{black}{ 10.3\%} nel 2011 a una sopravvivenza massima del
\textcolor{black}{53.7\% nel 2007.}

Le popolazioni europee di topino svernano nel Sahel occidentale (Mead
1979; Robinson \textit{et al.} 2008) e il ruolo di tale area nello
svernamento e nella preparazione al volo trans-sahariano in primavera
\`e essenziale (Sz\'ep 1995). L{\textquoteright}indice di piovosit\`a
del Sahel fornisce una misura indiretta delle condizioni di abbondanza
delle aree umide a cui \`e legata l{\textquoteright}alimentazione del
topino sia in migrazione sia durante lo svernamento (Morel \& Morel
1992). 

I nostri risultati sono in accordo con quanto si riscontra in
bibliografia: il tasso di sopravvivenza annuale, infatti, \`e
influenzato dalla variazione della piovosit\`a del Sahel, in
particolare con la piovosit\`a nel mese di ottobre. Anche in studi
precedenti la piovosit\`a nel Sahel occidentale \`e correlata con le
variazioni annuali della sopravvivenza (Robinson \textit{et al.} 2008).

La relazione positiva tra le variazioni nel tasso di sopravvivenza
annuale e l{\textquoteright}indice NAO dei mesi di giugno, luglio e
agosto potrebbe significare che in quel periodo un indice NAO con
valori positivi favorisce la sopravvivenza dei topini.
Nell{\textquoteright}area del Mediterraneo, valori positivi nella
stagione estiva indicano una maggiore piovosit\`a (Blad\'e \textit{et
al.} 2012). Un clima umido, con pi\`u precipitazioni, potrebbe impedire
l{\textquoteright}eccessivo disseccamento di corsi
d{\textquoteright}acqua e aree umide, fondamentali per la dieta
insettivora dei topini. Nel periodo riproduttivo (giugno e luglio)
un{\textquoteright}adeguata alimentazione \`e necessaria poich\'e
consente all{\textquoteright}individuo di nutrire sufficientemente se
stesso e i piccoli. In agosto \`e necessaria a consentire al topino di
accumulare il grasso che gli fornir\`a le energie necessarie a compiere
la migrazione.

Di particolare interesse \`e la relazione negativa tra la sopravvivenza
e l{\textquoteright}indice NAO di settembre, mese in cui avviene la
migrazione verso i territori di svernamento. Con indice NAO positivo e
quindi maggiori precipitazioni nel Mediterraneo diminuirebbe quindi la
sopravvivenza dei topini. In questo caso non sono stati trovati lavori
precedenti che analizzassero questa relazione. Una possibile
spiegazione \`e che le piogge eccessive ritardino la migrazione
obbligando il topino a fermarsi per lunghi periodi. Soste obbligate e
prolungate in aree non idonee come siti di stop-over potrebbero causare
fenomeni di competizione intra- e interspecifica per le risorse
disponibili, come \`e stato dimostrato per la migrazione attraverso il
golfo del Messico (Moore \& Yong 1991). 

I risultati ottenuti evidenziano che la componente climatica \`e un
elemento importante nella sopravvivenza del topino. In ogni periodo
dell{\textquoteright}anno le condizioni atmosferiche influenzano la
popolazione considerata, favorendone o sfavorendone la sopravvivenza.\\

\begin{table}[!h]
\centering
\small
\begin{tabular}{>{\raggedright\arraybackslash}p{.11\columnwidth}>{\raggedright\arraybackslash}p{.11\columnwidth}>{\raggedright\arraybackslash}p{.11\columnwidth}>{\raggedright\arraybackslash}p{.11\columnwidth}>{\raggedright\arraybackslash}p{.11\columnwidth}>{\raggedright\arraybackslash}p{.11\columnwidth}>{\raggedright\arraybackslash}p{.11\columnwidth}}
\toprule
\textbf{Modello} & \textbf{AICc} & \textbf{Delta AICc} & \textbf{AICc Weights} & \textbf{Model Likelihood} & \textbf{Num. Par} & \textbf{Deviance} \\
\toprule
%\showrowcolors
$\Phi\Phi$(.)p(t) & 1196.3454 & 0 & 0.99229 & 1 & 11 & 44.6692 \\
$\Phi\Phi$(t)p(.) & 1207.1475 & 10.8021 & 0.00448 & 0.0045 & 11 & 55.4713 \\
$\Phi\Phi$(t)p(t) & 1207.7988 & 11.4534 & 0.00323 & 0.0033 & 19 & 39.9043 \\
$\Phi\Phi$(.)p(.) & 1234.7593 & 38.4139 & 0 & 0 & 2 & 101.1934 \\
\bottomrule
\hiderowcolors
\end{tabular}
\caption{Modelli per la sopravvivenza e la probabilit\`a di cattura}
\label{Masoero_tab_1}
\end{table}

\section*{Bibliografia}
\begin{itemize}
  \item Blad\'e I., Liebmann B., Fortuny D. \& Van Oldenborgh G.J., 2012 -
Observed and simulated impacts of the summer NAO in Europe:
implications for projected drying in the Mediterranean region.
\textit{Climate Dynamics,} 39: 709-727.

  \item Cooch E. \& White G., 2012 - \textit{Program Mark: a gentle
introduction}. Evan Cooch \& Gay White Eds, 11th Revision.

  \item Cormack R.M., 1964 - Estimates of survival from the sighting of marked
animals. \textit{Biometrika}, 51: 429-438.

  \item Cowley E. \& Siriwardena G.M., 2005 - Long-term variation in survival
rates of Sand Martins \textit{Riparia riparia}: dependence on breeding
and wintering ground weather, age and sex, and their population
consequences. \textit{Bird Study, }52: 237--251.

  \item Dobson A.P., 1990 - Survival rates and their relationship to
life-history traits in some common British birds. \textit{Curr.
Ornithol.}, 7: 115--146.

  \item Jolly G.M., 1965 - Explicit estimates from capture-recapture data with
both death and immigration -stochastic model. \textit{Biometrika}, 52:
225-247.

  \item Mead C.J., 1979 - Colony fidelity and interchange in the Sand Martin.
\textit{Bird Study}, 26: 99-107.

  \item Moore F.R. \& Yong W., 1991 - Evidence of food-based~competition among
passerine migrants during~stopover. \textit{Behavioral Ecology And
Sociobiology}, 28: 85-90.

  \item Norman D. \& Peach W.J., 2013 - Density-dependent survival and
recruitment in a long-distance Palaearctic migrant, the Sand Martin
\textit{Riparia riparia}. \textit{Ibis}, 155: 284-296.

  \item Robinson R.A., Balmer D.E. \& Marchant J.H., 2008 - Survival rates of
Hirundines in relation to British and African rainfall. \textit{Ringing
\& Migration}, 24: 1--6.

  \item Seber G.A.F., 1965 - A note on the multiple recapture census.
\textit{Biometrika}, 52: 249-259.
\end{itemize}
