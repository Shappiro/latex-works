\setcounter{figure}{0}
\setcounter{table}{0}

\begin{adjustwidth}{-3.5cm}{0cm}
\pagestyle{CIOpage}
\authortoc{\textsc{Angelini J.}, \textsc{Scotti M.}}
\chapter*[L{\textquoteright}alimentazione del lanario. Parco
Regionale Gola Rossa e Frasassi]{L{\textquoteright}alimentazione del lanario
\textbf{\textit{Falco biarmicus feldeggii}}\textbf{ nel Parco
Regionale Gola della Rossa e di Frasassi (AN) Italia centrale}}
\addcontentsline{toc}{chapter}{L{\textquoteright}alimentazione del lanario - Parco Regionale Gola Rossa e Frasassi}

\textsc{Jacopo Angelini}$^{1*}$, \textsc{Massimiliano Scotti}$^{2**}$\\

\index{Angelini Jacopo} \index{Scotti Massimiliano}
\noindent\color{MUSEBLUE}\rule{27cm}{2pt}
\vspace{1cm}
\end{adjustwidth}



\marginnote{\raggedright $^1$C.T.S.Parco Regionale Gola della Rossa e di Frasassi
Via Marcellini 60041 Serra San Quirico (AN) \\
$^2$Parco Regionale Gola della Rossa e di Frasassi Via
Marcellini 60041 Serra San Quirico (AN) \\
\vspace{.5cm}
{\emph{\small $^*$Autore per la corrispondenza: \href{mailto:jaco.angelini@gmail.com}{jaco.angelini@gmail.com}}} \\
{\emph{\small $^**$Autore per la corrispondenza: \href{mailto:massimiliano.scotti@parcogolarossa.it}{mas\allowbreak si\allowbreak mi\allowbreak lia\allowbreak no.\allowbreak scot\allowbreak ti@\allowbreak par\allowbreak co\allowbreak go\allowbreak la\allowbreak ros\allowbreak sa.\allowbreak it}}} \\
\keywords{\textit{Falco biarmicus feldeggii}, Italia centrale (Marche), dieta}
{\textit{Falco biarmicus feldeggii}, Central
Italy (Marche), diet}
%\index{keywords}{\textit{Falco biarmicus feldeggii}} \index{keywords}{Italia centrale (Marche)} \index{keywords}{Dieta}
}
{\small

\noindent \textsc{\color{MUSEBLUE} Summary} / In the Regional Park Frasassi Rossa{\textquoteright}s Gorges (Central
Italy), we studied the diet of the lanner \textit{Falco biarmicus
feldeggii} in 4 nesting sites from 2008 to 2012. We combined 3
different methods of data collection. a) direct observations, b)
sampling collection at plucking places, c) pellet analysis. We
recognized 177 preys, belong to 23 \textit{taxa}. Birds represented
85\% of the prey number and 91\% of biomass, Mammals represented the
9,95\% of the prey number and 7,5\% of biomass. Starlings
\textit{Sturnus vulgaris} is the most preyed species: 26\% of the prey
number and 12, 9\% of biomass.
Among Mammals, the dormouse \textit{Glis glis} is the more frequent prey
(5\%).  Interestingly, we found also a bat \textit{Mynopterus
scheibersi}.
}

\vspace{1cm}
Il Parco Regionale Gola della Rossa e di Frasassi e le zone limitrofe
ospitano 4 coppie nidificanti di lanario \textit{Falco biarmicus
feldeggii}, corrispondenti a ben il 40\% della popolazione conosciuta
della specie nella Regione Marche (Angelini 2007). Questa area protetta
della regione Marche ha una superficie di circa 10.000 ettari ed \`e
costituita da imponenti gole calcaree, da praterie secondarie e da
ambienti agricoli a mosaico con estese foreste di roverella
\textit{Quercus pubescens} e di carpino nero \textit{Ostrya
carpinifolia,} nel versante collinare, ed estesi boschi di faggio
\textit{Fagus sylvatica} alle quote maggiori. Lo studio, relativo alla
dieta del lanario \textit{Falco biarmicus feldeggii,} \`e stato
effettuato durante la stagione riproduttiva dal 2008 al 2012 nei 4 siti
di nidificazione occupati dalla specie. Sono stati utilizzati
contemporaneamente tre metodi diversi di raccolta dati: 1) osservazioni
dirette (n=116); 2) posatoio di spiumata (n=32); raccolta borre (n=29).
Sono state identificate complessivamente 177 prede appartenenti a 23
\textit{taxa} diversi. Gli uccelli hanno costituito
l{\textquotesingle}85\% delle prede, i mammiferi il 10\% e i rettili
5\% (Tab. \ref{Angelini_tab_1}). La specie maggiormente predata \`e risultata essere lo
storno \textit{Sturnus vulgaris }(47 casi, 26\% del totale), seguito
dalla taccola \textit{Corvus monedula} (13\%), dalla gazza \textit{Pica
pica} (10\%), dal merlo \textit{Turdus merula }(7\%), dalla ghiandaia
\textit{Garrulus glandarius }(6\%), dal piccione domestico
\textit{Columba livia var. domestica}(6\%), dal colombaccio \textit{Columbus
palumbus} (5\%). Tra i Mammiferi sono stati predati 16 individui (10\%
del totale), di cui il ghiro \textit{Glis glis} rappresenta il 3\% del
totale. Molto interessante il ritrovamento, in un posatoio di spiumata,
dei resti di \textit{Rhinolophus ferrumequinum}, mai segnalato in
precedenza come specie preda in Italia. Tra i Rettili il genere
\textit{Podarcis} rappresenta il 2\% del totale. In alcune borre sono
stati inoltre trovati resti di Insetti, probabilmente predati nelle
praterie del Parco. Interessante \`e anche la predazione tra i
Falconidi del gheppio \textit{Falco tinnunculus} e tra gli Accipitridi
dello Sparviere \textit{Accipiter nisus. }Come biomassa complessiva la
taccola \textit{Corvus monedula} rappresenta il 20\% del totale e gli
Uccelli complessivamente rappresentano il 91\% della biomassa totale in
accordo con quanto evidenziato per l{\textquoteright}Italia centrale da
altri autori (Morimando et al. 1997; De Sanctis et al. 2009).

Il Parco Regionale ha partecipato alla redazione del Piano di Azione
nazionale della specie, elaborato dall{\textquoteright}ISPRA (Andreotti
\& Leonardi 2007) e ha attuato diverse azioni di conservazione dirette
e indirette della specie, previste dal Piano stesso, come la
regolamentazione dell{\textquoteright}attivit\`a di arrampicata nelle
aree frequentate dalla specie con il divieto assoluto temporale durante
il periodo riproduttivo e anche con la creazione di aree di tutela
integrale tutto l{\textquoteright}anno, portando la specie ad
utilizzare nuovi siti nell{\textquoteright}area protetta. Inoltre sono
state messe in sicurezza oltre 60 km di linee elettriche grazie al
finanziamento del progetto LIFE Natura {\textquotedblleft}Save the
Flyers{\textquotedblright}, con il posizionamento di isolanti dei
conduttori sopra i pali a media tensione, vista la mortalit\`a diretta
della specie. Possibili fattori di minaccia per la specie possono
considerarsi il furto delle uova o dei piccoli al nido, visti i
sequestri effettuati dal Corpo Forestale dello Stato in diverse parti d{\textquoteright}Italia
di lanari catturati illegalmente in natura, le linee elettriche,
l{\textquoteright}attivit\`a di arrampicata vicina ai nidi della specie
e gli impianti eolici.



\rowcolors{2}{white!60!MUSEBLUE}{white}
\begin{table}[!h]
\centering
\footnotesize
\scalebox{.9}{
\begin{tabular}{>{\raggedright\arraybackslash}p{.4\columnwidth}d{6.3}d{6.3}}
\hiderowcolors
\toprule
\textbf{Specie} & \mc{\textbf{\% sulle prede totali}} & \mc{\textbf{\raggedright \% biomassa}} \\
\toprule
\multicolumn{3}{c}{\textbf{\textit{AVES}}} \\
%\showrowcolors
\textit{Sturnus vulgaris} & 26 & 12,9 \\
\textit{Turdus merula} & 7 & 3 \\
\textit{Monticola solitarius} & 2,3 & 1,1 \\
\textit{Phoenicurus ochruros} & 1,25 & 0,4 \\
\textit{Corvus corone} & 3 & 7,4 \\
\textit{Garrulus glandarius} & 6 & 6,9 \\
\textit{Pica pica} & 10 & 12,5 \\
\textit{Corvus monedula} & 13 & 2 \\
\textit{Streptopelia turtur} & 1,25 & 1 \\
\textit{Columbus palumbus} & 5 & 11,5 \\
\textit{Columba livia var. domestica} & 6 & 12,5 \\
\textit{Falco tinnunculus} & 0,6 & 0,8 \\
\textit{Accipiter nisus} & 0,6 & 0,8 \\
\textit{Picus viridis} & 0,6 & 0,5 \\
\hiderowcolors
\textbf{Totale} & 85,4 & 91,3 \\
\toprule
\multicolumn{3}{c}{\textbf{\textit{MAMMALIA}}} \\
%\showrowcolors
\textit{Rinolophus ferrumequinum} & 0,6 & 0,1 \\
\textit{Glis glis} & 2,8 & 4 \\
\textit{Moscardinus avellanarius} & 1,25 & 0,2 \\
\textit{Apodemus sp.} & 2,3 & 0,4 \\ 
\textit{Rattus sp.} & 3 & 2,8 \\
\toprule
\textbf{Totale} & 9,95 & 7,5 \\
\hiderowcolors
\multicolumn{3}{c}{\textbf{\textit{REPTILIA}}} \\
\toprule
%\showrowcolors
\textit{Podarcis sicula} & 2,3 & 0,5 \\
\textit{Lacerta bilineata} & 1,25 & 0,4 \\
\textit{Anguis veronensis} & 0,6 & 0,2 \\
\textit{Chalcides chalcides} & 0,5 & 0,1 \\
\textbf{Totale} & 4,65 & 1,2 \\
\bottomrule
\end{tabular}
}
\caption{}
\label{Angelini_tab_1}
\end{table}

\section*{Ringraziamenti}

Si ringraziano per i preziosi consigli e la collaborazione; Alessandro
Andreotti, Marco Andreini, Bruno D{\textquoteright}Amicis, Giovanni
Leonardi, Mauro Magrini, Paolo Perna, Carlo Poiani, Stefano Sassaroli,
Simonetta Turbessi, Aurelio Vitali.

\section*{Bibliografia}
\begin{itemize}\itemsep0pt
	\item Andreotti A. \& Leonardi G. (a cura di), 2007 - Piano
d{\textquoteright}azione nazionale per il Lanario (\textit{Falco
biarmicus feldeggii}). \textit{Quad. Cons. Natura}, 24, Min. Ambiente
-- Ist. Naz. Fauna Selvatica.

	\item Angelini J., 2007 - Lanario \textit{Falco biarmicus} in Giacchini P. (a
cura di), 2007 \textit{Atlante degli Uccelli Nidificanti nella
Provincia di Ancona}. Provincia di Ancona, IX Settore Tutela
dell{\textquoteright}Ambiente -- Area Flora e Fauna: 98-99.

	\item De Sanctis A., Di Meo D., Pellegrini M. \& Sammarone L., 2009 - Breeding
Biology and Diet of the Lanner \textit{Falco biarmicus feldeggii} in
the Abruzzo Region, Central Appennines. \textit{Alula} XVI (1-2):
170-175. 

	\item Morimando F., Pezzo F. \& Draghi A., 1997 - Food habits of the Lanner
Falcon (\textit{Falco biarmicus} \textit{feldeggii}) in Central Italy.
\textit{Journal of Raptors Research} 31 (1): 40-43.
\end{itemize}
