\setcounter{figure}{0}
\setcounter{table}{0}

\begin{adjustwidth}{-3.5cm}{0cm}
\pagestyle{CIOpage}
\authortoc{\textsc{Sorace A.}, \textsc{Corradini A.}, \textsc{Dematris P.}, \textsc{De Zuliani E.}, \textsc{Mazzarani D.}, \textsc{Monaco E.}, \textsc{Muratore S.}, \textsc{Piroli R.}}
\chapter*[]{\bfseries
L{\textquoteright}avifauna nidificante nella Riserva naturale Regionale di Macchiatonda}
\addcontentsline{toc}{chapter}{L{\textquoteright}avifauna nidificante nella Riserva naturale Regionale di Macchiatonda}

\textsc{Alberto Sorace}$^{1*}$, \textsc{Augusto Corradini}$^2$, \textsc{Patrizio Dematris}$^2$, \textsc{Emanuele De Zuliani}$^2$, \textsc{Donatella Mazzarani}$^2$, \textsc{Ernesto Monaco}$^{1, 2}$, \textsc{Sergio Muratore}$^{1, 2}$, \textsc{Riccardo Piroli}$^2$ \\

\index{Sorace Alberto} \index{Corradi Augusto} \index{Demartis Patrizio} \index{De Zuliani Emanuele} \index{Mazzarani Donatella} \index{Monaco Ernesto} \index{Muratore Sergio} \index{Piroli Riccardo}
\noindent\color{MUSEBLUE}\rule{27cm}{2pt}
\vspace{1cm}
\end{adjustwidth}

\marginnote{\raggedright $^1$SROPU, Via Roberto Crippa 60 D/8 00125 Roma (Acilia) \\
$^2$Riserva naturale Regionale di Macchiatonda, via del Castello 40 00050 Santa Severa (Roma) \\
\vspace{.5cm}
{\emph{\small $^*$Autore per la corrispondenza: \href{mailto:sorace@fastwebnet.it}{so\allowbreak ra\allowbreak ce@\allowbreak fast\allowbreak web\allowbreak net.\allowbreak it}}} \\
\keywords{Riserva naturale Regionale di Macchiatonda, Roma, avifauna nidificante}
{Macchiatonda Natural Regional Reserve, Roma, breeding species}
}

{\small
\noindent \textsc{\color{MUSEBLUE} Summary} / The territories of breeding birds were mapped during spring 2013, in Macchiatonda Natural Reserve. On the whole, 34
breeding species were observed. \textit{Emberiza calandra} (1,68 cp/10 ha) and \textit{Cisticola juncidis} (1,39 cp/10
ha) were the most abundant. Other 9 species breed near the Reserve and use this area for feeding. The richness of
species was highest in agricultural areas, which also host most of species of conservation concern.\\
}

\vspace{1cm}
La Riserva Naturale regionale di Macchiatonda è un{\textquoteright}area umida residuale del litorale tirrenico, 50 km a nord di Roma, in
un territorio pianeggiante che si sviluppa tra il mare e i monti della Tolfa. L'area comprende circa 180 ha di campi
coltivati estensivamente e 70 ha di lagune costiere, laureto e prati alofili. La presenza di questi particolari habitat
unita a una ricca comunità ornitica acquatica e alla presenza di specie ornitiche parasteppiche nei coltivi ha promosso l{\textquoteright}area a SIC e ZPS (\textit{IT6030019}).

Nella primavera 2013 è stata avviata una ricerca per approfondire le conoscenze sull{\textquoteright}avifauna nidificante. Lungo una serie di percorsi, ripetuti per quattro volte da aprile a giugno, è stata mappata la presenza delle diverse specie ornitiche all{\textquoteright}interno dell{\textquoteright}area protetta.

Sono state rilevate 34 specie nidificanti possibili, probabili o certe: \textit{Cygnus olor}, \textit{Anas
platyrhynchos}, \textit{Coturnix coturnix},\textit{ Phasianus colchicus}, \textit{Ardea purpurea},\textit{ Tachybaptus
ruficollis}, \textit{Gallinula chloropus}, \textit{Fulica atra}, \textit{Columba livia} dom., \textit{Streptopelia
decaocto}, \textit{Streptopelia turtur}, \textit{Upupa epops}, \textit{Melanocorypha calandra}, \textit{Galerida
cristata}, \textit{Alauda arvensis}, \textit{Hirundo rustica}, \textit{Luscinia megarhynchos}, \textit{Turdus merula},
\textit{Cettia cetti}, \textit{Cisticola juncidis}, \textit{Acrocephalus scirpaceus}, \textit{Hippolais polyglotta},
\textit{Sylvia melanocephala}, \textit{Sylvia cantillans}, \textit{Sylvia atricapilla}, \textit{Parus major},
\textit{Corvus cornix}, \textit{Pica pica}, \textit{Corvus monedula}, \textit{Sturnus vulgaris}, \textit{Passer
italiae}, \textit{Passer montanus}, \textit{Carduelis carduelis}, \textit{Emberiza calandra}. 

Altre 9 specie nidificano in aree vicine e frequentano la Riserva in periodo riproduttivo per scopi trofici:
\textit{Falco tinnunculus} e \textit{Merops apiaster} (in passato nidificanti nella Riserva), \textit{Falco
peregrinus}, \textit{Columba palumbus,} \textit{Apus apus}, \textit{Delichon urbicum}, \textit{Cyanistes caeruleus,}
\textit{Serinus serinus} \ e \textit{Carduelis chloris}. Nell{\textquoteright}area protetta sono risultate dominanti le seguenti
specie: \textit{Emberiza calandra} (1,68 cp/10 ha; pi = 0,224), \textit{Cisticola juncidis} (1,39 cp/10 ha; pi =
0,186), \textit{Passer italiae} (0,66 cp/10 ha; pi = 0,087), \textit{Luscinia megarhynchos} (0,57 cp/10 ha; pi =
0,077), \textit{Cettia cetti} (0,53 cp/10 ha; pi = 0,071) e \textit{Galerida cristata} (0,49 cp/10 ha; pi = 0,066).

Come atteso, la ricchezza di specie raggiunge il suo valore più elevato (23 specie) nella zona agricola, essendo
l{\textquoteright}estensione di questi ambienti maggiore rispetto agli altri. Nella zona degli stagni retrodunali sono state tuttavia
osservate le più alte densità di coppie nidificanti (15,2 cp/10 ha) e la maggiore percentuale di non Passeriformi
(38,2\%), a conferma del loro valore conservazionistico ed ecologico. 

A parte \textit{Ardea purpurea}, le altre 11 specie a priorità di conservazione (All. I di Dir. 2009/147/CE; BirdLife
International 2004, Peronace et al. 2012) presenti nella Riserva nidificano in ambienti agricoli, anche di elevato
pregio conservazionistico. Tra le specie a priorità di conservazione spicca la presenza di \textit{Melanocorypha
calandra} (0,16 cp/10 ha; pi = 0,022), specie inserita in All. 1 della Dir. 2009/147/CE, estremamente localizzata nel
Lazio.

\section*{Bibliografia}
\begin{itemize}\itemsep0pt
	\item BirdLife International, 2004. \textit{Birds in Europe: population estimates, trends and conservation status. BirdLife International}. (BirdLife Conservation Series No.12), Cambridge.
	\item Peronace V., Cecere J.G., Gustin M. \& Rondinini C., 2012. Lista Rossa 2011 degli Uccelli nidificanti in Italia. \textit{Avocetta} 36: 11-58.
\end{itemize}