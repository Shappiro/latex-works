\setcounter{figure}{0}
\setcounter{table}{0}

\begin{adjustwidth}{-3.5cm}{0cm}
\pagestyle{CIOpage}
\authortoc{\textsc{Morganti N.}, \textsc{Mencarelli M.},
\textsc{Morici F.}}
\chapter*[L{\textquoteright}avifauna dell{\textquoteright}oasi
faunistica di San Gaudenzio]{\bfseries
Aggiornamento dell{\textquoteright}avifauna presente
nell{\textquoteright}oasi faunistica di San Gaudenzio}
\addcontentsline{toc}{chapter}{L{\textquoteright}avifauna dell{\textquoteright}oasi
faunistica di San Gaudenzio}

\textsc{Niki Morganti}$^{1*}$, \textsc{Mauro Mencarelli}$^{1*}$,
\textsc{Francesca Morici}$^{1*}$  \\

\index{Morganti Niki} \index{Mencarelli Mauro} \index{Morici Francesca}
\noindent\color{MUSEBLUE}\rule{27cm}{2pt}
\vspace{1cm}
\end{adjustwidth}


\marginnote{\raggedright $^1$Studio Naturalistico Diatomea, Senigallia (AN)\\
\vspace{.5cm}
{\emph{\small $^*$Autore per la corrispondenza: \href{mailto:info@studiodiatomea.it}{in\allowbreak fo@\allowbreak stu\allowbreak dio\allowbreak dia\allowbreak to\allowbreak mea.\allowbreak it}}} \\
\keywords{Oasi di San Gaudenzio (AN), avifauna, monitoraggio}
{Oasi di San Gaudenzio (AN), bird communities, monitoring}
%\index{keywords}{Oasi di San Gaudenzio} \index{keywords}{Avifauna} \index{keywords}{Monitoraggio}
}
{\small
\noindent \textsc{\color{MUSEBLUE} Summary} / Three years after the last survey, we have monitored wintering and
breeding birds in the Oasi of San Gaudenzio (Marche, central Italy).
The resulting check-list includes 125 species, but breeding species
show an considerable decrease, except for the aquatic birds.\\
}

\vspace{1cm}
A distanza di tre anni dall{\textquoteright}ultimo monitoraggio, sono
stati condotti i rilevamenti dell{\textquoteright}avifauna svernante e
nidificante nell{\textquoteright}Oasi faunistica di San Gaudenzio
(Senigallia, AN). Inoltre, si \`e stata aggiornata la check-list
dell{\textquoteright}Oasi attraverso una ricerca bibliografica e la
richiesta di dati in possesso di altri \textit{birdwatcher} che
frequentano l{\textquoteright}area di studio. 

La zona oggetto di indagine \`e l{\textquoteright}Oasi faunistica di San
Gaudenzio che si estende per 36 ettari nelle colline in destra
orografica del fiume Misa, a pochi chilometri
dall{\textquoteright}abitato di Senigallia (AN). In passato la zona \`e
stata sfruttata come cava di marna che, a seguito
dell{\textquoteright}abbandono dell{\textquoteright}attivit\`a
estrattiva, si \`e rinaturalizzata spontaneamente: attualmente
insistono diverse tipologie ambientali: due laghetti di diverse
dimensioni e profondit\`a e la relativa vegetazione ripariale, scarpate
rocciose nude, arbusteti con dominanza della Ginestra, piccole aree
boschive costituite principalmente dalla Roverella e, infine, la
struttura della fornace della cava oramai abbandonata da oltre 50 anni
(Furlani \& Morici 2006).

I rilevamenti sono stati condotti due volte al mese nel periodo compreso
tra dicembre 2012 e luglio 2013. La metodologia utilizzata \`e stata
quella dei punti di ascolto/osservazione con sosta di 15 minuti in
ognuno, in linea con i monitoraggi precedenti in modo cos\`i da
ottenere dati e risultati confrontabili col passato. Per
l{\textquoteright}elaborazione dei dati dei nidificanti \`e stato
calcolato l{\textquoteright}Indice Puntiforme di Abbondanza (IPA).

Nel periodo di ricerca sono state censite 77 specie, 11 in meno rispetto
ai monitoraggi del 2009-2010. Di queste 77, 54 sono le specie
nidificanti (69 nel 2009-2010); anche il valore IPA totale risulta
nettamente inferiore rispetto alla precedente ricerca: 442,5 nel
2012-2013 e 725,5 nel 2009-2010. La check-list annovera 125 specie, di
cui quattro (corriere piccolo, piro piro culbianco, topino e cincia
mora), osservate nel 2013, non erano state censite prima.

L{\textquoteright}Oasi faunistica di San Gaudenzio, pur se di limitata
estensione, mostra una notevole ricchezza avifaunistica: la ragione
principale di ci\`o pu\`o essere data dalla variet\`a di ambienti
presenti al suo interno. Rispetto al 2009-2010, i dati mostrano una
sensibile perdita della componente nidificante: le uniche specie che
mostrano un aumento sono quelle che nidificano in habitat umidi (es.
Folaga) e qualcuna di habitat boschivi, come il colombaccio. Tutte le
altre, in particolare le specie di habitat arbustivi, sono in netto
calo. Dal 2009 non si sono verificati cambiamenti nelle condizioni
vegetazionali o infrastrutturali dell{\textquoteright}area, pertanto le
ragioni di questi mutamenti della componente avifaunistica potrebbero
essere legate alle cattive stagioni primaverili e agli inverni rigidi
verificatisi negli ultimi 2 anni: in conseguenza di ci\`o, non solo a
San Gaudenzio ma anche a scala nazionale, sono scomparse alcune specie
o drasticamente calate le loro dimensioni di popolazione, come il
beccamoschino, lo strillozzo e il saltimpalo. Altre specie in netto
calo sono rondine, rondone comune, passera d{\textquoteright}Italia,
passera mattugia, ma le cause sono probabilmente da ricercare a pi\`u
vasta scala. 

\section*{Bibliografia}
\begin{itemize}\itemsep0pt
	\item Furlani M. \& Morici F., 2006 -- Caratteri naturalistici della cava di
San Gaudenzio. In: Villani V. e Mauro F. (a cura di).
	\item L{\textquoteright}Oasi di San Gaudenzio di Senigallia. Valori storici
ed ambientali. Gruppo {\textquotedblleft}Societ\`a e
Ambiente{\textquotedblright}, Senigallia.
\end{itemize}

