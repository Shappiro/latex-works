\setcounter{figure}{0}
\setcounter{table}{0}

\begin{adjustwidth}{-3.5cm}{0cm}
\pagestyle{CIOpage}
\authortoc{\textsc{Bedini G.}, \textsc{Paoli F.},
\textsc{Galli S.}, \textsc{Ceccherelli R.},
\textsc{Nuti F.}, \textsc{Montagnani D.},
\textsc{Baccetti N.}, \textsc{Azafzat H.},
\textsc{Bouagina A.}, \textsc{Zintu P.}, 
\textsc{Gherardi R.}}
\chapter*[Il recupero di un corrione biondo in Toscana]{Toscano per accidente: l{\textquotesingle}avventura italiana di
un giovane corrione biondo \textbf{\textit{Cursorior cursor}}\textbf{
prima del rimpatrio in Tunisia}}
\addcontentsline{toc}{chapter}{Il recupero di un corrione biondo in Toscana}

{\raggedleft
\textsc{Gianluca Bedini}$^{1}$, \textsc{Federica Paoli}$^{1}$,
\textsc{Silvia Galli}$^{1}$, \textsc{Renato Ceccherelli}$^{1}$,
\textsc{Fabio Nuti}$^{1}$, \textsc{David Montagnani}$^{2}$,
\textsc{Nicola Baccetti}$^{3}$, \textsc{Hichem Azafzat}$^{4}$,
\textsc{Adel Bouagina}$^{4}$, \textsc{Paola Zintu}$^{5}$, 
\textsc{Riccardo Gherardi}$^{1*}$ } \\

\index{Bedini Gianluca} \index{Paoli Federica} \index{Galli Silvia} \index{Ceccherelli Renato} \index{Nuti Fabio} \index{Montagnani David} \index{Baccetti Nicola} \index{Azafzat Hichem} \index{Bouagina Adel} \index{Zintu Paola} \index{Gheraridi Riccardo}
\noindent\color{MUSEBLUE}\rule{27cm}{2pt}
\vspace{1cm}
\end{adjustwidth}



\marginnote{\raggedright $^1$CRUMA -- LIPU, Via delle sorgenti 430, 57121 Livorno
(LI), Italy \\
$^2$Clinica Veterinaria Aurelia, via Aurelia
136/A, 57017 Stagno, Collesalvetti (LI), Italy \\
$^3$Istituto Superiore per la Protezione e la Ricerca
Ambientale (ISPRA), via Ca{\textquotesingle} Fornacetta 9, 40064 Ozzano
Emilia (BO), Italy \\
$^4$Association {\textquotedbl}Les Amis des
Oiseaux{\textquotedbl} (AAO), Ariana Center, Bureau C 208/209, 2080
Ariana, Tunisie \\
$^5$CRAS WWF, {\textquotedblleft}L{\textquoteright}Assiolo{\textquotedblright}, via
Donizetti Loc. Ronchi, 54100 Marina di Massa (MS), Italy \\
\vspace{.5cm}
{\emph{\small $^*$Autore per la corrispondenza: \href{mailto:ric_gherardi@hotmail.com}{ric\_\allowbreak ghe\allowbreak rar\allowbreak di@\allowbreak hot\allowbreak ma\allowbreak il.\allowbreak com}}} \\
\keywords{\textit{Cursorior cursor}, WRC, Toscana, Tunisia}
{\textit{Cursorior cursor}, WRC, Tuscany, Tunisia}
%\index{keywords}{\textit{Cursorior cursor}} \index{keywords}{WRC} \index{keywords}{Toscana} \index{keywords}{Tunisia}
}
{\small
\noindent \textsc{\color{MUSEBLUE} Summary} / On the 20 October 2011 a juvenile cream-colored courser
\textit{Cursorius cursor} was recovered after a heavy storm near Massa
-- Tuscany. The bird was hospitalized in two Tuscan WRC till its
complete rehabilitation. In June 2012, after 254 days from its
hospitalization, it was released in Bou Hedma National Park -- Tunisia.\\
}

\vspace{1cm}
Il corrione biondo \textit{Cursorius cursor}, \`e un Caradriforme di
190-210 mm di lunghezza, tipicamente legato ad ambienti semi-desertici
(Brichetti \& Fracasso 2004). In Italia risultano 125 segnalazioni
della specie relative al periodo 1817-2011, per un totale di 136-138
individui, (Verducci \textit{et al.} 2012). 

Il 20 ottobre 2011 un giovane esemplare di questa specie \`e stato
rinvenuto presso la zona industriale della citt\`a di Massa, in
Toscana. L{\textquoteright}animale \`e stato portato presso il C.R.A.S. W.W.F.
l{\textquoteright}Assiolo di Ronchi (MS) dalle persone che lo avevano
rinvenuto in difficolt\`a dopo un intenso temporale. Alla prima visita
il soggetto si presentava iporeattivo, bagnato e in leggero stato di
debilitazione. Come intervento di primo soccorso sono stati
somministrati fluidi riscaldati sottocute, vitamine e un antibiotico di
copertura.

Il giorno successivo \`e stato trasferito presso il C.R.U.M.A. della
L.I.P.U. di Livorno dove la terapia \`e proseguita per 7 giorni. Presso
tale Centro sono stati effettuati alcuni esami diagnostici di routine.
In seguito, sono stati prelevati campioni biologici per esami
specifici, effettuati presso laboratori esterni, necessari in
previsione del suo trasferimento in Nord Africa. 

Durante tutto il periodo di ricovero, il peso del corrione biondo \`e
stato monitorato a cadenza settimanale. Dopo un iniziale calo,
l{\textquoteright}animale ha ripreso ad alimentarsi regolarmente fino a
che il peso si \`e stabilizzato su valori normali per la specie.

Dopo un periodo iniziale di degenza in stabulazione stretta, finalizzata
a una miglior gestione e controllo delle sue condizioni,
l{\textquoteright}animale \`e stato trasferito in un box esterno
protetto da rete oscurante e dotato di arricchimento ambientale idoneo.
Questa nuova sistemazione gli ha permesso di godere di maggiori spazi
per muoversi e di maggior tranquillit\`a. 

In questo box l{\textquoteright}esemplare ha passato tutto il periodo
invernale grazie anche all{\textquoteright}aiuto di una lampada
riscaldante a infrarossi istallata per mitigare la temperatura di un
inverno che si \`e rivelato particolarmente rigido e nevoso per la
costa toscana.

Dopo circa un mese dal ricovero le sue condizioni fisiche risultavano
stabili e iniziava con successo i primi tentativi di volo.

A giugno, con l{\textquoteright}animale completamente ristabilito, \`e
stato possibile organizzare la sua liberazione in Tunisia, grazie anche
all{\textquoteright}intervento dell{\textquoteright}Ambasciatore
tunisino a Roma e alla compagnia di volo Tunisair.

Il 30 giugno 2012, dopo un giorno di osservazione in Tunisia presso le
strutture della \textit{Association  les Amis des Oiseaux}, e dopo 254
giorni dal suo rinvenimento, il corrione biondo \`e stato liberato in
una zona desertica del Parco Nazionale di Bou Hedma.

\begin{figure}[!h]
\centering
\includegraphics[width=.68\columnwidth]{Bedini_fig_1.jpg}
\caption{Corrione biondo con l{\textquoteright}ala destra aperta per mostrare l{\textquoteright}inizio della muta visibile nelle due remiganti primarie pi\`u interne}
\label{Bedini_fig_1}
\end{figure}

\section*{Ringraziamenti}

\`E intenzione degli autori ringraziare l{\textquoteright}ambasciatore
Tunisino a Roma Dott. Naceur Mestiri che si \`e prodigato per rendere
possibile l{\textquoteright}operazione e senza il cui aiuto tutta
questa avventura non solo sarebbe stata molto pi\`u complicata, ma non
avrebbe sicuramente ottenuto un cos\`i alto valore culturale. Inoltre,
i ringraziamenti degli autori vanno alla Dottoressa Sandra Nannipieri,
medico veterinario presso la ASL 6 di Livorno, per
l{\textquoteright}aiuto fornito nello svolgimento degli adempimenti
burocratici relativi all{\textquoteright}espatrio del corrione biondo e
alla Compagnia Aerea Tunisair per averci dato la possibilit\`a di fare
volare gratuitamente il corrione biondo da Fiumicino a Tunisi.



\section*{Bibliografia}
\begin{itemize}\itemsep0pt
	\item Brichetti P. \& Fracasso G., 2004 - \textit{Ornitologia Italiana.
2{\textdegree} volume Tetraonidae-Scolopacidae.} Alberto Perdisa
Editore, Ozzano dell{\textquotesingle}Emilia (BO): 396 pp.

	\item Verducci D., Biondi M., Sighele M. \& Norante N., 2012 - Revisione degli
avvistamenti e delle catture di corrione biondo \textit{Cursorius
cursor} in Italia con cenni sul suo status in Europa. \textit{U.D.I.
XXXVII:} 16-32.
\end{itemize}
