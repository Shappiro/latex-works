\setcounter{figure}{0}
\setcounter{table}{0}

\begin{adjustwidth}{-3.5cm}{0cm}
\pagestyle{CIOpage}
\authortoc{\textsc{Massa B.}, \textsc{La Mantia T.}}
\chapter*[Conservazione degli uccelli pelagici nelle isole Pelagie]{Conservazione della principale popolazione europea di berta
maggiore mediterranea \textbf{\textit{Calonectris diomedea}}\textbf{ e
altri uccelli pelagici nelle isole Pelagie}}
\addcontentsline{toc}{chapter}{Conservazione degli uccelli pelagici nelle isole Pelagie}

\textsc{Bruno Massa}$^{1*}$, \textsc{Tommaso La Mantia}$^{1}$ \\

\index{Massa Bruno} \index{La Mantia Tommaso}
\noindent\color{MUSEBLUE}\rule{27cm}{2pt}
\vspace{1cm}
\end{adjustwidth}



\marginnote{\raggedright $^1$Dipartimento SAF - Viale delle Scienze, Ed. 4,
Ingresso H, 90128 Palermo \\
\vspace{.5cm}
{\emph{\small $^*$Autore per la corrispondenza: \href{mailto:bruno.massa@unipa.it}{bruno.massa@unipa.it}}} \\
\keywords{\textit{Calonectris diomedea}, \textit{Puffinus
yelkouan}, \textit{Hydrobates pelagicus melitensis}, isole Pelagie,
conservazione}
{\textit{Calonectris diomedea}, Puffinus yelkouan, \textit{Hydrobates
pelagicus melitensis}, Pelagie islands, conservation}
%\index{keywords}{\textit{Calonectris diomedea}} \index{keywords}{\textit{Puffinus yelkouan}} \index{keywords}{\textit{Hydrobates pelagicus melitensis}} \index{keywords}{Isole Pelagie} \index{keywords}{Conservazione}
}
{\small
\noindent \textsc{\color{MUSEBLUE} Summary} / Authors report on the actions of the Life+ Project Nat/It 00093
{\textquotedbl}Pelagic Birds{\textquotedbl}, aimed to the conservation
of pelagic birds in the islands of Linosa and Lampedusa (Sicily,
Italy). One aim of the project concerns the eradication of the black
rat from the island of Linosa, where it causes a high loss of breeding
success of the scopoli's shearwater \textit{Calonectris diomedea}.\\
}

\vspace{1cm}
La Commissione Europea ha finanziato il progetto Life+ Nat/It 00093
{\textquotedbl}Pelagic Birds{\textquotedbl},
({\textquotedblleft}Conservation of the main European population of
\textit{Calonectris d. diomedea} and other pelagic birds on Pelagic
Islands{\textquotedblright})\textbf{ }il cui capofila \`e il
Dipartimento di Scienze agrarie e forestali
dell'Universit\`a di Palermo e i cui partner
beneficiari sono il Dipartimento Regionale Sviluppo Rurale e
Territoriale (ex Azienda Foreste Demaniali della Regione Siciliana) e
le associazioni ambientaliste Legambiente e Fare Ambiente (cfr.
http://www.pelagicbirds.eu/). Il progetto ha come scopo principale la
conservazione degli uccelli pelagici e in particolare
dell{\textquoteright}endemica berta maggiore mediterranea,
\textit{Calonectris diomedea,} nelle isole Pelagie.

L'isola di Linosa (Agrigento, Canale di Sicilia) ospita
la maggiore colonia europea di berta maggiore mediterranea: circa
10.000 coppie stimate, pari a oltre il 60\% della popolazione italiana
e a oltre il 20\% della popolazione europea; a livello globale, la
colonia \`e seconda solo a quella di Zembra (Tunisia). La berta
maggiore nidifica inoltre con una piccola popolazione
nell'isola di Lampedusa e con circa 200 coppie
nell'isolotto di Lampione. Le colonie si trovano
nell'area \textit{core} per la specie, in cui le
azioni di conservazione possono incidere in modo determinante sullo
stato di conservazione dell'intero taxon; le altre
grandi colonie si ritrovano infatti tutte nel Canale di Sicilia: oltre
che a Zembra (Tunisia), a Malta e Pantelleria (Sicilia). Il progetto
prevede di intervenire sulle popolazioni di ratto nero \textit{Rattus
rattus}, una delle specie che pi\`u interferisce con la sopravvivenza
delle berte nelle isole italiane (Baccetti \textit{et al}. 2009), oltre
che su due piante aliene: \textit{Carpobrotus} spp. e \textit{Nicotiana
glauca}.

Poich\'e si tratta di contesti territoriali circoscritti, e spesso
semplificati, le isole appaiono di frequente vulnerabili
all{\textquoteright}invasione di organismi esotici, come osservato in
molti casi su scala globale (Simberloff 1995) e nel Mediterraneo in
particolare (Pretto \textit{et al}. 2010; Pasta \& La Mantia 2008). 
Diversi studi (D{\textquoteright}Antonio 1990; Vil\`a \&
D{\textquoteright}Antonio 1998; Bourgeois \textit{et al}. 2005) hanno
evidenziato come \textit{Carpobrotus} spp. costituisca una risorsa
trofica d{\textquoteright}importanza cruciale per la dieta invernale
del ratto nero. \textit{N. glauca} colonizza esclusivamente ambienti
soggetti a forte disturbo antropico, luoghi generalmente frequentati
dai ratti. L{\textquoteright}eradicazione di queste due piante potrebbe
indirettamente ridurre l{\textquoteright}habitat idoneo al ratto nero.
Il progetto, della durata di 55 mesi (chiusura: 31 dicembre 2016),
prevede la realizzazione di 22 azioni, di cui 3 preparatorie, tre di
conservazione, 2 di monitoraggio, 7 di disseminazione e 7 di gestione.

\section*{Bibliografia}
\begin{itemize}\itemsep0pt
	\item Baccetti N., Capizzi D., Corbi F., Massa B., Nissardi S., Spano G. \&
Sposimo P., 2009 - Breeding shearwaters on Italian islands: population
size, island selection and co-existence with their main alien predator,
the Black rat. \textit{Riv. ital. Orn}., Milano, 78 (2): 83-100.

	\item Bourgeois K., Suehs C.M., Vidal \'E. \& M\'edail F., 2005 - Invasional
meltdown potential: Facilitation between introduced plants and mammals
on French Mediterranean islands. \textit{Ecoscience}, 12: 248-256. 

	\item D{\textquoteright}Antonio C.M., 1990 - Seed production and dispersal in
the non native, invasive succulent \textit{Carpobrotus edulis
}(Aizoaceae) in coastal strand communities of central California.
\textit{J. Appl. Ecol}., 27: 693-702. 

	\item Pasta S. \& La Mantia T., 2008 - Le specie vegetali aliene in alcuni SIC
siciliani: analisi del grado di invasivit\`a e misure di controllo.
\textit{Mem. Soc. it. Sci. nat. Mus. civ. Stor. nat. Milano}, Milano,
36 (1): 79. 

	\item Pretto F., Celesti-Grapow L., Carli E. \& Blasi C., 2010 - Influence of
past land use and current human disturbance on non-native plant species
on small Italian islands. \textit{Plant Ecol}., 210: 225-239;
Simberloff D., 1995 - Why do introduced species appear to devastate
islands more than mainland areas? \textit{Pacific Science}, 49: 87-97. 

	\item Vil\`a M. \& D{\textquoteright}Antonio C.M., 1998 - Fruit choice and
seed dispersal of invasive vs. non-invasive \textit{Carpobrotus
}(Aizoaceae) in coastal California. \textit{Ecology}, 79 (3):
1053-1060.
\end{itemize}
