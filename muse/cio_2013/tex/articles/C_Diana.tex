\setcounter{figure}{0}
\setcounter{table}{0}


\begin{adjustwidth}{-3.5cm}{-1cm}
\pagestyle{CIOpage}
\authortoc{\textsc{Diana F.}, \textsc{Pedrotti L.}, 
\textsc{Sartirana F.}, \textsc{Trotti P.},
\textsc{Galli L.}, \textsc{Bassi E.}}
\chapter*[Cure parentali nell{\textquoteright}aquila reale e nel
gipeto]{Cure parentali nell{\textquoteright}aquila reale
\textbf{\textit{Aquila chrysaetos}}\textbf{ e nel gipeto
}\textbf{\textit{Gypaetus barbatus}} \textbf{in una popolazione delle Alpi italiane}}
\addcontentsline{toc}{chapter}{Cure parentali nell{\textquoteright}aquila reale e nel
gipeto}
\end{adjustwidth}
\begin{adjustwidth}{-3.5cm}{0cm}
\textsc{Francesca Diana}$^{1*}$, \textsc{Luca Pedrotti}$^{1}$, 
\textsc{Fabiano Sartirana}$^{1}$, \textsc{Paolo Trotti}$^{1}$,
\textsc{Loris Galli}$^{2}$, \\\textsc{Enrico Bassi}$^{1**}$\\
\index{Diana Francesca} \index{Pedrotti Luca} \index{Sartirana Fabiano} \index{Trotti Paolo} \index{Galli Loris} \index{Bassi Enrico}
\noindent\color{MUSEBLUE}\rule{27cm}{2pt}
\vspace{1cm}
\end{adjustwidth}



\marginnote{\raggedright $^1$Consorzio del Parco Nazionale dello Stelvio Via De
Simoni 42, 23032 Bormio (SO) \\
$^2$Universit\`a degli Studi di Genova Corso Europa 26,
16132 Genova \\
\vspace{.5cm}
{\emph{\small $^*$Autore per la corrispondenza: \href{mailto:francesca.diana84@yahoo.it}{fran\allowbreak ce\allowbreak sca.\allowbreak dia\allowbreak na\allowbreak 84@\allowbreak ya\allowbreak hoo.\allowbreak it}}} \\
{\emph{\small $^**$Autore per la corrispondenza: \href{mailto:enrico.bassi76@gmail.com}{en\allowbreak ri\allowbreak co.\allowbreak bas\allowbreak si76@\allowbreak g\allowbreak ma\allowbreak il.\allowbreak com}}} \\
\keywords{Alpi, \textit{Aquila chrysaetos}, \textit{Gypaetus
barbatus}, biologia riproduttiva, cure parentali, comportamento}
{Alps, \textit{Aquila chrysaetos}, \textit{Gypaetus
barbatus}, breeding biology, parental care, behaviour}
%\index{keywords}{Alpi} \index{keywords}{\textit{Aquila chrysaetos}} \index{keywords}{\textit{Gypaetus
%barbatus}} \index{keywords}{Biologia riproduttiva} \index{keywords}{Cure parentali} \index{keywords}{Comportamento}
}
{\small
\noindent \textsc{\color{MUSEBLUE} Summary} / A field research on golden eagle \textit{Aquila chrysaetos} and bearded
vulture \textit{Gypaetus barbatus}
was performed during 2008-2011 in Stelvio National Park (central Alps,
Northern Italy), focusing on time budget of breeding pairs of both
species. The study area hosts a breeding population of 14 pairs of
golden eagle and 4 pairs of bearded vulture, nesting at short distance.
During 4 field seasons, 16 breeding events of golden eagle and 11 of
bearded vulture have been monitored for 289 observation days and
behaviour patterns of nesting adults of both species were recorded,
analysed and compared. Data analysis included the application of linear
mixed models. Results were compared between species and with other
study areas. Time dedicated to parental cares significantly differed
between sexes and species. In both species females spent more time than
males in parental cares activities, but golden eagle females dedicated
65\% of time in parental cares and males 33\%, while bearded vulture
females 54\% and males 38\%. Differences between species might be
linked to different feeding behaviour, use of the territory and
climatic conditions occurred in the respective breeding period.  \\
\noindent \textsc{\color{MUSEBLUE} Riassunto} / Nel quadriennio 2008-2011, nell{\textquoteright}area del Parco Nazionale
dello Stelvio e del suo intorno (Alpi centrali italiane), sono stati
condotti studi approfonditi inerenti le cure parentali delle coppie
riproduttive di aquila reale e gipeto. L{\textquoteright}area di studio
ospita 4 coppie di gipeto e 14 coppie di aquila reale, nidificanti tra
loro a breve distanza. Nel corso di 4 stagioni di campo sono state
monitorate 16 nidificazioni di aquila reale e 11 nidificazioni di
gipeto, durante 289 giornate di osservazione.
Nell{\textquoteright}ambito dello studio sono stati registrati,
confrontati e analizzati i comportamenti parentali di entrambe le
specie, utilizzando modelli lineari misti. I risultati sono stati
paragonati con quelli di altre aree di studio e si sono confrontati i
comportamenti delle due specie. Il tempo dedicato alle cure parentali
differisce significativamente tra le due specie e tra maschio e
femmina. In entrambe le specie la femmina dedica pi\`u tempo alle cure
parentali rispetto al maschio, ma le femmine di aquila reale utilizzano
il 65\% del tempo in tali attivit\`a e i maschi il 33\%, mentre nel
gipeto le femmine dedicano il 54\% del tempo alla prole rispetto al
38\% dei maschi. Le differenze tra le specie potrebbero essere connesse
alle differenti modalit\`a di ricerca del cibo, all{\textquoteright}uso
del territorio e alle rispettive condizioni climatiche registrate
durante il periodo riproduttivo. \\
}
\section*{Introduzione}

Sui rapaci, una grande mole di studi disponibili in letteratura riguarda
la densit\`a, l{\textquoteright}uso dell{\textquoteright}habitat, la
dieta e il successo riproduttivo mentre ricerche per
l{\textquoteright}analisi e la quantificazione del comportamento
parentale sono assai rare sia per la difficolt\`a di indagine sia per
l{\textquoteright}elevato sforzo di campo richiesto (Collopy 1984;
Margalida \& Bertran 2000).

Dal 2004, nel Parco Nazionale dello Stelvio (PNS) e nel suo intorno, \`e
in corso un monitoraggio intensivo delle popolazioni di aquila reale
\textit{Aquila chrysaetos} e gipeto \textit{Gypaetus barbatus. }Oltre
al monitoraggio ordinario delle coppie nidificanti, tra il 2008 e il
2011, si \`e inoltre impostato un programma di ricerca mirato ad
approfondire alcuni aspetti del comportamento e della biologia
riproduttiva delle due specie. In particolare, la presente ricerca ha
permesso di raccogliere informazioni sulle attivit\`a parentali degli
adulti impegnati nel ciclo riproduttivo, come i tempi di cova e la
frequenza dei cambi al nido, in rapporto con il sesso del genitore, con
le fasi del ciclo riproduttivo (cova, periodo di post-schiusa e
pre-involo) e con la fascia oraria. Informazioni di dettaglio, non
considerate nel presente lavoro, sono state raccolte anche
sull{\textquoteright}alimentazione, lo sviluppo dei giovani e
l{\textquoteright}influenza dell{\textquoteright}attivit\`a
antagonistica tra le due specie sul comportamento parentale e sulla
produttivit\`a (Bassi \textit{et al}., \textit{in stampa}). Per quanto
riguarda il gipeto i dati raccolti sono, al momento, gli unici
disponibili a livello dell{\textquoteright}arco alpino.

\section*{Area di studio}

L{\textquoteright}area di studio \`e localizzata nelle Alpi centrali
italiane e comprende il settore lombardo del Parco Nazionale dello
Stelvio (60.126 ha) in alta Valtellina (SO) e alta val Camonica (BS) e
alcune valli laterali poste nell{\textquoteright}intorno del Parco.
L{\textquoteright}area, caratterizzata dall{\textquoteright}abbondanza
di estese pareti rocciose calcaree e metamorfiche, boschi di conifere,
prati-pascoli e praterie alpine d{\textquoteright}alta quota, ospita
elevate densit\`a medie di ungulati selvatici (6,7
camosci/km\textsuperscript{2}, 5-25 cervi/km\textsuperscript{2} e 1150
stambecchi; Carro \& Pedrotti 2010). Siti di alimentazione artificiale
non sono mai stati allestiti. Nell{\textquoteright}area di studio sono
presenti 14 territori di aquila reale e 4 di gipeto (Bassi 2011).

\section*{Metodi}

Il comportamento degli adulti di aquila reale e gipeto impegnati nelle
cure parentali \`e stato studiato per 4 stagioni riproduttive
(2008-2011) nel corso delle seguenti tre fasi: cova (aquila reale 42
giorni; gipeto 53-55 giorni), post-schiusa (aquila reale 21 giorni;
gipeto 28 giorni) e pre-involo (dalla fine della fase di post-schiusa
fino all{\textquoteright}involo). 

I dati sono stati raccolti tramite osservazione diretta (\textit{focal
sampling}) dei nidi attivi in ciascuna stagione riproduttiva
utilizzando binocoli 10-12 x e cannocchiali 20-60 ingrandimenti. 

In generale, in etologia, il metodo del campionamento focale
(\textit{focal sampling} o campionamento dell{\textquoteright}animale
focale) consiste nell{\textquoteright}osservazione di un solo individuo
per un periodo di tempo stabilito (Martin \& Bateson 1986), durante il
quale vengono annotate tutte le sue azioni (\textit{time budget}).
Nell{\textquoteright}ambito della presente ricerca sono stati
utilizzati come animali focali entrambi gli adulti e i nidiacei di
ciascuna delle coppie riproduttive, osservabili da punti fissi di
osservazione; i comportamenti registrati sono stati standardizzati e
archiviati tramite la compilazione di schede di rilevamento. 

I punti fissi di osservazione erano posti a distanza compresa tra i 300
e i 2250 m dai nidi occupati di aquila reale (media= 1.071 m {\textpm}
677) e tra i 1000 e i 2510 m dai nidi usati dal gipeto (media= 1.251 m
{\textpm} 430), in modo da non arrecare disturbo alla nidificazione.
Maschio e femmina sono stati distinti in base a criteri morfologici,
comportamentali e ai segni di muta. 

Per l{\textquoteright}osservazione dei nidi (16 di aquila reale e 11 di
gipeto) sono stati complessivamente spesi 289 giorni di campo; 145 per
l{\textquoteright}aquila reale (1132 h) e 144 per il gipeto (1133 h),
con una durata media delle osservazioni pari a 7,8 h. 

Le variabili comportamentali registrate durante tale periodo sono state
suddivise in {\textquotedblleft}discontinue{\textquotedblright} e
{\textquotedblleft}continue{\textquotedblright}. Le attivit\`a
discontinue sono \textit{pattern} comportamentali di durata
relativamente breve che possono essere approssimati come punti nella
linea temporale, la cui caratteristica saliente \`e la frequenza,
espressa come numero di eventi per unit\`a di tempo. Le attivit\`a
discontinue analizzate, espresse come frequenza giornaliera (numero per
giornata), sono stati i cambi al nido e il trasporto di cibo al nido.
Sono state invece considerate come attivit\`a continue quelle azioni la
cui caratteristica principale \`e la durata del singolo
\textit{pattern} comportamentale, espressa in unit\`a di tempo
(minuti). 

Le principali attivit\`a continue registrate durante le osservazioni
oggetto di analisi sono state: cova delle uova, riscaldamento del
nidiaceo e sua alimentazione (tempo dedicato alla preparazione del
cibo, all{\textquoteright}imbeccata e all{\textquoteright}alimentazione
attiva da parte del giovane), sorveglianza (controllo del nidiaceo da
parte dell{\textquoteright}adulto, senza comportamenti di riscaldamento
o alimentazione) e l{\textquoteright}insieme delle cure parentali
(somma di tutti i precedenti comportamenti riferiti agli adulti).

Per il confronto della significativit\`a statistica delle eventuali
differenze nel \textit{time budget} (espresso come percentuale di
attivit\`a dedicate alle cure parentali) nelle due specie, e in
relazione al sesso e alla fase di sviluppo del giovane, sono stati
utilizzati modelli lineari misti. 

\section*{Risultati e discussione}

Nelle coppie seguite di aquila reale il periodo compreso tra la
deposizione e l{\textquoteright}involo \`e durato in media 117 giorni
(\textit{range} 111-129 giorni), mentre in quelle di gipeto \`e
risultato di durata significativamente maggiore con una media di 176
giorni (\textit{range} 162-193 giorni).

Nell{\textquoteright}aquila reale si sono registrati 126 cambi al nido
con una media di 2,5/giorno nella fase di cova (N= 87) e di 1,1 nella
fase di post schiusa (N= 39). 

Per il gipeto sono stati osservati 166 cambi al nido (media: 1,2
cambi/giorno), con una media pi\`u alta, pari a 1,6, nelle fasi di cova
(N= 69) e post schiusa (N= 36). 

Il numero di episodi di trasporto prede al nido per giornata da parte
dell{\textquoteright}aquila reale \`e stato pari a 0,4 nel periodo di
post schiusa e 0,5 in fase di pre-involo (N= 56); 31 di questi, si
riferiscono alla femmina (55\%), 2 ad adulti indeterminati (4\%) e 23
al maschio (41\%). 

Tali dati risultano sensibilmente inferiori a quanto riportato per
l{\textquoteright}Idaho (USA) da Collopy (1984), che indica una media
di 1,2 prede/giorno nel periodo di cova, 1,5 dalla schiusa alla quinta
settimana del giovane, 2,6 tra la sesta e l{\textquoteright}ottava
settimana e 1,6 nelle ultime settimane prima
dell{\textquoteright}involo. 

Nel gipeto entrambi i membri della coppia hanno contribuito al trasporto
di cibo al nido: la femmina nel 50\% dei casi osservati (N= 18) e il
maschio 44,4\% (N= 16), gli adulti indeterminati sono stati solo 2.

Dalla deposizione all{\textquoteright}involo, gli adulti di entrambe le
specie mostrano un comportamento parentale simile, per quanto riguarda
il tempo totale dedicato alla prole. Durante la fase di cova e nei
primi giorni successivi alla schiusa, l{\textquoteright}attivit\`a di
riscaldamento \`e costante mentre, con il progredire dello sviluppo del
pulcino, si osserva una graduale diminuzione, fino alla definitiva
cessazione nel periodo che precede l{\textquoteright}involo. Il tempo
percentuale medio dedicato alla cova, rispetto al tempo totale di
osservazione compreso tra la deposizione e l{\textquoteright}involo,
risulta significativamente differente tra le due specie (F= 4.00; p=
0.0461); il gipeto (stima {\textpm} e.s.= 0,33 {\textpm} 0,015) investe
pi\`u tempo in tale comportamento rispetto all{\textquoteright}aquila
reale (stima {\textpm} e.s.= 0,29 {\textpm} 0,014).

Nell{\textquoteright}aquila reale, nel periodo compreso tra la schiusa e
l{\textquoteright}involo, il pulcino \`e stato nutrito dalla femmina
per il 76,1\% del tempo (con un tempo medio di imbeccata o di
preparazione del cibo di 19 minuti); dal maschio per il 15,7\% (tempo
medio 14 minuti) e da un adulto indeterminato per
l{\textquoteright}8,2\%. La prima osservazione di giovane che si nutre
da solo \`e avvenuta dopo 40-51 giorni dalla schiusa, sulla media dei
nidi osservati.

Nel gipeto, nei primi giorni dopo la schiusa, il pulcino viene nutrito
dall{\textquoteright}adulto per un tempo massimo consecutivo di 70
minuti. Durante la fase di post schiusa il pulcino \`e stato nutrito
dalla femmina per il 75,5\% e dal maschio per il 24,5\%. 

Nella fase di pre-involo il pulcino \`e stato nutrito dalla femmina per
il 65,8\% e dal maschio per il 34,2\%. Nel gipeto, la prima
osservazione di un giovane che si nutre autonomamente \`e stata
registrata a 66 giorni di vita.

In entrambe le specie, sul totale delle osservazioni, le femmine (ff),
dalla cova fino all{\textquoteright}involo, investono pi\`u tempo nelle
cure parentali (intese come comprensive di cova delle uova,
riscaldamento, protezione da predatori e alimentazione del nidiaceo)
rispetto ai maschi (mm), (stima percentuale di tempo su tempo totale
{\textpm} e.s. ff = 0,476 {\textpm} 0,013; stima percentuale di tempo
su tempo totale {\textpm} e.s. mm = 0,304 {\textpm} 0,013; F= 90,97;
p{\textless} 0,001). 

Tenendo conto del totale del tempo che \`e stato dedicato alle cure
parentali da parte delle due specie durante le osservazioni, per
ciascuna specie, le percentuali suddivise per sesso sono risultate le
seguenti: aquila reale: ff 65,1, mm 32.8, indet. 2,1; gipeto: ff 54,4,
mm 37,8, indet. 7,8.

Analogamente le femmine di aquila reale sono risultate pi\`u dedite alle
cure parentali rispetto ai maschi, nelle prime tre settimane di vita
del pulcino (ff 71,6\%, mm 23,7\%, indet. 4,7\%) e, in maniera quasi
esclusiva, nella fase di preinvolo (ff 95,1\%, mm 4,9\%). 

Analoghi risultati sono stati ottenuti sulle Alpi svizzere tramite
l{\textquoteright}osservazione di 19 nidificazioni di aquila reale (per
un totale di 583 ore spese sul campo), in cui la femmina occupava il
nido per il 70,8\% {\textpm} 14,9\% del tempo giornaliero mentre il
maschio per il 18,4 {\textpm} 15,5\% (Jenny 1992).

Nel gipeto invece il contributo del maschio alle cure parentali
(espresso come percentuale di tempo dedicato alle cure parentali su
tempo totale di osservazione) \`e percentualmente maggiore rispetto al
maschio di aquila reale (stima {\textpm} e.s. aquila reale 0,192
{\textpm} 0,019; stima {\textpm} e.s. gipeto 0,273 {\textpm} 0,018; F=
5,04; p= 0,025). 

Trattando i dati in maniera analoga all{\textquoteright}unico studio
disponibile sul gipeto per l{\textquoteright}Europa (Pirenei spagnoli,
Margalida \& Bertran 2000), in cui entrambi i sessi contribuivano in
egual misura alle cure parentali (ff 52 {\textpm} 6,6\%; mm 48
{\textpm} 6,6\%; t\textsubscript{6}= -0,64; ns), nella presente
indagine il tempo dedicato alle cure parentali \`e risultato invece
pi\`u sbilanciato verso le femmine (61 {\textpm} 8,4\%) rispetto ai
maschi (39\% {\textpm} 8,4\%).

Una possibile interpretazione della differenza tra le due aree di studio
deriva dal fatto che, nel Parco Nazionale dello Stelvio, le temperature
medie di gennaio (mese che coincide con l{\textquoteright}inizio della
cova del gipeto in tale area di studio) sono notevolmente inferiori
rispetto a quelle pirenaiche (media delle minime: Alpi -12,2 vs Pirenei
-5{\textdegree}C). Le maggiori dimensioni della femmina potrebbero
infatti garantire un pi\`u efficace isolamento termico;
l{\textquoteright}influenza del clima sembra avvalorata anche dal fatto
che nel Parco Nazionale dello Stelvio, nell{\textquoteright}ambito del
presente studio, non sono mai state osservate interruzioni del
riscaldamento di uova e nidiacei. 

Nel gipeto i tempi di cova, che comunque risultano pi\`u equamente
distribuiti tra i due sessi rispetto a quanto avviene
nell{\textquoteright}aquila reale, possono essere una diretta
conseguenza delle modalit\`a di ricerca del cibo la cui localizzazione
risulta di pi\`u difficile previsione (Margalida \& Bertran 2000).
Infatti gli adulti di gipeto per individuare carcasse e singole ossa
disperse nel territorio perlustrano aree molto estese (nel Parco
Nazionale dello Stelvio la stima dell{\textquoteright}\textit{home
range} di ciascuna coppia di gipeto \`e pari a 300-500
km\textsuperscript{2}, Bassi \textit{ined}.), investendo un tempo
maggiore nella ricerca di cibo rispetto all{\textquoteright}aquila
reale, che utilizza territori di caccia generalmente posti a distanze
inferiori dai propri siti di nidificazione. Tale spiegazione
giustificherebbe anche il motivo per cui il numero di cambi al nido per
giornata sia inferiore nel gipeto rispetto a quanto registrato per
l{\textquoteright}aquila reale. 

\section*{Ringraziamenti}

Si ringraziano il Consorzio del Parco Nazionale dello Stelvio, il Corpo
Forestale dello Stato - C.T.A. di Bormio e la Societ\`a Tersia s.r.l.
di Savona per aver contribuito al finanziamento della ricerca nel corso
di tre anni consecutivi. Un particolare ringraziamento a Heinrich
Haller, Giuseppe Bogliani e David Jenny per i preziosi suggerimenti e
consigli nell{\textquoteright}impostazione della ricerca.

\section*{Bibliografia}
\begin{itemize}\itemsep0pt
	\item Bassi E., 2011 - Sintesi del censimento contemporaneo di aquila reale
(\textit{Aquila chrysaetos}) e gipeto (\textit{Gypaetus barbatus})
nell{\textquoteright}ambito dei progetti di monitoraggio delle
popolazioni nidificanti nel settore lombardo e trentino del Parco
Nazionale dello Stelvio. Anni 2004-2011. Parco Nazionale dello Stelvio.
Relazione interna, 46 pp.

	\item Bassi E., Diana F., Sartirana F., Trotti P., Galli L. \& Pedrotti P. 2015. - Analisi del successo riproduttivo
dell{\textquoteright}aquila reale (\textit{Aquila chrysaetos}) nel
Parco Nazionale dello Stelvio in relazione al ritorno del gipeto
(\textit{Gypaetus barbatus}) sulle Alpi. \textit{Pedrini P., Rossi F., Bogliani G., Serra L. \& Sustersic A. (a cura di) 2015. \emph{XVII Convegno Italiano di Ornitologia: Atti del convegno di Trento}. Ed. MUSE, 174 pp.}.

	\item Carro M. \& Pedrotti L. (a cura di), 2010 - Atlante del Parco Nazionale
dello Stelvio.

	\item Collopy M., 1984 - Parental care and feeding ecology of Golden eagle
nestlings. \textit{Auk}, 101:753-760.

	\item Margalida A. \& Bertran J., 2000 - Breeding behaviour of the Bearded
Vulture \textit{Gypaetus barbatus}: minimal sexual differences in
parental activities. \textit{Ibis,} 142: 225-234.

	\item Martin P. \& Bateson P., 1986 - \textit{Measuring Behaviour: an
introductory guide}. Cambridge University Press, Cambridge.

	\item Watson J., 1997 - \textit{The Golden eagle}. T\&D Poyser, London.
\end{itemize}
