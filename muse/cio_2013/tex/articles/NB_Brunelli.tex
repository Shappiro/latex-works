\setcounter{figure}{0}
\setcounter{table}{0}

\begin{adjustwidth}{-3.5cm}{0cm}
\pagestyle{CIOpage}
\authortoc{\textsc{Brunelli M.}, \textsc{Cento M.}, 
\textsc{Sarrocco S.}, \textsc{Biondi M.}, 
\textsc{Boano A.}, \textsc{De Santis E.}, 
\textsc{Fraticelli F.}, \textsc{Hueting S.}, 
\textsc{Meschini A.}, \textsc{Purificato G.}, 
\textsc{Scrocca R.}}
\chapter*[Atlante degli uccelli svernanti del Lazio]{\bfseries
Atlante degli uccelli d{\textquoteright}Italia in inverno: analisi dei
dati preliminari nel Lazio (2009/10 - 2012/13)}
\addcontentsline{toc}{chapter}{Atlante degli uccelli svernanti del Lazio}

\textsc{Massimo Brunelli}$^{1*}$, \textsc{Michele Cento}$^{1}$, 
\textsc{Stefano Sarrocco}$^{2}$, \textsc{Massimo Biondi}$^{1}$, 
\textsc{Aldo Boano}$^{1}$, \textsc{Emiliano De Santis}$^{1}$, 
\textsc{Fulvio Fraticelli}$^{1}$, \textsc{Steven Hueting}$^{1}$, 
\textsc{Angelo Meschini}$^{1}$, \textsc{Giovanni Purificato}$^{1}$, 
\textsc{Roberto Scrocca}$^{1}$ \\

\index{Brunelli Massimo} \index{Cento Michele} \index{Sarrocco Stefano} \index{Biondi Massimo} \index{Boano Aldo} \index{De Santis Emiliano} \index{Fraticelli Fulvio} \index{Hueting Steven} \index{Meschini Angelo} \index{Purificato Giovanni} \index{Scrocca Roberto}
\noindent\color{MUSEBLUE}\rule{27cm}{2pt}
\vspace{1cm}
\end{adjustwidth}



\marginnote{\raggedright $^1$\textbf{\textsuperscript{ }}SROPU,\textbf{ }Stazione
Romana per l{\textquoteright}Osservazione e la Protezione degli
Uccelli, Via Britannia 36, 00183 Roma, Italia \\
$^2$ARP, Agenzia Regionale per i Parchi, Via del
Pescaccio 96, 00166 Roma, Italia \\
\vspace{.5cm}
{\emph{\small $^*$Autore per la corrispondenza: \href{mailto:mss.brunelli@tin.it}{mss.brunelli@tin.it}}} \\
\keywords{Atlante, uccelli, inverno, Lazio}
{Atlas, birds, winter, Latium}
%\index{keywords}{Atlante} \index{keywords}{Uccelli} \index{keywords}{Inverno} \index{keywords}{Lazio}
}
{\small
\noindent \textsc{\color{MUSEBLUE} Summary} / We reported for the Latium region the results of the
first four years of the project
{\textquotedblleft}Atlas of the Birds of Italy in winter 2009/2010 --
2014/2015{\textquotedblright}. We recorded a total of 213
and an average of 48,4 (SD 24,2) species for square
unit of side 10 km, with high richness along coastal plain and hilly
areas, as well as around wetlands and lower values in the Apennines.
}


\section*{Introduzione}

La conoscenza della distribuzione e delle abbondanze degli uccelli
presenti in inverno sul territorio rappresenta una fonte di
informazione importante per attivare strumenti di conservazione delle
specie e gestione del territorio. Con queste finalit\`a il portale web
ornitho.it, e le associazioni che in esso si riconoscono, hanno
promosso il progetto  {\textquotedblleft}Atlante degli uccelli
d{\textquoteright}Italia in inverno 2009/2010 --
2014/2015{\textquotedblright}. In questo contributo presentiamo i
risultati ottenuti nel Lazio (aggiornati al 31 gennaio 2013) con
l{\textquoteright}obiettivo di analizzare i risultati raggiunti e
individuare le aree su cui indirizzare
prioritariamente lo sforzo di ricerca nella prossima stagione di
rilevamento.

\section*{Metodi}

I dati sono stati raccolti in varia misura da oltre 150 rilevatori dal
1{\textdegree} dicembre al 31 gennaio degli anni 2009-2013. La base
cartografica utilizzata \`e rappresentata dalla griglia UTM con unit\`a
di rilevamento quadrate di 10 km di lato. In ogni
unit\`a di rilevamento sono stati indagati tutti gli ambienti presenti
al fine di registrare il maggior numero di specie. Sono state escluse
le unit\`a di rilevamento ricadenti prevalentemente in regioni
confinanti e quelle costiere con ridottissima
estensione di superficie emersa. Le unit\`a di rilevamento indagate
sono state 206; le 5 in cui ricadono le isole Pontine sono state
accorpate in due unit\`a, pertanto elaborazioni e rappresentazioni
grafiche si riferiscono a 203 unit\`a di rilevamento. 

Per svolgere~inferenze sulle specie~attese~abbiamo elaborato alcune
analisi di \textit{geoprocessing,} utilizzando le specie osservate
nelle unit\`a di rilevamento e le 118 unit\`a di paesaggio regionali,
quest{\textquoteright}ultime considerate singolarmente, in prima
approssimazione, uniformi sia morfologicamente sia come uso del suolo.
A ogni unit\`a di paesaggio abbiamo assegnato un
valore pari al numero medio delle unit\`a di rilevamento in
essa ricadenti e tale valore lo
abbiamo successivamente riattribuito alle unit\`a di
rilevamento. Alle unit\`a di rilevamento con campionamenti incompleti
abbiamo assegnato i valori di quelle contigue che presentavano una
maggiore completezza di campionamento.

\section*{Risultati e discussione}
Complessivamente sono state rinvenute 213 specie, la ricchezza media per
unit\`a di rilevamento \`e risultata di 48,4 specie (DS 24,2); le
unit\`a di rilevamento con i valori maggiori sono quelle delle pianure
costiere (es. litorale romano) e quelle collinari (es. monti della
Tolfa), nonch\'e quelle con presenza di zone umide (es. lago di
Bolsena, litorale pontino, piana di Rieti). Le unit\`a di rilevamento
con i valori minori sono risultate invece quelle ricadenti lungo la
dorsale appenninica, caratterizzate da quote pi\`u elevate e dalla
presenza di estese aree montane.
Le unit\`a di paesaggio, analogamente, evidenziano i valori maggiori
nelle lagune costiere e nelle paludi salse (intervallo 117,0-65,5
specie), nei laghi dulciacquicoli (101,0-8,0 specie) e lungo le pianure
costiere (85,0-43,0 specie); i valori minori sono collegati alle conche
interne e ai rilievi carbonatici del Preappennino e
dell{\textquoteright}Appennino (47,3-3,7 specie), sebbene in queste
unit\`a la copertura dei rilevamenti non sia da ritenersi esaustiva.
Le elaborazioni effettuate hanno permesso di ricavare informazioni sul
numero di specie attese (Fig. \ref{Brunelli_fig_1}) e d{\textquoteright}individuare le UR
con una copertura ritenuta insufficiente, da visitare prioritariamente
nella prossima stagione di rilevamento.
Il confronto con la distribuzione della ricchezza
dell{\textquoteright}avifauna nidificante (Brunelli \textit{et al}.
2011) evidenzia come alcune aree mantengano alti valori di ricchezza
anche durante il periodo invernale (es. litorale romano, monti della
Tolfa, piana di Rieti), altre aumentino il loro valore (es. litorale
pontino) e altre lo diminuiscano sensibilmente (es. dorsale
appenninica).

\begin{figure}[!h]
\centering
\includegraphics[width=.74\columnwidth]{Brunelli_fig_1.png}
\caption{Numero di specie attese in inverno nelle particelle di 10 km di lato (nelle particelle \`e riportato il numero di specie osservate)}
\label{Brunelli_fig_1}
\end{figure}
\section*{Ringraziamenti}
Ringraziamo tutti i numerosi rilevatori, in particolare: Alessandro
Ammann, Gabriella Biondi, Fabrizio Bulgarini, Mario Cappelli, Monica
Carabella, Alberto Cardillo, Carlo Castellani, Emanuele G. Condello,
Davide de Rosa, Santino di Carlo, Brendan Doe, Roberto Gildi, Daniele
Iavicoli, Gigliola Magliocco, Alberto Manganaro, Fabrizio Mantero,
Riccardo Molajoli, Sergio Muratore, Alessio Rivola, Enzo Savo, Fabio
Scarf\`o, Alberto Sorace, Maurizio Sterpi, Marco Trotta, Claudio
Zanotti.
\section*{Bibliografia}
\begin{itemize}\itemsep0pt
	\item Brunelli M., Sarrocco S., Corbi F., Sorace A., Boano
A\textcolor{red}{.}, De Felici S., Guerrieri G., Meschini A. \& Roma S.
(a cura di), 2011 - \textit{Nuovo Atlante degli Uccelli Nidificanti nel
Lazio}. Edizioni ARP (Agenzia Regionale Parchi), Roma, 464 pp.
\end{itemize}
