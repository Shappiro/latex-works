\setcounter{figure}{0}
\setcounter{table}{0}

\begin{adjustwidth}{-3.5cm}{0cm}
\pagestyle{CIOpage}
\authortoc{\textsc{Morganti N.}, \textsc{Morici F.}, \textsc{Mencarelli M.}}
\chapter*[Monitoraggio dell{\textquoteright}avifauna del comune di
Senigallia (AN)]{Atlante faunistico del comune di Senigallia (AN): monitoraggio
dell{\textquoteright}avifauna del centro urbano}
\addcontentsline{toc}{chapter}{Monitoraggio dell{\textquoteright}avifauna del comune di
Senigallia (AN)}

\textsc{Niki Morganti}$^{1*}$, \textsc{Francesca Morici}$^{1*}$, \textsc{Mauro Mencarelli}$^{1*}$\\

\index{Morganti Niki} \index{Morici Francesca} \index{Mencarelli Mauro}
\noindent\color{MUSEBLUE}\rule{27cm}{2pt}
\vspace{1cm}
\end{adjustwidth}


\marginnote{\raggedright $^1$Studio Naturalistico Diatomea, Senigallia\\
\vspace{.5cm}
{\emph{\small $^*$Autore per la corrispondenza: \href{mailto:info@studiodiatomea.it}{in\allowbreak fo@\allowbreak stu\allowbreak dio\allowbreak dia\allowbreak to\allowbreak mea.\allowbreak it}}} \\
\keywords{Senigallia (AN), avifauna urbana, monitoraggio}
{Senigallia (AN), urban bird, community, monitoring}
%\index[keywords]{Senigallia} \index[keywords]{Avifauna urbana} \index[keywords]{Monitoraggio}
}
{\small
\noindent \textsc{\color{MUSEBLUE} Summary} / During the breeding season in 2013 was studied the bird community in the
center of the city Senigallia (AN). There have been recorded 57 species
of which 19 with some nesting and 17 non-nesting. There were surveyed
nests of house martin: in the village were counted 33 active nests.\\
}

\vspace{1cm}
Il progetto {\textquotedblleft}Atlante Faunistico del comune di
Senigallia (AN){\textquotedblright} \`e stato avviato nel 2009 con
l'obiettivo di predisporre un database con dati sulla
presenza, distribuzione e consistenza delle specie di Vertebrati
Tetrapodi (Mencarelli \textit{et al}. 2009). Grande attenzione \`e
stata dedicata alla classe degli Uccelli, veri e propri
{\textquotedblleft}indicatori{\textquotedblright} in grado di fornire
preziosi elementi sullo stato di salute dell'ambiente
e sulla biodiversit\`a. 

Nella stagione riproduttiva 2013 \`e stata
indagata l'avifauna nidificante nel centro urbano di
Senigallia. L'area di studio comprende il centro
storico di Senigallia, i quartieri residenziali i quali sono
caratterizzati dalla presenza di aree verdi di modeste dimensioni
(giardini privati, orti, ecc.), le aree verdi pubbliche, i centri
sportivi e una zona industriale. Quest'ultima,
localizzata a nord della citt\`a, \`e costituita da capannoni che
ospitano attivit\`a commerciali, artigianali e uffici; sono inoltre
presenti aree incolte e alcuni giardini privati.
L'area \`e stata suddivisa in 15 quadranti di 1 km di lato ed in ogni unit\`a di rilevamento sono stati individuati 4 punti di ascolto/osservazione, visitati tre volte, in cui applicare la
metodologia I.P.A. 

I monitoraggi sono stati avviati nel mese di marzo
2013 e sono terminati nel mese di luglio dello stesso anno. Per ogni
stazione \`e stata effettuata una sosta di 15 minuti per rilevare le
specie presenti. \`E stata compilata una scheda di rilevamento in cui
registrare la presenza delle specie utilizzando dei codici che, per la
stagione riproduttiva, indicano la categoria di nidificazione. Sono
state censite 57 specie di cui 19 con nidificazione certa (piccione
domestico, tortora dal collare, rondone, upupa, balestruccio, usignolo,
codirosso comune, merlo, capinera, cinciarella, cinciallegra, gazza,
taccola, storno, passera d'Italia, passera mattugia,
verzellino, verdone, cardellino), 8 con nidificazione probabile
(gheppio, torcicollo, pettirosso, cannaiola, codibugnolo, cornacchia
grigia, fringuello, strillozzo) e 12 con nidificazione possibile
(tortora selvatica, assiolo, civetta, rondine, ballerina bianca,
scricciolo, codirosso spazzacamino, usignolo di fiume, canapino, lu\`i
piccolo, pigliamosche, zigolo nero); 17 invece le specie non
nidificanti (fenicottero, garzetta, airone cenerino, falco pecchiaiolo,
falco di palude, albanella reale, albanella minore, poiana, lodolaio,
falco pellegrino, gabbiano comune, gabbiano reale, parrocchetto dal
collare, gruccione, cutrettola, stiaccino, sterpazzolina). 

Sono stati
inoltre censiti i nidi di balestruccio delle colonie presenti
nell'area del centro storico, il nucleo pi\`u antico
della citt\`a, per un totale di almeno 17 nidi attivi. Le specie pi\`u
presenti sono risultate: capinera (frequenza 0,86), cardellino (0,67),
passera d'Italia (0,96), tortora dal collare (0,71),
verdone (0,83) e verzellino (0,88). Risultano necessarie analisi pi\`u
approfondite riguardo alla nidificazione di alcune specie di rapaci
notturni (assiolo, civetta e barbagianni; relativamente al barbagianni
nel periodo di studio \`e stato rilevato un individuo morto investito),
della specie naturalizzata parrocchetto dal collare (\textit{Psittacula
krameri}) e riguardo al censimento dei nidi balestruccio:
complessivamente nel centro urbano sono stati contati 33 nidi attivi.

\section*{Bibliografia}
\begin{itemize}\itemsep0pt
	\item Mencarelli M., Morganti N. \& Morici F., 2009 - Progetto atlante
faunistico del comune di Senigallia: Avifauna primo anno di indagine. In: Brunelli M., Battisti
C., Bulgarini F., Cecere

	\item J.G., Fraticelli F., Giustin M., Sarrocco S. \& Sorace A. (A cura di)
Atti del XV Convegno Italiano di Ornitologia. Sabaudia, 14-18 ottobre 2009. \textit{Alula},
XVI (1-2): 680-682.
\end{itemize}
