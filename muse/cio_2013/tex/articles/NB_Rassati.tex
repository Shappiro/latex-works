\begin{otherlanguage}{english}
\setcounter{figure}{0}
\setcounter{table}{0}

\begin{adjustwidth}{-3.5cm}{0cm}
\pagestyle{CIOpage}
\authortoc{\textsc{Rassati G.}}
\chapter*[]{"Silviphobia" and urban bird populations}
\addcontentsline{toc}{chapter}{"Silviphobia" and urban bird populations}

\textsc{Gianluca Rassati}$^{1*}$\\

\index{Rassati Gianluca}
\noindent\color{MUSEBLUE}\rule{27cm}{2pt}
\vspace{1cm}
\end{adjustwidth}

\marginnote{\raggedright $^1$Via Udine 9, 33028 Tolmezzo (UD) \\
\vspace{.5cm}
{\emph{\small $^*$Autore per la corrispondenza: \href{mailto:itassar@tiscali.it}{i\allowbreak ta\allowbreak s\allowbreak sar@\allowbreak ti\allowbreak sca\allowbreak li.\allowbreak it}}} \\
\keywords{Alpi Carniche, selvicoltura, popolamento ornitico, area urbana, biodiversità, distrurbo antropico}
{Carnic Alps, silviculture, bird population, urban area, biodiversity, human disturbance}
} 
{\small
\noindent\textsc{\color{MUSEBLUE} Riassunto} / "Selvifobia" e popolamenti ornitici urbani. Avendo constatato un impoverimento della diversità arborea
urbana in zona alpina carnica, nel periodo compreso fra gli anni 1998-1999 e 2011-2012, se ne è voluto verificare
l’effetto sui popolamenti ornitici di Tolmezzo. I valori dei parametri del secondo periodo (Tab. \ref{Rassati_tab_1}) sono risultati più
bassi di quelli del primo, dimostrando un calo di diversità ornitica. Questo è stato determinato da specie legate agli
alberi, non contattate nel secondo periodo o il cui IKA è diminuito talvolta fortemente. \\
}

\vspace{1cm}
In the last decades, the decrease of the Alpine human populations and the departure of the young from
the rural world, together with the abandonment of the agricultural, silvicultural and pastoral activities, favoured the
spread of idea that the forest is literally devouring the mountain urban agglomerations, causing a kind of
"silviphobia" that brought to a rage against the trees within the urban agglomerations.

Based on such ideas, verified trough analysis of what is happened in last years in the villages of
Carnic Alps, we used temporal comparison data of the most extended urban area, Tolmezzo, to verify if the change in tree cover caused an impact on the bird populations.

The comparison was realized using the linear transect method (Bibby }\textit{\textcolor{black}{et
al.}} (2000), on a 1 km long transect in the years 1998 and 1999 and, 2011 and 2012. Each year, the transect was carried out once from 1}\textcolor{black}{\textsuperscript{st}}\textcolor{black}{ April to
15}\textcolor{black}{\textsuperscript{th}}\textcolor{black}{ May and a second time from
16}\textcolor{black}{\textsuperscript{th}}\textcolor{black}{ May to
30}\textcolor{black}{\textsuperscript{th}} June. During the survey only the individuals contacted
within a belt of 25 m on both sides were considered. For each year, only the higher value between the two periods was
considered.

\textcolor{black}{The monitored area (46$^{\circ}$24' N, 13$^{\circ}$01' E - 315 m a.s.l.) is flat and represented by an extensive residential zone. The arboreous and shrubby species are numerous, both native and alien.}

In the lapse of time included between the two periods in which the surveys were carried out several
old trees were cut, also because of various restorations and enlargement and realization of construction works and many
others, most of all of public property, were subjected to severe prunings especially regarding old branches. This
latter operation led to the death of a further number of trees: we estimated a decrease of 30\% of the total number of
trees, of 40\% of the total standing stock, of 70\% of the number of trees with dead parts or with cavities.

\textcolor{black}{Examinating Tab. \ref{Rassati_tab_1} we deduce a decrease of the value of the parameters between the first and the
second period of the survey: number of species (-10.71\%), KAI (-17.20\%), number of dominant species (-40.00\%) and
relative percent frequency (-32.81\%), percent of non-passerines species (-}\textcolor{black}{20.00\%); also the
diversity index (Shannon \& Weaver 1963) and the equitability index (Magurran 1988) are lower in the second period.}

\textcolor{black}{If we analyse in details the populations, we observe that even in presence of a high value of
similarity index (Sørensen 1948) of 0.83, there are some substantial differences. The first one is that the higher
number of species in 1998-1999 is determined by }\textit{\textcolor{black}{taxa}}\textcolor{black}{\ related to trees as
sparrowhawk, wryneck, green woodpecker, goldcrest, marsh }{tit, short-toed
treecreeper, not contacted in the second period. Secondly the higher KAI is determined not only by the above species
but also by the decrease of the values of arboreal }\textit{\textcolor{black}{taxa}}\textcolor{black}{\ contacted in
both periods as great spotted woodpecker (2.5 vs 0.5) and nuthatch (2.0 vs 0.5) but also chaffinch (7.0 vs 4.5) and
greenfinch (7.0 vs 5.5). The species found only in the second period, on the other hand, are not closely related to
trees, particularly the third one: cuckoo, magpie, starling. The magpie moreover began to use regularly the urban
habitats exactly in the first decade of the XXI century. These differences contribute moreover to decrease the indexes
and the diversity in the second period, indeed also the percentage of non-passerines species, due to the disappearance
of some picid species, decreases, simplifying the composition of the populations.}

What emerged shows that also in urban areas the silvicultural operations should be planned and managed
by specialised technicians to avoid, as now often happens also in public field, that such operations might be used as a
pretext to impoverish or eliminate the tree stands (e.g. substituting old trees and/or of native species with young
trees of alien species). To obtain concrete results it is however necessary to debunk the beliefs mentioned at the
beginning of this study, considering that also the actions carried out in the urban areas can substantially affect on
biodiversity.

\begin{table}[!h]
\centering
\begin{tabular}{>{\raggedright\arraybackslash}p{.2\columnwidth}>{\raggedright\arraybackslash}p{.07\columnwidth}>{\raggedright\arraybackslash}p{.07\columnwidth}>{\raggedright\arraybackslash}p{.07\columnwidth}>{\raggedright\arraybackslash}p{.07\columnwidth}>{\raggedright\arraybackslash}p{.07\columnwidth}>{\raggedright\arraybackslash}p{.07\columnwidth}>{\raggedright\arraybackslash}p{.07\columnwidth}}
\toprule
& \textbf{S} & \textbf{KAI} & \textbf{Nd} & \textbf{Fd\%} & \textbf{\% nP} & \textbf{H'} & \textbf{E} \\
\toprule
1998-1999 & 28 & 125 & 5 & 17.86 & 25 & 2.24 & 0.80 \\
2011-2012 & 25 & 103.5 & 3 & 12 & 20 & 2.05 & 0.76 \\
\bottomrule
\end{tabular}
\caption{Structure of the bird population in the two considered periods. S = number of species, KAI = mean kilometric abundance index (ind/km), Nd = number of dominant species, Fd\% = \% frequency of dominant species, \% nP = \% of non-passerines species, H' = diversity index, E = equitability index}
\label{Rassati_tab_1}
\end{table}

\section*{Bibliography}
\begin{itemize}\itemsep0pt
	\item Bibby C.J., Burgess N.D. \& Hill D.A., 2000 - \textit{Bird census tecniques.} Second Edition, Academic Press, London.
	\item \textcolor{black}{Magurran A. E., 1988 - }\textit{\textcolor{black}{Ecological diversity and its
measurement.}}\textcolor{black}{\ University Press, Cambridge.}
	\item \textcolor{black}{Shannon C.E. \& Weaver W., 1963 - }\textit{\textcolor{black}{A mathematical theory of
communication.}}\textcolor{black}{\ University of Illinois Press, Urbana.}
	\item S{\o}rensen T., 1948 - \textit{A method of establishing groups of equal amplitude in plant sociology based on similarity of species content and its application to analysis of the vegetation on Danish commons.}\ Det. Kong. Danske Vidensk. Selsk. Biol. Skr., 5: 1-34.
\end{itemize}
\end{otherlanguage}