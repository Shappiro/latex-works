\RequirePackage{atbegshi}
\AtBeginShipoutInit
\documentclass[10pt,twoside,openright,x11names,svgnames,italian,a4paper,dvipsnames,table]{memoir}
% add "showtrims" to class options to show the trimming lines

\usepackage{fontspec}
\setmainfont{Arial}

\setstocksize{303mm}{215mm}
\settrimmedsize{297mm}{210mm}{*}
\settrims{3mm}{3mm}
% Apply and enforce layout
\setlrmarginsandblock{5cm}{2cm}{*} % Left and right margin
\setulmarginsandblock{2cm}{3cm}{*}  % Upper and lower margin
\setlength{\footskip}{2cm}

%\setlength{\marginparsep}{1cm}
\checkandfixthelayout
\setlength{\marginparwidth}{3.5cm}

\usepackage[italian,english]{babel}
\usepackage{lmodern}
\usepackage{wallpaper}
\usepackage{tikz}
\usepackage{pagecolor}
\usetikzlibrary{shapes,positioning}
\usepackage[latin1]{inputenc}
\usepackage[italian]{babel}
\usepackage[T1]{fontenc}
\usepackage{enumitem}
\usepackage{amsmath}

\usepackage{eso-pic}

\newcommand\BackgroundFrontPic{%
\AddToShipoutPictureBG*{\includegraphics[width=\stockwidth, height=\paperheigth, keepaspectratio]{COVER_complete.png}}
}

\newcommand\BackgroundRetroPic{%
\AddToShipoutPictureBG*{\includegraphics[width=\stockwidth, height=\paperheigth, keepaspectratio]{RETRO_complete.png}}
}

\newcommand{\mc}[1]{\multicolumn{1}{c}{#1}} % handy shortcut macro

\setlength\parindent{0pt}

\usepackage{imakeidx}
\usepackage{enumitem}
\usepackage{titletoc}

\usepackage{textcomp}	

\usepackage[hyphens]{url} % For URL automated linebreaks
\usepackage{booktabs, longtable}
\usepackage{grffile}

\usepackage{xcolor}

\usepackage{titlesec}

\usepackage{amsmath}

\usepackage{afterpage}
\usepackage{lipsum}% for auto generating text

\usepackage{rotating}
\usepackage{minibox}
\usepackage{pdfpages}
\usepackage{subcaption}

\usepackage{changepage}


\usepackage{lipsum}
\usepackage[ISBN=978-88-531-0041-2]{ean13isbn}
\usepackage{graphicx}
\graphicspath{ {./img/} {./img/chap/} {./img/logo/} {./img/front/} {./img/icon/} {./img/back/} {./img/articles/} }

% Captions
\usepackage[labelfont={footnotesize,sf,bf},textfont={footnotesize,sf}]{caption}

%%% ROW A
\definecolor[named]{A1}{HTML}{FFF593}
\definecolor[named]{A2}{HTML}{FFEF3A}
\definecolor[named]{A3}{HTML}{FEED01}
\definecolor[named]{A4}{HTML}{FDCA01}
\definecolor[named]{A5}{HTML}{F9B700}
\definecolor[named]{A6}{HTML}{F59701}
\definecolor[named]{A7}{HTML}{F5A301}
\definecolor[named]{A8}{HTML}{F07901}
\definecolor[named]{A9}{HTML}{EA4E01}
\definecolor[named]{A10}{HTML}{CD4803}
\definecolor[named]{A11}{HTML}{C69121}
\definecolor[named]{A12}{HTML}{C37F1E}
\definecolor[named]{A13}{HTML}{B58636}
\definecolor[named]{A14}{HTML}{A4601F}

%%% ROW B
\definecolor[named]{B1}{HTML}{F4AA8D}
\definecolor[named]{B2}{HTML}{EC6863}
\definecolor[named]{B3}{HTML}{E7002A}
\definecolor[named]{B4}{HTML}{E94E2F}
\definecolor[named]{B5}{HTML}{E60003}
\definecolor[named]{B6}{HTML}{D70007}
\definecolor[named]{B7}{HTML}{B30006}
\definecolor[named]{B8}{HTML}{933907}
\definecolor[named]{B9}{HTML}{8C2B00}
\definecolor[named]{B10}{HTML}{4F1700}
\definecolor[named]{B11}{HTML}{2D0600}
\definecolor[named]{B12}{HTML}{7F8D98}
\definecolor[named]{B13}{HTML}{A0ABB1}
\definecolor[named]{B14}{HTML}{AFAEB3}

%%% ROW C
\definecolor[named]{C1}{HTML}{F1B0CE}
\definecolor[named]{C2}{HTML}{E86BA5}
\definecolor[named]{C3}{HTML}{E60084}
\definecolor[named]{C4}{HTML}{C80084}
\definecolor[named]{C5}{HTML}{AD0073}
\definecolor[named]{C6}{HTML}{930084}
\definecolor[named]{C7}{HTML}{741186}
\definecolor[named]{C8}{HTML}{5B004F}
\definecolor[named]{C9}{HTML}{1B0051}
\definecolor[named]{C10}{HTML}{4F250D}
\definecolor[named]{C11}{HTML}{240000}
\definecolor[named]{C12}{HTML}{0C0028}
\definecolor[named]{C13}{HTML}{5D7381}
\definecolor[named]{C14}{HTML}{817F84}

%%% ROW D
\definecolor[named]{D1}{HTML}{BBB1D6}
\definecolor[named]{D2}{HTML}{907EBA}
\definecolor[named]{D3}{HTML}{8D90C5}
\definecolor[named]{D4}{HTML}{6375B7}
\definecolor[named]{D5}{HTML}{3580C3}
\definecolor[named]{D6}{HTML}{4470B7}
\definecolor[named]{D7}{HTML}{8BA1D2}
\definecolor[named]{D8}{HTML}{0082CD}
\definecolor[named]{D9}{HTML}{006EB5}
\definecolor[named]{D10}{HTML}{0168B5}
\definecolor[named]{D11}{HTML}{0059A9}
\definecolor[named]{D12}{HTML}{004C92}
\definecolor[named]{D13}{HTML}{003B77}
\definecolor[named]{D14}{HTML}{504F54}

%%% ROW E
\definecolor[named]{E1}{HTML}{5DC6F3}
\definecolor[named]{E2}{HTML}{00B6EF}
\definecolor[named]{E3}{HTML}{01A5EC}
\definecolor[named]{E4}{HTML}{0060AA}
\definecolor[named]{E5}{HTML}{014EA0}
\definecolor[named]{E6}{HTML}{1A3793}
\definecolor[named]{E7}{HTML}{2E1D87}
\definecolor[named]{E8}{HTML}{004E8E}
\definecolor[named]{E9}{HTML}{00397E}
\definecolor[named]{E10}{HTML}{011C53}
\definecolor[named]{E11}{HTML}{004B7C}
\definecolor[named]{E12}{HTML}{373E5A}
\definecolor[named]{E13}{HTML}{003058}
\definecolor[named]{A14}{HTML}{000429}

%%% ROW F
\definecolor[named]{F1}{HTML}{0AB4CE}
\definecolor[named]{F2}{HTML}{15B1BD}
\definecolor[named]{F3}{HTML}{00A4DA}
\definecolor[named]{F4}{HTML}{00A2B9}
\definecolor[named]{F5}{HTML}{4EB693}
\definecolor[named]{F6}{HTML}{58B36E}
\definecolor[named]{F7}{HTML}{2BAA5B}
\definecolor[named]{F8}{HTML}{019E95}
\definecolor[named]{F9}{HTML}{009B71}
\definecolor[named]{F10}{HTML}{01994C}
\definecolor[named]{F11}{HTML}{415973}
\definecolor[named]{F12}{HTML}{405874}
\definecolor[named]{F13}{HTML}{575B67}
\definecolor[named]{F14}{HTML}{37363B}

%%% ROW G
\definecolor[named]{G1}{HTML}{C9D301}
\definecolor[named]{G2}{HTML}{97C000}
\definecolor[named]{G3}{HTML}{70B21A}
\definecolor[named]{G4}{HTML}{2FA829}
\definecolor[named]{G5}{HTML}{00A131}
\definecolor[named]{G6}{HTML}{019837}
\definecolor[named]{G7}{HTML}{01832D}
\definecolor[named]{G8}{HTML}{016821}
\definecolor[named]{G9}{HTML}{004D2B}
\definecolor[named]{G10}{HTML}{012F08}
\definecolor[named]{G11}{HTML}{005E66}
\definecolor[named]{G12}{HTML}{012E17}
\definecolor[named]{G13}{HTML}{002209}
\definecolor[named]{G14}{HTML}{1B1C20}
%%% COLORS
\usepackage{etoolbox}
\patchcmd{\chapter}{plain}{empty}{}{}

\newcommand{\chaptercolor}{G10}
\newcommand{\toprectanglecolor}{F10}
\newcommand{\pageboxcolor}{G9}
\newcommand{\backgroundrectanglecolor}{G8!80!white}
\newcommand{\decoratelinecolor}{G8}
\newcommand{\titlecolor}{G8}
\newcommand{\backpagecolor}{\chaptercolor}


% Links
\usepackage[pdftitle={Atti del Convegno Italiano di Ornitologia - Trento, 11-15 Settembre},
     pdfauthor={Sezione Zoologia dei Vertebrati, MUSE - Museo delle Scienze},
     colorlinks,linktocpage=true,linkcolor=\titlecolor,urlcolor=BrickRed,citecolor=MUSEBLUE,bookmarks]{hyperref}





% Command to hold chapter illustration image
\newcommand\chapterillustration{}

\usepackage{xcolor}

\newcommand\textlcsc[1]{\textsc{\MakeUppercase{#1}}}

\usepackage{dcolumn}
\newcolumntype{d}[1]{D{,}{.}{#1}}

\makeatletter
\newcommand{\fonttitle}{\chaptitlefont}
\makechapterstyle{CIO}{%
\def\chapterheadstart{\vspace*{\beforechapskip}}
\def\printchaptername{}
\def\printchapternum{}
\def\printchapternonum{}
\def\printchaptertitle##1{\raggedright \space \fonttitle \HUGE ##1}
\def\afterchaptertitle{\par\nobreak\vskip \afterchapskip}
\setlength\beforechapskip{0pt}%
%\setlength\afterchapskip{10pt}% 
}
\makeatother

\makeatletter
\renewcommand\chaptermark[1]{%
  \markboth{\MakeUppercase{#1}}{}
}
\makeatother

\titleclass{\part}{top} % make part like a chapter
\titleformat{\part}
[display]
{\raggedleft\normalfont\Huge\bfseries}
% "Part n", left blank
{}
{0pt}
{\color{MUSEBLUE}\huge\MakeUppercase}
%
\titlespacing*{\part}{0pt}{0pt}{20pt}

% Removed bfseries from TOC chapters
%\titleformat{\chapter}{\LARGE}{\thechapter.\ }{0em}{}


\makepagestyle{CIOpage}
\makeevenhead{CIOpage}{}{}{}
\makeoddhead{CIOpage}{}{}{}
\makeevenfoot{CIOpage}{\textbf{\thepage}\space / \space\textcolor{MUSEBLUE}{\textlcsc{Convegno italiano di ornitologia}}\space / Atti del convegno}{}{}
\makeoddfoot{CIOpage}{\vspace{\fill}Trento, 11-15 Settembre 2013 /\textcolor{MUSEBLUE}{\textlcsc{ Convegno italiano di ornitologia}} / \textbf{\thepage}}{}{}

\aliaspagestyle{chapter}{CIOpage}
\assignpagestyle{\part}{empty}

\setsecheadstyle{\color{MUSEBLUE}\scshape\large}
%\setsecnumdepth{part}
\setcounter{tocdepth}{0}

\newcommand{\keywords}[2]{\vspace{.2cm}\noindent \raggedright \textsc{\color{MUSEBLUE}Parole chiave}\\ #1 \\ \vspace{.2cm} \textsc{\color{MUSEBLUE}Keywords}\\ #2}

\usepackage{marginnote}
\reversemarginpar

\makeindex[title=,columns=2,options=-s CIOstyle.ist]

\renewcommand*{\marginfont}{\footnotesize}

\renewcommand*\copyright{{%
  \usefont{EU1}{lmr}{m}{n}\textcopyright}}

\titlespacing\section{0pt}{12pt plus 4pt minus 2pt}{0pt plus 2pt minus 2pt}


% Declaring "\authortoc" command to insert something (authors) before chapter name in TOC
\newcommand{\chaptertocindent}{\leftskip3em}
\makeatletter
\DeclareRobustCommand\authortoctext[1]{%
{\addvspace{10pt}\nopagebreak\chaptertocindent\relax
\rightskip \@tocrmarg\relax
\noindent\scshape\footnotesize#1\par\addvspace{-10pt}}}
\makeatother
\newcommand\authortoc[1]{%
  \gdef\chapterauthor{#1}%
  \addtocontents{toc}{\authortoctext{#1}}}

% Modify TOC appearance for chapters
\makeatletter
\def\l@chapter#1#2{\pagebreak[3]
   \vskip 1.0em plus 1pt  % space above chapter line
   \@tempdima 1.5em       % width of box holding chapter number
   \begingroup
     \parindent \z@ \rightskip \@pnumwidth
     \parfillskip -\@pnumwidth
     %\bfseries            % Boldface removed
     \leavevmode          % TeX command to enter horizontal mode.
     \chaptertocindent
     #1\dotfill \hbox to\@pnumwidth{\hss\normalfont\large\bfseries #2}\par
   \endgroup}
\makeatother


% For \male and \female symbols
\usepackage{wasysym}

\begin{document}
\pagestyle{empty}

\makeatletter
\renewcommand{\thetable}{\@arabic\c@table}
\@addtoreset{table}{chapter}
\renewcommand{\thefigure}{\@arabic\c@figure}
\@addtoreset{figure}{chapter}
\makeatother

\BackgroundFrontPic
\clearpage\null\newpage
\vspace*{\fill}
{
\normalsize
\raggedright
\textbf{\color{MUSEBLUE}A cura di} \\
Paolo Pedrini, Francesca Rossi, Giuseppe Bogliani, Lorenzo Serra e Anna Sustersic\\
\vspace{.5cm}
\textbf{\color{MUSEBLUE}Disegni} \\
Osvaldo Negra - MUSE - Museo delle Scienze di Trento \\
{\footnotesize
Re di Quaglie \emph{Crex crex}\\
Fraticello \emph{Sternula albifrons}\\
Nitticora \emph{Nycticorax nycticorax}\\
Pollo sultano \emph{Porphyrio porphyrio}\\
Logo - Pernice bianca \emph{Lagopus muta}\\
}
\vspace{.3cm}
\textbf{\color{MUSEBLUE}Grafica e impaginazione} \\
Aaron Iemma - MUSE - Museo delle Scienze di Trento\\
\vspace{.3cm}
\textbf{\color{MUSEBLUE}Foto di copertina} \\
Maurizio Bedin \\
{\footnotesize
Pernice bianca \emph{Lagopus muta}\\
}
\vspace{.3cm}
\textbf{\color{MUSEBLUE}Enti promotori ed organizzatori} \\
Centro italiano Studi Ornitologici \\
MUSE - Museo delle Scienze, Trento \\
Universit\`a degli Studi di Trento\\

\vspace{.3cm}
\textbf{\color{MUSEBLUE}Comitato Scientifico} \\
Daniela Campobello (segretario) - Universit\`a degli Studi di Palermo \\
Giuseppe Bogliani - Universit\`a degli Studi di Pavia \\
Claudio Carere - Universit\`a degli Studi della Tuscia, Viterbo \\
Dan Chamberlain - Universit\`a degli Studi di Torino \\
Marco Ciolli - Universit\`a degli Studi di Trento \\
Giacomo dell'Omo - Ornis Italica, Roma \\
Matteo Griggio - Konrad Lorentz Institute of Ethology, Vienna \\
Paolo Pedrini - MUSE - Museo delle Scienze, Trento \\
Fabrizio Sergio - Stazione Biologica di Do\~nana - CSIC, Siviglia, Spagna \\
Lorenzo Serra - ISPRA, Ozzano dell{\textquoteright}Emilia (BO) \\
Cecilia Soldatini - Universit\`a C\`a Foscari, Venezia \\
\vspace{.3cm}
\textbf{\color{MUSEBLUE}Comitato organizzatore} \\
Paolo Pedrini (coordinatore) - MUSE - Museo delle Scienze, Trento \\
Marco Ciolli - Universit\`a degli Studi di Trento \\
Samuela Caliari - MUSE - Museo delle Scienze, Trento \\
Daniela Campobello - Universit\`a degli Studi di Palermo \\
Elisa Maria Casati - MUSE - Museo delle Scienze, Trento \\
Sonia Caset - Universit\`a degli Studi di Trento \\
Francesca Chist\'e - Universit\`a degli Studi di Trento \\
Aaron Iemma - MUSE - Museo delle Scienze, Trento \\
Osvaldo Negra - MUSE - Museo delle Scienze, Trento \\
Francesca Rossi - MUSE - Museo delle Scienze, Trento \\
Cecilia Soldatini - Universit\`a Ca' Foscari, Venezia \\
Karol Tabarelli de Fatis - MUSE - Museo delle Scienze, Trento \\
Elisa Tessaro - MUSE - Museo delle Scienze, Trento \\
\vspace{.3cm}
\textbf{\color{MUSEBLUE}Stampato da} \\
Publistampa Arti Grafiche s.n.c. di Casagrande Silvio \& C. - 38057 Pergine Valsugana (TN), Via Dolomiti 36\\
\vspace{.2cm}
{\footnotesize
\textbf{\color{MUSEBLUE}Citazione consigliata} \\
Pedrini P., Rossi F., Bogliani G., Serra L. \& Sustersic A. (a cura di) 2015. \emph{XVII Convegno Italiano di Ornitologia: Atti del convegno di Trento}. Ed. MUSE, 176 pp. \\
}
\vspace{.7cm}
ISBN 978-88-531-0041-2 \\
\copyright  2015 MUSE - Museo delle Scienze, Trento \\
}

\newpage
\begin{adjustwidth}{-1cm}{-1cm}
{
\raggedright
\vspace*{4cm}
{\fontsize{1.5cm}{2em}\selectfont \color{MUSEBLUE!70!black} \textsc{XVII \\Convegno Italiano di Ornitologia}} \\[.2cm]
{\large Trento, 11 - 15 Settembre 2013} \\[.5cm]
{\HUGE \textbf{Atti del convegno}} \\[1cm]
}
\begin{center}
\begin{figure}[!h]
\includegraphics[width=.7\columnwidth]{logo.png}
\end{figure}
\end{center}
\vspace{7.5cm}
a cura di: Paolo Pedrini, Francesca Rossi, Giuseppe Bogliani, Lorenzo Serra e Anna Sustersic \\
\noindent\color{MUSEBLUE}\rule{27cm}{2pt}
\vspace*{\fill}
\end{adjustwidth}

\cleardoublepage
\pagestyle{CIOpage}
\pagenumbering{roman}
\begin{KeepFromToc}
	\tableofcontents
	%\listoftables
	%\listoffigures
\end{KeepFromToc}

\part{Introduzione}
\pagenumbering{arabic}

\chapterstyle{CIO}
{\large\itshape
Il biennale Convegno Italiano di Ornitologia che il CISO organizza in varie sedi d{\textquoteright}Italia, grazie alla collaborazione di Associazioni, enti e istituti di ricerca, rappresenta il momento scientifico nazionale pi\`u significativo per gli ornitologi italiani. L{\textquoteright}edizione svoltasi a Trento, tra l{\textquoteright}11 e il 13 settembre 2013, ha confermato la crescita dell{\textquoteright}ornitologia italiana, una tendenza gi\`a rilevata nelle precedenti edizioni. L{\textquoteright}incontro ha dato modo di apprezzare l{\textquoteright}intenso lavoro svolto negli ultimi anni e i diversi livelli di approfondimento degli studi in corso, dando vita a momenti di confronto e aggiornamento nelle diverse sessioni, workshop e tavole rotonde, previste nel ricco programma.


In questo volume edito da MUSE, sono stati raccolti 48 dei 165 lavori presentati in forma di comunicazioni (60) e poster (105), curati da oltre 400 autori. Gli abstract di tutti i lavori sono disponibili on line sul sito del CISO.


La ridotta frazione dei contributi qui pubblicati, rispetto alla grande quantit\`a di quelli proposti, \`e chiaro segno che in futuro i brevi contenuti del libro degli Atti lasceranno posto a lavori scientifici pi\`u completi, destinati alle riviste di settore nazionali e internazionali, o alla stessa Avocetta. Consapevoli di questo, siamo quindi oltremodo grati a tutti coloro che hanno risposto all{\textquoteright}invito, permettendo di tener fede all{\textquoteright}impegno assunto dagli organizzatori di pubblicare gli Atti, ora disponibili anche in \emph{.pdf} scaricabile dal sito del CISO. Per questo ringraziamo anche i referee che hanno lavorato in forma anonima e si sono fatti carico di valutare i lavori pervenuti e quanti hanno contribuito ai molti aspetti organizzativi, e in particolare a Aaron Iemma che ne ha curato la grafica e l{\textquoteright}impaginato, per la collaborazione nell{\textquoteright}aiuto redazione Giacomo Assandri, Mattia Brambilla, Alessandro Franzoi.


Un ringraziamento conclusivo va anche ai Colleghi del Comitato organizzatore e del Comitato scientifico, alla Segreteria del Convegno dell{\textquoteright}Universit\`a e a quanti altri hanno lavorato alla fase organizzativa dell{\textquoteright}evento e dei contenuti scientifici, cos\`i come ai curatori che hanno proposto, strutturato e condotto le tavole rotonde, le sessioni orali e i simposi a tema, che ricordiamo riproponendo il programma a seguire.
}

\vspace*{\fill}
\begin{figure}[!h]
\centering
\includegraphics[width=.7\columnwidth]{logo.png}
\end{figure}

\part{Programma}
\chapterstyle{CIO}
\pagestyle{CIOpage}
\addtocounter{figure}{1}
\addtocounter{table}{1}
\begin{center}
\vspace*{\fill}
\includegraphics[width=.8\columnwidth]{osv_5.png}
\vspace*{\fill}
\end{center}
\newpagecolor{white}\afterpage{\restorepagecolor}

\newpage
{
\setlength{\parindent}{0cm}
\textcolor{MUSEBLUE}{\textsc{\LARGE Mercoled\`i, 11 Settembre 2013}} \\
\vspace{.5cm}\\
\textbf{MUSE} \\
18:30 - 21:30 Benvenuto e registrazione \\
\vspace{.5cm}\\
\textcolor{MUSEBLUE}{\textsc{\LARGE Gioved\`i, 12 Settembre 2013}} \\
\vspace{.5cm}\\
\textbf{Universit\`a degli Studi di Trento - Dipartimento di Lettere e Filosofia} \\
08:00 - Apertura e registrazione \\
09:00 - Saluti di benvenuto \\ 
\vspace{.5cm} \\
{\bfseries \Large Effetti del cambiamento climatico}\\
{\bfseries Moderatori:} N. Saino, L. Serra\\
{\color{MUSEBLUE}{\hrule}}
\begin{description}[leftmargin=1cm,labelindent=1cm]\itemsep0pt
	\item[09:30 - M. Morganti] \emph{Uccelli migratori transahariani svernanti a nord del Sahara: un
adattamento al cambiamento climatico?} 
	\item[09:50 - A. Montemaggiori] \emph{Modelling the response of European breeding birds to climate
change: combining expert-base and statistical approach} 
	\item[10:10 - M. Sar\`a] \emph{Effetti del clima sul successo riproduttivo del grillaio \emph{Falco naumanni}} 
	\item[10:30 - G. Masoero] \emph{Trend della popolazione di topino \emph{Riparia riparia} nel Parco del Po e
della Collina Torinese}
\end{description}
{\color{MUSEBLUE}{\hrule height 2pt}}
\vspace{1cm}
{\bfseries \Large Ecologia e dinamiche di popolazione} \\
{\bfseries Moderatori:} R. Ambrosini, M. Sar\`a\\
{\color{MUSEBLUE}{\hrule}}
\begin{description}[leftmargin=1cm,labelindent=1cm]\itemsep0pt
	\item[11:20 - B. Sicurella] \emph{Livestock farming and hayfields may buffer barn swallow \emph{Hirundo
rustica} population decline}
	\item[11:40 - S. Tenan] \emph{Modelli gerarchici Bayesiani in ornitologia}
	\item[12:00 - P. Trotti] \emph{Ecologia e produttivit\`a del gufo reale \emph{Bubo bubo} in due aree di
studio della Lombardia}
	\item[12:20 - D. De Rosa] \emph{Distribuzione dei Picidi nel Parco Nazionale del Cilento e Vallo di
Diano}
\end{description}
{\color{MUSEBLUE}{\hrule height 2pt}}
\vspace{1cm}
{\bfseries \Large \raggedright Specie e popolazioni problematiche: conoscenza e gestione dei \\conflitti} \\
{\bfseries Moderatori:} C. Carere, A. Montemaggiori\\
{\color{MUSEBLUE}{\hrule}}
\begin{description}[leftmargin=1cm,labelindent=1cm]\itemsep0pt
	\item[11:20 - P. Sposimo] \emph{Gli uccelli alloctoni in Italia: analisi degli impatti e proposte gestionali}
	\item[11:40 - S. Calandri] \emph{Il problema dei colombi \emph{Columba livia} urbani affrontato con un
esperimento di riduzione della risorsa acqua}
	\item[12:00 - C. Carere] \emph{The use of the distress call to relocate starlings \emph{Sturnus vulgaris}
from urban roosts: effectiveness and limitations}
\end{description}
\newpage
{\bfseries 14:00 - 14:50}
\begin{description}[leftmargin=1cm,labelindent=1cm,style=unboxed]\itemsep0pt
	\item[Sessione plenaria - From conservation biology to ecosystem engineering: bridging the great divide
between research and action - Rapha\:el Arlettaz]
\end{description}
{\color{MUSEBLUE}{\hrule height 2pt}}
\vspace{1cm} 
{\bfseries \Large \raggedright Biologia riproduttiva} \\
{\bfseries Moderatori:} D. Campobello, M. Griggio\\
{\color{MUSEBLUE}{\hrule}}
\begin{description}[leftmargin=1cm,labelindent=1cm]\itemsep0pt
	\item[14:55 - N. Saino] \emph{Phenological variation in timing of molt and breeding in relation to
clock polymorphism in the barn swallow \emph{Hirundo rustica}}
	\item[15:15 - D. Nespoli] \emph{Somiglianza morfologica versus playback: una nuova specie di
sterpazzolina?}
	\item[15:35 - R. Di Maggio] \emph{Effects of microclimate on nest site selection and breeding success
of lesser kestrel \emph{Falco naumanni} in the Gela Plain (Sicily)}
	\item[15:55 - E. Bassi] \emph{Analisi del successo riproduttivo dell{\textquoteright}aquila reale \emph{Aquila
chrysaetos} nel Parco Nazionale dello Stelvio in relazione al ritorno
del gipeto \emph{Gypaetus barbatus} sulle Alpi}
\end{description}
{\color{MUSEBLUE}{\hrule height 2pt}}
\vspace{1cm}
{\bfseries \Large \raggedright Ecologia alpina - cambiamenti ad alta quota} \\
{\bfseries Moderatori:} D. E. Chamberlain, P. Pedrini\\
{\color{MUSEBLUE}{\hrule}}
\begin{description}[leftmargin=1cm,labelindent=1cm]\itemsep0pt
	\item[14:55 - A. Rolando] \emph{Skiing and bird diversity in the Alps}
	\item[15:15 - S. Imperio] \emph{Climate change and human disturbance effects on the population
dynamics of Alpine grouses: new insight from the western Italian Alps}
	\item[15:35 - T. Sitzia] \emph{Altitudinal and habitat selection of capercaillie \emph{Tetrao urogallus}: an
assessment of change after ten years in the Scanuppia Reserve (Central Alps, Northern Italy)}
	\item[15:55 - D. E. Chamberlain] \emph{Addressing the challenges of surveying birds at high altitude}
\end{description}
{\color{MUSEBLUE}{\hrule height 2pt}}
\vspace{1cm}
{\bfseries \Large \raggedright Comportamento e fisiologia} \\
{\bfseries Moderatori:} E. Caprio, L. Fusani\\
{\color{MUSEBLUE}{\hrule}}
\begin{description}[leftmargin=1cm,labelindent=1cm]\itemsep0pt
	\item[16:45 - D. Baldan] \emph{Effects of temporal shortage of food on social network and
leadership in house sparrow \emph{Passer domesticus} flocks}
	\item[17:05 - J. G. Cecere] \emph{Berte mannare: l{\textquoteright}effetto della luna sul comportamento della berta maggiore \emph{Calonectris diomedea}}
	\item[17:25 - D. Campobello] \emph{Nest attendance of conspecifics and heterospecifics as social
phenotypes affecting breeding lesser kestrels \emph{Falco naumanni}}
	\item[17:45 - A. Gagliardi] \emph{Dinamica di insediamento del cormorano \emph{Phalacrocorax carbo} nelle
garzaie e possibili effetti di competizione}
\end{description}
{\color{MUSEBLUE}{\hrule height 2pt}}
\newpage
{\bfseries \Large \raggedright Marine strategy e uccelli} \\
{\bfseries Moderatori:} N. Baccetti, M. Zenatello\\
{\color{MUSEBLUE}{\hrule}}
\begin{description}[leftmargin=1cm,labelindent=1cm]\itemsep0pt
	\item[16:45 - M. Zenatello] \emph{Spunta la luna dal monte? Strategie di foraggiamento delle berte
minori \emph{Puffinus yelkouan} dell{\textquoteright}isola di Tavolara}
	\item[17:05 - S. Sponza] \emph{Caratteristiche spaziali e temporali della migrazione del marangone
dal ciuffo \emph{Phalacrocorax aristotelis desmarestii} nell{\textquoteright}Adriatico}
settentrionale
	\item[17:25 B. Massa] \emph{Nocturnal activity of shearwaters at the colony: a study with radar}
	\item[17:45 - P. Aragno] \emph{Monitoraggio mensile di uccelli marini attraverso il Santuario
Pelagos: una collaborazione ISPRA-COT-CorsicaFerries}
\end{description}
{\color{MUSEBLUE}{\hrule height 2pt}}
\vspace{1cm} 
{\bfseries 18:05 - 19:00}
\begin{description}[leftmargin=1cm,labelindent=1cm,style=unboxed]\itemsep0pt
	\item[\bfseries  \raggedright Sessione Poster con promozione Avocetta - Journal of Ornithology]
	\item[\bfseries  \raggedright Tavola rotonda - Piano di Azione Nazionale per il marangone minore \emph{Phalacrocorax  pygmeus} - S. Volponi, M. Costa]
\end{description}
{\color{MUSEBLUE}{\hrule height 2pt}}
\vspace{1cm}
\textcolor{MUSEBLUE}{\textsc{\LARGE Venerd\`i, 13 Settembre 2013}}
\vspace{.5cm}\\
\textbf{Universit\`a degli Studi di Trento - Dipartimento di Lettere e Filosofia} \\ 
08:30 - Apertura e registrazione \\
09:10 - 09:30 Annunci\\
\vspace{.5cm}\\
{\bfseries \Large \raggedright Conservazione e gestione della biodiversit\`a: agricoltura, attivit\`a produttive e aree protette (I)} \\
{\bfseries Moderatori:} G. Bogliani, J. G. Cecere\\
{\color{MUSEBLUE}{\hrule}}
\begin{description}[leftmargin=1cm,labelindent=1cm]\itemsep0pt
	\item[09:30 - E. Cardarelli] \emph{Cambiamenti colturali delle risaie italiane e disponibilit\`a trofica per
gli Ardeidi}
	\item[09:50 - V. Longoni] \emph{Un esperimento di recupero del valore naturalistico delle risaie: il
progetto CORINAT}
	\item[10:10 - G. Assandri] \emph{Selezione dell{\textquoteright}habitat nel ciclo annuale in una popolazione spagnola
di capinere Sylvia atricapilla: l{\textquoteright}importanza della coltivazione non intensiva degli uliveti}
	\item[10:30 - P. Pedrini] \emph{Andamento demografico del re di quaglie \emph{Crex crex} nell{\textquoteright}Italia nord orientale}
\end{description}
{\color{MUSEBLUE}{\hrule height 2pt}}
\vspace{1cm}
{\bfseries \Large \raggedright Conservazione e gestione della biodiversit\`a: agricoltura, attivit\`a produttive e aree protette (II)} \\
{\bfseries Moderatori:} M. Brambilla, M. Fasola\\
{\color{MUSEBLUE}{\hrule}}
\begin{description}[leftmargin=1cm,labelindent=1cm]\itemsep0pt
	\item[11:20 - S. Alemanno] \emph{La coturnice appenninica \emph{Alectoris graeca} nel Parco Nazionale dei
Monti Sibillini}
	\item[11:40 - G.Tellini Florenzano] \emph{Pi\`u grandi, pi\`u vecchie: come rispondono le specie forestali diffuse alla trasformazione delle foreste italiane}
	\item[12:00 - S. G. Fasano] \emph{Importanza conservazionistica della Rete Natura 2000 in Liguria}
	\item[12:20 - G. Calvi] \emph{Monitoraggio degli uccelli comuni nidificanti e politiche di conservazione degli agro ecosistemi}
\end{description}
{\color{MUSEBLUE}{\hrule height 2pt}}
\vspace{1cm}
{\bfseries 11:20 - 12:40}
\begin{description}[leftmargin=1cm,labelindent=1cm,style=unboxed]\itemsep0pt
	\item[\bfseries  \raggedright Tavola rotonda - Conservazione delle collezioni museali - F. Barbagli, F. Foschi, B. Massa]
\end{description}
{\color{MUSEBLUE}{\hrule height 2pt}}
\vspace{1cm}
{\bfseries 14:00 - 14:50}
\begin{description}[leftmargin=1cm,labelindent=1cm,style=unboxed]\itemsep0pt
	\item[\bfseries  \raggedright Sessione plenaria - Will tropical avian frugivores fare better inside or outside protected areas? - Stuart Madsen]
\end{description}
{\color{MUSEBLUE}{\hrule height 2pt}}
\vspace{1cm}
{\bfseries \Large \raggedright Migrazione}\\
{\bfseries Moderatori:} D. Rubolini, U. Mellone\\
{\color{MUSEBLUE}{\hrule}}
\begin{description}[leftmargin=1cm,labelindent=1cm]\itemsep0pt
	\item[14:55 - R. Ambrosini] \emph{Modelling bird migration by means of ring recoveries}
	\item[15:15 - M. Panuccio] \emph{Priorit\`a di monitoraggio e conservazione per i rapaci migratori in
Italia}
	\item[15:35 - S. Lupi] \emph{Le condizioni fisiologiche influenzano la decisione di stopover
durante la migrazione nel codirosso spazzacamino \emph{Phoenicurus ochruros}, nel pettirosso \emph{Erithacus rubecula} e nel saltimpalo \emph{Saxicola torquata}}
\end{description}
{\color{MUSEBLUE}{\hrule height 2pt}}
\vspace{1cm}
{\bfseries \Large \raggedright Genetica di conservazione: casi studio e prospettive}\\
{\bfseries Moderatori:} H. Hauffe, C. Vernesi\\
{\color{MUSEBLUE}{\hrule}}
\begin{description}[leftmargin=1cm,labelindent=1cm]\itemsep0pt
	\item[14:55 - E. Randi] \emph{Le indagini molecolari in aiuto all{\textquoteright}ornitologia: l{\textquoteright}esempio del complesso \emph{Sylvia cantillans}}
	\item[15:15 - B. Crestanello] \emph{Il contributo della genetica della conservazione alla conoscenza e alla
gestione di alcune specie di Galliformi alpini: pernice bianca \emph{Lagopus muta}, fagiano di monte \emph{Tetrao tetrix} e gallo cedrone \emph{Tetrao urogallus}}
	\item[15:35 - A. Galimberti] \emph{La dieta vegetale dei Passeriformi in migrazione: dispersori di
biodiversit\`a o di specie aliene? Un approccio molecolare}
\end{description}
{\color{MUSEBLUE}{\hrule height 2pt}}
\vspace{1cm}
{\bfseries \Large \raggedright GIS e modelli ecologici}\\
{\bfseries Moderatori:} M. Ciolli, S. Imperio\\
{\color{MUSEBLUE}{\hrule}}\
\begin{description}[leftmargin=1cm,labelindent=1cm]\itemsep0pt
	\item[16:25 - M. Brambilla] \emph{I modelli di distribuzione per la conservazione di specie ornitiche:
oltre la semplice individuazione di aree idonee}
	\item[16:45 - C. Tattoni] \emph{L{\textquoteright}approccio metodologico alla Rete Ecologica del Trentino attraverso
l{\textquoteright}utilizzo dell{\textquoteright}avifauna come modello biologico}
	\item[17:05 - L. Nelli] \emph{Census methods of alpine galliforms: an experimental comparison}
	\item[17:25 - G. Londi] \emph{Studio dell'ecologia e monitoraggio delle popolazioni di \emph{Picus canus}, \emph{Dryocopus martius}, e \emph{Picoides tridactylus} nelle Alpi Carniche}
\end{description}
\vspace{1cm}
{\bfseries \Large \raggedright Canti, penne e uova: selezione sessuale in corso!}\\
{\bfseries Moderatori:} M. Griggio, G. Malacarne\\
{\color{MUSEBLUE}{\hrule}}
\begin{description}[leftmargin=1cm,labelindent=1cm]\itemsep0pt
	\item[16:25 - G. Malacarne] \emph{Selezione sessuale: sviluppi recenti}
	\item[16:45 - H. Hoi] \emph{Females produce more colourful eggs when males sing faster}
	\item[17:05 - S. Pirrello] \emph{Le femmine di storno \emph{Sturnus vulgaris} scelgono i maschi che sono
cresciuti in migliori condizioni ambientali}
	\item[17:25 - M. Griggio] \emph{Sexual selection, multiple traits and pair-bond behaviour in bearded
reedlings \emph{Panurus biarmicus}}
\end{description}
{\color{MUSEBLUE}{\hrule height 2pt}}
\vspace{1cm}
{\bfseries 17:45 - 19:00}
\begin{description}[leftmargin=1cm,labelindent=1cm,style=unboxed]\itemsep0pt
	\item[\bfseries  \raggedright Sessione poster]
	\item[Assemblea CISO]
\end{description}
{\color{MUSEBLUE}{\hrule height 2pt}}
\vspace{.5cm}\\
20:30 - Cena sociale - serata con eventi e visita al MUSE - Museo delle Scienze \\
\vspace{.5cm}

\noindent\textcolor{MUSEBLUE}{\textsc{\LARGE Sabato, 14 Settembre 2013}} \\
\vspace{.2cm}

{\bfseries Universit\`a degli Studi di Trento - Dipartimento di Lettere e Filosofia}\\
08:30 Apertura e registrazione \\
09:10 - 09:30 Annunci \\
\vspace{.2cm}

{\bfseries 9:30 - 10:20}
\begin{description}[leftmargin=1cm,labelindent=1cm,style=unboxed]\itemsep0pt
	\item[Sessione plenaria - Tales of the unexpected: new insights from satellite tracking common cuckoos
\emph{Cuculus canorus} - Chris Hewson]
\end{description}
{\color{MUSEBLUE}{\hrule height 2pt}}
\vspace{1cm}
{\bfseries \Large \raggedright Movimenti e tecnologie}\\
{\bfseries Moderatori:} D. Giunchi, M. Panuccio\\
{\color{MUSEBLUE}{\hrule}}
\begin{description}[leftmargin=1cm,labelindent=1cm]
	\item[10:55 - U. Mellone] \emph{Tra Europa e Africa: i movimenti di falco della Regina \emph{Falco eleonorae},
grillaio \emph{Falco naumanni} e biancone \emph{Circaetus gallicus} rivelati dalla telemetria satellitare}
	\item[11:15 - B. Massa] \emph{Winter movements of Scopoli{\textquoteright}s shearwaters \emph{Calonectris diomedea} breeding in Linosa island}
	\item[11:35 - M. Santini] \emph{L{\textquoteright}uso del radar per lo studio del comportamento negli uccelli}
\end{description}
{\color{MUSEBLUE}{\hrule height 2pt}}
\newpage
{\bfseries 10:55 - 11:55}
\begin{description}[leftmargin=1cm,labelindent=1cm,style=unboxed]\itemsep0pt
	\item[Tavola rotonda - Costituzione del "Comitato nazionale per la conservazione del fratino \emph{Charadrius alexandrinus} - A. Sartori, L. Serra, R. Tinarelli]
\end{description}
{\color{MUSEBLUE}{\hrule height 2pt}}
\vspace{1cm}
{\bfseries 12:00 - 13:20}
\begin{description}[leftmargin=1cm,labelindent=1cm,style=unboxed]\itemsep0pt
	\item[Tavola rotonda - La conservazione degli uccelli a scala nazionale: elementi di conoscenza, criticit\`a, prospettive future - C. Celada, M. Gustin]
\end{description}
{\color{MUSEBLUE}{\hrule height 2pt}}
\vspace{1cm}
{\bfseries \Large \raggedright Conservazione e monitoraggio dell{\textquoteright}avifauna urbana, pianificazione del territorio e barriere infrastrutturali}\\
{\bfseries Moderatori:} M. Dinetti, M. Fraissinet\\
{\color{MUSEBLUE}{\hrule}}
\begin{description}[leftmargin=1cm,labelindent=1cm]
	\item[12:00 - M. Dinetti] \emph{Avifauna urbana, pianificazione del territorio e barriere
infrastrutturali}
	\item[12:20 - C. Perticarari] \emph{Fattori che condizionano la nidificazione di alcune specie in vecchi
edifici: il caso dell{\textquoteright}Abbadia di Fiastra (MC)}
	\item[12:40 - M. Bon] \emph{Dall{\textquoteright}Atlante ornitologico all{\textquoteright}Atlante della Laguna: l'utilizzo dei dati
faunistici per la pianificazione territoriale nel Comune di Venezia}
	\item[13:00 - M. Fraissinet] \emph{Evoluzione della comunit\`a ornitica nidificante in un parco urbano di
impianto recente}
\end{description}
{\color{MUSEBLUE}{\hrule height 2pt}}
\vspace{1cm}
{\bfseries 14:40 - 16:00}
\begin{description}[leftmargin=1cm,labelindent=1cm,style=unboxed]\itemsep0pt
	\item[Tavola rotonda - Atlante degli Uccelli nidificanti e svernanti - R. Lardelli, L. Serra]
\end{description}
{\color{MUSEBLUE}{\hrule height 2pt}}
\vspace{1cm}
{\bfseries \Large \raggedright Tecnologie per l'ornitologia: dalla cattura, all'applicazione di strumenti, all'analisi dei dati}\\
{\bfseries Moderatori:} G. Dell'Omo, U. Mellone\\
{\color{MUSEBLUE}{\hrule}}
\begin{description}[leftmargin=1cm,labelindent=1cm]\itemsep0pt
	\item[14:40 - J. G. Cecere] \emph{Cattura di uccelli a scopo scientifico: normativa e iter autorizzativo}
	\item[15:00 - C. Catoni] \emph{Cattura e applicazione di strumenti per lo studio degli uccelli}
	\item[15:20 - G. Dell'Omo] \emph{Ornithology and technology}
	\item[15:40 U. Mellone] \emph{Monitoraggio dei movimenti a livello individuale}
\end{description}
{\color{MUSEBLUE}{\hrule height 2pt}}
\vspace{.5cm}

16:00 - 16:30 Premiazione Poster \\
16:30 - 17:00 Chiusura Convegno \\

\part{Contributi}
\setcounter{figure}{0}
\setcounter{table}{0}
{\hspace*{\fill} \emph{In ordine alfabetico secondo il primo Autore}}
\begin{center}
\vspace*{\fill}
\includegraphics[width=.95\columnwidth]{osv_3.png}
\vspace*{\fill}
\end{center}
\newpagecolor{white}\afterpage{\restorepagecolor}
\pagestyle{CIOpage}

\setcounter{figure}{0}
\setcounter{table}{0}

\begin{adjustwidth}{-3.5cm}{0cm}
\pagestyle{CIOpage}
\authortoc{Alemanno S., Ragni B.}
\chapter*[La coturnice nel Parco Nazionale dei Sibillini]{La coturnice appenninica \textbf{\textit{Alectoris
graeca}}\textbf{ nel Parco Nazionale dei Monti Sibillini}}
\addcontentsline{toc}{chapter}{La coturnice nel Parco Nazionale dei Sibillini}

\textsc{Simone Alemanno}$^{1}$, \textsc{Bernardino Ragni}$^{1*}$ \\

\index{Alemanno Simone} \index{Ragni Bernardino}
\noindent\color{MUSEBLUE}\rule{27cm}{2pt}
\vspace{1cm}
\end{adjustwidth}
\marginnote{\raggedright $^1$Dipartimento di Chimica, Biologia e
Biotecnologie, Universit\`a degli Studi di Perugia \\
\vspace{.5cm}
{\emph{\small $^*$Autore per la corrispondenza: \href{mailto:bernardino.ragni@unipg.it}{ber\allowbreak nar\allowbreak di\allowbreak no.\allowbreak ra\allowbreak gni@\allowbreak uni\allowbreak pg.\allowbreak it}}} \\
\keywords{\textit{Alectoris graeca}, Appennino Centrale, Parco
Nazionale Monti Sibillini}
{\textit{Alectoris graeca}, Central  Appennines, Monti
Sibillini National Park}
%\index{keywords}{\textit{Alectoris graeca}} \index{keywords}{Appennino Centrale} \index{keywords}{Parco Nazionale Monti Sibillini}
}
{
\small
\noindent \textsc{\color{MUSEBLUE} Summary} / The aim of this research is to define the \textit{status} (past and
present distribution, abundance, habitat and anthropogenic critical
factors) of rock partridge \textit{Alectoris graeca} in the umbrian
part of Sibillini mountains NP (Central Apennines). The species was
ascertained throughout 1970-2010 with a population density probably
conditioned by a heavy human pressure. \\
\noindent \textsc{\color{MUSEBLUE} Riassunto} / Lo studio si prefigge lo scopo di definire lo status della coturnice
\textit{Alectoris graeca} nella porzione umbra del Parco Nazionale dei
Monti Sibillini, Appennino Umbro-marchigiano, tramite la definizione
di: distribuzione pregressa e attuale, abbondanza specifica, selezione
dell{\textquoteright}habitat, presenza di fattori critici di origine
antropica. Il galliforme risulta ininterrottamente presente
nell{\textquoteright}area di studio, almeno nel quarantennio 1970-2010;
anche se l{\textquoteright}abbondanza specifica appare condizionata da
una pesante pressione antropica, la popolazione di coturnice
dell{\textquoteright}area protetta pu\`o rappresentare una importante
{\textquotedblleft}sorgente{\textquotedblright} per il circostante
{\textquotedblleft}bacino{\textquotedblright} montano dove, assieme
agli stessi fattori critici osservati nello studio, \`e presente una
diffusa attivit\`a di bracconaggio. \\
}


\section*{Introduzione}

La coturnice pu\`o essere considerata una specie bandiera (Caro
\textit{et al}. 2004) degli ambienti montani della catena appenninica.

La sua presenza in cCentro Italia risulta discontinua e localizzata in
sub-areali disgiunti (Brichetti \& Fracasso 2004) e diversi studi,
dagli anni Novanta dello scorso secolo, si sono occupati di \textit{A.
graeca }nell{\textquoteright}Appennino Umbro-marchigiano. Le ricerche,
in genere, l{\textquoteright}hanno interessata prevalentemente nella
fascia orografica d{\textquoteright}elezione, compresa tra i 1600--2200
m s.l.m. (Petretti 1985) entro la quale, nella Penisola centrale,
insistono vaste aree protette appenniniche; meno frequentemente gli
studi si sono svolti entro intervalli altitudinali inferiori.

Il presente studio triennale \`e stato portato a termine
all{\textquoteright}interno di un{\textquoteright}area protetta
nazionale, ma a una quota inferiore (compresa tra i 1168 ed i 1883 m
s.l.m.). Dal momento che a pi\`u di quindici anni
dall{\textquoteright}istituzione del Parco Nazionale dei Monti
Sibillini, la specie poteva risultare fortemente rarefatta o
addirittura scomparsa, analogamente a quanto accaduto in gran parte
dell{\textquoteright}Appennino (Magrini \& Gambaro 1997; Velatta
\textit{et al.} 2009) nel presente lavoro si \`e voluta saggiare tale
ipotesi, ricostruendo l{\textquoteright}areale pregresso della specie e
confrontandolo con quello attuale. Si sono inoltre indagati:
l{\textquoteright}abbondanza relativa della coturnice,
l{\textquoteright}eventuale scelta rispetto a fattori ambientali
predefiniti, gli eventuali fattori avversi, la presenza e la permanenza
della specie nell{\textquoteright}area.

\begin{figure}[!h]
\centering
\includegraphics[width=.8\columnwidth]{Alemanno_fig_1.png}

\caption{Ubicazione dell{\textquoteright}area di studio (ADS) nel Parco Nazionale dei Monti Sibillini (Appennino Umbro-marchigiano)}
\label{Alemanno_fig_1}
\end{figure}

\section*{Metodi}

L{\textquoteright}area di studio (ADS) riguarda l{\textquoteright}ambito
sud-occidentale del Parco, compreso nel SIC-ZPS
{\textquotedblleft}Monti Sibillini{\textquotedblright} in territorio
umbro. Essa \`e stata suddivisa in 56 celle (Unit\`a Spaziali di
Rilevamento, USR) quadrate -- di 1 km di lato -- a partire dal reticolo
UTM (Fig. \ref{Alemanno_fig_1}). L{\textquoteright}ADS \`e stata suddivisa in tre settori:
settentrionale (SS, 19 USR) centrale (SC, 21 USR) meridionale (SM, 6
USR).

Sono stati impiegati i seguenti metodi: inchiesta per intervista diretta, metodo cinegetico (MC) e
metodo naturalistico (MN).

Inchiesta: sono stati somministrati questionari a persone interessate
alla gestione fa\allowbreak u\allowbreak nis\allowbreak ti\allowbreak co-\allowbreak ve\allowbreak na\allowbreak to\allowbreak ria nel periodo precedente
l{\textquoteright}istituzione del Parco, per la determinazione dei siti
d{\textquoteright}involo e dei carnieri sui quali ricostruire
l{\textquoteright}areale pregresso di \textit{A. graeca}.

Il MC ha visto l{\textquoteright}impiego di cani da ferma addestrati per
la cerca e l{\textquoteright}individuazione dei galliformi
all{\textquoteright}interno di aree campione localizzate in ognuno dei
settori dell{\textquoteright}ADS. Tale metodo ha consentito la stima
del deme (Zunino \& Zullini 1995) post-riproduttivo e
l{\textquoteright}elaborazione dell{\textquoteright}indice cinegetico
di abbondanza della specie indagata,
l{\textquoteright}ICA\textsubscript{S} (Tab. \ref{Alemanno_tab_1}).

Il MN ha consentito il rilevamento diretto (osservazione e/o ascolto di
individui) e indiretto (osservazione e/o raccolta di feci, piume e
penne, orme e piste,
{\textquotedblleft}spollinatoi{\textquotedblright}, uova, nidi, ecc.)
della specie, percorrendo ripetutamente a piedi undici transetti (22
giornate, 167 km). L{\textquoteright}azione \`e stata integrata, in
periodo pre-riproduttivo, con la stimolazione al canto dei maschi
territoriali eventualmente presenti, tramite 65 stazioni di
emissione-ascolto, disposte sulla rete di transetti. Sui risultati
cos\`i ottenuti si basano: carte della distribuzione specifica e dei
maschi territoriali; indici specifici: di diffusione
ID\textsubscript{S}, chilometrico di abbondanza IKA\textsubscript{S},
puntiforme di abbondanza IPA\textsubscript{S }(Tab. \ref{Alemanno_tab_1}).

Tramite la rete di transetti si sono rilevate anche la presenza,
l{\textquoteright}abbondanza, la variazione nello spazio, di fattori
ecologici, favorevoli (14 categorie tra vegetazionali,
orografico-morfologiche e idriche) o avversi (6 categorie di disturbo
antropico) ritenuti significativi per la specie oggetto di studio.
Successivamente \`e stata verificata l{\textquoteright}esistenza di
correlazione tra gli indici di presenza, rilevati
nell{\textquoteright}ADS, e la
{\textquotedblleft}offerta{\textquotedblright} qualitativa e
quantitativa di tali fattori.

Quelli avversi sono stati {\textquotedblleft}pesati{\textquotedblright}
tramite punteggio ordinale di valore decrescente, basato
sull{\textquoteright}esperienza degli autori e di studiosi della specie
nell{\textquoteright}Appennino (Tab. \ref{Alemanno_tab_2}).

I dati raccolti sul campo sono stati saggiati per mezzo dei seguenti
\textit{test }statistici: \textit{Goodness of fit},
${\chi}$\textsuperscript{2}, nelle associazioni tra variabili;
Coefficiente di correlazione di Spearman, r\textsubscript{S}, nelle
relazioni tra variabili (Fowler \& Cohen 2002; Jacobs 1974).

\begin{table}[!h]
\centering
\begin{tabular}{>{\raggedright\arraybackslash}p{.2\columnwidth}>{\raggedright\arraybackslash}p{.7\columnwidth}}
\toprule
\textbf{Indice} & \textbf{Algoritmo} \\ \midrule
Indice di Diffusione specifica (IDS) & 
IDS = USRS / USRT; dove USRS: numero di celle nelle quali \`e accertata la presenza della specie S, USRT: totale delle celle esplorate; per IDS = 1, la diffusione della specie \`e la massima possibile, essa ha “saturato” tutto lo spazio disponibile, per IDS = 0, la diffusione specifica \`e nulla, essa \`e assente dall{\textquoteright}ADS \\ \midrule
Indice Cinegetico di Abbondanza specifica (ICAS) &
ICAS = NS / SI; dove: NS = numero di contatti stabiliti con la specie indipendentemente dagli individui; SI = estensione dell{\textquoteright}area effettivamente indagata espressa in km$^2$ \\ \midrule
Indice Chilometrico di Abbondanza specifica (IKAS) &
IKAS = $\sum$IPS / LT(R); dove: $\sum$IPS: sommatoria degli indici di presenza della specie S raccolti lungo il transetto T o la rete di transetti R; LT: lunghezza in chilometri del transetto T percorso, LR: lunghezza in km della rete di transetti R percorsa; per IKAS = 0, l{\textquoteright}abbondanza della specie S nel transetto T o nella rete R \`e nulla, la specie non vi \`e stata rilevata, IKAS> 0, l{\textquoteright}abbondanza della specie S nel transetto T o nella rete R \`e tanto pi\`u elevata quanto pi\`u il parametro \`e maggiore di 0 \\ \midrule
Indice Puntiforme di Abbondanza specifica (IPAS) &
IPAS = $\sum$RP / NP; dove, $\sum$RP: sommatoria delle risposte emesse dalla specie S rilevate dalle stazioni puntiformi P; NP: numero delle stazioni puntiformi dalle quali sono stati emessi richiami; per IPAS = 0, l{\textquoteright}abbondanza della specie S nell{\textquoteright}insieme dei punti-stazione \`e nulla, la specie non vi \`e stata rilevata, IPAS> 0, l{\textquoteright}abbondanza della specie S nell{\textquoteright}area di studio \`e tanto pi\`u elevata quanto pi\`u il parametro \`e maggiore di 0 \\ \midrule
Indice Chilometrico delle Risposte specifiche (IKRS) &
IKRS = $\sum$RS / LT(R); dove $\sum$RS: sommatoria delle risposte emesse dalla specie S e rilevati lungo il transetto T o la rete di transetti R; LT: lunghezza in chilometri del transetto T percorso, LR: lunghezza in km della rete di transetti R percorsa; per IKRS = 0, l{\textquoteright}abbondanza della specie S nel transetto T o nella rete R \`e nulla, la specie non vi \`e stata rilevata, IKRS> 0, l{\textquoteright}abbondanza della specie S nel transetto T o nella rete R \`e tanto pi\`u elevata quanto pi\`u il parametro \`e maggiore di 0 \\ \bottomrule
\end{tabular}

\caption{}
\label{Alemanno_tab_1}
\end{table}


\section*{Discussione}

L{\textquoteright}areale pregresso (1970--1990) presenta un
ID\textsubscript{C} (Tab. \ref{Alemanno_tab_1}) = 0,48; quello attuale \`e risultato
essere 0,50: il grado di sovrapposizione del secondo sul primo \`e 0,74
(Fig. \ref{Alemanno_fig_2}). L{\textquoteright}IKA\textsubscript{C} medio per
l{\textquoteright}intera ADS \`e 0,49; 0,59 per SS, 0,81 per SC e 0,43
per il SM.


\begin{figure}[!h]
\centering
	\includegraphics[width=.8\columnwidth]{Alemanno_fig_2.png}

\caption{Sovrapposizione degli areali pregresso (1970) e attuale (2010) della coturnice nell{\textquoteright}ADS}	
\label{Alemanno_fig_2}
\end{figure}



La stimolazione al canto ha ottenuto 8 risposte di maschi territoriali,
con IPA\textsubscript{C} = 0,12, valore uguale a quello ottenuto in una
precedente ricerca svolta nell{\textquoteright}intero comprensorio dei
Sibillini (Renzini \& Ragni 1998). Le distanze fra i centroidi delle
USR positive alla stimolazione risultano non significativamente diverse
(p{\textless}0,05) dal valore medio, pari a  m 2583;
l{\textquoteright}equiripartizione spaziale dei galli suggerisce la
{\textquotedblleft}saturazione{\textquotedblright} dello spazio
ecologicamente idoneo nell{\textquoteright}ADS.

I valori dell{\textquoteright}indice chilometrico costruito sulle
risposte, IKR\textsubscript{C}  e dello IPA\textsubscript{C} risultano
congruenti nel loro andamento spaziale: r\textsubscript{S}= 0,98
p{\textless}0,01. L{\textquoteright}ICA\textsubscript{C} medio
nell{\textquoteright}ADS \`e pari a 0,5: nel SS il valore \`e nullo,
nel SC \`e 1,45, nel SM 0,24. Vi \`e una congruenza parziale con quelli
risultanti dal MN, in quanto le
{\textquotedblleft}attese{\textquotedblright} per il SS risultano
insoddisfatte: ci\`o \`e imputabile a fattori contingenti che hanno
causato un minore sforzo di campionamento con MC in tale Settore. 

Riguardo alla selezione dell{\textquoteright}\textit{habitat}, la specie
sceglie, significativamente: cotico erboso continuo e discontinuo;
vegetazione arboreo-arbustiva frammentata e con basso grado di
copertura, associabile a una maggiore offerta in alimento ed in rifugio
dalla predazione di \textit{Aquila chrysaetos},  \textit{A. graeca} si
associa anche ad affioramenti rocciosi, rocce emergenti e ghiaioni,
utilizzati come rifugio e/o ricovero notturno;
nell{\textquoteright}arido paesaggio appenninico anche i punti
d{\textquoteright}acqua sembrano rappresentare un fattore favorevole
alla presenza della specie. Lo studio conferma la predilezione per i
versanti a solat\`io, mentre la fascia altitudinale prescelta \`e
quella posta fra i 1601 e i 1900 m.

Il rilevamento di campo tramite MN e MC ha consentito di raccogliere 33
dati di presenza e attivit\`a antropica, classificabili entro 6
categorie di disturbo (Tab. \ref{Alemanno_tab_2}) risultando interessate 11 delle 56 USR
dell{\textquoteright}ADS. Nel SS \`e accentuata la presenza di cani da
pastore sia liberamente vaganti che associati al pascolo in atto, i
quali possono ricadere nella condizione di randagi a causa della
mancanza di controllo umano. Tale presenza ha complicato
l{\textquoteright}attuazione del MC. Nel SM sono stati spesso osservati
escursionisti, \textit{mountain biker} e cercatori di funghi fuori dai
sentieri e saltuariamente autoveicoli fuori dal tracciato stradale.

La H\textsubscript{0}: {\textquotedblleft}la presenza e
l{\textquoteright}abbondanza dei fattori di disturbo antropico non
influenzano la presenza e l{\textquoteright}abbondanza della coturnice
appenninica nell{\textquoteright}area di studio{\textquotedblright},
saggiata tramite i rispettivi indici di abbondanza, \`e stata
rigettata, ottenendo un livello altamente significativo di relazione
inversa tra le due variabili (r\textsubscript{S }=-0,745
p{\textless}0,01).


%\rowcolors{2}{MUSEBLUE!40!white}{white}
\begin{table}[!h]
\centering
\small
\begin{tabular}{>{\raggedright\arraybackslash}p{.5\columnwidth}>{\raggedleft\arraybackslash}p{.13\columnwidth}>{\raggedleft\arraybackslash}p{.13\columnwidth}>{\raggedleft\arraybackslash}p{.13\columnwidth}}
\toprule
\textbf{Categoria disturbo antropico} & \textbf{Punteggio numerico ordinale} & \textbf{Numero di eventi riscontrati} & \textbf{Peso complessivo dato dalla categoria} \\
\toprule
%\showrowcolors
Cani da guardia-difesa liberamente vaganti & 6 & 3 & 18 \\
Cani da guardia-difesa associati a pascolo in atto & 5 & 7 & 35 \\
Escursionisti, deltaplanisti, cercatori di funghi & 4 & 7 & 28 \\ 
Veicoli a motore fuori dal tracciato stradale & 3 & 1 & 3 \\
Pascolo ovino in atto & 2 & 7 & 14 \\
Strada provinciale con traffico intenso & 1 & 8 & 8 \\
\bottomrule
\hiderowcolors
\end{tabular}
\caption{}
\label{Alemanno_tab_2}
\end{table}

\section*{Conclusioni}

Nella porzione pi\`u marginale, sia dal punto di vista ecologico che
geografico, dei Monti Sibillini \textit{A. graeca} costituisce, con
certezza, parte integrante e autoctona della zoocenosi appenninica nel
quarantennio 1970-2010. L{\textquoteright}area studiata sembra
rappresentare ancora oggi una \textit{source} per il vasto \textit{sink
}circostante (Ritchie 1997) il quale, oltre a soffrire le medesime
criticit\`a riscontrate nell{\textquoteright}ADS, \`e paradigma della
non corretta gestione faunistico-venatoria, con
l{\textquoteright}aggravante del bracconaggio.

La positiva realt\`a riscontrata nell{\textquoteright}ADS andrebbe
sostenuta con analisi delle dinamiche fitosociologiche, che valutino
quanto e con quale velocit\`a esse influenzano la specie, con un
monitoraggio specie-specifico a lungo termine, con \textit{best
practice} nell{\textquoteright}uso del territorio e degli ecosistemi,
da rivolgere alle attivit\`a agro-silvo-pastorali e a quelle
ludico-ricreative.

\section*{Ringraziamenti}

Gli Autori sono grati all{\textquoteright}Ente Parco Nazionale dei Monti
Sibillini, per aver consentito la ricerca e la fruizione del materiale
cartografico. Esprimono viva riconoscenza, per i contributi offerti, a:
Loreto Capotondo, Gianfranco Cavarischia, Vincenzo Di Felice, Giuseppe
Nobili, Fulvio Rosa, Massimiliano Cervosi, Rodolfo e Stefano Alemanno;
Guido Bellini, Nando Masciotti (gi\`a Provincia di Perugia); Giovanni
Bucciarelli, Manuele Cacciatori, Roberto Perucci (CTA-CFS); Luca
Convito (Provincia di Perugia), Nicola Felicetti (Studio LEA),
Francesco Renzini (CFS), Silvio Span\`o (UniGE).

\section*{Bibliografia}
\begin{itemize}\itemsep0pt
	\item Caro T., Engilis A. Jr, Fitzherbert E. \& Gardner T., 2004 - Preliminary
assessment of the flagship species concept at a small scale.
\textit{Animal Conservation}, 7: 63-70.

	\item Brichetti P. \& Fracasso G., 2004 - \textit{Ornitologia Italiana}.
\textit{Tetraonidae-Scolapacidae}. Casa editrice Alberto Perdisa
Editore, Bologna: 412 pp.

	\item Fowler J. \& Cohen L., 2002 - \textit{Statistica per Ornitologi e
Naturalisti}. British Trust for Ornithology, Franco Muzzio Editore,
Roma: 240 pp. 

	\item Jacobs J., 1974 - Quantitative measurement of food selection.
\textit{Oecologia}, 14: 413-417.

	\item Magrini M. \& Gambaro C., 1997 - \textit{Atlante Ornitologico
dell{\textquoteright}Umbria}. Regione dell{\textquoteright}Umbria,
Petruzzi Editore, Citt\`a di Castello: 239 pp.

	\item Petretti F., 1985 - La coturnice negli Appennini. \textit{Serie Atti e
Studi},  WWF Italia, 4: 2-24.

	\item Renzini F. \& Ragni B., 1998 - La coturnice nel Parco Nazionale dei
Monti Sibillini. \textit{L{\textquoteright}uomo e
l{\textquoteright}ambiente}. Universit\`a degli Studi di Camerino, 29:
1-40.

	\item Ritchie M. E., 1997 - Populations in a Landscape Context: Sources, Sinks
and Metapopulations. In: Bissonette J. A. (ed.), \textit{Wildlife and
Landscape Ecology. Effect of Patterns and Scale}. Springer-Verlag, New
York: 160-184.

	\item Velatta F., Lombardi G., Sergiacomi U. \& Viali P., 2010 - Monitoraggio
dell{\textquoteright}Avifauna Umbra. \textit{I Quaderni
dell{\textquoteright}Osservatorio} (NS), Regione Umbria: 1-390. 

	\item Zunino M. \& Zullini A., 1995 - Biogeografia, la dimensione
spaziale dell{\textquoteright}evoluzione. Casa Editrice Ambrosiana,
Milano: 310 pp.
\end{itemize}

\setcounter{figure}{0}
\setcounter{table}{0}

\begin{adjustwidth}{-3.5cm}{0cm}
\pagestyle{CIOpage}
\authortoc{Aluigi A., Fasano G. S., Baghino L., Campora M., Cottalasso R., Toffoli R.}
\chapter*[Rete Natura 2000 in Liguria]{Importanza conservazionistica della Rete Natura 2000 in
Liguria}
\addcontentsline{toc}{chapter}{Rete Natura 2000 in Liguria}

\textsc{Antonio Aluigi}$^{1}$, \textsc{Sergio G. Fasano}$^{1*}$,
\textsc{Luca Baghino}$^{1}$, \textsc{Massimo Campora}$^{1}$,
\textsc{Renato Cottalasso}$^{1}$, \\\textsc{Roberto Toffoli}$^{1}$ \\

\index{Aluigi Antonio} \index{Fasano G. Sergio} \index{Cottalasso Renato} \index{Toffoli Roberto}
\noindent\color{MUSEBLUE}\rule{27cm}{2pt}
\vspace{1cm}
\end{adjustwidth}



\marginnote{\raggedright $^1$Ente Parco del Beigua - Via Marconi 165, 16011
Arenzano GE - biodiv@parcobeigua.it \\
\vspace{.5cm}
{\emph{\small $^*$Autore per la corrispondenza: \href{mailto:fasanosg@gmail.com}{fa\allowbreak sa\allowbreak no\allowbreak sg@\allowbreak g\allowbreak ma\allowbreak il.\allowbreak com}}} \\
\keywords{Liguria, Parco del Beigua, monitoraggio comunit\`a}
{Liguria Region, Beigua Natural Park, monitoring, birds}
%\index{keywords}{Liguria} \index{keywords}{Parco del Beigua} \index{keywords}{Monitoraggio comunit\`a}
}
\noindent \textsc{\color{MUSEBLUE} Summary} / During 2006-2012,the Liguria Region developed a project to monitor
birds as part of an integrated system of surveys~(breeding bird
communitites in all sites, and a focus on target species in sample
areas).~ The monitoring took place in several Natura 2000 sites~as well
as~other areas.

We compared breeding birds communities detected in 5274 point counts,
confirming the conservation importance, both qualitative and
quantitative, of the sites belonging to the Natura 2000 network
compared with other areas. In addition, the likely trends of the
species exhibit significant differences within and outside Natura 2000
sites, with trends generally positive (or less negative) within the
Natura 2000 areas.\\
\noindent \textsc{\color{MUSEBLUE} Riassunto} / Tra il 2008 ed il 2012 nella Regione Liguria \`e tato condotto, mediante
l{\textquoteright}attuazione di un sistema integrato di censimenti, un
progetto di monitoraggio dell{\textquotesingle}avifauna nei siti della
Rete Natura 2000 e in altre aree di elevato interesse.

Il confronto delle comunit\`a ornitiche nidificanti rilevate in 5274
punti di ascolto conferma l{\textquotesingle}importanza
conservazionistica, sia qualitativa che quantitativa, delle aree
appartenenti alla Rete Natura 2000 rispetto alle altre aree. Inoltre, i
probabili andamenti delle specie evidenziano differenze significative
all{\textquotesingle}interno ed all{\textquotesingle}esterno dei siti
della Rete Natura 2000, con tendenze generalmente positive (o meno
negative) all{\textquotesingle}interno dei siti Natura 2000. 



\section*{Introduzione}

Tra il 2008 ed il 2012 \`e stato condotto un progetto di
{\textquotedblleft}monitoraggio della comunit\`a ornitica nelle ZPS e
nelle aree liguri a maggiore vocazionalit\`a avifaunistica ed
agricola{\textquotedblright} (Nicosia \textit{et al. }2009a, 2009b;
Fasano \textit{et al}. 2012, 2013). Il progetto,  promosso e finanziato
dalla Regione Liguria, e attuato dal Parco Naturale Regionale del
Beigua, rientra in un pi\`u vasto monitoraggio delle specie di
interesse conservazionistico, avviato dalla Regione nel 2007 in
adempimento alle Direttive 92/43/CEE
({\textquotedblleft}Habitat{\textquotedblright}) e 2009/147/CE
({\textquotedblleft}Uccelli{\textquotedblright}). Con il presente
contributo si \`e proceduto a testare eventuali differenze qualitative
e quantitative tra le aree afferenti alla Rete Natura 2000 e il resto
del territorio regionale.

\section*{Metodi}
Il progetto \`e basato su uno sforzo di campionamento maggiormente
approfondito in alcune aree della Rete Natura 2000 (sette ZPS e quattro
SIC), dove \`e incentrato sul monitoraggio di specie \textit{target}, e
sul censimento annuale della comunit\`a ornitica nidificante, sia nei
Siti Natura 2000 precedentemente individuati che in altri aggiuntivi e
in un numero variabile di  particelle UTM  (10 x 10 km di lato) in
parte ripetute e in parte scelte di anno in anno in modo da indagare
l{\textquoteright}intero territorio regionale nel corso dei cinque anni
previsti.

Per la caratterizzazione e il monitoraggio
dell{\textquoteright}ornitocenosi nidificante la tecnica di rilevamento
prescelta \`e stata quella dei punti di ascolto senza limiti di
distanza (Blondel \textit{et al}. 1981; Nicosia \textit{et al.} 2009b).
Considerando anche dati pregressi, per il periodo 2000-2012 \`e
disponibile un campione di 5274 stazioni di rilevamento (distribuite in
88 particelle UTM sul totale delle 90 afferenti alla Regione Liguria).
Per un pi\`u dettagliato confronto tra i siti liguri della Rete Natura
2000 rispetto alle altre aree, si \`e poi utilizzato un campione di
4142 punti d{\textquoteright}ascolto relativo agli anni 2008-2012; di
questi 2118 ricadono all{\textquoteright}interno di Siti Natura 2000
(7 ZPS e 49 SIC) e interessano 24 Aree Protette.

La valutazione degli andamenti delle specie comuni (anni 2000-2012;
Fasano \textit{et al}. 2012) \`e stata effettuata utilizzando il
software TRIM (TRends \& Indices for Monitoring Data), come indicato da
Gregory \textit{et al}. (2005) adottando una metodologia analoga a
quella applicata a livello nazionale per
l{\textquoteright}identificazione delle specie legate agli
agroecosistemi e ambienti boschivi (Fornasari \textit{et al.} 2002;
Tellini \textit{et al.} 2005; LIPU 2011); si \`e inoltre definita la
vocazione ambientale delle principali specie nidificanti. Di ciascun
gruppo, calcolando la media geometrica degli indici di popolazione
delle specie ad esso appartenenti (Gregory \textit{et al}. 2005), \`e
stato poi elaborato un indicatore\textbf{ }di stato di conservazione
complessivo.

Sono state quindi svolte ulteriori analisi di tendenza demografica,
verificando eventuali differenze esistenti all{\textquotesingle}interno
e all{\textquotesingle}esterno dei Siti della Rete Natura 2000. Per
fare ci\`o una particella \`e stata considerata afferente alla Rete
Natura 2000 quando almeno il 50\% dei suoi punti risultavano ricompresi
all{\textquotesingle}interno di uno o pi\`u Siti della rete;
successivamente l{\textquotesingle}appartenenza della particella alla
Rete Natura 2000 \`e stata utilizzata direttamente come covariata
categoriale nell{\textquotesingle}analisi di TRIM. Nel complesso 2603
punti d{\textquoteright}ascolto (il 49,4\% del campione relativo al
periodo 2000-2012) ricadono in Siti della rete e otto particelle, sulle
21 selezionate per il calcolo degli andamenti, rispondono ai criteri
sopra esposti.

\section*{Risultati e discussione}

Analizzando i valori medi per punto d{\textquoteright}ascolto di alcuni
parametri calcolati per le aree liguri della Rete Natura 2000
({\textquotedblleft}N2000{\textquotedblright}) rispetto alle altre aree
({\textquotedblleft}aa{\textquotedblright}), possiamo riscontrare come
le prime siano in maniera statisticamente significativa pi\`u
importanti per ci\`o che riguarda la diversit\`a, espressa attraverso
l{\textquoteright}indice di Shannon-Weaver (N2000: 1,86 {\textpm} 0,36;
aa: 1,80 {\textpm} 0,36;  t = 5,10, P {\textless} 0,001; MacArthur
1965), il valore nazionale corretto dall{\textquoteright}abbondanza
specifica (N2000: 33,5 {\textpm} 3,21; aa: 31,4 {\textpm} 3,21;  t =
21,16, P {\textless} 0,001; Brichetti \& Gariboldi 1992) e il numero di
specie incluse nell{\textquoteright}All. 1 della Direttiva
{\textquoteleft}Uccelli{\textquoteright} (N2000: 0,24 {\textpm} 0,494;
aa: 0,09 {\textpm} 0,326;  t = 11,35, P {\textless} 0,001), mentre non
si evidenziano differenze per quanto concerne la ricchezza specifica
(N2000: 7,75 {\textpm} 2,619; aa: 7,69 {\textpm} 2,523;  t = 0,83, P =
0,408). 

Considerando i parametri pi\`u strettamente legati al valore
conservazionistico dell{\textquoteright}ornitocenosi presente,
osserviamo poi come esistano differenze significative in relazione al
tipo di protezione cui \`e assoggettata l{\textquoteright}area nella
quale ricade il punto, e cio\`e se questo \`e al di fuori di aree
protette, in Aree Protette ({\textquoteleft}AP{\textquoteright}), in
Siti della Rete Natura 2000 oppure in territori che appartengano
contemporaneamente sia ad Aree Protette che a Siti della Rete Natura
2000 ({\textquoteleft}APeN2000{\textquoteright}). Il valore nazionale
corretto dall{\textquoteright}abbondanza specifica presenta
significative differenze fra le quattro categorie (APeN2000: 33,9
{\textpm} 3,28; N2000: 32,9 {\textpm} 3,01; AP: 32,2 {\textpm} 3,29;
aa: 31,3 {\textpm} 3,20; F\textsubscript{3,4135} = 172,55, P
{\textless} 0,001; test di Tukey, P {\textless} 0,05); il numero di
specie incluse nell{\textquoteright}All. 1 della Direttiva
{\textquotedblleft}Uccelli{\textquotedblright}, presenta
anch{\textquoteright}esso significative differenze fra le categorie
(APeN2000: 0,24 {\textpm} 0,487; N2000: 0,23 {\textpm} 0,506; AP: 0,08
{\textpm} 0,256; aa: 0,09 {\textpm} 0,331; F\textsubscript{3,4135} =
42,48, P {\textless} 0,001), che si raggruppano per\`o in due
sottoinsiemi omogenei (1: APeN2000 e N2000, 2: AP e aa; test di Tukey,
P {\textless} 0,05).  

Per quanto riguarda le tendenze demografiche (Fasano \textit{et al}.
2012), nel periodo 2000-2012 si \`e riscontrata una relativa
stabilit\`a delle popolazioni, ma con molte specie in diminuzione.
Sulle 54 specie considerate, il 39\% risultano tendenti
all{\textquoteright}aumento o stabili, il 37\% tendono alla
diminuzione, mentre per il 24\% la tendenza non \`e ancora
statisticamente definita. La tendenza alla diminuzione \`e pi\`u
marcata per le specie che nidificano in ambienti agrari e antropizzati
e per quelle che svernano nell{\textquoteright}Africa sub-sahariana.
Invece le specie migratrici intra-paleartiche e quelle legate alle
praterie presentano andamenti che tendono alla stabilit\`a, e le specie
che preferiscono ambienti boscati mostrano incremento moderato. 

L{\textquoteright}andamento dell{\textquoteright}indicatore complessivo,
calcolato per tutte le 54 specie, \`e simile e coerente sia
all{\textquoteright}interno che all{\textquoteright}esterno dei Siti
Natura 2000, ma con una tendenza meno negativa per questi ultimi.
L{\textquoteright}andamento \`e invece nettamente differenziato per le
specie forestali e di prateria, con tendenze positive entro i Siti
Natura 2000 e negative all{\textquoteright}esterno; mentre per
l{\textquoteright}indice relativo alle specie degli agroecosistemi la
tendenza, pur essendo simile, si inverte; ma ci\`o \`e probabilmente
riconducibile, nei settori  interessati dal campione analizzato, alla
scarsa disponibilit\`a di questi ambienti, dovuta
all{\textquoteright}abbandono delle attivit\`a agricole nelle aree
svantaggiate come quelle appenniniche.

Riscontriamo poi differenze significative all{\textquoteright}interno ed
all{\textquoteright}esterno dei Siti della Rete Natura 2000 per 13
specie (P {\textless} 0,05), delle quali quattro incluse
nell{\textquoteright}All. 1 della Direttiva
{\textquotedblleft}Uccelli{\textquotedblright}. Di queste ultime, in
tre casi osserviamo come l{\textquoteright}andamento degli indici
diverga, con tendenze positive (biancone \textit{Circaetus gallicus} e
magnanina comune \textit{Sylvia undata}) o di stabilit\`a (averla
piccola \textit{Lanius collurio}) all{\textquoteright}interno dei Siti
Natura 2000, e negative all{\textquoteright}esterno. La tottavilla
\textit{Lullula arborea}, che complessivamente presenta una diminuzione
moderata, mostra andamento simile nei due ambiti, ma con una tendenza
decisamente meno negativa all{\textquoteright}interno dei Siti Natura
2000.

Questi risultati, che confermano ulteriormente
l{\textquoteright}importanza conservazionistica della Rete Natura 2000,
sono probabilmente riconducibili non solo alle eventuali modalit\`a di
gestione attiva dei siti, ma anche al fatto che queste zone risultano,
con poche eccezioni, meno interessate (o meglio mitigate) da quei
processi che, come l{\textquotesingle}aumento incontrollato delle
superfici edificate, ha determinato drammatici cambiamenti nel
paesaggio e che, come evidenziato da Rete Rurale Nazionale \& LIPU
(2012, 2013), in certe condizioni \`e ad oggi probabilmente una delle
cause principali, se non la pi\`u importante, del declino degli uccelli
degli ambienti antropizzati e agrari.

\section*{Bibliografia}
\begin{itemize}\itemsep0pt
	\item Blondel J., Ferry C. \& Frochot B., 1981 - Point Counts with Unlimited
distance. In: Estimating Numbers of terrestrial birds. \textit{Studies
in Avian Ecologies,} 6: 414-420. 

	\item Brichetti P. \& Gariboldi A., 1992 - Un
{\guillemotleft}valore{\guillemotright} per le specie ornitiche
nidificanti in Italia. \textit{Riv.ital.Orn.,} 62: 73-87. 

	\item Fasano S. \& Aluigi A., 2007 - Dati preliminari sulla densit\`a
riproduttiva di Calandro \textit{Anthus campestris }e Magnanina comune
\textit{Sylvia undata}\textit{ }nel Parco del Beigua e nella ZPS
{\textquotedblleft}Beigua-Turchino{\textquotedblright} (GE-SV).
Abstract del XIV Convegno Italiano di Ornitologia. Trieste 26-30
settembre 2007: 47. 

	\item Fasano S., Baghino L. \& Aluigi A., 2009 - La
{\textquotedblleft}Canellona{\textquotedblright}: un \textit{hot-spot}
per l{\textquoteright}Averla piccola. (SIC IT1331402). Atti del XV
Convegno Italiano di Ornitologia. Parco Nazionale del Circeo, Sabaudia
(Latina) 14-18 ottobre 2009. \textit{Alula,} XVI (1-2): 544-546.

	\item Fasano S. G., Aluigi A., Baghino L., Campora M., Cottalasso R. \&
Toffoli R., 2012 - Monitoraggio della comunit\`a ornitica nelle
ZPS e nelle aree liguri di maggiore vocazionalit\`a avifaunistica e/o
agricola. Anno 2012. Regione Liguria -- Parco del Beigua, 235 pp.

	\item Fasano S.G., Cottalasso R., Campora M., Baghino L., Toffoli R. \& Aluigi
A. (a cura di), 2013 - Ambienti e Specie del Parco del Beigua e
dei Siti della Rete Natura 2000 funzionalmente connessi. Ente Parco
del Beigua, 100 pp. 

	\item Fornasari L., De Carli E., Brambilla S., Buvoli L., Maritan E. \&
Mingozzi T., 2002 - Distribuzione dell{\textquoteright}avifauna
nidificante in Italia: primo bollettino del progetto di
monitoraggio MITO 2000. \textit{Avocetta,} 26 (2): 59-115.

	\item Gregory R.D., van Strien A., Vorisek P., Gmelig Meyling A.W., Noble D.,
Foppen R. \& Gibbons D.W., 2005 - Developing indicators for European
birds. \textit{Phil. Trans. R. Soc. B.}, 360: 269-288.

	\item Nicosia E., Aluigi A., Fasano S. \& Toffoli R., 2009 - La Rete Natura
2000 in Liguria: caratterizzazione e con\allowbreak fronto di alcune realt\`a.
Atti del XV Convegno Italiano di Ornitologia. Parco Nazionale del
Circeo, Sabaudia (Latina) 14-18 ottobre 2009. \textit{Alula,} XVI
(1-2): 558-560.

	\item Nicosia E.,  Aluigi A., Fasano S., Baghino L., Campora M., Cottalasso
R., Toffoli R. \& Ballerini M., 2009b - Il monitoraggio della Rete
Natura 2000 in Liguria. Atti del XV Convegno Italiano di Ornitologia.
Parco Nazionale del Circeo, Sabaudia (Latina) 14-18 ottobre 2009.
\textit{Alula} XVI (1-2): 519-524. 

	\item MacArthur R.H., 1965 - Patterns of species diversity. \textit{Biol.
}\textit{Rev., }40:510-533. 

	\item LIPU, 2011 - Censimento dell{\textquoteright}avifauna per la
definizione del Farmland Bird Index a livello na\allowbreak zionale e
regionale in Italia. \textit{Farmland Bird Index e Woodland Bird Index
-- 2000-2010}. Rete Rurale Nazionale, 2007-2013. 

	\item Rete Rurale Nazionale \& LIPU, 2012 - Censimento
dell{\textquoteright}avifauna per la defnizione del Farmland Bird
Index a livello na\allowbreak zionale e regionale in Italia. \textit{Farmland
Bird Index e Woodland Bird Index -- 2000-2011}. Rete Rurale Nazionale,
2007-2013.

	\item Rete Rurale Nazionale \& LIPU, 2013 - \textit{Uccelli comuni in Italia.
Aggiornamento degli andamenti di popolazione al 2012}. Rete Rurale
Nazionale \& LIPU.

	\item Tellini Florenzano G., Buvoli L., Caliendo M.F., Rizzolli F. \&
Fornasari L., 2005 - Definizione dell{\textquoteright}ecologia
degli uccelli italiani mediante indici nazionali di selezione
d{\textquoteright}habitat. \textit{Avocetta}, 29 (n.s.): 148.
\end{itemize}

\setcounter{figure}{0}
\setcounter{table}{0}

\begin{adjustwidth}{-3.5cm}{0cm}
\pagestyle{CIOpage}
\authortoc{\textsc{Balestrieri R.}, \textsc{Posillico M.}, 
\textsc{Basile M.}, \textsc{Altea T.}, \textsc{Matteucci G.}}
\chapter*[Punti d{\textquoteright}ascolto: intensit\`a di
campionamento e contattabilit\`a]{
\textbf{Applicazione della tecnica dei punti d{\textquoteright}ascolto
in ambiente forestale: influenza dell{\textquoteright}intensit\`a di
campionamento e della contattabilit\`a delle singole specie sulla
caratterizzazione della comunit\`a ornitica}}
\addcontentsline{toc}{chapter}{Punti d{\textquoteright}ascolto: intensit\`a di
campionamento e contattabilit\`a}


\textsc{Rosario Balestrieri}$^{1*}$, \textsc{Mario Posillico}$^{1,2}$, 
\textsc{Marco Basile}$^{1}$, \textsc{Tiziana Altea}$^{2}$, 
 \textsc{Giorgio Matteucci}$^{1}$\\
 
 \index{Balestrieri Rosario} \index{Posillico Mario} \index{Basile Marco} \index{Altea Tiziana} \index{Matteucci Giorgio} 
\noindent\color{MUSEBLUE}\rule{27cm}{2pt}
\vspace{1cm}
\end{adjustwidth}



\marginnote{\raggedright $^1$Istituto di Biologia Agroambientale e
Forestale del CNR Via Salaria km 29.3, 00015 Monterotondo Scalo RM \\
$^2$Corpo Forestale dello Stato, Ufficio
Territoriale Biodiversit\`a di Castel di Sangro, Via Sangro, 45-67031
Castel di Sangro (AQ) \\
\vspace{.5cm}
{\emph{\small $^*$Autore per la corrispondenza: \href{mailto:rosario.balestrieri@ibaf.cnr.it}{ro\allowbreak sa\allowbreak rio.\allowbreak ba\allowbreak les\allowbreak trie\allowbreak ri@\allowbreak i\allowbreak baf.\allowbreak cnr.\allowbreak it}}} \\
\keywords{Alpi, Appennino, gestione forestale, punti
d{\textquoteright}ascolto}
{Alps, Apeninnes, forest management, point counts}
%\index{keywords}{Alpi} \index{keywords}{Appennino} \index{keywords}{Gestione forestale} \index{keywords}{Punti
%d{\textquoteright}ascolto}
}
{\small
\noindent \textsc{\color{MUSEBLUE} Summary} / Investigating the relationship between birds and forests is a key
argument for sustainable forestry and biodiversity conservation, as
birds are often the most numerous group of forest vertebrates, both as
number of species and individuals. We carried out our study within the
project LIFE ManFor CBD (www.manfor.eu), which aims to test and verify
in the field the effectiveness of forest management options in meeting
multiple objectives (timber production, environment protection and
biodiversity conservation, etc.), providing data, guidance and
indications of best-practice. Therefore, we search for an optimal
sampling scheme that account for the detectability of every species.
Data were collected in 5 forest stands in 5 areas from northern to
southern Italy, each one extended for about 30 ha. We randomly selected
19 -- 24 points in each area, surveyed four-five times each,
investigated through aural-visual survey. Detectability for every
species resulted in very different values, ranging from siskin (0.05),
to robin (1). Our research suggests that the majority of the species
among the forest bird community could be detected in the first and the
second survey.   \\
\noindent \textsc{\color{MUSEBLUE} Riassunto} / La relazione tra avifauna e habitat forestale \`e un argomento chiave
per la selvicoltura sostenibile e la conservazione della
biodiversit\`a, essendo gli Uccelli spesso il \textit{taxon} pi\`u
abbondante in tale habitat, sia per numero di specie che per numero di
individui. Il presente studio \`e stato portato avanti
nell{\textquoteright}ambito del progetto LIFE+ ManFor C.BD
(www.manfor.eu), che ha l{\textquoteright}obiettivo di verificare
l{\textquoteright}efficacia di diverse opzioni di gestione forestale
nel ottenere obiettivi multipli, tra cui la produzione di legname e la
conservazione della biodiversit\`a. Pertanto, \`e stato testato uno
schema di campionamento dell{\textquoteright}avifauna forestale, che
tenga conto della probabilit\`a di contatto di ogni specie. I dati sono
stati raccolti in 5 foreste di altrettante aree italiane, ognuna ampia
circa 30 ha. Sono stati selezionati casualmente 19 -- 24 punti in ogni
area, campionati 4 -- 5 volte tramite rilievi visivi e al canto. La
probabilit\`a di contatto di ogni specie \`e risultata molto varia, da
0.05 per il  lucherino a 1 per il pettirosso. Tale ricerca suggerisce
che la maggior parte delle specie incluse nella comunit\`a ornitica
forestale pu\`o essere rilevata nei primi due eventi di campionamento. \\
}
\section*{Introduzione}
{{La valutazione delle relazioni ecologiche
tra comunit\`a ornitica e ambiente forestale \`e di rilevante
importanza per una gestione forestale sostenibile, attenta anche alla
conservazione della biodiversit\`a essendo, spesso, gli Uccelli il
}\textit{{taxon}}{
di vertebrati forestali pi\`u numeroso sia in termini di individui che
in termini di specie (DeGraaf }\textit{{et
al. }}{1996). L{\textquoteright}alterazione
e la variazione dei parametri degli ambienti forestali (es. et\`a media
degli alberi o copertura fogliare) dovuta alle pratiche selvicolturali,
pu\`o, infatti, ripercuotersi sulla comunit\`a ornitica (cfr. ad es.
Caprio }\textit{{et
al.}}{ 2008, Gil-Tena
}\textit{{et
al.}}{ 2008, White
}\textit{{et
al.}}{ 2013) e, d{\textquoteright}altro
canto, la composizione della comunit\`a ornitica pu\`o rappresentare un
valido strumento nel predire i potenziali impatti negativi per tutta la
comunit\`a. Utile per l{\textquoteright}analisi della comunit\`a
ornitica forestale \`e la realizzazione di una
}\textit{{check-list}}{
delle specie presenti in un{\textquoteright}area. Le tecniche che
permettono di raggiungere tale obiettivo con risultati robusti (e.g.
inanellamento, rilievi a vista), in ambienti forestali, risultano
difficilmente applicabili sia per i costi che per le caratteristiche
fisiche degli habitat forestali. La tecnica dei punti
d{\textquoteright}ascolto \`e comunemente usata negli ambienti boscati
e rappresenta un buon compromesso tra adeguata copertura territoriale e
superamento delle problematiche menzionate (Fuller
}\textit{{et
al.}}{ 1984, Toms 2006). Questo metodo non
\`e tuttavia scevro da problemi (legati ad es. alla contattabilit\`a) e
la sua applicazione nel progetto LIFE+ ManFor CBD (\url{www.manfor.eu}) ne
rende necessaria una disamina per valutare le eventuali fonti di
}\textit{{bias}}. 
Lo scopo di questo lavoro \`e di valutare l{\textquoteright}applicazione del metodo dei punti
d{\textquoteright}ascolto valutando la capacit\`a di rilevare le varie
specie, intesa come probabilit\`a di contatto per ogni singola
specie.}

\section*{Metodi}

{
{La ricerca \`e stata condotta in 5
particelle forestali, ognuna delle quali estesa per circa 30 ha: una
pecceta sulle Alpi Giulie (\textasciitilde30 ha); una foresta mista di peccio
(}\textit{{Picea
abies}}{) e abete bianco
(}\textit{{Abies
alba}}{) sulle Dolomiti del Cadore (\textasciitilde25
ha); e 3 boschi di faggio
(}\textit{{Fagus sylvatica}}{)
sulle Prealpi Venete (\textasciitilde 33 ha), sull{\textquoteright}Appennino
centrale (\textasciitilde 30 ha) e sull{\textquoteright}Appennino meridionale
(\textasciitilde 30 ha). Ogni area era circondata da un{\textquoteright}ulteriore
area controllo (buffer), anch{\textquoteright}essa di circa 30 ha. In
ogni particella e relativo buffer sono stati individuati 19-24 punti
casuali, la cui distanza media \`e 155,8 m {\textpm} 15,2 DS), per un
totale di 111 punti. In ogni punto il rilievo \`e stato ripetuto da 3 a
5 volte, per un totale di 526 rilievi. Il singolo rilievo \`e
consistito nel riconoscimento di tutti gli individui visti o sentiti,
durante 5 minuti, dall{\textquoteright}alba alle ore 11. Per
massimizzare lo sforzo di campo e ottenere dati confrontabili, i
rilievi sono stati svolti entro un breve arco temporale (Maggio --
Giugno 2012), coincidente con il periodo di pi\`u frequenti
vocalizzazioni. L{\textquoteright}efficacia del campionamento, intesa
come specie rilevate rispetto al numero di specie attese, \`e stata
valutata con lo stimatore non parametrico di ricchezza specifica
Chao2-bc (Chao 2005). Le analisi sono state effettuate solo sulle
specie per le quali le informazioni di bibliografia indicassero
}\textit{{home range
}}{adeguati alla scala spaziale del
campionamento, per limitare problemi di autocorrelazione spaziale
(Brichetti e Fracasso 2008, 2010, 2011, 2013). Il modello utilizzato
per calcolare la probabilit\`a di contatto delle singole specie assume
che non ci siano differenze tra le aree, ed \`e implementato nel
software Presence (vers. 5.8).}~}

\section*{Risultati e discussione}

{
{L{\textquoteright}efficacia del
campionamento \`e risultata alta per tutte le aree. In totale sono
state rilevate 23 specie (max = 19; min = 13). Una probabilit\`a di
contatto inferiore a 0.5 \`e stata verificata per il 17\% delle specie,
mentre una probabilit\`a superiore a 0.8 \`e stata registrata per il
44\% delle specie. Cincia mora
}\textit{{Periparus ater}}{, fringuello
}\textit{{Fringilla coelebs}}{ e pettirosso
}\textit{{Erithacus rubecola}}{ sono risultate le specie pi\`u
rilevabili (p= 1), mentre lucherino
}\textit{{Carduelis spinus}}{, codibugnolo
}\textit{{Aegithalos caudatus}}{ e fiorrancino
}\textit{{Regulus ignicapilla}}{ quelle meno rilevabili (p =
0.05, 0.29, 0.31, rispettivamente).}}

{
{Dai risultati si evince in generale una
buona probabilit\`a di contatto delle specie (Tab. \ref{Balestrieri_tab_1}), anche per quelle
che iniziano a vocalizzare verso la fine di marzo. Inoltre lo studio
suggerisce che la maggior parte delle specie prettamente forestali
possa essere rilevata facilmente in una o due rilievi. Ci\`o conferma
che la struttura complessiva del disegno di campionamento e il rilievo
svolto attraverso la tecnica dei punti d{\textquoteright}ascolto sono
un metodo efficiente e utile per valutare alcuni parametri di
comunit\`a nei siti di studio. Ulteriori analisi sono in corso per
valutare la capacit\`a di questo approccio nel discriminare differenze
dei parametri di comunit\`a tra aree di dimensione inferiore ai 30 ha
soggette a diversi trattamenti selvicolturali.}}
\newpage
%\rowcolors{2}{MUSEBLUE!60!white}{white}
\begin{table}[!h]
\centering
\small
\begin{tabular}{>{\raggedright\arraybackslash}p{.4\columnwidth}>{\raggedright\arraybackslash}p{.25\columnwidth}>{\raggedright\arraybackslash}p{.25\columnwidth}}
\toprule
\textbf{Specie} & \textbf{Nome commune} & \textbf{Probablit\'a di contatto} \\
\toprule
%\showrowcolors
\textit{Aegithalos caudatos} & Codibugnolo & 0.2917 (0.0928) \\
\textit{Carduelis spinus} & Lucherino & 0.05 (0.0487) \\
\textit{Certhia brachydactyla} & Rampichino comune & 0.8333 (0.0761) \\
\textit{Certhia familiaris} & Rampichino alpestre & 0.3555 (0.1763) \\
\textit{Coccothraustes coccothraustes} & Frosone & 0.7997 (0.1795) \\
\textit{Cyanistes caeruleus} & Cinciarella & 0.875 (0.0675) \\
\textit{Erithacus rubecola} & Pettirosso & 1 \\
\textit{Ficedula albicollis} & Balia dal collare & 0.799 (0.181) \\
\textit{Fringilla coelebs} & Fringuello & 1 \\
\textit{Lophophanes cristatus} & Cincia dal ciuffo & 0.9 (0.0949) \\
\textit{Parus major} & Cinciallegra & 0.9583 (0.0408) \\
\textit{Periparus ater} & Cincia bigia & 1 \\
\textit{Phoenicurus phoenicurus} & Codirosso comune & 0.6972 (0.148) \\
\textit{Phylloscopus collybita} & Lu\'i piccolo & 0.8947 (0.0704) \\
\textit{Phylloscopus sibilatrix} & Lu\'i verde & 0.7358 (0.102) \\
\textit{Poecile palustris} & Cincia mora & 0.6667 (0.0962) \\
\textit{Pyrrhula pyrrhula} & Ciuffolotto & 0.8999 (0.0672) \\
\textit{Regulus ignicapilla} & Fiorrancino & 0.3069 (0.1291) \\
\textit{Regulus regulus} & Regolo & 0.5333 (0.1288) \\
\textit{Sitta europaea} & Picchio muratore & 0.9583 (0.0408) \\
\textit{Sylvia atricapilla} & Capinera & 0.7917 (0.0829) \\
\textit{Sylvia borin} & Beccafico & 0.7997 (0.1795) \\
\textit{Troglodytes troglodytes} & Scricciolo & 0.7917 (0.0829) \\
\bottomrule
\hiderowcolors
\end{tabular}
\caption{Probabilit\`a di contattare ogni specie, stimata secondo il modello implementato nel software PRESENCE vers. 5.8, che assume uguale probabilit\`a tra le aree (Hines 2006). I numeri in parentesi rappresentano gli errori standard}
\label{Balestrieri_tab_1}
\end{table}
\section*{Bibliografia}
\begin{itemize}\itemsep0pt
	\item Brichetti P. \& Fracasso G., 2008 - \textit{Ornitologia Italiana} Vol
5.
	\item Brichetti P. \& Fracasso G., 2010 - \textit{Ornitologia Italiana} Vol
6.
	\item Brichetti P. \& Fracasso G., 2011 - \textit{Ornitologia Italiana }Vol
7.
	\item Brichetti P. \& Fracasso G., 2013 - \textit{Ornitologia Italiana} Vol
8.
	\item Caprio E., Ellena I. \& Rolando A., 2008 - Assessing habitat/landscape
predictors of bird diversity in managed deciduous forests: a seasonal
and guild-based approach. \textit{Biodivers. Conserv., }18: 1287 --
1303.
	\item Chao A., 2005 - \textit{Species estimation and applications}
in~\textit{Encyclopedia of Statistical
Sciences,~2}\textit{\textsuperscript{nd}}\textit{ Edition, Vol. 12,
7907-7916}. Balakrishnan N, Read CB, Vidakovic B, eds. Wiley, New
York.
	\item DeGraaf R. \& Miller R.I., 1996 - \textit{Conservation of faunal
diversity in forested landscapes. }Chapman \& Hall, London.
	\item Fuller R.J. \& Langslow D.R., 1984 - Estimating numbers of birds by
point counts: how long should counts last? \textit{Bird Study,} 31:
195-202
	\item Gil-Tena A., Torras O. \& Saura S., 2008 - Relationships between forest
landscape structure and avian species richness in NE Spain.
\textit{Ardeola, }55 (1): 27 -- 40.
	\item Hines J.E., 2006 - PRESENCE -- Software to estimate patch occupancy and
related parameters USGS-PWRC. \texttt{http://www.mbr-pwrc.usgs.gov/software/presence.html}
	\item Toms J.D.,	 Schmiegelow F.K.A., Hannon S.J. \& Villard M-A., 2006 - Are
point counts of boreal songbirds reliable proxies for more intensive
abundance estimators? \textit{The Auk, }123 (2): 438- 454.
	\item White A.M., Zipkin E.F., Manley P.N. \& Schlesinger M.D., 2013 -
Conservation of avian diversity in the Sierra Nevada: moving beyond a
single-species management focus. \textit{PLoS ONE, }8 (5): e63088.
doi:10.1371/journal.pone.0063088.
\end{itemize}

\setcounter{figure}{0}
\setcounter{table}{0}

\begin{adjustwidth}{-3.5cm}{0cm}
\pagestyle{CIOpage}
\authortoc{\textsc{Bassi E.}, \textsc{Diana F.}, 
\textsc{Sartirana F.}, \textsc{Trotti P.}, 
 \textsc{Galli L.}, \textsc{Pedrotti L.}}
\chapter*[Successo riproduttivo dell'aquila reale in relazione al ritorno del gipeto]{Analisi del successo riproduttivo dell{\textquoteright}aquila
reale \textbf{\textit{Aquila chrysaetos}}\textbf{ nel Parco Nazionale
dello Stelvio in relazione al ritorno del gipeto
}\textbf{\textit{Gypaetus barbatus}}\textbf{ sulle Alpi}}
\addcontentsline{toc}{chapter}{Successo riproduttivo dell'aquila reale in relazione al ritorno del gipeto}

\end{adjustwidth}
\begin{adjustwidth}{-3.5cm}{-1cm}
\textsc{Enrico Bassi}$^{1*}$, \textsc{Francesca Diana}$^{1}$, 
\textsc{Fabiano Sartirana}$^{1}$, \textsc{Paolo Trotti}$^{1}$, 
 \textsc{Loris Galli}$^{2}$, \textsc{Luca Pedrotti}$^{1}$\\
 
 \index{Bassi Enrico} \index{Diana Francesca} \index{Sartirana Fabiano} \index{Trotti Paolo} \index{Galli Loris} \index{Pedrotti Luca}
\noindent\color{MUSEBLUE}\rule{27cm}{2pt}
\vspace{1cm}
\end{adjustwidth}



\marginnote{\raggedright $^1$Consorzio del Parco Nazionale dello Stelvio Via De
Simoni 42, 23032 Bormio (SO) \\
$^2$Universit\`a degli Studi di Genova Corso Europa 26,
16132 Genova \\
\vspace{.5cm}
{\emph{\small $^*$Autore per la corrispondenza: \href{mailto:enrico.bassi76@gmail.com}{en\allowbreak ri\allowbreak co.\allowbreak bas\allowbreak si76@\allowbreak g\allowbreak ma\allowbreak il.\allowbreak com}}} \\
\keywords{\textit{Gypaetus barbatus}, \textit{Aquila chrysaetos},
aquila reale, influenza interspecifica, \textit{no fly-zones}, Alpi
centrali}
{\textit{Gypaetus barbatus}, \textit{Aquila
chrysaetos}, interspecific influence, no fly-zones, central
Alps}
%\index{keywords}{\textit{Gypaetus barbatus}} \index{keywords}{\textit{Aquila chrysaetos}}
% \index{keywords}{Influenza interspecifica} \index{keywords}{\textit{No fly-zones}} \index{keywords}{Alpi centrali}
}
{\small
\noindent \textsc{\color{MUSEBLUE} Summary} / After the bearded vulture \textit{Gypaetus barbatus} extinction and the
subsequent reintroduction over the Alps, the first italian reproductive
nucleus settled in the Stelvio National Park (northern Italy). In the
period 1998-2013 it was characterized by high productivity rate (0.77
fledged juveniles/controlled pairs). In this area a density of 15.75
golden eagle \textit{Aquila chrysaetos} pairs/1000
km\textsuperscript{2} has been recorded, showing a productivity of 0.39
fledged juveniles/controlled pairs from 2005 to 2013. Furthermore in
this period many interspecific observations of aggressive interactions
have been recorded; starting from these data a research started to
assess whether the return of the vulture may have affected the
productivity of the golden eagle. In order to test this influence we
studied the breeding populations of golden eagle and bearded vulture
through focal sampling at nests from 2008 to 2011.

Statistical analysis showed that the productivity of the golden eagle
was positively correlated with the distance of the conspecific and
bearded vulture nearest pairs and negatively correlated with the height
of the snow in winter. The number of golden eagle pairs that laid eggs
was significantly higher in sites distant more than 5 km from bearded
vulture nests. Regarding the interactions observed, the golden eagle
was the most aggressive species. The 71.2\% of these interactions were
observed in the most recent territories defended by immature and
subadult golden eagle and especially within the areas called
{\textquoteleft}no fly zones{\textquoteright}, characterized by the
presence of recent and old nests.   \\
\noindent \textsc{\color{MUSEBLUE} Riassunto} / Dopo l{\textquoteright}estinzione del gipeto \textit{Gypaetus barbatus}
e la sua successiva reintroduzione sull{\textquoteright}arco alpino,
nel Parco Nazionale dello Stelvio (Alpi centrali) si \`e insediato il
primo nucleo riproduttivo italiano caratterizzato, nel periodo
1998-2013, da un{\textquoteright}elevata produttivit\`a pari a 0,77
giovani involati/coppie controllate. Tale area ospita anche una
popolazione di aquila reale \textit{Aquila chrysaetos} distribuita con
densit\`a pari a 15,75 coppie/1000 km\textsuperscript{2 }che, nel
periodo 2005-2013, ha mostrato una produttivit\`a di 0,39
giovani/coppie controllate. 
Inoltre in questo periodo sono state osservate tra le due specie
numerose interazioni aggressive; a partire da questi dati \`e stato
avviato uno studio a pi\`u livelli per valutare se il ritorno del
gipeto possa aver influenzato la produttivit\`a
dell{\textquoteright}aquila reale. Per testare questa influenza le
coppie nidificanti di entrambe le specie sono state oggetto dal 2008 al
2011 di studio intensivo tramite \textit{focal sampling} al nido.
Le analisi hanno mostrato che la produttivit\`a
dell{\textquoteright}aquila reale \`e correlata positivamente con la
distanza della coppia conspecifica e di gipeto pi\`u vicine e
negativamente con l{\textquoteright}altezza della neve in inverno.
Inoltre le coppie di aquila reale pi\`u lontane dai siti di gipeto
(5 km) sono quelle che depongono in maniera
significativamente maggiore rispetto ai siti vicini. Per quanto
riguarda le interazioni osservate, \`e l{\textquoteright}aquila reale
che ha mostrato una maggiore aggressivit\`a.  Il 71,2\% di tali
aggressioni \`e stato osservato nei territori di aquila reale di
neoformazione difesi da soggetti di et\`a immatura e subadulta e
soprattutto all{\textquoteright}interno delle aree di rispetto
denominate {\textquoteleft}\textit{no fly zones}{\textquoteright}
caratterizzate dalla presenza dei nidi (storici e attivi).
}



\section*{Introduzione}

Dopo l{\textquoteright}estinzione del gipeto sulle Alpi ai primi del
Novecento, nel 1986 sono stati avviati i primi rilasci
nell{\textquoteright}ambito del progetto internazionale di
reintroduzione per ricostituire metapopolazioni vitali in grado di auto
mantenersi. Dal 1998, nell{\textquoteright}intorno del Parco Nazionale
dello Stelvio (PNS) si \`e insediato il primo nucleo riproduttivo
italiano di 4 coppie (Bassi 2010) in simpatria con una consistente
popolazione di aquila reale (15,75 coppie/1000 km$^2$. Le
coppie di gipeto sono tutte incluse nell{\textquoteright}intorno del
settore lombardo del PNS (Sondrio e Brescia), a eccezione di alcuni
nidi ricadenti in Svizzera in corrispondenza del confine italiano. Nel
periodo 1998-2013 questo nucleo ha mostrato la maggiore produttivit\`a
a livello alpino, con un valore medio pari a 0,77 giovani involati per
coppia controllata (N= 54). Pi\`u modesta \`e stata invece la
produttivit\`a dell{\textquoteright}aquila reale, per il periodo
2005-2013 (pari a 0,39 giovani su 129 nidificazioni controllate). Le
date medie di deposizione e involo sono state rispettivamente il 17
gennaio e il 15 luglio nel gipeto e il 27 marzo e il 25 luglio
nell{\textquoteright}aquila reale.

Nel Parco Nazionale dello Stelvio sono stati individuati 
14 nidi di gipeto (quota media 2225.4 m
s.l.m., range 2028-2440 m; SD 102.5 m) e 124 nidi di aquila reale
(quota media 2035 m s.l.m., range 1317-2496 m; DS 230.5 m).

A partire da considerazioni relative al diverso successo riproduttivo
delle due specie negli ultimi anni, dalle frequenti osservazioni
occasionali di interazioni interspecifiche per lo pi\`u aggressive
(anche letali), dall{\textquoteright}anticipato ciclo riproduttivo del
gipeto rispetto all{\textquoteright}aquila reale e
dall{\textquoteright}evidenza che il 35,7\% dei 14 nidi usati dal
gipeto sono stati usurpati all{\textquoteright}aquila reale, si \`e
intrapreso uno studio a pi\`u livelli per valutare se il progressivo
ritorno del gipeto possa aver influenzato la produttivit\`a
dell{\textquoteright}aquila reale. 

Sono state pertanto formulate le seguenti ipotesi: 1) il gipeto,
dall{\textquoteright}epoca del suo insediamento, ha influenzato
negativamente la produttivit\`a dell{\textquoteright}aquila reale sul
breve periodo; 2) la quantit\`a di tempo speso
dall{\textquoteright}aquila reale nella difesa territoriale (in genere
rivolta verso conspecifici) pu\`o aumentare in maniera significativa
nei siti di compresenza del gipeto, tanto da indurla a
un{\textquoteright}ulteriore riduzione del tempo investito nelle cure
parentali e a un conseguente calo della sua produttivit\`a; 3) assunto
che i soggetti territoriali di entrambe le specie sono sedentari,
caratterizzati da una lunga aspettativa di vita (il monitoraggio
genetico ha infatti evidenziato la presenza di gipeti adulti tra i 10 e
i 24 anni di vita e aquile adulte di oltre 10 anni) e hanno una
conoscenza consolidata circa la localizzazione dei nidi delle coppie
vicine, esistono delle corrispondenti aree di rispetto
({\textquotedblleft}\textit{no-fly zones{\textquotedblright}}, al cui
interno gli individui territoriali non tollerano potenziali azioni e
incursioni quali soste prolungate, sorvoli, ecc.), identificabili coi
settori che includono i nidi, ben note alle coppie territoriali
confinanti, ma non ai soggetti erratici e alle coppie di pi\`u recente
insediamento.

\section*{Metodi}

Per testare la prima ipotesi, il successo riproduttivo
dell{\textquoteright}aquila reale (la variabile dicotomica
{\textquotedblleft}produttivit\`a Aquila 2004-10{\textquotedblright})
registrato per 14 coppie negli anni 2004-2010, coincidente col periodo
di insediamento di due nuove coppie di gipeto e di tre nuove coppie di
aquila, \`e stato analizzato mediante regressione logistica in
relazione a un set di variabili indipendenti: inclusione del territorio
nell{\textquoteright}area protetta, et\`a dei soggetti territoriali,
esperienza delle coppie, distanza dalla coppia di aquila pi\`u vicina,
distanza dalla coppia riproduttiva di aquila pi\`u vicina, altezza
invernale del manto nevoso (\textit{proxy} della mortalit\`a invernale
di ungulati selvatici e conseguente disponibilit\`a di carcasse),
distanza dalla coppia di gipeto pi\`u vicina e indice medio annuale di
densit\`a di aquile \textit{floaters} presenti nel mese di marzo.
Questo indice \`e stato desunto dai risultati dei conteggi simultanei
effettuati su un{\textquoteright}area di ampiezza media di 1055
km\textsuperscript{2}, tra il 2004 e il 2013, per  quantificare il
numero di adulti e \textit{floaters} nell{\textquoteright}arco di una
finestra spazio-temporale definita (Bassi 2014). La significativit\`a
delle variabili indipendenti \`e stata preliminarmente verificata
mediante analisi esplorative univariate. Successivamente \`e stato
applicato un modello di regressione logistica multivariata che
comprendeva, quali variabili indipendenti, quelle risultate
significative a livello univariato. La selezione del modello migliore
\`e stata effettuata tenendo conto delle differenze nei valori di
massima verosimiglianza dei modelli stessi (Hosmer \& Lemeshow 2000).

Per testare la seconda ipotesi invece \`e stato applicato il metodo
delle osservazioni mediante \textit{focal sampling} dei nidi occupati,
in cui sono stati calcolati i tempi dedicati alla cova e alle cure
parentali da parte degli adulti impegnati nella nidificazione
nell{\textquoteright}arco di 289 giornate di osservazione (media
giornaliera 7,8 h) distribuite su 4 anni (2008-11). I dati raccolti
sono stati confrontati tra coppie nidificanti vicine ({\textless} 5 km)
e lontane ({\textgreater} 5 km) da territori stabili di gipeto per
valutare se la vicinanza dell{\textquoteright}avvoltoio determinasse un
effettivo disturbo. Inoltre, nel corso del \textit{focal sampling,}
sono state registrate tutte le attivit\`a svolte al di fuori del nido
ed \`e stata calcolata la frequenza delle interazioni agonistiche
intra/interspecifiche. 

Per testare la terza ipotesi, lo spazio visibile dal punto di
osservazione \`e stato diviso in superfici unitarie dette
\textit{patches} (media 2.7 km\textsuperscript{2}) adattando il metodo
utilizzato da Haller (1996) per lo studio del comportamento spazio
temporale di coppie territoriali di aquila reale in Svizzera. 

Questa suddivisione \`e risultata funzionale
all{\textquoteright}assegnazione spazio temporale delle attivit\`a
delle coppie impegnate in azioni territoriali, di volo e di
sorveglianza al di fuori del nido. 

Ai \textit{patches} che includevano il nido attivo \`e stato assegnato
il codice 1, il codice 3 ai settori includenti altri nidi difesi dalla
coppia territoriale ma non occupati nell{\textquoteright}anno di
indagine e codice 5 a tutti gli altri settori in cui non erano presenti
nidi noti.

Complessivamente sono stati analizzati i dati provenienti dai
\textit{focal sampling} effettuati su 16 eventi di nidificazione di
aquila reale e 11 di gipeto. 

\section*{Risultati e discussione}

Una prima analisi esplorativa univariata indica come la variabile
dipendente {\textquotedblleft}produttivit\`a Aquila
2004-10{\textquotedblright} sia risultata significativamente correlata
a un set di variabili indipendenti quali la distanza dalla coppia di
aquila pi\`u vicina, {\textquotedblleft}Aquila NND{\textquotedblright}
(p{\textless}0,01), la media annuale dell{\textquoteright}altezza
nevosa in inverno, {\textquotedblleft}AN{\textquotedblright}
(p{\textless}0,01), la distanza dalla coppia di gipeto pi\`u vicina,
{\textquotedblleft}Gypaetus NND{\textquotedblright} (p{\textless}0,05),
e l{\textquoteright}indice di presenza invernale dei \textit{floaters}
di aquila reale censiti nel corso dei conteggi contemporanei effettuati
nel mese di marzo per gli anni 2004-13,
{\textquotedblleft}FLOAQ{\textquotedblright} (p{\textless}0,01). Il
modello migliore selezionato in base all{\textquoteright}analisi di
regressione logistica \`e risultato costituito dalle seguenti
variabili: Aquila NND (distanza dalla coppia di aquila pi\`u vicina) +
AN (altezza media del manto nevoso) + Gypaetus NND (distanza dalla
coppia di gipeti pi\`u vicina) + l{\textquoteright}interazione tra le
ultime due (Tab. \ref{Bassi_tab_1}), e classifica correttamente il 73\% dei casi. 

La produttivit\`a dell{\textquoteright}aquila reale nel periodo 2004-10
\`e risultata correlata negativamente con l{\textquoteright}altezza
della neve in inverno. Una maggiore nevosit\`a invernale causa tassi di
mortalit\`a pi\`u elevati negli ungulati in ambito alpino e quindi una
maggiore disponibilit\`a trofica di carcasse nella seconda parte della
stagione invernale stessa. Tuttavia l{\textquoteright}aumentata
disponibilit\`a alimentare pu\`o attrarre i \textit{floaters} che
costringono i soggetti riproduttivi a intensificare
l{\textquoteright}attivit\`a di difesa territoriale nel periodo di
inizio nidificazione (marzo), con esiti potenzialmente negativi sulla
produttivit\`a, poich\'e \`e ipotizzabile che venga modificato il tempo
dedicato alla cova e alle cure parentali (Jenny 1992). 

Il successo riproduttivo \`e invece positivamente correlato
all{\textquoteright}aumento della distanza dalla coppia territoriale
pi\`u vicina sia di aquila reale, sia di gipeto, lasciando presupporre
che elevate densit\`a di entrambe le specie possano determinare un
effetto congiunto di regolazione sulla dinamica della popolazione di
aquila reale, vicina al raggiungimento della capacit\`a portante (Tab.
\ref{Bassi_tab_2}). \`E possibile quindi affermare che negli anni di neo insediamento
di due nuove coppie di gipeto (2004-10), anche la distanza dalla coppia
territoriale di gipeto pi\`u vicina abbia influenzato la produttivit\`a
dell{\textquoteright}aquila reale. Pertanto si \`e ipotizzato che le
coppie di aquila nidificanti presso i territori di gipeto, oltre a
dover competere con i conspecifici, siano pi\`u frequentemente
sollecitate ad azioni di difesa contro individui di gipeto, rispetto ad
altre coppie di aquila poste a maggior distanza dal gipeto stesso, e
dunque siano sottoposte a un maggior grado di stress, a un aumentato
rischio di non deposizione e a una riduzione dei tempi di cova e di
cure parentali, in analogia con quanto dimostrato da Jenny (1992).

L{\textquotesingle}interazione Gypaetus NND * AN migliora in modo
significativo il modello e mette in evidenza che
l{\textquotesingle}effetto del gipeto sulla probabilit\`a di
riproduzione dell{\textquotesingle}aquila reale non \`e costante ma
dipende dalle condizioni dell{\textquotesingle}inverno. In inverni con
nevosit\`a al di sotto della media, la vicinanza del gipeto aumenta in
modo significativo la probabilit\`a di fallimento
dell{\textquotesingle}aquila reale; questo significa che le coppie di
aquila che si riproducono hanno una distanza media pi\`u alta dalla
coppia di gipeto pi\`u vicina. Si ipotizza che in inverni poco nevosi
la diminuzione di carcasse di ungulati sul territorio determini una
maggiore competizione trofica tra le due specie che probabilmente
tendono a interagire per lo sfruttamento delle medesime fonti
alimentari, numericamente pi\`u scarse e localizzate
nell{\textquoteright}area difesa da ciascuna coppia territoriale.

Al contrario, negli inverni con nevosit\`a al di sopra della media,
l{\textquotesingle}effetto
{\textquotedblleft}gipeto{\textquotedblright} non \`e visibile poich\'e
la distanza media dalla coppia di gipeto pi\`u vicina \`e analoga nelle
coppie che si riproducono e in quelle che falliscono. Queste condizioni
di nevosit\`a, garantendo una maggior disponibilit\`a di carcasse
distribuite su spazi pi\`u ampi, potrebbero diminuire sia il grado di
interazioni aggressive tra le due specie sia la loro parziale
competizione trofica. Inoltre, in inverni particolarmente nevosi,
l{\textquoteright}effetto di disturbo da parte di aquile
\textit{floaters }\`e verosimilmente maggiore rispetto al disturbo
operato dai pochi adulti territoriali di gipeto presenti
nell{\textquoteright}area.

Le osservazioni mediante \textit{focal sampling} sui nidi occupati non
hanno per\`o evidenziato alcuna differenza significativa della
percentuale di tempo investito nelle cure parentali tra coppie di
successo che nidificano vicine e lontane da nidi di gipeto, mentre \`e
emerso che il numero di coppie di aquila che intraprende la deposizione
\`e significativamente maggiore nei siti oltre 5 km da nidi di gipeto
(Chi-quadro= 5,57, 1 gl, p= 0,018), lasciando presupporre che la
presenza di coppie vicine di gipeto aumenti la frequenza delle coppie
di aquila reale che non iniziano la cova.  

Dall{\textquoteright}analisi di 132 interazioni, 91 sono avvenute tra
aquile e gipeti territoriali. L{\textquoteright}aquila reale ha
mostrato un{\textquoteright}aggressivit\`a significativamente maggiore
rispetto al gipeto attaccandolo nel 74\% dei casi (N= 91; Chi quadro=
20.3, p{\textless}0.01). Assumendo che le coppie di vecchio
insediamento (sedentarie e caratterizzate da una spiccata longevit\`a)
abbiano acquisito una conoscenza consolidata delle abitudini delle
coppie confinanti e delle loro {\textquotedblleft}\textit{no-fly
zones}{\textquotedblright} e, quindi, ne rispettino i confini, \`e
atteso che, laddove vi siano coppie di neoformazione e che nidificano
tra loro vicine, la frequenza delle interazioni aggressive sia
maggiore. Il 71,2\% delle aggressioni totali (inter e intra specifiche)
\`e stato infatti registrato in 2 soli territori in cui negli anni di
\textit{focal sampling} si sono insediate tre coppie di aquila reale di
neoformazione che hanno condotto la maggior parte degli attacchi.
Dall{\textquoteright}analisi spazio temporale la percentuale delle
interazioni aggressive tra le due specie \`e stata significativamente
maggiore (67\%) all{\textquoteright}interno dei \textit{patches} 1 e 3
e minore (33\%) nei \textit{patches} 5 che sono probabilmente meno
strategici in periodo riproduttivo (Chi quadro 12.03; 1 gl;
p{\textless}0.01). I \textit{patches} 1 e 3, aree assunte come
{\textquotedblleft}\textit{no-fly zones{\textquotedblright}},
identificano spazialmente quei settori che includono i nidi e in cui
gli individui territoriali non tollerano potenziali azioni e incursioni
(ad es. soste prolungate e sorvoli) da parte di intrusi e per questo
vengono suscitati a compiere azioni di attacco.
L{\textquoteright}ipotesi che le coppie neo insediate abbiano minor
conoscenza delle abitudini dell{\textquoteright}altra specie o ne
invadano intenzionalmente le aree di rispetto sembra confermata. 

Le interazioni avvengono con frequenza significativamente maggiore nei
\textit{patches} che includono i nidi (siti ad alto valore biologico
potenzialmente usurpabili e strategicamente importanti per il successo
riproduttivo della coppia nel medio periodo) ma che, per le voluminose
dimensioni che ne facilitano l{\textquoteright}individuazione da parte
dei rapaci, potrebbero svolgere anche un{\textquoteright}importante
funzione territoriale. In alcuni grandi rapaci, infatti, le specie
pi\`u forti, occupando molti dei nidi presenti in
un{\textquoteright}area, riducono la densit\`a riproduttiva della
specie pi\`u debole oppure la relegano verso nidi di minore qualit\`a
(White \& Cade 1971; Newton 1979). In quest{\textquoteright}ottica
l{\textquoteright}elevata frequenza osservata di interazioni aggressive
tra coppie di aquila di neoinsediamento e di gipeti territoriali pu\`o
spiegare, almeno in parte, il calo del successo riproduttivo
dell{\textquoteright}aquila. Quest{\textquoteright}ultima potrebbe
essere, infatti, sottoposta a un maggiore stress derivante dal rischio
di possibili interazioni/usurpazioni dei nidi da parte del gipeto che,
nidificando pi\`u precocemente rispetto all{\textquoteright}aquila,
potrebbe risultare favorito nel periodo di selezione dei siti e delle
pareti di nidificazione. Questa forma di competizione interspecifica
sembra attenuarsi fino a scomparire nelle coppie di aquila e gipeto che
nidificano da pi\`u anni a stretto contatto e che quindi avrebbero
verosimilmente imparato a conoscere e a rispettare i confini
territoriali delle coppie confinanti, riducendo in tal modo il numero
di interazioni aggressive.

\begin{table}[!h]
\centering
\begin{tabular}{>{\raggedright\arraybackslash}p{.05\columnwidth}>{\raggedright\arraybackslash}p{.45\columnwidth}>{\raggedright\arraybackslash}p{.1\columnwidth}>{\raggedright\arraybackslash}p{.1\columnwidth}>{\raggedright\arraybackslash}p{.1\columnwidth}}
\toprule
\textbf{N} & \textbf{Modello} & \textbf{LRT} & \textbf{$\Delta$LRT} & \textbf{P} \\
\toprule
%\showrowcolors
1 & \textit{Aquila} NND & 99.52 & & \\
2 & \textit{Aquila} NND + \textit{Aquila} NND repr & 98.68 & 0.84 & 0.36 \\
3 & \textit{Aquila} NND + FLOAQ & 98.29 & 1.23 & 0.27 \\
4 & \textit{Aquila} NND + AN & 90.94 & 8.58 & 0.001 \\
5 & \textit{Aquila} NND + AN + \textit{Gypaetus} NND & 86.86 & 4.08 & 0.04 \\
6 & \textit{Aquila} NND + AN + \textit{Gypaetus} NND + AN* \textit{Gypaetus} NND & 82.50 & 4.36 & 0.04 \\
\bottomrule
\hiderowcolors
\end{tabular}
\caption{Selezione del modello migliore che mette in relazione la produttivit\`a della popolazione di aquila reale del settore lombardo del Parco Nazionale dello Stelvio, per il periodo 2004-2010, con le seguenti variabili indipendenti: \textit{Aquila} NND, distanza dalla coppia di aquila reale pi\`u vicina, espressa mediante i valori di \textit{nearest neighbour distance}; \textit{Aquila} NND repr, distanza dalla coppia di aquila reale pi\`u vicina che si \'e riprodotta nello stesso anno; FLOAQ, densit\`a di \textit{floaters} di aquila reale rilevati durante il censimento contemporaneo effettuato in periodo tardo invernale; AN, altezza media del manto nevoso invernale; \textit{Gypaetus} NND, distanza dalla coppia di gipeto pi\`u vicina; LRT valore del \textit{likelihood ratio test}; $\Delta$LRT differenza tra il valore del LRT del modello testato rispetto al precedente}
\label{Bassi_tab_1}
\end{table}

\begin{table}[!h]
\centering
\begin{tabular}{>{\raggedright\arraybackslash}p{.05\columnwidth}>{\raggedright\arraybackslash}p{.25\columnwidth}>{\raggedright\arraybackslash}p{.12\columnwidth}>{\raggedright\arraybackslash}p{.12\columnwidth}>{\raggedright\arraybackslash}p{.12\columnwidth}>{\raggedright\arraybackslash}p{.12\columnwidth}}
\toprule
\textbf{N} & \textbf{Variabile} & \textbf{B} & \textbf{ES} & \textbf{P} & $\mathbf{e^{B}}$ \\
\toprule
%\showrowcolors
1 & \textit{Aquila} NND (+) & 0.359 & 0.154 & 0.019 & 1.42 \\
2 & \textit{Gypaetus} NND (+) & 0.217 & 0.090 & 0.016 & 1.24 \\
3 & AN (-) & -0.001 & 0.013 & 0.962 & 0.99 \\
4 & \textit{Gypaetus} NND * AN & -0.002 & 0.001 & 0.056 & 0.99 \\
5 & Costante & -2.275 & 1.425 & 0.110 & 0.10 \\
\bottomrule
\hiderowcolors
\end{tabular}
\caption{Modello di regressione logistica che mette in relazione la produttivit\`a della popolazione di aquila reale del settore lombardo del Parco Nazionale dello Stelvio, per il periodo 2004-2010, con la distanza tra coppie di aquila (“\textit{Aquila} NND”, espressa mediante i valori del \textit{nearest neighbour distance}) e con la distanza dalla coppia pi\`u vicina di gipeto, (\textit{Gypaetus} NND” – entrambi i fattori hanno una correlazione positiva con la produttivit\`a) e con l’altezza media del manto nevoso invernale, “AN”, che mostra al contrario una correlazione negativa}
\label{Bassi_tab_2}
\end{table}

\section*{Ringraziamenti}

Per suggerimenti nell{\textquoteright}impostazione della ricerca:
Heinrich Haller, Giuseppe Bogliani e David Jenny.
L{\textquoteright}Ufficio Fauna del PN Stelvio, gli oltre 200
volontari, gli Agenti forestali del CTA di Bormio, gli Agenti forestali
trentini dell{\textquoteright}Uff. distrettuale di Mal\'e e le Guardie
della Polizia Provinciale di Sondrio, Lecco e Brescia
impegnati nei censimenti contemporanei. Maurizio Bagnasco ed Enrico
Pregliasco (Tersia - Savona) per il finanziamento di 3 borse di 
studio.

\section*{Bibliografia}
\begin{itemize}\itemsep0pt
	\item Bassi E., 2010 -\textbf{ }Il Gipeto \textit{Gypaetus barbatus} sulle
Alpi: aggiornamento dei risultati del progetto internazionale di
reintroduzione. \textit{Ficedula} N.~44: 14-18; 

	\item Bassi E., 2014 - Sintesi dei risultati del {\textquotedblleft}XX
Censimento contemporaneo di Aquila reale e Gipeto nel Parco Nazionale
dello Stelvio e in aree limitrofe{\textquotedblright}, Parco Nazionale
dello Stelvio, Report interno. In collaborazione con: Bragalanti N.,
Buffa A. and Trotti P., 28 pp.

Haller H., 1996 \textsc{{}- }Der steinadler in der Graubunden.
Langfristige Unteruchungen zur Populationsokologie von \textit{Aquila
chrysaetos }im Zentrum der Alpen. \textit{Der Ornithologische
Beobachter, }9: 1-167. 

	\item Hosmer D.W. \& S. Lemeshow, 2000 - \textit{Applied logistic regression}
-- second edition. Wiley-Interscience, 373 pp. 

	\item Jenny D.,\textsc{ 1992 - }Bruterfolg und Bestandsregulation einer
alpinen Population des Steinadlers \textit{Aquila chrysaetos.}
\textit{Der Ornithologische Beobachter}\textsc{, 89: 1-43. }

	\item Newton I., 1979 \textit{{}- }\textit{Population Ecology of Raptors}.
Berkhamsted, UK. T \& AD Poyser.

	\item White C.M. \& Cade T.J., 1971 - Cliff-nesting raptors and Ravens along
the Colville River in arctic Alaska. \textit{Living Bird} 10: 107-50.
\end{itemize}

\setcounter{figure}{0}
\setcounter{table}{0}

\begin{adjustwidth}{-3.5cm}{0cm}
\pagestyle{CIOpage}
\authortoc{\textsc{Becciu P.}, \textsc{Stanzione V.}, 
\textsc{Massa B.}, \textsc{Dell'Omo G.}}
\chapter*[Riconoscimento parentale nella berta maggiore]{Riconoscimento parentale nella berta maggiore
\textbf{\textit{Calonectris diomedea}}\textbf{: un test con
l{\textquoteright}adozione incrociata dei pulcini}}
\addcontentsline{toc}{chapter}{Riconoscimento parentale nella berta maggiore}

\textsc{Paolo Becciu}$^{1*}$, \textsc{Viviana Stanzione}$^{1}$, 
\textsc{Bruno Massa}$^{2}$, \textsc{Giacomo Dell'Omo}$^{1}$ \\

\index{Becciu Paolo} \index{Stanzione Viviana} \index{Massa Bruno} \index{Dell'Omo Giacomo}
\noindent\color{MUSEBLUE}\rule{27cm}{2pt}
\vspace{1cm}
\end{adjustwidth}



\marginnote{\raggedright $^1$Ornis italica, Piazza Crati 15, 00199, Roma, IT \\
$^2$Dipartimento di Scienze Agrarie e Forestali,
Universit\`a di Palermo, Viale delle Scienze 13, 90128 Palermo, IT \\
\vspace{.5cm}
{\emph{\small $^*$Autore per la corrispondenza: \href{mailto:pablob989@gmail.com}{pablob989@gmail.com}}} \\
\keywords{\textit{Calonectris diomedea}, Linosa, adozione
incrociata, riconoscimento parentale}
{\textit{Calonectris diomedea}, Linosa island, cross
fostering, chicks recognition}
%\index{keywords}{\textit{Calonectris diomedea}} \index{keywords}{Linosa} \index{keywords}{Adozione incrociata} \index{keywords}{Riconoscimento parentale}
}
{\small
\noindent \textsc{\color{MUSEBLUE} Summary} / Scopoli{\textquoteright}s shearwater \textit{Calonectris diomedea} is a
pelagic seabird that breeds on small Mediterranean islands. During the
breeding season, adults return at night to the colony to feed their
chick while feeding themselves during the day. Most Procellariiformes
species, as shearwaters, feed their chick with highly nutrient stomach
oil, allowing chick{\textquoteright}s survival for several days and
toleration of irregular feeding attendance. It has been shown, by
T-maze tests, that parents are able to recognize their own chick by
smell. But how will they behave in the presence of a foreign chick? 
They might respond by feeding the adoptive chick as their own, or on
the contrary provide insufficient or no food to it. To test these
possibilities we cross-fostered 14 chicks and we measured their weight
and bill length once every three days for a nine-days period. Another
14 chicks, remained in their nests, were measured with the same
schedule to serve as control group. There was no significant difference
between the control and the experimental group, as both showed a normal
weight increase and similar bill growth. Hence, we showed that the
exchanged chicks received the same care as control chicks. Obviously,
these results do not prove nor exclude the lack of individual
recognition, but confirm that adult birds adopt any chick in their
nest. Further studies are required to test the importance of the
olfactory stimuli and other cues on chicks recognition.  \\
\noindent \textsc{\color{MUSEBLUE} Riassunto} / La berta maggiore \textit{Calonectris diomedea} \`e un uccello marino pelagico che nidifica in piccole isole del Mediterraneo. Durante il
periodo riproduttivo torna nella colonia durante la notte e resta in
mare aperto durante il giorno per nutrirsi. Le berte alimentano il
proprio pulcino con un olio prodotto nello stomaco, che permette al
pulcino di tollerare una frequenza irregolare di nutrizione. Il
genitore sembra riconoscere il proprio piccolo
dall{\textquoteright}olfatto e riesce a distinguerlo dai pulcini dei
nidi vicini. Questa informazione, ottenuta in passato con dei test di
scelta, non \`e stata verificata sul campo con la sostituzione dei
pulcini. Cosa succederebbe se un genitore trovasse nel nido un pulcino
diverso dal proprio? Le ipotesi in gioco sono che potrebbe alimentarlo
meno o allo stesso modo del proprio.
Per verificare queste due ipotesi abbiamo scambiato di nido 14 pulcini
provenienti da 7 nidi e monitorato ogni tre giorni per nove giorni
l{\textquoteright}andamento del peso corporeo.  Il gruppo di controllo
era costituito da 13 pulcini rimasti nei rispettivi nidi. Non sono
emerse differenze tra i pulcini di controllo e quelli scambiati che
hanno mostrato un normale incremento ponderale. Ci\`o ovviamente non
dimostra n\'e esclude la mancanza di un riconoscimento individuale, ma
suggerisce che stimoli di altro tipo garantiscono che
l{\textquoteright}adulto alimenti il pulcino nel nido a cui fa ritorno.
In test futuri si potranno controllare stimoli olfattivi. \\
}


\section*{Introduzione}


La berta maggiore mediterranea \textit{Calonectris diomedea} \`e un
Procellariforme coloniale caratterizzato da un elevato investimento
parentale, da un lungo periodo di incubazione (54 giorni) e allevamento
dei piccoli (90 giorni). Questa specie depone un solo uovo ed entrambi
i genitori nutrono il pulcino e partecipano al suo sviluppo fino
all{\textquoteright}involo. Con la crescita del pulcino diminuisce la
frequenza di imbeccata (Warham 1990) da parte dei genitori. I giovani
pulcini vengono nutriti con un particolare olio altamente energetico
secreto dallo stomaco dei genitori, che consente loro di sopportare
alcuni giorni di digiuno, cos\`i che il peso pu\`o subire variazioni a
causa della frequenza di nutrizione irregolare, mentre la crescita
corporea continua regolarmente.

Gli esemplari adulti riescono a trovare il proprio nido nel buio della
notte con l{\textquoteright}aiuto dell{\textquoteright}olfatto e della
vista, al loro arrivo nutrono il pulcino che li incita con il suo
pigolio insistente (\textit{begging}).

In questo studio, attraverso un esperimento di scambi incrociati tra
pulcini, abbiamo cercato di verificare l{\textquoteright}esistenza di
un riconoscimento genitore-figlio e la disponibilit\`a
all{\textquoteright}adozione di un pulcino non proprio. La tecnica
usata pone i genitori adottivi di fronte ad una serie di possibilit\`a
circa il nutrimento del pulcino adottivo: 1) nutrire il pulcino
estraneo come se fosse il proprio; 2) nutrirlo in modo anomalo rispetto
al comportamento standard; 3) non nutrirlo affatto. Da alcuni
esperimenti effettuati su altri Procellariformi nel passato con
tecniche di scelta, come il \textit{T-maze test}, \`e risultato
evidente che i genitori riconoscono i piccoli e il proprio nido
(Minguez 1997; Bonadonna \textit{et al.} 2004). Queste evidenze
potrebbero supportare la possibilit\`a del rifiuto del pulcino da parte
del genitore adottivo, o almeno una modalit\`a di nutrizione diversa a
causa del mancato riconoscimento del pulcino come proprio. 

\section*{Area di studio}

Il lavoro sul campo \`e stato svolto tra la fine di luglio e
l{\textquoteright}inizio di agosto del 2012 nell{\textquoteright}isola
di Linosa (Agrigento, isole Pelagie), durante il periodo di sviluppo
dei pulcini. Questa isola ospita la pi\`u grande colonia italiana di
berta maggiore mediterranea, stimata intorno alle 10.000 coppie (Massa
\& Lo Valvo 1986). 

\section*{Metodi}

I pulcini sperimentali (S; n=14) e quelli di controllo (C; n=14) sono
stati scelti casualmente tra i nidi gi\`a monitorati dal gruppo di
lavoro. Per lo studio sono stati usati solo nidi isolati, escludendo
quindi quelle situazioni in cui pi\`u coppie nidificano in una
cavit\`a. I pulcini sperimentali sono stati scambiati di nido a due
settimane di vita e, successivamente, sono stati misurati peso e
lunghezza del becco ad intervalli di tre giorni per quattro volte. Al
termine dell{\textquoteright}esperimento i pulcini scambiati sono stati
riposti nel loro nido e il loro sviluppo \`e proseguito fino
all{\textquoteright}involo (osservato nell{\textquoteright}ottobre
dello stesso anno). 

Per quanto riguarda la statistica si \`e scelta
un{\textquoteright}analisi della varianza a misure ripetute (GLM)
effettuata mediante SPSS 20.0 (SPSS Inc., Chicago, Illinois, U.S.A.). 

\section*{Risultati e discussione}

Le analisi non hanno evidenziato differenze tra il gruppo sperimentale e
il gruppo di controllo sia per quanto riguarda
l{\textquoteright}andamento del peso che
l{\textquoteright}accrescimento del becco (Fig. \ref{Becciu_fig_1}).

L{\textquoteright}esperimento \`e stato condotto al fine di valutare se
i genitori di berta maggiore alimentassero in maniera regolare un
pulcino non proprio, o se riconoscendolo non proprio, riducessero
l{\textquoteright}investimento parentale. Non essendo risultata alcuna
differenza di crescita nei parametri misurati tra i pulcini adottati e
quelli appartenenti al gruppo di controllo, possiamo asserire che i
pulcini adottati siano stati nutriti regolarmente dai genitori
adottivi. L{\textquoteright}elevata sensibilit\`a olfattiva di questi
uccelli, cos\`i come quella delle altre specie di Procellariformi
(Bonadonna \textit{et al.} 2004), aiuta loro a ritrovare nelle ore
pi\`u oscure della notte il proprio nido tra i molti presenti nella
colonia. Una spiegazione ai nostri risultati potrebbe quindi essere che
l{\textquoteright}odore molto forte del nido finisca con il mascherare
quello del pulcino, cos\`i da indurre l{\textquoteright}adulto a
nutrirlo come fosse il proprio. I pulcini emettevano comunque
vocalizzazioni e queste non hanno avuto effetti sulle modalit\`a di
alimentazione da parte dei genitori. Una seconda spiegazione, che non
esclude la precedente, consiste nella mancata selezione dei meccanismi
atti a un vero e proprio riconoscimento genitore-figlio, alla luce
della particolare modalit\`a di riproduzione di questi animali che ha
basse probabilit\`a di scambi tra nidiacei; ci\`o induce il genitore a
nutrire il pulcino che trova nel nido indipendentemente dal fatto che
sia il proprio. Bisogna inoltre considerare che
l{\textquoteright}investimento parentale in questa specie \`e
estremamente elevato e che il successo riproduttivo della coppia \`e
rappresentato dalle possibilit\`a offerte all{\textquoteright}unico
pulcino. Dopo aver investito energie nella deposizione
dell{\textquoteright}uovo e nella lunghissima incubazione, i genitori
intraprendono l{\textquoteright}ulteriore lunga fase di allevamento.
E{\textquoteright} probabile quindi che l{\textquoteright}identit\`a
del pulcino sia da loro posta in secondo ordine. \`E da precisare che
questo studio \`e stato condotto nella terza e quarta settimana di vita
dei pulcini, per cui non \`e escluso che il riconoscimento
genitore-figlio possa verificarsi solo durante una fase pi\`u avanzata,
come suggerito da uno studio di Storey \textit{et al. }(1992) sul
gabbiano tridattilo \textit{Rissa tridactyla}, in cui \`e stato
dimostrato che gli adulti iniziano a riconoscere il proprio pulcino
solo circa 15 giorni prima dell{\textquoteright}involo. \`E altres\`i
ipotizzabile che lo spostamento del materiale nei nidi in cui avviene
lo scambio incrociato sia un elemento chiave per la comprensione del
fenomeno dell{\textquoteright}adozione e del riconoscimento del nido,
come osservato negli uccelli delle tempeste \textit{Hydrobates
pelagicus} da Minguez (1997).


\begin{figure}[!h]
\centering
\includegraphics[width=.6\columnwidth]{Becciu_fig_1.png}
\caption{Andamento del peso e della lunghezza del becco (Media $\pm$ DS) dei pulcini adottati e dei pulcini di controllo durante il periodo di sperimentazione}
\label{Becciu_fig_1}
\end{figure}

\section*{Ringraziamenti}

Il lavoro da campo \`e stato sostenuto da \textit{Ornis italica} e dal
Dipartimento di Scienze Agrarie e Forestali
dell{\textquoteright}Universit\`a di Palermo
nell{\textquoteright}ambito del progetto LIFE+ Nat/It 00093
{\textquotedblleft}Pelagic Birds{\textquotedblright}.

\section*{Bibliografia}
\begin{itemize}\itemsep0pt
	\item Bonadonna F., Villafane M., Bajzak C. \& Jouventin P., 2004 -
Recognition of burrow{\textquoteright}s olfactory signature in blue
petrels, \textit{Halobaena caerulea}: an efficient discrimination
mechanism in the dark. \textit{Anim. Behav}, 67: 893-898.

	\item Massa B. \& Lo Valvo M., 1986 - Biometrical and biological
considerations on the Cory{\textquoteright}s Shearwater
\textit{Calonectris diomedea}. In: Medmaravis \& Monbailliu (eds),
\textit{Mediterranean Marine Avifauna}. Springer-Verlag, Berlin:
293-313.

	\item Minguez E., 1997 - Olfactory nest recognition by British storm-petrel
chicks. \textit{Anim. Behav.}, 53: 701-707. 

	\item Storey A.E., Anderson R.E., Porter J.M. \& McCharles A.M., 1992 -
Absence of parent--young recognition in kittiwakes: a re-examination.
\textit{Behaviour}, 120: 302--323.

	\item Warham J., 1990 - The petrels: Their ecology and breeding
systems. Academic Press, London, 452 pp.
\end{itemize}

\setcounter{figure}{0}
\setcounter{table}{0}


\begin{adjustwidth}{-3.5cm}{0cm}
\pagestyle{CIOpage}
\authortoc{\textsc{Calandri S.}, \textsc{Ragni B.}, 
\textsc{Andreini F.}}
\chapter*[Colombi urbani e la riduzione della
risorsa acqua]{\textcolor{black}{Il problema dei colombi
urbani} \textbf{\textit{\textcolor{black}{Columba
livia }}}\textbf{\textcolor{black}{affrontato con un
esperimento di riduzione della risorsa acqua}}}
\addcontentsline{toc}{chapter}{Colombi urbani e la riduzione della risorsa acqua}

\textsc{Simone Calandri}$^{1*}$, \textsc{Bernardino Ragni}$^{2}$, 
\textsc{Federica Andreini}$^{1}$ \\

\index{Calandri Simone} \index{Ragni Bernardino} \index{Andreini Federica}
\noindent\color{MUSEBLUE}\rule{27cm}{2pt}
\vspace{1cm}
\end{adjustwidth}



\marginnote{\raggedright $^1$Ufficio Ambiente, Comune di Spoleto \\
$^2$Dipartimento di \textcolor{black}{Chimica Biologia
Biotecnologie}, Universit\`a degli Studi di Perugia \\
\vspace{.5cm}
{\emph{\small $^*$Autore per la corrispondenza: \href{mailto:simone.calandri@comunespoleto.gov.it}{si\allowbreak mo\allowbreak ne.\allowbreak ca\allowbreak lan\allowbreak dri@\allowbreak co\allowbreak mu\allowbreak ne\allowbreak spo\allowbreak le\allowbreak to.\allowbreak gov.\allowbreak it}}} \\
\keywords{\textit{Columba livia}, popolazione urbana,
Spoleto}
{\textit{Columba livia}, urban population, Spoleto city}
%\index{keywords}{\textit{Columba livia}} \index{keywords}{Popolazione urbana} \index{keywords}{Spoleto}
}
{\small
\noindent \textsc{\color{MUSEBLUE} Summary} / In the town of Spoleto (Umbria), urban population of \textit{Columba
livia }is estimated at least 4000 individuals, one of the highest
population densities known in the literature.
The freshwater made available by the numerous public fountains, supports
this large population; in fact, previous research of behavioral ecology
has shown that water is a key resource for the urban pigeon. In this
study, we report the results of an experiment involving the temporary
removal of this water for a period of 30 consecutive days; this time
frame can intercept significant fractions of the reproductive cycle of
\textit{Columba livia}. The demographic trend has been monitored
through indicators of the population consisting of 5 roost, by 4 counts
night, before, during and after the experiment. After good starting
results, human- and  weather-linked events prevented the project to be
successful. However, the counts have shown a significant reduction of
pigeons (25\%) up to episodes of heavy rain; we believe that under
ideal conditions the procedure could have achieved its aim.  \\
\noindent \textsc{\color{MUSEBLUE} Riassunto} / La popolazione di colombi stanziali entro la cinta urbica di Spoleto \`e
stimata in almeno 4000 individui, con una delle densit\`a pi\`u alte
note in letteratura. Un precedente studio di ecologia comportamentale
propedeutico al progetto di contenimento, ha dimostrato come la
principale causa ambientale dell{\textquoteright}elevata densit\`a di
colombi urbani risiedesse nel cruciale rapporto che la specie
stabilisce con l{\textquoteright}acqua, una risorsa molto abbondante
nel centro storico della citt\`a, offerta da numerose fontane e lavatoi
pubblici. In questo studio si riportano i risultati di un esperimento
di riduzione temporanea della risorsa acqua per un periodo di 30 giorni
consecutivi; tale durata consente di intercettare frazioni
significative del ciclo riproduttivo del colombo urbano \textit{Columba
livia}. L{\textquoteright}andamento demografico \`e stato monitorato
tramite la stima numerica di indicatori di popolazione, rappresentati
da 5 \textit{roost}, per mezzo di 4 conteggi notturni precedenti,
contemporanei e successivi all{\textquoteright}esperimento. Nonostante
alcuni problemi tecnico-amministrativi e un periodo di precipitazioni
eccezionali, le quantificazioni hanno indicato una significativa
riduzione di colombi (25\%) fino agli episodi di pioggia abbondante.
Sulla base di questi risultati, la stima statistica del tempo
necessario per raggiungere il teorico valore
{\textquotedblleft}0{\textquotedblright} degli indicatori di
popolazione \`e stata di 40.4 giorni di riduzione della risorsa acqua. \\
}



\section*{Introduzione}

Le popolazioni sinantropiche di \textit{Columba livia }possono assumere
aspetti altamente critici per la conservazione del patrimonio
storico-artistico-architettonico, per la salute umana e per il pubblico
decoro nei centri storici italiani ed europei (Baldaccini \& Giunchi
2006). Quella insediatasi entro la cinta urbica di Spoleto (km$^2$ 0,75;
Umbria) \`e stimata in almeno 4000 individui, presentando cos\`i, una
delle densit\`a pi\`u alte note in letteratura (NOMISMA, 2003; Ragni
\textit{et al.} 2008). Nell{\textquoteright}ambito del progetto di
controllo di detta popolazione, la sperimentazione di numerosi metodi
di cattura e di dissuasione dei colombi non ha portato ad alcun
risultato significativo e utile. 

Lo studio dell{\textquoteright}ecologia comportamentale, sulla base di
protocolli d{\textquoteright}osservazione e sperimentali, ha portato a
constatare la stretta dipendenza di \textit{Columba livia
}dall{\textquoteright}acqua fresca, ferma o corrente. La presenza di
numerose fontane e lavatoi pubblici funzionanti, uniformemente
distribuiti entro la cinta urbica spoletana, pu\`o potenzialmente
essere considerata il fattore ecologico basilare per tale eccezionale
densit\`a di popolazione. \textit{E.g.} la Fontana del Foro risulta
essere una delle pi\`u frequentate: applicando il
\textit{Lincoln-Petersen-Chapman Index }(Krebs 1999) a un campione
catturato-marcato-rilasciato \`e stato possibile stimare in 2192-2561 i
colombi gravitanti giornalmente su di essa. Tale fontana \`e stata
sottoposta a un programma di osservazione diretta di 6 ore al giorno
per 10 giorni: il numero di contatti colombo-acqua nei giorni sereni
\`e stato significativamente superiore a quello osservato nei giorni
piovosi (${\chi}$\textsuperscript{2} = 92,34  p{\textless}0,001;
Grafico 1). La Fontana del Mascherone \`e risultata la seconda risorsa
d{\textquoteright}acqua pi\`u importante per i colombi spoletani, posta
a 250 metri dalla precedente. La chiusura sperimentale della Fontana
del Foro ha portato i piccioni a frequentare la seconda, con un
conseguente aumento significativo dei frequentatori (senza considerare
interferenza umana: ${\chi}$\textsuperscript{2} = 4,8 
0,01{\textless}p{\textless}0,05; considerando interferenza umana:
${\chi}$\textsuperscript{2} = 37,2 p{\textless}0,001; Grafico 2).
Osservazioni comparative svolte a Foligno (Umbria) hanno mostrato
identici \textit{pattern }eco-etologici dei colombi nei confronti della
stessa risorsa, nella fattispecie rappresentata dal fiume Topino che
attraversa la citt\`a storica. 

Pertanto, l{\textquoteright}ipotesi di ricerca da sottoporre a test
sperimentale \`e quella secondo la quale: sottraendo alla
disponibilit\`a dei colombi l{\textquoteright}acqua delle fontane
pubbliche cittadine, si ottiene un allontanamento di questi dal centro
storico di Spoleto motivato dalla necessit\`a di ricercare nuove fonti
idriche. Con la chiusura delle fontane ci si attende quindi una
drastica riduzione della popolazione per ridimensionamento
dell{\textquoteright}attivit\`a riproduttiva e per emigrazione: il
colombo rimane legato al luogo di nascita dove torna a nidificare una
volta raggiunta la maturit\`a sessuale, quindi sfavorire la
nidificazione implica agevolarne lo spostamento definitivo verso luoghi
pi\`u idonei alla cura e allevamento dei \textit{pulli}.

\section*{Metodi}

Sono state censite 36 fontane e punti d{\textquoteright}acqua nel centro
storico o in prossimit\`a di questo, verificandone il funzionamento.
L{\textquoteright}\textit{optimum }sperimentale prevede un
prosciugamento di essi \textit{ad libitum}, il cui unico termine \`e
rappresentato dal raggiungimento degli scopi sperimentali: rimozione
completa o altamente significativa della popolazione presente nel
centro storico.

L{\textquoteright}impraticabilit\`a socio-politica del raggiungimento
dell{\textquoteright}\textit{optimum }sperimentale ha indotto i
ricercatori \textcolor{black}{a fissare a 30 giorni consecutivi il
periodo sperimentale di essiccamento delle fontane}; lasso di tempo che
consente di intercettare frazioni significative delle due fasi critiche
del ciclo riproduttivo di \textit{C. livia}: deposizione e cova,
schiusa e cura parentale, della durata complessiva media di 55 giorni.

In considerazione dei dati disponibili sulle serie storiche regionali
delle precipitazioni meteoriche, con particolare riferimento allo
Spoletino (regione Umbria, 1997, 2004) il mese di luglio \`e
statisticamente risultato quello pi\`u siccitoso
dell{\textquoteright}anno; quindi il migliore ai fini
dell{\textquoteright}esperimento. Anche in questo caso, considerazioni
socio-politiche hanno indotto a ubicare cronologicamente
l{\textquoteright}esperimento tra i giorni 3 giugno e 3 luglio.
L{\textquoteright}andamento della sperimentazione \`e stato monitorato
tramite la stima numerica di indicatori di popolazione, rappresentati
da 5 siti, selezionati per fenologia, ubicazione, accessibilit\`a e
stabilit\`a. Per ogni sito sono stati effettuati 4 conteggi notturni
precedenti l{\textquoteright}inizio dell{\textquoteright}esperimento, 4
durante l{\textquoteright}attuazione dell{\textquoteright}operazione e
4 successivamente alla riapertura delle fontane.

\section*{Risultati e discussione}

La chiusura dei punti d{\textquoteright}acqua ha richiesto almeno 2
giorni (Tab. \ref{Calandri_tab_1}); la prima quantificazione post-chiusura degli
indicatori (Tab. 1; Graf. 3) ha registrato una significativa riduzione
di colombi (${\chi}$\textsuperscript{2} = 5,7 
0,01{\textless}p{\textless}0,05) pari al 24,7\%. Successivamente si
sono avute gravi, inattese, interferenze nell{\textquoteright}offerta
d{\textquoteright}acqua. La prima consiste nella riapertura della
fontana del Mascherone e di quella del Fortilizio dei Molini, due siti
strategici, avvenuta almeno 2 giorni prima del secondo conteggio
post-chiusura  (Tab. \ref{Calandri_tab_1}). Inoltre, tra le 14 e le 16 del giorno in cui
\`e stato effettuato il secondo conteggio notturno, il territorio
spoletino \`e stato interessato da forti precipitazioni (Tab. \ref{Calandri_tab_1}). Da
tale giorno, il diciassettesimo dall{\textquoteright}avvio
dell{\textquoteright}operazione Fontisecche, Spoleto,
l{\textquoteright}Umbria e gran parte della Penisola, sono state
investite da precipitazioni piovose, prolungate e abbondanti, a cadenza
pressoch\'e giornaliera (Tab. \ref{Calandri_tab_1}). I successivi conteggi degli
indicatori certificano la pronta risposta dei colombi alle mutate
condizioni dell{\textquoteright}habitat: offerta
d{\textquoteright}acqua abbondante, continua e diffusa; tanto che il
loro andamento nel tempo procede in modo profondamente diverso dalle
attese (Graf. 3): tra i giorni 26 e 38 i valori sono tornati
nell{\textquoteright}intervallo pre-chiusura a causa di aperture
abusive di fontane e presenza di precipitazioni; tra i giorni 51 e 67
(riapertura delle fontane e perdurare di precipitazioni meteoriche)
\textcolor{black}{il numero dei colombi conteggiati} sale
all{\textquoteright}intervallo superiore, 105-113, per effetto
dell{\textquoteright}involo dei nuovo nati. 

Per quanto riguarda l{\textquoteright}esperimento Fontisecche,
l{\textquoteright}unico tentativo che pu\`o essere esperito, al fine di
immaginarne l{\textquoteright}esito se le condizioni sperimentali
previste fossero state rispettate, lo si pu\`o leggere
nell{\textquoteright}equazione: 

\begin{center}
 y = - 2,3x + 93 
\end{center}

della retta di regressione lineare (R\textsuperscript{2} =1,0) che lega
le uniche due situazioni libere dalle gravi condizioni di perturbazione
dell{\textquoteright}esperimento: l{\textquoteright}ultimo conteggio
pre- e il primo conteggio post- chiusura delle fontane.
L{\textquoteright}algoritmo consente di stimare quanto tempo possa
occorrere, in tali condizioni, per raggiungere il teorico valore
{\textquotedblleft}0{\textquotedblright} degli indicatori di
popolazione; tale lasso di tempo \`e stimato in 40,4\textbf{ }giorni:
un intervallo non molto distante da quello, arbitrario, proposto dai
ricercatori. Concludendo, dal punto di vista concreto,
l{\textquoteright}esperimento Fontisecche non ha raggiunto lo scopo
prefissato, quello di una drastica e durevole riduzione della
popolazione sinantropica di \textit{Columba livia }nel centro storico
di Spoleto, probabilmente a causa di gravi perturbazioni delle
condizioni sperimentali previste. Tuttavia, indicazioni parziali
suggeriscono che, in assenza delle anzidette perturbazioni,
l{\textquoteright}intervento avrebbe potuto funzionare
\textcolor{black}{e }portare a esito positivo con un tempo di
sperimentazione moderatamente pi\`u lungo di quello previsto e, aspetto
non secondario, del tutto privo di costi pubblici, anzi risparmiando
preziosa acqua potabile. I risultati del presente lavoro, seppur non
conclusivi e basati su un disegno sperimentale non pienamente
realizzato, supportano l{\textquoteright}ipotesi secondo la quale la
rimozione degli accessi all{\textquoteright}acqua nei confronti dei
colombi pu\`o essere un efficace strumento di gestione volto alla
riduzione delle popolazioni cittadine.

{\footnotesize
\begin{longtable}{>{\raggedright\arraybackslash}p{.1\columnwidth}>{\raggedright\arraybackslash}p{.18\columnwidth}>{\raggedright\arraybackslash}p{.62\columnwidth}}
\toprule
\textbf{Giorno} &  \textbf{N$^{\circ}$ colombi} & \textbf{Eventi} \\
\toprule
\endfirsthead
\multicolumn{3}{l}{\footnotesize Continua dalla pagina precedente} \\
\toprule
\textbf{Giorno} &  \textbf{N$^{\circ}$ colombi} & \textbf{Eventi} \\
\toprule
\endhead
%\showrowcolors
1 &	84&	Primo conteggio pre-chiusura (26 Maggio) \\
2 & & \\		
3& 	82 &	Secondo conteggio pre-chiusura \\
4-5 & & \\		
6 &	82 &	Terzo conteggio pre-chiusura \\
7 & & \\		
8 & 93 &	Quarto conteggio pre-chiusura \\
9	&	& Chiusura fontane \\
10 &	&	Chiusura fontane (4 Giugno) \\
11-17 & & \\		
18 & 70	& Primo conteggio post-chiusura \\
19-23 & & \\		
24	& & Apertura abusiva Fontane Mascherone e Fortilizio dei Mulini \\
25	& & \\	
26	& 88 &	Secondo conteggio post-chiusura; Nubifragio \\
27	& &	 Pioggia \\
28	& &	Pioggia \\
29	& & Temporale e pioggia \\
30	& & \\	
31	& &	Pioggia \\
32	& &	Nubifragio \\
33	& &	Temporale e pioggia \\
34	& & \\	
35	& 90 &	Terzo conteggio post-chiusura; Pioggia \\
36	& & \\	
37	& &	Nubifragio e pioggia \\
38	& 86 &	Quarto conteggio post-chiusura; Temporale e pioggia \\
39	& &	Apertura fontane (3 Luglio); Temporale e pioggia \\
40	& &	Apertura fontane; Temporale e pioggia \\
41	& &	Apertura fontane; Temporale e pioggia \\
42-50 & & \\		
51	& 106 &	Primo conteggio post-apertura \\
52-55 & & \\		
56 & 105 & Secondo conteggio post-apertura \\
57-62 & & \\		
63	& 113 &	Terzo conteggio post-apertura \\
64-66 & & \\		
67 & 109 & Quarto conteggio post-apertura (31 Luglio) \\
\bottomrule
\hiderowcolors
\caption{Esperimento “Fontisecche”}
\label{Calandri_tab_1}
\end{longtable}
}

\begin{figure}[!h]
\centering
\includegraphics[width=.95\columnwidth]{Calandri_fig_1.png}
\caption{Fontana del Foro (Spoleto, Umbria): contatti colombi - acqua in relazione alle giornate piovose (PIOGGIA) e serene (SOLE)}
\label{Calandri_fig_1}
\end{figure}

\begin{figure}[!h]
\centering
\includegraphics[width=.95\columnwidth]{Calandri_fig_2.png}
\caption{Fontana del Mascherone (Spoleto, Umbria): contatti colombi - acqua mentre la vicina Fontana del Foro porta acqua (ACQUA) e non la porta (ASCIUTTA). Tra i giorni 16 e 19 si \`e verificato un forte e continuo disturbo antropico in prossimit\`a della Fontana del Mascherone che ha spaventato i colombi}
\label{Calandri_fig_2}
\end{figure}

\begin{figure}[!h]
\centering
\includegraphics[width=.95\columnwidth]{Calandri_fig_3.jpg}
\caption{Esperimento "Fontisecche" (Spoleto, Umbria). Conteggio dei colombi (ascisse) nei 67 giorni (ordinate) di sperimentazione. Chiusura temporanea fontane pubbliche (giorni 10-38) e riapertura definitiva (dal giorno 39)}
\label{Calandri_fig_3}
\end{figure}



\newpage 
\section*{Bibliografia}
\begin{itemize}\itemsep0pt
	\item \textcolor[rgb]{0.13725491,0.12156863,0.1254902}{Baldaccini N. E.,
Giunchi D., }\textcolor[rgb]{0.13725491,0.12156863,0.1254902}{2006 - 
}\textcolor[rgb]{0.13725491,0.12156863,0.1254902}{Le popolazioni urbane
di colombo: considerazioni sulla loro genesi e sulle metodologie di
gestione. }\textit{Biologia Ambientale}, 20 (2): 125-141. 

	\item Krebs J.C., 1999 - \textit{Ecological Methodology}. Addison Wesley
Longman, Menlo Park (CA): 581 pp.

	\item NOMISMA, 2003 - \textit{Valutazione dei costi economici e sociali dei
colombi in ambiente urbano}. Rapporto finale di ricerca. Bologna: 218
pp.

	\item Ragni B., Calandri S., Andreini F. \& Manni A.C., 2008 - Progetto
colombo urbano (\textit{Columba livia} forma \textit{domestica}) nel
centro storico della citt\`a di Spoleto. Universit\`a degli Studi di
Perugia, Comune di Spoleto, Perugia: 73 pp.

	\item Regione Umbria, 1997 - \textit{Relazione sullo stato
dell{\textquoteright}ambiente in Umbria}. Perugia: 344 pp.

	\item Regione Umbria,  2004 - \textit{Relazione sullo stato
dell{\textquoteright}ambiente in Umbria}. Perugia: 448 pp.
\end{itemize}
\setcounter{figure}{0}
\setcounter{table}{0}

\begin{adjustwidth}{-3.5cm}{0cm}
\pagestyle{CIOpage}
\authortoc{\textsc{Di Maggio R.}, \textsc{Campobello D.}, 
\textsc{Mascara R.}, \textsc{Sar\`a M.}}
\chapter*[Microclimate affects lesser kestrel breeding success in
Sicily]{Nest characteristics affect microclimate and breeding success of
lesser kestrel \textbf{\textit{Falco naumanni}}\textbf{ in the Gela
plain}}
\addcontentsline{toc}{chapter}{Microclimate affects lesser kestrel breeding success in
Sicily}

\textsc{Rosanna Di Maggio}$^{1*}$, \textsc{Daniela Campobello}$^{1}$, 
\textsc{Rosario Mascara}$^{2}$, \textsc{Maurizio Sar\`a}$^{1}$ \\

\index{Di Maggio Rosanna} \index{Campobello Daniela} \index{Mascara Rosario} \index{Sar\`a Maurizio}
\noindent\color{MUSEBLUE}\rule{27cm}{2pt}
\vspace{1cm}
\end{adjustwidth}



\marginnote{\raggedright $^1$STEBICEF Department (Biological, Chemical and
Pharmaceutical Sciences and Technologies), University of Palermo, Italy \\
$^2$Fondo Siciliano per la Natura, via Popolo, 6, 93015,
Niscemi (Caltanissetta, Italy) \\
\vspace{.5cm}
{\emph{\small $^*$Autore per la corrispondenza: \href{mailto:rosannadimaggio@gmail.com}{ro\allowbreak san\allowbreak na\allowbreak di\allowbreak mag\allowbreak gio@\allowbreak g\allowbreak ma\allowbreak il.\allowbreak com}}} \\
\keywords{\textit{Falco naumanni}, Sicilia, colonialit\`a
facoltativa, temperatura nel nido, umidit\`a}
{\textit{Falco naumanni}, Sicily,
facultative coloniality, nest temperature, humidity}
%\index{keywords}{\textit{Falco naumanni}} \index{keywords}{Sicilia} \index{keywords}{Colonialit\`a
%facoltativa} \index{keywords}{Temperatura nel nido} \index{keywords}{Umidit\`a}
}
{\small
\noindent \textsc{\color{MUSEBLUE} Summary} / Microclimate is an important factor for nest site selection and it
influences breeding success. Secondary-cavity nesters are compelled by
nest site availability to select existing nest structures to optimize
microclimate conditions. The lesser kestrel is a colonial raptor
breeding in pseudo-steppe habitats. It does not build a nest but breeds
under roof tiles or wall crevices of rural buildings. We studied 45
nest sites in five lesser kestrel \textit{Falco naumanni}\textbf{
}colonies located in the Gela Plain (Sicily). We measured microclimate
by placing thermo loggers inside nests so that we could record
temperature and relative humidity every hour, from laying to fledging
period. Our results revealed a significant effect of nest orientation
and nest type on the relative humidity, with the highest values in
nests under tiles placed in N-NW sides. Temperature was higher in hole
nests than in those under tiles. Nest orientation and nest type created
a specific microclimate that affected the reproductive outcome. In
particular, a nest under tiles in a S-SE side had a higher hatching
success while a nest in a hole in a S-SE side had a higher chance to
produce more fledglings.  \\
\noindent \textsc{\color{MUSEBLUE} Riassunto} / Il microclima \`e un importante aspetto della selezione del sito di
nidificazione e influenza il successo riproduttivo. Specie che non
costruiscono il nido ma usano strutture preesistenti
(\textit{secondary-cavity nesters}) sono particolarmente limitate dalla
disponibilit\`a del sito riproduttivo poich\'e obbligate a selezionare
strutture gi\`a esistenti al fine di ottimizzare le condizioni
microclimatiche. Il grillaio \`e un rapace coloniale che nidifica in
habitat pseudo-steppici. Nidifica sotto le tegole dei tetti o nei buchi
di edifici rurali abbandonati. In questo lavoro abbiamo studiato 45
nidi in cinque colonie di grillaio situate nella Piana di Gela. Il
microclima del nido \`e stato rilevato inserendo dei sensori termici,
in modo da registrare temperatura e umidit\`a relativa ogni ora del
giorno, dalla deposizione delle uova all{\textquoteright}involo dei
pulcini. I risultati rivelano un effetto
dell{\textquoteright}esposizione e della tipologia di nido
sull{\textquoteright}umidit\`a relativa con i valori pi\`u alti nei
nidi sotto le tegole esposte a N-NO mentre la temperatura risulta pi\`u
alta nei buchi rispetto alle tegole. L{\textquotesingle}esposizione e
la tipologia di nido creano uno specifico microclima che sembra
influenzare il successo riproduttivo. In particolare, un nido sotto le
tegole esposte a S-SE presenta un maggiore tasso di schiusa mentre un
nido in un buco esposto a S-SE invola pi\`u pulcini. \\
}


 
\section*{Introduction}

Specific microclimate within the nest (i.e. temperature and humidity) is
an important factor driving nest site selection and might influence
breeding success (Martin 1998; Lloyd \& Martin 2004). Some birds seem
to actively choose certain cavities but not all birds build their own
nests (Robert \textit{et al. }2010)\textit{.} Secondary-cavity nesters
are compelled by nest site availability to select existing nest
structures that minimize predation pressures while protecting eggs and
chicks against climatic variations (Sar\`a \textit{et al.} 2012). Such
species may optimize microclimate by selecting specific nest sites
characterized by specific exposures associated with suitable thermal
characteristics. Several studies have in fact shown the importance of
nest location and orientation with regard to solar radiation (e.g.
Lloyd \& Martin 2004). Nest microclimate can have important influences
on parental reproduction in secondary cavity nesting birds. The lesser
kestrel \textit{Falco naumanni} is a small raptor breeding in colonies
of variable size (2-60 pairs, Catry \textit{et al.} 2011; Sar\`a 2010)
in pseudo-steppe habitats, choosing its hole-nest in cliffs, under roof
tiles or wall crevices of rural buildings (Di Maggio \textit{et al.}
2013). It does not build a nest but lays its eggs directly on the
cavity floor after scraping the substrate. This species, former
considered {vulnerable}, has recently changed its
conservation status to {\textquotedblleft}least
concern{\textquotedblright}, due to conservation actions in part of its
range (I\~nigo \& Barov 2011). Also, the Gela plain lesser kestrel
population, the largest in Sicily, has grown consistently over the last
decade (Sar\`a 2010). In this species little is known about the effects
of nest thermal characteristics on reproductive success, so the aims of
this study are: 1) to determine the relationship between microclimate
and nest site characteristics; 2) to evaluate whether and how nest site
characteristics affects lesser kestrel breeding success.

\section*{Methods}

The Gela plain, located in south-eastern Sicily (Italy, 378070N,
148190E), is one of the largest plains in Italy (about 474
km\textsuperscript{2}). Due to limited precipitation (350 mm/year), the
agricultural landscape is composed of a mosaic of pseudo-steppes
dominated by artichoke \textit{Cynara} spp. fields and non-irrigated
crops. Across the plain, numerous rural buildings, often partially
destroyed or abandoned, host lesser kestrel nests (Di Maggio \textit{et
al.} 2014). Each colony is usually composed of a single building, or
two or more small houses very close, but with different sides.

The study was conducted in 2010, from April to July. We studied 45 nests
in five lesser kestrel colonies of different size. Colonies were
located within two different areas of the plain: the core area, where
the surface of all cropland land uses was ${\geq}$50 \% within a radius
of 500 m around the colony and an altitude between 0 ${\leq}$ 100 m
a.s.l and the edge, where the cropland area was {\textless} 50\% and
the altitude {\textgreater} 100 m a.s.l. We measured nest microclimate
by setting thermo loggers, so that temperature and relative humidity of
the nest were recorded each day every hour, from the laying (April-May)
to the fledging period (June-July). Each nest was checked at least 3-4
times in order to record the number of eggs and nestlings and then to
measure hatching and fledging rates. 

We used a generalized linear mixed model (GLMM, McCullagh \& Searle
2000) with a normal distribution and an identity link function to
describe the relationship between microclimate and the following
features: 1) colony location, 2) nest-type and 3) nest orientation.
Moreover, we used a second GLMM to test the relationship between
hatching and fledging rate and nest characteristics. In both analyses
we included colony and nest identities as random effects.

\section*{Discussion}

Our results revealed a significant effect of nest type (tiles or wall
holes) on the temperature inside nests with higher temperatures in
holes than under tiles (Fig. \ref{DiMaggio_fig_1}, Tab. \ref{DiMaggio_tab_1}). Furthermore, we recorded an
effect of nest orientation and nest type on the relative humidity with
the highest values in nests under tiles placed in N-NW sides (Figg. \ref{DiMaggio_fig_2} -
\ref{DiMaggio_fig_3}, Tab. \ref{DiMaggio_tab_2}). We did not find any effect of the colony location on both
temperature and relative humidity (Tab. \ref{DiMaggio_tab_1} - \ref{DiMaggio_tab_2}), and this result
suggested that differences in habitat and topography (i.e. altitude)
did not reflect a relevant change in microclimate inside nests.

Nest orientation and type determined when and for how long a nest is
exposed to direct solar radiation and wind.  Additionally, nest
orientation largely determined temperature and humidity within the
nest. Other factors (e.g., parent presence or number of chicks) are
warranted of further investigation for their potential effect on
microclimate. Nest orientation and nest type resulted in a specific
microclimate, affecting the reproductive success. In particular, a nest
under tiles in a S-SE building side would have a higher hatching
success (GLMM, F\textsubscript{side} = 3.114, P {\textless} 0.001,
F\textsubscript{type  } = 63.87, P {\textless} 0.001); additionally a
nest in a hole in a S-SE side had a higher chance to produce more
fledglings (GLMM, F\textsubscript{side} = 9.569, P {\textless} 0.001,
F\textsubscript{type  } = 5.634, P {\textless} 0.001; Figg. \ref{DiMaggio_fig_4} - \ref{DiMaggio_fig_5}).
Nest type and nest exposure compensate humidity, as a drier nest type
(hole) in a wetter side (N-NW) should have the same relative humidity
than the reverse combination (tile nest in dry S-SE side). Nest site
characteristics (i.e. type and orientation) have in turn a strong
impact on reproductive success through their effects on microclimate.
This latter could in conclusion minimize thermal requirements of eggs
and nestlings and improve reproductive success. 


\newpage
\begin{figure}[!h]
\centering
\begin{minipage}{0.49\textwidth}
\centering
	\includegraphics[width=.97\columnwidth]{DiMaggio_fig_1.png}
	\caption{Temperature inside lesser kestrel nests as a function of nest type (GLMM, N = 45)}
	\label{DiMaggio_fig_1}
\end{minipage}\hfill
\begin{minipage}{0.49\textwidth}
\centering
	\includegraphics[width=.97\columnwidth]{DiMaggio_fig_2.png}
	\caption{Relative humidity inside lesser kestrel nests as a function of nest type (GLMM, N = 45)}
	\label{DiMaggio_fig_2}
\end{minipage}
\end{figure}

\begin{figure}[!h]
\centering
	\includegraphics[width=.6\columnwidth]{DiMaggio_fig_3.png}
	\caption{Relative humidity inside lesser kestrel nests as a function of nest orientation (GLMM, N = 45)}
	\label{DiMaggio_fig_3}
\end{figure}

\begin{figure}[!h]
\centering
	\includegraphics[width=.95\columnwidth]{DiMaggio_fig_4.png}
	\caption{Hatching rate of lesser kestrel as a function of nest type and nest orientation (GLMM, N = 45)}
	\label{DiMaggio_fig_4}
\end{figure}

\begin{figure}[!h]
\centering
	\includegraphics[width=.95\columnwidth]{DiMaggio_fig_5.png}
	\caption{Fledging rate of lesser kestrel as a function of nest type and nest orientation (GLMM, N = 45)}
	\label{DiMaggio_fig_5}
\end{figure}

\vspace{1cm}\null

\begin{table}[!h]
\centering
\begin{tabular}{>{\raggedright\arraybackslash}p{.2\columnwidth}>{\raggedleft\arraybackslash}p{.2\columnwidth}>{\raggedleft\arraybackslash}p{.2\columnwidth}>{\raggedleft\arraybackslash}p{.2\columnwidth}}
\toprule
\textbf{Explanatory variable} & \textbf{Parameter estimate} & \textbf{F} & \textbf{P} \\
\toprule
%\showrowcolors
Intercept & 20.588 & & 1.000  \\
Colony location & 0.483 & 1.628 & 0.202 \\
Nest type & 1.242 & 612.208 & \textbf{< 0.001} \\
Nest orientation & 0.073 & 3.614 & 0.570 \\
\bottomrule
\hiderowcolors
\end{tabular}
\caption{Effects of colony location, nest type and nest orientation on temperature ($^\circ$C) measured inside lesser kestrel nests (GLMM, N = 45). In bold variables that significantly predicted nest temperatures}
\label{DiMaggio_tab_1}
\end{table}

\vspace{1cm}\null

\begin{table}[!h]
\centering
\begin{tabular}{>{\raggedright\arraybackslash}p{.2\columnwidth}>{\raggedleft\arraybackslash}p{.2\columnwidth}>{\raggedleft\arraybackslash}p{.2\columnwidth}>{\raggedleft\arraybackslash}p{.2\columnwidth}}
\toprule
\textbf{Explanatory variable} & \textbf{Parameter estimate} & \textbf{F} & \textbf{P} \\
\toprule
%\showrowcolors
Intercept & 51.922 & & 0.999 \\
Colony location & -0.892 & 0.445 & 0.505 \\
Nest type & -2.557 & 293.127 & \textbf{< 0.001} \\
Nest orientation & 6.366 & 30.905 & \textbf{< 0.001} \\
\hiderowcolors
\bottomrule
\end{tabular}
\caption{Effects of colony location, nest type and nest orientation on relative humidity (\%) measured inside lesser kestrel nests (GLMM, N = 45). In bold: variables that significantly predicted relative humidity}
\label{DiMaggio_tab_2}
\end{table}

\newpage
\section*{Acknowledgements}

We greatly thank Stefano Triolo, Jo\"elle Tysseire and Laura Zanca, for
assistance during field work. This project was supported by the Italian
Ministry of Education, University and Research (PRIN 2010/2011, 20108
TZKHC).


\section*{Bibliography}
\begin{itemize}\itemsep0pt
	\item Catry I., Franco A. M. A. \& Sutherland W. J., 2011 - Adapting
conservation efforts to face climate change: Modifying nest-site
provisioning for Lesser Kestrels.\textit{ Biol. Cons.}, 144 (3):
1111-1119.

	\item Di Maggio R., Campobello D. \& Sar\`a M., 2013 - Nest aggregation and
reproductive synchrony promote Lesser Kestrel \textit{falco naumanni}
seasonal fitness.\textit{ J. of Ornithol.,} 154: 901--910.

\item Di Maggio R., Mengoni C., Mucci N., Campobello D., Randi E. \& Sar\`a
M., 2014 - Do not disturb the family: roles of colony size and human
disturbance in the genetic structure of Lesser Kestrel. \textit{J.
Zool., }In press.

	\item I\~nigo A. \& Barov B., 2011 - \textit{Action plan for the Lesser
Kestrel Falco naumanni in the European Union}. SEO-BirdLife \& BirdLife
International for the European Commission.

	\item Lloyd J. D. \& Martin T. E., 2004 - Nest-site preference and maternal
effects on offspring growth.\textit{ Behav. Ecol., 15 }(5): 816-823.

	\item {Martin T. E., 1998 - Are microhabitat preferences of
coexisting species under selection and adaptive?
}\textit{{Ecology}}{, 79: 656--670.}

	\item McCullagh P. \& Searle S. R., 2000 - \textit{Generalized Linear and
Mixed Models.} Wiley-Interscience, New York.

	\item Robert M., Vaillancourt M. A. \& Drapeau P., 2010 - Characteristics of
nest cavities of Barrow{\textquoteright}s Goldeneyes in eastern Canada.
\textit{J. Field Ornithol.,} 81: 287--293.

	\item Sar\`a M., 2010 - Climate and land-use changes as determinants of Lesser
Kestrel \textit{falco naumanni} abundance in mediterranean cereal
steppes (Sicily).\textit{ Ardeola, }57(SPEC. DECEMBER): 3-22.

	\item Sar\`a M., Campobello D. \& Zanca L., 2012 - Effects of nest and colony
features on Lesser Kestrel (\textit{falco naumanni}) reproductive
success.\textit{ Avian Biol. Res., }5 (4): 209-217.
\end{itemize}

\setcounter{figure}{0}
\setcounter{table}{0}


\begin{adjustwidth}{-3.5cm}{-1cm}
\pagestyle{CIOpage}
\authortoc{\textsc{Diana F.}, \textsc{Pedrotti L.}, 
\textsc{Sartirana F.}, \textsc{Trotti P.},
\textsc{Galli L.}, \textsc{Bassi E.}}
\chapter*[Cure parentali nell{\textquoteright}aquila reale e nel
gipeto]{Cure parentali nell{\textquoteright}aquila reale
\textbf{\textit{Aquila chrysaetos}}\textbf{ e nel gipeto
}\textbf{\textit{Gypaetus barbatus}} \textbf{in una popolazione delle Alpi italiane}}
\addcontentsline{toc}{chapter}{Cure parentali nell{\textquoteright}aquila reale e nel
gipeto}
\end{adjustwidth}
\begin{adjustwidth}{-3.5cm}{0cm}
\textsc{Francesca Diana}$^{1*}$, \textsc{Luca Pedrotti}$^{1}$, 
\textsc{Fabiano Sartirana}$^{1}$, \textsc{Paolo Trotti}$^{1}$,
\textsc{Loris Galli}$^{2}$, \\\textsc{Enrico Bassi}$^{1**}$\\
\index{Diana Francesca} \index{Pedrotti Luca} \index{Sartirana Fabiano} \index{Trotti Paolo} \index{Galli Loris} \index{Bassi Enrico}
\noindent\color{MUSEBLUE}\rule{27cm}{2pt}
\vspace{1cm}
\end{adjustwidth}



\marginnote{\raggedright $^1$Consorzio del Parco Nazionale dello Stelvio Via De
Simoni 42, 23032 Bormio (SO) \\
$^2$Universit\`a degli Studi di Genova Corso Europa 26,
16132 Genova \\
\vspace{.5cm}
{\emph{\small $^*$Autore per la corrispondenza: \href{mailto:francesca.diana84@yahoo.it}{fran\allowbreak ce\allowbreak sca.\allowbreak dia\allowbreak na\allowbreak 84@\allowbreak ya\allowbreak hoo.\allowbreak it}}} \\
{\emph{\small $^**$Autore per la corrispondenza: \href{mailto:enrico.bassi76@gmail.com}{en\allowbreak ri\allowbreak co.\allowbreak bas\allowbreak si76@\allowbreak g\allowbreak ma\allowbreak il.\allowbreak com}}} \\
\keywords{Alpi, \textit{Aquila chrysaetos}, \textit{Gypaetus
barbatus}, biologia riproduttiva, cure parentali, comportamento}
{Alps, \textit{Aquila chrysaetos}, \textit{Gypaetus
barbatus}, breeding biology, parental care, behaviour}
%\index{keywords}{Alpi} \index{keywords}{\textit{Aquila chrysaetos}} \index{keywords}{\textit{Gypaetus
%barbatus}} \index{keywords}{Biologia riproduttiva} \index{keywords}{Cure parentali} \index{keywords}{Comportamento}
}
{\small
\noindent \textsc{\color{MUSEBLUE} Summary} / A field research on golden eagle \textit{Aquila chrysaetos} and bearded
vulture \textit{Gypaetus barbatus}
was performed during 2008-2011 in Stelvio National Park (central Alps,
Northern Italy), focusing on time budget of breeding pairs of both
species. The study area hosts a breeding population of 14 pairs of
golden eagle and 4 pairs of bearded vulture, nesting at short distance.
During 4 field seasons, 16 breeding events of golden eagle and 11 of
bearded vulture have been monitored for 289 observation days and
behaviour patterns of nesting adults of both species were recorded,
analysed and compared. Data analysis included the application of linear
mixed models. Results were compared between species and with other
study areas. Time dedicated to parental cares significantly differed
between sexes and species. In both species females spent more time than
males in parental cares activities, but golden eagle females dedicated
65\% of time in parental cares and males 33\%, while bearded vulture
females 54\% and males 38\%. Differences between species might be
linked to different feeding behaviour, use of the territory and
climatic conditions occurred in the respective breeding period.  \\
\noindent \textsc{\color{MUSEBLUE} Riassunto} / Nel quadriennio 2008-2011, nell{\textquoteright}area del Parco Nazionale
dello Stelvio e del suo intorno (Alpi centrali italiane), sono stati
condotti studi approfonditi inerenti le cure parentali delle coppie
riproduttive di aquila reale e gipeto. L{\textquoteright}area di studio
ospita 4 coppie di gipeto e 14 coppie di aquila reale, nidificanti tra
loro a breve distanza. Nel corso di 4 stagioni di campo sono state
monitorate 16 nidificazioni di aquila reale e 11 nidificazioni di
gipeto, durante 289 giornate di osservazione.
Nell{\textquoteright}ambito dello studio sono stati registrati,
confrontati e analizzati i comportamenti parentali di entrambe le
specie, utilizzando modelli lineari misti. I risultati sono stati
paragonati con quelli di altre aree di studio e si sono confrontati i
comportamenti delle due specie. Il tempo dedicato alle cure parentali
differisce significativamente tra le due specie e tra maschio e
femmina. In entrambe le specie la femmina dedica pi\`u tempo alle cure
parentali rispetto al maschio, ma le femmine di aquila reale utilizzano
il 65\% del tempo in tali attivit\`a e i maschi il 33\%, mentre nel
gipeto le femmine dedicano il 54\% del tempo alla prole rispetto al
38\% dei maschi. Le differenze tra le specie potrebbero essere connesse
alle differenti modalit\`a di ricerca del cibo, all{\textquoteright}uso
del territorio e alle rispettive condizioni climatiche registrate
durante il periodo riproduttivo. \\
}
\section*{Introduzione}

Sui rapaci, una grande mole di studi disponibili in letteratura riguarda
la densit\`a, l{\textquoteright}uso dell{\textquoteright}habitat, la
dieta e il successo riproduttivo mentre ricerche per
l{\textquoteright}analisi e la quantificazione del comportamento
parentale sono assai rare sia per la difficolt\`a di indagine sia per
l{\textquoteright}elevato sforzo di campo richiesto (Collopy 1984;
Margalida \& Bertran 2000).

Dal 2004, nel Parco Nazionale dello Stelvio (PNS) e nel suo intorno, \`e
in corso un monitoraggio intensivo delle popolazioni di aquila reale
\textit{Aquila chrysaetos} e gipeto \textit{Gypaetus barbatus. }Oltre
al monitoraggio ordinario delle coppie nidificanti, tra il 2008 e il
2011, si \`e inoltre impostato un programma di ricerca mirato ad
approfondire alcuni aspetti del comportamento e della biologia
riproduttiva delle due specie. In particolare, la presente ricerca ha
permesso di raccogliere informazioni sulle attivit\`a parentali degli
adulti impegnati nel ciclo riproduttivo, come i tempi di cova e la
frequenza dei cambi al nido, in rapporto con il sesso del genitore, con
le fasi del ciclo riproduttivo (cova, periodo di post-schiusa e
pre-involo) e con la fascia oraria. Informazioni di dettaglio, non
considerate nel presente lavoro, sono state raccolte anche
sull{\textquoteright}alimentazione, lo sviluppo dei giovani e
l{\textquoteright}influenza dell{\textquoteright}attivit\`a
antagonistica tra le due specie sul comportamento parentale e sulla
produttivit\`a (Bassi \textit{et al}., \textit{in stampa}). Per quanto
riguarda il gipeto i dati raccolti sono, al momento, gli unici
disponibili a livello dell{\textquoteright}arco alpino.

\section*{Area di studio}

L{\textquoteright}area di studio \`e localizzata nelle Alpi centrali
italiane e comprende il settore lombardo del Parco Nazionale dello
Stelvio (60.126 ha) in alta Valtellina (SO) e alta val Camonica (BS) e
alcune valli laterali poste nell{\textquoteright}intorno del Parco.
L{\textquoteright}area, caratterizzata dall{\textquoteright}abbondanza
di estese pareti rocciose calcaree e metamorfiche, boschi di conifere,
prati-pascoli e praterie alpine d{\textquoteright}alta quota, ospita
elevate densit\`a medie di ungulati selvatici (6,7
camosci/km\textsuperscript{2}, 5-25 cervi/km\textsuperscript{2} e 1150
stambecchi; Carro \& Pedrotti 2010). Siti di alimentazione artificiale
non sono mai stati allestiti. Nell{\textquoteright}area di studio sono
presenti 14 territori di aquila reale e 4 di gipeto (Bassi 2011).

\section*{Metodi}

Il comportamento degli adulti di aquila reale e gipeto impegnati nelle
cure parentali \`e stato studiato per 4 stagioni riproduttive
(2008-2011) nel corso delle seguenti tre fasi: cova (aquila reale 42
giorni; gipeto 53-55 giorni), post-schiusa (aquila reale 21 giorni;
gipeto 28 giorni) e pre-involo (dalla fine della fase di post-schiusa
fino all{\textquoteright}involo). 

I dati sono stati raccolti tramite osservazione diretta (\textit{focal
sampling}) dei nidi attivi in ciascuna stagione riproduttiva
utilizzando binocoli 10-12 x e cannocchiali 20-60 ingrandimenti. 

In generale, in etologia, il metodo del campionamento focale
(\textit{focal sampling} o campionamento dell{\textquoteright}animale
focale) consiste nell{\textquoteright}osservazione di un solo individuo
per un periodo di tempo stabilito (Martin \& Bateson 1986), durante il
quale vengono annotate tutte le sue azioni (\textit{time budget}).
Nell{\textquoteright}ambito della presente ricerca sono stati
utilizzati come animali focali entrambi gli adulti e i nidiacei di
ciascuna delle coppie riproduttive, osservabili da punti fissi di
osservazione; i comportamenti registrati sono stati standardizzati e
archiviati tramite la compilazione di schede di rilevamento. 

I punti fissi di osservazione erano posti a distanza compresa tra i 300
e i 2250 m dai nidi occupati di aquila reale (media= 1.071 m {\textpm}
677) e tra i 1000 e i 2510 m dai nidi usati dal gipeto (media= 1.251 m
{\textpm} 430), in modo da non arrecare disturbo alla nidificazione.
Maschio e femmina sono stati distinti in base a criteri morfologici,
comportamentali e ai segni di muta. 

Per l{\textquoteright}osservazione dei nidi (16 di aquila reale e 11 di
gipeto) sono stati complessivamente spesi 289 giorni di campo; 145 per
l{\textquoteright}aquila reale (1132 h) e 144 per il gipeto (1133 h),
con una durata media delle osservazioni pari a 7,8 h. 

Le variabili comportamentali registrate durante tale periodo sono state
suddivise in {\textquotedblleft}discontinue{\textquotedblright} e
{\textquotedblleft}continue{\textquotedblright}. Le attivit\`a
discontinue sono \textit{pattern} comportamentali di durata
relativamente breve che possono essere approssimati come punti nella
linea temporale, la cui caratteristica saliente \`e la frequenza,
espressa come numero di eventi per unit\`a di tempo. Le attivit\`a
discontinue analizzate, espresse come frequenza giornaliera (numero per
giornata), sono stati i cambi al nido e il trasporto di cibo al nido.
Sono state invece considerate come attivit\`a continue quelle azioni la
cui caratteristica principale \`e la durata del singolo
\textit{pattern} comportamentale, espressa in unit\`a di tempo
(minuti). 

Le principali attivit\`a continue registrate durante le osservazioni
oggetto di analisi sono state: cova delle uova, riscaldamento del
nidiaceo e sua alimentazione (tempo dedicato alla preparazione del
cibo, all{\textquoteright}imbeccata e all{\textquoteright}alimentazione
attiva da parte del giovane), sorveglianza (controllo del nidiaceo da
parte dell{\textquoteright}adulto, senza comportamenti di riscaldamento
o alimentazione) e l{\textquoteright}insieme delle cure parentali
(somma di tutti i precedenti comportamenti riferiti agli adulti).

Per il confronto della significativit\`a statistica delle eventuali
differenze nel \textit{time budget} (espresso come percentuale di
attivit\`a dedicate alle cure parentali) nelle due specie, e in
relazione al sesso e alla fase di sviluppo del giovane, sono stati
utilizzati modelli lineari misti. 

\section*{Risultati e discussione}

Nelle coppie seguite di aquila reale il periodo compreso tra la
deposizione e l{\textquoteright}involo \`e durato in media 117 giorni
(\textit{range} 111-129 giorni), mentre in quelle di gipeto \`e
risultato di durata significativamente maggiore con una media di 176
giorni (\textit{range} 162-193 giorni).

Nell{\textquoteright}aquila reale si sono registrati 126 cambi al nido
con una media di 2,5/giorno nella fase di cova (N= 87) e di 1,1 nella
fase di post schiusa (N= 39). 

Per il gipeto sono stati osservati 166 cambi al nido (media: 1,2
cambi/giorno), con una media pi\`u alta, pari a 1,6, nelle fasi di cova
(N= 69) e post schiusa (N= 36). 

Il numero di episodi di trasporto prede al nido per giornata da parte
dell{\textquoteright}aquila reale \`e stato pari a 0,4 nel periodo di
post schiusa e 0,5 in fase di pre-involo (N= 56); 31 di questi, si
riferiscono alla femmina (55\%), 2 ad adulti indeterminati (4\%) e 23
al maschio (41\%). 

Tali dati risultano sensibilmente inferiori a quanto riportato per
l{\textquoteright}Idaho (USA) da Collopy (1984), che indica una media
di 1,2 prede/giorno nel periodo di cova, 1,5 dalla schiusa alla quinta
settimana del giovane, 2,6 tra la sesta e l{\textquoteright}ottava
settimana e 1,6 nelle ultime settimane prima
dell{\textquoteright}involo. 

Nel gipeto entrambi i membri della coppia hanno contribuito al trasporto
di cibo al nido: la femmina nel 50\% dei casi osservati (N= 18) e il
maschio 44,4\% (N= 16), gli adulti indeterminati sono stati solo 2.

Dalla deposizione all{\textquoteright}involo, gli adulti di entrambe le
specie mostrano un comportamento parentale simile, per quanto riguarda
il tempo totale dedicato alla prole. Durante la fase di cova e nei
primi giorni successivi alla schiusa, l{\textquoteright}attivit\`a di
riscaldamento \`e costante mentre, con il progredire dello sviluppo del
pulcino, si osserva una graduale diminuzione, fino alla definitiva
cessazione nel periodo che precede l{\textquoteright}involo. Il tempo
percentuale medio dedicato alla cova, rispetto al tempo totale di
osservazione compreso tra la deposizione e l{\textquoteright}involo,
risulta significativamente differente tra le due specie (F= 4.00; p=
0.0461); il gipeto (stima {\textpm} e.s.= 0,33 {\textpm} 0,015) investe
pi\`u tempo in tale comportamento rispetto all{\textquoteright}aquila
reale (stima {\textpm} e.s.= 0,29 {\textpm} 0,014).

Nell{\textquoteright}aquila reale, nel periodo compreso tra la schiusa e
l{\textquoteright}involo, il pulcino \`e stato nutrito dalla femmina
per il 76,1\% del tempo (con un tempo medio di imbeccata o di
preparazione del cibo di 19 minuti); dal maschio per il 15,7\% (tempo
medio 14 minuti) e da un adulto indeterminato per
l{\textquoteright}8,2\%. La prima osservazione di giovane che si nutre
da solo \`e avvenuta dopo 40-51 giorni dalla schiusa, sulla media dei
nidi osservati.

Nel gipeto, nei primi giorni dopo la schiusa, il pulcino viene nutrito
dall{\textquoteright}adulto per un tempo massimo consecutivo di 70
minuti. Durante la fase di post schiusa il pulcino \`e stato nutrito
dalla femmina per il 75,5\% e dal maschio per il 24,5\%. 

Nella fase di pre-involo il pulcino \`e stato nutrito dalla femmina per
il 65,8\% e dal maschio per il 34,2\%. Nel gipeto, la prima
osservazione di un giovane che si nutre autonomamente \`e stata
registrata a 66 giorni di vita.

In entrambe le specie, sul totale delle osservazioni, le femmine (ff),
dalla cova fino all{\textquoteright}involo, investono pi\`u tempo nelle
cure parentali (intese come comprensive di cova delle uova,
riscaldamento, protezione da predatori e alimentazione del nidiaceo)
rispetto ai maschi (mm), (stima percentuale di tempo su tempo totale
{\textpm} e.s. ff = 0,476 {\textpm} 0,013; stima percentuale di tempo
su tempo totale {\textpm} e.s. mm = 0,304 {\textpm} 0,013; F= 90,97;
p{\textless} 0,001). 

Tenendo conto del totale del tempo che \`e stato dedicato alle cure
parentali da parte delle due specie durante le osservazioni, per
ciascuna specie, le percentuali suddivise per sesso sono risultate le
seguenti: aquila reale: ff 65,1, mm 32.8, indet. 2,1; gipeto: ff 54,4,
mm 37,8, indet. 7,8.

Analogamente le femmine di aquila reale sono risultate pi\`u dedite alle
cure parentali rispetto ai maschi, nelle prime tre settimane di vita
del pulcino (ff 71,6\%, mm 23,7\%, indet. 4,7\%) e, in maniera quasi
esclusiva, nella fase di preinvolo (ff 95,1\%, mm 4,9\%). 

Analoghi risultati sono stati ottenuti sulle Alpi svizzere tramite
l{\textquoteright}osservazione di 19 nidificazioni di aquila reale (per
un totale di 583 ore spese sul campo), in cui la femmina occupava il
nido per il 70,8\% {\textpm} 14,9\% del tempo giornaliero mentre il
maschio per il 18,4 {\textpm} 15,5\% (Jenny 1992).

Nel gipeto invece il contributo del maschio alle cure parentali
(espresso come percentuale di tempo dedicato alle cure parentali su
tempo totale di osservazione) \`e percentualmente maggiore rispetto al
maschio di aquila reale (stima {\textpm} e.s. aquila reale 0,192
{\textpm} 0,019; stima {\textpm} e.s. gipeto 0,273 {\textpm} 0,018; F=
5,04; p= 0,025). 

Trattando i dati in maniera analoga all{\textquoteright}unico studio
disponibile sul gipeto per l{\textquoteright}Europa (Pirenei spagnoli,
Margalida \& Bertran 2000), in cui entrambi i sessi contribuivano in
egual misura alle cure parentali (ff 52 {\textpm} 6,6\%; mm 48
{\textpm} 6,6\%; t\textsubscript{6}= -0,64; ns), nella presente
indagine il tempo dedicato alle cure parentali \`e risultato invece
pi\`u sbilanciato verso le femmine (61 {\textpm} 8,4\%) rispetto ai
maschi (39\% {\textpm} 8,4\%).

Una possibile interpretazione della differenza tra le due aree di studio
deriva dal fatto che, nel Parco Nazionale dello Stelvio, le temperature
medie di gennaio (mese che coincide con l{\textquoteright}inizio della
cova del gipeto in tale area di studio) sono notevolmente inferiori
rispetto a quelle pirenaiche (media delle minime: Alpi -12,2 vs Pirenei
-5{\textdegree}C). Le maggiori dimensioni della femmina potrebbero
infatti garantire un pi\`u efficace isolamento termico;
l{\textquoteright}influenza del clima sembra avvalorata anche dal fatto
che nel Parco Nazionale dello Stelvio, nell{\textquoteright}ambito del
presente studio, non sono mai state osservate interruzioni del
riscaldamento di uova e nidiacei. 

Nel gipeto i tempi di cova, che comunque risultano pi\`u equamente
distribuiti tra i due sessi rispetto a quanto avviene
nell{\textquoteright}aquila reale, possono essere una diretta
conseguenza delle modalit\`a di ricerca del cibo la cui localizzazione
risulta di pi\`u difficile previsione (Margalida \& Bertran 2000).
Infatti gli adulti di gipeto per individuare carcasse e singole ossa
disperse nel territorio perlustrano aree molto estese (nel Parco
Nazionale dello Stelvio la stima dell{\textquoteright}\textit{home
range} di ciascuna coppia di gipeto \`e pari a 300-500
km\textsuperscript{2}, Bassi \textit{ined}.), investendo un tempo
maggiore nella ricerca di cibo rispetto all{\textquoteright}aquila
reale, che utilizza territori di caccia generalmente posti a distanze
inferiori dai propri siti di nidificazione. Tale spiegazione
giustificherebbe anche il motivo per cui il numero di cambi al nido per
giornata sia inferiore nel gipeto rispetto a quanto registrato per
l{\textquoteright}aquila reale. 

\section*{Ringraziamenti}

Si ringraziano il Consorzio del Parco Nazionale dello Stelvio, il Corpo
Forestale dello Stato - C.T.A. di Bormio e la Societ\`a Tersia s.r.l.
di Savona per aver contribuito al finanziamento della ricerca nel corso
di tre anni consecutivi. Un particolare ringraziamento a Heinrich
Haller, Giuseppe Bogliani e David Jenny per i preziosi suggerimenti e
consigli nell{\textquoteright}impostazione della ricerca.

\section*{Bibliografia}
\begin{itemize}\itemsep0pt
	\item Bassi E., 2011 - Sintesi del censimento contemporaneo di aquila reale
(\textit{Aquila chrysaetos}) e gipeto (\textit{Gypaetus barbatus})
nell{\textquoteright}ambito dei progetti di monitoraggio delle
popolazioni nidificanti nel settore lombardo e trentino del Parco
Nazionale dello Stelvio. Anni 2004-2011. Parco Nazionale dello Stelvio.
Relazione interna, 46 pp.

	\item Bassi E., Diana F., Sartirana F., Trotti P., Galli L. \& Pedrotti P. 2015. - Analisi del successo riproduttivo
dell{\textquoteright}aquila reale (\textit{Aquila chrysaetos}) nel
Parco Nazionale dello Stelvio in relazione al ritorno del gipeto
(\textit{Gypaetus barbatus}) sulle Alpi. \textit{Pedrini P., Rossi F., Bogliani G., Serra L. \& Sustersic A. (a cura di) 2015. \emph{XVII Convegno Italiano di Ornitologia: Atti del convegno di Trento}. Ed. MUSE, 174 pp.}.

	\item Carro M. \& Pedrotti L. (a cura di), 2010 - Atlante del Parco Nazionale
dello Stelvio.

	\item Collopy M., 1984 - Parental care and feeding ecology of Golden eagle
nestlings. \textit{Auk}, 101:753-760.

	\item Margalida A. \& Bertran J., 2000 - Breeding behaviour of the Bearded
Vulture \textit{Gypaetus barbatus}: minimal sexual differences in
parental activities. \textit{Ibis,} 142: 225-234.

	\item Martin P. \& Bateson P., 1986 - \textit{Measuring Behaviour: an
introductory guide}. Cambridge University Press, Cambridge.

	\item Watson J., 1997 - \textit{The Golden eagle}. T\&D Poyser, London.
\end{itemize}

\begin{otherlanguage}{english}
\setcounter{figure}{0}
\setcounter{table}{0}

\begin{adjustwidth}{-3.5cm}{0cm}
\pagestyle{CIOpage}
\authortoc{\textsc{Gagliardi A.}, \textsc{Casola D.}, 
\textsc{Preantoni D.}, \textsc{Wauters L.}, 
\textsc{Martinoli A.}, \textsc{Fasola M.}}
\chapter*[Breeding interactions between great cormorants and grey
herons]{Interference between great cormorants \textbf{\textit{Phalacrocorax carbo}}\textbf{ and}\textbf{ herons
}\textbf{\textit{Ardea cinerea}}\textbf{ breeding in syntopy
}\textbf{in NW Italy}}
\addcontentsline{toc}{chapter}{Breeding interactions between great cormorants and grey
herons}

\textsc{Alessandra Gagliardi}$^{1*}$, \textsc{Daniela Casola}$^{2}$, 
\textsc{Damiano Preantoni}$^{2}$, \textsc{Lucas Wauters}$^{2}$, 
\textsc{Adriano Martinoli}$^{2}$, \textsc{Mauro Fasola}$^{3}$\\

\index{Gagliardi Alessandra} \index{Casola Daniela} \index{Preantoni Damiano} \index{Wauters Lucas} \index{Martinoli Adriano} \index{Fasola Mauro}
\noindent\color{MUSEBLUE}\rule{27cm}{2pt}
\vspace{1cm}
\end{adjustwidth}



\marginnote{\raggedright $^1$Istituto-Oikos, via Crescenzago 1 - 20134
Milano \\
$^2$Dipartimento di Scienze Teoriche e Applicate,
Universit\`a degli Studi dell{\textquotesingle}Insubria, via Dunant 3 -
21100 Varese \\
$^3$Dipartimento Scienze Terra Ambiente - Universit\`a degli Studi di
Pavia, via Ferrata 1 -- 27100 Pavia \\
\vspace{.5cm}
{\emph{\small $^*$Autore per la corrispondenza: \href{mailto:alessandra.gagliardi@uninsubria.it}{a\allowbreak les\allowbreak san\allowbreak dra.\allowbreak ga\allowbreak gliar\allowbreak di@\allowbreak u\allowbreak nin\allowbreak su\allowbreak bria.\allowbreak it}}} \\
\keywords{\textit{Phalacrocorax carbo}, \textit{Ardea cinerea},
siti di nidificazione, interazioni fra specie, andamento di
popolazione, Italia nord-occidentale}
{\textit{Phalacrocorax carbo}, \textit{Ardea cinerea}, nesting
site, interactions between species, population trend, north-west Italy}
%\index{keywords}{\textit{Phalacrocorax carbo}} \index{keywords}{\textit{Ardea cinerea}}
%\index{keywords}{Siti di nidificazione} \index{keywords}{Interazioni fra specie} \index{keywords}{Andamento di
%popolazione} \index{keywords}{Italia nord-occidentale}
}
{\small
\noindent \textsc{\color{MUSEBLUE} Summary} / Most of the colonies of great cormorant in Italy are located in sites
already used by breeding grey herons and other colonial
\textit{Ardeidae}. We analyzed data from long term monitoring programs
of breeding colonial herons and great cormorants (performed by the
research group {\textquotedblleft}Garzaie Italia{\textquotedblright},
and the Cormorant Colony Count Group) and field data collected in a
sample of colonies where the two species breed in synthopy, in order to
check any potential pattern of interference. In detail, we investigated
the long-term trend in the number of grey heron nests in the colonies
with and without cormorants, the spatial overlap of the nest
distribution among species and the vertical distribution of the nests.
The arrival of the great cormorant seems to have two effects on
co-breeding grey herons: a gradual diminution of the number of grey
heron nests and a gradual spatial and vertical segregation between the
two species inside the same colony.  \\
\noindent \textsc{\color{MUSEBLUE} Riassunto} / Molte delle colonie di cormorano \textit{Phalacrocorax carbo} in Italia
sono insediate in aree occupate anche da airone cenerino \textit{Ardea
cinerea}. In questo lavoro analizziamo in modo congiunto i dati
raccolti nell{\textquoteright}ambito del programma di monitoraggio
delle colonie di \textit{Ardeidae} coloniali in Italia nord-occidentale
e i dati rilevati in colonie ove le due specie sono compresenti per
verificare se la presenza del cormorano influenza
l{\textquotesingle}andamento delle popolazioni di airone cenerino e la
distribuzione spaziale dei nidi di questa specie
all{\textquoteright}interno delle colonie. Abbiamo inoltre rilevato il
posizionamento dei nidi di cormorano e airone cenerino come
sovrapposizione topografica e come tipo di alberi e altezze, in sei
colonie. L{\textquoteright}arrivo del cormorano sembra aver causato una
graduale diminuzione dell{\textquoteright}airone cenerino nelle colonie
miste, e una graduale segregazione orizzontale e verticale nella
localizzazione dei nidi delle due specie. Queste interferenze si sono
manifestate nonostante l{\textquoteright}apparente diponibilit\`a di
ampie superfici di ambiente idoneo alla nidificazione, e non occupate.
Abbiamo in programma di estendere i rilevamenti a tutte le colonie di
cormorano di Piemonte e Lombardia, al fine di confermate queste
interferenze tra specie coloniali. \\
}



\section*{Introduction}
The mainland breeding population of great cormorant in Italy has
increased rapidly since early 1990s, and the breeding population for
whole Italy in 2013 was estimated at around 4000 occupied nests in 48
colonies (Volponi \& CorMoNet.it, 2013), mainly located in the north
and particularly in Lombardy, where 10 colonies had 1100 breeding pairs
in 2013. Most colonies in Northwestern Italy are located in protected
areas (natural reserves, regional or natural parks, sites of the Natura
2000 network), and frequently in sites that had been already used for a
long time by breeding grey herons \textit{Ardea cinerea} and by other
colonial \textit{Ardeidae}. In NW Italy (Lombardy and Piedmont), 21
cormorant colonies out of 25 are located in heronries, of which 5 with
grey herons only and 16 plurispecific with 2 to 6 species of herons and
egrets, in a landscape strongly urbanized and exploited for
agricultural activities.

Since the availability of sites suitable for colonially breeding water
birds, i.e. sites wetlands safe from human disturbance, is limited
(Fasola \& Alieri, 1992), we predicted possible negative interactions
between great cormorants and grey herons, since these species tend to
use the same forest strata for breeding. In this study we investigate
whether the recent establishment of great cormorant exhibits any
pattern of interference, that could affect the settlement of the herons
and the cormorants in their colony sites. In detail, we aimed to check:
(i) the long-term trend in the number of grey heron nests in the
colonies without vs. those with cormorants; (ii) the spatial overlap of
the nest distribution among species; (iii) the vertical distribution of
the nests.

\section*{Methods}
Data from a long term monitoring program (from 1972 to 2012) of the
breeding colonial \textit{Ardeidae} throughout Northwestern Italy
(performed by the research group {\textquotedblleft}Garzaie
Italia{\textquotedblright}, see Fasola et al., 2011), and from nest
counts of great cormorant in Italy (Volponi \& the Cormorant Colony
Count Group, 2011) were used for time series and cross correlation
analysis, using the R environment (R Core Team, 2013). The time series
analysis was performed on the number of nests in the heronries within
the breeding range of the cormorant in NW Italy (160 sites with grey
herons, and 22 colonies with cormorants, from 1972 to 2012). The time
series data of colonies with both cormorants and grey herons were
aligned before cross correlation analysis, (for each colony the year of
great cormorant establishment was set as year zero).

Between early April and late July 2013, we explored the potential
effects of great cormorant establishment on the spatial distribution of
the herons breeding in six colonies that differed in the year of
settlement of cormorants (from recent establishment to long-time
presence of the species). All trees with nests were located and mapped
using GPS. Each occupied nest on each tree was assigned to the
cormorant, the grey heron or to other \textit{Ardeidae} species, by
identifying the occupants (nestlings or adults), or when they were not
in sight by nest shape and size. The overlap between cormorants and
grey herons in each colony was estimated by producing a 5 m buffer
around each tree occupied by the two species, and by calculating the
surface area of the intersection*s between these buffers for each
species, weighted by the total surface area of the colony. The vertical
distribution of the nests of the two species was investigated, using a
random sample of at least 30 trees in each of 6 colonies. Nest height
and tree height were measured using a hypsometer. Data were analyzed by
ANOVA, testing the effects of species, site and the species/site
interaction.

\section*{Results and discussion}
The time series analysis revealed a marked increase in the total number
of great cormorant nests since first breeding, as well as a continuing
positive trend of the grey heron in the colonies without cormorants. In
contrast to this positive trend, the number of grey herons nesting in
colonies with cormorants showed a sharp decreasing trend (Fig. \ref{Gagliardi_fig_1}). The
shift in the grey heron trend was concomitant with the increase of the
number of cormorant breeding pairs. This is confirmed by the results of
cross-correlation analysis of time series data: the presence of
cormorants affected grey herons (as shown by the cross correlation
coefficient exceeding the significant threshold r = 0.4) four years
after cormorant settled in the same colonies. The decreasing trend of
the grey herons, noticeable since about 2005 even in the colonies
without great cormorants, reflects a recent generalized decline of the
species throughout the entire NW Italy (unpublished results of the
{\textquotedblleft}Garzaie-Italia{\textquotedblright} research group).

The analysis of the spatial overlap in the areas used by cormorants and
by grey herons (Tab. \ref{Gagliardi_tab_1}) showed high overlap in the colonies of recent
cormorant nesting settlement (e.g. the colony
{\textquotedblleft}Brescia centro{\textquotedblright} where cormorants
first bred in 2011, overlap proportion = 61\%,), while lower overlap
was registered in the colonies where cormorants had been present since
long (e.g. {\textquotedblleft}Zerbaglia{\textquotedblright} where
cormorants bred since 2005, overlap proportion = 20\%). A particular
case is represented by the
{\textquotedblleft}Brabbia{\textquotedblright} colony, where there was
no overlap between the two species in 2013, due to the gradual
dislocation of the heronry that started in 2004, after the first
settlement of cormorant in the area. In all the surveyed colonies, the
longer was the time elapsed since the breeding settlement by
cormorants; the lower was the overlap between the nesting areas of the
two species. The relationship between overlap proportion and years
since the initial cormorant breeding settlement can be modeled by a
hyperbolic regression (Overlap = S/Year, t\textsubscript{(5)} = 4.12; p
= 0.0092), with the S parameter (representing the rate of overlap)
estimated as S = 0,330 {\textless} 0,878 {\textless} 1,426. 

The analysis of nest height of the two species, a two-way ANOVA with
{\textquotedblleft}species{\textquotedblright} and
{\textquotedblleft}site{\textquotedblright} as fixed factors and
{\textquotedblleft}nest tree species{\textquotedblright} as random
factor, did not show any random effect of the tree species. Thus, a
fixed factor ANOVA was carried out, which showed that great cormorant
nests were placed significantly higher (average 13.21 {\textpm} 5.60 m,
N = 439) than those of the grey heron (average 10.78 {\textpm} 6.76 m,
N = 346; F\textsubscript{(4,864) }= 121.27, p {\textless} 0.0001).

The {\textquotedblleft}site{\textquotedblright} variable also has an
effect, i.e. nests of the two species were placed at different heights
on a per-site basis (F\textsubscript{(6,864) }= 460.42, p {\textless}
0.0001). Time elapsed since breeding cormorant arrival seems again to
affect the vertical distribution of the two species (species/site
interaction F\textsubscript{(13,864) }= 15.53, p {\textless} 0.0001),
as resulted for the spatial overlap. In colonies with recent breeding
cormorant settlement, the great cormorant nests were higher than those
of the grey heron, and the differences in nest height between the two
species seem to decrease in the colonies where the two species have
coexisted for a longer period (Fig. \ref{Gagliardi_fig_2}).

\begin{figure}[!h]
\centering
\includegraphics[width=.8\columnwidth]{Gagliardi_fig_1.png}
\caption{Trends 1992-2012 in the total number of nests of great cormorant (squares) and of grey heron in the heronries without cormorants (triangles) and in those with cormorants (dots) within NW Italy. Continuous line: LOESS smoothing (5 year moving window local regression)}
\label{Gagliardi_fig_1}
\end{figure}

\begin{figure}[!h]
\centering
\includegraphics[width=.98\columnwidth]{Gagliardi_fig_2.png}
\caption{Box plots of nest height of Grey Heron and Cormorant, arranged from left to right in order of recent (colonies to the left of the graph) or longer (colonies to the right) colonization by the Great Cormorant}
\label{Gagliardi_fig_2}
\end{figure}

\begin{table}[!h]
\centering
\begin{tabular}{>{\raggedright\arraybackslash}p{.3\columnwidth}>{\raggedright\arraybackslash}p{.25\columnwidth}>{\raggedright\arraybackslash}p{.15\columnwidth}>{\raggedright\arraybackslash}p{.15\columnwidth}}
\toprule
\textbf{Colony name} & \textbf{Coordinates} & \textbf{Proportion of spatial overlap} & \textbf{Year of settlement of the great cormorant} \\
\toprule 
%\showrowcolors
Brescia centro (BS) & 45.507 N, 10.238 E & 61 & 2011 \\
Carpiano (MI) & 45.326 N, 9.243 E & 20 & 2010 \\
Villanterio (PV) & 45.213 N, 9.354 E & 0 & 2009 \\
Zerbaglia (LO) & 45.271 N, 9.642 E & 20 & 2006 \\
Zelata (PV) & 45.242 N, 9.003 E & 11 & 2005 \\
Brabbia (VA) & 45.776 N, 8.713 E & 0 & 2004 \\
\hiderowcolors
\bottomrule
\end{tabular}
\caption{Spatial overlap between the great cormorant and the grey heron, in the surface area occupied by the two species in mixed colonies}
\label{Gagliardi_tab_1}
\end{table}

\section*{Conclusions}


In conclusion, in the six sampled colonies, the arrival of the great
cormorant seems to have two effects on co-breeding grey herons: a
gradual diminution of the number of grey heron nests and a gradual
spatial and altitudinal segregation between the two species inside the
same colony. The Italian breeding population of the great cormorant is
still now only a small fraction of the overall European one (less than
1\%), well under the natural carrying capacity, and if its population
will further increase, we can expect stronger effects on the
distribution and number of breeding grey herons. The interference
between these two species occurs despite an apparent abundance of the
available breeding sites, since the surface area of the colony is only
a small fraction of the apparently suitable woodland, at least in some
colonies e.g. {\textquotedblleft} the
{\textquotedblleft}Zelata{\textquotedblright} colony. We plan to extend
our survey to all the heronries with cormorant in the study area, in
order to confirm these interference patterns.



\section*{Acknowledgements}
The authors would like to thank all the collaborators to the long term
monitoring programs, the {\textquotedblleft}Cormorant Colony
Count{\textquotedblright} and the {\textquotedblleft}Garzaie
Italia{\textquotedblright} research groups.

\section*{Bibliography}
\begin{itemize}\itemsep0pt

	\item Fasola M. \& Alieri R., 1992 - Conservation of heronry sites in North
Italian agricultural landscapes. \textit{Biological Conservation,} 62:
219-228.

	\item Fasola M., Merli E., Boncompagni E. \& Rampa A., 2011 - Monitoring heron
populations in Italy, 1972-2010. \textit{Journal of Heron Biology and
Conservation,} 1 (8): 1-10. \url{www.heronhonservation.org/vol1/art8}

	\item R Core Team, 2013 - R: A language and environment for statistical
computing. R Foundation for Statistical Computing, Vienna, Austria. \\ \url{http://www.R-project.org}

	\item Volponi S. \& CorMoNet.it, 2013 - Status of the breeding population of
Great Cormorants in Italy in 2012. -- In: Bregnballe T., Lynch J.,
Parz-Gollner R., Marion L., Volponi S., Paquet J-Y. \& van Eerden M.R.
(eds.) 2013. \textit{National reports from the 2012 breeding census of
Great Cormorants Phalacrocorax carbo in parts of the Western
Palearctic}. IUCN-Wetlands International Cormorant Research Group
Report. Technical Report from DCE -- Danish Centre for Environment and
Energy, Aarhus University. No. 22: 59-64.
\end{itemize}
\end{otherlanguage}
\setcounter{figure}{0}
\setcounter{table}{0}

\begin{adjustwidth}{-3.5cm}{0cm}
\pagestyle{CIOpage}
\authortoc{\textsc{Masoero G.}, \textsc{Tamietti A.}, 
\textsc{Caprio E.}}
\chapter*[Trend della popolazione di topino nel Parco del Po
torinese]{Trend della popolazione di topino \textbf{\textit{Riparia
riparia }}\textbf{nel Parco del Po e della Collina Torinese}}
\addcontentsline{toc}{chapter}{Trend della popolazione di topino nel Parco del Po
torinese}

\textsc{Giulia Masoero}$^{1*}$, \textsc{Alberto Tamietti}$^{2}$, 
\textsc{Enrico Caprio}$^{1}$\\

\index{Masoero Giulia} \index{Tamietti Alberto} \index{Caprio Enrico}
\noindent\color{MUSEBLUE}\rule{27cm}{2pt}
\vspace{1cm}
\end{adjustwidth}



\marginnote{\raggedright $^1$Dipartimento di Scienze della Vita e Biologia dei
Sistemi, Universit\`a di Torino \\
$^2$Parco del Po e della collina Torinese \\
\vspace{.5cm}
{\emph{\small $^*$Autore per la corrispondenza: \href{mailto:giulia.masoero@gmail.com}{giu\allowbreak lia.\allowbreak ma\allowbreak so\allowbreak e\allowbreak ro@\allowbreak g\allowbreak ma\allowbreak il.\allowbreak com}}} \\
\keywords{\textit{Riparia riparia}, Parco del Po e della Collina
torinese}
{\textit{Riparia riparia}, Po Park, Collina torinese Park}
%\index{keywords}{\textit{Riparia riparia}} \index{keywords}{Parco del Po e della Collina
%torinese}
}
{\small
\noindent \textsc{\color{MUSEBLUE} Summary} / The sand martin \textit{Riparia riparia} is a trans-saharian migrant
passerine. This species is classified as
{\textquotedblleft}vulnerable{\textquotedblright} in the 2011 Red List
of the Birds breeding in Italy and its population are declining in all
Europe. The present study shows the results of a ten years monitoring
on a sand martins breeding population along the southern Po river in
Turin. This research was promoted by the Po and Collina Torinese
Natural Park. The birds were captured and ringed at breeding sites
within the Park and the capture-mark-recapture data collected from 2002
to 2012 were analysed with the software MARK v 6.1 to obtain survival
rates. The annual survival rates were related to two meteoclimatic
indices, the Sahel rainfall index and the North Atlantic Oscillation
(NAO) index. The results showed an average survival rate of 33.8\%. The
annual survival rate resulted related to the two climatic indices
during some of the most important periods of sand martins life; Sahel
rainfall index during wintering and NAO index during breeding and
autumn migration. \\
\noindent \textsc{\color{MUSEBLUE} Riassunto} / Il topino \textit{Riparia riparia} \`e un passeriforme migratore
trans-sahariano classificato come
{\textquotedblleft}vulnerabile{\textquotedblright} nella Lista Rossa
2011 degli Uccelli nidificanti in Italia con popolazioni in declino in
tutta Europa. Il presente lavoro mostra i risultati di uno studio
decennale sulla nidificazione del topino lungo il Po a sud di Torino,
condotto dal Parco del Po e della Collina torinese attraverso
l{\textquoteright}inanellamento a scopo scientifico nelle colonie di
nidificazione. I dati di cattura-marcatura-ricattura raccolti dal 2002
al 2012 sono stati analizzati con il software MARK v. 6.1 al fine di
ottenere i dati relativi al tasso di sopravvivenza. Il tasso di
sopravvivenza annuale ottenuto \`e stato poi messo in relazione con gli
indici meteoclimatici di Oscillazione Nord Atlantica (NAO) e di
piovosit\`a del Sahel. I risultati ottenuti mostrano complessivamente
un tasso di sopravvivenza medio pari al 33.8\%. Il tasso di
sopravvivenza annuale \`e risultato dipendere in maniera significativa
dalla piovosit\`a del Sahel durante lo svernamento e
dall{\textquoteright}indice NAO durante il periodo riproduttivo e la
migrazione autunnale. 
}


\section*{Introduzione}

 Il topino \textit{Riparia riparia} \`e un migratore trans-sahariano
comune in Italia durante la stagione riproduttiva. Studi recenti hanno
dimostrato che il clima dei quartieri di svernamento africani \`e un
fattore di importanza cruciale nel ciclo annuale di un gran numero di
uccelli migratori su lunga distanza. Uno dei fattori pi\`u importanti
che appare collegato con la loro sopravvivenza \`e la piovosit\`a della
regione del Sahel. 

Durante il nostro studio abbiamo analizzato la relazione tra clima e
sopravvivenza durante tutto il ciclo vitale di adulti di topino, con
particolare attenzione ai quattro eventi fondamentali: lo svernamento,
la nidificazione e le migrazioni.

\section*{Area di studio}

L{\textquotesingle}area studiata \`e situata a sud di Torino, compresa
nei comuni di Carignano, Carmagnola, Lombriasco, La Loggia e
Moncalieri. Il territorio ricade nei confini del Parco del Po e della
Collina torinese ed \`e tutelato come parco regionale dal 1990 (L.R.
28/90 e successive modificazioni). 

\`E stato preso in esame un tratto di fiume Po e gli impianti di
estrazione di sabbia e ghiaia (cave) presenti lungo il corso del fiume
stesso. Il tratto in esame presenta una lunghezza lineare totale di 13
km e sono presenti 14 cave, 9 delle quali hanno presentato
nidificazioni di topino nel corso dello studio.

L{\textquoteright}ambiente fluviale \`e caratterizzato da
un{\textquoteright}area di bosco ripariale (saliceto) lungo le sponde
del fiume. Quest{\textquoteright}area, di dimensioni ridotte, separa il
fiume stesso dalla pianura coltivata a mais, cereali vernini, soia e
pioppeti.

L{\textquoteright}ambiente di cava attiva \`e invece costituito da un
lago artificiale le cui sponde sono oggetto di ripristino ambientale
finalizzato a ottenere boschi planiziali, ripariali o fasce arbustive.
Questo occupa la maggior parte area della cava. Una parte di estensione
minore \`e dedicata agli impianti produttivi, con cumuli di sabbia e
pietrisco, nastri trasportatori e piazzali e strade per
l{\textquoteright}accumulo o il trasporto degli inerti.

\section*{Metodi}

Tra il 2002 e il 2012 nei mesi di maggio, giugno e luglio si sono svolte
le operazioni di cattura e inanellamento in 9 cave attive di sabbia e
ghiaia adiacenti al fiume Po. Per l{\textquotesingle}analisi dei dati
di cattura-marcatura-ricattura \`e stato utilizzato il programma MARK,
utilizzando il modello \textit{Cormack-Jolly-Seber} (Cormack 1964;
Jolly 1965; Seber 1965). Una volta marcati gli animali vengono
rilasciati e quindi catturati nuovamente, riconosciuti, e rilasciati.
Ogni incontro \`e caratterizzato da una probabilit\`a di sopravvivenza
fra quello precedente e quello successivo. Questo parametro \`e detto  
$\Phi_{i}\Phi_{i}$. Il parametro $pp$ rappresenta invece la probabilit\`a di ricatturare
l{\textquoteright}animale in quell{\textquoteright}occasione (Cooch \&
White 2012).

Il tasso di sopravvivenza annuale \`e stato messo in relazione con
l{\textquotesingle}indice di piovosit\`a del Sahel e
l{\textquotesingle}indice di oscillazione Nord-Atlantico (NAO)
utilizzando modelli lineari generalizzati univariati (GLM). Come dati
di riferimento per l{\textquoteright}indice NAO sono stati utilizzati
gli indici mensili e stagionali di Hurrell.

\section*{Risultati}

Sono stati catturati 2254 individui adulti, per un totale di 314
individui ricontrollati negli anni successivi. Questi dati sono stati
analizzati con il programma MARK utilizzando come modello di partenza,
un modello in cui tasso di sopravvivenza (\textit{$\Phi $}) e di
ricattura (\textit{p}) variano con il tempo (t) (Cormack 1964; Jolly
1965; Seber 1965). Il test di \textit{Goodness of Fit} (  $\chi_{21}^{2}=$
  12.1793; P = 0.9347) indica che il modello descrive i dati in maniera
efficace. 

I modelli possibili per le variabili di sopravvivenza e di probabilit\`a
sono i quattro modelli per cui esse variano, o meno, in dipendenza dal
tempo (Tab. \ref{Masoero_tab_1}).

Il modello pi\`u probabile \`e quello con il minor valore di AICc e
quindi quello in cui la sopravvivenza rimane costante di anno in anno
mentre cambia la probabilit\`a di cattura. 

Il modello $\Phi\Phi(.)p(t)$ ci consente quindi di fornire una stima della probabilit\`a
di sopravvivenza non dipendente dal tempo. I modelli $\Phi\Phi(t)p(.)$ e
$\Phi\Phi(t)p(t)$ hanno valore di AICc molto simili e quindi confrontabili. Per
il confronto con i dati del clima si \`e scelto il modello$\Phi\Phi(t)p(t)$.
$\Phi\Phi(t)p(t)$ dipendente dal tempo rappresenta la probabilit\`a di sopravvivenza
di anno in anno. 

Il risultato dei GLM effettuati per valutare la relazione tra tasso di
sopravvivenza annuale e indice di piovosit\`a del Sahel di ogni mese ha
individuato una relazione significativa con il mese di ottobre (Beta =
0.00127, SE = 0.0005, p {\textless} 0.05).

I GLM effettuati per valutare la relazione tra tasso di sopravvivenza
annuale e indice NAO stagionale hanno individuato come modello migliore
quello con i mesi di giugno, luglio e agosto (NAO\_JJA) (Beta =
0.13465, SE = 0.04358, p {\textless} 0.05). I GLM tra il tasso di
sopravvivenza annuale e gli indici NAO mensili hanno evidenziato una
relazione significativa con l{\textquoteright}indice NAO del mese di
settembre (Beta = -0.13950, SE = 0.01933, p {\textless} 0.001).

\section*{Discussione}

Dall{\textquoteright}analisi dei dati di inanellamento si ottiene una
probabilit\`a di sopravvivenza del 33.8\%, risultato simile a quanto
rilevato da Cowley e Siriwardena (2005) ( 31\% per i
maschi e del 29\% per le femmine) oppure in quello di Norman e Peach
(2013) 38\% per i maschi e 31\% per le femmine. Il risultato da noi
ottenuto \`e conforme alle stime pubblicate derivanti dai ritrovamenti
di uccelli morti non sessati in Gran Bretagna (35,4\%; Dobson, 1990),
che suggeriscono che il nostro studio non ha eccessivamente
sottostimato la vera sopravvivenza (Norman \& Peach 2013).

Il tasso di sopravvivenza annuale fornisce invece valori che oscillano
di anno in anno, passando da una sopravvivenza minima
del\textcolor{black}{ 10.3\%} nel 2011 a una sopravvivenza massima del
\textcolor{black}{53.7\% nel 2007.}

Le popolazioni europee di topino svernano nel Sahel occidentale (Mead
1979; Robinson \textit{et al.} 2008) e il ruolo di tale area nello
svernamento e nella preparazione al volo trans-sahariano in primavera
\`e essenziale (Sz\'ep 1995). L{\textquoteright}indice di piovosit\`a
del Sahel fornisce una misura indiretta delle condizioni di abbondanza
delle aree umide a cui \`e legata l{\textquoteright}alimentazione del
topino sia in migrazione sia durante lo svernamento (Morel \& Morel
1992). 

I nostri risultati sono in accordo con quanto si riscontra in
bibliografia: il tasso di sopravvivenza annuale, infatti, \`e
influenzato dalla variazione della piovosit\`a del Sahel, in
particolare con la piovosit\`a nel mese di ottobre. Anche in studi
precedenti la piovosit\`a nel Sahel occidentale \`e correlata con le
variazioni annuali della sopravvivenza (Robinson \textit{et al.} 2008).

La relazione positiva tra le variazioni nel tasso di sopravvivenza
annuale e l{\textquoteright}indice NAO dei mesi di giugno, luglio e
agosto potrebbe significare che in quel periodo un indice NAO con
valori positivi favorisce la sopravvivenza dei topini.
Nell{\textquoteright}area del Mediterraneo, valori positivi nella
stagione estiva indicano una maggiore piovosit\`a (Blad\'e \textit{et
al.} 2012). Un clima umido, con pi\`u precipitazioni, potrebbe impedire
l{\textquoteright}eccessivo disseccamento di corsi
d{\textquoteright}acqua e aree umide, fondamentali per la dieta
insettivora dei topini. Nel periodo riproduttivo (giugno e luglio)
un{\textquoteright}adeguata alimentazione \`e necessaria poich\'e
consente all{\textquoteright}individuo di nutrire sufficientemente se
stesso e i piccoli. In agosto \`e necessaria a consentire al topino di
accumulare il grasso che gli fornir\`a le energie necessarie a compiere
la migrazione.

Di particolare interesse \`e la relazione negativa tra la sopravvivenza
e l{\textquoteright}indice NAO di settembre, mese in cui avviene la
migrazione verso i territori di svernamento. Con indice NAO positivo e
quindi maggiori precipitazioni nel Mediterraneo diminuirebbe quindi la
sopravvivenza dei topini. In questo caso non sono stati trovati lavori
precedenti che analizzassero questa relazione. Una possibile
spiegazione \`e che le piogge eccessive ritardino la migrazione
obbligando il topino a fermarsi per lunghi periodi. Soste obbligate e
prolungate in aree non idonee come siti di stop-over potrebbero causare
fenomeni di competizione intra- e interspecifica per le risorse
disponibili, come \`e stato dimostrato per la migrazione attraverso il
golfo del Messico (Moore \& Yong 1991). 

I risultati ottenuti evidenziano che la componente climatica \`e un
elemento importante nella sopravvivenza del topino. In ogni periodo
dell{\textquoteright}anno le condizioni atmosferiche influenzano la
popolazione considerata, favorendone o sfavorendone la sopravvivenza.\\

\begin{table}[!h]
\centering
\small
\begin{tabular}{>{\raggedright\arraybackslash}p{.11\columnwidth}>{\raggedright\arraybackslash}p{.11\columnwidth}>{\raggedright\arraybackslash}p{.11\columnwidth}>{\raggedright\arraybackslash}p{.11\columnwidth}>{\raggedright\arraybackslash}p{.11\columnwidth}>{\raggedright\arraybackslash}p{.11\columnwidth}>{\raggedright\arraybackslash}p{.11\columnwidth}}
\toprule
\textbf{Modello} & \textbf{AICc} & \textbf{Delta AICc} & \textbf{AICc Weights} & \textbf{Model Likelihood} & \textbf{Num. Par} & \textbf{Deviance} \\
\toprule
%\showrowcolors
$\Phi\Phi$(.)p(t) & 1196.3454 & 0 & 0.99229 & 1 & 11 & 44.6692 \\
$\Phi\Phi$(t)p(.) & 1207.1475 & 10.8021 & 0.00448 & 0.0045 & 11 & 55.4713 \\
$\Phi\Phi$(t)p(t) & 1207.7988 & 11.4534 & 0.00323 & 0.0033 & 19 & 39.9043 \\
$\Phi\Phi$(.)p(.) & 1234.7593 & 38.4139 & 0 & 0 & 2 & 101.1934 \\
\bottomrule
\hiderowcolors
\end{tabular}
\caption{Modelli per la sopravvivenza e la probabilit\`a di cattura}
\label{Masoero_tab_1}
\end{table}

\section*{Bibliografia}
\begin{itemize}
  \item Blad\'e I., Liebmann B., Fortuny D. \& Van Oldenborgh G.J., 2012 -
Observed and simulated impacts of the summer NAO in Europe:
implications for projected drying in the Mediterranean region.
\textit{Climate Dynamics,} 39: 709-727.

  \item Cooch E. \& White G., 2012 - \textit{Program Mark: a gentle
introduction}. Evan Cooch \& Gay White Eds, 11th Revision.

  \item Cormack R.M., 1964 - Estimates of survival from the sighting of marked
animals. \textit{Biometrika}, 51: 429-438.

  \item Cowley E. \& Siriwardena G.M., 2005 - Long-term variation in survival
rates of Sand Martins \textit{Riparia riparia}: dependence on breeding
and wintering ground weather, age and sex, and their population
consequences. \textit{Bird Study, }52: 237--251.

  \item Dobson A.P., 1990 - Survival rates and their relationship to
life-history traits in some common British birds. \textit{Curr.
Ornithol.}, 7: 115--146.

  \item Jolly G.M., 1965 - Explicit estimates from capture-recapture data with
both death and immigration -stochastic model. \textit{Biometrika}, 52:
225-247.

  \item Mead C.J., 1979 - Colony fidelity and interchange in the Sand Martin.
\textit{Bird Study}, 26: 99-107.

  \item Moore F.R. \& Yong W., 1991 - Evidence of food-based~competition among
passerine migrants during~stopover. \textit{Behavioral Ecology And
Sociobiology}, 28: 85-90.

  \item Norman D. \& Peach W.J., 2013 - Density-dependent survival and
recruitment in a long-distance Palaearctic migrant, the Sand Martin
\textit{Riparia riparia}. \textit{Ibis}, 155: 284-296.

  \item Robinson R.A., Balmer D.E. \& Marchant J.H., 2008 - Survival rates of
Hirundines in relation to British and African rainfall. \textit{Ringing
\& Migration}, 24: 1--6.

  \item Seber G.A.F., 1965 - A note on the multiple recapture census.
\textit{Biometrika}, 52: 249-259.
\end{itemize}

\setcounter{figure}{0}
\setcounter{table}{0}

\begin{adjustwidth}{-3.5cm}{-1cm}
\pagestyle{CIOpage}
\authortoc{\textsc{Pedrini P.}, \textsc{Brambilla M.}, 
\textsc{Florit F.}, \textsc{Martignago G.}, 
\textsc{Mezzavilla F.}, \textsc{Rassati G.}, 
\textsc{Silveri G.}}
\chapter*[Il re di quaglie in Italia nord-orientale]{Andamento demografico del re di quaglie \textbf{\textit{Crex crex}}\textbf{ nell{\textquoteright}Italia nord-orientale}}
\addcontentsline{toc}{chapter}{Il re di quaglie in Italia nord-orientale}

\textsc{Paolo Pedrini}$^{1*}$, \textsc{Mattia Brambilla}$^{1}$, 
\textsc{Fabrizio Florit}$^{2}$, \textsc{Gianfranco Martignago}$^{3}$, 
\textsc{Francesco Mezzavilla}$^{3}$, \textsc{Gianluca Rassati}$^{4,5}$, 
\textsc{Giancarlo Silveri}$^{3}$\\

\index{Pedrini Paolo} \index{Brambilla Mattia} \index{Florit Fabrizio} \index{Martignago Gianfranco} \index{Mezzavilla Francesco} \index{Rassati Gianluca} \index{Silveri Giancarlo}
\noindent\color{MUSEBLUE}\rule{27cm}{2pt}
\vspace{1cm}
\end{adjustwidth}



\marginnote{\raggedright $^1$Muse - Museo delle Scienze, Sezione di Zoologia dei
Vertebrati, Corso del Lavoro e della Scienza 3 - 38123 Trento, Italy \\
$^2$Regione autonoma Friuli Venezia Giulia, Direzione
centrale risorse rurali, agroalimentari e forestali, Servizio caccia,
risorse ittiche e biodiversit\`a, Ufficio studi faunistici, via
Sabbadini 31, 33100 Udine \\
$^3$Associazione Faunisti Veneti, C/o Museo Civico di
Storia Naturale S. Croce 1730 - 30135 Venezia \\
$^4$ Via Udine 9 - 33028 Tolmezzo (UD)\\
$^5$ Regione Autonoma Friuli Venezia Giulia, Ispettorato
agricoltura e foreste di Tolmezzo, Via San Giovanni Bosco 8 - 33028
Tolmezzo (UD)\\
\vspace{.5cm}
{\emph{\small $^*$Autore per la corrispondenza: \href{mailto:paolo.pedrini@muse.it}{paolo.pedrini@muse.it}}} \\
\keywords{\textit{Crex crex}, agricoltura, Alpi orientali, prati}
{\textit{Crex crex}, Eastern Alps,
agriculture, grassland}
%\index{keywords}{\textit{Crex crex}} \index{keywords}{Agricoltura} \index{keywords}{Alpi orientali} \index{keywords}{Prati}
}
{\small

\noindent \textsc{\color{MUSEBLUE} Summary} / We counted by means of nocturnal censuses the number of corncrake
\textit{Crex crex} calling males in selected sample areas in
north-eastern Italy (Trentino, Veneto, Friuli Venezia Giulia), in an
area considered to be the species{\textquotesingle} stronghold in
Italy. Counts were done in June-July, in the period 2000-2012.
Population trend was modelled under a Generalised Estimating Equation
(GEE) approach. Despite some differences in population trend across
regions (with the trend being more dramatic in Trentino and Veneto),
the species showed a significant decline during 2000-2012, a period
that is considered as favourable to the species at the European level.
This contrast with the general trend of the species suggested that
local factors are involved in driving the species decline; in
particular, unsustainable mowing management and agricultural
intensification in general are likely the most impacting factors. Also
land abandonment and, at a more local scale, urbanization, disturbance
and overgrazing may be associated with the species decline. \\
\noindent \textsc{\color{MUSEBLUE} Riassunto} / Abbiamo censito in orario notturno il numero di maschi cantori di re di
quaglie \textit{Crex crex} in aree campione selezionate in Italia
nord-orientale (Trentino, Veneto, Friuli Venezia Giulia), in
un{\textquotesingle}area ritenuta la roccaforte della specie a livello
nazionale. I censimenti sono stati svolti in giugno-luglio, nel periodo
2000-2012. Il trend della popolazione \`e stato modellizzato
utilizzando un approccio di tipo \textit{Generalised Estimating
Equation }(GEE). Nonostante alcune differenze nel trend della specie
tra le regioni (con un declino pi\`u evidente in Trentino e Veneto), la
specie ha mostrato un decremento significativo nel periodo 2000-2012,
che generalmente \`e considerato un periodo favorevole alla specie a
scala europea, con trend positivo nella maggior parte dei casi. Questo
contrasto con il trend generale della specie suggerisce la presenza di
fattori locali responsabili del declino della specie; in particolare,
pratiche di sfalcio non sostenibili e intensificazione
dell{\textquotesingle}agricoltura in generale rappresentano i fattori
pi\`u probabilmente importanti nel determinare il trend negativo della
specie. Anche l{\textquotesingle}abbandono delle aree agricole
marginali e, a una scala pi\`u locale, urbanizzazione, disturbo e
pascolo eccessivo sono probabilmente tra le cause concomitanti del
declino.
}



\section*{Introduzione}

Il re di quaglie \textit{Crex crex} \`e una specie di grande interesse
conservazionistico (SPEC 1; BirdLife International 2004), legata ad
ambienti prativi in larga parte dell{\textquotesingle}Eurasia (Cramp \&
Simmons 1980), dove \`e un visitatore estivo diffuso ma scarso alle
medie latitudini. La popolazione europea della specie ha subito un
drammatico declino durante i secoli XIX e XX (Green \textit{et al}.,
1997a; BirdLife International, 2004), le cui cause sono probabilmente
soprattutto l{\textquotesingle}intensificazione
dell{\textquotesingle}agricoltura e la sua meccanizzazione, lo sfalcio
precoce, i cambiamenti nelle pratiche agricole (Cramp \& Simmons 1980;
Broyer 1987; Kei\v{s}s 2003; Rassati \& Rodaro 2007; Moga \textit{et
al.} 2010). Dagli anni Novanta, diverse popolazioni hanno mostrato
segni di ripresa e incremento (O{\textquoteright}Brien \emph{et al}.
2006; Kei\v{s}s 2003), mentre altre hanno proseguito il declino.

In Europa centrale e occidentale, dove il re di quaglie appare legato
agli ambienti prativi da sfalcio (prati umidi, pascoli),
nell{\textquotesingle}ultimo ventennio la specie ha mostrato segnali di
recupero in buona parte del suo areale; per l{\textquoteright}Italia
mancano per\`o dati complessivi aggiornati a scala nazionale, anche se
dalla met\`a degli anni Novanta sono stati promossi sia monitoraggi a
scala nazionale che studi locali, i quali hanno permesso di
quantificare la popolazione, definire gli habitat di nidificazione e la
distribuzione, di fatto quasi esclusivamente limitata alla regione
alpina centro orientale (vedi Gustin \textit{et al.} 2009 e riferimenti
ivi citati). Con questo contributo abbiamo analizzato il trend
demografico della specie nella fascia prealpina e alpina di Trentino,
Veneto e Friuli Venezia Giulia, dove si rinviene la gran parte della
popolazione italiana. 

\section*{Metodi}

Abbiamo utilizzato i dati raccolti durante censimenti notturni dei
maschi cantori, condotti a partire dall{\textquoteright}anno 2000 in
alcuni siti campione ritenuti rappresentativi della distribuzione
regionale della specie. Stante l{\textquotesingle}esistenza di
variazioni nel corso della stagione riproduttiva di abbondanza locale e
distribuzione (Rassati 2001, 2004, 2009; Brambilla \& Pedrini 2011;
Pedrini \textit{et al.} 2012), abbiamo considerato dati il pi\`u
possibile omogenei dal punto di vista del periodo di censimento,
scegliendo i censimenti {\textquotedblleft}tardivi{\textquotedblright}
(giugno-luglio), in quanto gli unici disponibili per quasi tutti gli
anni indagati per tutte e tre le regioni considerate dallo studio. 

Le analisi sono state condotte attraverso un approccio di tipo
Generalised Estimating Equations, utilizzando il software TRIM (TRends
\& Indices for Monitoring data) 3.54. L{\textquotesingle}approccio
adottato da TRIM \`e quello delle
\textit{{generalized
estimating
equations}}{, che
consentono di stimare valori per i dati mancanti, tenendo conto di
}\textit{overdispersion} e autocorrelazione seriale (Pannekoek \& Van
Strien 2001; Soldaat
\textit{{et
al}}{.
}{2007; Ludwig
}\textit{{et
al.}}{
}{2008). Abbiamo
fissato pari a 1 il valore dell{\textquotesingle}indice di popolazione
per il 2000, primo anno
considerato}{.} Il
modello utilizzato \`e stato \textit{linear trend }con \textit{stepwise
selection }dei \textit{changepoints}, utilizzando i valori di
probabilit\`a proposti di default dal programma, dal momento che tale
modello di trend rappresentava in generale quello statisticamente
migliore (v. anche Pedrini \textit{et al}. 2012; Brambilla \& Pedrini
2013), e inserendo la regione come covariata, per verificare
l{\textquotesingle}esistenza di diversit\`a
nell{\textquotesingle}andamento demografico tra le tre aree, e tenerne
conto nell{\textquotesingle}elaborazione del trend complessivo.

\section*{Risultati}

Il trend complessivo della specie nel suo principale areale italiano nel
periodo 2000-2012 appare negativo, con una significativa tendenza al
declino e alcune fluttuazioni (Fig. 1). In generale, si assiste a un
calo con un accenno di ripresa nel 2007-2008, seguito da nuovo declino.


Tuttavia, emergono forti discrepanze a livello regionale (Tab. \ref{Pedrini_tab_1}),
confermate dall{\textquoteright}effetto significativo del fattore
regione nel modello complessivo (p = 0.005). In particolare, si
evidenzia un marcato declino in Trentino e in Veneto, mentre la
situazione, per quanto riguarda il Friuli, risulta sostanzialmente
stabile, sebbene i dati dei censimenti precoci suggeriscano un calo
anche in questa regione (vedi sotto). 

\section*{Discussione}

I risultati mostrano chiaramente come il re di quaglie sia ancora in
calo a livello nazionale. Confrontando la situazione italiana con
quella degli altri paesi europei, si nota come sia improbabile che il
calo mostrato in Italia possa dipendere da fattori di dinamica globale
delle popolazioni, come le condizioni riscontrate durante lo
svernamento o la migrazione. Nello stesso periodo, infatti, negli altri
stati europei il re di quaglie ha mostrato trend di popolazione
sostanzialmente positivi (Kei\v{s}s 2003, 2004; O{\textquoteright}Brien
\textit{et al.} 2006; BirdLife International 2013), seppur con marcate
fluttuazioni, come tipico per la specie (Rassati \& Tout 2002).
L{\textquoteright}esistenza di un effetto dovuto a fattori locali,
legati alla gestione e conservazione dei suoi habitat di nidificazione,
pare trovare conferma nella discrepanza tra le tendenze demografiche
mostrate nelle regioni da noi considerate, nonostante si tratti di aree
tra loro confinanti. In particolare, la situazione rilevata in Friuli
Venezia Giulia appare meno critica rispetto a quella trentina e veneta,
sebbene i conteggi per la prima parte della stagione suggeriscano un
calo di popolazione anche in questa regione.

Questi risultati confermano la necessit\`a di proseguire nel
monitoraggio di questa specie, il cui stato di conservazione in Italia
risulta {\textquotedblleft}cattivo{\textquotedblright} (Gustin
\textit{et al.} 2009), e il cui andamento demografico lascia presagire
una forte riduzione e contrazione di areale nei prossimi anni, se non
si riuscir\`a a fermare il declino delle popolazioni. In questo senso,
appare fondamentale promuovere forme di gestione degli ambienti prativi
che tengano conto delle esigenze della specie, soprattutto
nell{\textquotesingle}ambito del Piano di Sviluppo Rurale nelle regioni
alpine e in particolare nelle aree dove la densit\`a della specie
rimane relativamente alta (Pedrini \textit{et al}. 2012).

I fattori di minaccia principali per la specie sono essenzialmente
riconducibili a due opposte dinamiche: da un lato,
l{\textquotesingle}intensificazione delle pratiche agricole, con una
gestione dei prati non compatibile con le esigenze della specie
(sfalcio meccanizzato su ampie superfici, rimozione di elementi
marginali, fertilizzazione eccessiva), dall{\textquotesingle}altro,
l{\textquotesingle}abbandono di molte aree
{\textquotedblleft}periferiche{\textquotedblright}, con conseguente
ritorno del bosco e perdita di habitat per il re di quaglie e le altre
specie legate agli ambienti aperti. Localmente, altri fattori come
urbanizzazione, disturbo antropico e da animali domestici e il pascolo
eccessivo, possono avere un impatto negativo sulla specie.

\begin{figure}[!h]
\centering
\includegraphics[width=.8\columnwidth]{Pedrini_fig_1.png}
\caption{Trend di popolazione ($\pm$ errore standard) complessivo del re di quaglie nell'area di studio nel periodo 2000-2012}
\label{Pedrini_fig_1}
\end{figure}

\begin{table}[!h]
\centering
\small
\begin{tabular}{>{\raggedright\arraybackslash}p{.2\columnwidth}>{\raggedright\arraybackslash}p{.18\columnwidth}>{\raggedright\arraybackslash}p{.2\columnwidth}>{\raggedright\arraybackslash}p{.3\columnwidth}}
\toprule
\textbf{Area} & \textbf{Additivo $\pm$ ES} & \textbf{Moltiplicativo} $\pm$ ES & \textbf{Classificazione del trend} \\
\toprule
%\showrowcolors
Totale & -0.0625 $\pm$ 0.0083 & 0.9394 $\pm$ 0.0078 & Declino moderato (p < 0.01) \\
Trentino & -0.0813 $\pm$ 0.0216 & 0.9219 $\pm$ 0.0199 & Declino moderato (p < 0.01) \\
Veneto & -0.0849 $\pm$ 0.0174 & 0.9186 $\pm$ 0.0160 & Declino forte (p < 0.01) \\
Friuli Ven. Giulia & 0.0167 $\pm$ 0.0147 & 1.0168 $\pm$ 0.0149 & Stabile \\
\bottomrule
\hiderowcolors
\end{tabular}
\caption{Trend di popolazione del re di quaglie ($\pm$ errore standard) nell'area di studio nel periodo 2000-2012}
\label{Pedrini_tab_1}
\end{table}

\section*{Ringraziamenti}

Gli autori desiderano ringraziare tutti quanti hanno collaborato alla
ricerca e in particolare: S. Lombardo, Stefano Noselli, Franco
Rizzolli, Francesca Rossi, Karol Tabarelli de Fatis, Gilberto Volcan. 
Ricerca parzialmente finanziata da Progetto BIODIVERSIT\`A
(PAT-2001-05), Accordo di Programma per la Ricerca PAT, 2009-13. 

\section*{Bibliografia}
\begin{itemize}\itemsep0pt
	\item BirdLife International, 2004 - Birds in the European Union: a status
assessment. Wageningen: 

	\item BirdLife International, 2013 - Species factsheet: \textit{Crex crex}.
Available at \href{http://www.birdlife.org}{http:\allowbreak //www.\allowbreak bird\allowbreak li\allowbreak fe.\allowbreak org} (accessed 6 February 2013).

	\item Brambilla M. \& Pedrini P., 2013 - The introduction of subsidies for
grassland conservation in the Italian Alps coincided with population
decline in a threatened grassland species, the Corncrake \textit{Crex
crex}. \textit{Bird Study} 60: 404-408.

	\item Brambilla M. \& Pedrini P., 2011 - Intra-seasonal changes in local
pattern of corncrake Crex crex occurrence require adaptive conservation
strategies in Alpine meadows. \textit{Bird Conserv. Int}. 21:
388{}--393.

	\item Broyer J., 1987 - The habitat of the corncrake \textit{Crex crex} in
France. \textit{Alauda} (55): 161--186 (in French with English
summary). 

	\item Green R.E., Rocamora G. \& Sch\"affer N., 1997 - Populations, ecology
and threats to the Corncrake \textit{Crex crex }in Europe.
\textit{Vogelwelt, }118: 117{}--134.

	\item Gustin M., Brambilla M. \& Celada C., 2009 - Valutazione dello stato di
conservazione dell{\textquoteright}avifauna italiana. Roma: Ministero
dell{\textquoteright}Ambiente, della Tutela del Territorio e del Mare
\& LIPU/BirdLife Italia.

	\item Kei\v{s}s O., 2003 - Recent increases in numbers and the future of
Corncrake \textit{Crex crex} in Latvia. \textit{Ornis Hungarica,}
12{}--13: 151{}--156.

	\item Kei\v{s}s O., 2004 - Results of a survey of Corncrake \textit{Crex crex}
in Latvia, 1989{}--1995. \textit{Bird Census News,} 13: 73{}--76.

	\item Ludwig T., Storch I., \& W\"ubbenhorst J., 2008 - How the black grouse
was lost: Historic reconstruction of its status and distribution in
Lower Saxony (Germany). \textit{J. Ornithol}., (149): 587{}--596.

	\item Moga C. I., Hartel T., \& \"Ollerer K. 2010 - Status, microhabitat use
and distribution of the corncrake \textit{Crex crex }in a Southern
Transylvanian rural landscape, Romania. \textit{North-Western Journal
of Zoology}, 6 (1): 63-70.

	\item O{\textquoteright}Brien M., Green R.E. \& Wilson J. 2006 - Partial
recovery of the population of Corncrakes \textit{Crex crex }in Britain,
1993{}--2004. \textit{Bird Study,} 53: 213{}--224.

	\item Pannekoek J. \& Van Strien A.J., 2001 - TRIM (Trends and Indices for
Monitoring Data). Statistics Netherlands, Voorburg. 

	\item Pedrini P., Rizzolli F., Rossi F. \& Brambilla M., 2012 - Population
trend and breeding density of corncrake \textit{Crex crex} (Aves:
Rallidae) in the Alps: Monitoring and conservation implications of a
15-years survey in Trentino, Italy\textit{. }\textit{Ital. J. Zool}.,
79: 377{}--384.

	\item {Rassati G. 2001 - Il Re
di quaglie
}\textit{{Crex
crex}}{ durante
l{\textquoteright}anno 2000 in due aree campione in Carnia (Alpi
Orientali, Friuli-Venezia Giulia).
}\textit{{Avocetta,}}{
25 (1): 239.}

	\item {Rassati G., 2004 -
Evoluzione faunistica nelle aree rurali abbandonate. La presenza del Re
di quaglie
(}\textit{{Crex
crex}}{) e della Lepre
comune (}\textit{{Lepus
europaeus}}{).
}\textit{{Agribusiness
Paesaggio \&
Ambiente}}{ VII (1):
41-48.}

	\item {Rassati G.,
}{2009 -
}{The spring and summer
censuses of Corncrake
}\textit{{Crex
crex}}{ in three sample
areas of Carnia (Eastern Alps, Friuli-Venezia Giulia, North-eastern
Italy) (Years 2000-2005).
}{Gli Uccelli
d{\textquoteright}Italia XXXIV: 50-57.}

	\item {Rassati G. \& Rodaro
P., }{2007 - Habitat,
vegetation and land management of Corncrake
}\textit{{Crex
crex}}{ breeding sites
in Carnia (Friuli-Venezia Giulia, NE Italy).
}\textit{{Acrocephalus,}}{
28 (133): 61-68.}

	\item {Rassati, G. \& Tout
C.P., 2002 - }{The
Corncrake
(}\textit{{Crex
crex}}{) in
Friuli-Venezia Giulia (North-eastern Italy).
}\textit{{Avocetta,}}{
26 (1): 3-6.}

	\item {Soldaat L., Visser H.,
Roomen M. \& Strien A.,
}{2007 -
}{Smoothing and }trend
detection in waterbird monitoring data using structural time-series
{analysis and the
Kalman filter.
}\textit{{J.
Ornithol}}{.,
}{148:
}{S351}{{}--}{S357.}
\end{itemize}

\setcounter{figure}{0}
\setcounter{table}{0}

\begin{adjustwidth}{-3.5cm}{0cm}
\pagestyle{CIOpage}
\authortoc{\textsc{Termine R.}, \textsc{Massa B.}}
\chapter*[Nidificazione di svasso piccolo]{Nidificazione di svasso piccolo \textbf{\textit{Podiceps
nigricollis}}\textbf{ C. L. Brehm, 1831 al lago di Pergusa (Enna)}}
\addcontentsline{toc}{chapter}{Nidificazione di svasso piccolo}

\textsc{Rosa Termine}$^{1*}$, \textsc{Bruno Massa}$^{2}$ \\

\index{Termine Rosa} \index{Massa Bruno}
\noindent\color{MUSEBLUE}\rule{27cm}{2pt}
\vspace{1cm}
\end{adjustwidth}



\marginnote{\raggedright $^1$Laboratorio di Ingegneria Sanitaria Ambientale,
Universit\`a di Enna {\textquotedblleft}Kore{\textquotedblright},
Cittadella Universitaria, 94100 Enna, Italia  \\
$^2$Dipartimento di Scienze agrarie e forestali,
Universit\`a di Palermo, V.le delle Scienze, 90128 Palermo, Italia \\
\vspace{.5cm}
{\emph{\small $^*$Autore per la corrispondenza: \href{mailto:rosa.termine@unikore.it}{rosa.termine@unikore.it}}} \\
\keywords{\textit{Podiceps nigricollis}, nidificazione, lago
Pergusa, Sicilia, Italia}
{\textit{Podiceps nigricollis}, nesting, Pergusa lake,
Sicily, Italy}
%\index{keywords}{\textit{Podiceps nigricollis}} \index{keywords}{Nidificazione} \index{keywords}{Lago
%Pergusa} \index{keywords}{Sicilia}
}
{\small
\noindent \textsc{\textcolor{MUSEBLUE}{Summary}} / The black-necked grebe bred again in the Pergusa lake.
In Italy, the black- necked grebe is an irregular breeding bird;
nesting documented cases after 1950 number around twenty, mostly
restricted to a few pairs. In Sicily, apart from the nesting in 1957 in
the Pergusa lake and in 1966 in Scanzano lake, for thirty-four years
this species have been sighted occasionally and irregularly as summer
visitor; from 2000 to 2013 new cases of breeding have been recorded: in
2000 and 2005 in the province of Caltanissetta, in 2004, in 2006 and
2011 in the province of Siracusa, in 2010, 2012 and 2013 in Pergusa
lake (Enna). This lake has so far counted the largest number of
breeding pairs for Sicily and Italy; in 2013 a population of 120 adults
and 108 young has been censused. Over the last three years of breeding
(2010, 2012 and 2013) in Pergusa, a large presence of phanerophytes
\textit{Ruppia} and \textit{Potamogeton} with the
 formation of mats has been observed; it would be
interesting to research whether and how the presence of these mats is a
causal factor of the settling of nesting individuals of this species. \\
\noindent \textsc{\color{MUSEBLUE} Riassunto} / Lo svasso piccolo \textit{Podiceps nigricollis} \`e tornato a nidificare
al lago di Pergusa. In Italia, questa specie \`e nidificante
irregolare; i casi di riproduzione documentati dopo il 1950 sono circa
una ventina, perlopi\`u relativi a poche coppie. In Sicilia, a parte le
nidificazioni del 1957 al lago di Pergusa e del 1966 al lago di
Scanzano, per trentaquattro anni la specie \`e stata avvistata
occasionalmente e in modo irregolare come estivante; dal 2000 al 2013
sono stati documentati nuovi casi di nidificazioni: nel 2000 e nel 2005
in Provincia di Caltanissetta; nel 2004, nel 2006 e nel 2011 in
Provincia di Siracusa; nel 2010, nel 2012 e nel 2013 al lago di Pergusa
in Provincia di Enna. {Il lago di Pergusa }ha finora
contato il maggior numero di coppie nidificanti della specie per la
Sicilia e per l{\textquoteright}Italia; nell{\textquoteright}estate
2013 la popolazione \`e stata di 120 adulti, che hanno prodotto 108
giovani. Durante questi ultimi tre anni di nidificazioni (2010, 2012 e
2013) a Pergusa \`e stata rilevata una cospicua presenza delle
fanerofite \textit{Ruppia} e \textit{Potamogeton} con formazione di
\textit{mats}; sarebbe interessante indagare se e in che modo, la
presenza di tali \textit{mats} sia stato un fattore causale
dell{\textquoteright}insediamento dei soggetti nidificanti di questa
specie.
}




\section*{Introduzione}
La Sicilia, fino al 1800, era ricca di aree umide la cui distruzione
risale all{\textquoteright}ultimo secolo, passando da circa 100.000
ettari nel 1865 ai 47.174 dell{\textquoteright}inizio degli anni
{\textquoteleft}30 (Rallo \& Pandolfi 1988) fino ai soli 5000 ettari
attuali (Lo Valvo \textit{et al}. 1993); ci\`o ha causato
l{\textquoteright}estinzione locale di diverse specie di uccelli
acquatici. 

Il lago di Pergusa \`e uno dei pochi laghi naturali della Sicilia; esso
rappresenta una delle pi\`u importanti aree siciliane per la tutela
degli uccelli stanziali, svernanti e migratori, oltre a
rivestire{ un ruolo importante per} la nidificazione
di alcuni di loro.

In Italia, lo svasso piccolo \`e nidificante irregolare; i casi di
riproduzione documentati dopo il 1950, escludendo quelli possibili o
probabili, sono circa una ventina, perlopi\`u relativi a poche coppie
(Brichetti \& Fracasso 2013). In Sicilia tale specie era considerata
sedentaria e nidificante regolare da Benoit (1840) e Doderlein (1873).
Nel 1958 Krampitz segnal\`o 20-25 coppie nidificanti al lago di
Pergusa. Massa e Schenk (1983) invece lo ritenevano estinto in Sicilia
dal 1965. Successivamente Iapichino e Massa (1989) lo considerarono
nidificante occasionale, dopo la segnalazione del 1966 di una coppia
con 2 giovani al lago di Scanzano (PA). {Per
trentaquattro anni la specie \`e stata} avvistata
nell{\textquoteright}Isola solo con presenze occasionali e in modo
irregolare come estivante. Dal 2000 al 2012 sono stati documentati
alcuni nuovi casi di nidificazioni: nel 2000 sono state avvistate 1-2
coppie presso l{\textquoteright}invaso di Comunelli (CL, Mascara
2007); nel 2004 \`e stata avvistata una coppia con 2 pulli
presso il Pantano Cuba (SR) e una coppia con 4 \textit{juv.} al Pantano
Longarini (SR) (Corso 2005); nel 2005 \`e stata avvistata una coppia al
Comunelli e una coppia nell{\textquoteright}invaso di Cimia (CL, Mascara 2007); nel 2006 \`e stata accertata la nidificazione presso il
Pantano Baronello (SR) e avvistate 2-3 coppie al Pantano Cuba (Corso,
2007); nel 2011 \`e stata osservata una coppia con 2 pulli
presso il Pantano della Riserva Naturale Saline di Priolo (SR) (Di
Blasi, \textit{com. pers. }2011); in altri casi (Pantani di Pachino,
SR) la nidificazione \`e stata ritenuta possibile, ma non accertata. 

\section*{Area di studio}

Il lago di Pergusa, localizzato al centro della Sicilia tra i monti
Erei, ha una quota di 667 metri s.l.m. \`E l{\textquoteright}unico lago
endoreico siciliano;{ di origine tettonica, occupa la
parte pi\`u depressa di una struttura sinclinale pliocenica. La sua
fonte principale di alimentazione \`e rappresentata dalle
precipitazioni e dalle falde freatiche; esso, a causa
dell{\textquoteright}evaporazione estiva, \`e caratterizzato da acque
salmastre.}

{Vari interventi antropici, iniziati negli anni
{\textquoteright}30 con opere di bonifica e accentuati negli anni
{\textquoteright}60 e {\textquoteright}70 con
l{\textquoteright}emungimento di acqua dalle falde, hanno messo a serio
rischio la sua esistenza tanto da determinarne la quasi totale
riduzione dello specchio lacustre nell{\textquoteright}estate 2002.
Oggi, in seguito alla riduzione quasi totale
dell{\textquoteright}emungimento e grazie anche ad alcune stagioni
particolarmente piovose, il lago ha avuto una ripresa notevole,
occupando un{\textquoteright}area} di circa 141 ettari, di cui circa 35
nella cintura esterna sono ricoperti da \textit{Phragmites australis
}(Cav.) Trin.

{
Alla fine degli anni {\textquoteright}50 \`e stato realizzato un
autodromo, che cinge completamente il Lago a stretto contatto delle
sponde; la sua realizzazione ha determinato, oltre che consumo di
territorio, l{\textquoteright}isolamento della fascia riparia dalle
circostanti colline, con il depauperamento della zona ecotonale a causa
della consistente barriera lineare determinata dalla pista e dalle
strutture connesse.}

Il Lago \`e una Riserva Naturale Speciale, istituita dalla Legge
Regionale 71 del 1995 e gestita dalla Provincia Regionale di Enna;
{l{\textquoteright}area protetta ha
un{\textquoteright}estensione totale di 402,5 ha.} Fa anche parte della
Rete Natura 2000 (SIC-ZPS ITA060002 {\textquotedbl}Lago di
Pergusa{\textquotedbl}) e c{ome geosito del
{\textquotedblleft}Rocca di Cerere Geopark{\textquotedblright} rientra
nelle Reti dei Geoparchi Europea (EGN) e Globale (GGN).}
{I vincoli di protezione e la conseguente
regolamentazione delle attivit\`a motoristiche hanno migliorato le
condizioni del delicato ecosistema lacustre, diminuendone notevolmente
lo stress ambientale.}

Pur essendo di limitata estensione, Pergusa ospita una ricca fauna;
negli anni sono state censite 299 specie (Vertebrati e Invertebrati) di
cui 177 specie di uccelli tra nidificanti, svernanti e migratrici, tra
le quali diverse di importanza conservazionistica come
\textit{Porphyrio porphyrio}, \textit{Aythya nyroca},
\textit{Ixobrychus minutus}, \textit{Netta rufina}, \textit{Circus
aeruginosus}, incluse nell{\textquoteright}Allegato I della Dir.
2009/147/CE e nella Lista Rossa Italiana (Termine \textit{et al.},
2008). Numerose sono state le nidificazioni accertate al lago di
Pergusa; tra queste quella di\textit{ P. porphyrio}, di cui nel 2013
sono state censite 29 coppie, e quella di \textit{Himantopus
himantopus}{, }del quale nel 2013 \`e stato
controllato un nido con 4 uova e la nascita di 4 \textit{pulli}.

\section*{Metodi}

Il monitoraggio dello svasso piccolo \`e stato eseguito con
l{\textquotesingle}osservazione diretta sul campo, con
l{\textquoteright}ausilio di binocolo 10x42 e cannocchiale
25-50x80 con cadenza quindicinale e, a volte, anche pi\`u
frequentemente, percorrendo l{\textquoteright}intero perimetro del Lago
con un mezzo natante con motore elettrico per accedere a punti
altrimenti difficilmente osservabili dalle rive.

Nella stagione calda, i rilevamenti generalmente sono stati effettuati
nelle ore del mattino o del tardo pomeriggio, quando maggiore \`e
l{\textquoteright}attivit\`a trofica degli uccelli acquatici. I dati di
campo sono stati poi inseriti in un database.

\newcolumntype{s}{>{\centering\columncolor{white!80!MUSEBLUE}\arraybackslash}p{.06\columnwidth}}
\newcolumntype{M}{>{\centering\columncolor{white!55!MUSEBLUE}\arraybackslash}p{.06\columnwidth}}
\newcolumntype{E}{>{\centering\columncolor{white!30!MUSEBLUE}\arraybackslash}p{.06\columnwidth}}
\newcommand{\gw}{\cellcolor{white}}
\newcommand{\gr}{\cellcolor{white!5!MUSEBLUE}}
\newcommand{\gp}{\cellcolor{white!95!MUSEBLUE}}
\newcommand\crule[3][black]{\textcolor{#1}{\rule{#2}{#3}}}

%\newcolumntype{R}{>{\centering\columncolor{Gray}\arraybackslash}p{.08\columnwidth}}
%\newcolumntype{P}{>{\centering\columncolor{Gray}\arraybackslash}p{.08\columnwidth}}
\begin{table}[!h]
\centering
\scalebox{.8}{
\begin{tabular}{>{\raggedright\arraybackslash}p{.1\columnwidth}ssMMEEEEMMMs}
\hiderowcolors
\toprule
\textbf{Anno} & \gw \rotatebox{90}{\textbf{GEN}} & \gw \rotatebox{90}{\textbf{FEB}} & \gw \rotatebox{90}{\textbf{MAR}} & \gw \rotatebox{90}{\textbf{APR}} & \gw \rotatebox{90}{\textbf{MAG}} & \gw \rotatebox{90}{\textbf{GIU}} & \gw \rotatebox{90}{\textbf{LUG}} & \gw \rotatebox{90}{\textbf{AGO}} & \gw \rotatebox{90}{\textbf{SET}} & \gw \rotatebox{90}{\textbf{OTT}} & \gw \rotatebox{90}{\textbf{NOV}} & \gw \rotatebox{90}{\textbf{DIC}} \\
\toprule
 2004 & 6 & 5 & 10 &  &  &  &  &  &  &  &  &  \\
 \midrule
 2005 &  &  &  &  & 1 & 1 &  & 1 & 1 &  &  &  \\
 \midrule
 2006 &  & 2 &  & 1 & 4 & 2 &  & 3 & 2 & 1 &  &  \\
 \midrule
 2007 & 1 &  &  & 5 & 5 & 10 & 5 &  & 3 &  &  &  \\
 \midrule
 2008 & 4 & 2 & 2 & 2 & 6 &  & 4 & 3 &  &  &  & 4 \\
 \midrule
 2009 & 2 &  &  &  &  &  &  & 6 & 12 & 12 & 8 & 5 \\
 \midrule
 2010 & 4 & 4 &  & \gr & \gr 2 & \gr 6 & \gr 14 & \gp 16 & \gp 15 & \gp 15 & 6 & 2 \\
 \midrule
 2011 & 2 & 4 & 6 & 8 & 8 & 8 & 4 & 2 & 2 &  & 5 & 3 \\
 \midrule
 2012 & 8 & 8 & 15 & \gr 20 & \gr 40 & \gr 50 & \gr 50 & \gp 45 & \gp 46 & \gp 45 & 12 & 4 \\
 \midrule
 2013 & 5 & 5 & 10 & \gr 18 & \gr 41 & \gr 81 & \gr 128 & \gp 124 & \gp 120 & \gw &  \gw &  \gw \\
\bottomrule
\multicolumn{13}{c}{} \\
\multicolumn{13}{c}{\crule[white!80!MUSEBLUE]{.3cm}{.3cm} Svernamento \crule[white!55!MUSEBLUE]{.3cm}{.3cm} Migrazione \crule[white!30!MUSEBLUE]{.3cm}{.3cm} Estivazione \crule[white!5!MUSEBLUE]{.3cm}{.3cm} Riproduzione \crule[white!95!MUSEBLUE]{.3cm}{.3cm} Post-riproduzione} \\
\end{tabular}
}
\caption{Numero max di adulti di svasso piccolo osservati, in relazione alla fenologia annuale, dal 2004 al 2013
}
\label{Termine_tab_1}
\end{table}


\begin{table}[!h]
\small
\scalebox{.75}{
\begin{tabular}{>{\raggedleft\arraybackslash}p{.15\columnwidth}>{\raggedleft\arraybackslash}p{.13\columnwidth}>{\raggedleft\arraybackslash}p{.1\columnwidth}>{\raggedleft\arraybackslash}p{.13\columnwidth}>{\raggedleft\arraybackslash}p{.13\columnwidth}>{\raggedleft\arraybackslash}p{.13\columnwidth}>{\raggedleft\arraybackslash}p{.13\columnwidth}>{\raggedleft\arraybackslash}p{.15\columnwidth}}
\hiderowcolors
\toprule
\textbf{Data} & \textbf{Singoli} & \textbf{Coppie} & \textbf{Tot. Adulti} & \textbf{Pulli} & \textbf{Juv.} & \textbf{Tot. Pulli + Juv.} & \textbf{Os\allowbreak ser\allowbreak va\allowbreak to\allowbreak re$^{*}$} \\
\toprule
%\showrowcolors
\textbf{31.V.2012} &   & \textbf{20} & \textbf{40}&  29 &  & \textbf{29} & RT \\
\textbf{02.VI.2012} & & \textbf{25} & \textbf{50}&  42 &  & \textbf{42} & RT \\
\textbf{06.VII.2012} & & \textbf{25} & \textbf{50}&   & 49 & \textbf{49} & RT \\
\textbf{05.VIII.2012} & \textbf{5} & 20 & \textbf{45}&   & 48 & \textbf{48} & RT, NC \\
\textbf{12.IX.2012} & \textbf{46} &  & \textbf{46}&   & 40 & \textbf{40} & RT \\
\textbf{20.IV.2013} & \textbf{18} &  & \textbf{18}&   &  & \textbf{0} & RT \\
\textbf{14.V.2013} & \textbf{20} &  & \textbf{20}&   &  & \textbf{0} & RT \\
\textbf{24.V.2013} & \textbf{41} &  & \textbf{41}&   &  & \textbf{0} & RT, GC \\
\textbf{09.VI.2013} & \textbf{58} &  & \textbf{58}&   &  & \textbf{0} & RT \\
\textbf{22.VI.2013} & \textbf{39} & 21 & \textbf{81}&  50 &  & \textbf{50} & RT \\
\textbf{02.VII.2013} & \textbf{52} & 28 & \textbf{108}&  93 & 2 & \textbf{95} & RT \\
\textbf{13.VII.2013} & \textbf{50} & 38 & \textbf{126}&  102 & 2 & \textbf{104} & RT \\
\textbf{21.VII.2013} & \textbf{90} & 19 & \textbf{128}&  42 & 63 & \textbf{105} & RT \\
\textbf{05.VIII.2013} & \textbf{120} & 2 & \textbf{124}&  5 & 96 & \textbf{101} & RT \\
\textbf{11.VIII.2013} & \textbf{85}& 19 & \textbf{123}&  13 & 99 & \textbf{112} & RT \\
\textbf{03.IX.2013} & \textbf{120} &  & \textbf{120}&   & 108 & \textbf{108} & RT \\
\hiderowcolors
\multicolumn{8}{l}{}\\
\multicolumn{8}{l}{$^{*}$ GC: Giovanni Cumbo, NC: Natalino Cuti, RT: Rosa Termine}\\
\bottomrule
\end{tabular}
}
\caption{Osservazioni di svasso piccolo durante le nidificazioni del 2012 e del 2013}
\label{Termine_tab_2}
\end{table}

\begin{figure}[!h]
\centering
\includegraphics[width=.98\columnwidth]{Termine_fig_1.jpg}
\caption{Andamento del numero max di adulti di svasso piccolo osservati tra XI.2008 e IX.2013, in relazione alla fenologia annuale}
\label{Termine_fig_1}
\end{figure}

\section*{Risultati e discussione}

Le osservazioni di svasso piccolo, dal 2004 al settembre 2013, sono
riassunte in tabella 1, nella quale si riporta il numero massimo di
individui adulti osservati per mese, suddivise a seconda del periodo
fenologico annuale.

L{\textquoteright}andamento mensile, riferito a individui adulti censiti
da novembre del 2008 al settembre del 2013, sono riportate in figura 1.

Una seconda tabella (Tab. \ref{Termine_tab_2}), invece, riporta solo le osservazioni dello
svasso piccolo nel periodo tra maggio 2012 e settembre 2013,
differenziando tra adulti (singoli e coppie) e nuovi nati
(\textit{pulli} e \textit{juvenes}).

{Tra il 2003 e il 2004 tale specie \`e stata osservata
a Pergusa solo durante il periodo migratorio e durante il periodo di
svernamento} (Termine \textit{et al}. 2008){, mentre
presenze estive sono state registrate tra il 2005 e il 2009 (}Ientile
\textit{et al}. 2010; {Termine
}\textit{{et al.}}{ 2011). Le
osservazioni da novembre del 2008 a settembre del 2013 (Fig. \ref{Termine_fig_1}) hanno
permesso di rilevare una costante presenza estiva dello svasso piccolo
nel lago dal 2009 al 2013 e di accertarne la nidificazione negli anni
2010, 2012 (Fig. \ref{Termine_fig_2}) e 2013 (Fig. \ref{Termine_fig_3}).}

Con questa nota si documenta, quindi, il ritorno come nidificante dello
svasso piccolo \textit{Podiceps nigricollis} C. L. Brehm, 1831 al lago
di Pergusa, e la rilevanza di questa zona umida che ne ospita il
maggior numero di coppie nidificanti per la Sicilia e per
l{\textquoteright}intera Italia (Verducci \& Sighele 2013).

{Infatti, dopo 52 anni dalla storica nidificazione
(Krampitz 1958), nel 2010 al lago di Pergusa \`e stata accertata la}
presenza di 6 coppie nidificanti (Ientile \textit{et al}. 2010; Termine
\textit{et al}. 2011).

Nel 2012 presso la Riserva pergusina, oltre alle consuete presenze
invernali, a partire da fine maggio sono state osservate 20 coppie di
svasso piccolo; i primi \textit{pulli}, in numero di 29, sono stati
osservati il 31 maggio; durante il censimento del 2 giugno sono state
contate 25 coppie e 42 \textit{pulli}; a luglio sono state censite 25
coppie e 49 \textit{juv.} (Tab. \ref{Termine_tab_1}).

All{\textquoteright}inizio della primavera del 2013 sono stati osservati
presso la Riserva del lago di Pergusa 18 individui di svasso piccolo in
abito nuziale; le presenze della specie sono diventate notevolmente
pi\`u consistenti dopo il 24 maggio con 41 individui. A partire da
questa data, sono stati eseguiti ulteriori censimenti di tale specie a
cadenza massima quindicinale; i primi \textit{pulli}, in numero di 50,
sono stati osservati il 22 giugno; poi si \`e avuto un graduale
incremento fino al 21 luglio, arrivando a contare 128 adulti e 105
nuovi nati; fino all{\textquoteright}11 agosto sono state registrate
ulteriori nascite, raggiungendo il numero massimo di 112 tra
\textit{pulli} e \textit{juv.} (Tab. \ref{Termine_tab_2}).

Diversi sono i fattori che possono avere influito sulla nidificazione
dello svasso piccolo; tra questi la disponibilit\`a trofica, la
tipologia di vegetazione ripariale, la presenza di piante acquatiche
{emergenti, il rischio di predazione, la naturale
espansione della specie, la pressione antropica, etc. Probabilmente} al
lago di Pergusa la disponibilit\`a di risorse trofiche, tra cui la
notevole presenza di piccoli pesci, anfibi e invertebrati vari, ha
determinato una forte attrazione per gli svassi piccoli negli ultimi
anni, soprattutto nel periodo estivo.

Un altro fattore influente potrebbe essere stato
l{\textquoteright}espansione delle fanerofite \textit{Ruppia} sp. L. e
\textit{Potamogeton} \textit{pectinatus} L. con formazione di grandi
isole galleggianti (\textit{mats}) che gi\`a nel 2010 occupavano circa
1/3 dello specchio lacustre, mentre nel 2013 ne hanno occupato circa
3/4. La vegetazione galleggiante offre allo svasso piccolo un buon
substrato dove costruire il nido galleggiante, in grado di adeguarsi
alle variazioni del livello dell{\textquoteright}acqua.

\section*{Conclusioni}

{Il ripristino di condizioni ottimali del lago, sia in
termini di tutela, che ambientali, ha certamente favorito la
nidificazione dello svasso piccolo. Le nidificazioni della specie
avvenute a Pergusa nel 2010, 2012 e 2013 hanno coinciso con la presenza
di fanerofite galleggianti, situazione che non si \`e verificata nel
2011; infatti, tale vegetazione acquatica era pressoch\'e assente per
l{\textquoteright}eccessivo numero di
}\textit{{Cyprinus carpio}}{
(successivamente andati incontro a massiva moria
}nell{\textquoteright}autunno 2011 per la fioritura di
\textit{Prymnesium parvum }Carter, alga ittiotossica) il cui spettro
alimentare comprende anche alghe e macrofite acquatiche.
L{\textquoteright}assenza di tale vegetazione ha probabilmente
scoraggiato nel 2011 la riproduzione dello svasso piccolo nel Lago.

Pertanto sarebbe interessante comprendere se e in che modo la presenza
di tali isole galleggianti abbia rappresentato un fattore favorevole
all{\textquoteright}insediamento dei soggetti nidificanti di questa
specie. {Il lago di Pergusa ha una grande
vulnerabilit\`a; per la sua natura endoreica e la limitata estensione e
profondit\`a, \`e, infatti, molto sensibile ai cambiamenti climatici in
corso che portano ad una graduale diminuzione della piovosit\`a nella
regione. Inoltre, essendo strettamente a contatto con territori
urbanizzati, la sua sopravvivenza dipende anche dalle scelte da parte
dell{\textquoteright}uomo}; ci\`o dovrebbe far comprendere quanto sia
fondamentale effettuare una pianificazione urbanistica che preveda la
realizzazione di specifiche zone tampone a confine con gli ambienti
naturali, evitando la brusca interruzione di questi ambienti; cos\`i si
potrebbe ottenere il giusto compromesso tra le esigenze
socio-economiche e la conservazione di questa area protetta e quindi
anche della specie in argomento.

\begin{figure}[!h]
\centering
\includegraphics[width=.6\columnwidth]{Termine_fig_2.jpg}
\caption{Adulto con \textit{juvenes}, a Pergusa nel 2012 (Foto di R. Termine)}
\label{Termine_fig_2}
\end{figure}

\begin{figure}[!h]
\centering
\includegraphics[width=.7\columnwidth]{Termine_fig_3.jpg}
\caption{Svasso piccolo: adulto con \textit{pullus} sul dorso, a Pergusa nel 2013 (Foto di R. Termine)}
\label{Termine_fig_3}
\end{figure}

\section*{Ringraziamenti}
Rivolgiamo un doveroso ringraziamento alla Provincia Regionale di Enna,
ente gestore della R.N.S. lago di Pergusa, che ha promosso queste
ricerche, a Natalino Cuti e a Giovanni Cumbo per la generosa
disponibilit\`a.

\section*{Bibliografia}
\begin{itemize}\itemsep0pt
 \item {Benoit L., 1840 -
}\textit{{Ornitologia Siciliana}}{.
Stamperia G. Fiumara, Messina.}

 \item {Brichetti P.} \&{ Fracasso G., 2013 -
}\textit{{Ornitologia Italiana. Vol. 1.
Gaviidae-Falconidae}}{.
}\textit{{Edizione elettronica riveduta e
aggiornata}}{. Alberto Perdisa Ed., Bologna.}

 \item {Corso A., 2005 - }\textit{{Avifauna
di Sicilia}}{. L{\textquoteright}Epos Societ\`a
Editrice, Palermo.}

 \item {Corso A. in Ruggieri L. \& Sighele M. (red.), 2007 -
}\textit{{Annuario 2006}}{. EBN
Italia, Verona, 10 pp.}

 \item {Di Blasi F., 2011 - Comunicazione
personale}\textit{{. }}{LIPU Saline d
Priolo.}

 \item {Doderlein P., 1873 - Avifauna del Modenese e della
Sicilia. }\textit{{Giorn. Sci. Nat.
}}\textit{{Econom.,}}{ 5: 265-328.
Iapichino C. \& Massa B., 1989 - The Birds of Sicily.
}\textit{{British Ornithologist{\textquoteright}Union,
}}{Check-list, 11: 1-170.}

 \item {Ientile R., Termine R. \& Siracusa A. M., 2010 -
Nidificazione di Svasso piccolo }\textit{{Podiceps
nigricollis}}{ C. L. Brehm, 1831 (Aves
Podicipediformes) nella Riserva Naturale Speciale
Lago}\textit{{ }}{di Pergusa (Enna).
}\textit{{Naturalista  sicil}}{., S.
IV, XXXIV (3-4): 543-544.}

 \item {Krampitz H. E., 1958 - Weiteres uber die Brutvogel
Siziliens. }\textit{{J. Orn.}}{, 99:
39-58.}

 \item {Lo Valvo M., Massa B. \& Sar\`a M., 1993 - Uccelli e
paesaggio in Sicilia alle soglie del terzo millennio.
}\textit{{Naturalista sicil}}{., vol.
XVII, suppl.: 1- 371.}

 \item {Mascara R., 2007 - L{\textquoteright}avifauna degli
invasi artificiali di Cimia, Comunelli e Disueri (Caltanissetta,
Sicilia). Aggiornamento 1993-2006. }\textit{{Uccelli
d{\textquoteleft}Italia}}{, XXXII: 9-20.}

 \item {Massa B. \& Schenk H., 1983 - Similarit\`a tra le
avifaune della Sicilia, Sardegna e Corsica.
}\textit{{Lav. Soc. It.
Biogeografia}}{, 8 (1980): 757-799.}

 \item {Rallo G. \& Pandolfi M., 1988 -
}\textit{{Le zone umide del
Veneto}}{. Muzzio ed., Padova.}

 \item {Termine R., Canale E. D., Ientile R., Cuti N., Di
Grande C. S. \& Massa B., 2008 - Vertebrati della Riserva Naturale
Speciale e Sito d{\textquotesingle}Importanza Comunitaria Lago di
Pergusa. }\textit{{Naturalista
sicil.}}{, 32: 105-186.}

 \item {Termine R., Ientile R. \& Siracusa M. A., 2011 -
Nidificazione di Svasso piccolo nella Riserva Naturale Speciale del
Lago di Pergusa. }\textit{{Biologi
Italiani}}{, XLI, n{\textdegree} 2: 42-46.}

 \item {Verducci D. \& Sighele M., 2013 - La nidificazione
dello Svasso piccolo }\textit{{Podiceps
nigricollis}}{ in Italia.
}\textit{{U.D.I.}}{, XXXVIII: 39-48.}
\end{itemize}

\setcounter{figure}{0}
\setcounter{table}{0}

\begin{adjustwidth}{-3.5cm}{0cm}
\pagestyle{CIOpage}
\authortoc{\textsc{Trotti P.}, \textsc{Bassi E.}, 
\textsc{Bionda R.}, \textsc{Ferloni M.}, 
\textsc{Rubolini D.}}
\chapter*[Ecologia del gufo reale in due aree lombarde]{Ecologia e produttivit\`a del gufo reale \textbf{\textit{Bubo
bubo}}\textbf{ in due aree di studio della Lombardia}}
\addcontentsline{toc}{chapter}{Ecologia del gufo reale in due aree lombarde}

\textsc{Paolo Trotti}$^{1*}$, \textsc{Enrico Bassi}$^{2}$, 
\textsc{Radames Bionda}$^{3}$, \textsc{Maria Ferloni}$^{4}$, 
\textsc{Diego Rubolini}$^{1}$\\

\index{Trotti Paolo} \index{Bassi Enrico} \index{Bionda Radames} \index{Ferloni Maria} \index{Rubolini Diego}
\noindent\color{MUSEBLUE}\rule{27cm}{2pt}
\vspace{1cm}
\end{adjustwidth}


\marginnote{\raggedright $^1$Dipartimento di Bioscienze, Universit\`a degli Studi
di Milano, via Celoria 26, 20133, Milano \\
$^2$Consorzio del Parco Nazionale dello Stelvio, via De
Simoni 42, 23032, Bormio (SO) \\
$^3$Parco Naturale Alpe Veglia e Devero - Alta Valle
Antrona, viale Pieri 27, 28868 Varzo (VB) \\
$^4$Provincia di Sondrio Uff. Faunistico, via XXV Aprile,
23100, Sondrio \\
\vspace{.5cm}
{\emph{\small $^*$Autore per la corrispondenza: \href{mailto:paolotrotti6@libero.it}{paolotrotti6@libero.it}}} \\
\keywords{Alpi e Prealpi lombarde, \textit{Bubo bubo}, densit\`a,
successo riproduttivo, selezione habitat}
{Central Alps and Prealps, \textit{Bubo bubo},
density, reproductive success, habitat selection}
%\index{keywords}{Alpi e Prealpi lombarde} \index{keywords}{\textit{Bubo bubo}} \index{keywords}{Densit\`a}
%\index{keywords}{Successo riproduttivo} \index{keywords}{Selezione habitat}
}
{\small

\noindent \textsc{\color{MUSEBLUE} Summary} / CWe studied 31 pairs of eagle owl \textit{Bubo bubo} in two different
areas in northern Italy from 2010 to 2012: 6 pairs in area 1 in
Valtellina (Sondrio province) characterized by a \textit{Nearest
Neighbour Distance }(NND) of 4,280 {\textpm} 700 m with a density value
of 2.4 pairs 100 km\textsuperscript{{}2} and 25 pairs in area 2 in
Camonica Valley-Iseo lake surroundings (Bergamo and Brescia provinces)
with a NND of 2,971 {\textpm} 1,349 m and a density value of 5.2 pairs
100 km\textsuperscript{{}2}. In order to determine the habitat
preferences of the species we compared the landscape features within a
radius of 1000 m around each occupied (31) and unoccupied (33) cliffs.
Eagle owl preferred the most extensive and favourably orientated
cliffs, greater extension of open areas and woody crops that in our
study area were mainly represented by olive groves and vineyards. The
reproductive success was 44.4\% and the mean number of fledglings per
pair was 0.71. This parameter was positively influenced by the water
bodies and active quarries. \\
\noindent \textsc{\color{MUSEBLUE} Riassunto} / La ricerca ha indagato 31 coppie di gufo reale in due aree della
Lombardia dal 2010 al 2012: 6 coppie nell{\textquoteright}area 1 in
bassa Valtellina (provincia di Sondrio), caratterizzata da una NND di
4.280 {\textpm} 700 m e una densit\`a di 2,4 coppie per 100
km\textsuperscript{{}2}, e 25 coppie nell{\textquoteright}area 2 in
valle Camonica e nei dintorni del lago d{\textquoteright}Iseo (province
di Bergamo e Brescia) con NND di 2.971 {\textpm} 1.349 m e un valore di
densit\`a pari a 5,2 coppie per 100 km\textsuperscript{{}2}. Per
valutare quali variabili ambientali discriminassero i siti di presenza
da quelli di assenza, \`e stata eseguita un{\textquoteright}analisi
della selezione dell{\textquoteright}habitat
all{\textquoteright}interno di un buffer con raggio di 1000 m, tra i 31
siti di presenza e i 33 siti di assenza. La specie ha mostrato una
preferenza per le pareti rocciose pi\`u estese e meglio esposte al
sole, per maggiori estensioni di aree aperte e di colture legnose
rappresentate soprattutto da vigneti e oliveti. Il successo
riproduttivo \`e pari a 44,4\% e il numero medio di giovani involati
per coppia controllata \`e paria a 0,71. Tale parametro \`e influenzato
positivamente dalla maggiore estensione dei corpi idrici e delle cave
di versante.
}



\section*{Introduzione}

La ricerca si \`e posta l{\textquoteright}obiettivo di indagare la
presenza del gufo reale \textit{Bubo bubo,} in due aree della
Lombardia, al fine di valutarne la densit\`a, il successo riproduttivo,
le caratteristiche ambientali dei territori di nidificazione e
l{\textquoteright}influenza dell{\textquoteright}ambiente sul successo
riproduttivo. Inoltre \`e stato possibile confrontare il successo
riproduttivo di alcune coppie indagate (N= 11) con la serie storica
della produttivit\`a raccolta per le stesse coppie nel periodo
1999-2001 (Bassi 2001).

\section*{Metodi}

La ricerca si \`e concentrata in due aree lombarde: la prima, inclusa in
provincia di Sondrio (area 1: bassa Valtellina), ha estensione pari a
115 km\textsuperscript{2} e quota tra i 200 e i 1.150 m s.l.m.
L{\textquoteright}area 2 \`e invece situata tra le province di Bergamo
e Brescia e comprende le due sponde del lago d{\textquoteright}Iseo, la
val Cavallina, parte della val Borlezza e della valle Camonica, con
un{\textquoteright}estensione di 520 km\textsuperscript{2} e quota tra
i 185 e i 1200 m s.l.m. Nel triennio 2010-2012 sono state indagate
l{\textquoteright}area 1 e una porzione dell{\textquoteright}area 2
posta in provincia di Bergamo, gi\`a indagata intensivamente in passato
(Bassi 2001) mentre la porzione restante dell{\textquoteright}area 2,
posta in provincia di Brescia, \`e stata indagata nella sola stagione
riproduttiva 2012. I metodi utilizzati sono stati:
l{\textquoteright}ascolto sistematico del canto spontaneo degli adulti
territoriali (dicembre-marzo) e dei giovani (maggio-luglio), la
stimolazione con richiamo registrato (\textit{playback}) e la ricerca
diurna dei nidi e delle tracce di presenza. La distanza tra i territori
\`e stata calcolata con il metodo della \textit{Nearest Neighbour
Distance }(NND) mentre per il calcolo della densit\`a \`e stato
utilizzato il metodo dei \textit{buffer} che si basa sul valore medio
della NND utilizzato come raggio di un \textit{buffer }circolare
attorno al centro di ogni territorio; l{\textquoteright}area cos\`i
individuata costituisce l{\textquoteright}area di studio (Bionda 2002).
Per il calcolo della dispersione dei nidi \`e stato utilizzato il Test
G; valori superiori a 0,65 indicano una distribuzione uniforme dei
territori (Brown \& Rothery 1978). I principali parametri riproduttivi
calcolati sono stati il successo riproduttivo (percentuale delle coppie
riprodottesi con successo sulle coppie totali), il numero di giovani
involati su coppie totali e il numero di giovani involati su coppie di
successo.

Per le analisi ambientali nelle province considerate \`e stata
utilizzata la carta di uso del suolo D.U.S.A.F. 2.1 (Destinazione
d{\textquoteright}Uso dei Suoli Agricoli e Forestali) del 2007.

Le caratteristiche ambientali sono state calcolate
all{\textquoteright}interno di \textit{buffer }circolari di 1 km di
raggio attorno alle pareti di presenza e assenza. Le variabili
ambientali presenti all{\textquoteright}interno del \textit{buffer
}sono state raggruppate in 7 gruppi (aree aperte, aree boscate, colture
legnose, cespuglieti, corpi idrici, cave e aree urbanizzate). Le
caratteristiche della parete rocciosa sono state calcolate secondo la
metodologia adottata da Brambilla \textit{et al. }(2010) mentre, per
l{\textquoteright}esposizione della parete, \`e stato utilizzato un
sistema di punteggio che attribuisce valori maggiori alle pareti pi\`u
esposte al sole (N-NW=1; W=2; NE=3; E=4; SW=5; S-SE=6).

\`E stato effettuato un confronto descrittivo delle variabili ambientali
tra pareti occupate e pareti non occupate attraverso il test \textit{t
}di Student. L{\textquoteright}analisi ha fornito informazioni su quali
variabili potessero influenzare la presenza della specie. Per valutare
le variabili ambientali influenzanti la selezione
dell{\textquoteright}habitat \`e stata eseguita
un{\textquoteright}analisi di regressione logistica mentre, per
l{\textquoteright}analisi dell{\textquoteright}influenza
dell{\textquoteright}ambiente sul numero di giovani involati per coppia
\`e stato utilizzato un modello misto assumendo una distribuzione
poissoniana dell{\textquoteright}errore. 

\section*{Risultati e discussione}

Sono stati individuati 31 territori di gufo reale (6
nell{\textquoteright}area 1 e 25 nell{\textquoteright}area 2). Nel
settore bergamasco di quest{\textquoteright}ultima area sono stati
confermati, a distanza di 10 anni, 11 territori indagati nel periodo
1999-2001 (Bassi 2001) e scoperti 2 nuovi territori. Nel settore
bresciano di quest{\textquoteright}area sono stati riconfermati 4
territori noti (Bertoli \& Leo \textit{\textcolor{black}{ined.}}) e
scoperti 8 nuovi territori. La media NND per l{\textquoteright}area 1
\`e di 4280 {\textpm} 700 m (\textit{range} 3720-5330 m) (Bassi
\textit{et al. }2011) e per l{\textquoteright}area 2 di 2971 {\textpm}
1349 m (\textit{range} 847-5.435). 

I valori di densit\`a variano da 2,4 cp 100 km\textsuperscript{{}2},
per l{\textquoteright}area 1, a 5,2 cp 100 km\textsuperscript{{}2} per
l{\textquoteright}area 2. 

La densit\`a registrata nell{\textquoteright}area 1 rientra nella media
riportata per le Alpi (Casanova \& Galli 1998; Marchesi \textit{et al.
}1999; Bionda 2002; Bassi \textit{et al}. 2003) mentre, quella
dell{\textquoteright}area 2, costituisce uno dei valori pi\`u alti tra
quelli riportati a livello nazionale che sottolinea la particolare
vocazione di questo territorio caratterizzato
dall{\textquoteright}elevata disponibilit\`a di pareti rocciose a
ridosso di laghi e ampi fondovalle non eccessivamente urbanizzati.

Il test G ha evidenziato una distribuzione regolare e uniforme dei siti
con un valore di 0,94 per l{\textquoteright}area 1 e di 0,67 per
l{\textquoteright}area 2. Il successo riproduttivo \`e pari al 44,4\%,
in linea con alcuni studi alpini (Marchesi \textit{et al. }2002; Bionda
2002). Il numero medio di giovani involati per coppia controllata \`e
di 0,71 mentre il numero medio di giovani involati per coppia di
successo \`e di 1,61. Nelle due aree per cui sono disponibili serie
triennali di dati si \`e riprodotto con successo il 27,8\% (N= 18) e il
54,5\% (N= 33) delle coppie presenti. Il confronto del successo
riproduttivo nel periodo 1999-2001 per gli 11 territori della porzione
bergamasca dell{\textquoteright}area 2 (Bassi \textit{et al. }2003) con
quello per gli stessi nidi negli anni 2010-2012, ha evidenziato un
marcato declino (-40,7\%) cos\`i come il numero medio di giovani
involati sul totale delle coppie controllate \`e diminuito da 1,19 a
0,79. Tale risultato potrebbe dipendere dal progressivo consumo di
territorio sul fondovalle e dal sempre pi\`u marcato processo di
rimboschimento dei versanti che portano alla diminuzione di specie
preda importanti, legate alle aree pi\`u aperte.

Nel triennio 2010-2012, soltanto 5 delle 17 coppie seguite (29,4\%) si
sono riprodotte con successo ogni anno (min 1, max 3 giovani/coppia)
suggerendo l{\textquoteright}ipotesi che esistano coppie con una
maggiore \textit{fitness} riproduttiva, detentrici di territori
migliori da un punto di vista trofico e presentanti minori fattori di
mortalit\`a (ad es. elettrocuzione).

Le analisi statistiche hanno evidenziato una preferenza della specie per
le pareti rocciose pi\`u ampie e meglio esposte al sole, che possono
conferire una maggior protezione da eventuali disturbi e migliori
condizioni climatiche, una maggiore estensione di colture legnose
(vigneti e oliveti sui versanti pi\`u termofili che possono attrarre un
maggior numero di specie preda) e di aree aperte, fondamentali per la
caccia (Penteriani \textit{et al. }2001). Il numero di giovani involati
per coppia \`e influenzato positivamente
dall{\textquoteright}estensione dei corpi idrici e delle cave di
versante. La prima variabile, oltre a incrementare il potenziale
spettro trofico del predatore, \`e probabilmente una stima indiretta
della densit\`a della specie preda preferita (\textit{Rattus
norvegicus}) che, per il nord dell{\textquoteright}Italia, risulta
particolarmente consistente presso i corpi idrici (Sergio \textit{et
al}. 2004). Le cave, invece, influenzano positivamente il successo
riproduttivo sia perch\'e il divieto di accesso conferisce una maggior
protezione (al loro interno sono infatti interdette le attivit\`a di
caccia e arrampicata sportiva) sia perch\'e determinano un maggior
grado di biodiversit\`a che deriva dalla presenza di ecotoni e stagni
di cava (Bassi 2003). 

\section*{Ringraziamenti}

Si ringraziano sentitamente gli ornitologi Bertoli Roberto e Leo Rocco
per la generosa condivisione di alcuni dati e gli Agenti della Polizia
provinciale di Sondrio Mozzetti Ettore, Bernardara Enos, Ronconi
Antonio, Luciani Fausto, Pasini Massimiliano per
l{\textquoteright}attivit\`a di campo.

\section*{Bibliografia}
\begin{itemize}\itemsep0pt
	\item Bassi E., 2001 - Scelta del sito di nidificazione del Gufo reale
(\textit{Bubo bubo}, Strigiformes, Aves) nel Settore orientale delle
Prealpi Bergamasche. Tesi di Laurea, Universit\`a degli Studi di Pavia.

	\item Bassi E., 2003 - Importanza degli ambienti di cava per
l{\textquotesingle}insediamento del gufo reale \textit{Bubo bubo}.
\textit{Avocetta}, 27: 127.

	\item Bassi E., Bonvicini P. \& Galeotti P., 2003 - Successo riproduttivo e
selezione del territorio di nidificazione del Gufo reale \textit{Bubo
bubo} nelle Prealpi bergamasche\textit{. Avocetta}, 27: 97.

	\item Bassi E., Bionda R., Trotti P., Folatti M.G. \& Ferloni M., 2011 -
Mitigazione dell{\textquoteright}impatto delle linee elettriche per la
conservazione del gufo reale \textit{Bubo bubo} in provincia di
Sondrio. Atti XV Convegno Nazionale di Ornitologia, Cervia (RA), 22-25
settembre 2011 (in stampa).

	\item Bionda R., 2002 - Censimento di Gufo reale \textit{Bubo bubo} nella
provincia del Verbano Cusio Ossola. I Convegno Italiano Rapaci diurni e
notturni. Preganziol (TV) 9-10 marzo.

	\item Brambilla M., Bassi E., Ceci C. \& Rubolini D., 2010 - Environmental
factors affecting patterns of distribution and co-occurrence of two
competing raptor species. \textit{Ibis}, 152 (2): 310-322.

	\item Brown D. \& Rothery P., 1978 - Randomness and local regularity of points
in a plane. \textit{Biometrica}, 65: 115-122.

	\item Casanova M. \& Galli L., 1998 - Primi dati sulla biologia del Gufo
reale, \textit{Bubo bubo}, nel Finalese (Liguria occidentale).
\textit{Riv. ital. Orn.}, 68 (2): 167-174.

	\item Marchesi L., Pedrini P. \& Galeotti P., 1999 - Densit\`a e dispersione
territoriale del Gufo reale \textit{Bubo bubo} in provincia di Trento
(Alpi centro-orientali). \textit{Avocetta}, 23: 19-23.

	\item Marchesi L., Sergio F. \& Pedrini P., 2002 - Costs and benefits of
breeding in human-altered landscapes for the eagle owl \textit{Bubo
bubo}. \textit{Ibis}, 144, E164--E177.

	\item Penteriani V., Gallardo M., Roche P. \& Cazassus H., 2001 - Effects of
landscape spatial structure and composition on the settlement of the
eagle owl \textit{Bubo bubo }in a mediterranean habitat.
\textit{Ardea}, 89 (2): 331-340.

	\item Sergio F., Marchesi L. \& Pedrini P., 2004 - Integrating individual
habitat choice and regional distribution of a biodiversity indicator
and top predator. \textit{J Biogeogr.,} 31: 619--628.
\end{itemize}

\setcounter{figure}{0}
\setcounter{table}{0}

\begin{adjustwidth}{-3.5cm}{0cm}
\pagestyle{CIOpage}
\authortoc{\textsc{Verza E.}}
\chapter*[La pernice di mare e la sterna zampenere nel Delta del
Po]{Popolazione e scelta dell{\textquoteright}habitat riproduttivo
di pernice di mare \textbf{\textit{Glareola pratincola }}\textbf{e
sterna zampenere }\textbf{\textit{Gelochelidon nilotica}}\textbf{ nella
parte veneta del Delta del Po (Rovigo); analisi del periodo 2001-2012}}
\addcontentsline{toc}{chapter}{La pernice di mare e la sterna zampenere nel Delta del
Po}

\textsc{Emiliano Verza}$^{1*}$\\

\index{Verza Emiliano}
\noindent\color{MUSEBLUE}\rule{27cm}{2pt}
\vspace{1cm}
\end{adjustwidth}


\marginnote{\raggedright $^1$Ass. Sagittaria, via Badaloni 9, 45100 Rovigo \\
\vspace{.5cm}
{\emph{\small $^*$Autore per la corrispondenza: \href{mailto:sagittaria.at@libero.it}{sagittaria.at@libero.it}}} \\
\keywords{Delta del Po, \textit{Glareola pratincola},
\textit{Gelochelidon nilotica}, nidificazione}
{Po Delta, \textit{Glareola pratincola}, \textit{Gelochelidon nilotic}, nesting}
%\index{keywords}{Delta del Po} \index{keywords}{\textit{Glareola pratincola}}
%\index{keywords}{\textit{Gelochelidon nilotic}} \index{keywords}{Nidificazione}
}
{\small
\noindent \textsc{\color{MUSEBLUE} Summary} / Collared pratincole \textit{Glareola pratincola} and gull-billed tern
\textit{Gelochelidon nilotica}, breed in bracksih marsehs (valli) of
the Po river Delta (Rovigo district). The surveys started form their
first settlement: 2004 for \textit{Glareola pratincola}, 2001 for
\textit{Gelochelidon nilotica}.

In this area they nest only on islands or dried beds of marshes.
\textit{Gelochelidon nilotica} bred with 40 -- 246 pairs (mean 157)
(2001-12); \textit{Glareola pratincola} with 1-22 pairs (mean 10,4).

The most important limiting factors are spring downpours and artificial
rising of water levels. \\
\noindent \textsc{\color{MUSEBLUE} Riassunto} / La pernice di mare \textit{Glareola pratincola }e la sterna zampenere
\textit{Gelochelidon nilotica}\textbf{ }sono specie che nidificano
all{\textquoteright}interno dei complessi vallivi della parte veneta
del Delta del Po (provincia di Rovigo). Sono state monitorate con
apposite campagne di censimento a partire dal loro insediamento,
avvenuto nel 2004 per la pernice di mare e nel 2001 per la sterna
zampenere.
}

La riproduzione avviene esclusivamente su barene o fondali di valle,
temporaneamente prosciugati. Per il periodo 2001-2012 la sterna
zampenere \`e presente con una popolazione nidificante compresa tra 40
e 246 coppie (media 157); la pernice di mare, invece, con 1-22 coppie
(media 10,4).

I principali fattori limitanti sono rappresentati dagli acquazzoni
primaverili e da improvvisi innalzamenti del livello idrico indotti
artificialmente per motivi di produzione ittica.





\section*{Introduzione}

La pernice di mare e la sterna zampenere sono specie, in Veneto,
tipicamente costiere, concentrate per la maggior parte nel Delta del Po
(provincia di Rovigo). Dato il loro status a livello nazionale, vengono
seguite con apposite campagne di monitoraggio da oltre un decennio: i
dati raccolti e solo in parte gi\`a pubblicati (Fracasso \textit{et
al}. 2003; Verza \& Trombin 2012) vengono qui sintetizzati. 
L{\textquoteright}area di indagine riguarda tutta la parte venete del
Delta del Po, ricadente in provincia di Rovigo.

\section*{Metodi}

Il periodo considerato va dal primo insediamento delle specie, quindi
dal 2001 fino alla stagione riproduttiva 2012. I dati sono stati
raccolti con campagne di censimento standardizzate annuali; per ogni
annata \`e stata svolta almeno un{\textquoteright}uscita di censimento,
ma per molti anni sono stati svolti censimenti completi due volte al
mese, da marzo a giugno.

\section*{Risultati e discussione}

\textbf{Sterna zampenere} - La sterna zampenere ha iniziato a nidificare con certezza in Veneto nel
2001, all{\textquoteright}intero in una valle della provincia di Rovigo
(Fracasso \textit{et al}. 2003). Da allora la popolazione \`e andata
rapidamente consolidandosi, con progressiva espansione a molte valli
del Delta: dapprima il complesso delle valli
Sacchetta-Canocchione-Moraro, poi nella vicina valle
Ca{\textquoteright} Pasta, rimasta per alcuni anni unico sito di
nidificazione, successivamente all{\textquoteright}interno delle valli
di Porto Tolle e Rosolina. La media delle coppie nidificanti per il
periodo 2001 -- 2012 \`e di 157 coppie, escludendo le annate 2004, 2005
e 2006 per le quali la qualit\`a dei dati non \`e sufficiente. Tale
popolazione risulta essere 1/3 del popolamento italiano, stimato in
circa 550 coppie nel 2002 (Brichetti  \& Fracasso 2004).

La nidificazione avviene spesso in consociazione con altri Caradriformi,
tra cui sterna comune e fratino. Si riproduce esclusivamente
all{\textquoteright}interno delle valli, su barene e isolotti privi di
vegetazione o con vegetazione a salicornie e \textit{Phragmites
australis} rada. Il periodo di nidificazione pu\`o protrarsi fino al
mese di luglio. Gli adulti riproduttivi si alimentano quasi
esclusivamente nelle acque dolci dei limitrofi rami del Po e dei canali
di bonifica in ambiente agrario.

La sua riproduzione \`e influenzata sostanzialmente
dall{\textquoteright}andamento climatico stagionale e della gestione
antropica valliva. Forti piogge in maggio e giugno possono determinare
il fallimento della nidificazione o lo spostamento delle colonie in
altri siti. La gestione delle barene e dei livelli idrici, effettuata a
scopi produttivi, determina la creazione o la distruzione dei siti
idonei alla nidificazione

%\rowcolors{2}{white!60!MUSEBLUE}{white}
\begin{adjustwidth}{-3.5cm}{-1cm}
\begin{table}[!h]
\centering
\footnotesize
\scalebox{.8}{
\newcolumntype{S}{>{\centering\arraybackslash}p{.08\columnwidth}}
\begin{tabular}{>{\raggedright\arraybackslash}p{.2\columnwidth}SSSSSSSSSSSS}
\toprule
\textbf{Luogo}&\textbf{2001}&\textbf{2002}&\textbf{2003}&\textbf{2004}&\textbf{2005}&\textbf{2006}&\textbf{2007}&\textbf{2008}&\textbf{2009}&\textbf{2010}&\textbf{2011}&\textbf{2012} \\
\toprule
Valle Morosina (Rosolina)&&&&&&&&&&10-20&7-8&90 \\
\midrule
Valle Ca' Pasta (Porto Viro)&&&&&?&10-?&100-120&110&100-130&&13& \\
\midrule
Valle Canocchione - Moraro (Porto Viro)&&&&10-100&&&&&&105-115&177-191&90 \\
\midrule
Valle Sacchetta (Porto Viro)&40&230&220-240&10-100&&&&&&&& \\
\midrule
Valle Ca' Zuliani (Porto Tolle)&&&?&&4-6&&&&&&0-5&66 \\
\midrule
Valle Ripiego (Porto Tolle)&&&&&&&&&&5&& \\
\midrule
\textbf{Totale}&40&230&220-240&10-100&?&?&100-120&110&90-130&120-140&197-217&246 \\
\bottomrule
\end{tabular}
}
\caption{Numero di coppie nidificanti di sterna zampenere nel delta del Po veneto}
\label{Verza_tab_1}
\end{table}
\end{adjustwidth}

\textbf{Pernice di mare} - La nidificazione, in Veneto, \`e fatto piuttosto recente. Un primo
insediamento \`e avvenuto nel 2002 nella zona veneziana di Valle
Vecchia di Caorle, non pi\`u accertato gli anni successivi (AsFaVe,
2003). Dal 2004 la Pernice di mare nidifica regolarmente ogni anno
nell{\textquotesingle}area del Delta del Po (valli di Porto Viro e
Porto Tolle), con una popolazione che appare consolidata e costituita
per il periodo 2004-12 da una media di 10,4 coppie. Questa popolazione
rappresenta circa il 10\% di quella nazionale, stimata nel 2000-01 in
100-150 coppie (Brichetti \& Fracasso 2004).

Le aree tipiche della Pernice di mare sono quelle alofile costiere. Per
la riproduzione sceglie superfici fangose o anche argilloso-sabbiose
prive di vegetazione, con predilezione per i fondali di laghi
temporaneamente prosciugati e, secondariamente, per le barene nude
recentemente rimaneggiate. Tali barene possono anche presentare rada
vegetazione alofila (es: generi \textit{Salsola, Aster }e
\textit{Limonium}) e devono trovarsi sopra il livello idrico massimo
estivo. La nidificazione sino ad ora \`e avvenuta esclusivamente
all{\textquotesingle}interno delle valli, in particolare Scanarello, S.
Carlo e Ca{\textquoteright} Zuliani; la specie si \`e riprodotta anche
all{\textquoteright}interno delle valli Sacchetta, S. Leonardo,
Ca{\textquoteright} Pasta, Chiusa e Ripiego; vi sono inoltre sospetti
che abbia potuto riprodursi anche in Bagliona e Ca{\textquoteright}
Pisani. La specie si riproduce sia con coppie singole, sia a gruppi di
2 o 3 coppie, sia a gruppi pi\`u consistenti sino a 11 insieme in
un{\textquoteright}unica area valliva. Si riproduce anche in
consociazione con fratino e fraticello.

\`E evidente, quindi, come tutto il complesso vallivo sia potenzialmente
idoneo alla nidificazione della specie. Data la sua biologia
riproduttiva osservata in altre aree italiane (ad es. Ferrara) \`e
possibile che le piane di bonifica possano ospitare la nidificazione
della specie, o comunque altre aree esterne alle valli da caccia e da
pesca.

L{\textquoteright}insediamento riproduttivo avviene da maggio, con pulli
osservati sino alla prima met\`a di luglio. Per
l{\textquotesingle}alimentazione utilizza, invece, le aree agrarie e di
argine adiacenti le valli (ad es. i siti Punta Paltanara, medicai di
Ca{\textquotesingle} Zuliani, penisola di S. Margherita). Fattore
determinante \`e la presenza di zone prative (prati arginali, incolti,
medicai) in grado di fornire una sufficiente quantit\`a di insetti. Si
osserva quindi un continuo pendolarismo tra le valli e le zone agrarie
di bonifica.

La grande potenzialit\`a offerta dalle zone vallive venete rappresenta
un possibile fattore di crescita della popolazione nazionale. Il
principale fattore limitante \`e rappresentato proprio dalle pratiche
di gestione attiva esercitate in valle, che da un lato sono in grado di
creare di anno in anno idonei siti di nidificazione (creazione di
barene, rimodellamento, messa in asciutta di laghi),
dall{\textquotesingle}altro mettono spesso a repentaglio le nidiate
(improvviso innalzamento dei livelli idrici).

Un grave fattore limitante \`e inoltre rappresentato dagli eventi
atmosferici negativi estivi. Nel 2010 si \`e assistito alla distruzione
della colonia di Valle Ca{\textquoteright} Zuliani a causa delle forti
piogge del mese di giugno, con ricostituzione della stessa in valle
Sacchetta nel mese di luglio.

Al fine di favorire la nidificazione della specie \`e auspicabile la
realizzazione di apposite barene vallive, livellate, prive di
vegetazione e preferibilmente costituite da materiale fangoso.

\begin{table}[!h]
\centering
%\rowcolors{2}{white!40!MUSEBLUE}{white}
\newcolumntype{S}{>{\centering\arraybackslash}p{.035\columnwidth}}
\scalebox{.7}{
\begin{tabular}{>{\raggedright\arraybackslash}p{.2\columnwidth}SSSSSSSSSSSSSSSSSS}
\toprule
\textbf{Luogo}&\multicolumn{2}{c}{\textbf{2004}}&\multicolumn{2}{c}{\textbf{2005}}&\multicolumn{2}{c}{\textbf{2006}}&\multicolumn{2}{c}{\textbf{2007}}&\multicolumn{2}{c}{\textbf{2008}}&\multicolumn{2}{c}{\textbf{2009}}&\multicolumn{2}{c}{\textbf{2010}}&\multicolumn{2}{c}{\textbf{2011}}&\multicolumn{2}{c}{\textbf{2012}} \\
\toprule
Valli di Porto Viro&6&8&1&1& & & & & 1&3&10&13&3&7&1&1&0&1 \\
Valli di Porto Tolle&7&8&7&15&4&4&15&15&0&1&6&9&11&14&2&5&4&5 \\
\midrule
\textbf{Totale}&13&16&8&16&4&4&15&15&1&4&16&22&14&21&3&6&4&6 \\
\bottomrule
\end{tabular}
}
\caption{Numero di coppie nidificanti di pernice di mare nel delta del Po veneto}
\label{Verza_tab_1}
\end{table}

\section*{Bibliografia}
\begin{itemize}\itemsep0pt
 \item Associazione Faunisti Veneti, 2003. (Redattori: Bon M., Sighele M.,
Verza E.). Rapporto ornitologico per la Regione Veneto. Anno 2002.
\textit{Boll. Mus. civ. St. nat. Venezia}., 54 (2003): 123-160.

 \item Brichetti P., Fracasso G., 2004 - \textit{Ornitologia Italiana}. Vol. 2
Tetraonidae - Scolopacidae. Alberto Perdisa Editore, Bologna.

 \item Fracasso G., Verza E., Boschetti E. (a cura di), 2003.  \textit{Atlante
degli uccelli nidificanti in provincia di Rovigo}. Provincia di Rovigo
- Associazione Faunisti Veneti -- Gruppo di Studi Naturalistici
{\textquotedblleft}Nisoria{\textquotedblright}.

 \item Verza E., Trombin D. (a cura di), 2012. \textit{Le valli del Delta del
Po}. Ente Parco Regionale Veneto del Delta del Po. Apogeo Editore.
\end{itemize}


\addtocontents{toc}{\protect\newpage}
\part{Note brevi}
{\hspace*{\fill} \emph{In ordine alfabetico secondo il primo Autore}}\\
\begin{center}
\vspace*{\fill}
\includegraphics[width=.95\columnwidth]{osv_1.png}
\vspace*{\fill}
\end{center}
\newpagecolor{white}\afterpage{\restorepagecolor}


\setcounter{figure}{0}
\setcounter{table}{0}

\begin{adjustwidth}{-3.5cm}{0cm}
\pagestyle{CIOpage}
\authortoc{\textsc{Aluigi A.}, \textsc{Fasano G. S.}}
\chapter*[Valore ornitologico del Parco del Beigua e della ZPS Beigua-Turchino]{Valore ornitologico delle principali tipologie ambientali nel Parco del Beigua e nella ZPS Beigua-Turchino (GE-SV)}
\addcontentsline{toc}{chapter}{Valore ornitologico del Parco del Beigua e della ZPS Beigua-Turchino}

\textsc{Antonio Aluigi}$^{1}$, \textsc{Sergio G. Fasano}$^{1*}$ \\

\index{Aluigi Antonio} \index{Fasano G. Sergio}
\noindent\color{MUSEBLUE}\rule{27cm}{2pt}
\vspace{1cm}
\end{adjustwidth}


\marginnote{\raggedright $^1$Ente Parco del Beigua {}- Via Marconi 165, 16011
Arenzano GE - E-mail: biodiv@parcobeigua.it \\
\vspace{.5cm}
{\emph{\small $^*$Autore per la corrispondenza: \href{mailto:fasanosg@gmail.com}{fasanosg@gmail.com}}} \\
\keywords
{Parco del Beigua, diversit\`a ornitica, conservazione}
{Beigua Natural Park, birds diversity, conservation}
%\index{keywords}{Parco del Beigua} \index{keywords}{Diversit\`a ornitica} \index{keywords}{Conservazione}
}
{\small
\noindent \textsc{\color{MUSEBLUE} Summary } / A bird monitoring project~has been carried out~in the Beigua Natural 
Park between 2006-2012 as part of an integrated system of surveys. We
compared breeding bird communities detected in 1170 point counts~~to
identify the most important habitats evaluating specie richness, specie
diversity, ornithological value (Brichetti \& Gariboldi 1992), number
of Annex 1 species of  {\textquotedblleft}Birds
Directive{\textquotedblright}, and number of SPEC species. Our results
allowed us to identify pastures, grasslands, mediterranean bush, bare
rock and heterogeneous agricultural areas as the most important
habitats from a conservational perspective.
}
\vspace{1cm}

A partire dal 2006 l{\textquoteright}Ente Parco del Beigua ha attuato,
nell{\textquoteright}area protetta e nella connessa ZPS IT1331578
Beigua -- Turchino (che complessivamente occupano una superficie di
circa 145 chilometri quadrati), un dettagliato piano di monitoraggio
dell{\textquoteright}avifauna (Fasano \& Aluigi 2007, 2011; Fasano
\textit{et al. }2009, 2013), condotto applicando la metodica dei punti
d{\textquoteright}ascolto (Blondel \textit{et al}. 1981). 

Durante le attivit\`a di campo, al fine di poter mettere in relazione la
presenza e l{\textquoteright}abbondanza delle specie con le
caratteristiche ambientali, e quindi definire settori prioritari dal
punto di vista conservazionistico e gestionale, \`e stata stimata nel
raggio di 100 metri dal punto di rilevamento la copertura percentuale
delle categorie CORINE Land Cover di terzo livello. Nel corso dei
rilevamenti,  in 1170 punti d{\textquoteright}ascolto effettuati negli
anni 2006-2012, sono state rilevate 26 variabili ambientali, che
successive analisi di agglomerazione (matrice di somiglianza ricavata
mediante il calcolo della distanza euclidea; fusione delle entit\`a
secondo il metodo del legame medio tra gruppi) hanno associato in otto
cluster distinti, riconducibili ad ambienti dominati da: mosaici
agrari, boschi di latifoglie, boschi di conifere, boschi misti,
praterie, brughiere e cespuglieti, macchia mediterranea ed aree
rocciose. 

Tra queste otto tipologie ambientali sono state rilevate differenze
significative nel numero medio di specie per punto
d{\textquoteright}ascolto (F\textsubscript{7,1145} = 21,047, P
{\textless} 0,001) e nell{\textquoteright}indice di diversit\`a di
Shannon (F\textsubscript{7,1145} = 20,782, P {\textless} 0,001;
MacArthur 1965), i cui valori massimi sono stati osservati nei mosaici
agrari. In tale classe si riscontra anche il massimo valore
dell{\textquoteright}indice di equiripartizione. Prendendo in
considerazione il valore ornitologico nazionale delle specie
nidificanti calcolato da Brichetti \& Gariboldi (1992), e calcolando il
valore medio per punto d{\textquoteright}ascolto, abbiamo riscontrato,
tra le variabili ambientali esaminate, differenze statisticamente
significative per quanto riguarda il valore nazionale complessivo
(F\textsubscript{7,1145} = 16,211, P {\textless} 0,001), il valore
nazionale medio (F\textsubscript{7,1145} = 55,593, P {\textless} 0,001)
e il valore nazionale corretto dall{\textquoteright}abbondanza
specifica (F\textsubscript{7,1145} = 58,540, P {\textless} 0,001). I
valori nazionali medi e corretti dall{\textquoteright}abbondanza
specifica sono risultati nettamente pi\`u elevati per la macchia
mediterranea e gli ambienti rocciosi. Considerando infine il numero di
specie incluse nell{\textquoteright}All. 1 della Direttiva
{\textquotedblleft}Uccelli{\textquotedblright} (2009/147/CE) o la cui
conservazione risulti di particolare importanza per
l{\textquoteright}Europa (SPEC 2 e 3 secondo BirdLife International,
2004), si \`e evidenziato come la maggior rilevanza sia da attribuire
alle praterie e, a seguire, a brughiere e cespuglieti e boschi misti.
Tra i diversi ambienti \`e stata rilevata una differenza significativa
della presenza media per punto di ascolto sia di specie di All. 1 della
Direttiva {\textquotedblleft}Uccelli{\textquotedblright}
(F\textsubscript{7,1145} = 34,942, P {\textless} 0,001), sia di specie
di importanza europea (SPEC 2: F\textsubscript{7,1145} = 26,233, P
{\textless} 0,001; SPEC 3: F\textsubscript{7,1145} = 70,565, P
{\textless} 0,001). In particolare nella macchia mediterranea e nelle
praterie la presenza e l{\textquoteright}abbondanza di specie di All. 1
\`e risultata significativamente maggiore (test di Tukey, P {\textless}
0,05).

\section*{Bibliografia}
\begin{itemize}\itemsep0pt
	\item BirdLife International, 2004 - Birds in Europe: population
estimates, trends and conservation status. \textit{Cambridge UK}:
BirdLife International. BirdLife Conservation Series No. 12. 

	\item Blondel J., Ferry C. \& Frochot B., 1981 - Point Counts with Unlimited
distance. In: Estimating Numbers of terrestrial birds. \textit{Studies
in Avian Ecologies}, 6: 414-420. 

	\item Brichetti P. \& Gariboldi A., 1992 - Un
{\guillemotleft}valore{\guillemotright} per le specie ornitiche
nidificanti in Italia. \textit{Riv. ital. Orn.}, 62:73-87. 

	\item Fasano S. \& Aluigi A., 2007 - Dati preliminari sulla densit\`a
riproduttiva di Calandro \textit{Anthus campestris }e Magnanina comune
\textit{Sylvia undata} nel Parco del Beigua e nella ZPS
{\textquotedblleft}Beigua-Turchino{\textquotedblright} (GE-SV).
Abstract del XIV Convegno Italiano di Ornitologia. Trieste 26-30
settembre 2007: 47. 

	\item Fasano S.G. \& Aluigi A., 2011 - Variazioni interannuali ed
interstagionali nella densit\`a della magnanina comune \textit{Sylvia
undata} nel Parco del Beigua e nella ZPS Beigua-Turchino
(GE-SV) - Abstract del XVI Convegno Italiano di Ornitologia. Cervia -- Milano Marittima (Ravenna) 21-25 settembre 2011: 81-82. 

	\item Fasano S., Baghino L. \& Aluigi A., 2009 - La
{\textquotedblleft}Canellona{\textquotedblright}: un \textit{hot-spot}
per l{\textquoteright}Averla piccola. (SIC IT1331402). Atti del XV
Convegno Italiano di Ornitologia. Parco Nazionale del Circeo, Sabaudia
(Latina) 14-18 ottobre 2009. \textit{Alula} XVI (1-2): 544-546.

	\item Fasano S.G., Cottalasso R., Campora M., Baghino L., Toffoli R. \& Aluigi
A. (a cura di), 2013 - \textit{Ambienti e Specie del Parco del Beigua e
dei Siti della Rete Natura 2000 funzionalmente connessi.} Ente Parco
del Beigua, 100 pp. 

	\item MacArthur R.H., 1965 - Patterns of species diversity. \textit{Biol. Rev.} 40:510-533. 
\end{itemize}

\setcounter{figure}{0}
\setcounter{table}{0}

\begin{adjustwidth}{-3.5cm}{0cm}
\pagestyle{CIOpage}
\authortoc{\textsc{Angelini J.}, \textsc{Scotti M.}}
\chapter*[L{\textquoteright}alimentazione del lanario. Parco
Regionale Gola Rossa e Frasassi]{L{\textquoteright}alimentazione del lanario
\textbf{\textit{Falco biarmicus feldeggii}}\textbf{ nel Parco
Regionale Gola della Rossa e di Frasassi (AN) Italia centrale}}
\addcontentsline{toc}{chapter}{L{\textquoteright}alimentazione del lanario - Parco Regionale Gola Rossa e Frasassi}

\textsc{Jacopo Angelini}$^{1*}$, \textsc{Massimiliano Scotti}$^{2**}$\\

\index{Angelini Jacopo} \index{Scotti Massimiliano}
\noindent\color{MUSEBLUE}\rule{27cm}{2pt}
\vspace{1cm}
\end{adjustwidth}



\marginnote{\raggedright $^1$C.T.S.Parco Regionale Gola della Rossa e di Frasassi
Via Marcellini 60041 Serra San Quirico (AN) \\
$^2$Parco Regionale Gola della Rossa e di Frasassi Via
Marcellini 60041 Serra San Quirico (AN) \\
\vspace{.5cm}
{\emph{\small $^*$Autore per la corrispondenza: \href{mailto:jaco.angelini@gmail.com}{jaco.angelini@gmail.com}}} \\
{\emph{\small $^**$Autore per la corrispondenza: \href{mailto:massimiliano.scotti@parcogolarossa.it}{mas\allowbreak si\allowbreak mi\allowbreak lia\allowbreak no.\allowbreak scot\allowbreak ti@\allowbreak par\allowbreak co\allowbreak go\allowbreak la\allowbreak ros\allowbreak sa.\allowbreak it}}} \\
\keywords{\textit{Falco biarmicus feldeggii}, Italia centrale (Marche), dieta}
{\textit{Falco biarmicus feldeggii}, Central
Italy (Marche), diet}
%\index{keywords}{\textit{Falco biarmicus feldeggii}} \index{keywords}{Italia centrale (Marche)} \index{keywords}{Dieta}
}
{\small

\noindent \textsc{\color{MUSEBLUE} Summary} / In the Regional Park Frasassi Rossa{\textquoteright}s Gorges (Central
Italy), we studied the diet of the lanner \textit{Falco biarmicus
feldeggii} in 4 nesting sites from 2008 to 2012. We combined 3
different methods of data collection. a) direct observations, b)
sampling collection at plucking places, c) pellet analysis. We
recognized 177 preys, belong to 23 \textit{taxa}. Birds represented
85\% of the prey number and 91\% of biomass, Mammals represented the
9,95\% of the prey number and 7,5\% of biomass. Starlings
\textit{Sturnus vulgaris} is the most preyed species: 26\% of the prey
number and 12, 9\% of biomass.
Among Mammals, the dormouse \textit{Glis glis} is the more frequent prey
(5\%).  Interestingly, we found also a bat \textit{Mynopterus
scheibersi}.
}

\vspace{1cm}
Il Parco Regionale Gola della Rossa e di Frasassi e le zone limitrofe
ospitano 4 coppie nidificanti di lanario \textit{Falco biarmicus
feldeggii}, corrispondenti a ben il 40\% della popolazione conosciuta
della specie nella Regione Marche (Angelini 2007). Questa area protetta
della regione Marche ha una superficie di circa 10.000 ettari ed \`e
costituita da imponenti gole calcaree, da praterie secondarie e da
ambienti agricoli a mosaico con estese foreste di roverella
\textit{Quercus pubescens} e di carpino nero \textit{Ostrya
carpinifolia,} nel versante collinare, ed estesi boschi di faggio
\textit{Fagus sylvatica} alle quote maggiori. Lo studio, relativo alla
dieta del lanario \textit{Falco biarmicus feldeggii,} \`e stato
effettuato durante la stagione riproduttiva dal 2008 al 2012 nei 4 siti
di nidificazione occupati dalla specie. Sono stati utilizzati
contemporaneamente tre metodi diversi di raccolta dati: 1) osservazioni
dirette (n=116); 2) posatoio di spiumata (n=32); raccolta borre (n=29).
Sono state identificate complessivamente 177 prede appartenenti a 23
\textit{taxa} diversi. Gli uccelli hanno costituito
l{\textquotesingle}85\% delle prede, i mammiferi il 10\% e i rettili
5\% (Tab. \ref{Angelini_tab_1}). La specie maggiormente predata \`e risultata essere lo
storno \textit{Sturnus vulgaris }(47 casi, 26\% del totale), seguito
dalla taccola \textit{Corvus monedula} (13\%), dalla gazza \textit{Pica
pica} (10\%), dal merlo \textit{Turdus merula }(7\%), dalla ghiandaia
\textit{Garrulus glandarius }(6\%), dal piccione domestico
\textit{Columba livia var. domestica}(6\%), dal colombaccio \textit{Columbus
palumbus} (5\%). Tra i Mammiferi sono stati predati 16 individui (10\%
del totale), di cui il ghiro \textit{Glis glis} rappresenta il 3\% del
totale. Molto interessante il ritrovamento, in un posatoio di spiumata,
dei resti di \textit{Rhinolophus ferrumequinum}, mai segnalato in
precedenza come specie preda in Italia. Tra i Rettili il genere
\textit{Podarcis} rappresenta il 2\% del totale. In alcune borre sono
stati inoltre trovati resti di Insetti, probabilmente predati nelle
praterie del Parco. Interessante \`e anche la predazione tra i
Falconidi del gheppio \textit{Falco tinnunculus} e tra gli Accipitridi
dello Sparviere \textit{Accipiter nisus. }Come biomassa complessiva la
taccola \textit{Corvus monedula} rappresenta il 20\% del totale e gli
Uccelli complessivamente rappresentano il 91\% della biomassa totale in
accordo con quanto evidenziato per l{\textquoteright}Italia centrale da
altri autori (Morimando et al. 1997; De Sanctis et al. 2009).

Il Parco Regionale ha partecipato alla redazione del Piano di Azione
nazionale della specie, elaborato dall{\textquoteright}ISPRA (Andreotti
\& Leonardi 2007) e ha attuato diverse azioni di conservazione dirette
e indirette della specie, previste dal Piano stesso, come la
regolamentazione dell{\textquoteright}attivit\`a di arrampicata nelle
aree frequentate dalla specie con il divieto assoluto temporale durante
il periodo riproduttivo e anche con la creazione di aree di tutela
integrale tutto l{\textquoteright}anno, portando la specie ad
utilizzare nuovi siti nell{\textquoteright}area protetta. Inoltre sono
state messe in sicurezza oltre 60 km di linee elettriche grazie al
finanziamento del progetto LIFE Natura {\textquotedblleft}Save the
Flyers{\textquotedblright}, con il posizionamento di isolanti dei
conduttori sopra i pali a media tensione, vista la mortalit\`a diretta
della specie. Possibili fattori di minaccia per la specie possono
considerarsi il furto delle uova o dei piccoli al nido, visti i
sequestri effettuati dal Corpo Forestale dello Stato in diverse parti d{\textquoteright}Italia
di lanari catturati illegalmente in natura, le linee elettriche,
l{\textquoteright}attivit\`a di arrampicata vicina ai nidi della specie
e gli impianti eolici.



\rowcolors{2}{white!60!MUSEBLUE}{white}
\begin{table}[!h]
\centering
\footnotesize
\scalebox{.9}{
\begin{tabular}{>{\raggedright\arraybackslash}p{.4\columnwidth}d{6.3}d{6.3}}
\hiderowcolors
\toprule
\textbf{Specie} & \mc{\textbf{\% sulle prede totali}} & \mc{\textbf{\raggedright \% biomassa}} \\
\toprule
\multicolumn{3}{c}{\textbf{\textit{AVES}}} \\
%\showrowcolors
\textit{Sturnus vulgaris} & 26 & 12,9 \\
\textit{Turdus merula} & 7 & 3 \\
\textit{Monticola solitarius} & 2,3 & 1,1 \\
\textit{Phoenicurus ochruros} & 1,25 & 0,4 \\
\textit{Corvus corone} & 3 & 7,4 \\
\textit{Garrulus glandarius} & 6 & 6,9 \\
\textit{Pica pica} & 10 & 12,5 \\
\textit{Corvus monedula} & 13 & 2 \\
\textit{Streptopelia turtur} & 1,25 & 1 \\
\textit{Columbus palumbus} & 5 & 11,5 \\
\textit{Columba livia var. domestica} & 6 & 12,5 \\
\textit{Falco tinnunculus} & 0,6 & 0,8 \\
\textit{Accipiter nisus} & 0,6 & 0,8 \\
\textit{Picus viridis} & 0,6 & 0,5 \\
\hiderowcolors
\textbf{Totale} & 85,4 & 91,3 \\
\toprule
\multicolumn{3}{c}{\textbf{\textit{MAMMALIA}}} \\
%\showrowcolors
\textit{Rinolophus ferrumequinum} & 0,6 & 0,1 \\
\textit{Glis glis} & 2,8 & 4 \\
\textit{Moscardinus avellanarius} & 1,25 & 0,2 \\
\textit{Apodemus sp.} & 2,3 & 0,4 \\ 
\textit{Rattus sp.} & 3 & 2,8 \\
\toprule
\textbf{Totale} & 9,95 & 7,5 \\
\hiderowcolors
\multicolumn{3}{c}{\textbf{\textit{REPTILIA}}} \\
\toprule
%\showrowcolors
\textit{Podarcis sicula} & 2,3 & 0,5 \\
\textit{Lacerta bilineata} & 1,25 & 0,4 \\
\textit{Anguis veronensis} & 0,6 & 0,2 \\
\textit{Chalcides chalcides} & 0,5 & 0,1 \\
\textbf{Totale} & 4,65 & 1,2 \\
\bottomrule
\end{tabular}
}
\caption{}
\label{Angelini_tab_1}
\end{table}

\section*{Ringraziamenti}

Si ringraziano per i preziosi consigli e la collaborazione; Alessandro
Andreotti, Marco Andreini, Bruno D{\textquoteright}Amicis, Giovanni
Leonardi, Mauro Magrini, Paolo Perna, Carlo Poiani, Stefano Sassaroli,
Simonetta Turbessi, Aurelio Vitali.

\section*{Bibliografia}
\begin{itemize}\itemsep0pt
	\item Andreotti A. \& Leonardi G. (a cura di), 2007 - Piano
d{\textquoteright}azione nazionale per il Lanario (\textit{Falco
biarmicus feldeggii}). \textit{Quad. Cons. Natura}, 24, Min. Ambiente
-- Ist. Naz. Fauna Selvatica.

	\item Angelini J., 2007 - Lanario \textit{Falco biarmicus} in Giacchini P. (a
cura di), 2007 \textit{Atlante degli Uccelli Nidificanti nella
Provincia di Ancona}. Provincia di Ancona, IX Settore Tutela
dell{\textquoteright}Ambiente -- Area Flora e Fauna: 98-99.

	\item De Sanctis A., Di Meo D., Pellegrini M. \& Sammarone L., 2009 - Breeding
Biology and Diet of the Lanner \textit{Falco biarmicus feldeggii} in
the Abruzzo Region, Central Appennines. \textit{Alula} XVI (1-2):
170-175. 

	\item Morimando F., Pezzo F. \& Draghi A., 1997 - Food habits of the Lanner
Falcon (\textit{Falco biarmicus} \textit{feldeggii}) in Central Italy.
\textit{Journal of Raptors Research} 31 (1): 40-43.
\end{itemize}

\setcounter{figure}{0}
\setcounter{table}{0}

\begin{adjustwidth}{-3.5cm}{0cm}
\pagestyle{CIOpage}
\authortoc{\textsc{Assandri G.}, \textsc{Marotto P.}}
\chapter*[Il gabbiano reale nordico e il gabbiano reale pontico in
Piemonte. Una revisione critica]{Il gabbiano reale nordico \textbf{\textit{Larus
argentatus}}\textbf{ (Pontoppidan, 1763) e il gabbiano reale pontico
}\textbf{\textit{Larus cachinnans}}\textbf{ (Pallas, 1811) in Piemonte:
una revisione critica}}

\addcontentsline{toc}{chapter}{Il gabbiano reale nordico e il gabbiano reale pontico in
Piemonte}
\textsc{Giacomo Assandri}$^{1,2*}$, \textsc{Paolo Marotto}$^{1,2}$\\

\index{Assandri Giacomo} \index{Marotto Paolo}
\noindent\color{MUSEBLUE}\rule{27cm}{2pt}
\vspace{1cm}
\end{adjustwidth}



\marginnote{\raggedright $^1$Gruppo Piemontese Studi Ornitologici
{\textquotedblleft}F.A. Bonelli{\textquotedblright} ONLUS, Museo di
Storia Naturale, via San Francesco di Sales 188 - 10022 Carmagnola (TO) \\
$^2$Torino Birdwatching - Associazione EBN Italia, Via
Peyron 10 - 10143 Torino \\
\vspace{.5cm}
{\emph{\small $^*$Autore per la corrispondenza: \href{mailto:giacomo.assandri@gmail.com}{gia\allowbreak co\allowbreak mo.\allowbreak as\allowbreak san\allowbreak dri@\allowbreak g\allowbreak ma\allowbreak il.\allowbreak co\allowbreak m}}} \\
\keywords{\textit{Larus argentatus}, \textit{Larus cachinnans}, Piemonte}
{\textit{Larus argentatus}, \textit{Larus cachinnans}, Pedmont}
%\index{keywords}{\textit{Larus argentatus}} \index{keywords}{\textit{Larus cachinnans}} \index{keywords}{Piemonte}
} 
{\small
\noindent \textsc{\color{MUSEBLUE} Summary} / We present a revision of the status of Caspian and herring gulls
\textit{Larus cachinnans}\textit{ and Larus argentatus} in Piedmont (NW
Italy). Both species must be still considered vagrant in this region on
the basis of confirmed records.
}


\section*{Introduzione}

Sino al 2008, il gabbiano reale nordico era considerato, in Piemonte,
migratore irregolare e svernante occasionale sulla base di tre dati
storici certi e alcuni recenti in gran parte privi di documentazione.
Il gabbiano reale pontico era invece considerato accidentale sulla base
di quattro osservazioni adeguatamente documentate (Pavia \& Boano
2009). Come nel resto d{\textquoteright}Italia (Brichetti \& Fracasso
2006), anche in questa regione si sta recentemente assistendo a un
sensibile aumento delle segnalazioni di entrambe le specie; tuttavia
l{\textquoteright}identificazione di questi Laridi non \`e mai agevole
(Gibbins \textit{et al.} 2010) e, al fine di meglio definirne lo status
(secondo criteri AERC TAC 2003), si \`e resa necessaria la presente
revisione critica di tutti i dati disponibili. 

\section*{Metodi}

Sono state raccolte e analizzate tutte le segnalazioni note al 2012 e
quelle documentate fotograficamente sono state sottoposte alla
validazione di due esperti della COI, che hanno permesso di
classificare le segnalazioni in: confermate, non confermate, non
confermabili sulla base della documentazione disponibile o per
disaccordo dei validatori. Per entrambe le specie, basandosi sui
risultati della validazione, \`e stato inoltre calcolato un tasso di
errore sull{\textquoteright}identificazione:

\begin{center}
\textit{
(n{\textdegree} di foto erroneamente
identificate/n{\textdegree} di foto sottoposte a validazione)*100}
\end{center}

\section*{Risultati e discussione}

Per il gabbiano reale nordico sono stati analizzati 43 dati di presenza relativi
a 64 individui, successivi ai tre storici e riferiti al periodo
1987-2012. Di questi solo 5 sono corredati da foto (6 ind.), che hanno
permesso di confermare l{\textquoteright}identificazione
dell{\textquoteright}osservatore in 3 casi (3 ind.), mentre in 1 caso
(2 ind.) l{\textquoteright}identificazione \`e risultata errata (tasso
d{\textquoteright}errore=20\%) e in 1 caso (1 ind.) i validatori non
erano concordi. Il gabbiano reale pontico \`e stato segnalato per la prima volta
in Piemonte nel 2002 e al 2012 esistono 53 dati di presenza riferiti a
78 individui. Di questi 17 (20 ind.) sono corredati da foto, che hanno
permesso di confermare l{\textquoteright}identificazione
dell{\textquoteright}osservatore in 8 casi (9 ind.), mentre in 2 casi
(2 ind.) l{\textquoteright}identificazione \`e risultata errata (tasso
d{\textquoteright}errore=12\%), in 6 (7 ind.) non confermabile a causa
della scarsa qualit\`a della documentazione e in 1 caso (2 ind.) i
validatori non erano concordi. Il gabbiano reale nordico \`e stato contattato in
tutti i mesi compresi tra settembre e aprile, mentre il gabbiano reale pontico
tra novembre e aprile.

La presente revisione ha permesso di delineare il seguente quadro: I) Il
gabbiano reale nordico appare, diversamente da quanto noto finora, meno
frequente del gabbiano reale pontico in Piemonte, sebbene
quest{\textquoteright}ultimo sia stato contattato regolarmente solo a
partire 2006. II) Sulla base delle osservazioni documentate e
confermate, e quindi utilizzando un metodo conservativo di attribuzione
dello status, il gabbiano reale nordico \`e da considerarsi accidentale in
Piemonte con 6 segnalazioni, mentre il gabbiano reale pontico accidentale con 8.
Se la tendenza all{\textquoteright}incremento delle osservazioni si
manterr\`a tale, \`e prevedibile una rapida evoluzione di questo
status, che potrebbe coincidere con quello attuale, attribuito sulla
base di tutti i dati disponibili, anche non documentati (esclusi quelli
non validati): gabbiano reale nordico migratore regolare, svernante irregolare;
gabbiano reale pontico migratore irregolare, svernante irregolare. 

Sull{\textquoteright}origine dei soggetti piemontesi si \`e a conoscenza
di un gabbiano reale nordico ripreso nel marzo del 1958 sul Toce a Domodossola
(VB), inanellato nel luglio 1955 a Mellum, Germania (Moltoni 1973) e di
un gabbiano reale pontico inanellato da pullus a Kreminciuk (Ucraina) nel giugno
2010 e riosservato nel febbraio 2011 e 2012 presso
l{\textquoteright}Invaso del Meisino (TO) e pochi giorni dopo (nel
2012) anche sul lago di Varese. Tenendo conto delle difficolt\`a
oggettive di identificazione, confermate dai tassi di errore calcolati
nel presente contributo, \`e comunque ancora auspicabile da parte degli
osservatori un attivo sforzo di documentazione di queste due specie in
Piemonte.

\section*{Ringraziamenti}

Desideriamo ringraziare tutti gli osservatori che hanno reso disponibili
le loro segnalazioni e fotografie; Ottavio Janni e Michele Vigan\`o
(COI) per la validazione delle fotografie, Adriano Talamelli per la
comunicazione delle riletture.

\section*{Bibliografia}
\begin{itemize}\itemsep0pt
	\item AERC TAC, 2003 - AERC TAC{\textquoteright}s Taxonomic Recommendations. \\
	\url{http://www.aerc.eu/DOCS/AERCTAC.pdf}. 

	\item Brichetti P. \& Fracasso G., 2006 - \textit{Ornitologia Italiana} Vol. 3
-- \textit{Stercorariidae} - \textit{Caprimulgidae}. Alberto Perdisa Editore.

	\item Gibbins C., Small B. J. \& Sweeney J., 2010 - Identification of Caspian
Gull. \textit{British Birds}, 103: 142-183. 

	\item Moltoni E., 1973 - Elenco di parecchie centinaia di uccelli inanellati
all{\textquotesingle}estero e ripresi in Italia ed in Libia.
\textit{Rivista Italiana di Ornitologia, }43 (Suppl.): 1-182. 

	\item Pavia M. \& Boano G., 2009 - Check-list degli uccelli del Piemonte e
della Valle d{\textquoteright}Aosta aggiornata al dicembre 2008.
\textit{Rivista Italiana di Ornitologia}, 79: 23-47. 
\end{itemize}

\setcounter{figure}{0}
\setcounter{table}{0}

\begin{adjustwidth}{-3.5cm}{0cm}
\pagestyle{CIOpage}
\authortoc{\textsc{Baroni D.}, \textsc{Rapetti C.}}
\chapter*[L{\textquoteright}avifauna delle acque costiere]{L{\textquoteright}avifauna delle acque costiere del Mar Ligure centro-occidentale}
\addcontentsline{toc}{chapter}{L{\textquoteright}avifauna delle acque costiere}

\textsc{Daniele Baroni}$^{1*}$, \textsc{Carla Rapetti}$^{2}$\\

\index{Baroni Daniele} \index{Rapetti Carla}
\noindent\color{MUSEBLUE}\rule{27cm}{2pt}
\vspace{1cm}
\end{adjustwidth}



\marginnote{\raggedright $^1$Via Gaspare Buffa 4, 16158 -- Genova (GE) \\
$^2$Via Trento 14, 16011 -- Arenzano (GE) \\
\vspace{.5cm}
{\emph{\small $^*$Autore per la corrispondenza: \href{mailto:dbaroni12@gmail.com}{dbaroni12@gmail.com}}} \\
\keywords{Mar Ligure, uccelli acquatici, selezione
dell{\textquoteright}habitat}
{Ligurian Sea, diving birds, habitat selection}
%\index{keywords}{Mar Ligure} \index{keywords}{Uccelli acquatici} \index{keywords}{Selezione dell{\textquoteright}habitat}
}
{\small
\noindent \textsc{\color{MUSEBLUE} Summary} / Phenology and habitat selection of water birds in marine waters has been
surveyed in winter along a coastal transect of 62 km. During foraging
activity, harbors were preferred mostly by great crested grebe;
red-breasted merganser consistently selected reefs; arctic loon
selected waters along the shores.
}
\vspace{1cm}
\section*{Introduzione}
La selezione dell{\textquoteright}habitat e la fenologia di alcune
specie di uccelli acquatici sono state indagate in un settore del Mar
Ligure centro-occidentale; a livello regionale questi aspetti non sono
noti nel dettaglio e informazioni pregresse si trovano ad esempio in
Gorlier (1975); Andreotti \textit{et al.} (1991);  Borgo \textit{et
al.} (1991); Span\`o \textit{et al.} (1998); Ballardini \textit{et al.}
(2005). I dati sono stati raccolti percorrendo transetti costieri: tra
Genova Pra{\textquoteright} (GE) e Finale Ligure (SV), per un totale di
62 km, con frequenza mensile e tra Genova Pra{\textquoteright} (GE) e
Vado Ligure (SV), per un totale di 42 km, ogni decade.
L{\textquoteright}area di studio \`e caratterizzata da una forte
componente antropica e include due ampie aree portuali (Genova Voltri e
Savona), tratti di costa rocciosa e litorali ciottolosi o ghiaiosi. Il
periodo d{\textquoteright}indagine comprende il semestre ottobre-marzo,
dal 2008 al 2011 e la durata di ogni uscita ha interessato quasi tutte
le ore di luce, al fine di consentire il maggior rilevamento possibile
delle presenze. I transetti sono stati percorsi in macchina,
effettuando soste distribuite lungo il percorso in modo da monitorare
l{\textquoteright}intera area di studio mediante
l{\textquoteright}utilizzo di strumenti ottici (cannocchiale e
binocolo). L{\textquoteright}ordine progressivo con il quale sono stati
censiti i diversi settori dell{\textquoteright}area \`e stato
diversificato nell{\textquoteright}orario delle osservazioni,  al fine
di compensare l{\textquoteright}effetto delle variazioni di attivit\`a
degli uccelli durante il giorno. La selezione
dell{\textquoteright}habitat \`e stata quindi indagata mediante test
del $\tilde\chi^2$ al fine di verificare se le frequenze osservate e
attese in ogni habitat differissero significativamente. In caso
affermativo \`e stata utilizzata una statistica \textit{z} di
Bonferroni al fine di individuare quali tipologie fossero selezionate
positivamente o negativamente (Neu \textit{et al.} 1974).

\section*{Risultati e discussione}

Su 1826 osservazioni relative a 58 specie di uccelli acquatici, sono
state analizzate 4 specie meglio rappresentate e pi\`u significative a
livello di distribuzione, appartenenti alla \textit{guild} trofica dei
{\textquotedblleft}tuffatori{\textquotedblright}.
L{\textquoteright}elevata diversit\`a di habitat marini cartografati
nell{\textquoteright}area (Diviacco \& Coppo 2006) \`e stata ricondotta
a tre sole tipologie (Tab. \ref{Baroni_tab_1}). In tutti i casi le osservazioni hanno
evidenziato un utilizzo dell{\textquoteright}habitat che si discosta
dall{\textquoteright}atteso calcolato sulla disponibilit\`a ambientale
in termini di estensione (P{\textless}0.0001; Tab. \ref{Baroni_tab_1}). Lo smergo minore
\textit{Mergus serrator} evidenzia un massimo del numero di individui
nei mesi di gennaio e di dicembre. Analizzando le sole localizzazioni
relative a soggetti in attivit\`a trofica (N = 25) la costa rocciosa a
ridosso degli scogli \`e risultata essere la tipologia ambientale
fortemente selezionata, mentre l{\textquoteright}interno di strutture
portuali, lungo le banchine, \`e stato frequentato in minor misura e
non risulta selezionato attivamente. La maggior parte delle
segnalazioni (43\%) di strolaga mezzana \textit{Gavia arctica} \`e
relativa al mese di gennaio e le acque antistanti tratti di spiaggia
costituiscono l{\textquoteright}unica tipologia ambientale selezionata
dagli individui in alimentazione (N = 61). Lo svasso maggiore
\textit{Podiceps cristatus} evidenzia un picco delle segnalazioni in
febbraio (40\%), mentre per ci\`o che concerne gli individui in
attivit\`a trofica (N = 91) si evidenzia una preferenza per porti e
porticcioli. Infine, l{\textquoteright}analisi quadriennale dei dati
raccolti mensilmente sul cormorano \textit{Phalacrocorax carbo} sui 62
km di costa permette un{\textquoteright}analisi fenologica che
evidenzia maggiore variabilit\`a nei quantitativi osservati in
corrispondenza dei periodi migratori (ottobre e marzo) e un massimo in
gennaio ( ${x}$ = 46.7{\textpm}15.6).

La presenza massima di acquatici
{\textquotedblleft}tuffatori{\textquotedblright} nei mesi di gennaio e
febbraio \`e in linea con la fenologia nazionale delle specie
(Brichetti \& Fracasso 2003). Per ci\`o che concerne la selezione
dell{\textquoteright}habitat a fini trofici si evidenzia come la
predilezione di alcune specie per aree antropizzate (es. porticcioli)
sia probabilmente in funzione dell{\textquoteright}assenza, nel
contesto ligure, degli ambienti naturali d{\textquoteright}elezione per
le specie considerate, quali ad esempio le lagune costiere.



\newcolumntype{S}{>{\centering\arraybackslash}p{.1\columnwidth}}
\newcolumntype{s}{>{\centering\arraybackslash}p{.04\columnwidth}}
\newcolumntype{B}{>{\centering\arraybackslash}p{.12\columnwidth}}

\begin{adjustwidth}{1cm}{1cm}
\begin{table}[!h]
\centering
\footnotesize
\scalebox{.8}{
\begin{tabular}{>{\raggedright\arraybackslash}p{.15\columnwidth}ssBS|ssBS|ssBS}
\hiderowcolors
\toprule
& \multicolumn{4}{c}{\textbf{\textit{Podiceps cristatus}}} & \multicolumn{4}{c}{\textbf{\textit{Mergus serrator}}} & \multicolumn{4}{c}{\textbf{\textit{Gavia arctica}}} \\
\toprule
\textbf{Tipologie ambientali} & \textbf{pi att.} & \textbf{pi oss.} & \textbf{I.C. 95\% corr. Bonferroni}
& \textbf{Selezione} & \textbf{pi att.} & \textbf{pi oss.} & \textbf{I.C. 95\% corr. Bonferroni}
& \textbf{Selezione} & \textbf{pi att.} & \textbf{pi oss.} & \textbf{I.C. 95\% corr. Bonferroni}
& \textbf{Selezione} \\
\toprule
Ambiti portuali & 0,20 & 0,65 & 0,53-0,77 & selez. & 0,20 & 0,28 & 0,06-0,50 & non selez. & 0,20 & 0,05 & 0-0,12 & evitato \\
Acque antistanti spiagge & 0,47 & 0,25 & 0,14-0,36 & evitato & 0,47 & 0,00 & - & evitato & 0,47 & 0,84 & 0,73-0,95 & selez. \\
Falesie & 0,33 & 0,10 & 0,03-0,17 & evitato & 0,33 & 0,72 & 0,51-0,94 & selez. & 0,33 & 0,11 & 0,01-0,21 & evitato \\
\toprule
& \multicolumn{4}{c}{$\chi^{2}$ = 117,4; P < 0,0001} & \multicolumn{4}{c}{$\chi^{2}$ = 25,3; P < 0,0001} & \multicolumn{4}{c}{$\chi^{2}$ = 31,9; P < 0,0001} \\
\toprule
\end{tabular}
}
\caption{Frequenze attese e osservate di uso dell{\textquoteright}habitat e selezione delle tre tipologie}
\label{Baroni_tab_1}
\end{table}
\end{adjustwidth}

\section*{Bibliografia}
\begin{itemize}\itemsep0pt
	\item Andreotti A., Borgo E. \& Truffi G., 1991 -- Presenze di Strolaghe
(Gavia spp.) in Liguria. In: SROPU (red.). Atti V Conv. Ital. Orn.
\textit{Suppl. Ric. Biol. }\textit{Selvaggina}, 17: 449-451.

	\item Ballardini M., Calvini M., Nani B. \& Toffoli R., 2005 -- Osservazioni
su presenza e distribuzione di pulcinella di mare \textit{Fratercula
arctica} e gazza marina \textit{Alca torda} nel mar Ligure occidentale.
Atti XIII Conv. Ital. Orn. \textit{Avocetta}, 29: 167.

	\item Borgo E., Span\`o S. \& Truffi G., 1991 -- Eccezionale presenza di
edredoni in Liguria: dati quantitativi. In: Fasola M. (red.). atti II
Semin. Ital. Censim. Faunistici dei Vertebrati. \textit{Suppl. Ric.
Biol. Selvaggina}, 16: 297-300.

	\item Brichetti P. \& Fracasso G., 2003 -- \textit{Ornitologia italiana. Vol.
1 -- Gaviidae-Falconidae.} Alberto Perdisa Editore, Bologna, 463 pp.

	\item Diviacco G. \& Coppo S., 2006 -- \textit{Atlante degli habitat marini
della Liguria. Descrizione e cartografia delle praterie di }Posidonia
oceanica\textit{ e dei principali popolamenti marini costieri.} Regione
Liguria, Genova, 205 pp.

	\item Gorlier G., 1975 -- Osservazioni ornitologiche del litorale e della zona
di mare compresa tra Vado Ligure (SV) e Finale Ligure (SV).
\textit{Riv. Ital. Orn.,} 45: 61-67.

	\item Neu C.W., Byers C.R. \& Peek J.M., 1974 -- A technique for analysis of
utilization-availability data. \textit{J. Wildl. Manage.}, 38: 541-545.

	\item Span\`o S., Truffi G. \& Burlando B. (a cura di), 1998 --
\textit{Atlante degli Uccelli svernanti in Liguria.} Regione Liguria,
Genova, 253 pp.
\end{itemize}

\setcounter{figure}{0}
\setcounter{table}{0}

\begin{adjustwidth}{-3.5cm}{0cm}
\pagestyle{CIOpage}
\authortoc{\textsc{Basile M.}}
\chapter*[Dieta del gufo comune in habitat semi-urbano (Arzano,
Campania)]{Analisi della dieta del gufo comune \textbf{\textit{Asio
otus}}\textbf{ durante lo svernamento in habitat semi-urbano (Arzano,
Campania)}}
\addcontentsline{toc}{chapter}{Dieta del gufo comune in habitat semi-urbano (Arzano,
Campania)}


\textsc{Marco Basile}$^{1*}$ \\

\index{Basile Marco}
\noindent\color{MUSEBLUE}\rule{27cm}{2pt}
\vspace{1cm}
\end{adjustwidth}



\marginnote{\raggedright $^1$Associazione per la Ricerca, la Divulgazione e
l{\textquotesingle}Educazione Ambientale -- ARDEA \\
\vspace{.5cm}
{\emph{\small $^*$Autore per la corrispondenza: \href{mailto:marcob.nat@gmail.com}{marcob.nat@gmail.com}}} \\
\keywords{\textit{Asio otus}, Campania, dieta, borre,
habitat semi-urbano}
{\textit{Asio otus}, Campania, diet, pellets, suburban habitat}
%\index{keywords}{\textit{Asio otus}} \index{keywords}{Campania} \index{keywords}{Dieta} \index{keywords}{Borre}
%\index{keywords}{Habitat semi-urbano}
}
{\small
\noindent \textsc{\color{MUSEBLUE} Summary} / A long-eared owl \textit{Asio otus}\textbf{ }suburban roost was found in
2012 at Arzano (Campania, Southern Italy). The roost was occupied from
January till March. The analysis of pellets allowed to identify 6
different micromammals species plus indistinct bird remains.
Savi{\textquoteright}s pine vole is the most frequent prey (76\%).
Despite the low number of black rat and brown rat, amounting for 5\%
altogether, they form 23\% of the total biomass.
}
\section*{Introduzione}
Il gufo comune \textit{Asio otus} \`e uno strigiforme parzialmente
sedentario, nonch\'e migratore regolare e svernante regolare in Italia
(Galeotti 2003). Come tutti gli Strigiformi, questa specie produce
borre che si mantengono integre a lungo; ci\`o permette di effettuare
studi approfonditi sull{\textquoteright}alimentazione (Nappi 2011).
Indagini pregresse hanno messo in risalto la relazione tra le
popolazioni di arvicole e la composizione della dieta, evidenziando un
comportamento predatorio piuttosto specialista (Korpim\"aki 1992; Tome
2009). A differenza di quanto accade in Europa, tuttavia, in Italia il
gufo comune sembra essere meno specialista, avendo a disposizione uno
spettro di prede pi\`u ampio (Bertolino \textit{et al.} 2001). In
particolare, un comportamento alimentare generalista risulta pi\`u
evidente in habitat semiurbani, dove sono disponibili anche prede di
grosse dimensioni quali \textit{Rattus }sp. (Pirovano \textit{et al.}
2000).

In questo studio, vengono riportate informazioni sulla dieta
giornaliera, ottenute attraverso l{\textquoteright}analisi delle borre
raccolte presso un \textit{roost} sito ad Arzano (NA), a 79 m s.l.m. 

\section*{Metodi}

Il \textit{roost} era localizzato in un giardino privato, su un pino
marittimo \textit{Pinus pinaster}, in ambiente semi-urbano, con campi
coltivati o incolti e tessuto urbano. Le borre sono state raccolte ogni
giorno, conservate per almeno 24 h in un congelatore a -18
{\textdegree}C e dissezionate a secco. Per
l{\textquoteright}identificazione dei micro-Mammiferi si \`e fatto
riferimento a Nappi (2001), mentre gli Uccelli, come in altri studi,
sono stati considerati come un{\textquoteright}unica entit\`a (Galeotti
\& Canova 1994; Pirovano \textit{et al.} 2000). Per valutare se il
numero di specie rinvenute potesse essere considerato esaustivo dello
spettro alimentare \`e stato paragonato allo stimatore non parametrico
di ricchezza in specie Chao1-bc (Chao 2005). Questo stimatore stima la
ricchezza di specie teorica osservabile, calcolata dalle abbondanze di
ogni specie. Una sua comparazione con la ricchezza di specie reale
pu\`o dare un{\textquoteright}idea della rappresentativit\`a del
campione. La biomassa delle prede \`e stata calcolata utilizzando le
tabelle fornite da Lovari et al. (1976), ad eccezione che per le specie
appartenenti al genere \textit{Rattus}, per le quali sono state usate
le equazioni fornite da Di Palma e Massa (1981). Per gli Uccelli \`e
stata utilizzata, seguendo il medesimo autore,
un{\textquoteright}approssimazione di 10 g per individuo.

Il \textit{roost} \`e stato frequentato dal 6 gennaio
all{\textquoteright}8 marzo, per un totale di 53 giorni di presenza. Il
numero medio di individui \`e stato di tre (range 1-4). Sono state
raccolte 87 borre, che fanno riscontrare una produzione di 0.55 borre
al giorno per gufo. Sono state identificate 198 prede, tra cui \`e
stato possibile discernere 7 specie diverse di micro-Mammiferi, per un
valore totale di biomassa di 3788,7 g (Tab. \ref{Basile_tab_1}). 

\section*{Risultati e discussione}

La biodiversit\`a riscontrata \`e stata ritenuta esaustiva dello spettro
alimentare, come risultato dallo stimatore non parametrico Chao1 (mean
{\textpm} s.e. = 8.0 {\textpm} 0.5). Nel 57\% delle borre si sono
trovate una o due prede, mentre il pasto medio \`e risultato di 19-38
g. Il 76\% delle prede determinate appartiene alla specie
\textit{Microtus savii}, mentre solo il 5\% \`e stato attribuito a
\textit{Rattus rattus} o \textit{Rattus norvegicus}. Di contro, il 55\%
della biomassa \`e costituita da \textit{M. savii}, mentre il 23\% da
\textit{Rattus} sp. (pesi medi: \textit{R. rattus} = 58 g; \textit{R.
norvegicus} = 148 g). Gli uccelli sono un{\textquoteright}altra
importante fonte di alimentazione, costituendo il 9\% degli individui e
il 10\% della biomassa.

I ratti, nonostante costituiscano una fonte
d{\textquoteright}alimentazione potenzialmente non trascurabile,
sembrano essere predati in maniera occasionale, mentre
l{\textquoteright}arvicola di Savi risulta essere la preda principale.
Tali risultati appaiono in linea con quanto gi\`a noto per il nord
Italia (Pirovano \textit{et al.} 2000). Differentemente da altri studi,
lo spettro alimentare riscontrato risulta notevolmente ridotto
(Galeotti \& Canova 1994; Cecere \& Vicini 2000). Le cause di ci\`o
potrebbero essere ricercate in una scarsa qualit\`a
dell{\textquoteright}ambiente circostante, talvolta oggetto di
derattizzazioni. 

\begin{table}[!h]
\centering

\begin{tabular}{>{\raggedright\arraybackslash}p{.32\columnwidth}>{\raggedright\arraybackslash}p{.18\columnwidth}>{\centering\arraybackslash}p{.07\columnwidth}>{\centering\arraybackslash}p{.07\columnwidth}>{\centering\arraybackslash}p{.07\columnwidth}>{\centering\arraybackslash}p{.07\columnwidth}}
\toprule
\textbf{Specie} & \textbf{Nome comune} & \multicolumn{2}{c}{\textbf{Frequenza}} & \multicolumn{2}{c}{\textbf{Biomassa}} \\
& & \multicolumn{1}{c}{n} & \multicolumn{1}{c}{\%} & \multicolumn{1}{c}{g} & \multicolumn{1}{c}{\%} \\
\toprule
\multicolumn{4}{l}{\textit{Rodentia - Cricetidae}} \\
\toprule
%\showrowcolors
\textit{Microtus savii} & Arvicola di Savi & 151 & 76.26 & 2076.25 & 54.80 \\
\toprule
\hiderowcolors
\multicolumn{4}{l}{\textit{Rodentia - Muridae}} \\
\toprule
\textit{Apodemus sp.} & & 10 & 5.05 & 265 & 6.99 \\
\textit{Apodemus sylvaticus} & Topo selvatico & 4 & 2.02 & 106 & 2.80 \\
\textit{Mus domesticus} & Topo comune & 5 & 2.53 & 95 & 2.51 \\
\textit{Rattus rattus} & Ratto nero & 5 & 2.53 & 291.33 & 7.69 \\
\textit{Rattus norvegicus} & Ratto norvegese & 4 & 2.02 & 591.66 & 15.62 \\
\toprule
\hiderowcolors
\multicolumn{4}{l}{\textit{Soricomorpha - Soricidae}} \\
\toprule
%\showrowcolors
\textit{Crocidura suaveolens} & Crocidura minore & 1 & 0.51 & 3.5 & 0.09 \\
\toprule
\hiderowcolors
\textit{Aves - Passeriformes} & Uccelli & 18 & 9.09 & 360 & 9.50 \\
\toprule
\textbf{Totale} & & 198 & 100 & 3788.74 & 100 \\
\bottomrule
\end{tabular}
\caption{Specie rinvenute nelle borre con frequenza e biomassa assoluta e relative. La biomassa e la frequenza assolute sono espresse come numero di individui (n) e grammi (g). La biomassa e frequenza relative sono espresse in percentuali.}
\label{Basile_tab_1}
\end{table}

\section*{Bibliografia}
\begin{itemize}\itemsep0pt
	\item Read C.B. \& Vidakovic B. (eds), 2006 - \textit{Encyclopedia of
Statistical Sciences, 2nd Edition, Vol. 12}. Wiley, New York.

	\item Di Palma M.G. \& Massa B., 1981 - Contributo metodologico per lo studio
dell{\textquoteright}alimentazione dei rapaci. In: Farina A. (eds),
Atti I Convegno Italiano di Ornitologia Aulla (MS). \textit{Volume
monografico }(1982). 

	\item Galeotti P. \& Canova L., 1994 - Winter diet of Long-eared Owls
(\textit{Asio otus}) in the Po plain (northern Italy). \textit{J.
Raptor Research }28 (4): 265 - 268.

	\item Galeotti P., 2003 - Gufo comune. In Spagnesi M., Serra L. (eds).
\textit{Quad. Cons. Natura 16. }Min. Ambiente - Ist. Naz. Fauna
Selvatica.

	\item Korpim\"aki E., 1992 - Diet composition, prey choice, and breeding
success of Long-eared Owls: effects of multiannual fluctuations in food
abundance. \textit{Can. J. Zool}. 70: 2373 - 2381.

	\item Lovari S., Renzoni A. \& Fondi R., 1976 - The predatory habits of the
Barn Owl (\textit{Tyto alba }Scopoli) in relation to the vegetation
cover. \textit{Boll. Zool. }43: 173 - 191.

	\item Nappi A., 2001 - \textit{Micromammiferi d{\textquoteright}Italia.
}Edizioni Simone.

	\item Nappi A., 2011 - L{\textquoteright}analisi delle borre degli uccelli:
metodiche, applicazioni e informazioni. Un lavoro monografico.
\textit{Picus }37 (72): 106 - 120. 

	\item Pirovano A., Rubolini D., Brambilla S. \& Ferrari N., 2000 - Winter diet
of urban roosting Long-eared Owls \textit{Asio otus} in northern Italy:
the importance of the Brown Rat \textit{Rattus norvegicus}.
\textit{Bird Study }47: 242 - 244.

	\item Tome G., 2009 - Changes in the diet of Long-eared Owl \textit{Asio
otus}: seasonal patterns of dependence on vole abundance.
\textit{Ardeola }56 (1): 49 - 56. 
\end{itemize}


\setcounter{figure}{0}
\setcounter{table}{0}

\begin{adjustwidth}{-3.5cm}{0cm}
\pagestyle{CIOpage}
\authortoc{\textsc{Basile M.}, \textsc{Balestrieri R.},
\textsc{Buoninconti F.}, \textsc{Capobianco G.},
 \textsc{Altea T.}, \textsc{Matteucci G.},
\textsc{Posillico M.}}
\chapter*[Catturabilit\`a del rampichino comune]{
\textbf{Studio preliminare sulla catturabilit\`a del rampichino comune
}\textbf{\textit{Certhia brachydactyla}}}
\addcontentsline{toc}{chapter}{Catturabilit\`a del rampichino comune}


\textsc{Marco Basile}$^{1}$, \textsc{Rosario Balestrieri}$^{1*}$,
\textsc{Francesca Buoninconti}$^{1}$, \textsc{Giovanni Capobianco}$^{1}$,
 \textsc{Tiziana Altea}$^{2}$, \textsc{Giorgio Matteucci}$^{1}$,
\textsc{Mario Posillico}$^{1,2}$ \\

\index{Basile Marco} \index{Balestrieri Rosario} \index{Buoninconti Francesca} \index{Capobianco Giovanna} \index{Altea Tiziana} \index{Matteucci Giorgio} \index{Posillico Mario}
\noindent\color{MUSEBLUE}\rule{27cm}{2pt}
\vspace{1cm}
\end{adjustwidth}



\marginnote{\raggedright $^1$Istituto di Biologia Agroambientale e
Forestale del CNR Via Salaria km 29.3, 00015 Monterotondo Scalo RM \\
$^2$Corpo Forestale dello Stato, Ufficio
Territoriale Biodiversit\`a di Castel di Sangro, Via Sangro, 45-67031
Castel di Sangro (AQ) \\
{\textquotedblleft}L{\textquoteright}Assiolo{\textquotedblright}, via
Donizetti Loc. Ronchi, 54100 Marina di Massa (MS), Italy \\
\vspace{.5cm}
{\emph{\small $^*$Autore per la corrispondenza: \href{mailto:rosario.balestrieri@ibaf.cnr.it}{ro\allowbreak sa\allowbreak rio.\allowbreak ba\allowbreak les\allowbreak trie\allowbreak ri@\allowbreak i\allowbreak baf.\allowbreak cnr.\allowbreak it}}} \\
\keywords{Foresta demaniale regionale
{\textquotedblleft}Feudozzo{\textquotedblright}, \textit{Certhia brachydactyla}, catturabilit\`a}
{Feudozzo regional forest, Certhia brachydactyla, catchability}
%\index{keywords}{Foresta demaniale regionale {\textquotedblleft}Feudozzo{\textquotedblright}} \index{keywords}{\textit{Certhia brachydactyla}} \index{keywords}{Catturabilit\`a}
}
{\small
\noindent \textsc{\color{MUSEBLUE} Summary} / The short-toed treecreeper \textit{Certhia brachydactyla} is a bird
species strictly linked to old-growth forests. Nest and foraging sites,
indeed, are due to cavity, dead snag, ivy and other typical old-growth
forest characteristics. Thus, it could be an eligible species for
forest management research, as it is within the project LIFE ManFor
CBD. In order to do this kind of research in the near future, we need
to test the catchability of the short-toed treecreeper. This study was
carried out in the {\textquotedblleft}Feudozzo{\textquotedblright}
regional Forest, an old-growth forest of oak \textit{Quercus Cerris},
with relevant presence of beech \textit{Fagus sylvatica}. Catching
protocol consisted in the use of three 6-meters mist-nets, forming a
triangle around a tree, to which the base is located a playback-speaker
playing the treecreeper song, for a total of three triangles. In the
two-days session five individuals were captured plus one more
individual recaptured. Among these, four were born in the current
season. Capture frequency resulted in 0.4 individual/hour of capture.
}


\section*{Introduzione}
Il rampichino comune \textit{Certhia brachydactyla} \`e una specie
ornitica prettamente forestale, la cui presenza \`e strettamente
correlata a quella di alberi vetusti (Brichetti \& Fracasso 2011).
L{\textquoteright}area di nidificazione, in particolare, \`e legata a
quei caratteri tipici delle foreste miste, quali l{\textquoteright}alta
densit\`a di cavit\`a negli alberi, gli intricati fasci
d{\textquoteright}edera, nonch\'e (a una diversa scala spaziale) la
fitta copertura arborea (Cramp \& Perrins 1994). Inoltre il territorio
di nidificazione risulta particolarmente ristretto, con i casi estremi
registrati di: 0.4 ha in Germania orientale e 7.9 ha sulle Alpi
Marittime (Cramp \& Perrins 1994). Infine la specie \`e sedentaria e in
genere effettua limitati spostamenti, salvo il \textit{dispersal
}giovanile, comunque limitato (Cramp \& Perrins 1994). Date queste
peculiarit\`a, il rampichino comune \`e un buon soggetto di studio nel
contesto dell{\textquoteright}ecologia e gestione forestale. I
parametri demografici e la stima di popolazione potrebbero fornire
interessanti informazioni sugli effetti a breve e medio termine degli
interventi forestali. Nell{\textquoteright}ottica di effettuare studi
in tal senso, nell{\textquoteright}ambito del LIFE ManFor CBD, \`e
stato svolto nella Foresta demaniale regionale
{\textquotedblleft}Feudozzo{\textquotedblright} (Castel di Sangro, AQ)
uno studio preliminare sulla catturabilit\`a del rampichino comune, al
fine di valutare l{\textquoteright}efficacia delle trappole e lo sforzo
di campo.

\section*{Metodi}
La Foresta demaniale regionale
{\textquotedblleft}Feudozzo{\textquotedblright} \`e una foresta mista
di caducifoglie. La comunit\`a arborea \`e riconducibile alla cerreta
\textit{Quercus cerris}, con importanti presenze di faggio
\textit{Fagus sylvatica }nelle aree sommitali. Le trappole sono
costituite da tre \textit{mist-net }(maglia 16 mm, lunghezza 6 m,
altezza 2.4 m), montate a formare un triangolo intorno a un albero.
Alla base dell{\textquoteright}albero \`e stato posizionato un richiamo
elettroacustico per attrarre i rampichini aumentando le probabilit\`a
di cattura. Il richiamo elettroacustico proveniva da un lettore MP3
(Olympus DM-550), che emetteva, in modalit\`a ripetizione, un file
audio contenente il canto della specie. Le tre trappole sono state
posizionate lungo un transetto di circa 425 m (TR1 -- TR2 = 208 m; TR1
-- TR3 = 245 m; TR2 -- TR3 = 425 m), in maniera opportunistica, in
un{\textquoteright}area in cui era stata precedentemente rilevata la
specie. Le trappole sono state attive dal 9 al 10 agosto 2012, per un
totale di 15 ore. Le trappole sono state attive
dall{\textquoteright}alba alle 11, e dalle 16:30 al tramonto e
controllate ogni ora. Ogni rampichino catturato \`e stato marcato con
un anello ISPRA. Inoltre sono state prese le seguenti misure
biometriche: lunghezza della terza remigante primaria, corda massima
dell{\textquoteright}ala, lunghezza del becco, lunghezza del tarso,
peso; e le seguenti misure visive: et\`a, sesso (quando possibile),
accumulo di grasso, sviluppo dei muscoli pettorali. Tutte le misure
sono state prese, da un inanellatore abilitato, seguendo le direttive
ISPRA e con la strumentazione ufficiale. Lo sforzo di campo, invece,
\`e stato sostenuto da quattro persone, per un totale di 16:30 ore
(4.10 ore cadauno, compresa installazione e disinstallazione della
trappola). Le frequenze orarie di cattura sono state calcolate come
numero di catture/ore di funzionamento trappola, da cui poi \`e stata
ricavata la media aritmetica tra le trappole. La dimensione della
popolazione \`e stata stimata in maniera esplorativa secondo il modello
a variabilit\`a temporale Mt (EE), implementato dal software CARE-2
(Chao \& Yang 2003).

\section*{Risultati e discussione}
Sono stati marcati 5 individui, di cui uno ricatturato in una successiva
sessione, per un totale di 6 catture. La frequenza oraria media delle
catture \`e 0.4 catture/ora. La frequenza oraria media di cattura non
\`e risultata costante poich\'e in tutte le trappole i rampichini sono
stati catturati entro 2 ore dalla prima emissione del richiamo. Inoltre
l{\textquoteright}unica cattura pomeridiana \`e la ricattura di un
individuo inanellato 11 ore prima, nella stessa trappola. Infatti,
considerando le due trappole attive per due giorni (TR1-TR2), 4 catture
su 5 sono state effettuate il primo giorno. Dei 5 individui marcati, 4
sono risultati essere giovani nati nell{\textquoteright}anno in corso
(et\`a euring: 3), mentre 1 risulta essere adulto (et\`a euring: 4).
Con questi dati la popolazione \`e stata stimata in 9.8 individui
(IC\textsubscript{95\%} = 5 -- 15.6).~Tuttavia, trattandosi di uno
studio preliminare, volto a testare l{\textquoteright}efficacia del
metodo di cattura, l{\textquoteright}area effettivamente coperta dalle
trappole non \`e stimabile.

Nonostante il breve arco temporale indagato, sono stati ottenuti
risultati interessanti, innanzitutto l{\textquoteright}alta rispondenza
(risposta al richiamo e cattura) degli individui nelle prime ore del
mattino, dovuta alla spiccata territorialit\`a (Brichetti \& Fracasso
2011).  Sulla base delle esperienze effettuate, si suggerisce
l{\textquoteright}emissione in \textit{playback} di varie tipologie di
canto, riprodotte in maniera casuale per la realizzazione di ulteriori
sessioni di cattura sulla specie. Considerando, quindi, che
nell{\textquoteright}area indagata sono state ottenute 5 catture in tre
trappole diverse nelle prime 5 ore di attivit\`a, si pu\`o affermare
che, aumentando lo sforzo di campo, si potrebbe catturare buona parte
della popolazione nidificante entro una determinata area boscata.

\section*{Bibliografia}
\begin{itemize}\itemsep0pt
\item {
Brichetti P. \& Fracasso G., 2011- \textit{Ornitologia Italiana vol. 7 -
Paridae-Corvidae}. Oasi Alberto Perdisa Editore, Bologna, 493 pp.}

	\item {
Cramp S. \& Perrins C.M., 1994 - \textit{The Birds of the Western
Palearctic} Volume VIII. Oxford University Press, Oxford, New York, 899
pp. }

	\item {
Chao A. \& Yang H.C., 2003 - Program CARE-2 (for Capture-Recapture Part.
2). Program and user{\textquotesingle}s guide published at
\url{http://chao.stat.nthu.edu.tw}}
\end{itemize}

\setcounter{figure}{0}
\setcounter{table}{0}

\begin{adjustwidth}{-3.5cm}{0cm}
\pagestyle{CIOpage}
\authortoc{\textsc{Bedini G.}, \textsc{Paoli F.},
\textsc{Galli S.}, \textsc{Ceccherelli R.},
\textsc{Nuti F.}, \textsc{Montagnani D.},
\textsc{Baccetti N.}, \textsc{Azafzat H.},
\textsc{Bouagina A.}, \textsc{Zintu P.}, 
\textsc{Gherardi R.}}
\chapter*[Il recupero di un corrione biondo in Toscana]{Toscano per accidente: l{\textquotesingle}avventura italiana di
un giovane corrione biondo \textbf{\textit{Cursorior cursor}}\textbf{
prima del rimpatrio in Tunisia}}
\addcontentsline{toc}{chapter}{Il recupero di un corrione biondo in Toscana}

{\raggedleft
\textsc{Gianluca Bedini}$^{1}$, \textsc{Federica Paoli}$^{1}$,
\textsc{Silvia Galli}$^{1}$, \textsc{Renato Ceccherelli}$^{1}$,
\textsc{Fabio Nuti}$^{1}$, \textsc{David Montagnani}$^{2}$,
\textsc{Nicola Baccetti}$^{3}$, \textsc{Hichem Azafzat}$^{4}$,
\textsc{Adel Bouagina}$^{4}$, \textsc{Paola Zintu}$^{5}$, 
\textsc{Riccardo Gherardi}$^{1*}$ } \\

\index{Bedini Gianluca} \index{Paoli Federica} \index{Galli Silvia} \index{Ceccherelli Renato} \index{Nuti Fabio} \index{Montagnani David} \index{Baccetti Nicola} \index{Azafzat Hichem} \index{Bouagina Adel} \index{Zintu Paola} \index{Gheraridi Riccardo}
\noindent\color{MUSEBLUE}\rule{27cm}{2pt}
\vspace{1cm}
\end{adjustwidth}



\marginnote{\raggedright $^1$CRUMA -- LIPU, Via delle sorgenti 430, 57121 Livorno
(LI), Italy \\
$^2$Clinica Veterinaria Aurelia, via Aurelia
136/A, 57017 Stagno, Collesalvetti (LI), Italy \\
$^3$Istituto Superiore per la Protezione e la Ricerca
Ambientale (ISPRA), via Ca{\textquotesingle} Fornacetta 9, 40064 Ozzano
Emilia (BO), Italy \\
$^4$Association {\textquotedbl}Les Amis des
Oiseaux{\textquotedbl} (AAO), Ariana Center, Bureau C 208/209, 2080
Ariana, Tunisie \\
$^5$CRAS WWF, {\textquotedblleft}L{\textquoteright}Assiolo{\textquotedblright}, via
Donizetti Loc. Ronchi, 54100 Marina di Massa (MS), Italy \\
\vspace{.5cm}
{\emph{\small $^*$Autore per la corrispondenza: \href{mailto:ric_gherardi@hotmail.com}{ric\_\allowbreak ghe\allowbreak rar\allowbreak di@\allowbreak hot\allowbreak ma\allowbreak il.\allowbreak com}}} \\
\keywords{\textit{Cursorior cursor}, WRC, Toscana, Tunisia}
{\textit{Cursorior cursor}, WRC, Tuscany, Tunisia}
%\index{keywords}{\textit{Cursorior cursor}} \index{keywords}{WRC} \index{keywords}{Toscana} \index{keywords}{Tunisia}
}
{\small
\noindent \textsc{\color{MUSEBLUE} Summary} / On the 20 October 2011 a juvenile cream-colored courser
\textit{Cursorius cursor} was recovered after a heavy storm near Massa
-- Tuscany. The bird was hospitalized in two Tuscan WRC till its
complete rehabilitation. In June 2012, after 254 days from its
hospitalization, it was released in Bou Hedma National Park -- Tunisia.\\
}

\vspace{1cm}
Il corrione biondo \textit{Cursorius cursor}, \`e un Caradriforme di
190-210 mm di lunghezza, tipicamente legato ad ambienti semi-desertici
(Brichetti \& Fracasso 2004). In Italia risultano 125 segnalazioni
della specie relative al periodo 1817-2011, per un totale di 136-138
individui, (Verducci \textit{et al.} 2012). 

Il 20 ottobre 2011 un giovane esemplare di questa specie \`e stato
rinvenuto presso la zona industriale della citt\`a di Massa, in
Toscana. L{\textquoteright}animale \`e stato portato presso il C.R.A.S. W.W.F.
l{\textquoteright}Assiolo di Ronchi (MS) dalle persone che lo avevano
rinvenuto in difficolt\`a dopo un intenso temporale. Alla prima visita
il soggetto si presentava iporeattivo, bagnato e in leggero stato di
debilitazione. Come intervento di primo soccorso sono stati
somministrati fluidi riscaldati sottocute, vitamine e un antibiotico di
copertura.

Il giorno successivo \`e stato trasferito presso il C.R.U.M.A. della
L.I.P.U. di Livorno dove la terapia \`e proseguita per 7 giorni. Presso
tale Centro sono stati effettuati alcuni esami diagnostici di routine.
In seguito, sono stati prelevati campioni biologici per esami
specifici, effettuati presso laboratori esterni, necessari in
previsione del suo trasferimento in Nord Africa. 

Durante tutto il periodo di ricovero, il peso del corrione biondo \`e
stato monitorato a cadenza settimanale. Dopo un iniziale calo,
l{\textquoteright}animale ha ripreso ad alimentarsi regolarmente fino a
che il peso si \`e stabilizzato su valori normali per la specie.

Dopo un periodo iniziale di degenza in stabulazione stretta, finalizzata
a una miglior gestione e controllo delle sue condizioni,
l{\textquoteright}animale \`e stato trasferito in un box esterno
protetto da rete oscurante e dotato di arricchimento ambientale idoneo.
Questa nuova sistemazione gli ha permesso di godere di maggiori spazi
per muoversi e di maggior tranquillit\`a. 

In questo box l{\textquoteright}esemplare ha passato tutto il periodo
invernale grazie anche all{\textquoteright}aiuto di una lampada
riscaldante a infrarossi istallata per mitigare la temperatura di un
inverno che si \`e rivelato particolarmente rigido e nevoso per la
costa toscana.

Dopo circa un mese dal ricovero le sue condizioni fisiche risultavano
stabili e iniziava con successo i primi tentativi di volo.

A giugno, con l{\textquoteright}animale completamente ristabilito, \`e
stato possibile organizzare la sua liberazione in Tunisia, grazie anche
all{\textquoteright}intervento dell{\textquoteright}Ambasciatore
tunisino a Roma e alla compagnia di volo Tunisair.

Il 30 giugno 2012, dopo un giorno di osservazione in Tunisia presso le
strutture della \textit{Association  les Amis des Oiseaux}, e dopo 254
giorni dal suo rinvenimento, il corrione biondo \`e stato liberato in
una zona desertica del Parco Nazionale di Bou Hedma.

\begin{figure}[!h]
\centering
\includegraphics[width=.68\columnwidth]{Bedini_fig_1.jpg}
\caption{Corrione biondo con l{\textquoteright}ala destra aperta per mostrare l{\textquoteright}inizio della muta visibile nelle due remiganti primarie pi\`u interne}
\label{Bedini_fig_1}
\end{figure}

\section*{Ringraziamenti}

\`E intenzione degli autori ringraziare l{\textquoteright}ambasciatore
Tunisino a Roma Dott. Naceur Mestiri che si \`e prodigato per rendere
possibile l{\textquoteright}operazione e senza il cui aiuto tutta
questa avventura non solo sarebbe stata molto pi\`u complicata, ma non
avrebbe sicuramente ottenuto un cos\`i alto valore culturale. Inoltre,
i ringraziamenti degli autori vanno alla Dottoressa Sandra Nannipieri,
medico veterinario presso la ASL 6 di Livorno, per
l{\textquoteright}aiuto fornito nello svolgimento degli adempimenti
burocratici relativi all{\textquoteright}espatrio del corrione biondo e
alla Compagnia Aerea Tunisair per averci dato la possibilit\`a di fare
volare gratuitamente il corrione biondo da Fiumicino a Tunisi.



\section*{Bibliografia}
\begin{itemize}\itemsep0pt
	\item Brichetti P. \& Fracasso G., 2004 - \textit{Ornitologia Italiana.
2{\textdegree} volume Tetraonidae-Scolopacidae.} Alberto Perdisa
Editore, Ozzano dell{\textquotesingle}Emilia (BO): 396 pp.

	\item Verducci D., Biondi M., Sighele M. \& Norante N., 2012 - Revisione degli
avvistamenti e delle catture di corrione biondo \textit{Cursorius
cursor} in Italia con cenni sul suo status in Europa. \textit{U.D.I.
XXXVII:} 16-32.
\end{itemize}

\setcounter{figure}{0}
\setcounter{table}{0}

\begin{adjustwidth}{-3.5cm}{0cm}
\pagestyle{CIOpage}
\authortoc{\textsc{Biondi M.}, \textsc{Dragonetti M.},
\textsc{Giovacchini P.}, \textsc{Pietrelli L.}}
\chapter*[Preferenze ambientali dell{\textquoteright}occhione
nell{\textquoteright}Italia centrale]{Preferenze ambientali dell{\textquoteright}occhione \textbf{\textit{Burhinus oedicnemus}}\textbf{ in periodo riproduttivo nell{\textquoteright}Italia centrale}}
\addcontentsline{toc}{chapter}{Preferenze ambientali dell{\textquoteright}occhione
nell{\textquoteright}Italia centrale}

\textsc{Massimo Biondi}$^{1,3*}$, \textsc{Marco Dragonetti}$^{2}$,
\textsc{Pietro Giovacchini}$^{2}$, \textsc{Loris Pietrelli}$^{3}$ \\

\index{Biondi Massimo} \index{Dragonetti Marco} \index{Giovacchini Pietro} \index{Pietrelli Loris}
\noindent\color{MUSEBLUE}\rule{27cm}{2pt}
\vspace{1cm}
\end{adjustwidth}



\marginnote{\raggedright $^1$SROPU - Stazione Romana Osservazione e Protezione Uccelli, Via Britannia 36, 00183 Roma, Italia\\
$^2$GOM - Gruppo Ornitologico Maremmano\\
$^3$GAROL - Gruppo Attivit\`a Ricerche Ornitologiche del Litorale\\
\vspace{.5cm}
{\emph{\small $^*$Autore per la corrispondenza: \href{mailto:massimo.biondi54@gmail.com}{mas\allowbreak si\allowbreak mo.\allowbreak bion\allowbreak di\allowbreak 54@\allowbreak g\allowbreak ma\allowbreak il.\allowbreak com}}} \\
\keywords{\textit{Burhinus oedicnemus, }preferenze
ambientali, Lazio, Toscana}
{\textit{Burhinus oedicnemus}, habitat preference,
Latium, Tuscany}
%\index[keywords]{\textit{Burhinus oedicnemus}} \index[keywords]{Preferenze ambientali} \index[keywords]{Lazio} \index[keywords]{Toscana}
}
{\small
\noindent \textsc{\color{MUSEBLUE} Summary} / We analyzed the breeding habitat preferences of stone curlew
\textit{Burhinus oedicnemus} in central Italy. Data were collected from
194 breeding territories (111 in Tuscany and 83 in Latium). The species
selected 14 different biotopes (following Corine categories): 13 in Latium and 10 in Tuscany.\\
}
\vspace{1cm}

L{\textquoteright}occhione \textit{Burhinus oedicnemus} nidifica in
Toscana con una stima di 150-200 coppie {(Tinarelli
}\textit{{et al.}}{ 2009)} e nel
Lazio con 50-70 coppie {(Meschini in Brunelli
}\textit{{et al.}}{ 2011)}. I dati
sulle preferenze ambientali della specie sono sempre stati scarsi e
frammentari per la Toscana e parziali per il Lazio, eccetto che per la
popolazione viterbese (Meschini 2010).

Nel presente studio, per analizzare le preferenze riproduttive
dell{\textquoteright}occhione nell{\textquoteright}Italia centrale
abbiamo utilizzato le categorie fitosociologiche proposte dal progetto
{\textquotedblleft}CORINE Biotopes{\textquotedblright} (Devillers
\textit{et al}. 1991) modificate ad uso ornitologico (Boano 1997).
Unitamente alla tipologia ambientale \`e stata registrata la fascia
altimetrica utilizzata da ciascuna coppia. I dati raccolti hanno
interessato le provincie di Grosseto, Roma e Latina. Il periodo di
studio ha analizzato i dati inediti sia per la Toscana (1995-2011) sia
per il Lazio (2008-2012) raccolti nell{\textquoteright}arco della
stagione riproduttiva (1{\textdegree} marzo -- 30 settembre). Abbiamo
considerato occhioni nidificanti laddove sono stati registrati
individui con evidente comportamento territoriale protratto per pi\`u
di 15 giorni, oppure laddove siano state rinvenute uova e/o pulli. Sono
stati analizzati complessivamente 194 territori di cui 111 in Toscana e
83 nel Lazio. La specie ha selezionato 14 diverse tipologie ambientali
(13 nel Lazio; 10 in Toscana) e tra queste, cinque sono apparse
predominanti: praterie e steppe calcaree (37.6\%), gariga su suoli
calcarei (11.8), coltivazioni estensive e tradizionali (10.8), steppe
cerealicole (8.2\%) e greti fluviali (7.2\%) (Fig. \ref{Biondi_fig_1}). In Toscana la
specie ha selezionato praterie e steppe calcaree (45\%), caratterizzate
da pascoli permanenti, e coltivazioni estensive tradizionali (18\%), in
gran parte rappresentate da oliveti maturi a conduzione tradizionale e
da colture foraggere estensive collegate
all{\textquotesingle}allevamento ovino. Nel Lazio
l{\textquoteright}occhione si insedia per il 55.4\% in due ambienti
naturali con affioramenti calcarei pascolati (ovini, bovini, equini) di
cui uno, gariga su suoli calcarei, assente in Toscana e coincidente, in
gran parte, con l{\textquoteright}area della ZPS monti della Tolfa. La
specie dimostra inoltre di poter colonizzare ambienti succedanei a
quelli steppici naturali o semi-naturali insediandosi anche in habitat
fortemente antropizzati come i siti industriali attivi o abbandonati
(6.1\%) ove da alcuni anni risulta anche svernante (Lazio) (Biondi
\textit{et al.} 2011). Dal punto di vista altitudinale
l{\textquoteright}occhione sembra prediligere due fasce altimetriche:
0-50 m (29.8\%) e 101-200 m (27.8\%) con un limite altitudinale posto a
500-600 m (1.5\%). 

\begin{adjustwidth}{-3.5cm}{-1cm}
\begin{figure}[!h]
\centering
\includegraphics[width=1.2\columnwidth]{Biondi_fig_1.png}
\caption{Preferenze ambientali dell'occhione \textit{Burhinus oedicnemus} in Italia centrale}
\label{Biondi_fig_1}
\end{figure}
\end{adjustwidth}

\newpage
\section*{Bibliografia}
\begin{itemize}\itemsep0pt
	\item Biondi M., Pietrelli L., Scrocca R. \&  Meschini A., 2011 - New
Stone-Curlew \textit{Burhinus oedicnemus }wintering site in central
Italy. \textit{Wader Study Group Bulletin} 118 (1):63-64.

	\item Boano G., 1997 - Proposta di una classificazione degli habitat ad uso
ornitologico. In: Brichetti P. \& Gariboldi A. 1997. \textit{Manuale
pratico di ornitologia. }Edagricole, Bologna: 153-165.

	\item {Devillers P., Devillers-Terschuren J.P. \& CORINE
BIOTOPES EXPERTS GROUP, 1991 - }\textit{{CORINE
biotopes manual}}{. Part 2. Habitats of the European
Community. Commission of the European Community.
}{Brussels.}

	\item {Meschini A., 2010 -
}\textit{{L{\textquoteright}Occhione tra fiumi e
pietre.}}{ Edizioni Belvedere, Latina: pp.174.}

	\item {Meschini A., 2011 - L{\textquoteright}occhione
}\textit{{Burhinus oedicnemus}}{.
In.} Brunelli M., Sarrocco S., Corbi F., Sorace A., De Felici, Boano
A., Guerrieri G., Meschini A. \& Roma S. (a cura di) 2011 -
\textit{Nuovo atlante degli Uccelli Nidificanti nel Lazio}. Edizioni
ARP-Agenzia Regionale Parchi, Roma: 146-147.

	\item {Tinarelli R., Alessandria G., Giovacchini P., Gola L.,
Ientile R. \& Meschini A., 2009 - Consistenza e distribuzione
dell{\textquoteright}occhione in Italia: aggiornamento al 2008. In:
Giunchi D., Pollonara E. \& E. Baldaccini  (a cura di) 2009 -
L{\textquoteright}occhione (}\textit{{Burhinus
oedicnemus}}{): Biologia e conservazione di una specie
di interesse comunitario - Indicazioni per la gestione del territorio
e delle aree protette. Conservazione e gestione della natura.
}\textit{{Quaderni di
documentazione,}}{ 7: 45-50.}

\end{itemize}

\setcounter{figure}{0}
\setcounter{table}{0}

\begin{adjustwidth}{-3.5cm}{0cm}
\pagestyle{CIOpage}
\authortoc{\textsc{Brunelli M.}, \textsc{Cento M.}, 
\textsc{Sarrocco S.}, \textsc{Biondi M.}, 
\textsc{Boano A.}, \textsc{De Santis E.}, 
\textsc{Fraticelli F.}, \textsc{Hueting S.}, 
\textsc{Meschini A.}, \textsc{Purificato G.}, 
\textsc{Scrocca R.}}
\chapter*[Atlante degli uccelli svernanti del Lazio]{\bfseries
Atlante degli uccelli d{\textquoteright}Italia in inverno: analisi dei
dati preliminari nel Lazio (2009/10 - 2012/13)}
\addcontentsline{toc}{chapter}{Atlante degli uccelli svernanti del Lazio}

\textsc{Massimo Brunelli}$^{1*}$, \textsc{Michele Cento}$^{1}$, 
\textsc{Stefano Sarrocco}$^{2}$, \textsc{Massimo Biondi}$^{1}$, 
\textsc{Aldo Boano}$^{1}$, \textsc{Emiliano De Santis}$^{1}$, 
\textsc{Fulvio Fraticelli}$^{1}$, \textsc{Steven Hueting}$^{1}$, 
\textsc{Angelo Meschini}$^{1}$, \textsc{Giovanni Purificato}$^{1}$, 
\textsc{Roberto Scrocca}$^{1}$ \\

\index{Brunelli Massimo} \index{Cento Michele} \index{Sarrocco Stefano} \index{Biondi Massimo} \index{Boano Aldo} \index{De Santis Emiliano} \index{Fraticelli Fulvio} \index{Hueting Steven} \index{Meschini Angelo} \index{Purificato Giovanni} \index{Scrocca Roberto}
\noindent\color{MUSEBLUE}\rule{27cm}{2pt}
\vspace{1cm}
\end{adjustwidth}



\marginnote{\raggedright $^1$\textbf{\textsuperscript{ }}SROPU,\textbf{ }Stazione
Romana per l{\textquoteright}Osservazione e la Protezione degli
Uccelli, Via Britannia 36, 00183 Roma, Italia \\
$^2$ARP, Agenzia Regionale per i Parchi, Via del
Pescaccio 96, 00166 Roma, Italia \\
\vspace{.5cm}
{\emph{\small $^*$Autore per la corrispondenza: \href{mailto:mss.brunelli@tin.it}{mss.brunelli@tin.it}}} \\
\keywords{Atlante, uccelli, inverno, Lazio}
{Atlas, birds, winter, Latium}
%\index{keywords}{Atlante} \index{keywords}{Uccelli} \index{keywords}{Inverno} \index{keywords}{Lazio}
}
{\small
\noindent \textsc{\color{MUSEBLUE} Summary} / We reported for the Latium region the results of the
first four years of the project
{\textquotedblleft}Atlas of the Birds of Italy in winter 2009/2010 --
2014/2015{\textquotedblright}. We recorded a total of 213
and an average of 48,4 (SD 24,2) species for square
unit of side 10 km, with high richness along coastal plain and hilly
areas, as well as around wetlands and lower values in the Apennines.
}


\section*{Introduzione}

La conoscenza della distribuzione e delle abbondanze degli uccelli
presenti in inverno sul territorio rappresenta una fonte di
informazione importante per attivare strumenti di conservazione delle
specie e gestione del territorio. Con queste finalit\`a il portale web
ornitho.it, e le associazioni che in esso si riconoscono, hanno
promosso il progetto  {\textquotedblleft}Atlante degli uccelli
d{\textquoteright}Italia in inverno 2009/2010 --
2014/2015{\textquotedblright}. In questo contributo presentiamo i
risultati ottenuti nel Lazio (aggiornati al 31 gennaio 2013) con
l{\textquoteright}obiettivo di analizzare i risultati raggiunti e
individuare le aree su cui indirizzare
prioritariamente lo sforzo di ricerca nella prossima stagione di
rilevamento.

\section*{Metodi}

I dati sono stati raccolti in varia misura da oltre 150 rilevatori dal
1{\textdegree} dicembre al 31 gennaio degli anni 2009-2013. La base
cartografica utilizzata \`e rappresentata dalla griglia UTM con unit\`a
di rilevamento quadrate di 10 km di lato. In ogni
unit\`a di rilevamento sono stati indagati tutti gli ambienti presenti
al fine di registrare il maggior numero di specie. Sono state escluse
le unit\`a di rilevamento ricadenti prevalentemente in regioni
confinanti e quelle costiere con ridottissima
estensione di superficie emersa. Le unit\`a di rilevamento indagate
sono state 206; le 5 in cui ricadono le isole Pontine sono state
accorpate in due unit\`a, pertanto elaborazioni e rappresentazioni
grafiche si riferiscono a 203 unit\`a di rilevamento. 

Per svolgere~inferenze sulle specie~attese~abbiamo elaborato alcune
analisi di \textit{geoprocessing,} utilizzando le specie osservate
nelle unit\`a di rilevamento e le 118 unit\`a di paesaggio regionali,
quest{\textquoteright}ultime considerate singolarmente, in prima
approssimazione, uniformi sia morfologicamente sia come uso del suolo.
A ogni unit\`a di paesaggio abbiamo assegnato un
valore pari al numero medio delle unit\`a di rilevamento in
essa ricadenti e tale valore lo
abbiamo successivamente riattribuito alle unit\`a di
rilevamento. Alle unit\`a di rilevamento con campionamenti incompleti
abbiamo assegnato i valori di quelle contigue che presentavano una
maggiore completezza di campionamento.

\section*{Risultati e discussione}
Complessivamente sono state rinvenute 213 specie, la ricchezza media per
unit\`a di rilevamento \`e risultata di 48,4 specie (DS 24,2); le
unit\`a di rilevamento con i valori maggiori sono quelle delle pianure
costiere (es. litorale romano) e quelle collinari (es. monti della
Tolfa), nonch\'e quelle con presenza di zone umide (es. lago di
Bolsena, litorale pontino, piana di Rieti). Le unit\`a di rilevamento
con i valori minori sono risultate invece quelle ricadenti lungo la
dorsale appenninica, caratterizzate da quote pi\`u elevate e dalla
presenza di estese aree montane.
Le unit\`a di paesaggio, analogamente, evidenziano i valori maggiori
nelle lagune costiere e nelle paludi salse (intervallo 117,0-65,5
specie), nei laghi dulciacquicoli (101,0-8,0 specie) e lungo le pianure
costiere (85,0-43,0 specie); i valori minori sono collegati alle conche
interne e ai rilievi carbonatici del Preappennino e
dell{\textquoteright}Appennino (47,3-3,7 specie), sebbene in queste
unit\`a la copertura dei rilevamenti non sia da ritenersi esaustiva.
Le elaborazioni effettuate hanno permesso di ricavare informazioni sul
numero di specie attese (Fig. \ref{Brunelli_fig_1}) e d{\textquoteright}individuare le UR
con una copertura ritenuta insufficiente, da visitare prioritariamente
nella prossima stagione di rilevamento.
Il confronto con la distribuzione della ricchezza
dell{\textquoteright}avifauna nidificante (Brunelli \textit{et al}.
2011) evidenzia come alcune aree mantengano alti valori di ricchezza
anche durante il periodo invernale (es. litorale romano, monti della
Tolfa, piana di Rieti), altre aumentino il loro valore (es. litorale
pontino) e altre lo diminuiscano sensibilmente (es. dorsale
appenninica).

\begin{figure}[!h]
\centering
\includegraphics[width=.74\columnwidth]{Brunelli_fig_1.png}
\caption{Numero di specie attese in inverno nelle particelle di 10 km di lato (nelle particelle \`e riportato il numero di specie osservate)}
\label{Brunelli_fig_1}
\end{figure}
\section*{Ringraziamenti}
Ringraziamo tutti i numerosi rilevatori, in particolare: Alessandro
Ammann, Gabriella Biondi, Fabrizio Bulgarini, Mario Cappelli, Monica
Carabella, Alberto Cardillo, Carlo Castellani, Emanuele G. Condello,
Davide de Rosa, Santino di Carlo, Brendan Doe, Roberto Gildi, Daniele
Iavicoli, Gigliola Magliocco, Alberto Manganaro, Fabrizio Mantero,
Riccardo Molajoli, Sergio Muratore, Alessio Rivola, Enzo Savo, Fabio
Scarf\`o, Alberto Sorace, Maurizio Sterpi, Marco Trotta, Claudio
Zanotti.
\section*{Bibliografia}
\begin{itemize}\itemsep0pt
	\item Brunelli M., Sarrocco S., Corbi F., Sorace A., Boano
A\textcolor{red}{.}, De Felici S., Guerrieri G., Meschini A. \& Roma S.
(a cura di), 2011 - \textit{Nuovo Atlante degli Uccelli Nidificanti nel
Lazio}. Edizioni ARP (Agenzia Regionale Parchi), Roma, 464 pp.
\end{itemize}

\begin{otherlanguage}{english}
\setcounter{figure}{0}
\setcounter{table}{0}

\begin{adjustwidth}{-3.5cm}{0cm}
\pagestyle{CIOpage}
\authortoc{\textsc{Campioni L.}, \textsc{Del Mar Delgado M.},
\textsc{Bettega C.}, \textsc{Penteriani V.}}
\chapter*[Repeatability of movement behaviour]{Repeatability of movement parameters inside home range
boundaries in a long-lived species: the eagle owl \textbf{\textit{Bubo
bubo}}}
\addcontentsline{toc}{chapter}{Repeatability of movement behaviour}


\textsc{Letizia Campioni}$^{1*}$, \textsc{Maria Del Mar Delgado}$^{2}$,
\textsc{Chiara Bettega}$^{3}$, \textsc{Vincenzo Penteriani}$^{3,4}$ \\

\index{Campioni Letizia} \index{Del Mar Delgado Maria} \index{Bettega Chiara} \index{Penteriani Vincenzo}
\noindent\color{MUSEBLUE}\rule{27cm}{2pt}
\vspace{1cm}
\end{adjustwidth}



\marginnote{\raggedright $^1$MARE -- Marine and Environmental Sciences Centre,
ISPA - Instituto Universit\'ario, Lisbon, Portugal \\
$^2$Department of Biosciences, University of Helsinki,
Helsinki, Finland \\
$^3$Department of Conservation Biology, Estaci\'on
Biol\'ogica de Do\~nana, C.S.I.C., Seville, Spain \\
$^4$Research Unit of Biodiversity (UMIB, UO-CSIC-PA),
Oviedo University - Campus Mieres, Mieres, Spain \\
\vspace{.5cm}
{\emph{\small $^*$Autore per la corrispondenza: \href{mailto:letiziacampioni@hotmail.com}{le\allowbreak ti\allowbreak zia\allowbreak cam\allowbreak pio\allowbreak ni@\allowbreak hot\allowbreak ma\allowbreak il.\allowbreak com}}} \\
\keywords{Sierra Norte Spagna, ripetibilit\`a, comportamento di
movimento, variazione inter-individuale, \textit{Bubo bubo}}
{Sierra Norte Spain, repeatability, animal movement, individual
consistency, between-individual variation, \textit{Bubo bubo}}
%\index{keywords}{Sierra Norte} \index{keywords}{Spagna} \index{keywords}{Ripetibilit\`a} \index{keywords}{Comportamento di movimento} \index{keywords}{Variazione inter-individuale} \index{keywords}{\textit{Bubo bubo}}
}
{\small
\noindent \textsc{\color{MUSEBLUE} Summary} / Territorial species as the eagle owl \textit{Bubo bubo }that repeatedly
move within fixed home ranges are expected to have an extensive
knowledge of their surroundings. As a consequence, owls can be expected
to be highly repeatable in their movement parameters. We found that the
repeatability of speed, time step, total distance and step length of 26
breeding owls ranged between 15-25\% revealing a considerable
individual consistency.\\
}

\vspace{1cm}
Observed movement patterns are the response of the interaction between
environmental variables and individual state (B\"orger \textit{et al.}
2008). Surprisingly, even individuals of the same species experiencing
similar environmental condition can exhibit different behavioural
responses, being these responses highly repeatable within individuals
(Biro \& Adriaenssens 2013). Here, the variation we focus on is how
individuals of a long-lived, territorial species, the nocturnal eagle
owls \textit{Bubo bubo,} move during their daily activity over multiple
years. Eagle owls that repeatedly move within fixed home ranges are
expected to have an extensive knowledge of their surroundings. As a
consequence, they are expected to show to some extent a systematic
movement strategy based on available \textit{a priori} information
and/or based on an individual behavioural consistency. If it is so, we
can expect that movement patterns varied much less between repeated
daily trajectories than between different individuals. This study was
conducted in a hilly area of the Sierra Norte of Seville located in
south-western Spain. From 2004 to 2010, 26 breeding individuals (19
males and 7 females) from 19 nests were trapped (Campioni \textit{et
al}. 2013; Penteriani \textit{et al}. 2010) and fitted with a 30-g
radio-transmitter using a Teflon ribbon backpack harness (Biotrack,
UK). We radio tracked territory holders individually throughout the
night (from 1 h before sunset to 1 h after sunrise) during 290
continuous radio tracking sessions. Individual nightly movement
behaviour was characterised by four variables: 1) total distance, as
the sum of the distance between successive steps of the nightly
displacements; 2) step length, as the distance between successive
locations; 3) speed, as the step length divided by the time interval
between successive locations; and 4) time step, as the time elapsed
between successive moves. Then, movement variables were grouped and
analyzed at a daily temporal scale, where for each individual we
constructed log10-transformed movement parameter frequency
distributions (i.e., MPFD). Accordingly, we selected four statistics
able to comprehensively describe MPFDs: 1) minimum value, 2) maximum
value, 3) median, 4) geometric mean, and 5) coefficient of variation
(CV) and then we estimated their repeatability (R). The repeatability
analysis of MPFDs of breeding owls showed a considerable individual
consistency in all movement parameters ($\Delta $R: 15-25\%; Fig. \ref{Campioni_fig_1}).
Total distance was the parameter with the highest repeatability (mean
{\textpm} SE and 95\% CI; 0.29 {\textpm} 0.08; [0.127-0.44]). Moreover,
male was the sex showing higher repeatability of movement parameters (R
mean: 0.19, $\Delta $R = 0.04-0.30), though females seemed to be more
consistent than males with respect to time step parameter (R mean:
0.18, $\Delta $R = 0.01-0.36). Accordingly, 95\% CI of repeatability
estimates for most of the statistics were well above zero
(statistically significant at $\alpha $ = 0.05). These results
suggested that owls move following a consistent movement strategy,
i.e., with similar movement parameters every night while maintaining
some degree of variation across nights. Lastly, repeatability estimates
of different owls that have been owners of the same territory or mate
of the same pair were substantially smaller ($\Delta $R = 0.0-0.08).
Namely, between-territory variability was higher than within-territory
variability. Individuals behaving in similar environmental condition
seemed to show a substantial behavioural flexibility. We suggest
homogeneity of habitat in our study area and small home range size
(mean HR: {\textless} 250 ha) to be responsible for the moderate
between-individual variation in movement patterns. 

\begin{figure}[!h]
\centering
\includegraphics[width=.8\columnwidth]{Campioni_fig_1.png}
\caption{Boxplot of the four statistics used to describe the MPFD of speed movement parameter of 26 breeding owls with the relative values of repeatability}
\label{Campioni_fig_1}
\end{figure}

\section*{Acknowledgements}

The work was funded by two research projects of the Spanish Ministry of
Science and Innovation, the Ministry of Education and Science ---
C.S.I.C., the Junta of Andaluc{\i}\'a and LICOR43. During this work,
L.C. was supported by the post-doctoral grant (SFRH/\allowbreak BPD/\allowbreak 89904/\allowbreak 2012)
from FCT (Funda\c{c}ao Ciencia e Tecnologia, Portugal).

\section*{Bibliography}
\begin{itemize}\itemsep0pt
	\item Biro P.A., \& Adriaenssens B., 2013 - Predictability as a personality
trait: consistent differences in intraindividual behavioral variation.
\textit{American Naturalist.}

	\item B\"orger L., Danziel B.D. \& Fryxell J.M., 2008 - Are there general
mechanisms of animal home range behaviour? A review and prospects for
future research. \textit{Ecol Lett }11: 637--650.

	\item Campioni L., Delgado M.M., Louren\c{c}o R., Bastianelli G., Fern\'andez
N. \& Penteriani V., 2013 - Individual and spatio-temporal variations
in the home range behaviour of a long-lived, territorial species.
\textit{Oecologia} 172: 371--385.

	\item Penteriani V., Delgado M.M., Campioni L. \& Lourenco R. (2010).
Moonlight makes owls more chatty. \textit{PLoS One} 5 (1), e8696.
\end{itemize}
\end{otherlanguage}
\setcounter{figure}{0}
\setcounter{table}{0}

\begin{adjustwidth}{-3.5cm}{0cm}
\pagestyle{CIOpage}
\authortoc{\textsc{Comparato L.}, \textsc{Caprio E.},
\textsc{Boano G.}, \textsc{Rolando A.}}
\chapter*[Vocazionalit\`a ambientale della starna in provincia di
Asti]{Analisi di vocazionalit\`a ambientale della starna
\textbf{\textit{Perdix perdix}}\textbf{ nella Z.R.C.
{\textquotedbl}Casalino{\textquotedbl} in provincia di Asti}}
\addcontentsline{toc}{chapter}{Vocazionalit\`a ambientale della starna in provincia di
Asti}

\textsc{Laura Comparato}$^{1*}$, \textsc{Enrico Caprio}$^{1,2}$,
\textsc{Giovanni Boano}$^{3}$, \textsc{Antonio Rolando}$^{1}$ \\

\index{Comparato Laura} \index{Caprio Enrico} \index{Boano Giovanni} \index{Rolando Antonio}
\noindent\color{MUSEBLUE}\rule{27cm}{2pt}
\vspace{1cm}
\end{adjustwidth}



\marginnote{\raggedright $^1$Dipartimento di Scienze della Vita e Biologia dei
Sistemi, Universit\`a di Torino \\
$^2$Scuola di Biodiversit\`a di Villa Paolina, c/o
Consorzio Asti Studi Superiori - Piazzale F. De
Andre{\textquoteright}, 14100 Asti, Italy \\
$^3$Museo Civico di Storia Naturale di Carmagnola (TO) \\
\vspace{.5cm}
{\emph{\small $^*$Autore per la corrispondenza: \href{mailto:laura.comparato@tiscali.it}{lau\allowbreak ra.\allowbreak com\allowbreak pa\allowbreak ra\allowbreak to@\allowbreak tis\allowbreak ca\allowbreak li.\allowbreak it}}} \\
\keywords{\textit{Perdix perdix},provincia di Asti,
vocazionalit\`a ambientale, agroecosistemi}
{Asti province, \textit{Perdix perdix}, habitat suitability, agro-ecosystems}
%\index{keywords}{\textit{Perdix perdix}} \index{keywords}{Provincia di Asti} \index{keywords}{Vocazionalit\`a ambientale} \index{keywords}{Agroecosistemi}
}
{\small
\noindent \textsc{\color{MUSEBLUE} Summary} / This is a small-scale ecological study that evaluates the habitat
suitability for the grey partridge \textit{Perdix perdix} of an area of
historical presence. The results don't show a
reduction of suitable areas for the species. Therefore the causes of
this absence are due to rainy and snowy precipitations and/or
predation.
}



\section*{Introduzione}

La starna \textit{Perdix perdix} \`e una specie tipica di ambienti
coltivati che ha subito, in tutta Europa, un netto declino nella sua
distribuzione negli anni 1970-1990, imputabile a un crollo della
diversit\`a degli agro-ecososistemi, principalmente cerealicoli,
parallelamente a una mancata gestione degli ambienti.

Le modificazioni
dell{\textquoteright}agricoltura hanno determinato severi impatti sulla
biodiversit\`a, sopratutto sulle specie particolarmente legate agli
agro-ecosistemi e pertanto ritenute dei buoni indicatori (Donald
\textit{\textcolor[rgb]{0.0,0.0,0.039215688}{et
al}}\textcolor[rgb]{0.0,0.0,0.039215688}{. 2001).}

Scopo di questo studio \`e valutare modelli di vocazionalit\`a
ambientale per la starna a scala locale, in un'area di
presenza storica della specie, e identificare possibili minacce per la
sua conservazione. 

\section*{Metodi}
L'area di studio \`e compresa
all'interno della Zona di Ripopolamento e Cattura
(ZRC) {\textquotedbl}Casalino{\textquotedbl} (856 ha), territorio
collinare poco acclive, 140-261 m s.l.m., prevalentemente agricolo e
con bassa densit\`a di popolazione. La starna \`e una specie protetta
nelle province di Asti e Alessandria, interdetta alla caccia da met\`a
degli anni '90. 
{La presenza della starna \`e stata valutata tramite
conteggi pre-riproduttivi con l'Indice Chilometrico di
Abbondanza e uno post-riproduttivo, con l'ausilio di
cani da ferma, per l'anno 2012. Inoltre sono state
monitorate, con la tecnica del mappaggio, anche altre quattro specie
indicatrici di agro-ecosistemi (}\textit{{Emberiza
hortulana}}{, }\textit{{Miliaria
calandra}}{, }\textit{{Streptopelia
turtur }}{e }\textit{{Coturnix
coturnix}}{). I modelli di idoneit\`a ambientale sono
stati formulati utilizzando il software MaxEnt 3.1 (Phillips
}\textit{{et al.}}{ 2006) integrando
ai dati di presenza i monitoraggi della confinante Z.R.C.
{\textquotedbl}Val Cerrina{\textquotedbl} (AL), di cui si} ha a
disposizione una serie storica di monitoraggi effettuati tra il 2002 e
il 2009 dall'Osservatorio Faunistico della Provincia
di Alessandria{. }

L'approccio della massima entropia permette di
utilizzare le informazioni ambientali derivanti dalle interazioni con i
punti di presenza certa della specie, per poi generalizzarle
all'intero territorio indagato (Phillips \textit{et}
\textit{al}. 2006). 

Sono stati cos\`i identificati i fattori ambientali pi\`u importanti nel
determinare la presenza della specie, \`e stata generata una mappa di
distribuzione potenziale e idoneit\`a ambientale, per
l'intero territorio.

{Le variabili ambientali considerate sono state: uso
del suolo, ottenuto dai piani forestali territoriali della Regione
Piemonte (Regione Piemonte 2009), modello digitale del terreno,
pendenza ed esposizione dell'area di studio. Le mappe
di uso del suolo sono state aggiornate nel 2012 utilizzando foto aeree
recenti per mappare elementi lineari quali siepi o piccoli boschi. Le
categorie di uso del suolo sono state: boschi di latifoglie,
seminativi, siepi, prato-pascoli, torrenti e zone umide, boscaglie di
invasione, vigneti e urbanizzato, il tutto rasterizzato con immagini
con pixel 10x10m.  }

Il modello \`e stato fatto con i dati del monitoraggio per
l'anno 2004 in cui si ha la densit\`a massima della
starna e proiettato con le mappe di uso del suolo aggiornate al 2012.

I modelli sono stati selezionati sulla base
dell'analisi: i) delle curve di ROC che permettono di
determinare il limite soglia, valore al disopra del quale si ha la
massima idoneit\`a ambientale per la specie e ii)
dell'AUC che definisce la capacit\`a predittiva del
modello: il modello pu\`o considerarsi efficiente se il valore di AUC
supera lo 0,8 (Menel \textit{et al}. 2001).

\section*{Risultati e discussione}

Dai monitoraggi effettuati, possiamo concludere che la starna non \`e
pi\`u presente nell'area di studio, poich\'e non \`e
stato osservato alcun individuo n\'e trovato alcun segno di presenza
per l'anno 2012. 

I modelli ottenuti con MaxEnt per il 2004 e la proiezione degli stessi
sul 2012 mostrano un buon livello di efficienza con un AUC
{\textgreater} 0,8.

Punto di forza del presente lavoro \`e aver utilizzato i dati di
presenza delle quattro specie indicatrici di agro-ecosistemi (che
condividono l{\textquoteright}habitat della starna), monitorate nel
2012, per confrontare il modello di idoneit\`a
dell{\textquoteright}area di studio con quello derivante da specie
indicatrici di agroecosistemi, potenzialmente estendibile anche alla
starna, al fine di verificare lo stato attuale del territorio al fine
di verificare l{\textquoteright}idoneit\`a del territorio a ospitare la
specie; il valore AUC anche in questo caso \`e risultato maggiore di
{\textgreater} 0,8. Le variabili ambientali che hanno contribuito
positivamente alla costruzione di tutti i modelli sono risultate quelle
proprie di ambienti agricoli (seminativi, frutteti - vigneti ed
elementi lineari), le aree boscate sono invece risultate non idonee.
Tale risultato trova conferma in uno studio di Meriggi \textit{et al}.
(1991) che descrive come la starna selezioni gli elementi lineari
durante tutte le stagioni, ma con percentuale di utilizzo differente
secondo le stagioni, maggiore in inverno; le aree boscate sono invece
evitate tutto l'anno. Confrontando il modello del 2012
e quello del 2004 la superficie di territorio idoneo a ospitare la
specie non varia (314,69 ha per il 2004 e 314,31 ha per il 2012).  La
causa della riduzione della specie non sembra dunque imputabile a
un'alterazione nella gestione degli ambienti agricoli
poich\'e il territorio non ha subito una modificazione di utilizzo
negli ultimi trenta anni ed \`e sempre riconducibile a un ecosistema
rurale di tipo cerealicolo (fonte ISTAT).  

Le cause responsabili dell'assenza della starna in
quest{\textquoteright}area sono dunque imputabili ad altri fattori
quali l'andamento delle precipitazioni piovose
estivo-primaverili e nevose che possono influenzare negativamente le
dinamiche di popolazioni. L'andamento delle piogge
negli ultimi anni, tra giugno e luglio, potrebbe aver ridotto il
successo della covata e la disponibilit\`a di insetti nel periodo
estivo, fondamentali per le prime fasi vitali dei pulli.
L'inverno 2012 si \`e inoltre caratterizzato per
un'abbondante nevicata con temperature medie minime in
febbraio al disotto dei -10{\textdegree}C (picco -18.7 {\textdegree}C)
e coltre nevosa superiore ai 40 cm (valore neve cumulata 78 cm).
L'ostacolo non \`e la nevicata di per s\'e bens\`i il
gelo dei giorni successivi che pu\`o causare la morte per inedia data
l'incapacit\`a di raggiungere le risorse trofiche
(Cocchi \textit{et al.} 1993). Dal momento che il prelievo venatorio
per la specie non \`e previsto nelle due province limitrofe, tra i
fattori pi\`u importanti di limitazione vi pu\`o essere inoltre la
forte pressione predatoria esercitata in particolar modo dalla volpe
\textit{Vulpes vulpes} e dal gatto domestico.  

\section*{Ringraziamenti}

Ringraziamo l{\textquoteright}area Agricoltura della provincia di Asti
in particolare l{\textquoteright}Ufficio caccia, pesca e tartufi. Si
ringraziano inoltre l{\textquoteright}Osservatorio faunistico della
provincia di Alessandria per aver fornito i dati e tutte le persone che
hanno aiutato durante l'attivit\`a di campo

\section*{Bibliografia}
\begin{itemize}
	\item Cocchi R., Govoni M. \& Toso S., 1993 - La starna. \textit{Istituto
Nazionale per la Fauna Selvatica, Documenti Tecnici}, 14.

	\item Donald P. F., Green R. E. \& Heath M. F., 2001 - Agricultural
intensification and the collapse of Europe{\textquoteright}s farmland
bird populations. \textit{Proc. Roy. Soc}., Lond. B (268): 25--29.

	\item Manel S., Williams H.C. \& Ormerod S.J., 2001 - \textit{Evaluating
presence-absence models in ecology: the need to account for
prevalence}. \textit{Journal of Applied Ecology,} (38): 921-931.

	\item Meriggi A., Montagna D. \& Zacchetti D., 1991 - Habitat use by
partridges (\textit{Perdix perdix} and \textit{Alectoris rufa}) in an
area of northern Apennines, Italy. \textit{Bolletino di zoologia,}
(58): 85-90.

	\item Phillips S.J., Anderson R.P. \& Schapire R.E., 2006 - Maximum entropy
modeling of species geographic distributions. \textit{Ecological
Modeling,} (190): 231-259.
\end{itemize}

\setcounter{figure}{0}
\setcounter{table}{0}

\begin{adjustwidth}{-3.5cm}{0cm}
\pagestyle{CIOpage}
\authortoc{\textsc{De Giacomo U.}, \textsc{Guerrieri G.}}
\chapter*[Il nibbio bruno nella discarica di Malagrotta (Roma)]{Osservazioni sulla presenza di giovani di nibbio bruno \textbf{\textit{Milvus migrans}}\textbf{ nella discarica di Malagrotta
(Roma)}}
\addcontentsline{toc}{chapter}{Il nibbio bruno nella discarica di Malagrotta (Roma)}

\textsc{Umberto De Giacomo}$^{1*}$, \textsc{Gaspare Guerrieri}$^{1,2}$ \\

\index{De Giacomo Umberto} \index{Guerrieri Gaspare}
\noindent\color{MUSEBLUE}\rule{27cm}{2pt}
\vspace{1cm}
\end{adjustwidth}

\marginnote{\raggedright $^1$ALTURA - Associazione per la tutela degli uccelli
rapaci e dei loro ambienti \\
$^2$GAROL - Gruppo Attivit\`a Ricerche Ornitologiche del
Litorale \\
\vspace{.5cm}
{\emph{\small $^*$Autore per la corrispondenza: \href{mailto:udegiacomo@libero.it}{udegiacomo@libero.it}}} \\
\keywords{\textit{Milvus migrans}, giovani, discariche di rifiuti,
Italia centrale}
{\textit{Milvus migrans}, young, landfills, central
Italy}
%\index{keywords}{\textit{Milvus migrans}} \index{keywords}{Giovani} \index{keywords}{Discariche di rifiuti} \index{keywords}{Italia centrale}
}
{\small
\noindent \textsc{\color{MUSEBLUE} Summary} / Young black kites \textit{Milvus migrans}\textbf{ }were observed after
the fledglings, in Malagrotta landfill, main trophic area for suburban
Rome population (about 50 breeding pairs), in 2012 and 2013 years.
Their presence grows until the second week of August, while that of
other individuals, progressively decreases from the maximum value
recorded at the end of July. It seems evident, therefore, that many
young reside for a limited amount of time near the dump before
migrating, stopping and feeding at this site before migration.
}


\section*{Introduzione}

Presente da marzo ad agosto, il nibbio bruno, utilizza le discariche di
rifiuti quali aree di foraggiamento. Nel Lazio, la popolazione
riproduttiva pi\`u numerosa (39-54 coppie) \`e quella presente nel
circondario della discarica di Roma (Guerrieri \& De Giacomo 2012).

Scopo del lavoro \`e stato quello di analizzare la frequentazione della
discarica da parte dei giovani dell{\textquoteright}anno di nibbio
bruno, nel periodo successivo agli involi. Infatti, differenze nel
calendario fenologico tra le loro partenze migratorie e quelle degli
adulti, porrebbero l{\textquoteright}accento
sull{\textquoteright}importanza di questi impianti quali aree trofiche
e di sosta pre-migratoria soprattutto per i giovani.

\section*{Metodi}

La discarica di Malagrotta (41$^\circ$ 51'
21'' N - 12$^\circ$
20'  24''  E) si
estende per 200 ha nella periferia occidentale di Roma, citt\`a della
quale ospita i rifiuti (circa 4000 t/giorno) dal
1975. L{\textquoteright}area, posta a 4,5 km dal fiume Tevere, \`e
inserita in un contesto rurale in cui sono presenti anche attivit\`a
estrattive e industriali legate ai rifiuti. Tre aree boscate protette,
poste a breve distanza (1-9 km), costituiscono altrettanti nuclei per
le coppie di nibbio bruno nidificanti.

I rilievi, sono stati condotti a intervalli di 10 giorni, nel periodo
post-riproduttivo (dalla terza decade di giugno alla fine di agosto)
del 2012 e del 2013. Durante ogni se\allowbreak \allowbreak ssione sono stati effettuati
dei conteggi a tempo, uno ogni 6', sui nibbi in
alimentazione nell{\textquoteright}area dei rifiuti, censendo
separatamente i giovani dell{\textquoteright}anno e gli altri individui
(accorpando adulti e immaturi dell{\textquoteright}anno precedente). La
durata di ogni sessione \`e stata di 4 ore (12:00-15:00; N = 40) nel
2012, e di 10 ore nel 2013 (08:00-17:00, N = 100). Inoltre, nel 2013,
sono stati censiti, ogni 30', anche gli individui
posati (in riposo) in un raggio di 100 m all{\textquoteright}esterno
dell{\textquoteright}area di lavorazione dei rifiuti.

I confronti statistici, sono stati effettuati tra le medie dei rilievi a
carico dei due gruppi (adulti e giovani) per sessione (sia tra quelli
in attivit\`a trofica che tra i posati del 2013) e tra i due anni
(nell{\textquoteright}intervallo 12-15), e valutati tramite test z
(confronti duplici) e ANOVA (confronti multipli). 

\section*{Risultati e discussione}

Durante l{\textquoteright}estate del 2013, sono stati registrati
complessivamente 5134 contatti riguardanti individui in attivit\`a
trofica nell{\textquoteright}area dei rifiuti, di cui 623 (8,2 \%)
erano giovani dell{\textquoteright}anno. Questi ultimi, rilevati dalla
fine giugno (0,3 {\textpm} 1,0 DS; N = 100), sono aumentati
progressivamente fino a un massimo registrato nella seconda decade di
agosto (3,3 {\textpm} 1,0 DS; N = 100; ANOVA: F = 51,7; P {\textless}
0,05), quando rappresentavano il 63,9 \% degli individui in attivit\`a
trofica nella discarica. I restanti individui (adulti e immaturi nati
l{\textquoteright}anno precedente), partendo dal valore pi\`u elevato
rilevato all{\textquoteright}inizio dello studio (11,7 {\textpm} 7,9
DS, N = 100), hanno raggiunto quello pi\`u basso nella terza decade di
agosto (0,8 {\textpm} 1,1 DS; N = 100; ANOVA: F = 23.1, P {\textless}
0,05). Anche tra gli individui a riposo (n = 887), la presenza dei
giovani \`e aumentata dalla fine di luglio (0,8 {\textpm} 1,7 DS; N =
20) fino a raggiungere il valore pi\`u alto nella seconda decade di
agosto (10,2 {\textpm} 8,1 DS; N = 20. Fig. \ref{DeGiacomo_fig_1}), costituendo il 59,3 \%
dei nibbi censiti. Le differenze tra le decadi sono risultate
significative (ANOVA: F = 67.8, P {\textless} 0,001). Nel 2013, la
presenza dei giovani rilevati nelle ore centrali della giornata (pari a
1,0 {\textpm} 1,7 DS; N = 20) non \`e risultata diversa da quella
osservata nel 2012 (pari a 0.6 {\textpm} 0,8 DS, N = 20; z = 0,2; g.l.
= 558; n.s.).

Nei rapaci \`e nota la migrazione ritardata da parte dei giovani
(Kjell\'en 1992). In particolare nel nibbio bruno, si considera che gli
adulti lascino i quartieri riproduttivi 3-4 settimane prima dei giovani
(Newton 1979) e che questi costituiscano in agosto il 60 \% dei
migranti (Panuccio 2005).

Inoltre, prima della migrazione, aumenta la tendenza al gregarismo
creando una maggiore concentrazione di individui nelle zone di
{\textquotedblleft}\textit{stopover}{\textquotedblright} della
migrazione, tra le quali figurano per questa specie le discariche di
rifiuti (Panuccio \textit{et al.} 2005). Nel caso di Malagrotta, le
osservazioni effettuate all{\textquoteright}interno della discarica
indicano che i giovani dell{\textquoteright}anno tendono a raggiungere
precocemente quest{\textquoteright}area trofica dopo
l{\textquoteright}involo e che aumentano la loro consistenza
progressivamente sino alla met\`a del mese di agosto, quando
rappresentano la maggior parte della popolazione di nibbi presente nel
sito.

\section*{Conclusioni}

I risultati ottenuti confermano quindi che, presso questa discarica, i
giovani nibbi tendano a trattenersi nell{\textquoteright}area e a
partire in ritardo rispetto alla gran parte degli adulti, probabilmente
per accumulare le riserve di energia necessarie a permettere loro di
superare la migrazione, al pari di quanto accade per altre specie
(Baker 1978). L{\textquoteright}applicazione della 1999/31/CE, che
prevede la chiusura delle discariche di rifiuti solidi urbani non
trattati in Europa, avr\`a ripercussioni negative sulla specie
(Peronace \textit{et al.} 2012) e in particolare su questa fascia
d{\textquoteright}et\`a, che \`e anche quella maggiormente soggetta
alla pressione selettiva (Sergio \textit{et al.} 2011).

\begin{figure}[!h]
\centering
\includegraphics[width=.8\columnwidth]{DeGiacomo_fig_1.png}
\caption{Numero medio $\pm$ E.S. (errore standard) di giovani dell{\textquoteright}anno vs. altre classi di et\`a (adulti e subadulti) del nibbio bruno in attivit\`a trofica e in riposo, rilevato per decadi dal 20 luglio al 30 agosto 2013 nella discarica di Malagrotta (RM). Juv/at = giovani in attivit\`a trofica; A+S/at = tutti gli altri individui (adulti e subadulti) in attivit\`a trofica; Juv/rp = giovani in riposo; A+S/rp = altri individui (adulti e subadulti) in riposo}
\label{DeGiacomo_fig_1}
\end{figure}

\section*{Bibliografia}
\begin{itemize}\itemsep0pt
	\item Baker R.R., 1978 - \textit{The Evolutionary Ecology of Animal
Migration}. Holmes \& Meier Publishers, New York, 1012 pp.

	\item Guerrieri G. \& De Giacomo U., 2012 - Nibbio bruno \textit{Milvus
migrans}. In: Aradis A., Sarrocco S. \& Brunelli M. Analisi dello
status e della distribuzione dei rapaci diurni nidificanti nel Lazio.
\textit{Quaderni Natura e Biodiversit\`a} 2/2012 ISPRA: 23-29.

	\item Kjell\'en N.,1992 - Differential timing of autumn migration between sex
and age groups in raptors at Faltserbo, Sweden. \textit{Ornis
scandinavica,} 23: 420-434.

	\item Newton J., 1979 - \textit{Population ecology of Raptors}. T.\& A.D.
Poyser, London, 399 pp.

	\item Panuccio M., 2005 - Dati sulla presenza del Nibbio bruno \textit{Milvus
migrans} in due discariche di rifiuti urbani. \textit{Alula,} 12/1-2:
189-192.

	\item Peronace V., Cecere J.C., Gustin M. \& Rondinini C., 2011 - Lista Rossa
2011 degli Uccelli Nidificanti in Italia. \textit{Avocetta,} 36: 11-58.

	\item Sergio F., Tavecchia G., Blas G., L\'opez L., Tanferna A. \& Hiraldo F.,
2011 - Variation in age-structured vital rates of a long-lived raptor:
Implications for population growth. \textit{Basic and Applied Ecology},
12: 107-115.
\end{itemize}

\setcounter{figure}{0}
\setcounter{table}{0}

\begin{adjustwidth}{-3.5cm}{0cm}
\pagestyle{CIOpage}
\authortoc{\textsc{Florit F.}, \textsc{Rassati G.}}
\chapter*[]{\textbf{Il monitoraggio del re di quaglie }\textbf{\textit{Crex crex}}\textbf{ in Friuli Venezia Giulia (anni 2000-2012)}}
\addcontentsline{toc}{chapter}{Il monitoraggio del re di quaglie in Friuli Venezia Giulia}

\textsc{Fabrizio Florit}$^{1*}$, \textsc{Gianluca Rassati}$^{2**}$ \\

\index{Florit Fabrizio} \index{Rassati Gianluca}
\noindent\color{MUSEBLUE}\rule{27cm}{2pt}
\vspace{1cm}
\end{adjustwidth}

\marginnote{\raggedright $^1$Regione autonoma Friuli Venezia Giulia - Servizio caccia, risorse ittiche e biodiversità - Ufficio studi faunistici Via Sabbadini, 31 - 33100 Udine \\
$^2$Regione autonoma Friuli Venezia Giulia - Ispettorato Agricoltura e Foreste di Tolmezzo Via San Giovanni Bosco, 8 - 33028 Tolmezzo \\
\vspace{.5cm}
{\emph{\small $^*$Autore per la corrispondenza: \href{mailto:fabrizio.florit@regione.fvg.it}{fa\allowbreak bri\allowbreak zio.\allowbreak flo\allowbreak rit@\allowbreak re\allowbreak gio\allowbreak ne.\allowbreak fvg.\allowbreak it}}} \\
{\emph{\small $^{**}$Autore per la corrispondenza: \href{mailto:gianluca.rassati@regione.fvg.it}{gian\allowbreak lu\allowbreak ca.\allowbreak ras\allowbreak sa\allowbreak ti@\allowbreak re\allowbreak gio\allowbreak ne.\allowbreak fvg.\allowbreak it}}} \\
\keywords{\textit{Crex crex}, monitoraggio, Friuli Venezia Giulia}
{\textit{Crex crex}, monitoring, Friuli Venezia Giulia}
} 
{\small
\noindent \textsc{\color{MUSEBLUE} Summary} / Autonomous Region of Friuli Venezia Giulia has been coordinating annual surveys of Corncrake since 2000. The data collected between 2000 and 2012 by Regional Forest Service rangers and Office for fauna studies are presented.
Number of singing males is analyzed in relation to consistence and trend of regional population which is the most
important within Italian breeding range. Population trend over last thirteen years, although with annual fluctuations, is negative. The geocoded information on the distribution of the species is the basis for implementing specific
measures for the conservation of grassland habitats, mainly located in the pre-alpine and alpine areas of Friuli
Venezia Giulia.\\
}
\section*{Introduzione}
La Lista Rossa 2011 degli Uccelli Nidificanti in Italia classifica il re di quaglie \textit{Crex crex} vulnerabile (VU) secondo il criterio D1 - popolazione minore di 1000 individui maturi (Peronace \textit{et al}. 2012). Nel contesto del ristretto areale di nidificazione italiano il Friuli Venezia Giulia ospita la più consistente popolazione di questa specie di notevole interesse conservazionistico. 


\section*{Metodi}
Un programma di monitoraggio del re di quaglie coordinato dalla Regione autonoma Friuli Venezia Giulia è stato avviato nel 2000 e si è avvalso del personale del Corpo forestale regionale (CFR) e dell{\textquoteright}Ufficio studi faunistici (Gottardo \textit{et al}. 2001 e 2003; Florit \& Rassati 2005, 2009, 2010, 2013 e 2014).

I dati sono stati archiviati in una banca dati georeferita utile per estrarre informazioni dettagliate sulla presenza,
\ localizzazione e distribuzione della specie nel territorio regionale. La distribuzione in sintesi viene rappresentata in Unità di Rilevamento della Carta Tecnica Regionale Numerica (CTR, griglia di dimensioni pari a $3200$ x $2800$ m), e riguarda il numero di maschi cantori censiti in periodo riproduttivo da metà maggio a fine giugno durante indagini notturne.


\section*{Risultati e discussione}
Nella tabella (Tab. \ref{Florit_tab_1}) sono riportati i dati relativi al monitoraggio regionale del re di quaglie nel periodo 2000-2012. 
Negli ultimi tredici anni l{\textquoteright}andamento demografico, seppure con fluttuazioni annuali, appare negativo. Rispetto ai primi tre anni di indagine la consistenza della popolazione regionale evidenzia un netto calo. In particolare nell{\textquoteright}ultimo biennio, a fronte di uno sforzo di indagine superiore (155 e 148 Unità di Rilevamento indagate rispettivamente nel 2011 e 2012), la consistenza della popolazione appare paragonabile ai valori minimi registrati nel biennio 2005 e 2006 (122 e 79 UR indagate). 

Nel biennio 2011-2012 la distribuzione dei maschi cantori nel territorio regionale è circoscritta a tre nuclei
principali: Carnia, Prealpi Carniche e Prealpi Giulie (Fig. \ref{Florit_fig_1} e \ref{Florit_fig_2}). Alla diminuzione della popolazione è corrisposta una contrazione dell{\textquoteright}areale di nidificazione.

L{\textquoteright}apparente abbandono di siti storici di nidificazione indica la necessità di continuare il monitoraggio del re di quaglie, coordinando le indagini a livello locale anche con le regioni limitrofe al fine di condividere comuni
strategie di gestione degli habitat prativi, calibrate in funzione delle peculiarità delle diverse aree rurali
(Rassati 2004; Rassati \& Tout 2002).

\begin{table}[!h]
\centering
\footnotesize
\begin{tabular}{>{\raggedright\arraybackslash}p{.14\columnwidth}>{\raggedleft\arraybackslash}p{.05\columnwidth}>{\raggedleft\arraybackslash}p{.05\columnwidth}>{\raggedleft\arraybackslash}p{.05\columnwidth}>{\raggedleft\arraybackslash}p{.05\columnwidth}>{\raggedleft\arraybackslash}p{.05\columnwidth}>{\raggedleft\arraybackslash}p{.05\columnwidth}>{\raggedleft\arraybackslash}p{.05\columnwidth}>{\raggedleft\arraybackslash}p{.05\columnwidth}>{\raggedleft\arraybackslash}p{.05\columnwidth}>{\raggedleft\arraybackslash}p{.05\columnwidth}>{\raggedleft\arraybackslash}p{.05\columnwidth}>{\raggedleft\arraybackslash}p{.05\columnwidth}>{\raggedleft\arraybackslash}p{.05\columnwidth}}
\toprule
	& \textbf{2000} & \textbf{2001} & \textbf{2002} & \textbf{2003} & \textbf{2004} & \textbf{2005} & \textbf{2006} & \textbf{2007} & \textbf{2008} & \textbf{2009} & \textbf{2010} & \textbf{2011} & \textbf{2012}\\
	\toprule
	\textbf{Tot n. \male\male } & 325 & 199 & 205 & 157 & 146 & 91 & 76 & 133 & 115 & 96 & 120 & 87 & 76 \\
	\textbf{UR occupate} & 93 & 79 & 81 & 6 & 60 & 45 & 33 & 58 & 54 & 46 & 51 & 38 & 36 \\
	\textbf{UR indagate} & 177 & 198 & 202 & 202 & 169 & 122 & 79 & 147 & 153 & 167 & 155 & 155 & 148 \\
	\bottomrule
\end{tabular}
\caption{Risultati del monitoraggio del re di quaglie in Friuli Venezia Giulia negli anni 2000-2012 (n. \male\male: numero di maschi cantori; UR: Unità di Rilevamento, corrispondenti ad un elemento della griglia della CTR)}
\label{Florit_tab_1}
\end{table}

\begin{figure}[!h]
\centering
\includegraphics[width=.6\columnwidth]{Florit_fig_1.jpg}
\caption{Distribuzione del re di quaglie rilevata nel 2011 nelle celle indagate}
\label{Florit_fig_1}
\end{figure}

\begin{figure}[!h]
\centering
\includegraphics[width=.6\columnwidth]{Florit_fig_2.jpg}
\caption{Distribuzione del re di quaglie rilevata nel 2012 nelle celle indagate}
\label{Florit_fig_2}
\end{figure}

\newpage
\section*{Ringraziamenti}
Per la raccolta dei dati in campo ringraziamo il personale del Corpo forestale regionale, dei Parchi naturali regionali e dell{\textquoteright}Ufficio studi faunistici.

Un ringraziamento a Renato Castellani, Matteo De Luca, Bruno Dentesani, Luca Dorigo, Sergio Gollino, Roberto Parodi, Davide Pasut, Pierluigi Taiariol, Paolo Utmar e Valter Simonitti, per i dati inediti forniti. 

\section*{Bibliografia}
\begin{itemize}\itemsep0pt
	\item Florit, F. \& Rassati, G., 2005 - Il Re di quaglie \textit{Crex crex }in Friuli Venezia Giulia: 5 anni di monitoraggio (2000-2004). \textit{Avocetta}, 29: 110.
	\item Florit, F. \& Rassati, G., 2009 - Aggiornamento sull{\textquoteright}attività di monitoraggio del Re di quaglie \textit{Crex crex }promosso dalla Regione autonoma Friuli Venezia Giulia: anni 2007-2008. \textit{Alula}, 16 (1-2): 92-93.
	\item Florit F. \& Rassati G., 2010 - Corncrake (\textit{Crex crex}) monitoring in Friuli Venezia Giulia (North-eastern Italy). Abstracts Bird Numbers 2010 “Monitoring, indicators and targets” 18th Conference of the European Bird Census Council, Càceres, Spain, 22-26 March 2010.
	\item Florit, F. \& Rassati, G., 2013 - Il monitoraggio di una specie prioritaria per l{\textquoteright}Unione Europea promosso dalla Regione autonoma Friuli Venezia Giulia: il Re di quaglie \textit{Crex crex}. \textit{Rivista italiana di Ornitologia}, 82 (1-2): 177-179.
	\item Florit, F. \& Rassati, G., 2014 - Distribuzione del re di quaglie \textit{Crex crex} in Friuli Venezia Giulia in relazione alla rete regionale di aree naturali tutelate. Atti XVI Conv. It. Orn., Cervia (RA), 22-25 settembre 2011. Scritti, Studi e Ricerche di Storia Naturale della Repubblica di San Marino: 162-164.
	\item Gottardo, E., Luise, R., Zorzenon, T., Ota, D. \& Florit F., 2001 - Il censimento del Re di quaglie \textit{Crex crex} nel Friuli-Venezia Giulia nel 2000. \textit{Avocetta}, 25: 212.
	\item Gottardo, E., Luise, R., Zorzenon, T., Ota, D., Di Gallo, M., Facchin, G. \& Florit, F., 2003 - Il censimento del Re di quaglie \textit{Crex crex }in Friuli-Venezia Giulia negli anni 2001 e 2002\textit{. Avocetta}, 27: 111.
	\item Peronace, V., Cecere, J.C., Gustin, M. \& Rondinini, C. 2012 - Lista Rossa 2011 degli Uccelli Nidificanti in Italia. \textit{Avocetta}, 36(1): 11-58.
	\item Rassati G., 2004 - Evoluzione faunistica nelle aree rurali abbandonate. La presenza del Re di quaglie (\textit{Crex crex}) e della Lepre comune (\textit{Lepus europaeus}). \textit{Agribusiness Paesaggio \& Ambiente}, VII (1): 41-48.
	\item Rassati, G. \& Tout, C.P., 2002 - The Corncrake (\textit{Crex crex}) in Friuli-Venezia Giulia (North-eastern Italy). \textit{Avocetta}, 26 (1): 3-6.
\end{itemize}
\setcounter{figure}{0}
\setcounter{table}{0}

\begin{adjustwidth}{-3.5cm}{0cm}
\pagestyle{CIOpage}
\authortoc{\textsc{Fusari M.}, \textsc{Morganti N.}}
\chapter*[Migrazione dei rapaci nel Parco del Conero dal 2007 al 2013]{La migrazione primaverile dei rapaci nel Parco Regionale del
Conero: risultati delle osservazioni dal 2007 al 2013}
\addcontentsline{toc}{chapter}{Migrazione dei rapaci nel Parco del Conero dal 2007 al 2013}

\textsc{Maurizio Fusari}$^{1*}$, \textsc{Niki Morganti}$^{2**}$ \\

\index{Fusari Maurizio} \index{Morganti Niki}
\noindent\color{MUSEBLUE}\rule{27cm}{2pt}
\vspace{1cm}
\end{adjustwidth}


\marginnote{\raggedright $^1$Studio Faunistico Chiros, Macerata \\
$^2$Studio Naturalistico Diatomea, Senigallia (AN) \\
\vspace{.5cm}
{\emph{\small $^*$Autore per la corrispondenza: \href{mailto:chiros.studio@libero.it}{chiros.studio@libero.it}}} \\
{\emph{\small $^{**}$Autore per la corrispondenza: \href{mailto:info@studiodiatomea.it}{info@studiodiatomea.it}}} \\
\keywords{Conero, rapaci, migrazione}
{Conero, raptors, migration}
%\index{keywords}{Conero} \index{keywords}{Rapaci} \index{keywords}{Migrazione}
}
{\small
\noindent \textsc{\color{MUSEBLUE} Summary} / Conero mount is one of the most important site for pre-breeding migrator
raptors crossing the Adriatic sea to fly to Balkans and east-Europe.
Authors counted almost 47.000 raptors of 26 different species during
the years 2007-2013. The migration direction of 90\% of birds were
SW-NE, honey buzzard and marsh harrier were the most common spotted
species.
}



\section*{Introduzione}

Le prime osservazioni del fenomeno migratorio sul promontorio del Conero
svolte da Marco Borioni alla fine degli anni {\textquoteright}80 misero
subito in evidenza la rilevanza dell{\textquoteright}area per la
migrazione primaverile dei rapaci (Borioni 1997). Il Parco Regionale
del Conero, a partire dal 1999 ha avviato un monitoraggio
standardizzato per approfondire le conoscenze sulla migrazione
nell{\textquoteright}area protetta. Nel presente contributo si
riportano in sintesi i risultati riferiti al periodo 2007-2013. 
L{\textquoteright}area di studio ricade all{\textquoteright}interno del
Parco del Conero, situato lungo la costa al centro delle Marche. Il
monte Conero \`e il secondo promontorio roccioso dopo il Gargano che
gli uccelli incontrano lungo la loro rotta migratoria, risalendo il
litorale adriatico da sud. La morfologia dell{\textquoteright}area
presenta falesie calcaree a ridosso del mare e la zona interna con
colline e valli. Il Conero segna il confine settentrionale virtuale del
clima mediterraneo lungo l{\textquoteright}Adriatico, e la vegetazione
che lo caratterizza rispecchia questo fattore con presenza diffusa di
macchia mediterranea e boschi di leccio e roverella nei versanti
rivolti a sud e sud-est; mentre, nei versanti rivolti a nord, la
vegetazione \`e pi\`u mesofila, con ornielli e carpini. Oltre ai gi\`a
citati campi coltivati, sono presenti molti vigneti e numerosi
arbusteti dominati da ginestra; risultano ancora numerosi i boschi a
conifere (principalmente pino d{\textquoteright}Aleppo) derivanti dai
rimboschimenti degli anni {\textquoteright}30 del secolo scorso.

\section*{Metodi}

I rilevamenti sono stati svolti in localit\`a Gradina del Poggio, sito
che permette di avere un{\textquoteright}ampia visuale verso sud-ovest,
e quindi di controllare la direzione principale di provenienza dei
rapaci durante la migrazione preriproduttiva.

I campi di rilevamento sono stati svolti dal 15 aprile al 31 maggio
dalle ore 9 alle 19. Le osservazioni sono state annotate su schede
standard utili alla raccolta di dati relativi a: orario di
osservazione, specie, numero di individui, sesso, et\`a, direzioni di
provenienza e scomparsa e altezza di volo, oltre alle informazioni
relative alle condizioni meteorologiche e sulla visibilit\`a.

\section*{Risultati e discussione}

Sono state censite 26 specie in migrazione, per un totale di quasi
47.000 rapaci contati (Tab.1). Il falco pecchiaiolo \textit{Pernis
apivorus}, ad eccezione del 2007, \`e sempre stata la specie pi\`u
comune seguita dal falco di palude \textit{Circus aeruginosus}: questi
due rapaci negli anni hanno costituito tra il 68\% e il 91\% del totale
delle osservazioni. Nel periodo considerato dopo un picco registrato
negli anni 2008 e 2009, si \`e verificato un calo delle osservazioni
che hanno toccato il minimo nel 2012 (60\% in meno rispetto al 2009).
L{\textquoteright}andamento giornaliero del flusso migratorio evidenzia
il massimo passaggio nel periodo compreso tra la fine di aprile e la
prima met\`a di maggio, periodo coincidente con il picco migratorio del
falco pecchiaiolo. Nel presente lavoro sono state analizzate le
variazioni nella migrazione di 5 specie per cui l{\textquoteright}area
del Conero sembra rivestire particolare importanza; falco pecchiaiolo,
falco di palude, albanella pallida \textit{Circus macrorus}, albanella
minore \textit{Circus pygargus} e falco cuculo \textit{Falco
vespertinus}. L{\textquoteright}andamento negli anni del flusso
migratorio del falco pecchiaiolo rispecchia quanto gi\`a evidenziato
per il totale dei rapaci, con un picco tra il 2008 e il 2010 e un calo
negli anni successivi, mentre il falco di palude mostra un andamento
molto pi\`u costante. Il falco pecchiaiolo, inoltre, evidenzia un trend
nella mediana del passaggio anticipata anno dopo anno. Le altre due
specie di \textit{Circus} mostrano un andamento quasi coincidente con
un maggior numero di individui registrato negli anni 2007, 2011 e 2013.
Il falco cuculo, dopo una vera e propria
{\textquotedblleft}invasione{\textquotedblright} registrata nel 2008 e
una quasi {\textquotedblleft}scomparsa{\textquotedblright}
nell{\textquoteright}anno successivo, ha mantenuto nei successivi anni
un andamento piuttosto costante. Dall{\textquoteright}analisi oraria
delle osservazioni, si nota come la migrazione abbia due picchi, uno
nella tarda mattinata (il falco pecchiaiolo attraversa
l{\textquoteright}area prevalentemente al mattino) e uno nella seconda
met\`a del pomeriggio (il falco di palude sorvola il Conero
principalmente di pomeriggio). L{\textquoteright}analisi della
direzione di provenienza e di scomparsa dei rapaci, evidenzia come la
migrazione abbia un netto asse preferenziale SW-NE (circa il 90\% dei
rapaci).

Il Conero si conferma un{\textquoteright}area fondamentale soprattutto
per i rapaci che intendono raggiungere la Penisola balcanica e
l{\textquoteright}Est Europa e che utilizzano il promontorio come
{\textquotedblleft}trampolino di lancio{\textquotedblright} prima del
difficile attraversamento dell{\textquoteright}Adriatico.
\newcolumntype{S}{>{\raggedleft\arraybackslash}p{.074\columnwidth}}
\begin{longtable}{>{\raggedright\arraybackslash}p{.28\columnwidth}SSSSSSS}
\toprule
\textbf{Specie} & \textbf{2007} & \textbf{2008} & \textbf{2009} & \textbf{2010} & \textbf{2011} & \textbf{2012} & \textbf{2013} \\
\toprule
\endfirsthead
\multicolumn{8}{l}{\textit{\footnotesize Continua dalla pagina precedente}} \\
\toprule
\textbf{Specie} & \textbf{2007} & \textbf{2008} & \textbf{2009} & \textbf{2010} & \textbf{2011} & \textbf{2012} & \textbf{2013} \\
\toprule
\endhead
 Falco pecchiaiolo & 1765 & 5381 & 6546 & 4644 & 2734 & 1711 & 2366 \\
 Nibbio bruno & 59 & 50 & 44 & 29 & 40 & 13 & 47 \\
 Nibbio reale & 7 & 11 & 3 & 7 & 6 & 5 & 5 \\
 Biancone & 4 & 11 & 5 & 2 & 11 & 6 & 3 \\
 Falco di palude & 1896 & 1960 & 2347 & 1569 & 1716 & 1025 & 1621 \\
 Albanella reale & 8 & 6 & 2 & 6 & 4 & 5 & 15 \\
 Albanella pallida & 19 & 8 & 9 & 9 & 34 & 3 & 29 \\
 Albanella minore & 259 & 97 & 135 & 112 & 279 & 73 & 224 \\
 Albanella ind. & 27 & 28 & 10 & 8 & 9 & 23 & 53 \\
 Sparviere & 60 & 22 & 30 & 24 & 22 & 54 & 108 \\
 Sparviere levantino & 1 & 0 & 0 & 0 & 0 & 0 & 0 \\
 Poiana & 225 & 80 & 80 & 93 & 143 & 121 & 460 \\
 Poiana delle steppe & 1 & 0 & 1 & 1 & 0 & 0 & 0 \\
 Poiana coda bianca & 1 & 0 & 1 & 1 & 0 & 1 & 1 \\
 Poiana calzata & 0 & 0 & 0 & 0 & 1 & 0 & 0 \\
 Aquila anatraia minore & 0 & 1 & 0 & 0 & 0 & 1 & 0 \\
 Aquila delle steppe & 0 & 1 & 0 & 0 & 0 & 0 & 0 \\
 Aquila imperiale & 0 & 0 & 0 & 1 & 0 & 0 & 0 \\
 Aquila minore & 1 & 1 & 1 & 1 & 1 & 1 & 2 \\
 Accipitridi ind. & 26 & 7 & 8 & 0 & 0 & 56 & 9 \\
 Falco pescatore & 23 & 16 & 25 & 21 & 15 & 12 & 17 \\
 Grillaio & 6 & 2 & 0 & 1 & 0 & 2 & 3 \\
 Gheppio & 186 & 114 & 239 & 319 & 195 & 119 & 171 \\
 Gheppio/Grillaio & 65 & 84 & 63 & 52 & 38 & 52 & 138 \\
 Falco cuculo & 238 & 1471 & 61 & 325 & 267 & 232 & 365 \\
 Smeriglio & 0 & 0 & 1 & 0 & 0 & 0 & 0 \\
 Lodolaio & 141 & 112 & 115 & 157 & 180 & 137 & 219 \\
 Falco della regina & 0 & 1 & 5 & 3 & 1 & 0 & 0 \\
 Sacro & 2 & 0 & 3 & 0 & 0 & 1 & 0 \\
 Pellegrino ssp. \textit{calidus} & 0 & 0 & 1 & 4 & 1 & 0 & 0 \\
 Falconidi ind. & 8 & 43 & 13 & 26 & 0 & 29 & 20 \\
 \toprule
 \hiderowcolors
 \textbf{ Totale } & \textbf{ 5028 } & \textbf{ 9507 } & \textbf{ 9748 } & \textbf{ 7415 } & \textbf{ 5697 } & \textbf{ 3682 } & \textbf{ 5876} \\
\bottomrule
\multicolumn{8}{l}{} \\
\caption{Numero rapaci censiti nel periodo 2007-2013}
\label{Fusari_tab_1}
\end{longtable}

Proprio in considerazione della rilevanza dell{\textquoteright}area
sarebbe auspicabile riuscire a implementare le conoscenze acquisite con
i monitoraggi che permettano di comprendere la percentuale di esemplari
che dal monte Conero si dirigono verso le coste croate e
l{\textquoteright}ampliamento del periodo di indagine al fine di avere
un quadro completo dell{\textquoteright}intero fenomeno migratorio.



\section*{Ringraziamenti}

Si ringrazia l{\textquoteright}Ente Parco Regionale del Conero, Marco
Borioni, Maria Rosa Baldoni, Mina Pascucci, Vittorio e nonno Franco.


\section*{Bibliografia}
\begin{itemize}\itemsep0pt
 \item Borioni M., 1997 -- \textit{Ali in un Parco}. Printem Edizioni, Ancona,
95 pp.
\end{itemize}


\setcounter{figure}{0}
\setcounter{table}{0}

\begin{adjustwidth}{-3.5cm}{0cm}
\pagestyle{CIOpage}
\authortoc{\textsc{Giacchini P.}}
\chapter*[Diversit\`a ornitica basso-montana di Fiordimonte (MC -
Marche)]{\bfseries
L{\textquoteright}importanza degli ambienti aperti basso-montani per la
biodiversit\`a ornitica nelle marche: il caso di Fiordimonte (MC)}
\addcontentsline{toc}{chapter}{Diversit\`a ornitica basso-montana di Fiordimonte (MC - Marche)}

\textsc{Paolo Giacchini}$^{1*}$ \\

\index{Giacchini Paolo}
\noindent\color{MUSEBLUE}\rule{27cm}{2pt}
\vspace{1cm}
\end{adjustwidth}



\marginnote{\raggedright $^1$Hystrix srl - via Castelfidardo 7 -- 61032 Fano (PU)  \\
\vspace{.5cm}
{\emph{\small $^*$Autore per la corrispondenza: \href{mailto:paolo.giacchini@hystrix.it}{pao\allowbreak lo.\allowbreak giac\allowbreak chi\allowbreak ni@\allowbreak hys\allowbreak trix.\allowbreak it}}} \\
\keywords{Marche, biodiversit\`a ornitica, comunit\`a nidificante}
{Marche region, bird population biodiversity, breeding
population}
}
{\small
\noindent \textsc{\color{MUSEBLUE} Summary} / This study represents an in-depth analysis of the bird population
biodiversity nesting in a low mountainous area of the Marche region.
The breeding population of \textit{Lanius collurio} showing a maximum
density of 0,65 territorial males/10 ha; \textit{Coturnix coturnix,
Lullula arborea, Alauda arvensis, Emberiza citrinella, Miliaria
calandra} have been considered of particular interest.
}



\section*{Introduzione}

Il presente lavoro rappresenta un approfondimento
dell{\textquoteright}analisi della biodiversit\`a ornitica nidificante
in un{\textquoteright}area basso-montana della regione Marche.

L{\textquoteright}area di studio riguarda un territorio di circa 460 ha
nei comuni di Fiastra e Fiordimonte (MC), al confine con il Parco
Nazionale dei monti Sibillini e con il SIC {\textquotedblleft}val di
Fibbia -- valle dell{\textquoteright}Acquasanta{\textquotedblright}. Si
tratta di una dorsale che sale da 850 m s.l.m. fino a circa 1400 m
s.l.m., caratterizzata da coltivi (orzo, erba medica, erbai polifiti)
nella parte basale, macchie boscate nei fossi e nelle pendenze meno
coltivabili, pascoli pi\`u o meno cespugliati nella parte intermedia
fino a quelli sommitali. Presenti ma in modo ridotto, gli affioramenti
rocciosi.
L{\textquoteright}area \`e compresa in parte in
un{\textquoteright}azienda agri-turistico-venatoria (Fiordimonte), con
all{\textquoteright}interno una zona addestramento cani (ZAC) di tipo
C. 

\section*{Metodi}

Le osservazioni sono state condotte nel periodo maggio-luglio 2007,
ripetute a maggio-giugno 2012 e 2013, con sessioni di ascolto al canto
e osservazione diretta da punti fissi e su transetti lineari. 

\section*{Risultati e discussione}

I rilevamenti hanno individuato, in particolare, la seguente comunit\`a
ornitica nidificante (Tab. \ref{Giacchini_tab_1}).

Tra gli elementi di maggior interesse spicca la popolazione di averla
piccola, ancora diffusa come nidificante nelle Marche, pur avendo
mostrato locali, evidenti, contrazioni numeriche. 

La consistenza rilevata nel 2007, gi\`a elevata, \`e aumentata nel 2012
giungendo a 30 maschi territoriali in aree aperte cespugliate con
\textit{Rosa canina, Juniperus communis, Prunus spinosa}, che usano in
prevalenza per la riproduzione. La densit\`a rilevata \`e di 0,50
maschi territoriali /10 ha nel 2007 e di 0,65 maschi/10 ha nel 2012;
considerando singole aree a maggior densit\`a, si arriva fino a valori
di 2,07 maschi/10 ha (19 maschi territoriali in un{\textquoteright}area
di 92 ha nel 2012). 

La distribuzione altitudinale dei territori va da 870 a 1330 m s.l.m. A
differenza di quanto rilevato in altre aree marchigiane (Morelli \&
Pandolfi 2011), non sembra esservi un rapporto privilegiato dei siti
riproduttivi con il reticolo stradale, n\'e con abitazioni rurali. 

Importante \`e anche la presenza di altre specie non comuni in ambito
regionale, tra cui lo zigolo giallo, con almeno 9 siti riproduttivi, e
le concentrazioni di quaglia, allodola, tottavilla e strillozzo. Per
quaglia e allodola si \`e riscontrata una certa contrazione nel 2012,
pur mostrando ancora una presenza diffusa.

Lo studio ha evidenziato la ricchezza ornitica di
quest{\textquoteright}area; nonostante la presenza di un istituto di
attivit\`a venatoria, ma anche per la vicinanza degli istituti di
protezione (in particolare il Parco Nazionale dei monti Sibillini), le
consistenze delle popolazioni di specie legate ad ambienti aperti
appaiono di estremo interesse, essendo non del tutto comuni ad altre
aree marchigiane simili, esterne ad aree protette. 

\begin{table}[!h]
\centering
\begin{tabular}{>{\raggedright\arraybackslash}p{.4\columnwidth}>{\raggedright\arraybackslash}p{.2\columnwidth}>{\raggedright\arraybackslash}p{.2\columnwidth}}
\toprule
\textbf{Specie} & \textbf{N. coppie 2007} & \textbf{N. coppie 2012} \\
\toprule
%\showrowcolors
Quaglia comune \textit{Coturnix coturnix} & circa 30 & circa 20 \\
Gheppio \textit{Falco tinnunculus} & 1 & ? \\
Tottavilla \textit{Lullula arborea} & >30 & circa 20 \\
Allodola \textit{Alauda arvensis} & >120 & circa 40 \\
Calandro \textit{Anthus campestris} & 4 & 2 \\
Codirossone \textit{Monticola saxatilis} & 1 & ? \\
Sterpazzola \textit{Sylvia communis} & 4 & 4 \\
Culbianco \textit{Oenanthe oenanthe} & 5 & 1 \\
Averla piccola \textit{Lanius collurio} & 23 & 30 \\
Zigolo giallo \textit{Emberiza citrinella} & 9 & 3 \\
Strillozzo \textit{Miliaria calandra} & 45-60 & 30-40 \\
\bottomrule
\hiderowcolors
\end{tabular}
\caption{Comunit\'a ornitica di maggior interesse nidificante nell{\textquoteright}area di studio}
\label{Giacchini_tab_1}
\end{table}

\section*{Ringraziamenti}
Si ringrazia Pietro Spadoni per la partecipazione ai rilevamenti.

\section*{Bibliografia}
\begin{itemize}\itemsep0pt
	\item Morelli F. \& Pandolfi M., 2011 - Breeding habitat and nesting site of
the red-backed shrike \textit{Lanius collurio} in farmland of the
Marche region, Italy. \textit{Avocetta}, 35: 43-49.
\end{itemize}

\setcounter{figure}{0}
\setcounter{table}{0}

\begin{adjustwidth}{-3.5cm}{0cm}
\pagestyle{CIOpage}
\authortoc{\textsc{Laddaga L.}, \textsc{Luoni F.},
\textsc{Martinoli A.}, \textsc{Soldato G.}}
\chapter*[Monitoraggio ornitologico del lago di Varese e della
palude Brabbia]{Lago di Varese e palude Brabbia: sintesi dei primi anni di monitoraggio del Gruppo locale di conservazione LIPU}
\addcontentsline{toc}{chapter}{Monitoraggio ornitologico del lago di Varese e della palude Brabbia}

\textsc{Lorenzo Laddaga}$^{1*}$, \textsc{Federica Luoni}$^{1**}$,
\textsc{Alessio Martinoli}$^{1}$, \textsc{Giovanni Soldato}$^{1}$ \\

\index{Laddaga Lorenzo} \index{Luoni Federica} \index{Martinoli Alessio} \index{Soldato Giovanni}
\noindent\color{MUSEBLUE}\rule{27cm}{2pt}
\vspace{1cm}
\end{adjustwidth}



\marginnote{\raggedright $^1$LIPU BirdLife Italia -- Via Udine 3/A, 43122 Parma  \\
\vspace{.5cm}
{\emph{\small $^*$Autore per la corrispondenza: \href{mailto:l.laddaga@libero.it}{l.\allowbreak lad\allowbreak da\allowbreak ga@\allowbreak li\allowbreak be\allowbreak ro.\allowbreak it}}} \\
{\emph{\small $^**$Autore per la corrispondenza: \href{mailto:federica.luoni@lipu.it}{fe\allowbreak de\allowbreak ri\allowbreak ca.\allowbreak lu\allowbreak oni@\allowbreak li\allowbreak pu.\allowbreak it}}} \\
\keywords{GLC, palude Brabbia, lago di Varese, censimenti avifauna}
{LCG, Brabbia Marsh, Varese lake, bird census}
%\index{keywords}{GLC} \index{keywords}{Palude Brabbia} \index{keywords}{Lago di Varese} \index{keywords}{Censimenti avifauna}
}
{\small
\noindent \textsc{\color{MUSEBLUE} Summary} / Since 2009 a group of LIPU volunteers (LCG) undertakes bird surveys in
the Brabbia marsh Natural Reserve and Varese lake. In Brabbia marsh
were contacted 162 species; in particular, 20 species of wintering
waterbirds, and 19 nesting were found. On lake Varese the number of
aquatic species surveyed is 29 in the winter and 19 in spring.
}



\section*{Introduzione}

Dal 2009 la LIPU ha dato via al progetto GLC (Gruppi Locali di
Conservazione), al fine di monitorare sia il livello di conservazione,
che l{\textquoteright}andamento delle popolazioni di uccelli
nidificanti e svernanti nelle IBA o nei siti Rete Natura 2000, tramite
l{\textquoteright}attivit\`a di gruppi coordinati di volontari. Presso
la Riserva Naturale Oasi LIPU palude Brabbia \`e operante un GLC che
svolge monitoraggi ornitologici nell{\textquoteright}area della riserva
e sull{\textquoteright}adiacente lago di Varese, anche grazie al
contributo dell{\textquoteright}Unione Europea
nell{\textquoteright}ambito del progetto LIFE10 NAT IT241 TIB
{\textquotedblleft}Trans Insubria Bionet{\textquotedblright}. 

\section*{Metodi}

I metodi si differenziano in funzione della stagione e
dell{\textquoteright}area: sul lago viene svolto mensilmente, da marzo
a ottobre, un transetto in barca, mentre, nei mesi invernali i
monitoraggi vengono effettuati da punti fissi da riva. In palude
Brabbia vengono svolti transetti lineari per il censimento delle specie
legate agli ambienti boschivi e osservazioni da terra per la fauna
acquatica, oltre a due censimenti da barca nei mesi di aprile e maggio.
Per alcune specie di particolare interesse conservazionistico vengono
svolti censimenti \textit{ad hoc} durante la stagione riproduttiva. Per
ognuna delle due aree viene poi stilato un elenco mensile complessivo
aggiungendo anche osservazioni non standardizzate. 

\section*{Risultati e discussione}

In totale in palude Brabbia sono state contattate 162 specie; in
particolare sono state rilevate 20 specie di acquatici svernanti, e 19
nidificanti. La specie pi\`u abbondante, a parte il cormorano
\textit{Phalacrocorax carbo} presente con un \textit{roost} invernale
di oltre 1000 individui, \`e risultata essere
l{\textquoteright}alzavola \textit{Anas crecca} che conta contingenti
di quasi 100 individui nel periodo invernale, seguita dal germano reale
\textit{Anas platyrhynchos} che raggiunge anch{\textquoteright}esso la
massima concentrazione in inverno e che nidifica con circa una decina
di coppie. 

Per il lago di Varese il numero di specie acquatiche contattate \`e di
29 nel periodo invernale, tra le quali risulta essere dominante il
gabbiano comune \textit{Chroicocephalus ridibundus} presente con un
contingente di circa 2500 individui. Sono 19, invece, le specie
acquatiche nidificanti, tra cui svasso maggiore \textit{Podiceps
cristatus} e folaga \textit{Fulica atra} risultano essere le pi\`u
comuni.

Nell{\textquoteright}area della Riserva \`e presente anche una garzaia
di circa 100 nidi occupati al 70\% da airone cenerino \textit{Ardea
cinerea} e per la restante parte da nitticora \textit{Nycticorax
nycticorax}. L{\textquoteright}airone rosso \`e presente, invece, con 5
coppie nell{\textquoteright}area della palude Brabbia, mentre sulla
sponda nord del lago di Varese dal 2011 si \`e, inoltre, insediata una
garzaia di airone rosso e cenerino composta rispettivamente da 5 e 7
coppie. La popolazione nidificante di tarabusino \textit{Ixobrychus
minutus} \`e risultata costante durante gli anni di censimento (8
coppie sul lago di Varese e 1-2 coppie in palude Brabbia), ma in netto
calo rispetto a quanto riportato dall{\textquoteright}atlante dei
nidificanti della provincia di Varese che stima circa 20 coppie
nidificanti in palude Brabbia fino alla fine degli anni
{\textquoteright}90 (Gagliardi \textit{et al}. 2007). 

\`E stata accertata la nidificazione di una coppia di moretta tabaccata
\textit{Aythya nyroca} nella Riserva per tutti gli anni di osservazione
e, nel corso della stagione 2012, di 4 coppie sul lago di Varese, dove
\`e presente un piccolo nucleo stanziale di 8 individui che sono stati
contattati negli anni anche durante i censimenti invernali. 

Il falco di palude \textit{Circus aeruginosus} \`e presente sia
nell{\textquoteright}area delle Riserva che sul lago di Varese nel
periodo primaverile ed estivo durante tutti gli anni di monitoraggio
con un numero di individui da 3 a 9 (aprile 2011), in palude Brabbia la
specie \`e stata osservata anche durante i mesi invernali. Durante i
mesi estivi \`e stato osservato dal 2010, almeno un giovane, il che fa
supporre una probabile nidificazione della specie
all{\textquoteright}interno dell{\textquoteright}IBA, anche se non \`e
stato ritrovato il sito di nidificazione. Questa risulterebbe la prima
nidificazione della specie in provincia di Varese, essa infatti non \`e
presente negli atlanti dei nidificanti 2003-2005 (Gagliardi \textit{et
al}. 2007) e 1983-1987 (Guenzani \& Saporetti 1988).

\section*{Ringraziamenti}

Gli autori ringraziano in primo luogo i volontari del GLC lago di Varese
e palude Brabbia.

Un ringraziamento anche a tutti coloro che hanno permesso la
realizzazione di questi monitoraggi e supervisionato i lavori svolti.
Grazie quindi a: Barbara Ravasio, Claudio Celada, Giorgia Gaibani,
Massimo Soldarini. 

Un grazie anche al personale di provincia di Varese e ai partner del
progetto LIFE TIB Fondazione Cariplo e Regione Lombardia.

\section*{Bibliografia}
\begin{itemize}\itemsep0pt
	\item Gagliardi A., Guenzani W., Pratoni D.G., Saporetti F. \& Tosi G. (a cura
di), 2007 {}-- \textit{Atlante Ornitologico Georeferenziato della
provincia di Varese. Uccelli nidificanti 2003-2005}. Provincia di
Varese; Civico Museo Insubrico di Storia Naturale di Induno Olona;
Universit\`a degli Studi dell{\textquoteright}Insubria, sede di Varese,
295 pp.

	\item Guenzani W. \& Saporetti F., 1988 {}- \textit{Atlante degli uccelli
nidificanti in provincia di Varese}. Edizioni Lativia.
\end{itemize}


\setcounter{figure}{0}
\setcounter{table}{0}

\begin{adjustwidth}{-3.5cm}{0cm}
\pagestyle{CIOpage}
\authortoc{\textsc{La Mantia T.}, \textsc{Massa B.},
\textsc{Pipitone S.}, \textsc{R\"uhl J.}}
\chapter*[Gli uccelli come vettori di dispersione dei semi]{\bfseries
Il ruolo degli uccelli come vettori di dispersione durante le
successioni secondarie}
\addcontentsline{toc}{chapter}{Gli uccelli come vettori di dispersione dei semi}

\textsc{Tommaso La Mantia}$^{1*}$, \textsc{Bruno Massa}$^1$,
\textsc{Sergio Pipitone}$^1$, \textsc{Juliane R\"uhl}$^1$ \\

\index{La Mantia Tommaso} \index{Massa Bruno} \index{Pipitone Sergio} \index{R\"uhl Juliane}
\noindent\color{MUSEBLUE}\rule{27cm}{2pt}
\vspace{1cm}
\end{adjustwidth}



\marginnote{\raggedright $^1$Dipartimento SAF - Viale delle Scienze, Ed. 4, Ingresso H, 90128
Palermo \\
\vspace{.5cm}
{\emph{\small $^*$Autore per la corrispondenza: \href{mailto:tommaso.lamantia@unipa.it}{tom\allowbreak ma\allowbreak so.\allowbreak la\allowbreak man\allowbreak ti\allowbreak a@\allowbreak uni\allowbreak pa.\allowbreak it}}} \\
\keywords{Pantelleria, Uccelli, semi, specie arbustive}
{Pantelleria, Birds, secondary
succession, seed dispersion}
%\index{keywords}{Pantelleria} \index{keywords}{Semi} \index{keywords}{Specie arbustive}
}
{\small
\noindent \textsc{\color{MUSEBLUE} Summary} / Birds play an important role in the spread of seeds of shrubs and trees.
The abandonment processes that characterized the European countries and
in particular the countries of the Mediterranean have led to rapid
processes of colonization by vegetation. The speed with which these
processes occur depends in an essential way from the success of woody
plants to colonize the abandoned areas and is influenced in an
important way by birds. There are few quantitative data on the role
played by birds. We started a study on the role played by birds in
secondary successions in the island of Pantelleria, island affected by
heavy phenomena of abandonment. The test was conducted by placing
perches structured so you can collect the feces of birds. Preliminary
results indicate that: 1) the amount of seeds of woody species
dispersed by birds decreases with increasing distance of the mother
plants; 2) dispersion of seeds by birds is the most important type for
the colonization of abandonment field: 3) rats and rabbits also play an
important role in the dispersion of the woody species, but only in the
first few tens of meters from the mother plants.\\
}

\section*{Introduzione}


Nel XX secolo, l{\textquoteright}Europa \`e stata caratterizzata da
forti processi di abbandono dell{\textquoteright}agricoltura. Nelle
aree agricole abbandonate non soggette a disturbi (pascolo, incendio),
le dinamiche della successione secondaria (cio\`e il processo di
ricostituzione della vegetazione dopo che la copertura vegetale
preesistente \`e stata distrutta da un disturbo) hanno portato alla
formazione di comunit\`a vegetali pre-forestali e forestali. La
velocit\`a con cui avvengono questi processi dipende in maniera
essenziale dal successo delle piante legnose a colonizzare le aree
abbandonate e viene influenzato in maniera importante dagli uccelli
(R\"uhl \& Schnittler 2011). Per valutare il ruolo degli uccelli \`e
stato avviato uno studio in Sicilia, a Pantelleria, isola interessata
da forti fenomeni di abbandono (La Mantia et al., 2008), attraverso la
collocazione di posatoi per testare due ipotesi: 1) se la quantit\`a di
semi delle specie legnose della macchia dispersi dagli uccelli
diminuisce con l{\textquoteright}aumento della distanza della macchia;
2) se la dispersione ornitocora \`e pi\`u importante della dispersione
non-ornitocora per la colonizzazione degli ex-coltivi. 

\section*{Metodi}

Nel febbraio 2011 sono state posizionate 7 repliche di coppie
{\textquotedblleft}posatoi{\textquotedblright} e
{\textquotedblleft}vaschetta a terra{\textquotedblright} in giovani
ex-coltivi (vigneti-cappereti). In ogni replica, una coppia di posatoi
e vaschette (distanza tra di loro 10 m) sono stati messi,
rispettivamente, a 30, 60 e 120 m dal limite tra
l{\textquoteright}ex-coltivo e un{\textquoteright}area di macchia
mediterranea. Tra novembre 2011 e marzo 2012 sono stati effettuati
quattro controlli nei posatoi e nelle vaschette e quindi raccolti i
semi contenuti al loro interno che sono stati suddivisi in tre grandi
gruppi: 1) arbusti della macchia (= \textit{Pistacia lentiscus},
\textit{Arbutus unedo}, \textit{Phillyrea latifolia}, \textit{Myrtus
communis}, \textit{Daphne gnidium}, \textit{Teline monspessulana}); 2)
piante un tempo coltivate (\textit{Vitis vinifera}, \textit{Capparis}
\textit{spinosa}); 3) altre piante legnose (\textit{Rubus ulmifolius},
\textit{Rubia peregrina}, \textit{Prasium majus}). Sono stati raccolti
in totale 2272 semi di specie legnose, la maggior parte tra novembre e
dicembre. Nelle vaschette sono stati trovati feci di ratti e conigli, e
all{\textquoteright}interno di queste erano presenti semi. Con
riferimento agli arbusti della macchia, sono stati trovati molto pi\`u
semi nei posatoi che nelle vaschette (ca. il doppio nelle distanze di
30 e 90 m, e ca. 10 volte di pi\`u a 60 m). Inoltre, solo nel caso dei
posatoi vi \`e una diminuzione continua del numero dei semi con
distanza crescente dalla macchia, mentre nel caso delle vaschette il
numero dei semi a 30 m era uguale a quello a 90 m, e nel caso delle
feci trovate dentro le vaschette, il numero dei semi era pi\`u alto a
30 m, basso a 60 m ed intermedio a 90 m. Per le piante un tempo
coltivate, non sono state trovate delle differenze tra il numero di
semi nei posatoi e nelle vaschette a 30 e 60 m, mentre a 90 m sono
stati trovati molto pi\`u semi nelle vaschette che nei posatoi.
Inoltre, sono stati quasi assenti i semi di vite e cappero nelle feci
trovate all{\textquoteright}interno delle vaschette. I semi delle
{\textquotedblleft}altre legnose{\textquotedblright} nelle vaschette
aumentano in quantit\`a con l{\textquoteright}incremento della distanza
della macchia, mentre quelli nelle feci nelle vaschette diminuiscono
lungo il \textit{transect}. In termini assoluti, solo a 60 m sono stati
trovati pi\`u semi nei posatoi che nelle vaschette. 

\section*{Conclusioni}

I risultati del primo anno di raccolta indicano che: 1) la quantit\`a di
semi delle specie legnose della macchia dispersi dagli uccelli negli
ex-coltivi diminuisce con l{\textquoteright}aumento della distanza
delle piante madri; 2) la dispersione ornitocora \`e la tipologia pi\`u
importante per la colonizzazione degli ex-coltivi da parte delle specie
della macchia; 3) anche i ratti e i conigli svolgono un ruolo
importante di dispersione delle specie legnose della macchia, ma solo
nelle prime decine di metri dalla macchia.

\section*{Ringraziamenti}

Ricerca condotta nell{\textquoteright}ambito del progetto MIUR-PRIN
{\textquotedblleft}Strategie nazionali per la mitigazione dei
cambiamenti climatici in sistemi arborei agrari e forestali
(CARBOTREES){\textquotedblright}.

\section*{Bibliografia}
\begin{itemize}\itemsep0pt
	\item La Mantia T., R\"uhl J., Pasta S., Campisi D. \& Terrazzino G., 2008 -
Structural analysis of woody species in Mediterranean old fields.
\textit{Plant Biosystems}, 142 (3): 462-471.

	\item R\"uhl J. \& Schnittler M., 2011 - An empirical test of neighbourhood
effect and safe-site effect in abandoned mediterranean vineyards.
\textit{Acta Oecol}., 37: 71-78.
\end{itemize}

\setcounter{figure}{0}
\setcounter{table}{0}

\begin{adjustwidth}{-3.5cm}{0cm}
\pagestyle{CIOpage}
\authortoc{\textsc{Liuzzi C.}, \textsc{Mastropasqua F.},
\textsc{Todisco S.}}
\chapter*[La collezione ornitologica di Vincenzo de Romita]{La collezione ornitologica di Vincenzo de Romita}
\addcontentsline{toc}{chapter}{La collezione ornitologica di Vincenzo de Romita}

\textsc{Cristiano Liuzzi}$^{1*}$, \textsc{Fabio Mastropasqua}$^1$,
\textsc{Simone Todisco}$^1$ \\

\index{Liuzzi Cristiano} \index{Mastropasqua Fabio} \index{Todisco Simone}
\noindent\color{MUSEBLUE}\rule{27cm}{2pt}
\vspace{1cm}
\end{adjustwidth}



\marginnote{\raggedright $^1$Associazione Centro Studi de Romita, C. da Tavarello
n. 362/A, Monopoli (BA) \\
\vspace{.5cm}
{\emph{\small $^*$Autore per la corrispondenza: \href{mailto:c.liuzzi@wwfoasi.it}{c.\allowbreak li\allowbreak uz\allowbreak zi@\allowbreak wwf\allowbreak oa\allowbreak si.\allowbreak it}}} \\
\keywords{Bari, collezione ornitologica, de Romita}
{Bari, ornithological collection, de Romita}
%\index{keywords}{Bari} \index{keywords}{Collezione ornitologica} \index{keywords}{De Romita}
}
{\small
\noindent \textsc{\color{MUSEBLUE} Summary} / We report the results of a scientific review of V. de Romita
ornithological collection. The collection includes 994 samples
belonging to 318 different species, with an average of 3.1 specimens
per species, collected mainly between 1871 and 1893 and coming mostly
from the province of Bari (79\%). All collected data were published in
the book {\textquotedbl}Avifauna Puglia ... 130 years
later{\textquotedbl}.\\
}




\section*{Introduzione}
Si riportano i risultati del progetto {\textquotedblleft}Museo de
Romita{\textquotedblright} (finanziato dalla Regione Puglia), che ha
previsto la revisione della collezione del Prof. de Romita, custodita
presso l{\textquoteright}IISS Vivante-Pitagora nel centro di Bari. Il
Museo de Romita nasce per ospitare i reperti collezionati dal de Romita
e da lui stesso donati al Regio istituto tecnico e nautico di Bari
(oggi IISS Vivante-Pitagora). Con la supervisione scientifica
dell{\textquoteright}ISPRA e la collaborazione con
l{\textquoteright}Associazione Or. Me., \`e stato possibile programmare
un piano d{\textquoteright}analisi dei singoli reperti. 

\section*{Metodi}

La revisione \`e stata realizzata in un anno (09/2011-08/2012) da tre
operatori; i reperti sono stati controllati secondo i numeri
progressivi dell'inventario, effettuando un controllo
incrociato tra il catalogo e quanto riportato sui cartellini apposti
nel tempo dai diversi curatori della collezione, confrontando il tutto
con gli scritti del de Romita (1884, 1889, 1899, 1900). I dati
dell'inventario e i nuovi acquisiti sono stati
informatizzati. I reperti, dopo essere stati classificati, sono stati
sottoposti a un intervento di pulizia e suddivisi in base allo stato di
conservazione in modo da esporre quelli integri e interessanti, e
conservare i reperti danneggiati, nella speranza che possano un giorno
essere riparati. Gli esemplari di difficile identificazione sono stati
analizzati dettagliatamente e sottoposti in un secondo tempo anche al
parere di esperti ornitologi.

\section*{Risultati e discussione}

Il Catalogo cos\`i ottenuto consta di 994 reperti, per un totale di 472
esemplari di non Passeriformi e 522 di Passeriformi, riferibili a 318
specie; tra queste, la pi\`u rappresentata \`e \textit{Motacilla
flava}, con 115 individui, mentre la media per specie \`e di 3,1. 

La collezione comprende esemplari catturati dal 1869 al 1940. Gli anni
pi\`u prolifici per la raccolta vanno dal 1871 al 1893, periodo durante
il quale sono stati acquisiti 486 reperti (min. 27 nel 1871, max. 100
nel 1891), pi\`u del 50\% del totale. Il 7,9\% dei reperti elencati nel
catalogo non \`e stato riscontrato durante la revisione; dei restanti,
il 6,4\% \`e conservato in pelle mentre l'85,7\% \`e
montato.

La maggior parte dei reperti \`e di origine pugliese (866; circa
l{\textquoteright}87\%); tra le altre regioni italiane le pi\`u
rappresentate sono: Piemonte (17), Friuli-Venezia Giulia (10) e
Lombardia (7). Solo 4 esemplari provengono dall{\textquoteright}estero.
Infine 77 reperti sono di origine ignota (cartellini assenti,
illeggibili ecc.). La provincia con il maggior numero di reperti \`e
Bari, con 781 esemplari (79\%).

Si segnalano tra le specie non riportate in precedenza dal de Romita: 1
ind. di \textit{Charadrius pecuarius} proveniente da Bari (1908):
sottoposta alla COI (se convalidata sarebbe la prima per
l{\textquoteright}Italia); 1 ind. di \textit{Phalaropus tricolor}
proveniente da Bari (24 ottobre 1889), sottoposto al parere della COI;
1 ind. di \textit{Turdus eunomus} proveniente da Bari (1901), prima
segnalazione regionale. Inoltre sono stati sottoposti a parere della
COI: 1 ind. di \textit{Sylvia hortensis} con alcune caratteristiche
riconducibili alla ssp. \textit{crassirostris}; 1 ind. di
\textit{Ficedula hypoleuca} con alcune caratteristiche riconducibili
alla ssp. \textit{speculigera}.

Inoltre nella collezione sono presenti anche 1 ind. di \textit{Cursorius
cursor} (Gallipoli 1899), 1 ind. di \textit{Charadrius asiaticus} (Bari
1898), 1 ind. di \textit{Calidris maritima} (Bari 1904), 1 ind. di
\textit{Calandrella rufescens} (Bari 1875), 1 ind. di
\textit{Eremophila alpestris} (Bari 1877).

L{\textquoteright}elenco completo della collezione \`e parte integrante
del volume {\textquotedblleft}Avifauna pugliese{\dots} 130 anni
dopo{\textquotedblright} pubblicato al termine del progetto, ed \`e
consultabile anche sul sito
\href{http://www.museoderomita.it}{www.\allowbreak mu\allowbreak se\allowbreak o\allowbreak de\allowbreak ro\allowbreak mi\allowbreak ta.\allowbreak it}.

\section*{Bibliografia}
\begin{itemize}\itemsep0pt
	\item De Romita V., 1884 - Avifauna Pugliese, Catalogo sistematico degli
uccelli. \textit{Rist.anast. a cura di Arnaldo Forni Editore}, Bari. 

	\item De Romita V., 1889 - Aggiunte all'ornitologia pugliese.
\textit{Annuario del Regio Istituto Tecnico Nautico di Bari}. 

	\item De Romita V., 1899 - Nuove aggiunte all'ornitologia
pugliese. \textit{Annuario del Regio Istituto Tecnico Nautico di Bari.}

	\item De Romita V., 1900 - Materiali per una fauna barese. In: La Sorsa S. (a
cura di). \textit{La Terra di Bari sotto l{\textquoteright}aspetto
storico, economico e naturale, vol.III, Vecchi, Trani. pp. 245-338},
ried. Levante editori, Bari 1986. 
\end{itemize}

\setcounter{figure}{0}
\setcounter{table}{0}

\begin{adjustwidth}{-3.5cm}{0cm}
\pagestyle{CIOpage}
\authortoc{\textsc{Massa B.}, \textsc{La Mantia T.}}
\chapter*[Conservazione degli uccelli pelagici nelle isole Pelagie]{Conservazione della principale popolazione europea di berta
maggiore mediterranea \textbf{\textit{Calonectris diomedea}}\textbf{ e
altri uccelli pelagici nelle isole Pelagie}}
\addcontentsline{toc}{chapter}{Conservazione degli uccelli pelagici nelle isole Pelagie}

\textsc{Bruno Massa}$^{1*}$, \textsc{Tommaso La Mantia}$^{1}$ \\

\index{Massa Bruno} \index{La Mantia Tommaso}
\noindent\color{MUSEBLUE}\rule{27cm}{2pt}
\vspace{1cm}
\end{adjustwidth}



\marginnote{\raggedright $^1$Dipartimento SAF - Viale delle Scienze, Ed. 4,
Ingresso H, 90128 Palermo \\
\vspace{.5cm}
{\emph{\small $^*$Autore per la corrispondenza: \href{mailto:bruno.massa@unipa.it}{bruno.massa@unipa.it}}} \\
\keywords{\textit{Calonectris diomedea}, \textit{Puffinus
yelkouan}, \textit{Hydrobates pelagicus melitensis}, isole Pelagie,
conservazione}
{\textit{Calonectris diomedea}, Puffinus yelkouan, \textit{Hydrobates
pelagicus melitensis}, Pelagie islands, conservation}
%\index{keywords}{\textit{Calonectris diomedea}} \index{keywords}{\textit{Puffinus yelkouan}} \index{keywords}{\textit{Hydrobates pelagicus melitensis}} \index{keywords}{Isole Pelagie} \index{keywords}{Conservazione}
}
{\small
\noindent \textsc{\color{MUSEBLUE} Summary} / Authors report on the actions of the Life+ Project Nat/It 00093
{\textquotedbl}Pelagic Birds{\textquotedbl}, aimed to the conservation
of pelagic birds in the islands of Linosa and Lampedusa (Sicily,
Italy). One aim of the project concerns the eradication of the black
rat from the island of Linosa, where it causes a high loss of breeding
success of the scopoli's shearwater \textit{Calonectris diomedea}.\\
}

\vspace{1cm}
La Commissione Europea ha finanziato il progetto Life+ Nat/It 00093
{\textquotedbl}Pelagic Birds{\textquotedbl},
({\textquotedblleft}Conservation of the main European population of
\textit{Calonectris d. diomedea} and other pelagic birds on Pelagic
Islands{\textquotedblright})\textbf{ }il cui capofila \`e il
Dipartimento di Scienze agrarie e forestali
dell'Universit\`a di Palermo e i cui partner
beneficiari sono il Dipartimento Regionale Sviluppo Rurale e
Territoriale (ex Azienda Foreste Demaniali della Regione Siciliana) e
le associazioni ambientaliste Legambiente e Fare Ambiente (cfr.
http://www.pelagicbirds.eu/). Il progetto ha come scopo principale la
conservazione degli uccelli pelagici e in particolare
dell{\textquoteright}endemica berta maggiore mediterranea,
\textit{Calonectris diomedea,} nelle isole Pelagie.

L'isola di Linosa (Agrigento, Canale di Sicilia) ospita
la maggiore colonia europea di berta maggiore mediterranea: circa
10.000 coppie stimate, pari a oltre il 60\% della popolazione italiana
e a oltre il 20\% della popolazione europea; a livello globale, la
colonia \`e seconda solo a quella di Zembra (Tunisia). La berta
maggiore nidifica inoltre con una piccola popolazione
nell'isola di Lampedusa e con circa 200 coppie
nell'isolotto di Lampione. Le colonie si trovano
nell'area \textit{core} per la specie, in cui le
azioni di conservazione possono incidere in modo determinante sullo
stato di conservazione dell'intero taxon; le altre
grandi colonie si ritrovano infatti tutte nel Canale di Sicilia: oltre
che a Zembra (Tunisia), a Malta e Pantelleria (Sicilia). Il progetto
prevede di intervenire sulle popolazioni di ratto nero \textit{Rattus
rattus}, una delle specie che pi\`u interferisce con la sopravvivenza
delle berte nelle isole italiane (Baccetti \textit{et al}. 2009), oltre
che su due piante aliene: \textit{Carpobrotus} spp. e \textit{Nicotiana
glauca}.

Poich\'e si tratta di contesti territoriali circoscritti, e spesso
semplificati, le isole appaiono di frequente vulnerabili
all{\textquoteright}invasione di organismi esotici, come osservato in
molti casi su scala globale (Simberloff 1995) e nel Mediterraneo in
particolare (Pretto \textit{et al}. 2010; Pasta \& La Mantia 2008). 
Diversi studi (D{\textquoteright}Antonio 1990; Vil\`a \&
D{\textquoteright}Antonio 1998; Bourgeois \textit{et al}. 2005) hanno
evidenziato come \textit{Carpobrotus} spp. costituisca una risorsa
trofica d{\textquoteright}importanza cruciale per la dieta invernale
del ratto nero. \textit{N. glauca} colonizza esclusivamente ambienti
soggetti a forte disturbo antropico, luoghi generalmente frequentati
dai ratti. L{\textquoteright}eradicazione di queste due piante potrebbe
indirettamente ridurre l{\textquoteright}habitat idoneo al ratto nero.
Il progetto, della durata di 55 mesi (chiusura: 31 dicembre 2016),
prevede la realizzazione di 22 azioni, di cui 3 preparatorie, tre di
conservazione, 2 di monitoraggio, 7 di disseminazione e 7 di gestione.

\section*{Bibliografia}
\begin{itemize}\itemsep0pt
	\item Baccetti N., Capizzi D., Corbi F., Massa B., Nissardi S., Spano G. \&
Sposimo P., 2009 - Breeding shearwaters on Italian islands: population
size, island selection and co-existence with their main alien predator,
the Black rat. \textit{Riv. ital. Orn}., Milano, 78 (2): 83-100.

	\item Bourgeois K., Suehs C.M., Vidal \'E. \& M\'edail F., 2005 - Invasional
meltdown potential: Facilitation between introduced plants and mammals
on French Mediterranean islands. \textit{Ecoscience}, 12: 248-256. 

	\item D{\textquoteright}Antonio C.M., 1990 - Seed production and dispersal in
the non native, invasive succulent \textit{Carpobrotus edulis
}(Aizoaceae) in coastal strand communities of central California.
\textit{J. Appl. Ecol}., 27: 693-702. 

	\item Pasta S. \& La Mantia T., 2008 - Le specie vegetali aliene in alcuni SIC
siciliani: analisi del grado di invasivit\`a e misure di controllo.
\textit{Mem. Soc. it. Sci. nat. Mus. civ. Stor. nat. Milano}, Milano,
36 (1): 79. 

	\item Pretto F., Celesti-Grapow L., Carli E. \& Blasi C., 2010 - Influence of
past land use and current human disturbance on non-native plant species
on small Italian islands. \textit{Plant Ecol}., 210: 225-239;
Simberloff D., 1995 - Why do introduced species appear to devastate
islands more than mainland areas? \textit{Pacific Science}, 49: 87-97. 

	\item Vil\`a M. \& D{\textquoteright}Antonio C.M., 1998 - Fruit choice and
seed dispersal of invasive vs. non-invasive \textit{Carpobrotus
}(Aizoaceae) in coastal California. \textit{Ecology}, 79 (3):
1053-1060.
\end{itemize}

\setcounter{figure}{0}
\setcounter{table}{0}

\begin{adjustwidth}{-3.5cm}{0cm}
\pagestyle{CIOpage}
\addtocontents{toc}{\protect\newpage}
\authortoc{\textsc{Massari S.}, \textsc{Grattini N.},
\textsc{Fiozzi A.}, \textsc{Ferando E.},
\textsc{Cavaletti E.}}
\chapter*[Analisi delle presenze presso il CRAS Parcobaleno]{\bfseries
Analisi delle presenze ornitiche del Centro recupero (CRAS) di
Parcobaleno (Mantova) - periodo 2003/2012}
\addcontentsline{toc}{chapter}{Analisi delle presenze presso il CRAS Parcobaleno}

\textsc{Simone Massari}$^{1*}$, \textsc{Nunzio Grattini}$^{2}$,
\textsc{Andrea Fiozzi}$^{1}$, \textsc{Eleonora Ferando}$^{1}$,
\textsc{Enrico Cavaletti}$^{1}$\\

\index{Massari Simone} \index{Grattini Nunzio} \index{Fiozzi Andrea} \index{Ferando Eleonora} \index{Cavaletti Enrico}
\noindent\color{MUSEBLUE}\rule{27cm}{2pt}
\vspace{1cm}
\end{adjustwidth}



\marginnote{\raggedright $^1$Centro recupero fauna selvatica WWF Parcobaleno -
Parcobaleno, via Guerra 4/B, 46100 Mantova \\
$^2$Gruppo Ricerche
Avifauna Mantovano (GRAM) - Centro visite del Parco San Lorenzo, Strada
Falconiera, 46020 Pegognaga~(MN) \\
\vspace{.5cm}
{\emph{\small $^*$Autore per la corrispondenza: \href{mailto:simomassari@inwind.it}{simomassari@inwind.it}}} \\
\keywords{Parcobaleno, Mantova, Lombardia,
rapaci, Centro recupero}
{Parcobaleno, Mantua, Lombardy, birds of prey,
recovery}
%\index{keywords}{Parcobaleno} \index{keywords}{Mantova} \index{keywords}{Lombardia} \index{keywords}{Rapaci} \index{keywords}{Centro recupero}
}
{\small
\noindent \textsc{\color{MUSEBLUE} Summary} / The wildlife recovery centre Parcobaleno of Mantova (Lombardy), located
in an abandoned municipal area, in the decade 2003-2012 hosted 865
individuals of 74 different species. Data underline a significant
increase in the number of birds from 2003 to 2012, with a positive peak
in the 2010. The annual average of recovered animals is of 86.5
individuals, with significant fluctuations. Standard deviation over
years is higher for the number of individuals (35.75) than for the
number of species (8,19). The most represented species are:
\textit{Athene noctua }(255 individuals), \textit{Falco tinnunculus}
(160 individuals), \textit{Asio otus} (68 individuals) and
\textit{Strix aluco} (32 individuals), an interesting sample to be used
for ornithological surveys. Crossing provenience data with soil use
maps, it has been discovered a correlation between arrival frequency of
birds from a particular area and the possible abundance of habitat
suitable for their breeding and survival. The majority of little owl
individuals come from the municipalities of Mantova and Roncoferraro
where it is plenty of large old historical and rural buildings. The
kestrel, instead, appears to be more uniformly distributed (Grattini
2008), even if it prefers rural landscapes (Grattini \& Longhi 2010). \\
}
\section*{Introduzione}
Negli anni '90 il WWF locale di Mantova ha elaborato
una proposta di gestione di un{\textquoteright}area comunale
abbandonata per garantire, tra le altre attivit\`a, la costituzione di
un Centro di Recupero della Fauna Selvatica in difficolt\`a. Per
attuare tale proposta ha costituito l'Associazione
Anticitt\`a, che ne \`e a tutt{\textquoteright}oggi il gestore. Nasce
cos\`i il centro di educazione ambientale di PARCOBALENO, situato in
prossimit\`a del Parco storico di Bosco Virgiliano e al confine con la
riserva Naturale della {\textquotedblleft}Vallazza{\textquotedblright}.
Un'area che si presta al reinserimento in natura degli
animali riabilitati.

\section*{Metodi}
In questo lavoro vengono presentati i dati delle presenze ornitiche
ospitate al CRAS (Centro Recupero Animali Selvatici) di Parcobaleno,
nel periodo 2003-12, analizzando l{\textquoteright}andamento de: il
numero di uccelli ricoverati nei 10 anni; il numero di specie totale;
il numero di specie suddiviso per anno nonch\'e una breve analisi
territoriale per le specie pi\`u ricorrenti. 

\section*{Risultati e discussione}
In totale sono stati ricoverati 865 individui, appartenenti a 74 specie.
I dati evidenziano l{\textquoteright}aumento significativo del numero
di uccelli ospitati dall{\textquoteright}anno di apertura (2003) al
2012, con un numero massimo registrato nel 2010.\textcolor{red}{ }

La media annua delle presenze \`e di 86,5 individui, anche se nel
periodo analizzato, si sono manifestate delle oscillazioni
significative. La d.s. misurata tra gli anni \`e maggiore per il valore
del numero di individui (35,75) che per il valore del numero di specie
(8,19). 

Fra le specie pi\`u numerose si ricordano: civetta comune \textit{Athene
noctua}, gheppio \textit{Falco tinnunculus}, gufo comune \textit{Asio
otus} e allocco \textit{Strix aluco}, un interessante campione
potenzialmente utilizzabile per eventuali studi ornitologici. 

Nel corso dei 10 anni, sono giunti al Centro 255 civette, 160 gheppi, 68
gufi comuni e 32 allocchi. Incrociando i dati di provenienza con carte
dell{\textquoteright}uso del suolo si sono individuate delle
correlazioni tra la frequenza dei ricoveri provenienti da una
determinata area e la presenza di habitat adatti alla sopravvivenza e
alla riproduzione della specie. 

A titolo di esempio, dalle analisi dei ritrovamenti della civetta
comune, il maggior numero di individui o nidiate della specie
provengono maggiormente da Mantova e Roncoferraro (sono esclusi
dall{\textquoteright}elaborazione dei dati i soggetti provenienti dai
comuni veronesi ed emiliani di cui non si dispongono dati
sull'utilizzo del suolo). Ipotizziamo che questi
territori forniscano numerosi siti adatti alla nidificazione per la
presenza di centri storici di una certa dimensione (come a Mantova) o
edifici rurali diffusi, come nel caso di Roncoferraro e nei Comuni di
Suzzara, Bagnolo S. Vito e S. Benedetto Po; sembra invece che
l{\textquoteright}area morenica sia meno adatta alla presenza della
civetta comune. Dai dati raccolti emerge tuttavia che la specie \`e
presente in tutto il territorio provinciale. 

Il gheppio \textit{Falco tinnunculus,} rispetto alla civetta comune,
risulterebbe omogeneamente distribuito (sono esclusi
dall{\textquoteright}elaborazione i comuni veronesi ed emiliani),
probabilmente per la sua attitudine a nidificare in nidi abbandonati di
\textit{Corvidae}. Non si individua infatti un{\textquoteright}area
d{\textquoteright}elezione, a conferma di quanto riscontrato
recentemente nel mantovano (Grattini 2008). Anche per il gheppio
tuttavia, sembrano essere privilegiati i contesti rurali (comuni come
Marcaria, Suzzara, S. Benedetto Po, Marmirolo) che probabilmente
offrono sia siti di nidificazione, sia importanti zone di alimentazione
(Grattini \& Longhi  2010).  

Questa preliminare analisi, evidenzia come i CRAS possano contribuire,
nel corso degli anni, alla costituzione di una buona base di dati
ornitologici, che sono utilizzabili per studi sulle dinamiche di
popolazione di alcune specie ornitiche di interesse conservazionistico
(rapaci diurni e notturni \textit{in primis}). 

\section*{Bibliografia}
\begin{itemize}
	\item Grattini N.,\textsc{ 2008 - }Distribuzione, consistenza ed espansione
territoriale del Gheppio \textit{Falco} \textit{tinnunculus}
nidificante in Provincia di Mantova.
\textbf{\textmd{\textit{Alula}}}\textbf{\textmd{, Volume XV (1/2):
189-194.}}

	\item Grattini N. \& Longhi D\textsc{., 2010. - }Avifauna del mantovano
(Lombardia, Italia settentrionale).\textit{ Nat. Bresc.}, 37: 143-181.
\end{itemize}

\setcounter{figure}{0}
\setcounter{table}{0}

\begin{adjustwidth}{-3.5cm}{0cm}
\pagestyle{CIOpage}
\authortoc{\textsc{Mezzavilla F.}, \textsc{Martignago G.},
\textsc{Silveri G.}}
\chapter*[Il re di quaglie nelle Prealpi venete orientali]{Monitoraggio del re di quaglie \textbf{\textit{Crex crex}}\textbf{ nelle Prealpi venete orientali}}
\addcontentsline{toc}{chapter}{Il re di quaglie nelle Prealpi venete orientali}

\textsc{Francesco Mezzavilla}$^{1*}$, \textsc{Gianfranco Martignago}$^{1}$,
\textsc{Giancarlo Silveri}$^{2}$\\

\index{Mezzavilla Francesco} \index{Martignago Gianfranco} \index{Silveri Giancarlo}
\noindent\color{MUSEBLUE}\rule{27cm}{2pt}
\vspace{1cm}
\end{adjustwidth}



\marginnote{\raggedright $^1$Associazione Faunisti Veneti, Museo di Storia
Naturale, S. Croce 1730, I-30135 Venezia \\
$^2$LIPU Pedemontana Trevigiana \\
\vspace{.5cm}
{\emph{\small $^*$Autore per la corrispondenza: \href{mailto:f.mezza@libero.it}{f.mezza@libero.it}}} \\
\keywords{\textit{Crex crex}, Prealpi Trevigiane, monitoraggio
popolazione}
{\textit{Crex crex}, Prealpi Trevigiane, population
monitoring}
%\index{keywords}{\textit{Crex crex}} \index{keywords}{Prealpi Trevigiane} \index{keywords}{Monitoraggio
%popolazione}
}
{\small
\noindent \textsc{\color{MUSEBLUE} Summary} / The corn crake \textit{Crex crex} has been surveyed in eastern venetian Prealps since 1995. 615 singing males were detected within four macro-areas in a surface of ​​120.8 km$^2$. The data analysis performed by the program TRIM 3.5, allowed us to detect a trend in strong decline of - 8.1\% year.\\
}
\section*{Introduzione}
Il re di quaglie \textit{Crex crex }in Veneto \`e stato poco indagato
fino all{\textquoteright}inizio degli anni Novanta del secolo scorso.
Solo a partire dal 1995 ha preso avvio una campagna di monitoraggio che
ha interessato  il massiccio del monte Grappa (106
km\textsuperscript{2}) e successivamente nel 1997
l{\textquoteright}altopiano del Cansiglio (7,9 km\textsuperscript{2}),
nel 1999 i versanti meridionali del monte Cesen (3,5
km\textsuperscript{2}) e nel 2000 il col Visentin (3,4
km\textsuperscript{2}) (Basso \textit{et al}. 1999). Il massiccio del
monte Grappa e l{\textquoteright}altopiano del Cansiglio sono stati
censiti tutti gli anni, mentre nel col Visentin i censimenti non sono
stati svolti nel 2002, 2003 e nel monte Cesen nel 2007, 2008 e 2011. 

\section*{Metodi}

Il monitoraggio \`e stato effettuato tutti gli anni nel mese di giugno e
solo in pochi casi  nella prima decade di luglio. Il censimento si \`e
basato sulla ricerca dei maschi in canto spontaneo e in mancanza di
alcun richiamo, dopo circa 10 minuti si \`e fatto ricorso al metodo del
\textit{play back}, con successivo ascolto delle eventuali risposte.
Negli anni {\textquoteright}90 sono stati effettuati duplici conteggi
nei mesi di maggio e di giugno, ma avendo verificato che il massimo
dell{\textquoteright}attivit\`a canora dei maschi si aveva soprattutto
in giugno, dal 2000 si \`e optato per una sola sessione da svolgere in
questo secondo mese. Complessivamente i punti di ascolto sono stati 59,
cos\`i suddivisi: monte Grappa 37, monte Cesen 4, monte Visentin 5,
altopiano del Cansiglio 13. Nel corso del monitoraggio sono stati
censiti complessivamente 615 maschi cantori. 

\section*{Risultati e discussione}

Il massiccio del monte Grappa con una media annuale di 23,6 maschi ( N =
18 ; SD = 8,9) ha ospitato la popolazione pi\`u numerosa, contrapposta 
al monte Cesen che ha evidenziato la media annua pi\`u bassa di soli
2,2 ( N = 11; SD = 0,6) maschi cantori. Il monte Visentin con una media
di 9,6 maschi (N = 12 ; SD = 3,5) e il Cansiglio con 3,6 ( N = 16 ;  SD
= 2,9) hanno evidenziato situazioni intermedie.  Per valutare il trend
delle presenze negli anni indagati \`e stato  impiegato il programma
TRIM  3.53 (TRends \& Indices for Monitoring data).
Dall{\textquoteright}analisi di ogni area monitorata,  il suo status 
\`e risultato  piuttosto vario, con diminuzioni comprese tra il -7,1 \%
anno del monte Visentin e -1,5 \% anno del Cansiglio. In
quest{\textquoteright}ultima localit\`a per\`o, tra il 2009 e il 2012,
si \`e rilevato un forte calo delle presenze con un solo maschio in
canto. Il trend complessivo nelle Prealpi venete, analizzato sul totale
dei maschi cantori rilevati a partire dal 2000 nelle quattro macroaree,
ha confermato un  andamento negativo ({\textquotedblleft}forte
declino{\textquotedblright}) con valori  pari a -8,1 \%  anno (Fig. \ref{Mezzavilla_fig_1})
, simile alle tendenze rilevate in provincia di Trento (Pedrini
\textit{et al}. 2015) e in altri stati dell{\textquoteright}Europa
occidentale (Green \textit{et al}. 1997).

I motivi della progressiva diminuzione del re di quaglie nelle Prealpi
venete non sono del tutto chiari. In alcune aree fra quelle indagate,
come la piana del Cansiglio, le modalit\`a di sfalcio primaverile dei
prati hanno progressivamente modificato il suo habitat, cos\`i come la
presenza massiccia di ungulati selvatici e domestici (caprini, ovini)
al pascolo, potrebbe aver inciso sul grado di disturbo (Mezzavilla \&
Lombardo 2011). Considerato il forte livello di filopatria,
caratteristico di questa specie (Schaffer 1999), in molti casi si \`e
verificato che bastano 2-3 anni successivi di sfalcio precoce dei
prati, perch\'e in seguito non si insedi pi\`u nello stesso sito. Anche
la progressiva espansione del bosco e l{\textquoteright}abbandono dei
prati e dei pascoli, ha diffusamente ridotto il suo spazio vitale. In
alcuni siti del monte Grappa, in anni diversi \`e stato verificato il
suo insediamento presso le stalle, dove l{\textquoteright}abbondanza di
liquami favoriva uno sviluppo molto esteso di \textit{Urtica dioica}.
La scomparsa di questa specie a causa di interventi
dell{\textquoteright}uomo, cos\`i come l{\textquoteright}eliminazione
di aree dominate da \textit{Epilobium angustifolium}, collegate al
pascolo ovino, caprino e bovino hanno completamente eliminato 
specifici habitat occupati in fase riproduttiva. Tutto ci\`o a fronte
di incentivi comunitari erogati dal Piano di sviluppo rurale  per il
mantenimento degli habitat, che per\`o non sono mai stati utilizzati in
Veneto.

\begin{figure}[!h]
\centering
\includegraphics[width=.95\columnwidth]{Mezzavilla_fig_1.png}
\caption{Trend del re di quaglie rilevato negli anni 2000-2012 nell{\textquoteright}area di studio}
\label{Mezzavilla_fig_1}
\end{figure}

\section*{Bibliografia}

\begin{itemize}\itemsep0pt
	\item Basso E., Martignago G., Silveri G. \& Mezzavilla F., 1999 - Censimenti
del Re di quaglie \textit{Crex crex} nelle Prealpi Venete Orientali.
\textit{Avocetta}, 23: 115.

	\item Green R., Rocamora G. \& Schaffer N., 1997 - Populations, ecology and
threats to the Corncrake \textit{Crex crex} in Europe.
\textit{Vogelwelt}, 118:117-134.

	\item Mezzavilla F. \& Lombardo S., 2011 - Censimenti del Re di quaglie
\textit{Cex crex}, in Cansiglio (1997-2008). Atti 2$^\circ$
Convegno Aspetti naturalistici della provincia di Belluno. Tipografia
Piave. Pp. 165-170.

	\item Schaffer N., 1999 - Habitatwahl und Partnerschaftssystem von Tupfelralle
\textit{Porzana porzana} und Wachtelkonig \textit{Crex crex}.
\textit{Okologie der Vogel} (\textit{Ecol. Birds}), 21:1-267.
\end{itemize}
\setcounter{figure}{0}
\setcounter{table}{0}

\begin{adjustwidth}{-3.5cm}{0cm}
\pagestyle{CIOpage}
\authortoc{\textsc{Morganti N.}, \textsc{Morici F.}, \textsc{Mencarelli M.}}
\chapter*[Monitoraggio dell{\textquoteright}avifauna del comune di
Senigallia (AN)]{Atlante faunistico del comune di Senigallia (AN): monitoraggio
dell{\textquoteright}avifauna del centro urbano}
\addcontentsline{toc}{chapter}{Monitoraggio dell{\textquoteright}avifauna del comune di
Senigallia (AN)}

\textsc{Niki Morganti}$^{1*}$, \textsc{Francesca Morici}$^{1*}$, \textsc{Mauro Mencarelli}$^{1*}$\\

\index{Morganti Niki} \index{Morici Francesca} \index{Mencarelli Mauro}
\noindent\color{MUSEBLUE}\rule{27cm}{2pt}
\vspace{1cm}
\end{adjustwidth}


\marginnote{\raggedright $^1$Studio Naturalistico Diatomea, Senigallia\\
\vspace{.5cm}
{\emph{\small $^*$Autore per la corrispondenza: \href{mailto:info@studiodiatomea.it}{in\allowbreak fo@\allowbreak stu\allowbreak dio\allowbreak dia\allowbreak to\allowbreak mea.\allowbreak it}}} \\
\keywords{Senigallia (AN), avifauna urbana, monitoraggio}
{Senigallia (AN), urban bird, community, monitoring}
%\index[keywords]{Senigallia} \index[keywords]{Avifauna urbana} \index[keywords]{Monitoraggio}
}
{\small
\noindent \textsc{\color{MUSEBLUE} Summary} / During the breeding season in 2013 was studied the bird community in the
center of the city Senigallia (AN). There have been recorded 57 species
of which 19 with some nesting and 17 non-nesting. There were surveyed
nests of house martin: in the village were counted 33 active nests.\\
}

\vspace{1cm}
Il progetto {\textquotedblleft}Atlante Faunistico del comune di
Senigallia (AN){\textquotedblright} \`e stato avviato nel 2009 con
l'obiettivo di predisporre un database con dati sulla
presenza, distribuzione e consistenza delle specie di Vertebrati
Tetrapodi (Mencarelli \textit{et al}. 2009). Grande attenzione \`e
stata dedicata alla classe degli Uccelli, veri e propri
{\textquotedblleft}indicatori{\textquotedblright} in grado di fornire
preziosi elementi sullo stato di salute dell'ambiente
e sulla biodiversit\`a. 

Nella stagione riproduttiva 2013 \`e stata
indagata l'avifauna nidificante nel centro urbano di
Senigallia. L'area di studio comprende il centro
storico di Senigallia, i quartieri residenziali i quali sono
caratterizzati dalla presenza di aree verdi di modeste dimensioni
(giardini privati, orti, ecc.), le aree verdi pubbliche, i centri
sportivi e una zona industriale. Quest'ultima,
localizzata a nord della citt\`a, \`e costituita da capannoni che
ospitano attivit\`a commerciali, artigianali e uffici; sono inoltre
presenti aree incolte e alcuni giardini privati.
L'area \`e stata suddivisa in 15 quadranti di 1 km di lato ed in ogni unit\`a di rilevamento sono stati individuati 4 punti di ascolto/osservazione, visitati tre volte, in cui applicare la
metodologia I.P.A. 

I monitoraggi sono stati avviati nel mese di marzo
2013 e sono terminati nel mese di luglio dello stesso anno. Per ogni
stazione \`e stata effettuata una sosta di 15 minuti per rilevare le
specie presenti. \`E stata compilata una scheda di rilevamento in cui
registrare la presenza delle specie utilizzando dei codici che, per la
stagione riproduttiva, indicano la categoria di nidificazione. Sono
state censite 57 specie di cui 19 con nidificazione certa (piccione
domestico, tortora dal collare, rondone, upupa, balestruccio, usignolo,
codirosso comune, merlo, capinera, cinciarella, cinciallegra, gazza,
taccola, storno, passera d'Italia, passera mattugia,
verzellino, verdone, cardellino), 8 con nidificazione probabile
(gheppio, torcicollo, pettirosso, cannaiola, codibugnolo, cornacchia
grigia, fringuello, strillozzo) e 12 con nidificazione possibile
(tortora selvatica, assiolo, civetta, rondine, ballerina bianca,
scricciolo, codirosso spazzacamino, usignolo di fiume, canapino, lu\`i
piccolo, pigliamosche, zigolo nero); 17 invece le specie non
nidificanti (fenicottero, garzetta, airone cenerino, falco pecchiaiolo,
falco di palude, albanella reale, albanella minore, poiana, lodolaio,
falco pellegrino, gabbiano comune, gabbiano reale, parrocchetto dal
collare, gruccione, cutrettola, stiaccino, sterpazzolina). 

Sono stati
inoltre censiti i nidi di balestruccio delle colonie presenti
nell'area del centro storico, il nucleo pi\`u antico
della citt\`a, per un totale di almeno 17 nidi attivi. Le specie pi\`u
presenti sono risultate: capinera (frequenza 0,86), cardellino (0,67),
passera d'Italia (0,96), tortora dal collare (0,71),
verdone (0,83) e verzellino (0,88). Risultano necessarie analisi pi\`u
approfondite riguardo alla nidificazione di alcune specie di rapaci
notturni (assiolo, civetta e barbagianni; relativamente al barbagianni
nel periodo di studio \`e stato rilevato un individuo morto investito),
della specie naturalizzata parrocchetto dal collare (\textit{Psittacula
krameri}) e riguardo al censimento dei nidi balestruccio:
complessivamente nel centro urbano sono stati contati 33 nidi attivi.

\section*{Bibliografia}
\begin{itemize}\itemsep0pt
	\item Mencarelli M., Morganti N. \& Morici F., 2009 - Progetto atlante
faunistico del comune di Senigallia: Avifauna primo anno di indagine. In: Brunelli M., Battisti
C., Bulgarini F., Cecere

	\item J.G., Fraticelli F., Giustin M., Sarrocco S. \& Sorace A. (A cura di)
Atti del XV Convegno Italiano di Ornitologia. Sabaudia, 14-18 ottobre 2009. \textit{Alula},
XVI (1-2): 680-682.
\end{itemize}

\setcounter{figure}{0}
\setcounter{table}{0}

\begin{adjustwidth}{-3.5cm}{0cm}
\pagestyle{CIOpage}
\authortoc{\textsc{Morganti N.}, \textsc{Mencarelli M.},
\textsc{Morici F.}}
\chapter*[L{\textquoteright}avifauna dell{\textquoteright}oasi
faunistica di San Gaudenzio]{\bfseries
Aggiornamento dell{\textquoteright}avifauna presente
nell{\textquoteright}oasi faunistica di San Gaudenzio}
\addcontentsline{toc}{chapter}{L{\textquoteright}avifauna dell{\textquoteright}oasi
faunistica di San Gaudenzio}

\textsc{Niki Morganti}$^{1*}$, \textsc{Mauro Mencarelli}$^{1*}$,
\textsc{Francesca Morici}$^{1*}$  \\

\index{Morganti Niki} \index{Mencarelli Mauro} \index{Morici Francesca}
\noindent\color{MUSEBLUE}\rule{27cm}{2pt}
\vspace{1cm}
\end{adjustwidth}


\marginnote{\raggedright $^1$Studio Naturalistico Diatomea, Senigallia (AN)\\
\vspace{.5cm}
{\emph{\small $^*$Autore per la corrispondenza: \href{mailto:info@studiodiatomea.it}{in\allowbreak fo@\allowbreak stu\allowbreak dio\allowbreak dia\allowbreak to\allowbreak mea.\allowbreak it}}} \\
\keywords{Oasi di San Gaudenzio (AN), avifauna, monitoraggio}
{Oasi di San Gaudenzio (AN), bird communities, monitoring}
%\index{keywords}{Oasi di San Gaudenzio} \index{keywords}{Avifauna} \index{keywords}{Monitoraggio}
}
{\small
\noindent \textsc{\color{MUSEBLUE} Summary} / Three years after the last survey, we have monitored wintering and
breeding birds in the Oasi of San Gaudenzio (Marche, central Italy).
The resulting check-list includes 125 species, but breeding species
show an considerable decrease, except for the aquatic birds.\\
}

\vspace{1cm}
A distanza di tre anni dall{\textquoteright}ultimo monitoraggio, sono
stati condotti i rilevamenti dell{\textquoteright}avifauna svernante e
nidificante nell{\textquoteright}Oasi faunistica di San Gaudenzio
(Senigallia, AN). Inoltre, si \`e stata aggiornata la check-list
dell{\textquoteright}Oasi attraverso una ricerca bibliografica e la
richiesta di dati in possesso di altri \textit{birdwatcher} che
frequentano l{\textquoteright}area di studio. 

La zona oggetto di indagine \`e l{\textquoteright}Oasi faunistica di San
Gaudenzio che si estende per 36 ettari nelle colline in destra
orografica del fiume Misa, a pochi chilometri
dall{\textquoteright}abitato di Senigallia (AN). In passato la zona \`e
stata sfruttata come cava di marna che, a seguito
dell{\textquoteright}abbandono dell{\textquoteright}attivit\`a
estrattiva, si \`e rinaturalizzata spontaneamente: attualmente
insistono diverse tipologie ambientali: due laghetti di diverse
dimensioni e profondit\`a e la relativa vegetazione ripariale, scarpate
rocciose nude, arbusteti con dominanza della Ginestra, piccole aree
boschive costituite principalmente dalla Roverella e, infine, la
struttura della fornace della cava oramai abbandonata da oltre 50 anni
(Furlani \& Morici 2006).

I rilevamenti sono stati condotti due volte al mese nel periodo compreso
tra dicembre 2012 e luglio 2013. La metodologia utilizzata \`e stata
quella dei punti di ascolto/osservazione con sosta di 15 minuti in
ognuno, in linea con i monitoraggi precedenti in modo cos\`i da
ottenere dati e risultati confrontabili col passato. Per
l{\textquoteright}elaborazione dei dati dei nidificanti \`e stato
calcolato l{\textquoteright}Indice Puntiforme di Abbondanza (IPA).

Nel periodo di ricerca sono state censite 77 specie, 11 in meno rispetto
ai monitoraggi del 2009-2010. Di queste 77, 54 sono le specie
nidificanti (69 nel 2009-2010); anche il valore IPA totale risulta
nettamente inferiore rispetto alla precedente ricerca: 442,5 nel
2012-2013 e 725,5 nel 2009-2010. La check-list annovera 125 specie, di
cui quattro (corriere piccolo, piro piro culbianco, topino e cincia
mora), osservate nel 2013, non erano state censite prima.

L{\textquoteright}Oasi faunistica di San Gaudenzio, pur se di limitata
estensione, mostra una notevole ricchezza avifaunistica: la ragione
principale di ci\`o pu\`o essere data dalla variet\`a di ambienti
presenti al suo interno. Rispetto al 2009-2010, i dati mostrano una
sensibile perdita della componente nidificante: le uniche specie che
mostrano un aumento sono quelle che nidificano in habitat umidi (es.
Folaga) e qualcuna di habitat boschivi, come il colombaccio. Tutte le
altre, in particolare le specie di habitat arbustivi, sono in netto
calo. Dal 2009 non si sono verificati cambiamenti nelle condizioni
vegetazionali o infrastrutturali dell{\textquoteright}area, pertanto le
ragioni di questi mutamenti della componente avifaunistica potrebbero
essere legate alle cattive stagioni primaverili e agli inverni rigidi
verificatisi negli ultimi 2 anni: in conseguenza di ci\`o, non solo a
San Gaudenzio ma anche a scala nazionale, sono scomparse alcune specie
o drasticamente calate le loro dimensioni di popolazione, come il
beccamoschino, lo strillozzo e il saltimpalo. Altre specie in netto
calo sono rondine, rondone comune, passera d{\textquoteright}Italia,
passera mattugia, ma le cause sono probabilmente da ricercare a pi\`u
vasta scala. 

\section*{Bibliografia}
\begin{itemize}\itemsep0pt
	\item Furlani M. \& Morici F., 2006 -- Caratteri naturalistici della cava di
San Gaudenzio. In: Villani V. e Mauro F. (a cura di).
	\item L{\textquoteright}Oasi di San Gaudenzio di Senigallia. Valori storici
ed ambientali. Gruppo {\textquotedblleft}Societ\`a e
Ambiente{\textquotedblright}, Senigallia.
\end{itemize}


\setcounter{figure}{0}
\setcounter{table}{0}

\begin{adjustwidth}{-3.5cm}{-1cm}
\pagestyle{CIOpage}
\authortoc{\textsc{Morici F.}, \textsc{Mencarelli M.},
\textsc{Sebastianelli C.}, \textsc{Morganti N.}}
\chapter*[Disturbo alla nidificazione del fratino e misure di protezione]{Studio del disturbo alla nidificazione del fratino \textbf{\textit{Charadrius alexandrinus}} e misure di
protezione dei nidi lungo i litorali di Senigallia e Montemarciano (AN)
}
\addcontentsline{toc}{chapter}{Disturbo alla nidificazione del fratino e misure di protezione}
 
\end{adjustwidth}
\begin{adjustwidth}{-3.5cm}{0cm}
\textsc{Francesca Morici}$^{1*}$, \textsc{Mauro Mencarelli}$^{1*}$,
\textsc{Claudio Sebastianelli}$^{2**}$, \textsc{Niki Morganti}$^{1*}$  \\

\index{Morici Francesca} \index{Mencarelli Mauro} \index{Sebastianelli Claudio} \index{Morganti Niki}
\noindent\color{MUSEBLUE}\rule{27cm}{2pt}
\vspace{1cm}
\end{adjustwidth}


\marginnote{\raggedright $^1$Studio Naturalistico Diatomea, Senigallia
\textit{info@studiodiatomea.it} \\
$^2$A.R.C.A., Senigallia \\
\vspace{.5cm}
{\emph{\small $^*$Autore per la corrispondenza: \href{mailto:info@studiodiatomea.it}{in\allowbreak fo@\allowbreak stu\allowbreak dio\allowbreak dia\allowbreak to\allowbreak mea.\allowbreak it}}} \\
{\emph{\small $^{**}$Autore per la corrispondenza: \href{mailto:info@associazionearca.eu}{in\allowbreak fo@\allowbreak as\allowbreak so\allowbreak cia\allowbreak zio\allowbreak ne\allowbreak ar\allowbreak ca.\allowbreak eu}}} \\
\vspace{.5cm}
\keywords{Senigallia e Montemarciano, \textit{Charadrius
alexandrinus}, disturbo dei nidi, misure di conservazione, gabbiette di
esclusione, perimetrazione dei nidi}
{Senigallia and Montemarciano, \textit{Charadrius
alexandrinus}, nest disturbance, conservation measures,
exclosure box, nest fencing}
%\index{keywords}{Senigallia e Montemarciano} \index{keywords}{\textit{Charadrius
%alexandrinus}} \index{keywords}{Disturbo dei nidi} \index{keywords}{Misure di conservazione} \index{keywords}{Gabbiette di esclusione} \index{keywords}{Perimetrazione dei nidi}
}
{\small
\noindent \textsc{\color{MUSEBLUE} Summary} / We investigated the types of disturbance
and their effects on kentish plover \textit{Charadrius
alexandrinus} nests, also analyzing
the effectiveness of some protection measures. Individuals
responded differently to disturbance returning to the nest on average after 10 minutes if
disturbed by humans, after 60 minutes or abandoning the nest if
disturbed by motor-vehicles. Nest protection operated by exclusion
boxes against predators (dogs, crows) was not satisfying as 50\% to
100\% of the nests were abandoned because of disturbance determined by
humans attracted by the presence of the box. Fencing operated by red
and white strings around a small triangular area largely improved nest
survival in tourist beaches.\\
}
\vspace{1cm}




Nella stagione riproduttiva 2013 abbiamo studiato i diversi tipi di
disturbo che si osservano sui nidi di fratino e valutato i loro effetti
sulla nidificazione analizzando anche l'efficacia
delle azioni di tutela intraprese al fine di attuare nei prossimi anni
misure di protezione mirate pi\`u efficaci. Attualmente la specie
nidifica con regolarit\`a nelle Marche lungo i litorali di Senigallia,
Montemarciano e Fermo (Morganti \textit{et al}. 2009) (Mencarelli
\textit{et al. }2013). La ricerca ha interessato il litorale del Comune
di Senigallia (AN) e Montemarciano (AN). Sono state individuate quattro
aree distinte: Cesanella e Cesano (litorale Nord) caratterizzate da
spiaggia sabbiosa e dune embrionali; Marzocca (litorale Sud) e
Montemarciano caratterizzati da spiaggia ghiaiosa con vegetazione
dunale rada. Per lo studio del disturbo sono stati monitorati 5 nidi
con osservazioni dirette della durata variabile dai 15 ai 120 minuti,
condotte tra le ore 10:00 e le ore 13:00. Sono state annotate le
seguenti informazioni: tipologia di disturbo e distanza dal nido,
risposta del fratino al disturbo e periodo in cui il nido rimane
incustodito. Per la protezione dei nidi sono state adottate due
strategie: 1. perimetrazione triangolare con nastro plastificato bianco
e rosso per evitare il calpestio; 2. gabbiette di esclusione anti
predazione con rete a maglia larga chiuse superiormente. Nel 2013 sono
state adottate misure di protezione in 14 nidi con esiti differenti a
seconda della zona in cui sono state utilizzate.
L{\textquoteright}utilizzo delle gabbiette ha visto per il litorale
Nord il 50\% dei nidi con esito positivo, per il litorale Sud il 100\%
dei nidi con esito negativo e per Montemarciano il 50\% dei nidi con
esito positivo. L{\textquoteright}utilizzo della recinzione ha visto
per il litorale Sud l{\textquoteright}80\% dei nidi con esito positivo
e per Montemarciano il 100\% dei nidi con esito positivo. Infine,
l{\textquoteright}utilizzo della recinzione con la gabbietta, adottata
solo per un nido nella zona del litorale Sud, ha avuto esito negativo.
L'analisi del disturbo ai nidi ha permesso di
classificarne le tipologie e individuarne le pi\`u frequenti,
determinare i tempi di allontanamento dal nido e identificare quelle
che ne determinano l'abbandono. La tipologia di
disturbo pi\`u ricorrente \`e stata la presenza antropica, con un
disturbo medio di 10 minuti e un ritorno alla cova tra i 4 e i 20
minuti dopo la scomparsa del disturbo. I mezzi meccanici hanno creato
un disturbo medio di oltre 60 minuti e un ritorno al nido variabile da
60 minuti al non ritorno. Il prolungarsi di questi due disturbi ha
causato l'abbandono di almeno 3 nidi. Le misure di
protezione adottate non sono state utilizzate in maniera standard lungo
il litorale. Mentre le gabbiette sono state efficaci
nell'area del litorale Nord, al litorale Sud e
Montemarciano la presenza delle gabbiette ha incuriosito i bagnanti
che, avvicinandosi troppo al nido, ne hanno causato
l'abbandono. In queste zone sono state utilizzate con
maggiore successo le recinzioni che, delimitando pi\`u genericamente
l'area di nidificazione, attirano meno
l{\textquoteright}attenzione dei bagnanti. L{\textquoteright}utilizzo
contemporaneo di recinzione e gabbietta lungo il litorale Sud ha
comportato gli stessi problemi osservati con l{\textquoteright}utilizzo
della sola gabbietta. Dal 2014 verranno diversificate le tipologie di
recinzioni in base alla zona del litorale in cui si trovano i nidi. I
dati raccolti durante il monitoraggio del disturbo hanno fornito
elementi utili alle future azioni di conservazione. Nonostante alcuni
casi di predazione, non c{\textquoteright}\`e stato abbandono della
covata per presenza di cani o cornacchie: i fratini hanno risposto con
la fuga, per poi fare ritorno al nido. In un caso \`e stata osservata
un'interazione con un conspecifico:
l'individuo in cova ha abbandonato per pochi minuti le
uova per allontanare l'intruso. Preoccupa
particolarmente il disturbo arrecato dai mezzi meccanici. Sarebbe
necessario adottare misure di protezione condivise tra amministrazioni
locali, operatori e coloro che si occupano della tutela della specie.

\section*{Ringraziamenti}
Gli autori ringraziano Claudia Latini e Caterina Abbrugiati, tirocinanti
della Facolt\`a di Scienze dell{\textquoteright}UNIVPM di Ancona, che
hanno partecipato al monitoraggio.

\section*{Bibliografia}
\begin{itemize}\itemsep0pt
	\item Mencarelli M., Morici F., Sebastianelli C. \& Morganti N., 2013 -
\textcolor{black}{Il Fratino (}\textit{\textcolor{black}{Charadrius
alexandrinus}}\textcolor{black}{) nidificante sul litorale di
Senigallia (AN) e Montemarciano (AN): distribuzione, problematiche e
strategie di conservazione. Stagioni 2009-2012. }U.D.I. XXXVIII: 67-76
(2013).

	\item Morganti N., Fusari M., Mencarelli M., Morici F., Pascucci M. \& Marini
G., 2009 - Aspetti ecologici della nidificazione di \textit{Charadrius
alexandrinus }lungo il litorale marchigiano. In: Brunelli M., Battisti
C., Bulgarini F., Cecere J.G., Fraticelli F., Giustin M., Sarrocco S.
\& Sorace A. (A cura di). Atti del XV Convegno Italiano di Ornitologia.
Sabaudia, 14-18 ottobre 2009\textit{. Alula}, XVI (1-2): 252-254.
\end{itemize}

\setcounter{figure}{0}
\setcounter{table}{0}

\begin{adjustwidth}{-3.5cm}{0cm}
\pagestyle{CIOpage}
\authortoc{\textsc{Nissardi S.}, \textsc{Zucca C.},
\textsc{Cherchi F.}, \textsc{Atzori J.}}
\chapter*[Density and home range of purple gallinule]{
Densit\`a e \textit{home range} di pollo sultano \textit{Porphyrio porphyrio} in un{\textquoteright}area protetta della Sardegna meridionale (Parco Naturale Regionale Molentargius Saline)}
\addcontentsline{toc}{chapter}{Density and home range of purple gallinule}

\textsc{Sergio Nissardi}$^{1*}$, \textsc{Carla Zucca}$^{1*}$,
\textsc{Fabio Cherchi}$^{2}$, \textsc{Jessica Atzori}$^{2}$  \\

\index{Nissardi Sergio} \index{Rotelli Luca} \index{Fabio Cherchi} \index{Jessica Atzori}
\noindent\color{MUSEBLUE}\rule{27cm}{2pt}
\vspace{1cm}
\end{adjustwidth}


\marginnote{\raggedright $^1$Anthus snc, Via Luigi Canepa 3, 09129 Cagliari
(Italy), Fax 070496956 \\
$^2$Via del Trifoglio n. 10, 09045 Quartu S.E. (Italy) \\
\vspace{.5cm}
{\emph{\small $^*$Autore per la corrispondenza: \href{mailto:anthus@anthus.info}{anthus@anthus.info}}} \\
\keywords{Parco naturale regionale Molentargius Saline, Sardegna,
\textit{Porphyrio porphyrio}, densit\`a, \textit{home range}}
{Molentargius Saline natural regional Park, Sardinia, \textit{Porphyrio porphyrio}, density, home range}
%\index{keywords}{\textit{Porphyrio porphyrio}} \index{keywords}{Densit\'a} \index{keywords}{\textit{Home range}}
}
{\small
\noindent \textsc{\color{MUSEBLUE} Summary} / The aim of the study was to evaluate the density, abundance and home
ranges of purple gallinule \textit{Porphyrio porphyrio} in Molentargius
natural regional Park.
The overall abundance was estimated at 55-75 (through line transects and
point counts), of which 30-40 in Ecosistema Filtro (a portion of 37
ha), where the density was evaluated at 0.8-1.1 pairs / ha.
Through banding with color rings, research of the banded birds by visual
census and radio-tracking we evaluated that the home range can be less
than one hectare, suggesting a strong sedentariness, at least for
adults, and confirming the density values obtained from the censuses.\\
}
\vspace{1cm}

{
Lo studio, svolto per conto del Parco naturale regionale Molentargius
Saline (Sardegna), \`e stato finalizzato ad approfondire le conoscenze
su densit\`a e comportamento territoriale del pollo sultano
\textit{Porphyrio porphyrio}, all{\textquoteright}interno di questa
zona umida che comprende 150 ha di bacini ad acque dolci, fra cui un
impianto di fitodepurazione di 37 ha (Ecosistema Filtro) nel quale si
sono svolte le attivit\`a di cattura e marcatura degli animali. Data la
particolare etologia di questa specie (si riproduce praticamente per
tutto l{\textquoteright}anno, con picchi di deposizioni a
dicembre-gennaio e marzo-maggio
(\textcolor[rgb]{0.0,0.0,0.039215688}{Schenk 1993; Grussu 1999)}, lo
studio ha interessato un{\textquoteright}intera annualit\`a (da ottobre
2010 a settembre 2011), al fine di valutare eventuali variazioni
intra-annuali nella densit\`a (o nella contattabilit\`a) o nel
comportamento territoriale. Lo studio si \`e svolto grazie a: 1)
censimenti periodici mediante transetti lineari e punti di
osservazione/ascolto; 2) cattura e inanellamento (marcando gli animali
anche con anello colorato per consentirne il riconoscimento mediante
osservazione a distanza); 3) campagne di lettura per localizzare gli
individui marcati; 4) radio-\textit{tracking}.}

{
I censimenti hanno evidenziato una stima prudenziale di 55-75
coppie/nuclei familiari in tutto il sistema di Molentargius, di cui
30-40 nel solo Ecosistema Filtro (densit\`a pari a 0,8-1,1 coppie
ha\textsuperscript{{}-1} di superficie, inclusi gli specchi
d{\textquoteright}acqua). Il valore cos\`i ottenuto ammonta a oltre il
10\% del totale di 450-600 coppie riportato da Andreotti (2001) per la
popolazione sarda. \`E stato catturato e marcato un campione di 7
individui, di cui 6 equipaggiati con trasmittente. La ricerca di
individui inanellati ha avuto modesti risultati in rapporto allo sforzo
di rilevamento, dato che nel periodo di studio sono stati identificati
2 soli individui, di cui uno era stato inanellato precedentemente, il
6/09/2007, ed \`e stato osservato il 3/05/2011, quindi con un
intervallo fra la marcatura e la rilettura di 1335 giorni. Il secondo,
catturato e inanellato il 20 febbraio 2011, \`e stato osservato il
21/05/2011, con un intervallo fra la marcatura e la rilettura di 90
giorni. Un terzo individuo, inanellato il 12 settembre 2009, \`e stato
osservato, dopo la conclusione del monitoraggio, il 9 giugno 2013 (1366
giorni dopo la cattura).}

{
Per quanto concerne lo studio degli spostamenti individuali, dei 6
individui a cui \`e stata applicata la radio, solo 3 hanno
l{\textquoteright}hanno tenuta per un tempo superiore ai 3 giorni:
l{\textquoteright}individuo marcato con anello AC, catturato una prima
volta il 30 dicembre 2010, ha perso la radio dopo 2 giorni; ripreso il
primo ottobre 2011 \`e stato monitorato per 85 giorni, quindi fino a
359 giorni dalla prima cattura, sempre all{\textquoteright}interno di
una superficie valutabile in 0,17-0,71 ha (intervallo ottenuto
considerando un possibile errore di circa 15 m nella localizzazione
mediante radio-\textit{tracking}); l{\textquoteright}individuo AD
marcato il 9 gennaio \`e stato seguito per 35 giorni con spostamenti in
un{\textquoteright}area di 0,24-0,81 ha; l{\textquoteright}individuo
AF, marcato il 20 febbraio, \`e stato seguito per 30 giorni fino alla
perdita della radio e riosservato dopo 90 giorni dalla prima cattura,
si \`e mosso in un{\textquoteright}area di 0,29-0,69 ha. \`E da
rilevare inoltre che gli individui AC e AD occupano territori in gran
parte sovrapposti, la cui superficie complessiva \`e valutabile in
circa un ettaro. Questo dato potrebbe indicare che si tratti di una
coppia territoriale, essendo AC certamente maschio e AD probabilmente
femmina, in quanto decisamente pi\`u piccolo, anche se con misure non
discriminanti. Questi risultati sembrano confermare i valori di
densit\`a ottenuti attraverso i censimenti ed evidenziano un
comportamento sostanzialmente sedentario, almeno degli adulti (tutti
gli individui monitorati per un periodo significativo erano adulti) che
si muovono entro territori ristretti, valutabili in meno di un ettaro,
che possono essere frequentati dai medesimi individui durante
l{\textquoteright}intero ciclo annuale.}


\section*{Bibliografia}

\begin{itemize}\itemsep0pt
	\item Andreotti A. (a cura di), 2001 - Piano d{\textquoteright}azione
nazionale per il Pollo sultano (\textit{Porphyrio porphyrio}).
Quad. Cons. Natura, 8, Min. Ambiente - Ist. Naz. Fauna
Selvatica.

	\item Grussu M., 1999 - Status and breeding ecology of the Purple Swamp-hen in
Italy. British Birds 4 (92): 183-192.

	\item Schenk H., 1993 - Pollo sultano. In: Meschini E., Frugis S.(eds),
Atlante degli uccelli nidificanti in Italia. Suppl. Ric. Biol.
Selvaggina, XX: 107.

\end{itemize}

\setcounter{figure}{0}
\setcounter{table}{0}

\begin{adjustwidth}{-3.5cm}{0cm}
\pagestyle{CIOpage}
\authortoc{\textsc{Partel P.}, \textsc{Rotelli L.}}
\chapter*[Progetto radiotelemetrico sul Gallo cedrone]{Il progetto radiotelemetrico sul
Gallo cedrone \textit{Tetrao urogallus} nel Parco Naturale
Paneveggio - Pale di San Martino}
\addcontentsline{toc}{chapter}{Progetto radiotelemetrico sul Gallo cedrone}

\textsc{Piergiovanni Partel}$^{1*}$, \textsc{Luca Rotelli}$^{2}$  \\

\index{Partel Piergiovanni} \index{Rotelli Luca}
\noindent\color{MUSEBLUE}\rule{27cm}{2pt}
\vspace{1cm}
\end{adjustwidth}


\marginnote{\raggedright $^1$Parco Naturale Paneveggio Pale di San Martino \\
$^2$Universit\`a di Friburgo \\
\vspace{.5cm}
{\emph{\small $^*$Autore per la corrispondenza: \href{mailto:piergiovanni.partel@parcopan.org}{pier\allowbreak gio\allowbreak van\allowbreak ni.\allowbreak par\allowbreak tel@\allowbreak par\allowbreak co\allowbreak pan.\allowbreak org}}} \\
\keywords{Trentino, \textit{Tetrao urogallus}, radiotelemetria, successo
riproduttivo}
{Trentino, \textit{Tetrao urogallus}, radiotelemetry, breeding success}
%\index{keywords}{Trentino} \index{keywords}{\textit{Tetrao urogallus}} \index{keywords}{Radiotelemetria} \index{keywords}{Successo riproduttivo}
}
{\small
\noindent \textsc{\color{MUSEBLUE} Summary} / The Paneveggio - Pale di San Martino Provincial Park has been carrying
out a radiotelemetry project for four years. During this period 22
cocks and 6 hens were captured, tagged and intensively tracked. In the
meanwhile counts were carried out both in spring, on the leeks, to
determine the number of displaying cocks and thus the population size
and structure and in summer, by means of pointing dogs, with the aim to
determine the breeding success.\\
}


\section*{Introduzione}

{Negli ultimi decenni le popolazioni alpine
di gallo cedrone }\textit{{Tetrao
urogallus}}{ hanno subito
un{\textquoteright}importante riduzione delle loro consistenze sulle
Alpi italiane. Tra le principali cause di declino vengono indicate la
perdita, il degrado e la frammentazione degli habitat, alcune pratiche
selvicolturali, ma anche un aumento della pressione predatoria, i
cambiamenti climatici e i disturbi causati da alcune attivit\`a
antropiche come gli sport invernali (Storch 2000).}

{Nelle aree centrali e meridionali delle
Alpi, le azioni di conservazione a favore della specie fino ad ora
}si\textcolor{red}{ }{sono basate
prevalentemente su indicazioni generiche piuttosto che su solide
conoscenze maturate nel corso di studi mirati. Mancano, infatti,
conoscenze approfondite inerenti le relazioni esistenti tra il gallo
cedrone e il suo habitat e quindi delle effettive cause che minacciano
le popolazioni di questo Tetraonide.}

{Per questo motivo dal 2009 il Parco
Naturale Paneveggio-Pale di San Martino ha avviato un progetto di
ricerca sulla biologia ed ecologia della specie, condotto con
l{\textquoteright}ausilio della radiotelemetria.}

{La ricerca \`e stata realizzata in
collaborazione con l{\textquoteright}Universit\`a di Friburgo e il
Servizio Foreste e fauna della Provincia autonoma di Trento.}

\section*{Metodi}

{Dal 2009 al 2012, 22 maschi e 6 femmine
sono stati catturati, marcati con radiocollari VHF del peso di 20 gr. 
e intensamente localizzati, per mezzo di
}\textit{{homing
in}}{ e triangolazione. Nello stesso
periodo sono stati eseguiti censimenti primaverili sulle arene di
canto, per conoscere la consistenza e la struttura della popolazione.
Inoltre sono stati realizzati censimenti estivi, con
l{\textquoteright}ausilio di cani da ferma, al fine di determinare il
successo riproduttivo della specie.}

\section*{Risultati e discussione}

{La popolazione di gallo cedrone
nell{\textquoteright}area del Parco e nei territori limitrofi risulta
essere vitale, con la presenza di arene di canto spesso ben
strutturate, caratterizzate da un numero medio di maschi per arena di
canto superiore a 2, con
}\textit{{lek}}{
frequentati anche fino da 10 maschi. Tra i dati ottenuti risalta il
basso successo riproduttivo (0,63 pulli per femmina), controbilanciato
da un tasso di sopravvivenza piuttosto elevato degli adulti (68\%). In
media solo il 35\% delle femmine in estate \`e accompagnata da nidiata,
mentre il numero medio di pulli per covata \`e di 1,85. Il 65\% dei
nidi viene perso durante la cova. Di questi l'80\%
risulta essere predato, mentre nel rimanente 20\% dei casi i nidi
vengono abbandonati dalle femmine a causa del disturbo antropico. Tra
le cause di mortalit\`a degli adulti, importante risulta essere
soprattutto la predazione, la cui incidenza sulla popolazione pu\`o
variare notevolmente di anno in anno, oltre a quella da collisione
contro i cavi degli impianti di risalita (Rotelli 2012). }

{Questi primi risultati, anche se non ancora
definitivi, tuttavia sono importanti per poter implementare misure di
conservazione in grado di ridurre i fattori che influenzano in modo
negativo il trend delle popolazioni di gallo cedrone
nell{\textquoteright}area meridionale delle Alpi.}

\section*{Bibliografia}
\begin{itemize}\itemsep0pt
	\item Storch I., 2000 - Grouse Status Survey and Conservation Action
Plan 2000-2004. WPA/BirdLife/SSC Grouse Specialist Group. IUCN, Gland.
Switzerland and Cambridge, UK and the World Pheasant Association,
Reading, UK. X+, 112 pp.

	\item Rotelli L., 2012 - Risultati dell{\textquoteright}attivit\`a
svolta nell{\textquoteright}ambito del progetto sul Gallo cedrone nel
Parco Naturale Paneveggio -- Pale di San Martino nel periodo marzo
2011 -- febbraio 2012 (con integrazioni fino al 31 marzo 2012).
Dattiloscritto a cura del Dipartimento di Ecologia e Gestione della
Fauna Selvatica dell{\textquoteright}Universit\`a di Freiburg
(Germania).
\end{itemize}


\begin{otherlanguage}{english}
\setcounter{figure}{0}
\setcounter{table}{0}

\begin{adjustwidth}{-3.5cm}{0cm}
\pagestyle{CIOpage}
\authortoc{\textsc{Rassati G.}}
\chapter*[]{"Silviphobia" and urban bird populations}
\addcontentsline{toc}{chapter}{"Silviphobia" and urban bird populations}

\textsc{Gianluca Rassati}$^{1*}$\\

\index{Rassati Gianluca}
\noindent\color{MUSEBLUE}\rule{27cm}{2pt}
\vspace{1cm}
\end{adjustwidth}

\marginnote{\raggedright $^1$Via Udine 9, 33028 Tolmezzo (UD) \\
\vspace{.5cm}
{\emph{\small $^*$Autore per la corrispondenza: \href{mailto:itassar@tiscali.it}{i\allowbreak ta\allowbreak s\allowbreak sar@\allowbreak ti\allowbreak sca\allowbreak li.\allowbreak it}}} \\
\keywords{Alpi Carniche, selvicoltura, popolamento ornitico, area urbana, biodiversità, distrurbo antropico}
{Carnic Alps, silviculture, bird population, urban area, biodiversity, human disturbance}
} 
{\small
\noindent\textsc{\color{MUSEBLUE} Riassunto} / "Selvifobia" e popolamenti ornitici urbani. Avendo constatato un impoverimento della diversità arborea
urbana in zona alpina carnica, nel periodo compreso fra gli anni 1998-1999 e 2011-2012, se ne è voluto verificare
l’effetto sui popolamenti ornitici di Tolmezzo. I valori dei parametri del secondo periodo (Tab. \ref{Rassati_tab_1}) sono risultati più
bassi di quelli del primo, dimostrando un calo di diversità ornitica. Questo è stato determinato da specie legate agli
alberi, non contattate nel secondo periodo o il cui IKA è diminuito talvolta fortemente. \\
}

\vspace{1cm}
In the last decades, the decrease of the Alpine human populations and the departure of the young from
the rural world, together with the abandonment of the agricultural, silvicultural and pastoral activities, favoured the
spread of idea that the forest is literally devouring the mountain urban agglomerations, causing a kind of
"silviphobia" that brought to a rage against the trees within the urban agglomerations.

Based on such ideas, verified trough analysis of what is happened in last years in the villages of
Carnic Alps, we used temporal comparison data of the most extended urban area, Tolmezzo, to verify if the change in tree cover caused an impact on the bird populations.

The comparison was realized using the linear transect method (Bibby }\textit{\textcolor{black}{et
al.}} (2000), on a 1 km long transect in the years 1998 and 1999 and, 2011 and 2012. Each year, the transect was carried out once from 1}\textcolor{black}{\textsuperscript{st}}\textcolor{black}{ April to
15}\textcolor{black}{\textsuperscript{th}}\textcolor{black}{ May and a second time from
16}\textcolor{black}{\textsuperscript{th}}\textcolor{black}{ May to
30}\textcolor{black}{\textsuperscript{th}} June. During the survey only the individuals contacted
within a belt of 25 m on both sides were considered. For each year, only the higher value between the two periods was
considered.

\textcolor{black}{The monitored area (46$^{\circ}$24' N, 13$^{\circ}$01' E - 315 m a.s.l.) is flat and represented by an extensive residential zone. The arboreous and shrubby species are numerous, both native and alien.}

In the lapse of time included between the two periods in which the surveys were carried out several
old trees were cut, also because of various restorations and enlargement and realization of construction works and many
others, most of all of public property, were subjected to severe prunings especially regarding old branches. This
latter operation led to the death of a further number of trees: we estimated a decrease of 30\% of the total number of
trees, of 40\% of the total standing stock, of 70\% of the number of trees with dead parts or with cavities.

\textcolor{black}{Examinating Tab. \ref{Rassati_tab_1} we deduce a decrease of the value of the parameters between the first and the
second period of the survey: number of species (-10.71\%), KAI (-17.20\%), number of dominant species (-40.00\%) and
relative percent frequency (-32.81\%), percent of non-passerines species (-}\textcolor{black}{20.00\%); also the
diversity index (Shannon \& Weaver 1963) and the equitability index (Magurran 1988) are lower in the second period.}

\textcolor{black}{If we analyse in details the populations, we observe that even in presence of a high value of
similarity index (Sørensen 1948) of 0.83, there are some substantial differences. The first one is that the higher
number of species in 1998-1999 is determined by }\textit{\textcolor{black}{taxa}}\textcolor{black}{\ related to trees as
sparrowhawk, wryneck, green woodpecker, goldcrest, marsh }{tit, short-toed
treecreeper, not contacted in the second period. Secondly the higher KAI is determined not only by the above species
but also by the decrease of the values of arboreal }\textit{\textcolor{black}{taxa}}\textcolor{black}{\ contacted in
both periods as great spotted woodpecker (2.5 vs 0.5) and nuthatch (2.0 vs 0.5) but also chaffinch (7.0 vs 4.5) and
greenfinch (7.0 vs 5.5). The species found only in the second period, on the other hand, are not closely related to
trees, particularly the third one: cuckoo, magpie, starling. The magpie moreover began to use regularly the urban
habitats exactly in the first decade of the XXI century. These differences contribute moreover to decrease the indexes
and the diversity in the second period, indeed also the percentage of non-passerines species, due to the disappearance
of some picid species, decreases, simplifying the composition of the populations.}

What emerged shows that also in urban areas the silvicultural operations should be planned and managed
by specialised technicians to avoid, as now often happens also in public field, that such operations might be used as a
pretext to impoverish or eliminate the tree stands (e.g. substituting old trees and/or of native species with young
trees of alien species). To obtain concrete results it is however necessary to debunk the beliefs mentioned at the
beginning of this study, considering that also the actions carried out in the urban areas can substantially affect on
biodiversity.

\begin{table}[!h]
\centering
\begin{tabular}{>{\raggedright\arraybackslash}p{.2\columnwidth}>{\raggedright\arraybackslash}p{.07\columnwidth}>{\raggedright\arraybackslash}p{.07\columnwidth}>{\raggedright\arraybackslash}p{.07\columnwidth}>{\raggedright\arraybackslash}p{.07\columnwidth}>{\raggedright\arraybackslash}p{.07\columnwidth}>{\raggedright\arraybackslash}p{.07\columnwidth}>{\raggedright\arraybackslash}p{.07\columnwidth}}
\toprule
& \textbf{S} & \textbf{KAI} & \textbf{Nd} & \textbf{Fd\%} & \textbf{\% nP} & \textbf{H'} & \textbf{E} \\
\toprule
1998-1999 & 28 & 125 & 5 & 17.86 & 25 & 2.24 & 0.80 \\
2011-2012 & 25 & 103.5 & 3 & 12 & 20 & 2.05 & 0.76 \\
\bottomrule
\end{tabular}
\caption{Structure of the bird population in the two considered periods. S = number of species, KAI = mean kilometric abundance index (ind/km), Nd = number of dominant species, Fd\% = \% frequency of dominant species, \% nP = \% of non-passerines species, H' = diversity index, E = equitability index}
\label{Rassati_tab_1}
\end{table}

\section*{Bibliography}
\begin{itemize}\itemsep0pt
	\item Bibby C.J., Burgess N.D. \& Hill D.A., 2000 - \textit{Bird census tecniques.} Second Edition, Academic Press, London.
	\item \textcolor{black}{Magurran A. E., 1988 - }\textit{\textcolor{black}{Ecological diversity and its
measurement.}}\textcolor{black}{\ University Press, Cambridge.}
	\item \textcolor{black}{Shannon C.E. \& Weaver W., 1963 - }\textit{\textcolor{black}{A mathematical theory of
communication.}}\textcolor{black}{\ University of Illinois Press, Urbana.}
	\item S{\o}rensen T., 1948 - \textit{A method of establishing groups of equal amplitude in plant sociology based on similarity of species content and its application to analysis of the vegetation on Danish commons.}\ Det. Kong. Danske Vidensk. Selsk. Biol. Skr., 5: 1-34.
\end{itemize}
\end{otherlanguage}
\setcounter{figure}{0}
\setcounter{table}{0}

\begin{adjustwidth}{-3.5cm}{0cm}
\pagestyle{CIOpage}
\authortoc{\textsc{Sartirana F.}, \textsc{Valfiorito R.}}
\chapter*[\textit{Status} e distribuzione
dell{\textquoteright}aquila reale in provincia di Imperia]{Status e distribuzione dell{\textquoteright}aquila reale \textbf{\textit{Aquila chrysaetos}}\textbf{ }\textbf{\textit{
}}\textbf{in provincia di Imperia}}
\addcontentsline{toc}{chapter}{\textit{Status} e distribuzione
dell{\textquoteright}aquila reale in provincia di Imperia}

\textsc{Fabiano Sartirana}$^{1*}$, \textsc{Rudy Valfiorito}$^2$  \\

\index{Sartirana Fabiano} \index{Valfiorito Rudy}
\noindent\color{MUSEBLUE}\rule{27cm}{2pt}
\vspace{1cm}
\end{adjustwidth}


\marginnote{\raggedright $^1$Via Don Minzoni 14/19, 18100 Imperia\\
$^2$Corso Verbone 199, 18036 Soldano (IM) \\
\vspace{.5cm}
{\emph{\small $^*$Autore per la corrispondenza: \href{mailto:fabianosartirana@libero.it}{fa\allowbreak bia\allowbreak no\allowbreak sar\allowbreak ti\allowbreak ra\allowbreak na@\allowbreak li\allowbreak be\allowbreak ro.\allowbreak it}}} \\
\keywords{\textit{Aquila chrysaetos}, coppie territoriali,
densit\`a, sito riproduttivo, provincia di Imperia}
{\textit{Aquila chrysaetos}, territorial pairs,
density, breeding site, province of Imperia.}
}
%\index{keywords}{Densit\`a} \index{keywords}{Sito riproduttivo} \index{keywords}{Provincia di Imperia}
{\small
\noindent \textsc{\color{MUSEBLUE} Summary} / The most southern alpine 
\textit{Aquila chrysaetos} population, located in the province of Imperia (Liguria, Italy), includes 8 territorial pairs (4.6 pairs/1000 km\textsuperscript{2}). Between the 12 known
eyries, the lowest is located at 380 m a.s.l., probably the minimum
altitude recorded across the italian Alps. \\
}
\vspace{1cm}

Il presente studio si pone come obiettivo
l{\textquoteright}aggiornamento dello status e della distribuzione
dell{\textquoteright}aquila reale (\textit{Aquila chrysaetos}) nella
provincia di Imperia (1156 km\textsuperscript{2}), che ospita la
popolazione pi\`u meridionale delle Alpi. L{\textquoteright}area
indagata \`e caratterizzata dalla presenza di un{\textquoteright}estesa
copertura boscosa e di ambienti tipicamente mediterranei, come macchia
e gariga e da ridotte praterie alpine frammiste a estese zone rocciose
calcaree. Di conseguenza i territori di caccia comprendono una
variet\`a di habitat maggiore rispetto alle tipiche zone alpine, alla
quale \`e verosimilmente associata un{\textquoteright}alta diversit\`a
specifica di prede. Ci\`o \`e attestato anche
dall{\textquoteright}osservazione di numerosi atti di predazione, in
particolare nei confronti di camoscio, capra domestica, cinghiale,
marmotta, volpe, gatto domestico, lepre comune, fagiano comune e
starna.
Partendo da osservazioni di campagna e da dati pregressi di
nidificazione, sono state effettuate delle uscite mirate a stabilire la
distribuzione delle coppie territoriali e ad individuare i siti di
nidificazione.
La specie fino al 2013 era presente con 8 coppie territoriali (densit\`a
4.6/1000 km\textsuperscript{2}, calcolata sull{\textquoteright}intera
area occupata dai territori presunti delle coppie). In seguito a
ripetute osservazioni sul campo si \`e ipotizzata una nona coppia tra i
territori di Imperia e Sanremo (per una densit\`a a 5.1 coppie/1000
km\textsuperscript{2}). Questi valori risultano essere inferiori a
quelli delle principali aree alpine (per es. nel Parco Naturale
Dolomiti Friulane: 16.8/1000 km\textsuperscript{2}
n=10,\textsuperscript{ } Borgo\textsc{ 2009}) e superiori a quelli
appenninici (per es. nell{\textquoteright}Appennino Umbro-marchigiano:
2.9/1000 km\textsuperscript{2} n=13, Magrini \textit{et al. }2001).
Sono stati individuati 18 nidi (2.2 per coppia, Min. 1 Max. 6), situati
ad un{\textquoteright}altitudine media di 1005 m (Min. 380 m, Max. 1720
m., N=14). Un sito riproduttivo - utilizzato sia nel 2012 che nel 2013
- \`e stato localizzato ad un{\textquoteright}altitudine di 380 m, che
rappresenta probabilmente la quota pi\`u bassa per le Alpi, nonch\'e
una delle minori per l{\textquoteright}Italia.
Dei 18 nidi noti, 15 (83.3\%) ricadono in aree della Rete Natura 2000
(le ZPS sono pressoch\'e sovrapposte ai SIC), 12 (66.7\%) in aree a
divieto di caccia e 8 (44,4\%) si trovano all{\textquoteright}interno
del Parco Naturale Regionale delle Alpi Liguri. Uno solo dei nidi
considerati si trova al di fuori dei confini provinciali.

\section*{Ringraziamenti}
Si ringraziano sia per il materiale fotografico che per le osservazioni
sul campo le seguenti persone: Gian Pietro Pittaluga, Franco Bianchi e
Daniele Chianea.

\section*{Bibliografia}
\begin{itemize}\itemsep0pt
	\item Borgo A., 2009 - L{\textquoteright}Aquila reale ecologia,
biologia e curiosit\`a sulla regina del Parco Naturale delle
\textit{Dolomiti Friulane.} Parco Naturale delle Dolomiti Friulane, 192
pp.

	\item Magrini M., Perna P., Angelini J. \& Armetano L., 2001\textsc{ -
}Tendenza delle popolazioni di Aquila reale \textit{Aquila chrysaetos},
il Lanario \textit{Falco }\textit{biarmicus} e il Pellegrino\textit{
Falco peregrinus }nelle Marche e in Umbria. \textit{Avocetta, }25: 57.
\end{itemize}

\setcounter{figure}{0}
\setcounter{table}{0}

\begin{adjustwidth}{-3.5cm}{0cm}
\pagestyle{CIOpage}
\authortoc{\textsc{Sarrocco S.}, \textsc{Capizzi D.}, \textsc{Pizzol I.}, 
\textsc{Scalisi M.}}
\chapter*[Il monitoraggio delle specie della Direttiva Uccelli nel
Lazio]{\textbf{Il monitoraggio delle specie della Direttiva Uccelli
(2009/147/CE) nel Lazio: stima dei parametri di alcuni
}\textbf{\textit{taxa}}\textbf{ di interesse conservazionistico}}
\addcontentsline{toc}{chapter}{Il monitoraggio delle specie della Direttiva Uccelli nel
Lazio}

\textsc{Stefano Sarrocco}$^{1*}$, \textsc{Dario Capizzi}$^1$, \textsc{Ivana Pizzol}$^{1,2}$, 
\textsc{Marco Scalisi}$^1$  \\

\index{Sarrocco Stefano} \index{Capizzi Dario} \index{Pizzol Ivana} \index{Scalisi Marco}
\noindent\color{MUSEBLUE}\rule{27cm}{2pt}
\vspace{1cm}
\end{adjustwidth}


\marginnote{\raggedright $^1$Agenzia Regionale Parchi, Area Biodiversit\`a e
Geodiversit\`a \\
$^2$Universit\`a della Tuscia di Viterbo, DEB \\
\vspace{.5cm}
{\emph{\small $^*$Autore per la corrispondenza: \href{mailto:ssarrocco@regione.lazio.it}{s\allowbreak sar\allowbreak roc\allowbreak co@\allowbreak re\allowbreak gio\allowbreak ne.\allowbreak la\allowbreak zio.\allowbreak it}}} \\
\keywords{Lazio, Monitoraggio, Direttiva Uccelli, Modelli}
{Latium, monitoring, Birds, modeling}
%\index{keywords}{Lazio} \index{keywords}{Monitoraggio} \index{keywords}{Modelli} \index{keywords}{Direttiva Uccelli}
}
{\small
\noindent \textsc{\color{MUSEBLUE} Summary} / We modeled the regional (Latium, central Italy) suitability distribution
of 14 breeding birds of European conservation concern by Maxent
software. All the AUC values of the ROC curve obtained by Maxent are
higher than 0,8. Some species showed a wide suitability distribution
(e.g. honey buzzard, red-backed shrike), others less than 1000
km\textsuperscript{2} (e.g. rock partridge, white-backed woodpecker).\\
}

\section*{Introduzione}

Le direttive europee {\textquotedblleft}Habitat{\textquotedblright} e
{\textquotedblleft}Uccelli{\textquotedblright} prevedono il
monitoraggio dello stato di conservazione degli habitat e delle specie,
con l{\textquoteright}obiettivo del loro mantenimento in uno stato
favorevole. I principali parametri da misurare sono: \textit{range
}della specie e dell{\textquoteright}habitat, consistenza della
popolazione e variazioni nel tempo, ampiezza e qualit\`a
dell{\textquoteright}habitat. La Regione Lazio per rispondere a questi
obblighi normativi si \`e dotata di una Rete regionale di Monitoraggio
della Biodiversit\`a, organizzata in un centro regionale (\textit{Focal
Point}), alcuni centri tematici e una rete capillare di laboratori
territoriali (presso le aree naturali protette). In questo contributo
sono presentati i modelli predittivi, ottenuti con
l{\textquoteright}uso del software \textit{Maxent} (Phillips \textit{et
al}. 2006), su 14 specie di interesse comunitario nidificanti nel
Lazio: coturnice \textit{Alectoris graeca}, falco pecchiaiolo
\textit{Pernis apivorus}, nibbio bruno \textit{Milvus migrans},
biancone \textit{Circaetus gallicus}, aquila reale \textit{Aquila
chrysaetos}, succiacapre \textit{Caprimulgus europaeus}, picchio
dalmatino \textit{Dendrocopos leucotos}, calandra \textit{Melanocorypha
calandra}, calandrella \textit{Calandrella brachydactyla}, tottavilla
\textit{Lullula arborea}, calandro \textit{Anthus campestris}, balia
dal collare \textit{Ficedula albicollis}, averla piccola \textit{Lanius
collurio} e ortolano \textit{Emberiza hortulana}. Obiettivi
dell{\textquoteright}uso di questi modelli sono la definizione
dell{\textquoteright}area di distribuzione potenziale e la stima delle
consistenze delle popolazioni regionali.

\section*{Area di studio}

L{\textquoteright}area di studio coincide con il territorio della
Regione Lazio, esteso 17.207 km\textsuperscript{2}. I dati ambientali
(\textit{environmental data}) utilizzati sono costituiti da 13 strati
informativi derivanti dall{\textquoteright}uso del suolo (CUS della
Regione Lazio, 2004), oltre a 9 strati relativi ad aspetti morfologici
(DTM, pendenze, morfologia), idrografici e demografici (centri abitati
e densit\`a demografica); tutti gli strati sono stati trasformati in
distanze minime di ogni punto dalla categoria considerata, ad eccezione
degli aspetti morfologici che hanno mantenuto il valore originario. 

\section*{Metodi}

I dati di presenza delle specie (\textit{samples}) provengono dal nuovo
atlante regionale degli uccelli nidificanti (Brunelli \textit{et al}.
2011), oltre a ulteriori dati raccolti nel 2011-2013 che hanno permesso
di ottenere le densit\`a di alcune specie lungo percorsi campione
(\textit{line transect method}: Jarvinen \& Vaisanen 1976). Per un
confronto sullo stato di conservazione favorevole delle specie
analizzate si \`e fatto riferimento ai valori di FRV
(\textit{Favourable Reference Value}) riportati da Gustin \textit{et
al}. (2009). I modelli sono stati prodotti in un formato \textit{Raster
ArcGis} con celle quadrate di 100 metri di lato; questi sono stati
successivamente riclassificati in tre classi di idoneit\`a attraverso
la modalit\`a del \textit{natural break} (algoritmo di Jenks). Per il
calcolo dell{\textquoteright}area idonea di ogni specie non sono state
considerate le celle ricadenti nella classe di idoneit\`a minore. 

\section*{Risultati e discussione}

In tabella 1 sono indicati i valori dell{\textquoteright}area sotto la
curva ROC (AUC) ottenuti nell{\textquoteright}elaborazione dei modelli
di idoneit\`a: la soglia di efficienza dei modelli si attesta su valori
superiori a 0,8 che dimostra una buona capacit\`a dei modelli di
predire la distribuzione della specie. L{\textquoteright}estensione
dell{\textquoteright}area a media e alta idoneit\`a evidenzia un gruppo
di specie a distribuzione ristretta (\textit{Aquila chrysaetos},
\textit{Ficedula albicollis}, \textit{Alectoris graeca}  e
\textit{Dendrocopos leucotos}) con valori di idoneit\`a minori di 1000
km\textsuperscript{2 }(va tenuto presente che le specie a distribuzione
ristretta assumono valori di AUC maggiori rispetto a quelle a
distribuzione ampia), un secondo gruppo costituito da specie ad ampia
distribuzione. Nella tabella sono evidenziate le densit\`a e le
consistenze di alcune specie, registrate nel corso dei rilievi e
riprese dalla bibliografia pi\`u recente (Sorace \textit{et al}. 2011,
cfr. Brunelli \textit{et al}. 2011, Aradis \textit{et al}. 2012),
confrontate con i valori di FRV o le Indicazioni di Conservazione (Ind.
Cons.) proposti da Gustin \textit{et al}. (2009). In alcuni casi le
densit\`a medie regionali sono simili o superiori a quelle soglia
(\textit{Lanius collurio} e \textit{Dendrocopos leucotos}), in altri
casi inferiori (\textit{Melanocorypha calandra}, \textit{Calandrella
brachydactyla} e \textit{Lullula arborea}). Per alcuni valori di FVR la
soglia \`e riferita all{\textquoteright}intera popolazione peninsulare
o appenninica, come nel caso della Balia dal collare; per questa
specie, in questo lavoro, \`e stata effettuata una stima per
estrapolazione della consistenza della popolazione: considerando la
sola area ad alta idoneit\`a e la densit\`a minima registrata nella
regione si ottiene una stima di oltre 1700 coppie, ben oltre il 50\% di
quella indicata per l{\textquoteright}intera Italia peninsulare.


\begin{table}[!h]
\centering
%\rowcolors{2}{white!60!MUSEBLUE}{white}
\newcolumntype{S}{>{\raggedleft\arraybackslash}p{.1\columnwidth}}
\scalebox{.7}{
\begin{tabular}{>{\raggedright\arraybackslash}p{.2\columnwidth}SSSSSSSSS}
\toprule
\textbf{Specie} & \textbf{N} & \textbf{AUC} & \textbf{Media idoneit\`a [km$^2$]} & \textbf{ Alta idoneit\`a [km$^2$]} & \textbf{ Tot. Area idoneit\`a [km$^2$]} & \textbf{Densit\`a media cp/10 [ha]} & \textbf{ Popol. Lazio coppie} & \textbf{FVR Cp. o cp/10 [ha]} & \textbf{Ind. Cons. cp/10 [ha]} \\
\toprule
\textit{Aquila chrysaetos} &  20 & 0,993 & 452,72 & 188,13 & 640,85 & - & 11 & 170$^{*}$ & - \\
\textit{Lanius collurio} &  546 & 0,805 & 8901,61 & 5735,64 & 14637,25 & 0,65 & - & 0,5 & 0,5 \\
\textit{Ficedula albicollis} &  44 & 0,99 & 463,62 & 394,74 & 858,36 & 2,46 & 1718,3 & 3000$^{*}$ & 2,5 \\
\textit{Circaetus gallicus} &  144 & 0,91 & 5421,56 & 2586,8 & 8008,36 & - & 54-82 & - & 1,0 $^{**}$ \\
\textit{Melanocorypha calandra} &  53 & 0,966 & 1359,69 & 858,02 & 2217,71 & 3,15 & - & 6 & 3,5 \\
\textit{Calandrella brachydactyla} &  45 & 0,949 & 2502,38 & 1213,15 & 3715,53 & 2,82 & - & 10 & 3,5 \\
\textit{Anthus campestris} &  172 & 0,931 & 3593,46 & 1462 & 5055,46 & 1,32 & - & 2,5 & 1,5 \\
\textit{Alectoris graeca} &  38 & 0,991 & 269,93 & 157,48 & 427,41 & 0,16 & 171-342 & - & - \\
\textit{Dendrocopos leucotos} &  269 & 0,87 & 315,46 & 236,19 & 551,65 & 0,5 & 160-210 & - & 5-6 \\
\textit{Milvus migrans} &  326 & 0,916 & 4275,44 & 1600,03 & 5875,47 & - & 77-117 & 700$^{*}$ & - \\
\textit{Emberiza hortulana} &  32 & 0,979 & 2646,45 & 1012,16 & 3658,61 & - & - & - & 8 \\
\textit{Pernis apivorus} &  24 & 0,995 & 6644,65 & 3951,19 & 10595,84 & 2,00 $^{**}$ & - & - & 0,2 $^{**}$ \\
\textit{ Caprimulgus europaeus } &  116 & 0,94 & 5353,04 & 1582,33 & 6935,37 & - & - & 5,0 $^{**}$ & - \\
\textit{ Lullula arborea } &  199 & 0,91 & 4542,39 & 1755,57 & 6297,96 & 0,58 & - & 3 & 1 \\
\bottomrule
\hiderowcolors
\multicolumn{10}{l}{}\\
\multicolumn{10}{l}{ $^{**}$ =  valori densit\`a espressi in coppia/100km$^2$}\\
\multicolumn{10}{l}{$^*$ = popolazione Italia peninsulare} \\
\end{tabular}
}
\caption{Numero di dati, valori di AUC, dimensione delle aree a media e alta idoneit\`a, densit\`a media e consistenza della popolazione delle specie analizzate nel Lazio. Sono riportati i valori di FVR e le Indicazioni di Conservazione proposti da Gustin \textit{et al}. (2009)}
\label{Sarrocco_tab_1}
\end{table}

\newpage
\section*{Bibliografia}
\begin{itemize}\itemsep0pt
 \item Aradis A., Sarrocco S. \& Brunelli M. 2012 - Analisi dello
status e della distribuzione dei rapaci diurni nidificanti nel Lazio.
Quaderni Natura e Biodiversit\`a 2/2012 ISPRA, ARP Lazio, 139 pp.

 \item Brunelli M., Sarrocco S., Corbi F., Sorace A., Boano A., De Felici S.,
Guerrieri G., Meschini A. \& Roma S. (a cura di), 2011 - Nuovo
Atlante degli Uccelli Nidificanti nel Lazio. \textit{Edizioni ARP (Agenzia
Regionale Parchi)}, Roma, 464 pp.

 \item Gustin M., Brambilla M. \& Celada C. (a cura di), 2009 - Valutazione
dello Stato di Conservazione dell{\textquoteright}avifauna italiana.
Ministero dell{\textquoteright}Ambiente e della Tutela del Territorio e
del Mare, Lega Italiana Protezione Uccelli (LIPU), 1153 pp.

 \item Jarvinen O. \& Vaisanen R.A., 1976 - Finnish Line Transect Censuses.
\textit{Ornis Fennica}, 53: 115-118.

 \item Phillips S. J., Anderson R. P. \& Schapire R. E., 2006 - Maximum entropy
modeling of species geographic distributions. \textit{Ecological
Modelling}, Vol 190 (3-4): 231-259.

 \item Sorace A., Properzi S., Guglielmi S., Riga F., Trocchi V. \& Scalisi M.,
2011 - La Coturnice nel Lazio: status e piano
d{\textquoteright}azione. \textit{Edizione ARP}, Roma; 80 pp.
\end{itemize}

\setcounter{figure}{0}
\setcounter{table}{0}

\begin{adjustwidth}{-3.5cm}{0cm}
\pagestyle{CIOpage}
\authortoc{\textsc{Sighele M.}, \textsc{Lerco R.}}
\chapter*[Zigolo delle nevi nel Parco della Lessinia e innevamento
al suolo]{Presenza invernale dello zigolo delle nevi
\textbf{\textit{Plectrophenax nivalis}}\textbf{ nel Parco della
Lessinia (Verona, Veneto): relazione con l{\textquoteright}innevamento
al suolo}}
\addcontentsline{toc}{chapter}{Zigolo delle nevi nel Parco della Lessinia e innevamento
al suolo}

\textsc{Maurizio Sighele}$^{1*}$, \textsc{Roberto Lerco}$^1$  \\

\index{Sighele Maurizio} \index{Lerco Roberto}
\noindent\color{MUSEBLUE}\rule{27cm}{2pt}
\vspace{1cm}
\end{adjustwidth}


\marginnote{\raggedright $^1$Associazione Verona Birdwatching, Via Lungolor\`i,
5/A, 37127 Verona \\
\vspace{.5cm}
{\emph{\small $^*$Autore per la corrispondenza: \href{mailto:info@veronabirdwatching.org}{in\allowbreak fo@ve\allowbreak ro\allowbreak na\allowbreak bir\allowbreak dwatch\allowbreak ing.\allowbreak org}}} \\
\keywords{\textit{Plectrophenax nivalis}, Lessinia, Verona, Veneto, svernamento, neve}
{\textit{Plectrophenax nivalis}, Lessinia, Verona,
Veneto, wintering}
}
{\small
\noindent \textsc{\color{MUSEBLUE} Summary} / Snow bunting \textit{Plectrophenax
nivalis}\textcolor[rgb]{0.14901961,0.14901961,0.14901961}{ has only
been a regular visitor to the Natural Regional Park of Lessinia
(Verona, Veneto) for the last eight winters}. Authors compared data of
bunting sightings to the presence of snow on the ground, suggesting
that snow buntings have never been observed in absence of snow. No
relation was found between number of buntings and amount of snow on the
ground.\\
}
\vspace{1cm}



Nel XX secolo lo zigolo delle nevi (\textit{Plectrophenax nivalis}) era
segnalato assai raramente nel Parco Naturale Regionale della Lessinia
(Verona, Veneto), con poche osservazioni note tra gli anni
'60 e '80 per le quali era
considerato migratore irregolare o, in modo pi\`u ottimistico, di
presenza abituale (De Franceschi 1993; Sauro 1980). In realt\`a fino al
2001 si possono contare rare e sporadiche segnalazioni e non sono noti
periodi di svernamento. A partire dall{\textquoteright}inverno
2003/2004, invece, questo passeriforme artico \`e stato segnalato ogni
anno nel territorio del Parco, anche se le osservazioni di un numero
significativo di individui che si sono ripetute per alcune settimane
sono state riscontrate solo negli ultimi 8 inverni, cio\`e dal
2005/2006 (Sighele \& Parricelli 2007, 2009, 2010, 2011, 2012, 2013).
La ragione del maggiore numero di segnalazioni dello zigolo delle nevi
potrebbe essere correlata alla maggiore diffusione delle informazioni,
cartacee o telematiche, e all{\textquoteright}individuazione di una
particolare localit\`a, Bocca di Selva, dove \`e pi\`u facile
incontrare la specie negli inverni pi\`u rigidi, con conseguente
aumento di visitatori e di \textit{feedback}. In questi anni \`e
sembrato che la presenza degli zigoli delle nevi in Lessinia fosse
legata all{\textquoteright}innevamento al suolo, ma questa ipotesi,
avanzata anche 150 anni orsono (Perini 1858; De Betta 1863), fino ad
ora non era mai stata suffragata da dati certi. Sono stati pertanto
analizzati e confrontati i dati relativi alla presenza dello zigolo
delle nevi e dell{\textquoteright}innevamento in Lessinia nei 6 inverni
tra il 2007/2008 e il 2012/2013. In questo periodo lo zigolo delle nevi
\`e stato osservato ogni inverno, prevalentemente tra la prima decade
di dicembre e la seconda decade di marzo; la data pi\`u precoce
rilevata \`e il 3 dicembre (2012) e quella pi\`u tardiva il 12 aprile
(2013). La media degli individui contati in questi 6 inverni \`e
superiore a 20 indd. (20,3), con un conteggio massimo di 38 indd. nel
marzo 2010. Confrontando le registrazioni dei periodi di osservazione
di questa specie con la presenza di innevamento al suolo si osserva che
gli zigoli non sono mai stati segnalati in caso di mancanza di neve
(Fig. 1). Nell{\textquoteright}unico inverno con poche precipitazioni
considerato da questa ricerca (2011/2012), le osservazioni della specie
sono coincise col periodo di presenza di neve al suolo. Quando il
periodo di innevamento si \`e prolungato per tutto
l{\textquoteright}inverno, da dicembre a marzo, la durata delle
osservazioni \`e solitamente risultata limitata a 6-10 settimane, con
l{\textquoteright}eccezione del 2012/2013, quando gli zigoli sono stati
presenti tra dicembre e aprile. Non si sono invece evidenziati aspetti
significativi legati all{\textquoteright}entit\`a
dell{\textquoteright}innevamento. I risultati di questa analisi,
pertanto, indicano che la presenza dello zigolo delle nevi nel Parco
della Lessinia \`e strettamente legata alla presenza di neve al suolo,
seppur non alla sua entit\`a; vi \`e quindi una maggior difficolt\`a
nell{\textquoteright}osservare questa specie quando le precipitazioni
nevose sono assenti.

\begin{figure}[!h]
\centering
\includegraphics[width=.8\columnwidth]{Sighele_fig_1.png}
\caption{Zigoli delle nevi e innevamento nel Parco della Lessinia}
\label{Sighele_fig_1}
\end{figure}

\section*{Ringraziamenti}
Si ringrazia il Corpo Forestale dello Stato, Comando di Verona, per la
concessione dei dati relativi all{\textquoteright}innevamento al suolo.

\section*{Bibliografia}

\begin{itemize}\itemsep0pt
	\item Arrigoni Degli Oddi E., 1899 - Note ornitologiche della provincia di
Verona. Atti Soc. ital. Sc. Nat., 38 (1/2): 75-190.

	\item De Betta E., 1863 - Materiali per una fauna veronese. Tip.
Vicentini e Franchini, 144 pp.

	\item {De Franceschi P., 1993 - Parco Naturale Regionale della Lessinia - Piano ambientale: Zoologia
-- Vertebrati. {Regione del Veneto e Comunit\`a Montana della Lessinia.}}

	\item Perini G., 1858 - Uccelli Veronesi. Tip. Vicentini, 320 pp.

	\item Sauro E., 1980 - \textcolor{black}{Alcune variazioni della flora e della
fauna lessinica dal 1950 ad oggi. }La Lessinia - Ieri oggi
domani - Quaderno culturale: 25-26.

	\item Sighele M. \& Parricelli P., 2007 - Resoconti ornitologici del Parco
della Lessinia. Anno 2006. Parco Naturale della Lessinia \&
Verona Birdwatching, 24 pp.

	\item Sighele M. \& Parricelli P., 2009 - Resoconti ornitologici del Parco
della Lessinia. Anno 2008. Parco Naturale della Lessinia \&
Verona Birdwatching, 32 pp.

	\item Sighele M. \& Parricelli P., 2010 - Resoconti ornitologici del Parco
della Lessinia. Anno 2009. Parco Naturale della Lessinia \&
Verona Birdwatching, 32 pp.

	\item Sighele M. \& Parricelli P., 2011 - Resoconti ornitologici del Parco
della Lessinia. Anno 2010. Parco Naturale della Lessinia \&
Verona Birdwatching, 32 pp.

	\item Sighele M. \& Parricelli P., 2012 - Resoconti ornitologici del Parco
della Lessinia. Anno 2011. Parco Naturale della Lessinia \&
Verona Birdwatching, 32 pp.

	\item Sighele M. \& Parricelli P., 2013 - Resoconti ornitologici del Parco
della Lessinia. Anno 2012. Parco Naturale della Lessinia \&
Verona Birdwatching, 32 pp.
\end{itemize}

\setcounter{figure}{0}
\setcounter{table}{0}

\begin{adjustwidth}{-3.5cm}{0cm}
\pagestyle{CIOpage}
\authortoc{\textsc{Sorace A.}, \textsc{Corradini A.}, \textsc{Dematris P.}, \textsc{De Zuliani E.}, \textsc{Mazzarani D.}, \textsc{Monaco E.}, \textsc{Muratore S.}, \textsc{Piroli R.}}
\chapter*[]{\bfseries
L{\textquoteright}avifauna nidificante nella Riserva naturale Regionale di Macchiatonda}
\addcontentsline{toc}{chapter}{L{\textquoteright}avifauna nidificante nella Riserva naturale Regionale di Macchiatonda}

\textsc{Alberto Sorace}$^{1*}$, \textsc{Augusto Corradini}$^2$, \textsc{Patrizio Dematris}$^2$, \textsc{Emanuele De Zuliani}$^2$, \textsc{Donatella Mazzarani}$^2$, \textsc{Ernesto Monaco}$^{1, 2}$, \textsc{Sergio Muratore}$^{1, 2}$, \textsc{Riccardo Piroli}$^2$ \\

\index{Sorace Alberto} \index{Corradi Augusto} \index{Demartis Patrizio} \index{De Zuliani Emanuele} \index{Mazzarani Donatella} \index{Monaco Ernesto} \index{Muratore Sergio} \index{Piroli Riccardo}
\noindent\color{MUSEBLUE}\rule{27cm}{2pt}
\vspace{1cm}
\end{adjustwidth}

\marginnote{\raggedright $^1$SROPU, Via Roberto Crippa 60 D/8 00125 Roma (Acilia) \\
$^2$Riserva naturale Regionale di Macchiatonda, via del Castello 40 00050 Santa Severa (Roma) \\
\vspace{.5cm}
{\emph{\small $^*$Autore per la corrispondenza: \href{mailto:sorace@fastwebnet.it}{so\allowbreak ra\allowbreak ce@\allowbreak fast\allowbreak web\allowbreak net.\allowbreak it}}} \\
\keywords{Riserva naturale Regionale di Macchiatonda, Roma, avifauna nidificante}
{Macchiatonda Natural Regional Reserve, Roma, breeding species}
}

{\small
\noindent \textsc{\color{MUSEBLUE} Summary} / The territories of breeding birds were mapped during spring 2013, in Macchiatonda Natural Reserve. On the whole, 34
breeding species were observed. \textit{Emberiza calandra} (1,68 cp/10 ha) and \textit{Cisticola juncidis} (1,39 cp/10
ha) were the most abundant. Other 9 species breed near the Reserve and use this area for feeding. The richness of
species was highest in agricultural areas, which also host most of species of conservation concern.\\
}

\vspace{1cm}
La Riserva Naturale regionale di Macchiatonda è un{\textquoteright}area umida residuale del litorale tirrenico, 50 km a nord di Roma, in
un territorio pianeggiante che si sviluppa tra il mare e i monti della Tolfa. L'area comprende circa 180 ha di campi
coltivati estensivamente e 70 ha di lagune costiere, laureto e prati alofili. La presenza di questi particolari habitat
unita a una ricca comunità ornitica acquatica e alla presenza di specie ornitiche parasteppiche nei coltivi ha promosso l{\textquoteright}area a SIC e ZPS (\textit{IT6030019}).

Nella primavera 2013 è stata avviata una ricerca per approfondire le conoscenze sull{\textquoteright}avifauna nidificante. Lungo una serie di percorsi, ripetuti per quattro volte da aprile a giugno, è stata mappata la presenza delle diverse specie ornitiche all{\textquoteright}interno dell{\textquoteright}area protetta.

Sono state rilevate 34 specie nidificanti possibili, probabili o certe: \textit{Cygnus olor}, \textit{Anas
platyrhynchos}, \textit{Coturnix coturnix},\textit{ Phasianus colchicus}, \textit{Ardea purpurea},\textit{ Tachybaptus
ruficollis}, \textit{Gallinula chloropus}, \textit{Fulica atra}, \textit{Columba livia} dom., \textit{Streptopelia
decaocto}, \textit{Streptopelia turtur}, \textit{Upupa epops}, \textit{Melanocorypha calandra}, \textit{Galerida
cristata}, \textit{Alauda arvensis}, \textit{Hirundo rustica}, \textit{Luscinia megarhynchos}, \textit{Turdus merula},
\textit{Cettia cetti}, \textit{Cisticola juncidis}, \textit{Acrocephalus scirpaceus}, \textit{Hippolais polyglotta},
\textit{Sylvia melanocephala}, \textit{Sylvia cantillans}, \textit{Sylvia atricapilla}, \textit{Parus major},
\textit{Corvus cornix}, \textit{Pica pica}, \textit{Corvus monedula}, \textit{Sturnus vulgaris}, \textit{Passer
italiae}, \textit{Passer montanus}, \textit{Carduelis carduelis}, \textit{Emberiza calandra}. 

Altre 9 specie nidificano in aree vicine e frequentano la Riserva in periodo riproduttivo per scopi trofici:
\textit{Falco tinnunculus} e \textit{Merops apiaster} (in passato nidificanti nella Riserva), \textit{Falco
peregrinus}, \textit{Columba palumbus,} \textit{Apus apus}, \textit{Delichon urbicum}, \textit{Cyanistes caeruleus,}
\textit{Serinus serinus} \ e \textit{Carduelis chloris}. Nell{\textquoteright}area protetta sono risultate dominanti le seguenti
specie: \textit{Emberiza calandra} (1,68 cp/10 ha; pi = 0,224), \textit{Cisticola juncidis} (1,39 cp/10 ha; pi =
0,186), \textit{Passer italiae} (0,66 cp/10 ha; pi = 0,087), \textit{Luscinia megarhynchos} (0,57 cp/10 ha; pi =
0,077), \textit{Cettia cetti} (0,53 cp/10 ha; pi = 0,071) e \textit{Galerida cristata} (0,49 cp/10 ha; pi = 0,066).

Come atteso, la ricchezza di specie raggiunge il suo valore più elevato (23 specie) nella zona agricola, essendo
l{\textquoteright}estensione di questi ambienti maggiore rispetto agli altri. Nella zona degli stagni retrodunali sono state tuttavia
osservate le più alte densità di coppie nidificanti (15,2 cp/10 ha) e la maggiore percentuale di non Passeriformi
(38,2\%), a conferma del loro valore conservazionistico ed ecologico. 

A parte \textit{Ardea purpurea}, le altre 11 specie a priorità di conservazione (All. I di Dir. 2009/147/CE; BirdLife
International 2004, Peronace et al. 2012) presenti nella Riserva nidificano in ambienti agricoli, anche di elevato
pregio conservazionistico. Tra le specie a priorità di conservazione spicca la presenza di \textit{Melanocorypha
calandra} (0,16 cp/10 ha; pi = 0,022), specie inserita in All. 1 della Dir. 2009/147/CE, estremamente localizzata nel
Lazio.

\section*{Bibliografia}
\begin{itemize}\itemsep0pt
	\item BirdLife International, 2004. \textit{Birds in Europe: population estimates, trends and conservation status. BirdLife International}. (BirdLife Conservation Series No.12), Cambridge.
	\item Peronace V., Cecere J.G., Gustin M. \& Rondinini C., 2012. Lista Rossa 2011 degli Uccelli nidificanti in Italia. \textit{Avocetta} 36: 11-58.
\end{itemize}
\setcounter{figure}{0}
\setcounter{table}{0}

\begin{adjustwidth}{-3.5cm}{0cm}
\pagestyle{CIOpage}
\authortoc{\textsc{Trotta M.}}
\chapter*[Successo alimentare del chiurlo maggiore in periodo invernale]{Ulteriori dati sul successo alimentare del chiurlo maggiore \textbf{\textit{Numenius arquata}}\textbf{ in periodo invernale nel Parco Nazionale del Circeo (Lazio, Italia Centrale)}}
\addcontentsline{toc}{chapter}{Successo alimentare del chiurlo maggiore in periodo invernale}

\textsc{Marco Trotta}$^{1*}$ \\

\index{Trotta Marco}
\noindent\color{MUSEBLUE}\rule{27cm}{2pt}
\vspace{1cm}
\end{adjustwidth}


\marginnote{\raggedright $^1$SROPU -- Stazione Romana Osservazione e Protezione Uccelli, Via Britannia 36, 00183 Roma, Italia \\
\vspace{.5cm}
{\emph{\small $^*$Autore per la corrispondenza: \href{mailto:marcotrot@gmail.com}{marcotrot@gmail.com}}} \\
\keywords{Parco Nazionale del Circeo, Lazio, \textit{Numenius arquata}, successo alimentare}{Circeo National Park, Latium, \textit{Numenius arquata}}
}
{\small
\noindent \textsc{\color{MUSEBLUE} Summary} / The foraging success of the Eurasian curlew \textit{Numenius
arquata} in winter was studied in a coastal region of southern Latium. Comparing the data collected in the period 1997-98/2000-01 with the results of this study, an increase of the foraging attempts in contrast
with a decrease in the feeding success was found. This result is probably caused by a decrease of the density of earthworms in the feeding areas.\\
}
\vspace{1cm}

Nel periodo 1/12/12-28/2/13 \`e stata condotta
un{\textquoteright}indagine sul successo di foraggiamento del chiurlo
maggiore \textit{Numenius arquata} in un sito di svernamento del Lazio
meridionale. I rilievi sono stati effettuati in alcune aree adibite al
pascolo situate all{\textquoteright}interno del P.N. del Circeo ed
utilizzate dalla specie per alimentarsi. Sono stati eseguiti, su
soggetti in attivit\`a di foraggiamento, 66 campionamenti della durata
di 3 minuti ciascuno. I dati raccolti sono stati confrontati con i
risultati registrati nelle medesime aree di alimentazione in quattro
stagioni invernali, nel periodo 1997-98/2000-01 (Trotta 2008). Il
chiurlo maggiore ha effettuato 9.2 tentativi/minuto, il successo di
foraggiamento \`e stato di 1.14 prede/minuto. Su 227 prede, il 93\% \`e
rappresentato da Artropodi e il 7\% da lombrichi.
{Sulle coste tedesche del mar Baltico, in zone esposte
a marea, Rippe \& Dierschke (1997) registrano un successo di cattura
fino a tre volte maggiore,}\textcolor{red}{ }{sebbene
le specie predate siano differenti;}\textcolor{red}{
}{questo risultato \`e determinato dalla facilit\`a
con cui i chiurli riescono a individuare il polichete
}\textit{{Nereis diversicolor}}{
nelle velme }{coperte da acqua a profondit\`a compresa
tra 3 e 15 cm.}{ }Nel presente studio, il successo di
foraggiamento \`e pi\`u basso di quello rilevato da Berg (1993) in
ambienti prativi della Svezia centrale e risulta inferiore anche alla
media registrata nel P.N. del Circeo nel quadriennio 1997-98/2000-01
(Tab. \ref{Trotta_tab_1}), sebbene la frequenza di tentativi sia maggiore. Una chiave di
lettura potrebbe essere rappresentata dal marcato decremento di
lombrichi catturati, solo il 7\% in questa indagine rispetto al 18\%
registrato nelle stesse aree di foraggiamento nel 1997-98/2000-01. Il
decremento \`e ancora pi\`u evidente se si confrontano i dati raccolti
nel 1998-99 in una indagine sui fattori di disturbo nelle aree di
foraggiamento; su un campione di 70 prede i lombrichi rappresentano
infatti il 42,7\% del totale (Trotta 2003). Nella dieta del chiurlo
maggiore, gli oligocheti della famiglia \textit{Lumbricidae} occupano
una porzione rilevante in termini di biomassa e rappresentano una
risorsa di elevato valore nutrizionale (Camus \textit{et al.} 2001).
L{\textquoteright}impiego di alcuni fertilizzanti e il ricorso a
profonde lavorazioni annuali dei terreni sono tra gli interventi che
possono causare una forte riduzione dei lombrichi (Chiarini \& Conte
2010). Densit\`a basse di questa preda potrebbero non consentire ai
chiurli di soddisfare il fabbisogno energetico giornaliero, con gravi
conseguenze sulla sopravvivenza degli individui; nei mesi invernali,
infatti, \`e di fondamentale importanza per gli uccelli avere
un{\textquoteright}adeguata disponibilit\`a trofica che permetta di
accumulare le provviste energetiche per fronteggiare le condizioni
meteorologiche avverse (Owen 1980; Van Gils \textit{et al.} 2006). Con
future ricerche sarebbe interessante indagare il comportamento del
chiurlo maggiore nelle ore successive al tramonto, in modo da
comprendere se il modesto successo alimentare registrato nel P.N. del
Circeo sia compensato da un{\textquoteright}intensa attivit\`a trofica
notturna.

\begin{table}[!h]
\centering
\small
\begin{tabular}{>{\raggedright\arraybackslash}p{.2\columnwidth}>{\raggedright\arraybackslash}p{.15\columnwidth}>{\raggedright\arraybackslash}p{.15\columnwidth}>{\raggedright\arraybackslash}p{.15\columnwidth}>{\raggedright\arraybackslash}p{.15\columnwidth}}
\toprule
& \textbf{Successo/\allowbreak minuto} & \textbf{Tentativi/\allowbreak minuto} & \textbf{Lombrichi/\allowbreak minuto} & \textbf{Minuti di campionamento} \\
\toprule
2012-13	& 1.14 & 9.2 & 0.08 & 198 \\
\midrule
1997-98/2000-01	& 1.41 & 7.3 & 0.26 & 298 \\
\bottomrule
\end{tabular}
\caption{Tentativi di cattura e successo alimentare del chiurlo maggiore nel P.N. del Circeo; sono riportati in tabella, per comparazione, anche i dati raccolti nel quadriennio 1997-98/2000-01}
\label{Trotta_tab_1}
\end{table}

\section*{Bibliografia}
\begin{itemize}\itemsep0pt
	\item Berg \r{A}., 1993 - Food resources and foraging success of Curlews
\textit{Numenius arquata} in different farmland habitats. \textit{Ornis
Fennica}, 70: 22-31.
	\item Camus A., Bernard J.L. \& Granval P., 2001 - Bordi dei campi e
lombrichi. A cura di ANUU Migratoristi, ZENECA Sopra: 14-15.
	\item Chiarini F. \& Conte L., 2010 - Avvicendamenti, consociazioni e
fertilit\`a del suolo in agricoltura biologica. Veneto Agricoltura, 56
pp.
	\item Owen M., 1980 - Wild geese of the world: their life history and
ecology. Batsford, London, 236 pp. 
	\item Rippe H. \& Dierschke V., 1997 - Picking out the plum jobs: feeding
ecology of curlews Numenius arquata in a Baltic Sea wind flat.
\textit{Mar. Ecol. Prog. Ser.}, 159: 239-247.
	\item Trotta M., 2003 - Alimentazione del Chiurlo maggiore \textit{Numenius
arquata} in periodo invernale e analisi dei fattori di  disturbo nelle
aree di foraggiamento. \textit{Avocetta}, 27: 23.
	\item Trotta M., 2008 - Strategie di foraggiamento del Chiurlo maggiore
\textit{Numenius arquata} e differenze di successo alimentare tra sessi
in un sito di svernamento dell{\textquoteright}Italia centrale.
\textit{Avocetta}, 32: 41-46. 
	\item Van Gils J.A., Spaans B., Dekinga A. \& Piersma T.,
2006 - Foraging in a tidally structured environment
by Red Knots (Calidris canutus): ideal, but not free. \textit{{Ecol.}}{, 87 (5):
1189-202.}
\end{itemize}
\setcounter{figure}{0}
\setcounter{table}{0}

\begin{adjustwidth}{-3.5cm}{0cm}
\pagestyle{CIOpage}
\authortoc{\textsc{Trotta M.}, \textsc{Panuccio M.}, \textsc{Dell{\textquoteright}Omo G.}}
\chapter*[La dieta del gheppio in un paesaggio agricolo
dell'Italia centrale]{La dieta del gheppio \textbf{\textit{Falco
tinnunculus}}\textbf{ nidificante in un paesaggio agricolo
dell'Italia centrale}}
\addcontentsline{toc}{chapter}{La dieta del gheppio in un paesaggio agricolo
dell'Italia centrale}

\textsc{Marco Trotta}$^{1*}$, \textsc{Michele Panuccio}$^{2,3}$, \textsc{Giacomo
dell{\textquoteright}Omo}$^{3}$ \\

\index{Trotta Marco} \index{Panuccio Michele} \index{Dell'Omo Giacomo}
\noindent\color{MUSEBLUE}\rule{27cm}{2pt}
\vspace{1cm}
\end{adjustwidth}


\marginnote{\raggedright $^1$SROPU -- Stazione Romana Osservazione e Protezione
Uccelli, Via Britannia 36, 00183 Roma, Italia \\
$^2$Ente Regionale Roma Natura \\
$^3$\textit{Ornis italica}, Roma \\
\vspace{.5cm}
{\emph{\small $^*$Autore per la corrispondenza: \href{mailto:marcotrot@gmail.com}{marcotrot@gmail.com}}} \\
\keywords{\textit{Falco tinnunculus}, Italia centrale, dieta, paesaggio agricolo}
{\textit{Falco tinnunculus}, Central Italy, diet, farmland}
}
{
\small
\noindent \textsc{\color{MUSEBLUE} Summary} / In this study, we analysed the diet of breeding kestrels \textit{Falco tinnunculus} in a Mediterranean area. The 46,3\% of the biomass of
preys is represented by birds and the rest is distributed among
mammals, reptiles and insects. The results suggest a wider diet
composition of the kestrels breeding in the study area, despite of
other European populations show a predator-prey relationships with
voles and other rodents. \\
}
\vspace{1cm}

Nella stagione riproduttiva 2012 abbiamo analizzato, attraverso la
raccolta di borre e resti di prede, il regime alimentare del gheppio
\textit{Falco tinnunculus} in un contesto tipico della campagna
agricola romana, quello della Riserva Naturale di Decima-Malafede
(Lazio). La raccolta \`e stata effettuata ispezionando 20 cassette nido
posizionate su tralicci dell{\textquoteright}alta tensione. Sono state
esaminate 91 borre integre, oltre a numerosi resti alimentari, per un
totale di {228 prede. Il 46,3\% della biomassa predata
\`e rappresentato da uccelli, la restante parte \`e distribuita tra
mammiferi (27,1\%), rettili (19,1\%) e insetti (7,6\%, Tab.1).}
{L{\textquoteright}}elevata frequenza di specie
ornitiche associate {ad ambienti agricoli e la
presenza in tarda primavera di individui appena involati, quindi pi\`u
inesperti, }hanno {probabilmente determinato, nel
contesto ambientale indagato, una maggiore predazione di uccelli. Le
preferenze alimentari sono indirizzate principalmente verso i
Passeriformi di piccole dimensioni,}

{tra cui il genere }\textit{{Passer}}
\`e il pi\`u rappresentato (26,8\% degli uccelli predati).
{Tra gli insetti catturati la specie dominante \`e il
coleottero }\textit{{Pentodon
bidens}} { (60,0\%), un ruolo importante \`e rivestito
anche dagli ordini degli Odonati (8,8\%) e degli Ortotteri (8,0\%). }I
mammiferi sono rappresentati da \textit{Microtus savii} e
sporadicamente da alcuni individui di \textit{Apodemus }spp.;
l{\textquoteright}apporto trofico dei Soricomorfi \`e irrilevante
(0,4\%). {I Lacertidi costituiscono
l{\textquoteright}81,4\% dei rettili, le cui catture sono ripartite tra
}\textit{{Podarcis }}{spp. (44,4\%) e
}\textit{{Lacerta bilineata}} {
(37,0\%); tra le prede compaiono anche
}\textit{{Tarentola mauritanica
}}{(1,3\%) e }\textit{{Chalcides
chalcides }}{(0,9\%). Precedenti studi hanno
evidenziato come la dinamica delle popolazioni europee di gheppio sia
influenzata dalla densit\`a dei micro-Mammiferi (Korpim\"aki 1985;
Village 1990; Korpim\"aki \& Norrdahl 1991; Fargallo 
}\textit{{et al.}}{ 2009), che
rappresentano quindi una componente dominante della dieta. Lo spettro
alimentare rilevato dalla nostra indagine appare invece pi\`u
diversificato. Si ritiene che tale risultato sia espressione della
}notevole plasticit\`a ecologica del gheppio, in grado di sfruttare le
risorse trofiche pi\`u abbondanti all'interno del
proprio territorio di caccia (Ferguson-Lees 2001; Costantini \textit{et
al.} 2005).

\begin{table}[!h]
\centering
\begin{tabular}{>{\raggedright\arraybackslash}p{.3\columnwidth}>{\raggedleft\arraybackslash}p{.2\columnwidth}>{\raggedleft\arraybackslash}p{.2\columnwidth}}
\toprule
\textbf{Taxa} & \textbf{Tot. prede} & \textbf{Biomassa \%} \\
\toprule
\textit{Crocidura spp.}	& 1&	0.3 \\
\toprule
\hiderowcolors
\textbf{\textit{Soricomorpha}} & 1&	0.3 \\
\toprule
%\showrowcolors
\textit{Apodemus spp.}	&2	&2.2 \\
\textit{Microtus savii}&	6	&4.6 \\
\textit{Rodentia ind.}	&26&	20.0 \\
\toprule
\hiderowcolors
\textit{\textbf{Rodentia}}	&34	&26.8 \\
\toprule
\textit{\textbf{Mammalia}}	&35&	27.1 \\
\toprule
%\showrowcolors
\textit{Motacilla alba}	&2	&1.5 \\
\textit{Troglodytes troglod.}&	1&	0.3 \\
\textit{Turdus merula}	&3	&9.9 \\
\textit{Luscinia megarhynchos}&	1	&0.8 \\
\textit{Sylvia atricapilla}&	2&	1.4 \\
\textit{Sturnus vulgaris}	&3&	8.8 \\
\textit{Passer italiae}&	3&	3.3 \\
\textit{Passer spp.}	&8	&7.8 \\
\textit{Carduelis carduelis}&	5	&2.9 \\
\textit{Passeriformi ind.}& 	13&	9.5 \\
\toprule
\hiderowcolors
\textit{\textbf{Aves}}	&41	&46.3 \\
\toprule
%\showrowcolors
\textit{Lacerta bilineata}&	10	&12.8 \\
\textit{Podarcis siculus}	&3	&1.4 \\
\textit{Podarcis muralis}	&5&	1.6 \\
\textit{Podarcis spp.}&	4	&1.5 \\
\textit{Tarentola mauritanica}&	3	&0.9 \\
\textit{Chalcides chalcides}&	2	&0.9 \\
\toprule
\hiderowcolors
\textit{\textbf{Reptilia}}	&27&	19.1 \\
\toprule
%\showrowcolors
\textit{Pentodon bidens}&	75&	5.5 \\
\textit{Coccinellidae ind.}	&3	&0.1 \\
\textit{Anacridium aegyptium}&	8&	0.6 \\
\textit{Gryllotalpa gryllotalpa}&	1&	0.1 \\
\textit{Orthoptera ind.}	&1&	0.0 \\
\textit{Odonata ind.}	&11	&0.4 \\
\textit{Lepidoptera ind.}	&2&	0.0 \\
\textit{Insecta ind.}	&24&	0.9 \\
\toprule
\hiderowcolors
\textit{\textbf{Insecta}}	&125&7.6 \\
\toprule
\textbf{Totale}	&\textbf{228}&	\textbf{100.0} \\
\bottomrule
\end{tabular}
\caption{Dieta della popolazione di gheppio nidificante nella R.N.R. di Decima-Malafede}
\label{Trotta2_tab_1}
\end{table}

\section*{Ringraziamenti}

Si ringrazia Giulio Fancello e l'Ente Regionale
RomaNatura nonch\'e TERNA per l{\textquoteright}installazione e il
mantenimento delle cassette nido sui tralicci della rete elettrica e
per il continuo supporto alle attivit\`a di monitoraggio durante la
nidificazione dei gheppi.

\section*{Bibliografia}

\begin{itemize}\itemsep0pt
	\item Costantini D., Casagrande S., Di Lieto G., Fanfani, A. \&
Dell{\textquoteright}Omo G., 2005 - Consistent differences in feeding
habits between neighbouring breeding kestrels. \textit{Behaviour,} 142:
1409-1421.

	\item Fargallo J.A., Martinez-Padilla J., Vinuela J., Blanco G., Torre I,
Vergara P. \& De Neve L., 2009 -  Kestrel-Prey Dynamic in a
Mediterranean Region: The Effect of Generalist Predation and Climatic
Factors. \textit{PLoS one} 4 (2): e4311. 

	\item Ferguson-Lees J. \& Christie D.A., 2001 - \textit{Raptors of the world}.
London: Christopher Helm, 320 pp.

	\item Korpim\"aki E., 1985 - Diet of the Kestrel \textit{Falco tinnunculus} in
the breeding season. \textit{Ornis Fenn.,} 62: 130-137.

	\item Korpim\"aki E. \& Norrdahl K., 1991 - Numerical and functional responses
of kestrels, short-eared owls, and long-eared owls to vole densities.
\textit{Ecology,} 72: 814-826.

	\item Village A., 1990 - \textit{The Kestrel}. T \& AD Poyser, Londra, 352 pp.
\end{itemize}

%\input{tex/articles/NB_Velat.tex}


\newpage\null
\pagestyle{empty}
\part{Indice analitico degli autori}
\pagestyle{empty}
\begin{center}
\vspace*{\fill}
\includegraphics[width=.8\columnwidth]{osv_2.png}
\vspace*{\fill}
\end{center}
%\begin{center}
%\vspace*{\fill}
%\includegraphics[width=1\columnwidth]{osv_5.png}
%\vspace*{\fill}
%\end{center}
\pagestyle{empty}
\newpagecolor{white}\afterpage{\restorepagecolor}
\pagestyle{empty}
\printindex
\pagestyle{empty}
%\part{Indice delle parole chiave}
%\begin{center}
%\vspace*{\fill}
%\includegraphics[width=1\columnwidth]{osv_5.png}
%\vspace*{\fill}
%\end{center}
%\newpagecolor{white}\afterpage{\restorepagecolor}
%\pagestyle{CIOpage}
%\printindex[keywords]

\newpage
\pagestyle{empty}
\clearpage\null\newpage
\null\newpage
\null\newpage
\BackgroundRetroPic
\null\newpage
\end{document}