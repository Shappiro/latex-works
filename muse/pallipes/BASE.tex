\PassOptionsToPackage{usenames,dvipsnames,table,xcdraw}{xcolor}
\documentclass[11pt,a4paper,italian,twoside,openany]{memoir}

\usepackage{lscape}
\usepackage{eso-pic}
\usepackage{graphicx}
\usepackage{calc}
\usepackage{background}
%\usepackage{trebuchet}
%%%Keyboard inputs and layout
\usepackage[utf8]{inputenc}
\usepackage[italian]{babel}
\usepackage[T1]{fontenc}
\usepackage[square,sort&compress,colon,authoryear,numbers]{natbib}  %% customize citation style in text
\usepackage{subcaption}
\usepackage{ragged2e}
\usepackage{fontspec}

\usepackage{tikz}

\newcommand{\pie}[1]{%
\begin{tikzpicture}
 \draw (0,0) circle (2pt);\fill (2pt,0) arc (0:#1:2pt) -- (0,0) -- cycle;
\end{tikzpicture}%
}

%%%Tables, environments, graphics
\usepackage{float}
\usepackage{longtable}
\usepackage{tikz}
\usepackage{stfloats}
\usepackage{paralist} %Inline lists
%\usepackage{noitemsep}
\usepackage{booktabs}
\graphicspath{ {./img/} }
\usepackage{mdframed,longtable}

%%%Colors
\usepackage{color}
\definecolor{lightgray}{gray}{0.9}
\definecolor{verylightgray}{gray}{0.95}
\definecolor{veryblue}{HTML}{1b6bb0}

%%%Spacings
\usepackage[left=2cm,right=2cm,top=2cm,bottom=2cm]{geometry}

%Add pdf files 
\usepackage{pdfpages}

%Captions
\usepackage[labelfont={footnotesize,sf,bf},textfont={footnotesize,sf}]{caption}

%%%%%%%%%%%%%%%%%%%%%%%%%%%%%%%%%%%%%%%%%%%%%%
%%%%%%%%%%%%  REDEFINED COMMANDS  %%%%%%%%%%%%
%%%%%%%%%%%%%%%%%%%%%%%%%%%%%%%%%%%%%%%%%%%%%%
\newcommand { \mysize }{ \footnotesize { }}
\definecolor{grey}{gray}{0.5} % 0-nero; 1-bianco
\renewcommand{\labelitemi}{\textcolor{grey}{$\bullet$}}
\newcommand{\HRule}{\rule{\linewidth}{0.2mm}}
\newcommand{\etal}{et al. }
\newcommand{\ie}{\emph{i}.\emph{e}. }
\newcolumntype{P}[1]{>{\raggedright\arraybackslash}p{#1}}
\newcommand{\clearemptydoublepage}{\newpage{\pagestyle{empty}\cleardoublepage}}
\newsubfloat{figure}
\setcounter{secnumdepth}{3}
\setcounter{tocdepth}{3}
\newcommand\BackgroundPic{%
  \put(0,0){%
    \parbox[b][\paperheight]{\paperwidth}{%
      \vfill
      \flushleft
      \includegraphics[width=\paperwidth,height=\paperheight,%
      keepaspectratio]{bar.jpg}%
      \vfill
}}}
\setlength{\abovecaptionskip}{15pt plus 3pt minus 2pt} % Set spacing between figure and caption leaving 3 upper and 2 lower points of free adaptation

\newcommand\invisiblesection[1]{
	\refstepcounter{section}
	\addcontentsline{toc}{section}{\protect\numberline{\thesection}#1}
	\sectionmark{#1}
}
%%%%%%%%%%%%%%%%%%%%%%%%%%%%%%%%%%%%%%%%%%%%%%
%%%%%%%%  END OF REDEFINED COMMANDS  %%%%%%%%%
%%%%%%%%%%%%%%%%%%%%%%%%%%%%%%%%%%%%%%%%%%%%%%

%Links
\usepackage[pdftitle={Piano di Gestione del Gambero di fiume in Trentino},
     pdfauthor={FEM - Fondazione Edmund Mach / MUSE - Museo delle Scienze},
     colorlinks,linktocpage=true,linkcolor=RoyalBlue,urlcolor=BrickRed,citecolor=OliveGreen,bookmarks]{hyperref}
     

%%%%%%%%%%%%%%%%%%%%%%%%%%%%%%%%%%%%%%%%%%%%%%
%%%%%%%%%%%%%%%%  MEMOIR STYLE %%%%%%%%%%%%%%%
%%%%%%%%%%%%%%%%%%%%%%%%%%%%%%%%%%%%%%%%%%%%%%  
\makepagestyle{LIFEstyle} 
\nouppercaseheads
\setlength{\headwidth}{\dimexpr\textwidth}
\makerunningwidth{LIFEstyle}{\headwidth}
\makeevenhead{LIFEstyle}{
  }{}{
  } 
\makeoddhead{LIFEstyle}{
  }{}{\raisebox{25pt}{\parbox{.32\columnwidth}{\vspace{-20pt}\raggedleft\color{veryblue}\scshape\bfseries\leftmark}}} 
\makeoddfoot{LIFEstyle}{}{}{\hspace*{\fill}\Large\bfseries\thepage\hspace{-1.5cm}}
\makeevenfoot{LIFEstyle}{\hspace{-1.5cm}\Large\bfseries\thepage\hspace*{\fill}}{}{}
\copypagestyle{chapter}{plain}
\makeoddfoot{chapter}{}{}{}
\makeevenfoot{chapter}{}{}{}
\makeatletter
\makepsmarks{LIFEstyle}{%
  \createmark{chapter}{left}{nonumber}{ }{.\ }
  \createmark{section}{right}{shownumber}{ }{.\ }}
\makeatother
\chapterstyle{section}
%\renewcommand{\thechapter}{}

\author{\textsl{FEM - Fondazione Edmund Mach / MUSE - Museo delle Scienze}}
%%%%%%%%%%%%%%%%%%%%%%%%%%%%%%%%%%%%%%%%%%%%%%
%%%%%%%%%%%%%%  END MEMOIR STYLE %%%%%%%%%%%%%
%%%%%%%%%%%%%%%%%%%%%%%%%%%%%%%%%%%%%%%%%%%%%% 


%%%%%%%%%%%%%%%%%%%%%%%%%%%%%%%%%%%%%%%%%%%%%%
%%%%%%%%%%%%%  GLOSSARY SECTION  %%%%%%%%%%%%%
%%%%%%%%%%%%%%%%%%%%%%%%%%%%%%%%%%%%%%%%%%%%%%
\usepackage[acronym,toc]{glossaries}

\newacronym{utm}{UTM}{\emph{Universal Trasverse Mercator}, particolare proiezione della superficie terrestre su un piano}
\newglossaryentry{database}
{
  name={\emph{database}},
  description={Archivio di dati in cui le informazioni contenute sono organizzate tramite
  				un particolare modello logico in modo tale da consentire la gestione efficiente degli stessi,
  				e l'interfacciamento con linguaggi di interrogazione e/o \emph{software}},
  sort=database
}
\makeglossaries
%Rimuove il punto alla fine della descrizione lunga nel glossario
\renewcommand*{\glspostdescription}{}

%%%%%%%%%%%%%%%%%%%%%%%%%%%%%%%%%%%%%%%%%%%%%%
%%%%%%%%%  END OF GLOSSARY SECTION  %%%%%%%%%%
%%%%%%%%%%%%%%%%%%%%%%%%%%%%%%%%%%%%%%%%%%%%%%

\newcommand\BackgroundPicOne{
  \put(0,0){
    \parbox[b][\paperheight]{\paperwidth}{%
      \vfill
      \centering
      \includegraphics[width=\paperwidth,height=\paperheight]{1.pdf}
      \vfill
    }
  }
}
\newcommand\BackgroundPicTwo{
  \put(0,0){
    \parbox[b][\paperheight]{\paperwidth}{%
      \vfill
      \centering
      \includegraphics[width=\paperwidth,height=\paperheight]{2.pdf}
      \vfill
    }
  }
}
\newcommand\BackgroundPicTree{
  \put(0,0){
    \parbox[b][\paperheight]{\paperwidth}{%
      \vfill
      \centering
      \includegraphics[width=\paperwidth,height=\paperheight]{3.pdf}
      \vfill
    }
  }
}
\newcommand\BackgroundPicFour{
  \put(0,0){
    \parbox[b][\paperheight]{\paperwidth}{%
      \vfill
      \centering
      \includegraphics[width=\paperwidth,height=\paperheight]{4.pdf}
      \vfill
    }
  }
}

\newcommand\BackgroundPicFive{
  \put(0,0){
    \parbox[b][\paperheight]{\paperwidth}{%
      \vfill
      \centering
      \includegraphics[width=\paperwidth,height=\paperheight]{5.pdf}
      \vfill
    }
  }
}



\setlength{\footskip}{
  \paperheight-(1.2in+\voffset+\topmargin+\headheight+\headsep+\textheight)
}
\setlength{\footskip}{
  \paperheight-(1.2in+\voffset+\topmargin+\headheight+\headsep+\textheight)
}

\newcommand\DeactivateBG{\backgroundsetup{contents={}}}
\newcommand\ActivateBG{
  \backgroundsetup{
    scale=1,
    opacity=1,
    angle=0,
    color=black,
    contents={%
     \ifodd\value{page}
       \includegraphics[width=\paperwidth,height=\paperheight]{3.pdf}
      \else
       \includegraphics[width=\paperwidth,height=\paperheight]{4.pdf}
      \fi
    }
  }
}

\usepackage{enumitem} 
\setlist[enumerate]{itemsep=0mm}

\setmainfont{Trebuchet MS}
\begin{document}
\DeactivateBG
\AddToShipoutPicture*{\BackgroundPicOne}
\newgeometry{left=3cm,right=2cm,top=2cm,bottom=2cm}

\flushleftright

\thispagestyle{empty}

\begin{center}
   \vspace*{\fill}
   \includegraphics[width=\columnwidth]{gambero.jpg} \\
   \vspace*{\fill}
\end{center}

\newcommand\crule[3][black]{\textcolor{#1}{\rule{#2}{#3}}} % RECTANGLE FILLED WITH COLOR | \crule[color]{width}{height}

\newpage
\AddToShipoutPicture*{\BackgroundPicTwo}
\restoregeometry
\thispagestyle{LIFEstyle}
\flushleft
\noindent

\vspace*{\fill}

\textbf{\color{veryblue}TESTI} \\
Maria Cristina Bruno / FEM - Fondazione Edmund Mach \\
Sonia Endrizzi / FEM \& MUSE - Museo delle Scienze \\
Andrea Gandolfi / FEM \\
Heidi C. Hauffe / FEM \\
\vspace{.5cm}
\textbf{\color{veryblue}COORDINAMENTO} \\
Paolo Pedrini / MUSE \\
\vspace{.5cm}
\textbf{\color{veryblue}IMPAGINAZIONE}\\
Aaron Iemma / MUSE \\
\vspace{.5cm}
\textbf{\color{veryblue}RINGRAZIAMENTI} \\
Si ringraziano per la revisione critica del documento: Marcello Corazza (Progetto LIFE+T.E.N.); Valeria Fin e Lucio Sottovia (Servizio sviluppo sostenibile e aree protette della Provincia autonoma di Trento), il Servizio Foreste e fauna della Provincia Autonoma di Trento, Massimiliano Scalici (Università Roma III, Dipartimento di Scienze), Simone Tenan e Aaron Iemma (MUSE - Museo delle Scienze). Si ringraziano inoltre i partecipanti al tavolo dedicato al gambero organizzato in occasione della \emph{Final Conference} del LIFE+T.E.N.
\vspace{.5cm}
\href{http://www.lifeten.tn.it/actions/preliminary_actions/pagina8.html}{\includegraphics[width=.8\columnwidth]{lifeten_muse.png}} \\
\textbf{\footnotesize \emph{Pubblicazione realizzata nell'ambito dell'azione C10 "Azione dimostrativa di tutela di specie: salvaguardia delle popolazioni autoctone del gamberi di fiume", Progetto LIFE11/NAT/IT/000187 "T.E.N." (Trentino Ecological Network)} - \url{www.lifeten.tn.it} } \\
\vspace{.5cm}
\textbf{\color{veryblue}COORDINAMENTO AZIONE C10} \\
Marcello Corazza / PAT, \emph{\href{mailto:marcello.corazza@provincia.tn.it}{marcello.corazza@provincia.tn.it}} \\
\vspace{.5cm}
\textbf{\color{veryblue}COORDINAMENTO PROGETTO LIFE+T.E.N.} \\
Claudio Ferrari / PAT, \emph{\href{mailto:claudio.ferrari@provincia.tn.it}{claudio.ferrari@provincia.tn.it}} \\
\vspace{2cm}

{\footnotesize \emph{Il presente studio deriva dall'implementazione dell'Accordo di Programma 2016-2018 (del. 2406 del 20 dicembre 2016) tra PAT e FEM, per la realizzazione dell'iniziativa denominata Collaborazione con la Fondazione Edmund Mach (prot. 124726 di data 6.3.2017) alla redazione del Piano di Gestione del gambero di fiume definito nella Convenzione tra la Provincia autonoma di Trento e il Museo delle Scienze di Trento, sottoscritta in data 1 settembre 2016 (prot. n. 459014 di data 2.9.2016), approvata con determinazione del Dirigente del Servizio Sviluppo Sostenibile e Aree Protette n. 46 di data 01 Luglio 2016}}
\vspace{.2cm}

\newpage
\begin{KeepFromToc}
  \tableofcontents
\end{KeepFromToc}

\newpage
\AddToShipoutPicture*{\BackgroundPicTwo}
\pagestyle{LIFEstyle}
\chapter{Premessa}
Il piano di gestione per il gambero di fiume in Trentino è stato realizzato nell'ambito dell'azione C10 "\emph{Azione dimostrativa di tutela di specie: salvaguardia delle popolazioni autoctone di gambero di fiume}" del Progetto LIFE+TEN (Trentino Ecological Network, LIFE11NAT/IT/000187, \url{www.lifeten.tn.it}), cofinanziato dall'Unione Europea e coordinato dal Servizio Sviluppo Sostenibile e Aree Protette della Provincia autonoma di Trento in partnership con il Museo delle Scienze di Trento (MUSE) avviato nel 2012 e concluso nel 2017. L'esperienza, maturata nel corso dei cinque anni di svolgimento del progetto, ha messo in luce la necessità di adottare misure urgenti per la conservazione della specie sul territorio provinciale. Ogni anno si assiste infatti alla scomparsa di popolazioni che, già segregate in aree limitate, non riescono a far fronte a eventi catastrofici di origine naturale o antropica, al degrado ambientale e alla diffusione della cosiddetta "peste del gambero", causata dall'oomicete parassita \emph{Aphanomyces astaci} (Shikora 1906). Il presente piano di gestione ha quindi l'obiettivo di fornire le indicazioni necessarie per l'attuazione di interventi utili ad affrontare l'emergenza estintiva e a gestire correttamente le popolazioni e i loro habitat garantendone così la conservazione nel tempo. 
Nello specifico il piano intende fornire:
\begin{itemize}

  \item informazioni sulle caratteristiche della specie (biologia ed ecologia), sulle conoscenze acquisite attraverso studi recenti effettuati Trentino (distribuzione, stato di conservazione, variabilità genetica delle popolazioni native e principali minacce, oltre alla distribuzione di popolazioni alloctone) e sulla normativa vigente in materia di specie minacciate e alloctone;
  \item metodi per il monitoraggio e lo studio delle popolazioni e dei loro habitat;
  \item un piano d'azione per la realizzazione e la gestione di allevamenti, di progetti di ripopolamento/reintroduzione e di creazione/riqualificazione di habitat;
  \item metodi per l'identificazione, la prevenzione e la mitigazione dei fattori di minaccia;
  \item modalità di organizzazione dei dati e delle informazioni;
  \item proposte per l'attuazione di azioni divulgative;
  \item priorità d'intervento e organizzazione delle azioni d'intervento.
\end{itemize}

\chapter{Introduzione}
\AddToShipoutPicture*{\BackgroundPicTwo}
\label{chap_intro}
Il gambero d'acqua dolce ha un ruolo chiave nel mantenimento dell'equilibrio degli ecosistemi acquatici essendo in grado di influenzare la densità e la distribuzione degli altri organismi, attraverso il suo comportamento onnivoro e bioturbatore, oltre a rappresentare una fonte alimentare importante per pesci, uccelli e mammiferi. Questi crostacei costituiscono inoltre da sempre una risorsa alimentare importante per le economie locali di molti Paesi in tutto il mondo. Nel corso dell'ultimo secolo le specie europee hanno però subito una forte regressione dovuta principalmente al degrado e alla scomparsa degli habitat acquatici oltre che all'introduzione di specie alloctone responsabili della diffusione della peste del gambero. Le specie alloctone, maggiormente feconde e resistenti ai disturbi ambientali, sono state introdotte in alcuni Paesi europei per motivi commerciali, diffondendosi poi rapidamente in tutto il continente. Il territorio europeo ospita così attualmente 5 specie native riconducibili ai generi \emph{Astacus} e \emph{Austropotamobius}, e undici specie aliene appartenenti ai generi \emph{Procambarus}, \emph{Pacifastacus}, \emph{Orconectes}, e \emph{Cherax}, introdotti da Stati Uniti d'America e Australia. In Italia sono presenti tre specie autoctone, \emph{Austropotamobius pallipes} (Lereboullet, 1858), diffusa in tutte le regioni ad eccezione di Sicilia e piccole isole, \emph{Astacus astacus} (Linnaeus 1758) e \emph{Austropotamobius torrentium} (Schrank 1803) presenti con piccole popolazioni nelle regioni nord-orientali \cite{De Luise 2006} \cite{Fureder 2007} \cite{Aquiloni 2010}, e sei specie alloctone: \emph{Astacus leptodactylus} (Eschscholtz 1823), \emph{Orconectes limosus} (Rafinesque 1817) e \emph{Pacifastacus leniusculus} (Dana 1852), presenti in alcune regioni dell'Italia centro-settentrionale, \emph{Procambarus clarkii} (Girard 1852), ampiamente diffuso in quasi tutte le regioni \cite{Kouba 2014} e \emph{Cherax destructor} (Clark 1936), presente con una popolazione stabile in Lazio \cite{Scalici 2009}, mentre la specie \emph{Procambarus fallax} (Hagen 1870) è stata segnalata solo di recente in Toscana \cite{Nonnis Marzano 2009}. In Trentino è attualmente segnalata la presenza della specie nativa \emph{A. pallipes} e di due specie aliene invasive, \emph{O. limosus} e \emph{P. clarkii}. Le specie alloctone sono responsabili di gravi squilibri dei sistemi ambientali con conseguenze che si riflettono anche sull'uomo. La maggior fecondità, che caratterizza queste specie, comporta infatti lo sviluppo di popolazioni più numerose, con un conseguente maggior consumo di risorse rispetto alle specie autoctone, determinando il declino delle comunità di macrofite, di invertebrati e vertebrati legati all'ambiente acquatico. L'intensa attività di scavo esercitata da queste specie per la costruzione delle tane, a differenza delle specie autoctone che sfruttano rifugi naturali creati dalle irregolarità del substrato, determina inoltre un'intensa azione di bioturbazione. Questi effetti ambientali negativi si traducono, nella maggior parte dei casi, in gravi danni economici per la distruzione delle colture, la diminuzione del pescato e la destabilizzazione delle sponde \cite{Faller 2016}.
\section{Il gambero di fiume \emph{Austropotamobius pallipes}}
\subsection{Sistematica}
Phylum: \emph{Arthropoda} \\
Subphylum: \emph{Crustacea} \\
Classe: \emph{Malacostraca} \\
Ordine: \emph{Decapoda} \\
Famiglia: \emph{Astacidae} \\
Genere: \emph{Austropotamobius} \\
Specie: \emph{Austropotamobius pallipes} \\

La tassonomia del gambero di fiume ha subìto numerose revisioni nel corso degli ultimi settant'anni ed è ancora oggi oggetto di dibattito nella comunità scientifica. I primi metodi di classificazione, fondati esclusivamente su caratteri morfologici, sono stati nel tempo integrati e quindi sostituiti da quelli basati su tecniche molecolari sempre più evolute, senza però fare chiarezza sulla reale struttura filogenetica della specie. Nonostante le analisi genetiche abbiano evidenziato l'esistenza di numerose linee evolutive, rimane aperta la discussione sul significato tassonomico di tale variabilità. Il numero di specie, sottospecie e varietà proposto da diversi autori dipendono infatti dalle tecniche di analisi utilizzate, dall'interpretazione filogenetica e dal concetto di sottospecie adottato da ciascuno di essi. 

\ActivateBG

Le prime indagini molecolari condotte con marcatori allozimatici \cite{Santucci 1997} \cite{Lortscher 1998} e mitocondriali \cite{Grandjean 2000} \cite{Fratini 2005} hanno evidenziato la presenza di due gruppi principali caratterizzati da una marcata diversità genetica riconducibili alle specie \emph{A. pallipes} e \emph{A. italicus} (Faxon 1914), già proposte dalla classificazione su base morfologica di Karaman \cite{Karaman 1962} (Tabella \ref{tab_1}). \emph{Austropotamobius italicus}, secondo alcuni autori, sarebbe caratterizzato dalla presenza di quattro sottospecie: \emph{A. i. italicus} (Faxon 1914) distribuita in Italia sull'Appennino tosco-emiliano, in Spagna e nel sud della Francia, \emph{A. i. carsicus} \cite{Karaman 1962} in Italia nord orientale, Francia e Alpi Dinariche, \emph{A. i. carinthiacus} (Albrecht 1961) in Italia centrale e nord-occidentale, Austria e Svizzera e \emph{A. i. meridionalis} \cite{Fratini 2005} in Italia centro-meridionale e Slovenia, con zone di contatto tra diversi taxa rilevate in Italia e in particolare in Piemonte tra \emph{A. pallipes} e \emph{A. italicus} e sull'Appennino marchigiano tra \emph{A. i. carsicus} e \emph{A. i. meridionalis} \cite{Santucci 1997} \cite{Grandjean 2000} \cite{Zaccara 2004} \cite{Fratini 2005} \cite{Cataudella 2006}. L'esistenza della specie \emph{A. berndhauseri} in Svizzera e della sottospecie \emph{A. i. lusitanicus} in Spagna, proposta su base morfologica da diversi autori \cite{Bott 1950} \cite{Karaman 1962} \cite{Brodski 1983} \cite{Largiader 2000}, sarebbe stata esclusa da analisi molecolari. \emph{A. berndhauseri} sarebbe infatti stata ricondotta alla sottospecie \emph{A. i. carinthiacus} mentre \emph{A. i. lusitanicus} alla sottospecie \emph{A. i. italicus} \cite{Santucci 1997} \cite{Grandjean 2000}.

In considerazione delle limitazioni legate all'utilizzo di marcatori mitocondriali, che permettono di ottenere informazioni relative alla sola linea materna, offrendo un quadro solo parziale della complessità del pattern filogeografico di \emph{A. pallipes}, Chiesa \etal \cite{Chiesa 2011} hanno di recente sfruttato un approccio integrato, per valutare la diversità e la struttura genetica di \emph{Austropotamobius} in Italia, combinando analisi mitocondriali e di fingerprinting del DNA nucleare (\emph{Amplified fragment length polymorphism}, AFLP). Le analisi mitocondriali, basate sul sequenziamento del marcatore COI (citocromo ossidasi I), hanno evidenziato la presenza di quattro diversi gruppi di aplotipi, già individuati da precedenti autori \cite{Trontelj 2005} \cite{Stefani 2011} utilizzando lo stesso marcatore, ma solo parzialmente corrispondenti con quelli descritti da Grandjean \etal \cite{Grandjean 2002} e Fratini \etal \cite{Fratini 2005} attribuiti a differenti specie e sottospecie. Inoltre, la diversità genetica misurata ai marcatori nucleari AFLP tra le popolazioni italiane analizzate, non sembra supportare l'esistenza delle differenti specie e sottospecie ipotizzate sulla base delle analisi mitocondriali \cite{Chiesa 2011}. L'esistenza di un'unica specie, \emph{A. pallipes}, in Italia sarebbe ulteriormente confermata da recenti analisi morfometriche e meristiche condotte sulle stesse popolazioni \cite{Scalici 2012}.

In considerazione di questi risultati e del concetto di specie biologica, quale entità costituita da gruppi di popolazioni naturali interfeconde e riproduttivamente isolate da altri gruppi, sono stati condotti esperimenti di riproduzione tra le presunte specie \emph{A. pallipes} e \emph{A. italicus}, al fine di indagare il grado di isolamento riproduttivo esistente tra di esse \cite{Ghia 2011}. La totale assenza di barriere riproduttive riscontrata tra i due taxa fa presupporre anche in questo caso l'esistenza di un'unica specie sebbene resti ancora da accertare l'effettiva fertilità della prole generata.

Alla luce della più recente letteratura scientifica relativa alle popolazioni italiane di \emph{Austropotamobius}, si ritiene pertanto che la distinzione in differenti sottospecie non sia sufficientemente supportata. Tuttavia, il riconoscimento di differenti linee evolutive mitocondriali (‘\emph{carinthiacus}', ‘\emph{meridionalis}' e ‘\emph{carsicus}') e l'individuazione di diverse unità gestionali (Management Units, MU) sono funzionali al sostegno delle attività di tutela del gambero. L'analisi di marcatori genetici (mitocondriali e nucleari) ha infatti evidenziato in diversi studi una marcata variabilità genetica intra e interpopolazione in \emph{A. pallipes} con un hot spot di diversità nelle regioni alpine e mediterranee, e con un gradiente di variabilità inversamente proporzionale alla latitudine \cite{Grandjean 2000b} \cite{Gouin 2001} \cite{Baric 2005} \cite{Fratini 2005} \cite{Gouin 2006} \cite{Bertocchi 2008}. Tale struttura filogeografica è stata ricondotta ai processi microevolutivi innescati dalle oscillazioni climatiche del Quaternario \cite{Grandjean 2000b}. L'alternanza di periodi glaciali e interglaciali avrebbe infatti determinato l'isolamento di popolazioni in aree rifugio nel Sud dell'Europa (Iberia, Italia e Balcani meridionali) e la successiva nuova espansione verso nord \cite{Hewitt 1996} con conseguente sviluppo del gradiente latitudinale di diversità genetica. In questo processo le Alpi possono aver giocato sia un ruolo di connessione che di separazione delle popolazioni rappresentando da un lato il luogo di origine dei maggiori sistemi idrografici europei, e dall'altro una insormontabile barriera generata della morfologia del territorio. Nell'arco alpino si sarebbero quindi innescati importanti fenomeni dispersione e isolamento delle popolazioni con conseguente creazione di numerose linee evolutive. 

\small
\begin{longtable}[c]{@{}llllll@{}}
\toprule
\textbf{Autore}        & \textbf{Genere}           & \textbf{Specie}       & \textbf{Sottospecie}  & \textbf{Varietà} & \textbf{Analisi}          \\* \midrule
\endfirsthead
\multicolumn{6}{l}{\footnotesize\bfseries Continua dalla pagina precedente} \\
\textbf{Autore}        & \textbf{Genere}           & \textbf{Specie}       & \textbf{Sottospecie}  & \textbf{Varietà} & \textbf{Analisi}          \\* \midrule
\endhead
%
\bottomrule
\endfoot
%
\endlastfoot
%
\rowcolor[HTML]{EFEFEF} 
Lereboullet, 1858      & \textit{Austropotamobius} & \textit{pallipes}     & \textit{pallipes}     &       & morfologica    \\
Bott, 1950, 1972       & \textit{Austropotamobius} & \textit{pallipes}     & \textit{pallipes}     &       & morfologica    \\
            & \textit{}      & \textit{}  & \textit{italicus}     &       &     \\
            & \textit{}      & \textit{}  & \textit{lusitanicus}  &       &     \\
            & \textit{}      & \textit{berndhauseri} & \textit{}  &       &     \\
\rowcolor[HTML]{EFEFEF} 
Karaman, 1962, 1963    & \textit{Austropotamobius} & \textit{pallipes}     & \textit{pallipes}     &       & morfologica    \\
\rowcolor[HTML]{EFEFEF} 
            & \textit{}      & \textit{}  & \textit{bispinosus}   &       &     \\
\rowcolor[HTML]{EFEFEF} 
            & \textit{}      & \textit{italicus}     & \textit{italicus}     &       &     \\
\rowcolor[HTML]{EFEFEF} 
            & \textit{}      & \textit{}  & \textit{lusitanicus}  &       &     \\
\rowcolor[HTML]{EFEFEF} 
            & \textit{}      & \textit{}  & \textit{carsicus}     &       &     \\
Albrecht, 1982         & \textit{Austropotamobius}          & \textit{pallipes}   &            & \textit{pallipes}         & morfologica    \\
            &     &            &            & \textit{lombardicus}      &     \\
            &     &            &            & \textit{carnthiacus}      &     \\
            &     &            &            & \textit{trentinicus}      &     \\
\rowcolor[HTML]{EFEFEF} 
Brodski, 1983          & \textit{Austropotamobius} & \textit{pallipes}     & \textit{pallipes}     &       & biochimica  \\
\rowcolor[HTML]{EFEFEF} 
            & \textit{}      & \textit{}  & \textit{bispinosus}   &       & biogeografica     \\
\rowcolor[HTML]{EFEFEF} 
            & \textit{}      & \textit{italicus}     & \textit{italicus}     &       & ecologica     \\
\rowcolor[HTML]{EFEFEF} 
            & \textit{}      & \textit{}  & \textit{lusitanicus}  &       &     \\
\rowcolor[HTML]{EFEFEF} 
            & \textit{}      & \textit{}  & \textit{carsicus}     &       &     \\
Santucci, 1997         & \textit{Austropotamobius} & \textit{pallipes}     & \textit{}  &       & allozimatica \\
            & \textit{}      & \textit{italicus}     & \textit{carsicus}     &       &     \\
\rowcolor[HTML]{EFEFEF} 
Grandjean et al., 1998 & \textit{Austropotamobius} & \textit{pallipes}     & \textit{pallipes}     &       & mtDNA RFLP     \\
\rowcolor[HTML]{EFEFEF} 
            & \textit{}      & \textit{}  & \textit{italicus}     &       &     \\
\rowcolor[HTML]{EFEFEF} 
            & \textit{}      & \textit{}  & \textit{lusitanicus}  &       &     \\
Grandjean et al., 2000 & \textit{Austropotamobius} & \textit{pallipes}     & \textit{}  &       & mt 16S rRNA    \\
            & \textit{}      & \textit{italicus}     & \textit{italicus}     &       &     \\
            & \textit{}      & \textit{}  & \textit{carinthiacus} &       &     \\
            & \textit{}      & \textit{}  & \textit{carsicus}     &       &     \\
            & \textit{}      & \textit{berndhauseri} & \textit{}  &       &     \\
\rowcolor[HTML]{EFEFEF} 
Fratini et al., 2005   & \textit{Austropotamobius} & \textit{pallipes}     & \textit{pallipes}     &       & mt 16S rRNA    \\
\rowcolor[HTML]{EFEFEF} 
            & \textit{}      & \textit{}  & \textit{bispinosus}   &       &     \\
\rowcolor[HTML]{EFEFEF} 
            & \textit{}      & \textit{italicus}     & \textit{italicus}     &       &     \\
\rowcolor[HTML]{EFEFEF} 
            & \textit{}      & \textit{}  & \textit{carinthiacus} &       &     \\
\rowcolor[HTML]{EFEFEF} 
            & \textit{}      & \textit{}  & \textit{carsicus}     &       &     \\
\rowcolor[HTML]{EFEFEF} 
            & \textit{}      & \textit{}  & \textit{meridionalis} &       &     \\
Chiesa et al., 2011    & \textit{Austropotamobius} & \textit{pallipes}     & \textit{}  &       & COI mtDNA\\
& & & & & marcatori AFLP\\* \bottomrule

\caption{Proposte classificative per \emph{Austropotamobius pallipes complex} dal 1858 ad oggi con indicazione del metodo su cui si basa la classificazione}
\label{tab_1}\\
\end{longtable}
\normalsize

\subsection{Morfologia}
Il corpo del gambero d'acqua dolce è formato da ventuno segmenti o metameri, distinti in tre regioni principali: capo, torace e addome (figure \ref{fig_1} e \ref{fig_2}). Ciascun metamero, ad eccezione dell'ultimo, porta un paio di appendici variamente modificate e adattate a svolgere diverse funzioni.

\begin{figure}
  \centering
  \includegraphics[width=.8\columnwidth]{fig_1.jpg}
  \caption{Esemplare di \emph{A. pallipes} raccolto in Trentino, Rio Santa Colomba (foto S. Endrizzi)}
  \label{fig_1}
\end{figure}


Il capo è la regione che si estende dal rostro (prolungamento anteriore appuntito) al solco cefalico (punto di fusione con il torace). Ai due lati del rostro sono presenti gli occhi peduncolati. Il capo è costituito da sei metameri, che portano complessivamente sei paia di appendici: occhi composti, antennule bifide, antenne, mandibole e due paia di mascelle. Le prime tre paia di appendici svolgono funzioni sensoriali mentre le altre tre presentano una funzione masticatoria e triturante. Il torace, compreso tra il solco cefalico e l'addome, è costituito da otto metameri caratterizzati da otto paia di appendici: tre paia di massillipedi a funzione masticatoria e cinque paia di pereiopodi a funzione prevalentemente locomotoria. Il primo paio di pereiopodi è caratterizzato da chele ben sviluppate costituite da una porzione mobile ed una fissa articolate sul carpo. Le chele rappresentano uno strumento di fondamentale importanza per l'animale in quanto utilizzate nelle attività di predazione, difesa, escavazione del substrato per la creazione di rifugi e nell'accoppiamento. La perdita di entrambe le chele può quindi ridurre la sopravvivenza dell'individuo. Il secondo e terzo paio di pereiopodi presentano invece chele molto piccole utilizzate, nell'attività di foraggiamento, per raccogliere piccole particelle organiche e microrganismi bentonici e portarli alla bocca. Il quarto e quinto paio di pereiopodi sono totalmente sprovvisti di chele e svolgono una funzione puramente locomotoria. Il segmento basale interno dell'ultima Appendice toracica porta l'orifizio maschile mentre quello femminile si trova a livello del terzo paio di pereiopodi. Il capo e il torace sono saldati fra loro a formare il cefalotorace, struttura che nasconde la divisione metamerica di queste due regioni. 
\newpage
\begin{figure}[H]
  \centering
  \includegraphics[width=.8\columnwidth]{fig_2.jpg}
  \caption{Morfologia del gambero d'acqua dolce (Modificato da \url{http://usdbiology.com/})}
  \label{fig_2}
\end{figure}

L'addome è costituito da sette metameri ben distinguibili e articolati fra loro a formare una struttura molto mobile. Ciascun metamero, ad eccezione dell'ultimo, è caratterizzato da pleopodi ridotti e variamente modificati. Una evidente differenziazione di tali arti consente di individuare i due sessi; infatti nella femmina le appendici del primo segmento addominale sono molto ridotte e poco visibili ad occhio nudo; nel maschio, invece, le due paia anteriori di appendici sono modificate in organi copulatori detti gonopodi (figura \ref{fig_3}), molto evidenti e diretti in avanti per favorire il flusso dello sperma. Nelle femmine i pleopodi sono utilizzati per trattenere le uova e i giovani nati. In entrambi i sessi il sesto paio di appendici, maggiormente sviluppato rispetto agli altri e di forma allargata, costituisce gli uropodi che, assieme al telson, ultimo metamero addominale privo di appendici, forma il ventaglio caudale. Questa struttura favorisce la spinta retro-propulsiva, generata dalla muscolatura, che permette all'animale di nuotare rapidamente all'indietro per sfuggire ai predatori. Le femmine utilizzano inoltre il ventaglio caudale per proteggere le uova, ripiegandolo ventralmente. 

\begin{figure}[H]
  \centering
  \includegraphics[width=.8\columnwidth]{fig_3.jpg}
  \caption{\emph{A. pallipes}: a) maschio (visibili i gonopodi), b) femmina, c) femmina con uova (foto S. Endrizzi, B. Maiolini)}
  \label{fig_3}
\end{figure}

Il corpo di \emph{Austropotamobius pallipes} presenta uno scheletro esterno o esoscheletro, costituito da chitina, una sostanza proteica che conferisce al rivestimento membranoso consistenza cornea che, in relazione allo sviluppo, e in particolare alle fasi delle varie mute che si susseguono durante il ciclo vitale, presenta vari gradi di impregnazione di carbonato di calcio. L'esoscheletro forma il carapace, struttura particolarmente robusta che a livello del cefalotorace si estende lateralmente a formare le camere branchiali. La presenza di un esoscheletro sclerificato fornisce protezione agli organi interni, sostegno oltre ad agevolare la locomozione dell'animale, ma non consente il suo accrescimento. Pertanto i gamberi per crescere devono liberarsi del vecchio carapace e formare un nuovo rivestimento, tale fenomeno è detto muta o ecdisi ed è regolato ormonalmente. La crescita corporea è limitata ai brevi periodi in cui esso è abbastanza molle da espandersi, ovvero nel periodo immediatamente successivo alla muta.

Nel gambero, come in tutti i malcostraci, la muta (che dura da qualche minuto ad alcune ore) consiste nella formazione di una frattura dorsale dell'esoscheletro tra cefalotorace e addome. Il crostaceo gonfia il proprio corpo mediante l'assorbimento di acqua e sguscia fuori dal vecchio esoscheletro aiutandosi con il movimenti degli arti. L'animale che emerge è rivestito da una cuticola molle e, non potendo reggersi sulle zampe, rimane appoggiato sul fondo, steso su un lato estendendo i pereiopodi il più possibile, mentre i tessuti si gonfiano e si allungano grazie a una massima ritenzione idrica. Questa fase del ciclo vitale è delicata, poiché senza protezione, e dura 3-4 giorni, che l'animale trascorre generalmente all'interno di rifugi, al riparo dai predatori. Terminato il processo di muta l'animale ritorna attivo e riprende ad alimentarsi accumulando materiale proteico e sali minerali, fino alla successiva fase di pre-muta durante la quale l'animale smette di nutrirsi e rimuove dal vecchio esoscheletro il calcio. In questo stadio, che dura alcuni giorni, lo strato di cellule posto sotto l'esoscheletro si stacca e comincia a produrre il nuovo involucro, la fine della pre-muta culmina con la fessurazione del carapace.

Durante il processo di muta i decapodi hanno l'opportunità di rigenerare le appendici mutilate (figura \ref{fig_4}). I risultati della rigenerazione sono totalmente indistinguibili negli individui giovani, mentre negli adulti sono evidenti per le dimensioni solitamente inferiori delle appendici rigenerate. Nel gambero di fiume i sessi sono separati. Di norma il maschio sviluppa nel corso della vita chele di dimensioni maggiori rispetto alla femmina, che vanta però un addome assai più largo in quanto atto ad ospitare le uova.

\begin{figure}
  \centering
  \includegraphics[width=.8\columnwidth]{fig_4.jpg}
  \caption{\emph{A. pallipes}, esempio di rigenerazione della chela destra (foto S. Endrizzi)}
  \label{fig_4}
\end{figure}


Il gambero d'acqua dolce presenta una muscolatura metamerica striata collegata all'esoscheletro mediante tendini chitinosi. Il movimento degli arti e della regione addominale è determinato dalla contrazione e dal rilassamento dei muscoli flessori ed estensori, e la maggior parte della massa muscolare si trova a livello addominale. Il movimento è tipicamente reptante, si muovono sui pereiopodi e utilizzano le chele per arrampicarsi sulle superfici ripide. Il movimento propulsivo all'indietro è utilizzato solo occasionalmente come meccanismo di fuga.

\begin{figure}
  \centering
  \includegraphics[width=.8\columnwidth]{fig_5.jpg}
  \caption{Anatomia del gambero d'acqua dolce (modificato da \url{http://usdbiology.com/})}
  \label{fig_5}
\end{figure}


La respirazione è branchiale, le branchie lamellari, in numero di 18 per lato, sono collocate nelle camere branchiali laterali del cefalotorace e protette dall'esoscheletro (figura \ref{fig_5}), esse permettono gli scambi gassosi e ionici tra l'emolinfa e l'ambiente esterno, sono continuamente irrorate dalla corrente d'acqua che entra dalla parte posteriore del carapace, richiamata dai movimenti delle appendici del secondo paio di mascelle (scafognatiti) ed esce anteriormente da due fori posti lateralmente alla bocca. \emph{Austropotamobius pallipes} aumenta il tasso di ventilazione non appena diminuisce l'ossigeno o aumenta la concentrazione di anidride carbonica; ciò consente all'individuo di adattare la propria respirazione all'ambiente.

Il sistema circolatorio dei gamberi è di tipo lacunare aperto ed è costituito da un cuore uniloculare posto dorsalmente nel cefalotorace, da arterie e seni (figura \ref{fig_5}); l'emolinfa entra nel seno pericardico e penetra nel cuore attraversando tre aperture laterali dette osti, provviste di valvole per impedire il reflusso dell'emolinfa. La contrazione del cuore spinge l'emolinfa nelle arterie, ovvero: 1) cranialmente, un'arteria oftalmica (cervello), un paio di arterie antennali (organi genitali, occhi, antenne e mandibole) ed epatiche; 2) caudalmente, un'arteria addominale posteriore (muscoli addominali e intestino) ed una sternale (appendici toraciche e addominali). Dalle arterie il sangue si riversa in una cavità, detta emocele, in cui sono contenuti gli organi vitali dell'animale, poi l'emolinfa torna al cuore attraverso il seno ventrale, e di qua giunge alle branchie, cede anidride carbonica, assume ossigeno e, attraverso i vasi branchiali efferenti, raggiunge il seno pericardico. Il cuore dei crostacei è una pompa aspirante; quando si decontrae, dilatandosi, aspira emolinfa dal seno pericardico.

L'apparato genitale maschile comprende due testicoli, due canali deferenti e relative ghiandole androgene. I testicoli si estendono dorsalmente nella regione toracica in direzione cranio-caudale, fondendosi in seguito in una struttura impari (configurazione a Y), da ciascuno di essi partono spermidotti lunghi e convoluti, che sboccano in prossimità di una papilla genitale a livello del quinto paio di pereiopodi. L'apparato genitale femminile è composto da due ovari e due ovidutti; i primi hanno medesima collocazione dorsale dei testicoli, estendendosi fino al secondo segmento addominale con l'analoga struttura allungata contraddistinta dalla loro fusione caudale (forma a Y). Gli ovidutti, sono brevi, e sboccano a livello del terzo paio di pereiopodi. I gonopodi dei maschi e gli orifizi degli ovodutti delle femmine sono chiaramente visibili sulla faccia ventrale del torace (figura \ref{fig_3}). La fecondazione esterna avviene tramite il trasferimento da parte del maschio di spermatofore tubuliformi alla femmina, la quale provvede alla loro conservazione in appositi ricettacoli sino all'ovodeposizione.

Il sistema digestivo è composto dall'apparato boccale, un breve esofago, stomaco e intestino (figura \ref{fig_5}). Lo stomaco è diviso in una regione cardiale anteriore e in una pilorica posteriore. La regione cardiale, più grande, contiene una coppia di grossi pezzi calcarei discoidali detti gastroliti, che costituiscono una riserva di calcio che si scioglie poco prima della muta fornendo al sangue la quantità di calcio necessaria per formare il nuovo carapace. Il cibo che giunge allo stomaco attraverso l'esofago, oltre a subire degradazione chimica, viene ulteriormente triturato da una particolare struttura detta mulino gastrico, ovvero una parte dello stomaco pilorico provvista di formazioni dentiformi, costituite da numerose piastre cuticolari calcificate, messe in movimento da specifici muscoli associati alla parete dello stomaco. Il materiale si sposta poi verso lo stomaco pilorico, dove una serie di setole filtranti impediscono alle particelle di grandi dimensioni di entrare nell'intestino. L'intestino medio, con funzione digestiva, è breve e privo di chitina; presenta un organo molto ramificato, l'epatopancreas, che secerne enzimi digestivi che permettono il successivo assorbimento dei nutrienti.

I gamberi, come tutti i crostacei sia marini che di acqua dolce espellono il 70-90\% delle loro scorie azotate sotto forma di ammoniaca, mentre la restante parte viene espulsa sotto forma di urea, acido urico, aminoacidi ed altri composti. L'ammoniaca può essere liberata per mezzo delle branchie, ma l'apparato escretore vero e proprio è costituito dalla ghiandola del verde posta alla base delle antenne, le cui cellule svolgono una funzione assorbente selettiva nei confronti delle sostanze presenti nell'emolinfa. 

Il sistema nervoso è formato da un complesso gangliare cerebrale sopraesofageo e da due gangli nervosi per ogni segmento, collegati tra loro ai muscoli da fibre nervose raggruppate in fasci. 

Gli organi di senso sono distinti in esterocettori ed enterocettori. I primi sono distribuiti sulla superficie corporea e permettono di percepire gli stimoli derivanti dall'ambiente esterno quali luce, odori, suoni e stimoli idrodinamici. Gli enterocettori sono invece inseriti più profondamente e permettono di rispondere agli stimoli interni, come le variazioni di pressione e di concentrazione delle sostanze chimiche presenti nei fluidi corporei. I chemiorecettori a funzione olfattoria sono distribuiti sulle antennule mentre quelli del gusto si trovano sulle varie parti boccali e sui pleiopodi. La percezione della gravità, e quindi della posizione del corpo nello spazio, è resa possibile dalle statocisti poste alla base delle antennule. Gli stimoli idrodinamici, importanti per la percezione della direzione della corrente e degli organismi presenti nelle vicinanze (competitori, prede e predatori) sono avvertiti da migliaia di recettori distribuiti sulla superficie corporea e sulle appendici. I gamberi hanno occhi composti, ovvero costituiti da una serie di unità visive dette ommatidi. 

\subsection{Ciclo vitale e riproduzione}
L'accoppiamento di \emph{A. pallipes} avviene in autunno nei mesi di ottobre e novembre, generalmente quando la temperatura dell'acqua si approssima ai 10 $^{\circ}$C. In questo periodo il maschio cerca la femmina in modo più attivo. L'accoppiamento rappresenta una fase molto delicata che può mettere a rischio la vita stessa degli individui, in particolar modo della femmina. 

\begin{figure}
  \centering
  \includegraphics[width=.5\columnwidth]{fig_6.jpg}
  \caption{Individui di \emph{O. limosus} in accoppiamento (foto: S. Endrizzi)}
  \label{fig_6}
\end{figure}

Durante questa fase, infatti, il maschio rovescia la femmina sul dorso e la afferra con le chele per tenerla ferma causando a volte mutilazioni o addirittura l'uccisione (figura \ref{fig_6}). Una volta posizionati ventre a ventre, il maschio emette il liquido spermatico che, a contatto con l'acqua, gelifica ed assume la forma di numerosi piccoli sacchetti di forma vermicolare contenenti gli spermatozoi, le spermatofore, che vengono fatte aderire agli sterniti toracici del cefalotorace della femmina, attorno agli orifizi esterni degli ovidotti, tra il terzo ed il quinto paio di arti toracici.

Le femmine si rintanano quindi in rifugi individuali, dove portano a termine la maturazione e l'emissione degli ovociti in tempi cha variano da pochi giorni a un mese a seconda della temperatura dell'acqua. Appoggiate sul dorso, ripiegano l'addome che si riempie del secreto mucoso delle ghiandole tegumentarie rivestendo le setole dei pleopodi che vengono così saldati fra loro in modo da formare una sorta di sacca, in cui si depositano, dopo 2-3 giorni, le uova emesse dagli ovidotti. Al momento della deposizione delle uova, il cui numero varia da 30 a circa 100 e con diametro da 1 a 2 mm a seconda della taglia della madre, le ghiandole sessuali della femmina producono un secreto che, grazie all'azione dei pleopodi, si diffonde verso le spermatofore provocandone la disgregazione; gli spermatozoi così liberati fecondano le uova, collocate nelle sacche ovigere. Il secreto si trasforma in un secondo momento in una membrana che isola ciascun uovo e che salda le uova ai pleopodi e all'addome in generale (figura \label{fig_7}a). Questo processo dura mediamente 2-3 ore e successivamente la femmina si rintana per tutto l'inverno fino al momento della schiusa, ossigenando e pulendo in continuazione le proprie uova. Il periodo d'incubazione varia generalmente in rapporto alla temperatura dell'acqua e può oscillare fra i cinque e i sette mesi, ma tipicamente le uova schiudono a primavera. Il comportamento della madre si modifica quando si avvicina il momento della schiusa: dilata il telson, lo solleva aritmicamente e pettina le sue uova in modo più attivo. 

Tutti i gamberi d'acqua dolce, \emph{A. pallipes} incluso, non hanno fasi larvali libere, e tutte le fasi di sviluppo avvengono all'interno dell'uovo. La schiusa avviene con l'apertura del guscio in due valve che rimangono attaccate al peduncolo. Dall'uovo emerge una fase giovanile (detta larva di prima fase) che possiede la maggior parte delle appendici dell'adulto benché le chele abbiano le estremità ripiegate ad amo ed il cefalotorace sia sproporzionato; misura 8-9 mm e pesa mediamente intorno ai 20 mg. I giovanili rimangono passivamente attaccati con le chele uncinate ai pleiopodi della femmina. In questo stadio le larve sono lecitotrofiche e si nutrono, per uno fino a un massimo di dieci giorni, delle riserve contenute nel sacco vitellino che viene via via riassorbito. Il carapace della larva durante il primo stadio larvale è ancora relativamente molle ed elastico, consentendo una crescita in lunghezza e in peso. A sei-otto giorni dalla nascita avviene la prima muta, il giovane gamberetto (larva di seconda fase), lungo circa 12 mm si distacca dalla madre e comincia l'alimentazione attiva e la vita autonoma sul fondo dei corsi d'acqua. I giovani per i primi giorni rimangono a pochi centimetri dalla madre, per poter eventualmente correre al riparo quando la femmina, in caso di pericolo, richiama a sé i giovani con l'emissione di ferormoni d'allarme (figura \ref{fig_7} b, c, d). Con la muta successiva, generalmente in autunno, si passa a larve di terza fase, di circa 20-30 mm, identificate come gamberi di età 0+, ovvero di età pari a un'estate di vita e quindi inferiori a un anno. 

\begin{figure}
  \centering
  \includegraphics[width=.8\columnwidth]{fig_7.jpg}
  \caption{\emph{A. pallipes}: a) addome di femmina con uova; b, c) femmina con prole (larve di seconda fase), d) giovane gambero, larva di seconda fase (foto S. Endrizzi, B. Maiolini)}
  \label{fig_7}
\end{figure}

Durante il primo anno di vita, il giovane gambero, continuando a crescere in dimensioni, compie 5-6 mute (ricambio dell'esoscheletro). Il gambero di fiume generalmente raggiunge la maturità sessuale al terzo o quarto anno di vita, quando i maschi hanno raggiunto una lunghezza totale (dalla punta del rostro fino al telson) di 60-70 mm e le femmine di 55-60 mm e hanno terminato, in seguito a muta, la lunghezza isometrica di determinate caratteristiche morfologiche (ad es. le chele). I gamberi d'acqua dolce hanno crescita indeterminata con mute più rare, con una frequenza massima di 1-2 volte l'anno, dopo il raggiungimento della maturità sessuale fino alla morte dell'animale, con taglie pari a circa 120 mm e oltre, e peso di 70-80 gr. Esiste dimorfismo sessuale nel peso e nelle dimensioni del corpo e delle chele, che risultano maggiori nel maschio rispetto a alla femmina. Una volta raggiunta la maturità sessuale, infatti il tasso di crescita nei due sessi si differenzia. Questo è dovuto principalmente al maggior numero di mute nel maschio rispetto alla femmina. Queste ultime infatti dovendo trattenere le uova fino a inizio estate sono in grado di affrontare una sola muta nel periodo estivo, mentre il maschio muta due volte.

Il tasso di crescita nei gamberi può mostrare ritmi molto diversi tra popolazioni \cite{Lowery 1988} in quanto risulta influenzato in modo inversamente proporzionale alla latitudine e alla quota \cite{Vogt 2012} \cite{Scalici 2008}. Il fattore che maggiormente influisce sul tasso di crescita è la temperatura dell'acqua, che determina i ritmi stagionali. Così i gamberi che vivono sulle Alpi presenteranno un tasso di crescita inferiore rispetto a quelli del centro e del Sud Italia. \emph{A. pallipes} è la specie europea più piccola e a crescita più lenta, e può vivere anche più di 10 anni \cite{Kouba 2015}.

\subsection{Ecologia}
\subsubsection{Habitat} 
\label{subsub_hab}
Il gambero di fiume \emph{A. pallipes} è in grado di vivere bene in diverse tipologie di habitat acquatici, colonizza prevalentemente torrenti e piccoli corsi d'acqua montani e collinari caratterizzati da acque limpide, fresche, ben ossigenate (salmonicole). In passato era spesso presente anche in piccoli laghi e raccolte d'acqua naturali o artificiali purché caratterizzate da acque fresche e ben ossigenate, nonché in tratti sorgivi dei fiumi maggiori e nei fontanili. Le quote ottimali di \emph{A. pallipes} sono difficili da definire, poiché altri fattori come la latitudine influenzano la temperatura dell'acqua. In Italia la specie si trova fino a 1200 m s.l.m., quando le condizioni termiche lo consentono \cite{Endrizzi 2013}.

Preferisce ruscelli e torrenti con fondo di roccia, ghiaia e sabbia, ma è presente anche in quelli della fascia collinare e prealpina caratterizzati da ciottoli, fango, limo, foglie e rami, radici sommerse e vegetazione acquatica. L'habitat dev'essere inoltre diversificato in modo da rendere disponibile una buona quantità di rifugi rappresentati da anfratti rocciosi, tronchi e ceppi sommersi, radici di alberi, banchi di macrofite, lettiere di foglie e rami, e tane scavate dagli individui stessi lungo le rive \cite{Souty-Grosset 2006}. In mancanza di rifugi naturali infatti è in grado di scavare tane nei sedimenti fini delle rive sotto la vegetazione erbacea riparia. Questo comportamento è tipico degli adulti, mentre gli individui giovanili sono più spesso confinati in aree con acqua poco profonda e molti rifugi e conducono invece vita interstiziale, occupando per lo più lo spazio tra i ciottoli del fondo. La presenza di habitat eterogenei riduce gli effetti della predazione sui giovani e diminuisce il rischio di cannibalismo durante il periodo di muta nei gamberi adulti \cite{Ackefors 1996}. 
La luce rappresenta un parametro importante per la selezione degli habitat ottimali, in quanto \emph{A. pallipes} preferisce corsi d'acqua ben ombreggiati, con una buona copertura vegetale riparia, che impedisce al sole un diretto irraggiamento della superficie dell'acqua. L'attività del gambero è sostanzialmente crepuscolare e notturna. Questa specie tollera ampi intervalli di temperatura e possiede una buona resistenza, con valori ottimali estivi compresi tra 15 e 18$^{\circ}$C; e valori prossimi a 25$^{\circ}$C tollerati solo per brevi periodi. Resistono nel periodo invernale a temperature prossime allo zero.

Nonostante \emph{A. pallipes} sia in grado di tollerare basse concentrazioni di ossigeno per lunghi periodi di tempo, per la sopravvivenza delle popolazioni è necessaria una buona ossigenazione dell'acqua (>60 \% sat.). \emph{A. pallipes} è invece strettamente stenoionico con valori ottimali di pH compresi tra 6 e 9, necessita di una concentrazione di calcio disciolto variabile tra 2.7 ppm e 140 ppm, con valori ottimali tra 50 e 100 ppm, e una concentrazione di HCO3 da 6 a 430 ppm \cite{Smith 1996}. Infine, \emph{A. pallipes} preferisce acque con basso livello di inquinamento e di sedimento fine e, sebbene sia stato in passato considerato un buon indicatore di qualità delle acque, in realtà la sua presenza è stata rilevata anche in ambienti con bassa qualità ecologica, in cui si è dimostrato in grado di tollerare bene eutrofizzazione e acidificazione delle acque con un valore medio di pH di 7,9 nei torrenti in Piemonte \cite{Favaro 2010}. E' comunque sensibile all'inquinamento organico che riduce il livello di ossigeno nell'ambiente acquatico, e chimico, in particolare ai pesticidi.

Pertanto, \emph{A. pallipes} va considerato un indicatore di qualità degli habitat intesa soprattutto come eterogeneità di ambienti e disponibilità di rifugi piuttosto che di bioindicatore, e gli è pertanto attribuito il ruolo di "specie bandiera" \cite{Fureder 2003}.

In presenza di disagio ambientale particolarmente grave, \emph{A. pallipes} può uscire dall'acqua e sopravvivere per alcune ore, muovendosi a una velocità di 54 m h-1 \cite{Pond 1975} in cerca di un altro corso idrico vicino, poiché la respirazione branchiale può avvenire anche in atmosfera umida. L'interramento, comportamento messo in atto anche per difendersi dalle rigide temperature invernali, rappresenta un'altra possibile reazione a fenomeni d'inquinamento.

\subsubsection{Dieta}
In \emph{Austropotamobius pallipes} la dieta varia in relazione alla taglia: carnivora nei primi anni di vita e vegetariana - detritivora, prevalentemente fitofaga con preferenza per la macrofita Fontinalis antipiretica \cite{Arrignon 1996}, negli adulti. La componente carnea del suo regime alimentare è costituita da prede vive, ricercate tra i macroinvertebrati acquatici quali molluschi, i cui gusci forniscono una importante fonte di calcio, larve di insetti, piccoli crostacei bentonici e anellidi oltre ad anfibi e le loro larve nonchè piccoli pesci \cite{Mancini 1986}. L'alimentazione di \emph{A. pallipes} varia anche a seconda della tipologia del corso d'acqua e, sebbene la componente carnea sia preponderante, si ciba anche di materia vegetale (alghe e macrofite), semi e resti di frutti (castagne, ciliegie, ecc.). Importante è anche la caratteristica dei mesohabitat occupati: nelle zone superficiali poco profonde il nutrimento è prevalentemente di origine vegetale, mentre nelle fosse questo è costituito per la maggior parte dalla componente animale \cite{Mason 1974}. Fenomeni di cannibalismo sono frequenti e sono accentuati soprattutto in situazioni di stress, per mancanza di risorse alimentari e sovrappopolamento dell'habitat \cite{Elgar 1992}. Tale comportamento è messo in atto generalmente dagli individui adulti nei confronti dei giovani o degli individui indeboliti da malattie. Il cibo viene afferrato con le piccole chele del secondo e terzo paio di pereiopodi e portato alla bocca, dove tutte le appendici boccali, ricchissime di setole con funzione tattile e olfattiva, lo selezionano mentre le mandibole lo triturano. Essendo una specie fotofoba la ricerca del cibo avviene dal tardo pomeriggio alle prime ore del mattino, con un picco di attività concentrato nel mezzo della notte.

\subsubsection{Comportamento}
Il gambero è fondamentalmente un animale solitario con tendenze territoriali più accentuate nei maschi, soprattutto in concomitanza del periodo riproduttivo quando aumenta anche la mobilità erratica in entrambi i sessi. Pur non potendo essere definiti notturni, i gamberi hanno abitudini marcatamente crepuscolari e lucifughe. Sono poco attivi di giorno, quando stanno nascosti tra tronchi e ceppi sommersi, banchi di macrofite, lettiere di foglie e rami, anfratti rocciosi o in tane da essi stessi scavate lungo le rive dei corsi d'acqua. Di notte gli animali sono molto attivi, e impegnati nella caccia, attuata muovendosi sul fondo dei torrenti con le chele protese in avanti, pronte all'attacco, nell'intento di procacciarsi attivamente le prede. I gamberi si nutrono lentamente ma non a lungo, tendendo a spostarsi in altre zone dopo un breve periodo, comportamento che riduce il rischio di cattura da parte dei predatori (pesci, uccelli acquatici e piccoli mammiferi carnivori).
Nonostante \emph{A. pallipes} venga considerato territoriale, non è raro individuare popolazioni concentrate su poche decine di metri quadri lungo i corsi d'acqua, in caso di condizioni ambientali particolarmente propizie (clima, disponibilità di rifugi, cibo, ossigeno, ioni calcio disciolti, ecc…). Spostamenti verso il fondo o la superficie si verificano negli stagni in rapporto all'andamento della temperatura con la profondità. Spostamenti e migrazioni nello stesso corso d'acqua da monte a valle e viceversa sono spesso dovute alle condizioni ambientali avverse o all'eccessiva densità di popolazione.

\subsection{Parassiti e patogeni} 
I gamberi di fiume sono ospiti di una vasta gamma di commensali, parassiti e patogeni, tra cui virus, batteri, funghi, protisti, trematodi, cestodi, acantocefali, nematodi, anellidi branchiobdellidi, platelminti temnocefalidi e artropodi. Il carapace funge da un lato da barriera limitando l'attacco da parte di agenti patogeni e di parassiti e dall'altro da substrato adatto agli ectoparassiti che, in numero elevato, potrebbero ridurre la mobilità dei gamberi. Tuttavia, attraverso il processo naturale della muta, il numero di ectoparassiti viene periodicamente ridotto.

Il movimento di gamberi, soprattutto di specie alloctone, operato dall'uomo per le attività di acquacoltura e di acquariofilia, ha comportato la traslocazione di un certo numero di parassiti e di patogeni da un ambiente all'altro con effetti anche gravi sulle popolazioni native (vedi oltre, cap. \ref{subsec_diffusionepat}).

Per quel che riguarda \emph{A. pallipes}, una revisione recente su parassiti, commensali, patogeni e malattie dei gamberi condotta su tutto l'areale di distribuzione \cite{Longshaw 2016} \cite{Quaglio 2011} elenca la presenza dei seguenti patogeni:
\begin{enumerate}[label={\alph*)}]
  \item Virus: ApBV e WSSV \\
  Il virus intranucleare bacilliforme di \emph{A. pallipes} (ApBV) (la posizione tassonomica è ancora sconosciuta; è limitato all'epatopancreas e all'intestino) è stato ritenuto responsabile o una concausa di mortalità ed estinzione di popolazioni di \emph{A. pallipes}. Il virus che causa la malattia delle macchie bianche o "\emph{white spot syndrome virus}" (WSSV) appartiene alla famiglia \emph{Nimaviridae}. É una patologia mortale in gamberi e altri malacostraca sia giovani che adulti, che muoiono da 3 a 10 giorni dal momento della comparsa della malattia. Il virus infetta un'ampia varietà di cellule del tessuto epiteliale delle branchie, delle ghiandole antennali, del tessuto emopoietico, nervoso, connettivo e dell'epitelio intestinale. La malattia si trasmette principalmente per ingestione di tessuti infetti (cannibalismo, predazione), per contatto diretto, tramite vettori quali rotiferi, vermi policheti, isopodi e crostacei non decapodi (\emph{Artemia salina}) e meno frequentemente tramite acqua contaminata.

  \item Batteri: \emph{Aeromonas hydrophila}, \emph{Citrobacter sp}., \emph{C. freundii}, \emph{Nocardia sp}., \emph{Proteus morganii}, \emph{P. vulgaris}, \emph{Pseudomonas aeruginosa}, \emph{P. flourescens}, \emph{P. putida}; \emph{Hafnia alvei}.\\
  I batteri isolati nelle batteriemie asintomatiche e nelle setticemie dei gamberi d'acqua dolce sono diffusi nell'ambiente, sia nell'acqua, sia nel sedimento e colonizzano l'esoscheletro e l'intestino. Nella batteriemia asintomatica e nella setticemia di \emph{A. pallipes} sono coinvolti batteri Gram negativi dei generi \emph{Acinetobacter}, \emph{Aeromonas}, \emph{Citrobacter}, \emph{Flavobacterium}, \emph{Pseudomonas}, \emph{Proteus} \cite{Quaglio 2011}. Le batteriemie sono probabilmente favorite in situazioni di stress e dalla presenza di ferite nel carapace; spesso sono asintomatiche, ma in alcuni casi evolvono in setticemie con sintomatologia clinica (letargia, perdita del tono muscolare, scarsa reattività agli stimoli). Al genere Nocardia appartengono più di 80 batteri Gram positivi, largamente diffusi in natura nel suolo come saprofiti su materiali organici in decomposizione. Attualmente risulta un'unica segnalazione di infezione da \emph{Nocardia sp}. in un campione adulto di \emph{A. pallipes}, prelevato nel fiume Avon, in Inghilterra durante un'indagine su un episodio di peste \cite{Alderman 1986}. \\
  \emph{Hafnia alvei} è un bacillo Gram negativo, anaerobico facoltativo della famiglia \emph{Enterobatteriaceae}, che può essere presente come commensale nell'intestino di mammiferi, uccelli ed in ambiente naturale (acqua e suolo), causa una batteriemia asintomatica. É un batterio opportunista che può avere un ruolo patogeno in animali gravemente compromessi dall'azione di inquinanti.
  \item Funghi: \emph{Aphanomyces astaci}, \emph{Acremonium sp}., \emph{Alternaria sp}., \emph{A. frigidophilus}, \emph{A. repetans}, \emph{Cryptococcus gammari}, \emph{Fusarium sp}., \emph{F. oxysporum}, \emph{F. solani}, \emph{Plectosporium tabacinum}, \emph{Geotrichum sp}., \emph{Gliocladium sp}., \emph{Penicillium sp}., \emph{Saprolegnia sp}., \emph{S. australis}, \emph{S. parasitica}, \emph{Trichoderma sp}. 

  La più conosciuta e devastante malattia in grado di contagiare i gamberi d'acqua dolce, è la micosi nota come "peste del gambero", causata dall'oomicete \emph{Aphanomyces astaci}. La patologia di origine americana, diffusa con l'introduzione di gamberi alloctoni americani, come \emph{P. clarkii}, \emph{O. limosus} e \emph{P. leniusculus}, che sono resistenti all'infezione e agiscono da vettori biologici, può provocare una mortalità del 100\% nelle specie autoctone europee ed è quindi la causa della loro scomparsa in molte aree geografiche. 

  Attualmente, sono noti 5 ceppi del patogeno \cite{Svoboda 2017}: 1) gruppo A (genotipo As) comprende ceppi isolati dai gamberi europei \emph{Astacus astacus} e \emph{A. leptodactylus}; 2) gruppo B (genotipo PsI), ceppi isolati da \emph{Pacifastacus leniusculus} di origine californiana; 3) gruppo C (genotipo PsII) ceppi provenienti da \emph{P. leniusculus} di origine canadese; 4) gruppo D (tipo Pc) ceppi isolati da \emph{Procambarus clarkii}; 5) gruppo E isolato da \emph{Orconectes limosus} in Repubblica Ceca. I ceppi A-C sono adattati a gamberi d'acqua fredda (4-21$^{\circ}$C), i ceppi del gruppo D originano in regioni subtropicali del sud est degli Stati Uniti d'America e si sono adattati a crescere a temperature di 20-26$^{\circ}$ C, e risultano più virulenti a temperature superiori a 20$^{\circ}$ C rispetto ai ceppi di acqua fredda, e sono in grado di crescere più rapidamente. I ceppi del gruppo E hanno una crescita ottimale a 22$^{\circ}$C e produzione di spore tra 5-26$^{\circ}$C. La sensibilità all'infezione e il decorso clinico variano notevolmente tra le differenti specie di gambero: specie europee appartenenti alla famiglia Astacidae, tra cui \emph{A. pallipes}, risultano ad esempio estremamente sensibili. La trasmissione avviene per via orizzontale diretta, da gambero infetto o portatore asintomatico (specie nordamericane) a gambero sano, o indirettamente. Infatti, le zoospore possono essere veicolate, tra bacini idrici diversi tramite acqua contaminata, sulla superficie corporea di uccelli, mammiferi acquatici, specie ittiche traslocate da aree infette ad aree indenni (le zoospore possono sopravvivere nel muco cutaneo e nel tratto intestinale dei pesci oltre che essere presenti nell'acqua di trasporto) e su attrezzatura da pesca, natanti, stivali e vestiario. La riproduzione avviene in modo asessuato, tramite zoospore, mobili e biflagellate adatte alla dispersione in acqua, che vengono attratte chemiotatticamente dalla cuticola dei crostacei, a cui la spora aderisce tramite i flagelli e la perfora grazie all'azione di enzimi lipolitici e chitinolitici. Le ife vegetative penetrano attraverso l'esoscheletro del gambero e si incistano. Successivamente le ife, che invadendo l'ipoderma si propagano fino agli organi vitali e provocano la morte del gambero a causa della produzione di una neurotossina. Le ife producono sporangi che fuoriescono dalla cuticola ricoprendola localmente, prima della morte dell'ospite, di un sottile feltro biancastro. Dall'animale moribondo si liberano nell'ambiente acquatico un elevato numero di nuove spore che possono rimanere vitali da una a otto settimane. I sintomi dell'infezione sono di tipo comportamentale e consistono in un aumento dell'attività motoria seguito da apatia. I gamberi infetti sono inoltre attivi in pieno giorno, tendono ad uscire dall'acqua, si muovono in modo scoordinato, instabile con perdita dell'equilibrio, non tentano di fuggire alla cattura e rimangono inerti se trattenuti. Al culmine dell'infezione si rovesciano sul dorso, muovono convulsamente le appendici e non riescono a raddrizzarsi; possono morire in questa posizione per paralisi. Frequentemente si verifica il distacco di arti o porzioni di essi. Sia in gamberi alloctoni portatori sani che in gamberi autoctoni con infezione subacuta si evidenziano aree ulcerate e brunastre, per deposito di melanina nella cuticola. La risposta immunitaria contro \emph{A. astaci} è più efficace e rapida nelle specie appartenenti alla famiglia dei Cambaridae e consiste nel contenimento della diffusione delle ife entro la porzione più superficiale della cuticola, grazie ad un'abbondante produzione di melanina. Possono così essere presenti gamberi alloctoni apparentemente sani, ma portatori del patogeno che sarà comunque in grado di completare il ciclo biologico con la produzione di zoospore letali per le specie autoctone. L'Italia è stata il primo paese in Europa colpito da epizoozia di peste del gambero, probabilmente di origine nordamericana. Il primo episodio di peste fu descritto in Lombardia, nel 1859 a nord del fiume Po ed altri ne sono seguiti in tutta la pianura padana, da Torino a Trieste. La peste si propagò in tutta l'Europa centrale fino ai paesi baltici, scandinavi ed in Russia. Dopo aver decimato le popolazioni di gamberi indigeni, tra la seconda metà del 1800 e i primi decenni del 1900, in Italia, Francia, Germania, Austria, Belgio, Paesi Bassi, Polonia, Danimarca, Estonia, Lettonia, Lituania, Finlandia, Svezia, Ungheria, Romania, Bulgaria, Russia e Paesi Slavi, la peste ha diminuito la sua incidenza restando circoscritta ad alcuni focolai nell'Europa orientale e settentrionale. Negli ultimi decenni l'epidemia è tornata a minacciare i gamberi europei interessando anche le acque interne della Norvegia, Spagna, Gran Bretagna, Irlanda, Turchia, Svizzera e Repubblica Ceca. La principale fonte di diffusione in Europa tra il 1960 e il 2000 è stata l'importazione incontrollata di gamberi americani.

  I miceti del genere \emph{Acremonium} causano rammollimento bilaterale del cefalotorace e macchie brunastre nelle branchie, non causano mortalità e possono essere considerati come organismi saprofiti. Anche i generi \emph{Penicillium}, \emph{Gliocadium}, \emph{Trichoderma}, \emph{Geotrichum} e \emph{Alternaria} sono stati riscontrati sia su gamberi sani che su gamberi morti, colonizzano ed invadono le parti molli della cuticola, le ferite dell'esoscheletro, le branchie e le uova. Si ritengono microrganismi saprofiti, la loro presenza nei gamberi spesso testimonia il degrado delle condizioni ambientali.

  \emph{Fusarium} è un vasto genere di funghi che si riproducono per conidiospore. Le specie di Fusarium sono ampiamente diffuse nell'ambiente, nel suolo e sulle piante e sono riportate anche in organismi acquatici, e sono un agente patogeno opportunista del gambero d'acqua dolce. La patogenicità del fungo può essere correlata a ferite dell'esoscheletro, ad inquinamento ambientale o fattori fisico-chimici sfavorevoli, stressanti, che diminuiscono le difese dell'animale. Le condizioni che favoriscono l'infezione sono frequenti in acquacoltura. L'infezione da Fusarium solani può manifestarsi con lesioni su esoscheletro, branchie ed emocele. La morte, in condizioni sperimentali, si può verificare anche dopo diversi mesi dall'infezione e la malattia può avere un decorso anche molto lungo; la mortalità è causata da alterazioni fisiologiche per interferenza con la muta, alle esotossine prodotte dal fungo, a disturbi di pressione osmotica e ad alterazioni delle concentrazioni di sodio e cloro nell'emolinfa. La presenza del fungo è accompagnata da depositi di melanina nella cuticola e nelle branchie; la malattia perciò viene definita malattia dell'addome bruno o "\emph{brown abdomen disease}". \emph{Fusarium} in associazione con altri funghi, batteri chitinolitici Gram negativi e \emph{Pseudomonas sp}. può provocare la "ruggine dei gamberi" o "\emph{Burn Spot Disease}", che determina lesioni alle branchie e ai muscoli. All'esterno i punti di infezione si presentano come macchie nere-rossastre (da cui il nome) che possono degenerare fino a vere e proprie lacerazioni. Ha un lungo decorso e porta a una mortalità abbastanza modesta, spesso dovuta a infezioni batteriche secondarie.

  Gli oomiceti del genere \emph{Saprolegnia} hanno diffusione cosmopolita nelle acque dolci, sono muffe acquatiche saprofite od opportunistiche. Includono specie responsabili di significative infezioni in pesci ed altri organismi acquatici. Infezioni di \emph{A. pallipes} causate da Saprolegnia interessano in particolare le uova dei gamberi, in condizioni di allevamento. In questo caso l'oomicete colonizza le uova morte e da qui si estende a quelle vicine, causando così gravi perdite di uova negli allevamenti e nell'incubazione artificiale delle uova. Negli adulti provoca lesioni cuticolari e può essere causa di elevata mortalità in condizioni di allevamento, quando fattori ambientali, bassa concentrazione di ossigeno disciolto ed elevato tasso di solidi sospesi, inducono stress e predispongono gli animali ad infezioni fungine.
  
  \item Microsporidia: \emph{Thelohania contejeani}\\ 
  I \emph{Microsporidia} sono patogeni intracellulari obbligati, con 14 specie descritte per i gamberi dulcacquicoli. Il ciclo vitale è dimorfico all'interno dell'ospite. La muscolatura è il sito primario dell'infezione, i segni clinici sono limitati a apatia, riduzione dei movimenti della coda e opacità della muscolatura. \emph{Thelohania contejeani} è responsabile della malattia della porcellana, così chiamata a causa della colorazione lattiginosa assunta dalla muscolatura addominale. Si tratta di una delle malattie più diffuse e gravi di \emph{A. pallipes} ed è generalmente presente nelle popolazioni selvatiche con un'incidenza variabile (da < 1.0 \% al 30 \%), colpendo indistintamente esemplari immaturi e adulti. La "malattia della porcellana" è stata ritenuta causa del rapido calo numerico di varie popolazioni, ma non sembra in grado di determinare mortalità di massa. La malattia provoca la degenerazione dei tessuti muscolari e si diffonde tipicamente per necrofagia e cannibalismo sugli individui malati, con decorso che può durare anche molti mesi. 
  \item Mesomycetozoea: \emph{Psorospermium sp}.\\ 
  I \emph{Mesomycetozoea} (\emph{Ichthyosporea} o DRIP clade) sono un piccolo gruppo di protisti, morfologicamente simili a miceti, per lo più parassiti di pesci e altri animali. \emph{Psorospermium spp}. è stato rinvenuto anche nei tessuti di \emph{A. pallipes}. Il ciclo vitale di questo organismo è solo parzialmente noto; nel gambero si sviluppa da una forma ameboide che subisce aumenti di taglia consecutivi e cambiamenti di forma terminanti con la formazione di una parete cellulare in forme mature, presenti in vari tessuti. L'infezione generalmente non è letale ma può diventarlo in associazione con altri organismi invasivi. Gli individui infetti possono mostrare caratteristiche macchie arancioni sul carapace.

  \item Protisti: \emph{Cothurnia sp}., \emph{C. sieboldii}, \emph{Epistylis sp}.\\ 
  La maggior parte dei protisti trovati sui gamberi sono ciliati ectoparassiti, presenti in grande numero (soprattutto in acque di qualità poco elevata) sul carapace e occasionalmente sulle branchie, tra cui in particolare i Ciliati del genere \emph{Epistylis}. Sono solo raramente associati a fenomeni di mortalità.

  \item Nematodi (non identificati)\\
  I nematodi associati con i gamberi sono generalmente considerati commensali, sono presenti principalmente sulle branchie (come è il caso del nematode non identificato rinvenuto su \emph{A. pallipes}) e non compromettono la sopravvivenza dell'ospite essendo generalmente presenti in numero limitato. 

  \item Artropodi: acari (non identificati) \\
  Gli acari d'acqua dolce hanno un ciclo vitale complesso che include una fase larvale parassita seguita da una protoninfa quiescente , una deutoninfa predatrice, una tritoninfa quiescente, e infine un adulto predatore. Sono in genere presenti sulle branchie, come è il caso dell'acaro non identificato rinvenuto su un esemplare di \emph{A. pallipes}.

  \item Annelidi, Branchiobdellidi: \emph{Branchiobdella astaci}, \emph{B. hexadonta}, \emph{B. italica}, \emph{B. parasitica}, \emph{B. pentadonta}\\
  I branchiobdellidi sono ectocommensali o ectosimbionti dei gamberi, sono ermafroditi e depongono i loro bozzoli sul carapace degli ospiti. L'impatto dei branchiobdellidi sui gamberi è dibattuto, alcuni autori li reputano patogeni (se attaccati alle branchie), altri suggeriscono possano far parte di una simbiosi con la loro attività di pulizia (se sulla superficie corporea) e quindi portare a un aumento del tasso di crescita dell'ospite.

  \emph{Branchiobdella parasitica}, \emph{B. italica} e \emph{B. pentadonta} sono specie commensali e facoltativamente parassite, vivono preferenzialmente sull'esoscheletro dell'ospite mentre \emph{B. astaci} e \emph{B. hexodonta} sono tipicamente presenti sulle branchie. \emph{B. hexodonta} è l'unica specie per la quale sia stato dimostrato un comportamento parassita in quanto si nutre dei tessuti dell'ospite. L'intensità di infestazione da branchiobdellidi è in genere troppo bassa per causare mortalità diretta ma le ferite causate possono rappresentare la via d'ingresso per altri patogeni. 
\end{enumerate}

\subsection{Parassiti e patogeni in popolazioni italiane di \emph{A. pallipes}}
\label{sub_parapat}
Per quel che riguarda la patologia più pericolosa per \emph{A. pallipes}, l'afanomicosi, in Italia, dopo le prime segnalazioni risalenti alla metà del 1800, non sono stati più descritti episodi e pertanto si pensava che questa patologia fosse scomparsa dal territorio nazionale. Tuttavia, nell'estate 2009 sono stati rinvenuti individui di\emph{A. pallipes}, prelevati in conseguenza a gravi morie in torrenti della provincia di Isernia, con gravi lesioni riferibili a peste del gambero. Successivamente, segnalazioni di focolai di infestazione sono pervenute per popolazioni di \emph{A. pallipes} in Valdobbiadene (TV) nel 2010 e sul fiume Chiese, in località Ponte Santa Lucia in Trentino, in località Montevecchia in Lombardia, nel Comune di Crognaleto in Abruzzo, in nel Comune di Rocchetta al Volturno in Molise e in Friuli, nel 2011. Per quel che riguarda le infestazioni in popolazioni alloctone, sono stati segnalati focolai in popolazioni di \emph{P. clarki} in Lombardia e Toscana nel 2008, e in \emph{C. destructor} a Mogliano Veneto (TV) nel 2009-2010, in \emph{C. quadricarinatus} allevati in Sicilia nel 2011 (Quaglio 2011; Pretto 2011) e in \emph{O. limosus} nei laghi di Levico, Caldonazzo e Canzolino in Trentino nel 2012 \cite{Minghetti 2012} \cite{Minghetti 2012b}.

Per quel che riguarda micosi meno devastanti dell'afanomicosi, ma altrettanto rilevanti soprattutto per i programmi di reintroduzione e traslocazione di popolazioni, è stato descritto da Quaglio \etal \cite{Quaglio 2008} un grave episodio di mortalità da Saprolegniaceae in una troticoltura in provincia di Belluno nel corso di una prova sperimentale di allevamento intensivo di \emph{A. pallipes} tra il 2004 e il 2005, che ha causato mortalità nei gamberi tenuti in vasche non correttamente gestite. L'esame microscopico a fresco, micologico, batteriologico e istopatologico ha evidenziato la presenza di Saprolegniaceae in branchie, arti e addome di tutti i campioni, e \emph{Fusarium sp}. in un solo esemplare. La mortalità è stata attribuita ad infezione da \emph{Saprolegniaceae}, sviluppatesi in maniera abnorme a causa della mancata rimozione dei residui di alimento.

Quaglio \etal \cite{Quaglio 2006} hanno analizzato popolazioni di \emph{A. pallipes} raccolte in corsi d'acqua in Emilia Romagna e Friuli Venezia Giulia in estate-autunno 2003, conducendo esami parassitologici, micologici, batteriologici e istopatologici, e rilevando la presenza dei protozoi \emph{Epystilis spp}. e \emph{Cothurnia sieboldii} e di \emph{Branchiobdella italica}, \emph{B. astaci}, \emph{B. parasita} e \emph{B. hexodonta} come ectocommensali nella camera branchiali, filamenti branchiali ed esoscheletro. Sono stati inoltre isolati i miceti \emph{Fusarium spp}. e \emph{Saprolegnia sp}. da lesioni della cuticola. Nei casi in cui sono stati trovati gamberi morti o moribondi, la analisi hanno dimostrato la presenza di \emph{Thelohania contejeani} e del batterio enterico \emph{Citrobacter freundii}. La presenza del microsporide \emph{T. contejeani} in \emph{A. pallipes} è stata anche accertata in diverse popolazioni della Liguria monitorate nel periodo 1993-1999 \cite{Mori 2000}, in provincia di Belluno dal 2004 al 2006 \cite{Quaglio 2011b} e in Trentino dal 2011 al 2012 \cite{Endrizzi 2013}.
Popolazioni di \emph{A. pallipes} portanti sulla superficie ventrale del carapace i branchiobdellidi \emph{B. italica} e \emph{B. parasita}, che però non inficiavano lo stato di salute dei gamberi, sono state registrate nelle province di Firenze, Modena, Parma, Piacenza, Alessandria, Asti, Cuneo, Torino, nel corso degli anni '90 \cite{Gelder 1999}. Numerose popolazioni infestate con \emph{B. italica} sono state rinvenute in Alto Adige nel 1999-2002 \cite{Oberkofler 2002} e nel Lazio nel 2005 \cite{Scalici 2010}, e in Trentino nel 2014-2015 (in popolazioni allevate per il progetto LIFE+TEN, (figura \ref{fig_8}) anche in questo caso i branchiobdellidi erano attaccati esternamente al carapace e non apparivano nutrirsi dei tessuti dell'ospite.

\begin{figure}
  \centering
  \includegraphics[width=.8\columnwidth]{fig_8.jpg}
  \caption{Esemplare di A. pallipes (Trentino, popolazione del Rio Laguna, in Valsugana) con \emph{Branchiobdellidae} ectoparassiti (Foto B. Maiolini).}
  \label{fig_8}
\end{figure}


\subsubsection{Ruolo di \emph{A. pallipes} nella trasmissione di SEV e NEI nei salmonidi}
La Setticemia Emorragica Virale (SEV) e la Necrosi Ematopoietica Infettiva (NEI) sono due infezioni sistemiche ad eziologia virale che colpiscono molte specie di salmonidi. La Direttiva 67/1991 CE, che stabilisce le norme per la commercializzazione di animali e prodotti d'acquacoltura, sancisce che episodi di infezione sono soggetti a notifica obbligatoria in tutti gli Stati Membri. SEV e NEI sono inoltre oggetto di sorveglianza in Italia secondo Il Decreto Legislativo 148/2008, che obbliga gli allevamenti che effettuano semine di materiale ittico in acque pubbliche a ottenere e mantenere l'indennità. Al fine di verificare il ruolo di \emph{A. pallipes} come possibile vettore degli agenti di SEV e NEI, \cite{Minghetti 2012b} ha svolto un'indagine prelevando 35 gamberi da 2 troticolture diagnosticate infette per tali patologie. Tutti gli esemplari analizzati sono risultati negativi suggerendo che \emph{A. pallipes} non giochi un ruolo nella trasmissione di tali patologie tra i salmonidi. Indagini più approfondite, condotte su un numero maggiore di popolazioni potenzialmente infette, sarebbero però necessarie per poter escludere con più certezza tale eventualità.

\subsection{Allevamento}
L'allevamento di gamberi è un metodo molto efficiente per ottenere, in condizioni ottimali, un elevato numero d'individui a costi relativamente bassi. Il target dell'allevamento è la produzione di gamberi di precise categorie di età, adatte per usi vari (inizio di altre colture, reintroduzioni, consumo alimentare, ecc.) \cite{Keller 1988}. Quando l'allevamento tiene in considerazione tutti gli aspetti biologici, fisiologici e genetici di specie autoctone di gambero, può rappresentare un valido supporto alla conservazione, protezione e ampliamento delle popolazioni native \cite{Policar 2015}. Questo tipo di approccio è utilizzato in numerosi Paesi europei per specie di Astacidae. In Italia, numerosi centri di riproduzione di \emph{A. pallipes} sono stati creati nell'ambito di diversi progetti di conservazione e ripopolamento, come il Progetto Europeo LIFE CRAINat che ha previsto la realizzazione di otto centri distribuiti in Abruzzo, Lombardia e Molise \cite{AA.VV. 2014} e il Progetto Europeo LIFE RARITY per il quale sono stati avviati 2 centri in Friuli-Venezia Giulia \cite{RARITY}. In Italia, i centri di riproduzione di A. pallipes sono quindi strutture destinate alla fecondazione, all'incubazione delle uova (naturale o artificiale) e all'allevamento dei giovani destinati alla reintroduzione/rinforzo (\emph{sensu} IUCN) in corsi d'acqua dove le popolazioni originarie sono scomparse e/o in rarefazione. Gli allevamenti sono di due tipologie, ovvero tutte le fasi del ciclo vitale possono essere allevate a) in ambiente artificiale (strutture all'interno di un edificio), b) in ambiente semi-naturale (strutture in ambiente esterno). Spesso gli allevamenti utilizzano impianti di piscicolture già esistenti e, in genere, sono realizzati in modo intensivo, ovvero ogni categoria di età viene allevata con elevate densità di individui in vasche speciali attrezzate con un sistema di ricircolo dell'acqua, che assicura temperatura e ossigenazione ottimali, e con ripari artificiali; l'alimentazione dei gamberi è artificiale e controllata. Le vasche, generalmente di forma rettangolare o quadrata, possono essere realizzate in cemento, mattoni o in vetroresina e devono essere, qualora la collocazione sia esterna, ombreggiate per mantenere idonea la temperatura dell'acqua nei mesi estivi. Va comunque menzionata la possibilità di allevare i gamberi anche in modo estensivo, ovvero con gamberi di varie classi di età tenuti insieme in stagni naturali senza la necessità di fornire alimento artificiale supplementare.
La qualità dell'acqua è fondamentale per il successo dell'allevamento di \emph{A. pallipes}. I valori raccomandati di tempertaura, ossigeno disciolto, pH e ioni Ca++ \cite{AA.VV. 2014} sono riportati in tabella \ref{tab_2}. La concentrazione di calcio è particolarmente importante, in quanto il calcio è indispensabile, oltre che per le normali funzioni cellulari, per la costituzione di un esoscheletro con un livello di calcificazione adeguato. Nel caso di acque a bassa durezza, oltre all'apporto minerale fornito con il mangime artificiale e necessario provvedere alla distribuzione di molluschi acquatici (es. \emph{Lymnaea}) in vasca che, oltre all'integrazione nutritiva in sostanze proteiche, con la loro conchiglia calcarea rimediano a questa carenza \cite{De Luise 2012}.

\begin{table}[]
\centering
\begin{tabular}{@{}lll@{}}
\toprule
\textbf{Parametro}       & \textbf{Intervallo o valore ottimale} & \textbf{Soglia limite}    \\ \midrule
Unità  di pH  & 6,8/8,2         & \textless 6 e \textgreater  9   \\
\rowcolor[HTML]{EFEFEF}  Ossigeno disciolto (ppm) & 6/saturazione   & -   \\
Temperatura ($^{\circ}$C)        & 14/15           & \textless 0 e \textgreater 20  \\
\rowcolor[HTML]{EFEFEF}  Ca+2 (ppm)    & 50/100          & \textless 5 e \textgreater 130 \\ \bottomrule
\end{tabular}
\caption{Parametri chimico-fisici richiesti per l'allevamento di \emph{A. pallipes}}
\label{tab_2}
\end{table}

Poichè negli allevamenti viene utilizzata preferibilmente acqua di superficie o di falda, va tenuto presente che l'acqua di falda o di risorgiva ha un intervallo termico limitato e costante durante tutto l'anno mentre le acque di superficie seguono la temperatura dell'aria, con effetti molto importanti sui tempi del ciclo biologico della specie e dell'accrescimento del novellame. La temperatura dell'acqua ha una grande iportanza nel determinare i tempi di incubazione delle uova. La durata standard per \emph{A. pallipes} è di 120-226 giorni (1500-1716 gradi/giorno), ma la manipolazione della temperatura può accelerare altre fasi del ciclo biologico. Con acqua a temperatura costante (generalmente 12-13 $^{\circ}$C) le uova schiudono con un paio di mesi in anticipo, all'inizio della primavera, ma i gamberi giovani si accrescono più lentamente rispetto a un allevamento con acqua superficiale, dove le temperature estive sono più elevate (anche 18-20 $^{\circ}$C). 

L'allevamento segue delle fasi che corrispondono al ciclo biologico dei gamberi: riproduzione, incubazione delle uova, crescita dei piccoli (tabella \ref{tab_3}). Gli allevamenti, sia in incubatoio che in condizioni semi-naturali all'esterno, devono disporre di un numero sufficiente di vasche per poter stabulare separatamente: i) i riproduttori (maschi e femmine possono essere allevati separatamente o insieme, con un rapporto da 1:1 a un massimo di 1:4); ii) gli animali in accoppiamento (da ottobre a metà dicembre); iii) le femmine ovigere e/o con larve; iv) i giovani per il periodo di sviluppo fino alla liberazione in natura. 

\begin{table}[]
\centering
\begin{tabular}{p{.2\columnwidth}p{.3\columnwidth}p{.3\columnwidth}l@{}}
\toprule
\textbf{Periodo}  & \textbf{Vasche interne} & \textbf{Vasche o stagni esterni}        \\ \midrule
Agosto-settembre  & Cattura dei riproduttori & Cattura dei riproduttori \\
\rowcolor[HTML]{EFEFEF}  Settembre-ottobre & Mantenimento dei gamberi separati per sesso & Maschi e femmine insieme \\
Ottobre & Maschi e femmine insieme & -- \\
\rowcolor[HTML]{EFEFEF} Dicembre          & Monitoraggio delle femmine ovigere e allontanamento dei maschi in altre vasche & -- \\
Aprile-maggio     & Monitoraggio delle femmine ovigere e ritorno dei maschi nei torrenti di origine & Monitoraggio delle femmine ovigere e prelievo dei maschi per il ritorno nei torrenti di origine \\
\rowcolor[HTML]{EFEFEF}Luglio & Verifica della presenza dei piccoli e allontanamento delle femmine in altre vasche o ritorno nei torrenti di origine & Verifica della presenza dei piccoli e allontanamento delle femmine in altre vasche o ritorno nei torrenti di origine \\
Settembre & Prelievo dei piccoli gamberi di circa 3 mesi di età & Prelievo dei piccoli gamberi di circa 3 mesi di età \\ \bottomrule
\end{tabular}
\caption{Schema delle fasi di allevamento sulla base della diversa tipologia di impianto (da \cite{AA.VV. 2014})}
\label{tab_3}
\end{table}

Durante il periodo dell'accoppiamento, i gamberi devono essere mantenuti in condizioni di semioscurità, con cicli luce/oscurità 8:16 o, se le vasche sono all'aperto, queste devono essere coperte con materiale che permetta il passaggio della luce per mantenere un ritmo cicardiano di illuminazione. Vanno inoltre forniti un numero appropriato di ricoveri (è raccomandabile almeno un ricovero per ogni gambero), in genere rappresentati da tubi in PVC \cite{Policar 2015}; inoltre, se possibile vanno immessi vegetali acquatici. Dopo l'accoppiamento, le femmine ovigere vanno separate e stabulate in speciali gabbie dette "da parto", sospese in acqua e con fondo forato per permettere alle larve appena staccate dalla madre di cadere sul fondo senza venire predate. É importante che, durante la fase di incubazione delle uova, le femmine siano mantenute in condizioni ottimali (di temperatura, ossigenazione e qualità dell'acqua, illuminazione; con presenza di appropriati rifugi, e in isolamento da altre femmine e da maschi) per evitare ogni stress in quanto femmine troppo attive o nervose possono perdere tutte le uova (fenomeno che però può avvenire anche naturalmente per insufficiente attaccamento delle uova ai pleiopodi, e che può interessare dal 18 al 90\% delle uova a seconda delle condizioni di coltura, si veda la bibliografia in Policar \& Kozák \cite{Policar 2015}).

Le vasche utilizzate per allevare i piccoli gamberi fino al raggiungimento della taglia di 2 cm (taglia alla quale si procede al rilascio in natura) vanno fornite di ricoveri artificiali (mattoni forati posizionati lungo tutto il perimetro del contenitore, unitamente ad altri tipi di nascondigli costituiti da tunnel cilindrici scavati in materiali plastici, da trucioli in PVC e da vegetazione acquatica). Il mangime (pellettato commerciale completo e bilanciato, che deve rimanere compatto in acqua per almeno 48 ore) va depositato in una o più mangiatoie costituite da un piattino colorato, utile per controllare e tarare la somministrazione del cibo evitando di inquinare l'ambiente di allevamento. L'alimento viene distribuito giornalmente in quantità pari al 5\% della biomassa dei riproduttori presenti, all'1\% di quella delle femmine ovigere e ad libitum per le larve. L'alimento artificiale va integrato con vegetali acquatici (principalmente del genere Fontinalis ed Elodea) immessi nelle vasche e ricchi di micro e macrofauna, e con naupli vivi di Artemia salina distribuiti ad libitum alle sole larve nella prima settimana di vita indipendente, o in alternativa con alimento omogeneizzato per uso umano. 

Dettagli sulle caratteristiche di costruzione degli impianti di allevamento e sulle modalità operative testate in popolazioni italiane sono descritti nei risultati del progetto LIFE CRAINat \cite{AA.VV. 2014}, da utilizzare come riferimento sia per l'allevamento in incubatoio che in semi-intensivo all'aperto.


\section{Specie autoctone e alloctone in Provincia di Trento}
Dai dati di rilievo delle popolazioni di gambero in Trentino svolte sia nell'ambito del progetto LIFE+TEN \cite{Endrizzi 2014} che in altre ricerche recenti \cite{Ciutti 2013} \cite{Endrizzi 2013} \cite{Cappelletti 2014}, risultano presenti sul territorio della Provincia di Trento tre specie di gamberi, descritte brevemente qui di seguito (figura \ref{fig_9}) (per maggiori dettagli identificativi, si veda Mazzoni \etal, 2004 \cite{Mazzoni 2004}).

\subsection{\emph{Austropotamobius pallipes}} 
Specie nativa dell'Europa, compresa l'Italia peninsulare. Di medie dimensioni, può raggiungere, e in alcuni casi superare, i 12 cm di lunghezza totale. Presenta carapace granuloso, con alcune spine nella parte posteriore del solco cefalico, e un solo paio di creste post orbitali, sebbene in alcuni casi risulti leggermente visibile un secondo paio. Il rostro è breve e liscio, con bordi divergenti dalla regione apicale verso quella oculare, denti laterali piccoli e cresta mediana liscia e poco marcata. Le chele, robuste e granulose, presentano margine interno irregolare e colore più chiaro nella parte ventrale rispetto al resto del corpo. La colorazione del corpo, generalmente bruna o olivastra, può variare notevolmente. Per ulteriori dettagli su morfologia, biologia, ecologia della specie si veda il capitolo \ref{chap_intro}.

\subsection{\emph{Orconectes limosus}}
Specie aliena invasiva originaria degli Stati Uniti orientali, è stata introdotta in Europa verso la fine del XIX secolo. E' attualmente segnalata in 22 Paesi europei. Di piccole-medie dimensioni misura mediamente 6 - 9 cm ma può raggiungere anche 12 cm di lunghezza. Presenta carapace liscio, caratterizzato da spine in corrispondenza del solco cefalico, e un solo paio di creste post orbitali con spine apicali. Il rostro presenta bordi quasi paralleli con denti laterali ben evidenti, apice liscio senza cresta mediana. Le chele sono poco sviluppate, lisce, con margine interno regolare, punta uncinata con bande nere-arancio e presentano una spina a livello del carpo. Di colore bruno o olivastro, presenta caratteristiche bande rossastre trasversali sulla porzione dorsale di ciascun segmento addominale.

Si tratta di una specie molto aggressiva, a crescita rapida e sviluppo precoce. Il periodo di vita medio si aggira attorno ai due-tre anni, sebbene possano in alcuni casi superare i quattro anni. La femmina, molto feconda, è in grado di produrre oltre 400 uova. E' in grado di vivere bene anche in ambienti degradati, caratterizzati da acque inquinate, scarsamente ossigenate ed eutrofiche. Sono scavatori e possono quindi essere responsabile dell'instabilità delle sponde. Date le origini americane, questa specie è un potenziale vettore della peste del gambero.

\subsection{\emph{Procambarus clarkii}}
Specie aliena invasiva, originaria del Sud degli Stati Uniti, è stata introdotta in Spagna per l'allevamento nel 1973 ed è attualmente segnalata in 13 Paesi Europei.

Di dimensioni medio - grandi, misura tra i 10 e i 20 cm di lunghezza totale. Il carapace è granuloso con un solo paio di creste post-orbitali e numerose spine laterali. Il rostro presenta bordi divergenti dalla regione apicale verso quella oculare, con denti laterali presenti ma piccoli. Le chele sono sviluppate, con margine interno irregolare e sono ricoperte da spine e tubercoli. L'adulto presenta colorazione rossa con bande scure nella porzione dorsale dell'addome mentre i giovani sono di colore grigiastro. Si tratta di una specie che presenta elevata fecondità e vita breve; in condizioni naturali vivono infatti dai 12 ai 18 mesi. Molto aggressiva, può avere gravi impatti sull'equilibrio degli ecosistemi invasi causando la drastica riduzione di vegetazione, invertebrati acquatici e anfibi e, di conseguenza, alcune specie di uccelli acquatici. E' in grado di vivere bene in ambienti fortemente degradati con acque poco ossigenate, salmastre e inquinate sopportando anche ampie escursioni termiche e sopravvivendo per lunghi periodi fuori dall'acqua, in condizioni di umidità. Predilige ambienti caratterizzati da substrati melmosi, in cui esercita un'intensa attività di scavo per la costruzione delle tane causando gravi problemi di instabilità delle sponde e danni alle coltivazioni aggravati anche dall'intensa attività di foraggiamento. E' vettore della peste del gambero e di patogeni e tossine nocive anche per l'uomo \cite{Mazzoni 2004} \cite{Souty-Grosset 2006} \cite{Kouba 2014}.

\begin{figure}
  \centering
  \includegraphics[width=.9\columnwidth]{fig_9.png}
  \caption{Chiave dicotomica e caratteri identificativi delle specie di gambero presenti in Trentino (foto: S. Endrizzi, F. Mezzo, Mazzoni et al., 2004)}
  \label{fig_9}
\end{figure}

\section{Aspetti normativi}
\label{sec_normativa}
\emph{Austropotambius pallipes} è inserito nella Lista Rossa della IUCN e classificato come "Endangered A2ce" ossia minacciato di estinzione a causa di una riduzione della popolazione superiore al 50\% osservata negli ultimi 10 anni e dovuta al declino degli habitat disponibili e all'introduzione di competitori e parassiti alloctoni (\cite{IUCN 2013}). La tutela di \emph{A. pallipes} rientra nel quadro più ampio delle strategie comuni a livello globale e delle normative per la protezione dell'ambiente e della biodiversità. A livello internazionale sono state promulgate diverse convenzioni attinenti a questi scopi:
\begin{enumerate}[label=\roman*]
  \item la Convenzione di Ramsar (1971), ufficialmente "Convenzione per la tutela delle zone umide di importanza internazionale", ratificata in Italia ed entrata in esecuzione attraverso il D.P.R. n. 448 del 13/03/1978 ("Esecuzione della Convenzione relativa alle zone umide di importanza internazionale, firmata a Ramsar il 02/02/1971"), rappresenta l'unico trattato internazionale moderno per la tutela delle zone umide, e quindi implicitamente per la tutela degli habitat preferenziali del gambero di fiume; 
  \item la Convenzione di Berna (1979), ufficialmente " Convenzione per la conservazione della vita selvatica e dei suoi biotopi in Europa", ratificata in Italia attraverso la L. n. 503 del 05/08/1981 (Ratifica ed esecuzione della convenzione relativa alla conservazione della vita selvatica e dell'ambiente naturale in Europa, con allegati, adottata a Berna il 19 settembre 1979),  promuove la cooperazione dei Paesi sottoscriventi nella conservazione delle specie selvatiche e dei loro habitat naturali, e richiede ai paesi sottoscriventi di "controllare strettamente l'introduzione di specie non-native (articolo 11.2.b)". E' quindi applicabile per il controllo dell'introduzione di gamberi alieni; inoltre, nell'appendice \cellcolor{Goldenrod}IIIvengono elencate le specie di fauna selvatica per le quali gli stati sottoscriventi devono regolamentare lo sfruttamento, onde non compromettere la sopravvivenza di tali specie, adottando misure quali il divieto temporaneo o locale di sfruttamento, ove necessario, onde ripristinare una densità soddisfacente delle popolazioni. \emph{A. pallipes} è inserito nell'allegato III.
  \item la Convenzione sulla Diversità Biologica di Rio de Janeiro (1992), entrata in vigore in Italia con L. n. 309 del 13/12/1993, rettifica L. n. 82 del 25/03/1994. Gli obiettivi della convenzione sono la conservazione della diversità biologica, l'utilizzazione durevole dei suoi elementi e la ripartizione giusta ed equa dei vantaggi derivanti dallo sfruttamento delle risorse genetiche. La convenzione sottolinea il ruolo delle comunità locali e delle popolazioni autoctone in materia di conservazione della biodiversità e impegna le parti a introdurre misure di controllo ed eliminazione delle specie aliene nocive e a prevenire ulteriori invasioni.
\end{enumerate}

A livello europeo la tutela del gambero di fiume e/o i suoi habitat sono inclusi specificatamente nella Direttiva Habitat (Direttiva n. 92/43/CEE relativa alla conservazione degli habitat naturali e seminaturali e della flora e della fauna selvatiche), approvata il 21 maggio 1992 dalla Commissione europea, recepita in Italia dal D.P.R. n. 357 del 08/09/1997, modificato ed integrato dal D.P.R. n. 120 del 12/03/2003. Scopo della Direttiva Habitat è "salvaguardare la biodiversità mediante la conservazione degli habitat naturali, nonché della flora e della fauna selvatiche nel territorio europeo degli Stati membri al quale si applica il trattato" (art 2). Per il raggiungimento di questo obiettivo la Direttiva stabilisce misure volte ad assicurare il mantenimento o il ripristino, in uno stato di conservazione soddisfacente, degli habitat e delle specie di interesse comunitario elencati nei suoi allegati. La Direttiva è costruita intorno a due pilastri: la costituzione della "Rete Natura 2000", un network di aree protette che collega e tutela tutti i Siti d'Importanza Comunitaria (SIC), i siti mirati alla conservazione di habitat e specie elencati rispettivamente negli allegati I e II, e il regime di tutela delle specie elencate negli allegati \cellcolor{BurntOrange}IVe V. A. pallipes  è inserito negli allegati \cellcolor{LimeGreen}II e \cellcolor{Red}Vdella Direttiva Habitat (specie animali e vegetali d'interesse comunitario la cui conservazione richiede la designazione di zone speciali di conservazione (All. II) e il cui prelievo e sfruttamento in natura potrebbe formare oggetto di misure di gestione (All. V). 

In Trentino la "Legge provinciale sulle foreste e sulla protezione della natura" (Legge Provinciale 23 maggio 2007 n. 11) è finalizzata a migliorare la stabilità fisica e l'equilibrio ecologico del territorio forestale e montano, nonché a conservare e a migliorare la biodiversità espressa dagli habitat e dalle specie, attraverso una equilibrata valorizzazione della multifunzionalità degli ecosistemi, al fine di perseguire un adeguato livello possibile di stabilità dei bacini idrografici, dei corsi d'acqua e di sicurezza per l'uomo, di qualità dell'ambiente e della vita e di sviluppo socio-economico della montagna. La legge provinciale nel titolo IV, capo II, art 26 (tutela della fauna) sancisce che"\dots  è vietato uccidere, distruggere, danneggiare, catturare, detenere e commerciare esemplari o parti di essi, in qualsiasi stadio di sviluppo, appartenenti alle specie animali individuate dal regolamento\dots". Il decreto n. 23-25/Leg 2009 del Presidente della Provincia definisce il regolamento di attuazione del titolo IV, capo \cellcolor{LimeGreen}II (della legge provinciale sopra citata, sancendo che "\dots sono protette tutte le specie di anfibi, di rettili, nonché le specie di invertebrati elencate nell'allegato C a questo regolamento". \emph{A. pallipes} è elencato nell'allegato C.

\chapter{Studio e conservazione di \emph{Austropotamobius pallipes} in Trentino}
\DeactivateBG
\AddToShipoutPicture*{\BackgroundPicTwo}

Il gambero di fiume è per dimensioni il maggiore rappresentante della comunità macrobentonica dei corsi d'acqua e dei laghi del Trentino costituendo un elemento fondamentale per il mantenimento dell'equilibrio degli ecosistemi acquatici. Ampiamente diffuso e quindi ben conosciuto dalle comunità locali fino agli anni ‘50, è andato via via riducendo la sua presenza fino a divenire quasi del tutto sconosciuto alle nuove generazioni. In considerazione dell'elevato grado di minaccia attribuito alla specie in Europa e alle scarse conoscenze sullo status del gambero in Provincia, sono stati di recente condotti studi e monitoraggi \cite{Ciutti 2013} \cite{Endrizzi 2013}, utili a ottenere dati aggiornati sulla distribuzione e sulle caratteristiche delle popolazioni ancora presenti. Nell'ambito dell'azione C10 del Progetto Europeo Life+ TEN, "Azione dimostrativa di tutela di specie: salvaguardia delle popolazioni autoctone di gambero di fiume", sono inoltre stati attuati degli interventi di reintroduzione e di riqualificazione di habitat volti a promuovere la conservazione della specie sul territorio.

\section{Distribuzione storica e attuale} 
In Trentino è stata attualmente accertata la presenza di trenta popolazioni di gambero di fiume A. pallipes diffuse in otto bacini idrografici: Adige, Avisio, Brenta, Cismon, Chiese, Fersina, Noce e Sarca (tabella \ref{tab_4}). 
Le popolazioni rilevate occupano diverse tipologie di habitat quali fiumi, torrenti, ruscelli, laghi e stagni distribuiti tra i 200 e i 1200 m di quota. Il confronto tra i dati storici relativi alla presenza della specie nelle aree indagate della Provincia, ottenuti da fonti bibliografiche (\cite{Albrecht 1982} \cite{Pagotto 1995} \cite{Fratini 2005} \cite{Paoli 2008} e Formulari Natura 2000) e da interviste alla popolazione locale, e quelli ricavati dai recenti studi hanno però evidenziato una forte contrazione dell'areale di distribuzione del gambero. Due popolazioni della specie nativa sono state sostituite dalla specie alloctona \emph{O. limosus} nei laghi di Caldonazzo e Levico (figura 10). In quest'ultimo, è stata registrata una risalita di alcuni esemplari nel tratto finale del Rio Vignola, affluente del lago di Levico, intermittente nel tratto a monte. Trentotto popolazioni segnalate in passato sono risultate estinte (figura \ref{fig_10}) e, oltre il 70\% di queste sarebbero scomparse nel corso degli ultimi quindici anni. I risultati della ricerca indicano che in Trentino le popolazioni native di A. pallipes sono quasi del tutto scomparse dai grandi fiumi di fondovalle, che maggiormente risentono delle alterazioni delle caratteristiche fisiche e idrologiche e dell'inquinamento e dove maggiori sono gli impatti generati dalle attività antropiche e dalla diffusione di specie aliene invasive, in accordo con i dati riportati in letteratura che identificano il degrado degli habitat come una delle principali cause di estinzione di questa specie in Europa \cite{Fureder 2003b} \cite{Renai 2006} \cite{Sint 2007} \cite{Trouilhe 2007} \cite{Brusconi 2008} \cite{Aquiloni 2010}. Le popolazioni residue risultano attualmente isolate nei piccoli corsi d'acqua e bacini montani che ancora conservano un'elevata naturalità.

\footnotesize
\begin{longtable}[c]{p{.1\columnwidth}p{.05\columnwidth}p{.15\columnwidth}p{.15\columnwidth}p{.008\columnwidth}p{.06\columnwidth}p{.06\columnwidth}p{.04\columnwidth}p{.06\columnwidth}p{.06\columnwidth}l}
\toprule
{\cellcolor{white}}\textbf{\rotatebox[origin=c]{90}{Data rilevamento}} & \textbf{\rotatebox[origin=c]{90}{Bacino}} & \textbf{\rotatebox[origin=c]{90}{Corpo idrico}} & \textbf{\rotatebox[origin=c]{90}{Vincolo di tutela habitat}} & \textbf{\rotatebox[origin=c]{90}{Quota}} & \textbf{\rotatebox[origin=c]{90}{Latitudine}} & \textbf{\rotatebox[origin=c]{90}{Longitudine}} & \textbf{\rotatebox[origin=c]{90}{Specie}} & \textbf{\rotatebox[origin=c]{90}{Dati storici}} & \textbf{\rotatebox[origin=c]{90}{Caratteristiche idrologiche}} & \textbf{\rotatebox[origin=c]{90}{Fonte}} \\* \midrule
\endfirsthead
%
\multicolumn{11}{l}{\bfseries \footnotesize Continua dalla pagina precedente} \\
\toprule
\textbf{\rotatebox[origin=c]{90}{Data rilevamento}} & \textbf{\rotatebox[origin=c]{90}{Bacino}} & \textbf{\rotatebox[origin=c]{90}{Corpo idrico}} & \textbf{\rotatebox[origin=c]{90}{Vincolo di tutela habitat}} & \textbf{\rotatebox[origin=c]{90}{Quota}} & \textbf{\rotatebox[origin=c]{90}{Latitudine}} & \textbf{\rotatebox[origin=c]{90}{Longitudine}} & \textbf{\rotatebox[origin=c]{90}{Specie}} & \textbf{\rotatebox[origin=c]{90}{Dati storici}} & \textbf{\rotatebox[origin=c]{90}{Caratteristiche idrologiche}} & \textbf{\rotatebox[origin=c]{90}{Fonte}} \\* \midrule
\endhead
\rowcolor[HTML]{EFEFEF} 30/08/2012 & Adige  & Fossa Maestra Terlago & \texttt{ZSC IT3120110} (in parte, oltre a aree non vincolate) &  541 & 5106385  & 658800 & A. p.  &   & P & 1   \\
19/07/2012 & Adige  & Lago Cei   & \texttt{ZSC IT3120081}  & 918 & 5090570  & 656528 & A. p.  &   & P & 1   \\
\rowcolor[HTML]{EFEFEF} 19/07/2012 & Adige  & Lago Lagabis   & Nessuno & 925 & 5090455  & 656673 & nn   &   & P & 2   \\
30/08/2011 & Adige  & Lago Lamar & \texttt{ZSC IT3120087}  & 725 & 5110651  & 659411 & ex   & 2007  & P & 1   \\
\rowcolor[HTML]{EFEFEF} 20/09/2011 & Adige  & Lago Santo Lamar & \texttt{ZSC IT3120087}  & 717 & 5109924  & 659091 & A. p.  &   & P & 1   \\
2012  & Adige  & Lago Terlago   & \texttt{ZSC IT3120110}  & 418 & 5106391  & 658697 & ex   & 2002  & P & 2   \\
\rowcolor[HTML]{EFEFEF} 29/07/2011 & Adige  & Rio - Zambana Vecchia & Nessuno & 218 & 5113525  & 659960 & nn   &   & P & 1   \\
6/7/2011  & Adige  & Rio Carpine & Nessuno & 378 & 5108860  & 663857 & A. p.  &   & P & 1   \\
\rowcolor[HTML]{EFEFEF} 29/07/2011 & Adige  & Rio Castel Monreale   & Nessuno & 365 & 5118858  & 666049 & nn   &   & P & 1   \\
21/07/2011 & Adige  & Rio Cortesano  & Nessuno & 615 & 5109933  & 665081 & nn   &   & P & 1   \\
\rowcolor[HTML]{EFEFEF} 26/07/2011 & Adige  & Rio dei Gamberi & Nessuno & 656 & 5111528  & 665775 & ex   & 1950  & P & 1   \\
29/07/2011 & Adige  & Rio di Faedo   & Nessuno & 489 & 5117918  & 666255 & nn   &   & P & 1   \\
\rowcolor[HTML]{EFEFEF} 29/07/2011 & Adige  & Rio Netta  & Nessuno & 221 & 5111647  & 660017 & nn   &   & P & 1   \\
18/08/2009 & Adige  & Rio Salà   & Nessuno & 578 & 5103755  & 667792 & nn   &   & P & 1   \\
\rowcolor[HTML]{EFEFEF} 2012  & Adige  & Roggia Bondone & Nessuno & 311 & 5097193  & 663136 & nn   &   & P & 1   \\
20/09/11  & Adige  & Roggia Gardolo & Nessuno & 325 & 5108458  & 663782 & A. p.  &   & P & 1   \\
\rowcolor[HTML]{EFEFEF} 20/09/2011 & Adige  & Roggia Gardolo 2 & Nessuno & 199 & 5107761  & 663517 & A. p.  &   & P & 1   \\
30/08/2012 & Adige  & Roggia Terlago & \texttt{ZSC IT3120110} (in parte, oltre a aree non vincolate) & 521 & 5107010  & 657192 & ex   & 2011  & P & 1   \\
\rowcolor[HTML]{EFEFEF} 2012  & Adige  & Sorgente Fersura & \texttt{ZSC-ZPS IT3120156}   & 178 & 5081571  & 655709 & ex   & 2008  & P & 2   \\
19/07/2012 & Adige  & Torrente Arione & Nessuno & 907 & 5091260  & 657516 & A. p.  &   & P & 2   \\
\rowcolor[HTML]{EFEFEF} 28/09/2011 & Adige  & Torrente Valsorda   & Nessuno & 330 & 5097030  & 665916 & A. p.  &   & P  & 1   \\
20/2/2017 & Avisio & Alta Val Camesio & Nessuno & 1060  & 5129899  & 690748 & ex   & 1975 ca &  & 3   \\
\rowcolor[HTML]{EFEFEF} 20/2/2017 & Avisio & Bellamonte & Nessuno & 1366  & 5132064  & 704724 & ex   & 1975 ca &  & 3   \\
11/7/2012 & Avisio & Lago Santo Cembra   & Nessuno & 1199  & 5118102  & 670439 & A. p.  &   & P & 1   \\
\rowcolor[HTML]{EFEFEF} 20/2/2017 & Avisio & Palui de Malgola & Nessuno & 1400  & 5130270  & 702507 & ex   & 1966  &  & 3   \\
15/07/2012 & Avisio & Rio di Regnana & Nessuno & 627 & 5116573  & 674725 & nn   &   & P & 1   \\
\rowcolor[HTML]{EFEFEF} 3/8/2012  & Avisio & Rio Favorine   & Nessuno & 774 & 5118418  & 673032 & nn   &   & P & 1   \\
7/8/2012  & Avisio & Rio Fornei & Nessuno & 747 & 5115483  & 667700 & A. p.  &   & P & 1   \\
\rowcolor[HTML]{EFEFEF} 2/8/2012  & Avisio & Rio Lisignago  & Nessuno & 518 & 5114680  & 668363 & nn   &   & P & 1   \\
29/07/2012 & Avisio & Rio Mercal & Nessuno & 958 & 5117295  & 671484 & ex   & 2010  & P & 1   \\
\rowcolor[HTML]{EFEFEF} 3/8/2012  & Avisio & Rio Molino & Nessuno & 809 & 5121994  & 676802 & nn   &   & P & 1   \\
26/07/2011 & Avisio & Rio Pramalga   & Nessuno & 641 & 5111157  & 666341 & nn   &   & P & 1   \\
\rowcolor[HTML]{EFEFEF} 7/7/2012  & Avisio & Rio Scorzai & Nessuno & 739 & 5116220  & 670550 & ex   & 1990  & P & 1   \\
26/07/2011 & Avisio & Rio Vallalta   & Nessuno & 542 & 5112566  & 666712 & nn   &   & S & 1   \\
\rowcolor[HTML]{EFEFEF} 13/08/2012 & Avisio & Rivo di Brusago & Nessuno & 635 & 5121028  & 678187 & nn   &   & P & 1   \\
12/1/2017 & Avisio & Rosta Cavazal  & Nessuno & 857 & 5128845  & 690433 & A. p.  &   & P & 3   \\
\rowcolor[HTML]{EFEFEF} 20/2/2017 & Avisio & Rosta dei Ciasani   & Nessuno & 907 & 5128490.89 & 693879.32 & ex   & 1966  &  & 3   \\
20/2/2017 & Avisio & Rosta del Novelli   & Nessuno & 878 & 5127530  & 688068 & ex   & 1966  &  & 3   \\
\rowcolor[HTML]{EFEFEF} 20/2/2017 & Avisio & Rosta del Sgneches & Nessuno & 926 & 5128631  & 696504 & ex   & 1966  &  & 3   \\
13/08/2012 & Avisio & Torrente Avisio - Loc Piscine  & Nessuno & 614 & 5121825  & 678053 & ex   & 2000  & P & 1   \\
\rowcolor[HTML]{EFEFEF} 2012  & Brenta & Canali risorgiva - Fontanazzo  & \texttt{ZSC-ZPS IT3120030} & 262 & 5099067  & 701434 & ex   & 2006  & P & 2   \\
2/7/2010  & Brenta & Fiume Brenta   & Nessuno & 438 & 5096997  & 679293 & nn   &   & P & 1   \\
\rowcolor[HTML]{EFEFEF} 25/08/2011 & Brenta & Fiume Brenta tra Grigno - Levico & Nessuno & 242 & 5097854  & 704818 & ex   & 1950  & P & 1   \\
2/8/2011  & Brenta & Lago Caldonazzo & \texttt{ZSC IT3120042} e aree non vincolate   & 449 & 5099398  & 673724 & O. l., ex  & 2004  & P & 1   \\
\rowcolor[HTML]{EFEFEF} 2/8/2011  & Brenta & Lago Levico e tratto terminale affluente Rio Vignola & \texttt{ZSC IT3120039} e aree non vincolate   & 447 & 5099616  & 675333 & O. l., ex  & 2004  & P & 1, 2  \\
4/8/2015  & Brenta & Rio Laguna & Nessuno & 280 & 5099200  & 703244 & A. p.  &   & P & 1   \\
\rowcolor[HTML]{EFEFEF} 1/7/2010  & Brenta & Rio Mandola & Nessuno & 648 & 5096369  & 671372 & nn   &   & P & 1   \\
1/7/2010  & Brenta & Rio Merdar & Nessuno & 452 & 5100707  & 672653 & nn   &   & P & 1   \\
\rowcolor[HTML]{EFEFEF} 2/7/2010  & Brenta & Rio S Giuliana & Nessuno & 443 & 5096617  & 679263 & nn   &   & S & 1   \\
26/07/2012 & Brenta & Rio Vena - Inghiaie   & \texttt{ZSC-ZPS IT3120038}   & 450 & 5096486  & 679036 & ex   & 2011  & P & 2   \\
\rowcolor[HTML]{EFEFEF} 21/06/2012 & Brenta & Rio Vignola & Nessuno & 455 & 5099700  & 675449 & O. l.  &   & P & 2   \\
1/7/2010  & Brenta & Rio Vignola & \texttt{ZSC IT3120043} & 561 & 5101389  & 674742 & nn   &   & S & 1   \\
\rowcolor[HTML]{EFEFEF} 25/08/2011 & Brenta & Stagni Bellasio & Nessuno & 241 & 5098449  & 703730 & nn   &   & S & 1   \\
1/7/2010  & Brenta & Torrente Centa & Nessuno & 550 & 5093690  & 673982 & nn   &   & P & 1   \\
\rowcolor[HTML]{EFEFEF} 25/08/2011 & Brenta & Torrente Larganza - Roncegno   & Nessuno & 153 & 5102331  & 685948 & nn   &   & P & 1   \\
25/08/2011 & Brenta & Torrente Resenzuola   & Nessuno & 232 & 5097344  & 706249 & ex   & 1995  & P & 1   \\
\rowcolor[HTML]{EFEFEF} 1/7/2010  & Brenta & Torrente Trambario & Nessuno & 692 & 5095399  & 671976 & nn   &   & P & 1   \\
4/7/2012  & Chiese & Emissario Lago Roncone & Nessuno & 804 & 5093637  & 629754 & nn   &   & S & 1   \\
\rowcolor[HTML]{EFEFEF} 24/09/2012 & Chiese & Fiume Chiese & Nessuno & 696 & 5089483  & 624578 & A. p.  &   & P & 2   \\
24/09/2012 & Chiese & Fiume Chiese & Nessuno & 464 & 5084877  & 624943 & nn   &   & P & 2   \\
\rowcolor[HTML]{EFEFEF} 24/09/2012 & Chiese & Fiume Chiese & Nessuno & 538 & 5088662  & 626521 & nn   &   & P & 2   \\
6/9/2012  & Chiese & Lago Ampola & \texttt{ZSC IT3120076}  & 750 & 5080979  & 628215 & A. p.  &   & P & 2   \\
\rowcolor[HTML]{EFEFEF} 4/7/2012  & Chiese & Lago Roncone   & Nessuno & 810 & 5093848  & 629920 & ex   & 1980  & P & 1   \\
1/8/2012  & Chiese & Palù di Boniprati   & \texttt{ZSC IT3120066}  & 1176  & 5087824  & 624202 & ex   & 2010  & P & 2   \\
\rowcolor[HTML]{EFEFEF} 1/8/2012  & Chiese & Rio Cimego & Nessuno & 1171  & 5087356  & 624053 & A. p.  &   & P & 2   \\
4/7/2012  & Chiese & Rio Roncone & Nessuno & 800 & 5094114  & 630172 & nn   &   & P & 1   \\
\rowcolor[HTML]{EFEFEF} 7/9/2012  & Chiese & Torrente Palvico & Nessuno & 428 & 5077724  & 623493 & nn   &   & P & 2   \\
2016  & Cismon & Lago Villa Welsperg   & Nessuno & 1047  & 5120091  & 721250 & A. p.  &   & P & 4   \\
\rowcolor[HTML]{EFEFEF} 15/10/2010 & Fersina & Canale Canzolino-Costa & Nessuno & 512 & 5105297  & 672757 & nn   &   & P & 1   \\
4/8/2011  & Fersina & Lago Canzolino & Nessuno & 537 & 5105580  & 672131 & O. l.  &   & P & 1   \\
\rowcolor[HTML]{EFEFEF} 28/06/2011 & Fersina & Lago Costa & \texttt{ZSC IT3120139}  & 500 & 5104893  & 673019 & ex   & 2010  & P & 1   \\
19/07/2012 & Fersina & Lago Lases & \texttt{ZSC IT3120049}  & 636 & 5112005  & 671517 & ex   & 2007  & P & 1   \\
\rowcolor[HTML]{EFEFEF} 4/8/2011  & Fersina & Lago Madrano   & Nessuno & 550 & 5105899  & 671578 & O. l.  &   & P & 1   \\
25/06/2010 & Fersina & Lago Piazze & Nessuno & 1021  & 5113907  & 676306 & ex   & 2004  & P & 1   \\
\rowcolor[HTML]{EFEFEF} 2012  & Fersina & Lago Pudro & Nessuno & 520 & 5105236  & 671752 & nn   &   & P & 2   \\
26/05/2011 & Fersina & Lago Restel & Nessuno & 878 & 5106356  & 673494 & A. p.  &   & P & 1   \\
\rowcolor[HTML]{EFEFEF} 26/06/2010 & Fersina & Lago Santa Colomba & \texttt{ZSC  IT3120102} & 924 & 5110089  & 668468 & ex   & 2007  & P & 1   \\
25/06/2010 & Fersina & Lago Serraia   & \texttt{ZSC IT3120034} (e aree non vincolate) & 970 & 5111744  & 674431 & ex   & 2004  & P & 1   \\
\rowcolor[HTML]{EFEFEF} 30/08/2012 & Fersina & Rio Campo  & Nessuno & 976 & 5112391  & 674802 & nn   &   & P & 2   \\
19/08/2011 & Fersina & Rio Farinella  & Nessuno & 552 & 5107039  & 667808 & A. p.  &   & P & 1   \\
\rowcolor[HTML]{EFEFEF} 21/07/2011 & Fersina & Rio Mala   & Nessuno & 856 & 5107088  & 676438 & nn   &   & P & 1   \\
6/7/2011  & Fersina & Rio Nero   & Nessuno & 532 & 5104720  & 673778 & A. p.  &   & P & 1   \\
\rowcolor[HTML]{EFEFEF} 19/08/2011 & Fersina & Rio S. Colomba & Nessuno & 651 & 5108299  & 668232 & A. p.  &   & P & 1   \\
1/8/2012  & Fersina & Rio Val Brutta & Nessuno & 665 & 5107725  & 671017 & nn   &   & P & 2   \\
\rowcolor[HTML]{EFEFEF} 2/8/2012  & Fersina & Rio Val Guarda & Nessuno & 499 & 5104632  & 672929 & ex   & 2011  & P & 2   \\
31/07/2012 & Fersina & Rio Valbrutta  & \texttt{ZPS IT3120139}  & 901 & 5109185  & 672081 & nn   &   & P & 2   \\
\rowcolor[HTML]{EFEFEF} 1/7/2010  & Fersina & Rio Valgrande  & Nessuno & 463 & 5104150  & 670591 & nn   &   & S & 1   \\
21/07/2011 & Fersina & Rio Vergini & Nessuno & 977 & 5108806  & 677475 & nn   &   & P & 1   \\
\rowcolor[HTML]{EFEFEF} 21/07/2011 & Fersina & Rio Viarago & Nessuno & 777 & 5105698  & 674841 & nn   &   & S & 1   \\
15/10/2010 & Fersina & Torrente Fersina - Ponte Alto  & Nessuno & 353 & 5104799  & 666581 & nn   &   & P & 1   \\
\rowcolor[HTML]{EFEFEF} 29/03/2013 & Noce & Laghetto pesca Mezzocorona & Nessuno & 251 & 5121506  & 661545 & A. p.  &   & P & 1   \\
5/9/2012  & Noce & Rio - Maso Milano   & Nessuno & 272 & 5122267  & 657747 & nn   &   & P & 2   \\
\rowcolor[HTML]{EFEFEF} 12/9/2012 & Noce & Rio Dercolo & Nessuno & 270 & 5123764  & 658777 & nn   &   & P & 2   \\
23/08/2012 & Noce & Rio di Fai & Nessuno & 212 & 5117400  & 661772 & nn   &   & P & 2   \\
\rowcolor[HTML]{EFEFEF} 5/9/2012  & Noce & Rio Tuazen & Nessuno & 305 & 5125384  & 658537 & A. p.  &   & P & 2   \\
4/9/2012  & Noce & Torrente Lovernatico  & Nessuno & 272 & 5123008  & 658760 & nn   &   & P & 2   \\
\rowcolor[HTML]{EFEFEF} 18/09/2012 & Noce & Torrente Noce  & Nessuno & 284 & 5125030  & 658710 & nn   &   & P & 2   \\
20/09/2012 & Noce & Torrente Noce - Forra santa Giustina & \texttt{ZSC IT3120060}  & 488 & 5134212  & 658462 & A. p.  &   & P & 2   \\
\rowcolor[HTML]{EFEFEF} 18/09/2012 & Noce & Torrente Noce - La Rocchetta   & \texttt{ZSC-ZPS IT3120061}   & 273 & 5121893  & 659195 & nn   &   & P & 2   \\
23/08/2012 & Noce & Torrente Noce - La Rupe   & \texttt{ZSC IT3120054}  & 207 & 5117003  & 661804 & ex   & 2006  & P & 2   \\
\rowcolor[HTML]{EFEFEF} 12/9/2012 & Noce & Torrente Pongaiola & Nessuno & 299 & 5126270  & 658904 & nn   &   & P & 2   \\
5/9/2012  & Noce & Torrente Rinassico & Nessuno & 275 & 5123630  & 658893 & nn   &   & P & 2   \\
\rowcolor[HTML]{EFEFEF} 18/09/2012 & Noce & Torrente Sporeggio & Nessuno & 267 & 5122514  & 658805 & nn   &   & P & 2   \\
17/09/2012 & Sarca  & Lago Toblino   & \texttt{ZSC IT3120055}  & 246 & 5101794  & 652092 & nn   &   & P & 2   \\
\rowcolor[HTML]{EFEFEF} 2014  & Sarca  & Lago Lagolo & Nessuno & 937 & 5100507  & 655243.73 & P.c. &   & P & 5   \\
13/09/2012 & Sarca  & Rio Andogno & Nessuno & 488 & 5102765  & 646757 & A. p.  &   & P & 1   \\
\rowcolor[HTML]{EFEFEF} 6/7/2012  & Sarca  & Rio Banale & Nessuno & 448 & 5100646  & 645590 & ex   & 1950  & P & 1   \\
4/7/2012  & Sarca  & Rio Bolbeno & Nessuno & 549 & 5099330  & 634438 & nn   &   & P & 1   \\
\rowcolor[HTML]{EFEFEF} 1/8/2012  & Sarca  & Rio Cavaione   & Nessuno & 779 & 5097977  & 640492 & nn   &   & P & 1   \\
12/8/2012 & Sarca  & Rio Cioc - Troticoltura   & Nessuno & 500 & 5100526  & 636824 & A. p.  &   & P & 2   \\
\rowcolor[HTML]{EFEFEF} 6/7/2012  & Sarca  & Rio Comano & Nessuno & 408 & 5100204  & 645057 & nn   &   & P & 1   \\
30/07/2012 & Sarca  & Rio Diga   & Nessuno & 485 & 5101498  & 639787 & nn   &   & S & 1   \\
\rowcolor[HTML]{EFEFEF} 8/10/2012 & Sarca  & Rio El Pison - troticoltura & Nessuno & 483 & 5101903 & 639765  & A. p.   & & P   & 2 & &   &  & \\
13/09/2012 & Sarca  & Rio Folon  & Nessuno & 550 & 5099671  & 635538 & A. p.  &   & P & 1   \\
\rowcolor[HTML]{EFEFEF} 1/8/2012  & Sarca  & Rio Madice & Nessuno & 690 & 5098487  & 641519 & nn   &   & P & 1   \\
11/7/2012 & Sarca  & Rio Pill   & Nessuno & 835 & 5098394  & 640382 & nn   &   & P & 1   \\
\rowcolor[HTML]{EFEFEF} 13/07/2012 & Sarca  & Rio Poia   & Nessuno & 552 & 5099361  & 645737 & ex   & 2008  & P & 1   \\
30/07/2012 & Sarca  & Rio Ponte Pià  & Nessuno & 483 & 5101499  & 640395 & nn   &   & P & 1   \\
\rowcolor[HTML]{EFEFEF} 9/7/2012  & Sarca  & Rio Pravert & Nessuno & 507 & 5101454  & 640358 & ex   & 1950  & P & 1   \\
6/7/2012  & Sarca  & Rio Premione 1 & Nessuno & 644 & 5101184  & 644966 & nn   &   & P & 1   \\
\rowcolor[HTML]{EFEFEF} 9/7/2012  & Sarca  & Rio Premione 2 & Nessuno & 475 & 5100631  & 645167 & nn   &   & P & 1   \\
13/09/2012 & Sarca  & Rio Rango  & Nessuno & 805 & 5098195  & 640347 & A. p.  &   & P & 1   \\
\rowcolor[HTML]{EFEFEF} 4/7/2012  & Sarca  & Rio Squero & Nessuno & 642 & 5099303  & 635981 & ex   & 1980  & P & 1   \\
25/07/2012 & Sarca  & Rio Tanfurin 2 & Nessuno & 411 & 5100190  & 643338 & nn   &   & P & 1   \\
\rowcolor[HTML]{EFEFEF} 20/07/2012 & Sarca  & Rio Tavodo & Nessuno & 494 & 5102635  & 646653 & nn   &   & P & 1   \\
25/07/2012 & Sarca  & Rio Tignerone  & Nessuno & 421 & 5100311  & 643220 & nn   &   & S & 1   \\
\rowcolor[HTML]{EFEFEF} 17/09/2012 & Sarca  & Rio Val del Bus & Nessuno & 247 & 5102574  & 653186 & nn   &   & P & 2   \\
6/9/2012  & Sarca  & Roggia Calavino & Nessuno & 293 & 5101939  & 653282 & ex   & 2010  & P & 2   \\
\rowcolor[HTML]{EFEFEF} 21/6/12   & Sarca  & Roggia Vezzano & Nessuno & 473 & 5104645  & 656173 & A. p.  &   & P & 1   \\
20/07/2012 & Sarca  & Torrente Ambiez & Nessuno & 614 & 5104338  & 646157 & nn   &   & P & 1   \\
\rowcolor[HTML]{EFEFEF} 23/07/2012 & Sarca  & Torrente Dal   & Nessuno & 476 & 5097118  & 644774 & nn   &   & P & 1   \\
12/8/2012 & Sarca  & Torrente Dal   & \texttt{ZSC IT3120070}  & 524 & 5094227  & 644314 & A. p.  &   & P & 2   \\
\rowcolor[HTML]{EFEFEF} 24/09/2012 & Sarca  & Torrente Ponale - troticoltura   & Nessuno & 397 & 5080559  & 640443 & A. p.  &   & P & 2   \\
4/7/2012  & Sarca  & Torrente Sarca & Nessuno & 525 & 5099645  & 634401 & nn   &   & P & 1   \\* \bottomrule
\caption{Elenco dei siti indagati con indicazione della data dell'ultimo campionamento/monitoraggio, il bacino idrografico, il corpo idrico e vincoli di tutela (per l'intero corpo idrico o solo una parte) , la quota, le coordinate geografiche (UTM), le specie rilevate (A.p. = \emph{A. pallipes}; O.l. = \emph{O. limosus}; P.c. = \emph{Procambarus clarkii}; nn. = nessuna specie rilevata; ex = estinzione di popolazioni di \emph{A. pallipes}, la data dell'ultima segnalazione di \emph{A. pallipes}, caratteristiche idrologiche del sito (P = permanente; S = stagionale; A = artificiale) e note, fonte del dato (1 = Endrizzi et al., 2013; 2= Ciutti et al., 2013; 3 = segnalazione guardie forestali; 4 = segnalazione Parco Paneveggio)}
\label{tab_4}\\
\end{longtable}
\normalsize

\begin{figure}
  \centering
  \begin{subfigure}[t]{.8\textwidth}
    \centering
    \includegraphics[width=1\columnwidth]{fig_10_a.jpg}
    \caption{A}
    \label{fig_10a}
  \end{subfigure}
  \begin{subfigure}[t]{.8\textwidth}
    \centering
    \includegraphics[width=1\columnwidth]{fig_10_b.jpg}
    \caption{B}
    \label{fig_10b}
  \end{subfigure}  
  \caption{A) stazioni indagate per la presenza di gamberi autoctoni e alloctoni (cfr tabella \ref{tab_4}); B) distribuzione storica e attuale delle popolazioni di gambero di specie autoctone e alloctone nelle aree indagate del Trentino (\cite{Paoli 2008} \cite{Ciutti 2013} \cite{Endrizzi 2013} e Formulari Natura 2000)}
  \label{fig_10}
\end{figure}

\section{Abbondanza e struttura delle popolazioni}
\ActivateBG
Le indagini condotte negli ultimi anni hanno permesso di ricavare dati utili a comprendere lo stato di conservazione di parte delle popolazioni rilevate in Trentino. I dati biometrici, la caratterizzazione per sesso, la stima di abbondanza, consentono infatti di ottenere informazioni sulla struttura e la dinamica delle singole popolazioni, essenziali nella definizione di piani di gestione e di conservazione \cite{Grandjean 1997} \cite{Scalici 2008}. L'attribuzione dell'età nei crostacei è molto complessa e richiede un'approfondita conoscenza delle singole popolazioni; per questo motivo l'analisi della struttura delle popolazioni è più spesso basata sulle classi di taglia. Questo tipo di analisi permette di indagare nel dettaglio lo status delle popolazioni osservate e di valutare potenziali rischi per la loro conservazione. La distribuzione degli individui di una popolazione in numerose classi di taglia che siano ben rappresentate da entrambi i sessi e in misura maggiore da femmine, rappresenta la situazione ottimale. Studi di laboratorio \cite{Carral 1994} \cite{Carral 2000} evidenziano infatti una maggiore efficienza riproduttiva con un rapporto sessi sbilanciato verso le femmine, data la capacità di ciascun maschio di accoppiarsi con almeno 8 femmine \cite{Reynolds 1992}.

Le popolazioni più abbondanti tra quelle studiate, sono quelle del Lago Restel, della Roggia di Terlago, del Rio Laguna, del Lago Santo di Cembra e del Lago Costa, per le quali sono stati rilevati, nel periodo estivo di massima attività dei gamberi, valori di CPUE (Catch Per Unit Effort: n gamberi catturati/n operatori/tempo impiegato per la cattura) compresi tra 0,88 e 0,30. Valori inferiori allo 0,05 sono invece stati osservati per le popolazioni del Rio Carpine e del Rio Nero. Per quanto riguarda il rapporto sessi, la situazione ottimale è stata rilevata nelle popolazioni di: Torrente Valsorda, Rio Farinella, Rio Santa Colomba, Rio Rango, Roggia di Gardolo, Lago Santo di Lamar, Lago Santo di Cembra e Roggia di Terlago (con M/F < 1) mentre le condizioni peggiori, con un numero di maschi pari a più del doppio rispetto a quello delle femmine, sono state riscontrate nel Rio Carpine e nel Lago Costa (figura \ref{fig_11}).


\begin{figure}
  \centering
  \includegraphics[width=.8\columnwidth]{fig_11.jpg}
  \caption{Abbondanza (CPUE) e rapporto sessi (M/F) rilevati nelle popolazioni di \emph{A. pallipes} studiate in Trentino tra il 2010 e il 2012}
  \label{fig_11}
\end{figure}

Il ciclo vitale dei gamberi è risultato condizionato dalle diverse temperature registrate nei vari siti di campionamento, con periodi di attività dei gamberi, di sviluppo delle uova e di muta tardivi alle quote maggiori rispetto al fondovalle. Un maggior tasso di crescita è stato osservato inoltre in popolazioni campionate nei corsi d'acqua rispetto a quelle di lago \cite{Endrizzi 2013}. Ne consegue quindi che individui appartenenti a popolazioni diverse sono caratterizzati da ritmi di accrescimento differenti. Classi di taglia differenziate per entrambi i sessi, e rapporto sessi in favore delle femmine sono stati rilevati a: Roggia di Terlago, Roggia di Gardolo, Rio Laguna, Lago Restel e Laghetto di Mezzocorona. 

\begin{figure}[!h]
  \centering
    \includegraphics[width=.8\columnwidth]{fig_12_a.png}
  \caption{Distribuzione delle classi di taglia nelle popolazioni di A. pallipes studiate in Trentino tra il 2010 e il 2012. Numero di individui (in ascissa) per classe di taglia da 5mm ciscuna (in ordinata) distinti per maschi (in blu) e femmine (in rosso) - Adige, Brenta e Noce}
  \label{fig_12_a}
\end{figure}

Situazioni allarmanti si sono invece riscontrate ad esempio nel lago Costa, nel Rio Nero, nel Rio Folon e nella Roggia di Vezzano che presentano una distribuzione non omogenea dei due sessi nelle diverse classi di taglia (figura \ref{fig_12}). Popolazioni caratterizzate da una struttura poco equilibrata avranno minori capacità di resilienza nei confronti di eventi di disturbo come la variazione di parametri ambientali o l'infestazione da patogeni. Così, ad esempio, la popolazione del Lago Costa campionata nel 2010, pur presentando densità piuttosto elevata era caratterizzata da una struttura del tutto squilibrata in termini di rapporto sessi e distribuzione delle classi di taglia ed è risultata estinta l'anno successivo. La necessità di raccogliere periodicamente questo tipo di informazioni risulta quindi evidente per l'identificazione delle popolazioni minacciate da sottoporre a tempestivi interventi per la loro conservazione. 
Eventi estremi possono comunque causare la riduzione o l'estinzione improvvisa anche in popolazioni ben strutturate come accaduto per la Roggia di Terlago e il laghetto di Mezzocorona rispettivamente per la cattiva gestione del corpo idrico e per la diffusione della peste del gambero. 

Il monitoraggio svolto nelle aree protette della Rete Natura 2000 della Provincia di Trento da Ciutti e Cappelletti \cite{Ciutti 2013} riporta dati di classi di taglia e di rapporto sessi su ulteriori popolazioni rinvenute. Tuttavia, le classi di taglia non sono direttamente confrontabili con i dati qui esposti poiché calcolate sulla misura del carapace totale anziché del cefalotorace. Le catture effettuate con le nasse possono inoltre aver interferito sull'analisi delle classi di taglia e del rapporto sessi in quanto escludono la possibilità di catturare individui di piccola taglia permettendo invece di catturare prevalentemente maschi di grandi dimensioni, maggiormente mobili e aggressivi. Tali metodi sono quindi principalmente utili al rilievo di presenza/assenza di specie, che costituiva il focus dei monitoraggi effettuati, ma non permettono un'analisi realistica dello stato delle popolazioni. In ogni caso, tali dati possono essere considerati per avere una prima idea sulle possibili condizioni delle popolazioni osservate. Nel dettaglio, valori di rapporti sessi sbilanciati verso le femmine sono stati osservati per le popolazioni del lago di Cei (0,3), del Torrente Arione (0,7), e della Forra di Santa Giustina (0,3), mentre il rapporto sessi è risultato a favore dei maschi al lago d'Ampola (1,5).

\begin{figure}[!h]
  \centering
    \includegraphics[width=.8\columnwidth]{fig_12_b.png}
  \caption{Distribuzione delle classi di taglia nelle popolazioni di A. pallipes studiate in Trentino tra il 2010 e il 2012. Numero di individui (in ascissa) per classe di taglia da 5mm ciscuna (in ordinata) distinti per maschi (in blu) e femmine (in rosso) - AVisio e Fersina}
  \label{fig_12_b}
\end{figure}
\begin{figure}
  \centering
    \includegraphics[width=.8\columnwidth]{fig_12_c.png}
  \caption{Distribuzione delle classi di taglia nelle popolazioni di A. pallipes studiate in Trentino tra il 2010 e il 2012. Numero di individui (in ascissa) per classe di taglia da 5mm ciscuna (in ordinata) distinti per maschi (in blu) e femmine (in rosso) - Sarca}
  \label{fig_12_c}
\end{figure}

\section{Caratterizzazione genetica}
\label{sec_gene}
Numerose popolazioni di gambero del Trentino sono state analizzate da un punto di vista genetico nel corso del 2015 \cite{Gandolfi 2015}. La caratterizzazione genetica dei campioni è stata effettuata seguendo un doppio approccio, basato sull'utilizzo di due differenti tipologie di marcatori molecolari. L'analisi di sequenza di un marcatore del DNA mitocondriale (mtDNA) ha consentito di inserire gli individui e le popolazioni oggetto di studio in un contesto tassonomico e filogeografico, per confronto con i dati più rilevanti disponibili in letteratura scientifica e relativi a numerose popolazioni italiane di \emph{Austropotamobius pallipes}. Come marcatore del mtDNA è stato analizzato il locus della citocromo ossidasi I (COI), comunemente utilizzato in tassonomia molecolare animale per il DNA barcoding degli individui e per il quale numerosi dati sono disponibili in letteratura. In particolare, i dati raccolti nell'ambito di recenti progetti svolti in Italia nell'ambito di progetti LIFE finalizzati alla conservazione di \emph{Austropotamobius} (Life CRAINat \cite{CRAINAT} e Life RARITY \cite{RARITY} ) hanno permesso un confronto diretto dei dati e un'interpretazione della diversità osservata in Trentino.

Il secondo approccio adottato è costituito dall'analisi, a livello individuale, di un appropriato numero di marcatori microsatellite (simple sequence repeats, SSR) del DNA nucleare (nDNA). Tali marcatori offrono la possibilità di investigare la struttura genetica di popolazione e di stimare eventuali tracce di flusso genico tra le popolazioni, interpretabili in relazione alle capacità di dispersione della specie e alla connettività naturale tra gli ambienti o, diversamente, in relazione alle pratiche gestionali precedentemente adottate e ai pregressi eventi di traslocazione di individui. La quantità di informazione offerta dai marcatori SSR è tanto maggiore quanto più differenziati sono individui e popolazioni e quanto più alto è il numero di marcatori analizzati. Per queste ragioni, ai fini della caratterizzazione genetica delle popolazioni trentine, si è prestata particolare attenzione a utilizzare un adeguato numero di marcatori SSR, riuscendo a caratterizzarne ben otto, contro i tre/sei comunemente utilizzati in tutti i lavori precedenti aventi come oggetto di studio il genere \emph{Austropotamobius} \cite{Baric 2005} \cite{Bertocchi 2008} \cite{Matallanas 2013}. Poiché i dati SSR non sono direttamente confrontabili con dati analoghi ottenuti in differenti studi (Life RARITY) - o non sono addirittura disponibili per altre indagini (Life CRAINat) - l'interpretazione dei risultati è limitata alle popolazioni del Trentino e dell'Alto Adige incluse nell'analisi (tabella \ref{tab_5}).

Le informazioni congiunte derivate dall'analisi dei due tipi di marcatori genetici sono in grado di suggerire l'identificazione di differenti unità gestionali, definite sulla base delle differenziazioni genetiche osservate tra i siti di campionamento e finalizzate al mantenimento della biodiversità naturale rilevata entro e tra i siti stessi. 

\subsection{Popolazioni trentine analizzate geneticamente}
Complessivamente sono stati analizzati in Trentino 223 individui da 20 siti di campionamento per il marcatore mitocondriale e 171 individui da 14 siti di campionamento per il marcatore nucleare (Figure 13 e 14). Nell'ambito dello stesso studio, sono inclusi per confronto numerosi individui (105 per il mtDNA e 108 per il nDNA da sei siti di campionamento) dell'Alto Adige e un singolo individuo del Friuli Venezia Giulia (tabella \ref{tab_5}).

\subsection{Analisi del mtDNA}
Una sequenza di buona qualità della citocromo ossidasi I (COI) del mtDNA, per una lunghezza complessiva di 773 pb (paia basi), è stata ottenuta per 329 individui analizzati (tabella \ref{tab_5}).

Le differenze, in termini di mutazioni, tra gli aplotipi osservati sono state rappresentate in forma di network (\emph{Median Joining Network}), includendo nell'analisi differenti dataset di riferimento. Poiché le sequenze di riferimento disponibili hanno lunghezze (numero di paia-basi, pb) differenti, tre successive analisi sono state effettuate: i) includendo le sole sequenze degli individui delle Province di Bolzano e Trento (773 pb); ii) aggiungendo alle precedenti le sequenze disponibili per campioni dell'area alpina occidentale e dell'Appennino (695 pb; \cite{Bernini 2016}; progetto Life CRAINat); iii) aggiungendo alle precedenti le sequenze disponibili per campioni del Friuli Venezia Giulia (310 pb; Bertucci Maresca 2015; RARITY).
Le sequenze ottenute per i campioni della Provincia di Trento sono state analizzate in primo luogo unitamente alle sequenze, di pari lunghezza (773 pb), disponibili per i campioni della Provincia di Bolzano e per un campione del Friuli Venezia Giulia (figura \ref{fig_13}). Complessivamente sono stati individuati 16 aplotipi, di cui 14 presenti in Trentino. 

Un primo aplotipo (hAp01) è presente con una frequenza fortemente dominante (79,3\% complessivamente e 70,4\% in Provincia di Trento) e risulta fissato in 11 siti di campionamento del Trentino e in 5 dell'Alto Adige (tabella \ref{tab_5}). Il secondo aplotipo più frequente (hAp07) è osservato nel 4,6\% degli individui totali, e nel 6,7\% degli individui della Provincia di Trento. I rimanenti aplotipi non raggiungono mai frequenze superiori al 5\%. I 16 aplotipi sono raggruppati in tre aplogruppi, tra loro differenziati da un elevato numero di mutazioni nucleotidiche e entro i quali gli aplotipi sono differenziati da un numero minimo di mutazioni. Un aplogruppo (I) include campioni trentini e altoatesini, uno (II) include solamente campioni trentini e il terzo (III) è costituito dal solo aplotipo unicamente rinvenuto nel singolo campione del Friuli Venezia Giulia (figura \ref{fig_13}). La distribuzione di frequenza dei 14 differenti aplotipi e dei due aplogruppi entro i siti di campionamento del Trentino è mostrata in figura \ref{fig_14}.

\begin{figure}[!h]
  \centering
  \includegraphics[width=.95\columnwidth]{fig_13.png}
  \caption{\emph{Median Joining Network} dei 16 aplotipi mitocondriali individuati su 329 individui (Trentino = 223, Alto Adige = 105, Friuli 0 1) analizzati, su una lunghezza complessiva di 773 pb del marcatore COI del mtDNA. Ogni differente aplotipo è rappresentato da un cerchio, la cui area è proporzionale alla frequenza dell'aplotipo stesso; i colori sono riferiti ai diversi siti di campionamento. Cerchi neri (senza alcuna denominazione dell'aplotipo) = aplotipi non rinvenuti, ogni segmento perpendicolare alla linea che unisce due cerchi indica una singola mutazione nucleotidica}
  \label{fig_13}
\end{figure}
\newpage

\begin{figure}[!h]
  \centering
  \includegraphics[width=.95\columnwidth]{fig_14.png}
  \caption{Distribuzione di frequenza dei 14 diversi aplotipi (a) e dei due differenti aplogruppi (b) mitocondriali rinvenuti al locus COI nei siti di campionamento trentini. L'area dei singoli grafici a torta è proporzionale al numero di individui analizzati entro sito. I colori sono riferiti ai diversi aplotipi (a) o aplogruppi (b)}
  \label{fig_14}
\end{figure}

Il confronto delle sequenze del presente studio mediante l'allineamento di 663 pb di lunghezza con i dati pubblicati in Bernini \cite{Bernini 2016}, relativi a numerose popolazioni dell'Area Alpina Centro Occidentale e dell'Appennino, ha permesso di collocare gli aplogruppi in cui ricadono gli individui trentini in un contesto tassonomico: gli aplotipi dell'aplogruppo I (figura \ref{fig_13}), seppur differenti rispetto a quelli descritti da Bernini \cite{Bernini 2016}, sono inclusi in un clade riferibile alla sottospecie \emph{Austropotamobius italicus carinthiacus} (figura \ref{fig_15}). Gli aplotipi dell'aplogruppo \cellcolor{LimeGreen}II (figura \ref{fig_13}) risultano invece inclusi in un cluster riferibile alla sottospecie \emph{Austropotamobius italicus carsicus} (figura \ref{fig_15}). In particolare, mentre Bernini \cite{Bernini 2016} individuavano due cladi (\emph{Eastern} e \emph{Western}) entro la sottospecie \emph{Austropotamobius italicus carsicus}, i campioni trentini costituiscono un terzo clade entro la sottospecie, significativamente differenziato dagli altri due (Trentino clade in figura \ref{fig_15}).
\newpage

\begin{figure}[H]
  \centering
  \includegraphics[width=.95\columnwidth]{fig_15.png}
  \caption{Median Joining Network degli aplotipi mitocondriali individuati in Trentino Alto Adige per il marcatore COI del mtDNA, confrontati con le sequenze disponibili in Bernini \protect\cite{Bernini 2016}. Il network è costruito su un allineamento ridotto a 695 pb rappresentanti la regione omologa sequenziata in entrambi gli studi. Ogni differente aplotipo è rappresentato da un cerchio, la cui area è proporzionale alla frequenza dell'aplotipo stesso (solo per i campioni del Trentino Alto Adige); i differenti colori indicano le diverse provenienze per gli individui analizzati nel presente studio, o la classificazione in unità tassonomiche (\emph{A. i. carinthiacus}, \emph{A. i. carsicus Western clade} e \emph{A. i. carsicus Eastern clade}) proposta per i diversi campioni analizzati in Bernini \protect\cite{Bernini 2016}. Cerchi neri (senza alcuna denominazione dell'aplotipo) = aplotipi non rinvenuti; ogni segmento perpendicolare alla linea che unisce due cerchi indica una singola mutazione nucleotidica}
  \label{fig_15}
\end{figure}

Il successivo confronto con le sequenze pubblicate in Bertucci Maresca \cite{Bertucci Maresca 2015}, relative a numerose popolazioni del Friuli Venezia Giulia e ad alcuni individui Trentini e Lombardi, seppur basato sull'allineamento di sole 310 pb, ha consentito di rifinire ulteriormente il quadro tassonomico e di definire la struttura geografica dei differenti aplogruppi riscontrati nel Italia Settentrionale (figura \ref{fig_16}). In particolare, rispetto al quadro precedentemente delineato, è possibile evidenziare come la maggior parte dei campioni provenienti dal Friuli Venezia Giulia siano attribuibili alla sottospecie \emph{Austropotamobius italicus meridionalis}, con l'eccezione di alcuni campioni del bacino del Rosandra, riferibili al clade Trentino di \emph{A. i. carsicus}, seppure con aplotipi diversi da quelli trentini. 

Il quadro d'insieme così ottenuto, riassunto in figura \ref{fig_17}, evidenzia l'esistenza di almeno tre cladi principali, riferibili alle sottospecie \emph{A. i. carinthiacus}, \emph{A. i. meridionalis} e \emph{A. i. carsicus}, l'ultimo dei quali risulta suddiviso in quattro aplogruppi secondari. Da un punto di vista geografico, i cladi e i sotto-cladi mostrano una distribuzione sostanzialmente definita, con l'eccezione di alcune zone di contatto o distribuzione simpatrica o parapatrica.

Il Trentino è caratterizzato dalla presenza di due linee mitocondriali riferibili a \emph{A. i. carinthiacus} e a \emph{A. i. carsicus}. Come evidenziato in figura \ref{fig_14}, la distribuzione geografica dei due aplogruppi è abbastanza definita, con un aplogruppo prevalente nel macro-bacino dell'Adige e un secondo aplogruppo quasi esclusivo del Bacino del Brenta. 

\newpage

\begin{figure}[H]
  \centering
  \includegraphics[width=.95\columnwidth]{fig_16.png}
  \caption{Median Joining Network degli aplotipi mitocondriali individuati in Trentino Alto Adige per il marcatore COI del mtDNA, confrontati con le sequenze disponibili in Bernini \protect\cite{Bernini 2016}  e dal progetto LIFE RARITY \protect\cite{Bertucci Maresca 2015}. Il network è costruito su un allineamento ridotto a 310 pb rappresentanti la regione omologa sequenziata nei tre differenti studi. Ogni differente aplotipo è rappresentato da un cerchio, la cui area è proporzionale alla frequenza dell'aplotipo stesso; i differenti colori indicano le diverse provenienze per gli individui analizzati In Trentino Alto Adige, o al bacino idrografico di provenienza degli individui analizzati in Bernini \protect\cite{Bernini 2016} e in \protect\cite{Bertucci Maresca 2015}. Cerchi neri (senza alcuna denominazione dell'aplotipo) = aplotipi non rinvenuti; ogni segmento perpendicolare alla linea che unisce due cerchi indica una singola mutazione nucleotidica}
  \label{fig_16}
\end{figure}

\begin{figure}[H]
  \centering
  \includegraphics[width=.95\columnwidth]{fig_17.png}
  \caption{Schema di sintesi delle relazioni genetiche tra le linee evolutive mitocondriali, e le relative distribuzioni geografiche, individuate In Trentino Alto Adige e per confronto con i dati pubblicati \protect\cite{Bertucci Maresca 2015} \protect\cite{Bernini 2016}}
  \label{fig_17}
\end{figure}

\subsection{Analisi del nDNA}
Nell'ambito dell'azione C10 del Progetto Life+TEN, 171 individui della Provincia di Trento, insieme a 108 campioni di riferimento della Provincia di Bolzano e un individuo del Friuli Venezia Giulia, sono stati caratterizzati a otto marcatori SSR (tabella \ref{tab_5}).

In generale, il livello di polimorfismo osservato entro le popolazioni trentine è limitato: da un minimo di 4 a un massimo di 23 alleli per marcatore analizzato, e da un numero medio di alleli minimo di 1.29, al Lago Santo di Lamar, a un massimo di 5.25, a Rio Valsorda (tabella \ref{tab_5}). La variabilità osservata entro sito di campionamento è risultata generalmente maggiore in Trentino rispetto a quella osservata in Alto Adige. Nonostante il polimorfismo dei marcatori sia risultato moderato, ben 24 alleli (su un totale di 80) sono risultati privati di una singola popolazione, evidenziando quindi una certa differenziazione tra i siti.

Attraverso un'analisi di clustering bayesiano, implementata nel software Structure \cite{Pritchard 2001}, l'informazione genetica fornita dai marcatori SSR è stata utilizzata - senza ulteriori informazioni a priori, quali ad esempio la provenienza geografica di ciascun individuo - per definire i) il numero più probabile di gruppi di individui geneticamente distinti (K, clusters), ii) i rapporti di distanza genetica relativa tra i gruppi inferiti, e iii) la proporzione di appartenenza (q, ancestry) di ciascun individuo a ciascuno dei gruppi individuati. Nell'impossibilità di includere dati da altri progetti, l'analisi è stata condotta su tutto il dataset disponibile, rappresentativo di diverse popolazioni delle Provincie di Trento e Bolzano, oltre a un individuo del Friuli.

L'analisi di clustering bayesiano ha indicato K = 4 come il numero più probabile di gruppi di individui geneticamente distinti (clusters) entro il dataset microsatellite complessivo (figura \ref{fig_18}a). In figura \ref{fig_18}c sono mostrate le proporzioni di appartenenza (q value o ‘ancestry') di ciascun individuo ai quattro cluster genetici individuati. 

Un primo cluster include le popolazioni della sinistra orografica dell'Adige e dei suoi affluenti; un secondo cluster include le popolazioni della destra dell'Adige e le popolazioni altoatesine; un terzo cluster include le sole popolazioni del torrente Valsorda, un sito meridionale della destra orografica dell'Adige, e di Roggia Grigno, nel Bacino del Brenta; l'ultimo cluster è rappresentato dalla sola popolazione di Villa Welsperg, nel Bacino del Cismon. Alcune popolazioni, in particolare Noce e Krebus, mostrano un livello significativo di admixture (flusso genico), evidente in quasi tutti gli individui, tra i due cluster caratteristici delle popolazioni di sinistra e destra orografica dell'Adige. Mentre la similarità tra i due siti, Noce e Krebus, è spiegabile in termini di contiguità geografica, non è possibile determinare se l'origine della loro composizione genetica, determinata dalle mescolanza tra due differenti cluster genetici, sia riconducibile a un flusso genico naturale oppure mediato dall'uomo, attraverso la traslocazione di individui dalla sinistra orografica dell'Adige. Il terzo cluster genetico individuato, caratteristico del Brenta, e il quarto cluster, rappresentato dalla popolazione del Cismon, non sono presenti in modo significativo nelle popolazioni dell'Alto Adige.
\newpage
\begin{figure}[H]
  \centering
  \includegraphics[width=.95\columnwidth]{fig_18.png}
  \caption{Analisi di clustering bayesiano sui dati microsatellite. a) Selezione del numero più probabile di cluster geneticamente distinti (K) rappresentati nel dataset: per K = 4 la probabilità a posteriori del modello è massima (Pritchard e Wen, 2004); b) Albero di distanza genetica relativa tra i quattro cluster individuati dall'analisi; c) Risultati dell'analisi di clustering bayesiano individuale: per ciascun individuo, in ogni sito analizzato, è presentata la proporzione di appartenenza (q, ‘ancestry') ai quattro cluster genetici individuati. Abbreviazioni utilizzate: SA1 = Sinistra Adige 1, SA2 = Sinistra Adige 2, SA3 = Sinistra Adige 3, DA1 = Destra Adige 1, DA2 = Destra Adige 2, DA3 = Destra Adige 3, NO = Noce, AV1 = Avisio 1, AV2 = Avisio 2, FE1 = Fersina 1, FE2 = Fersina 2, FE3 = Fersina 3, FE4 = Fersina 4, BR = Brenta, CI = Cismon, FR = Friuli}
  \label{fig_18}
\end{figure}


La figura \ref{fig_19} mostra la diversità genetica osservata ai marcatori microsatellite in relazione alla distribuzione geografica dei siti analizzati. Appare evidente che, nonostante la difficoltà di interpretazione dei dati a causa della mancanza di un più ampio dataset rappresentativo di un più vasto areale geografico, la differenziazione osservata tra i siti ricalchi in qualche misura la separazione idrografica tra bacini differenti.

\begin{figure}[H]
  \centering
  \includegraphics[width=.95\columnwidth]{fig_19.png}
  \caption{Risultati dell'analisi di clustering bayesiano sui dati microsatellite visualizzati sulla mappa dei siti di campionamento. Per ciascun sito analizzato è presentata la proporzione di appartenenza (‘ancestry') di ciascun individuo ai quattro cluster genetici individuati}
  \label{fig_19}
\end{figure}


\subsection{Confronto e integrazione dei risultati dell'analisi del mtDNA e del nDNA}
L'analisi del mtDNA evidenzia come tutti gli aplotipi rinvenuti in Trentino, riferibili a due differenti linee mitocondriali, siano privati di questa regione, ovvero non siano mai stati identificati in altre zone dell'areale naturale di distribuzione della specie. Questo dato è sufficiente a suggerire una gestione delle popolazioni trentine che non preveda in alcun modo attività di ripopolamento o introduzione con materiale proveniente da differenti zone dell'areale. Oltre a ciò, i risultati del nDNA suggeriscono una ulteriore differenziazione, a scala più fine, delle popolazioni trentine. Anche tale differenziazione deve essere tenuta in considerazione e preservata in una corretta pianificazione delle attività di conservazione delle popolazioni. 
Ai fini di una corretta gestione delle popolazioni, i flussi genici stimati sulla base dei dati genetici dovrebbero essere innanzitutto interpretati come effetto di migrazione naturale. Tuttavia è doveroso precisare come non si possa a priori escludere l'effetto di un intervento mediato dall'uomo: in altri termini alcuni individui di gambero, pur in assenza di notizie e/o atti ufficiali, potrebbero essere stati oggetto di traslocazione da un sito ad un altro. Un contatto secondario tra popolazioni ha infatti, in ogni caso, come effetto immediato un aumento della diversità genetica entro la popolazione ricevente il flusso genico e una riduzione della differenziazione tra le popolazioni sorgente e ricevente. L'aumento di diversità entro la popolazione ricevente deve tuttavia essere valutato positivamente, e conseguentemente tutelato, solo nel caso in cui sia riconducibile ad un flusso genico naturale, mediato quindi da migrazione attiva e naturale degli individui. Viceversa, un aumento di diversità dovuto a flusso genico di origine antropica, mediato dalla traslocazione di individui, deve essere valutato negativamente e conseguentemente contrastato per impedire il fenomeno dell'introgressione genetica. Nell'impossibilità di escludere un impatto antropico, tutti gli individui - o in estrema ratio tutte le popolazioni - che portino tracce di introgressione dovrebbero essere esclusi dalle pratiche gestionali (riproduzione ex situ, traslocazione).

Sulla base delle differenziazioni genetiche osservate tra i siti di campionamento, tenendo conto delle informazioni ottenute sia a livello mitocondriale che nucleare, sono state individuate sette unità gestionali (Management Units, MU) distinte (A-G. Vedi figura \ref{fig_20}), che presentano una concordanza, seppure non perfetta, con l'idrogeografia della regione di studio. Di queste, due unità sono rappresentate, rispettivamente da tre (A in figura \ref{fig_20}) e sei popolazioni (B in figura \ref{fig_20}) mentre cinque sono rappresentate da un'unica popolazione (C-G in \ref{fig_20}).
Le popolazioni 4, 6 e 15, unitamente alle popolazioni altoatesine, risultano simili da un punto di vista mitocondriale e nucleare e vengono quindi individuate come un'unica MU (A in figura \ref{fig_20}), approssimativamente ascrivibile alle destra orografica del macrobacino dell'Adige. E' opportuno evidenziare come una traccia di introgressione nucleare, riconducibile a un flusso genico da una popolazione della sinistra orografica del macrobacino dell'Adige, sia osservata nella popolazione 15 (e nella popolazione altoatesina di Krebus). Nell'impossibilità di escludere un'origine antropica di tali eventi di introgressione, le popolazioni 4 e 6 dovrebbero essere favorite come possibili popolazione sorgente per attività di ripopolamento all'interno della destra orografica dell'Adige. In particolare, la popolazione 6 presenta sia una diversità genetica sia un'abbondanza stimata leggermente superiori a quelle della popolazione 4 e si configurerebbe quindi come migliore popolazione sorgente. Per il ripristino di popolazioni attualmente estinte, la scelta della popolazione sorgente potrebbe alternativamente basarsi su un criterio di prossimità geografica con le popolazioni disponibili, e in ogni caso il mescolamento di diverse popolazioni sorgente dovrebbe prudenzialmente essere escluso.

Le popolazioni 9, 2, 12, 11, 1 e 8 risultano omogenee da un punto di vista mitocondriale e nucleare e vengono quindi inserite in un'unica MU (B in figura \ref{fig_20}) ascrivibile, approssimativamente, alla sinistra orografica del macrobacino dell'Adige. Le popolazioni 11 e 1 presentano tuttavia una traccia di introgressione a livello nucleare, riconducibile a un flusso genico dalla popolazione 3 (o dalla popolazione 16), mentre la popolazione 8 presenta una traccia di introgressione a livello mitocondriale, condividendo un aplotipo - seppure con frequenza limitata - con la popolazione 16. Le migliori candidate come popolazione sorgente nell'ambito della MU B (sinistra orografica dell'Adige) sarebbero quindi le popolazioni 9, 2 e 12. Tra queste, tutte fissate per un unico aplotipo mitocondriale, la popolazione 12 presenta una ricchezza allelica ai marcatori microsatellite leggermente superiore, mentre la popolazione 9 presenta una abbondanza stimata superiore alle altre.

La popolazione 10, Lago Restel, è caratterizzata dalla massima diversità aplotipica osservata, determinata dalla presenza di sei aplotipi mitocondriali riferibili ai due aplogruppi presenti sul territorio trentino. Due aplotipi dell'aplogruppo I e uno dell'aplogruppo \cellcolor{LimeGreen}II sono privati di questa popolazione. Diversamente, da un punto di vista nucleare, la popolazione 10 risulta omogenea con numerose popolazioni della sinistra orografica del macrobacino dell'Adige. La compresenza nel sito di campionamento dei due aplogruppi, la presenza di aplotipi privati, l'uniformità genetica a livello nucleare e la posizione geografica del sito sembrerebbero suggerire questo stesso come possibile zona di contatto secondario, in tempi remoti, tra linee evolutive distinte (aplogruppi I e II); In tempi più recenti, un flusso genico con le popolazioni più prossime dell'Adige, avrebbe cancellato a livello nucleare - ma non mitocondriale - le tracce di questo evento. L'interpretazione alternativa della diversità osservata entro questa popolazione come effetto di traslocazione non è supportata - seppure in modo non conclusivo - dalla osservazione dei tre aplotipi qui presenti con frequenze elevate e invece assenti in tutti gli altri siti studiati così come nella letteratura scientifica disponibile. Conseguentemente, la popolazione 10 viene qui considerata come MU a sé stante (C in figura \ref{fig_20}).

La popolazione 28, rappresentata da un unico individuo di origine Friulana è caratterizzata da un aplotipo mai osservato in Trentino. L'esiguità del dato ottenuto su un singolo individuo è d'altronde fortemente supportata dal confronto effettuato, a livello mitocondriale tra il presente dataset e i risultati del progetto Life RARITY. E' quindi assolutamente plausibile che le popolazioni del Friuli debbano costituire una o più MU, in ogni caso ben distinte da quelle individuate sul territorio trentino. 

Le popolazioni 17, 16 e 3 risultano completamente differenziate tra loro sulla base degli aplotipi mitocondriali, seppur riferibili allo stesso aplogruppo, e parzialmente differenziate tra loro sulla base del dato microsatellite e sono conseguentemente state identificate con MU distinte (D, E e F, rispettivamente). In assenza di dati nucleari relativi ad un più ampio areale, queste tre MUs possono essere riferite sensu lato al macrobacino del fiume Brenta.

Le indagini genetiche sulle popolazioni trentine di gambero di fiume permettono di delineare uno scenario relativamente chiaro circa le popolazioni sorgente da cui poter eventualmente prelevare individui per operazioni di ripopolamento. Attenendosi ad un criterio di massima cautela, si può concludere che:
\begin{itemize}\itemsep0pt
  \item nel macrobacino dell'Adige si evidenzia una chiara distinzione genetica fra la destra e la sinistra orografica. Qualsiasi intervento dovrebbe, conseguentemente, basarsi su questo risultato. E' bene ricordare come questo tipo di differenziamento si sia già evidenziato in organismi diversi, dalle piante annuali, passando agli anfibi, rettili ed uccelli per arrivare fino ai mammiferi \cite{Schonswetter 2004} \cite{Pecchioli 2006} \cite{Vernesi 2016}. Si tratterebbe di un pattern biogeografico molto chiaro dalle importanti implicazioni in campo conservazionistico e gestionale.
  \item Il macrobacino del Brenta si pone come geneticamente ben distinto da quello dell'Adige. Al suo interno è molto evidente la diversificazione genetica fra i differenti siti.
  \item I dati genetici escludono molto chiaramente la possibilità di usare altre popolazioni dell'arco alpino o appenniniche come sorgente per attività di ripopolamento in Trentino.
\end{itemize}

\begin{figure}
  \centering
  \includegraphics[width=.8\columnwidth]{fig_20.png}
  \caption{Sintesi interpretativa dei risultati: per ogni sito di campionamento i cui campioni sono stati analizzati sia al marcatore mitocondriale sia ai microsatelliti nucleari sono riportate le frequenze aplotipiche (e di aplogruppo) mitocondriali e le proporzioni di appartenenza (‘ancestry') individuali ai quattro cluster genetici stimati con i marcatori nucleari. Le frecce rosa e azzurre indicano probabili tracce di flusso genico (naturale o antropico) mediato da individui di sesso femminile o maschile, rispettivamente. I due cerchi esterni colorati riassumono la differenziazione (colori diversi) tra siti di campionamento sulla base dell'informazione mitocondriale (cerchio più esterno) e nucleare (cerchio più interno); i gradienti tra colori, entro sito, indicano l'effetto del flusso genico. Le informazioni congiunte dei due tipi di marcatori genetici suggeriscono l'identificazione di differenti unità gestionali, codificate ciascuna con una differente lettera (A - G)}
  \label{fig_20}
\end{figure}

\footnotesize
\begin{landscape}
\begin{longtable}[c]{@{}p{.08\columnwidth}p{.0083\columnwidth}p{.0083\columnwidth}p{.0083\columnwidth}p{.0083\columnwidth}p{.0083\columnwidth}p{.0083\columnwidth}p{.0083\columnwidth}p{.0083\columnwidth}|p{.0083\columnwidth}p{.0083\columnwidth}p{.0083\columnwidth}p{.0083\columnwidth}p{.0083\columnwidth}p{.0083\columnwidth}p{.0083\columnwidth}p{.0083\columnwidth}|p{.0083\columnwidth}p{.0083\columnwidth}|p{.0083\columnwidth}p{.0083\columnwidth}p{.0083\columnwidth}p{.0083\columnwidth}p{.0083\columnwidth}p{.0083\columnwidth}p{.0083\columnwidth}p{.0083\columnwidth}p{.0083\columnwidth}p{.0083\columnwidth}@{}}
\toprule
\multicolumn{19}{l}{\textbf{mtDNA COI}}  & \multicolumn{10}{l}{\textbf{nDNA SSR}}  \\
\midrule
{\cellcolor{white}}  &{\cellcolor{white}}  & \multicolumn{7}{l}{\textbf{Aplogruppo I}}   & \multicolumn{8}{l}{\textbf{Aplogruppo II}}  & \multicolumn{2}{l}{\textbf{A. III}} & \textbf{}  & \multicolumn{9}{l}{\textbf{Na}}\\
\midrule
\textbf{Sito} & \textbf{N} & \rotatebox[origin=c]{45}{\textbf{01}} & \rotatebox[origin=c]{45}{\textbf{02}} & \rotatebox[origin=c]{45}{\textbf{03}} & \rotatebox[origin=c]{45}{\textbf{04}} & \rotatebox[origin=c]{45}{\textbf{05}} & \rotatebox[origin=c]{45}{\textbf{06}} & \rotatebox[origin=c]{45}{\textbf{16}} & \rotatebox[origin=c]{45}{\textbf{07}} & \rotatebox[origin=c]{45}{\textbf{08}} & \rotatebox[origin=c]{45}{\textbf{09}} & \rotatebox[origin=c]{45}{\textbf{10}} & \rotatebox[origin=c]{45}{\textbf{11}} & \rotatebox[origin=c]{45}{\textbf{12}} & \rotatebox[origin=c]{45}{\textbf{13}} & \rotatebox[origin=c]{45}{\textbf{14}} & \rotatebox[origin=c]{45}{\textbf{15}} & \rotatebox[origin=c]{90}{\textbf{N privati}} & \rotatebox[origin=c]{45}{\textbf{N}} & \rotatebox[origin=c]{45}{\textbf{Aitali11}} & \rotatebox[origin=c]{45}{\textbf{Aas3040}} & \rotatebox[origin=c]{45}{\textbf{Aitali9}} & \rotatebox[origin=c]{45}{\textbf{AT37}} & \rotatebox[origin=c]{45}{\textbf{Ap3}} & \rotatebox[origin=c]{45}{\textbf{Aitali1}} & \rotatebox[origin=c]{45}{\textbf{Aitali5}} & \rotatebox[origin=c]{45}{\textbf{Aas6}} & \rotatebox[origin=c]{45}{\textbf{Na medio}} \\*
\midrule
\endfirsthead
\multicolumn{29}{l}{\footnotesize\textbf{Continua dalla pagina precedente}} \\
\toprule
& \multicolumn{18}{l}{\textbf{mtDNA COI}}  & \multicolumn{10}{l}{\textbf{nDNA SSR}}  \\ \midrule
& \textbf{}  & \multicolumn{7}{l}{\textbf{Aplogruppo I}}   & \multicolumn{8}{l}{\textbf{Aplogruppo II}}  & \multicolumn{2}{l}{\textbf{A. III}} & \textbf{}  & \multicolumn{9}{l}{\textbf{Na}}\\
\midrule
\textbf{Sito di campionamento} & \textbf{N} & \rotatebox[origin=c]{45}{\textbf{01}} & \rotatebox[origin=c]{45}{\textbf{02}} & \rotatebox[origin=c]{45}{\textbf{03}} & \rotatebox[origin=c]{45}{\textbf{04}} & \rotatebox[origin=c]{45}{\textbf{05}} & \rotatebox[origin=c]{45}{\textbf{06}} & \rotatebox[origin=c]{45}{\textbf{16}} & \rotatebox[origin=c]{45}{\textbf{07}} & \rotatebox[origin=c]{45}{\textbf{08}} & \rotatebox[origin=c]{45}{\textbf{09}} & \rotatebox[origin=c]{45}{\textbf{10}} & \rotatebox[origin=c]{45}{\textbf{11}} & \rotatebox[origin=c]{45}{\textbf{12}} & \rotatebox[origin=c]{45}{\textbf{13}} & \rotatebox[origin=c]{45}{\textbf{14}} & \rotatebox[origin=c]{45}{\textbf{15}} & \rotatebox[origin=c]{90}{\textbf{N privati}} & \rotatebox[origin=c]{45}{\textbf{N}} & \rotatebox[origin=c]{45}{\textbf{Aitali11}} & \rotatebox[origin=c]{45}{\textbf{Aas3040}} & \rotatebox[origin=c]{45}{\textbf{Aitali9}} & \rotatebox[origin=c]{45}{\textbf{AT37}} & \rotatebox[origin=c]{45}{\textbf{Ap3}} & \rotatebox[origin=c]{45}{\textbf{Aitali1}} & \rotatebox[origin=c]{45}{\textbf{Aitali5}} & \rotatebox[origin=c]{45}{\textbf{Aas6}} & \rotatebox[origin=c]{45}{\textbf{Na medio}} \\*
\midrule
\endhead
%
Roggia Gardolo& 15  & 15  & 0 & 0 & 0 & 0 & 0 & 0 & 0 & 0 & 0 & 0 & 0 & 0 & 0 & 0 & 0 &   & 15  & 3  & 4 & 3 & 3& 6   & 2 & 2 & 4& 3  \\
Rio Carpine & 15  & 15  & 0 & 0 & 0 & 0 & 0 & 0 & 0 & 0 & 0 & 0 & 0 & 0 & 0 & 0 & 0 &   & 14  & 3  & 3 & 2 & 2 (1)  & 6   & 2 & 2 & 3& 2.43 \\
Rio Valsorda  & 15  & 0 & 0 & 0 & 0 & 0 & 0 & 0 & 0 & 7 & 1 & 7 & 0 & 0 & 0 & 0 & 0 & 1   & 14  & 3  & 6 (1)   & 4 & 4& 7 (1) & 3 (1)   & 6 & 8 (3)  & 4.25 \\
Lago Santo Lamar  & 14  & 13  & 0 & 0 & 1 & 0 & 0 & 0 & 0 & 0 & 0 & 0 & 0 & 0 & 0 & 0 & 0 &   & 14  & 1  & 1 & 1 & 1& 3   & 1 & 1 & 5 (2)  & 1.29 \\
Roggia Terlago& 10  & 10  & 0 & 0 & 0 & 0 & 0 & 0 & 0 & 0 & 0 & 0 & 0 & 0 & 0 & 0 & 0 &   &   &  &   &   &  & &   &   &  &  \\
Lago di Cei & 7 & 4 & 0 & 0 & 0 & 3 & 0 & 0 & 0 & 0 & 0 & 0 & 0 & 0 & 0 & 0 & 0 & 1   & 7 & 1  & 1 & 1 & 1& 2   & 1 & 1 & 3& 1.38 \\
Torrente Arione   & 8 & 6 & 0 & 0 & 2 & 0 & 0 & 0 & 0 & 0 & 0 & 0 & 0 & 0 & 0 & 0 & 0 &   & 8 & 1  & 1 & 1 & 1& 2   & 1 & 1 & 3& 1.38 \\
Rio Fornei  & 15  & 14  & 0 & 0 & 0 & 0 & 0 & 0 & 0 & 0 & 0 & 0 & 0 & 0 & 1 & 0 & 0 &   & 11  & 3  & 3 & 3 & 3& 3   & 2 & 3 & 5 (1)  & 2.86 \\
Lago Santo Cembra & 15  & 15  & 0 & 0 & 0 & 0 & 0 & 0 & 0 & 0 & 0 & 0 & 0 & 0 & 0 & 0 & 0 &   & 14  & 3  & 3 & 2 & 3& 6   & 2 & 2 & 4& 3.13 \\
Lago Restel & 19  & 7 & 5 & 1 & 0 & 0 & 0 & 0 & 0 & 3 & 2 & 0 & 1 & 0 & 0 & 0 & 0 & 3   & 13  & 3  & 4 (1)   & 2 & 3& 8   & 2 & 2 & 7 (2)  & 3.33 \\
Rio Nero  & 13  & 13  & 0 & 0 & 0 & 0 & 0 & 0 & 0 & 0 & 0 & 0 & 0 & 0 & 0 & 0 & 0 &   & 12  & 3  & 4 & 2 & 3& 5   & 2 & 2 & 4& 3.13 \\
Rio Farinella & 14  & 14  & 0 & 0 & 0 & 0 & 0 & 0 & 0 & 0 & 0 & 0 & 0 & 0 & 0 & 0 & 0 &   & 14  & 3  & 4 & 3 & 3& 5   & 2 & 2 & 5& 3.43 \\
Rio S. Colomba& 1 & 1 & 0 & 0 & 0 & 0 & 0 & 0 & 0 & 0 & 0 & 0 & 0 & 0 & 0 & 0 & 0 &   &   &  &   &   &  & &   &   &  &  \\
Torrente Noce - Forra di Santa Giustina & 8 & 8 & 0 & 0 & 0 & 0 & 0 & 0 & 0 & 0 & 0 & 0 & 0 & 0 & 0 & 0 & 0 &   & 7 & 1  & 3 & 3 (1)   & 2& 4   & 2 & 3 & 1& 2.29 \\
Roggia Grigno (+Rio Laguna)& 15  & 0 & 0 & 0 & 0 & 0 & 0 & 0 & 0 & 0 & 0 & 0 & 0 & 4 & 4 & 7 & 0 & 2   & 14  & 3 (1)& 3 & 4 & 3& 4   & 1 & 5 (1)   & 4& 3.17 \\
Lago Welsperg & 15  & 0 & 0 & 0 & 0 & 0 & 0 & 0 & 15  & 0 & 0 & 0 & 0 & 0 & 0 & 0 & 0 &   & 14  & 2  & 3 & 2 (1)   & 3& 4 (4) & 2 (1)   & 3 & 4 (2)  & 2.5  \\
Roggia di Vezzano-Mulino & 3 & 3 & 0 & 0 & 0 & 0 & 0 & 0 & 0 & 0 & 0 & 0 & 0 & 0 & 0 & 0 & 0 &   &   &  &   &   &  & &   &   &  &  \\
Rio Andogno & 10  & 10  & 0 & 0 & 0 & 0 & 0 & 0 & 0 & 0 & 0 & 0 & 0 & 0 & 0 & 0 & 0 &   &   &  &   &   &  & &   &   &  &  \\
Rio Rango   & 9 & 9 & 0 & 0 & 0 & 0 & 0 & 0 & 0 & 0 & 0 & 0 & 0 & 0 & 0 & 0 & 0 &   &   &  &   &   &  & &   &   &  &  \\
Rio Folon   & 2 & 0 & 0 & 0 & 0 & 0 & 2 & 0 & 0 & 0 & 0 & 0 & 0 & 0 & 0 & 0 & 0 & 1   &   &  &   &   &  & &   &   &  &  \\
Lago di Cornino   & 1 & 0 & 0 & 0 & 0 & 0 & 0 & 0 & 0 & 0 & 0 & 0 & 0 & 0 & 0 & 0 & 1 & 1   &   & na & 1 & 2 & 2& na  & 1 (1)   & 1 & 2 (2)  & 1.5  \\
Bolzano   & 22  & 22  & 0 & 0 & 0 & 0 & 0 & 0 & 0 & 0 & 0 & 0 & 0 & 0 & 0 & 0 & 0 &   & 24  & 1  & 1 & 1 & 1& 2   & 1 & 1 & 1& 1.13 \\
Hyppolith   & 20  & 20  & 0 & 0 & 0 & 0 & 0 & 0 & 0 & 0 & 0 & 0 & 0 & 0 & 0 & 0 & 0 &   & 20  & 1  & 1 & 1 & 1& 3 (1) & 1 & 1 & 3& 1.5  \\
Bressanone  & 10  & 9 & 0 & 0 & 0 & 0 & 0 & 1 & 0 & 0 & 0 & 0 & 0 & 0 & 0 & 0 & 0 & 1   & 10  & 2 (1)& 4 (1)   & 4 (1)   & 1& 6 (2) & 2 (1)   & 1 & 6 (1)  & 3.25 \\
Teis & 10  & 10  & 0 & 0 & 0 & 0 & 0 & 0 & 0 & 0 & 0 & 0 & 0 & 0 & 0 & 0 & 0 &   & 10  & 1  & 3 & 1 & 1& 3   & 1 & 1 & 9 (2)  & 2.5  \\
Angelbach   & 24  & 24  & 0 & 0 & 0 & 0 & 0 & 0 & 0 & 0 & 0 & 0 & 0 & 0 & 0 & 0 & 0 &   & 24  & 1  & 2 & 1 & 1& 2   & 1 & 1 & 4 (1)  & 1.63 \\
Krebus & 19  & 19  & 0 & 0 & 0 & 0 & 0 & 0 & 0 & 0 & 0 & 0 & 0 & 0 & 0 & 0 & 0 &   & 20  & 1  & 4 & 2 & 2& 6   & 1 & 2 & 5& 2.88 \\
Totale & 329 & 261 & 5 & 1 & 3 & 3 & 2 & 1 & 15  & 10  & 3 & 7 & 1 & 4 & 5 & 7 & 1 & 10  & 264 & 5  & 11  & 8 & 6& 22  & 6 & 7 & 30   & 2.15 \\* \bottomrule
\caption{Per ogni sito di campionamento sono riportati, per l'analisi del DNA mitocondriale (mtDNA COI): il numero di individui caratterizzati (N), il numero di differenti aplotipi mitocondriali (hAp\_01 - hAp16) e il numero di aplotipi privati (N privati, ovvero presenti solo in una determinata popolazione); per l'analisi del DNA nucleare (nDNA SSR): il numero di individui caratterizzati (N), il numero di alleli (Na) e il numero di alleli privati (in parentesi) per ciascun marcatore microsatellite analizzato, oltre al numero medio di alleli (Na medio) su tutti i loci nucleari}
\label{tab_5}\\
\end{longtable}
\end{landscape}
\normalsize

\section{Caratterizzazione degli habitat}
Le popolazioni residue di gambero di fiume presenti in Trentino risultano attualmente isolate in fiumi, torrenti, ruscelli, laghi, stagni, che ancora conservano un'elevata naturalità, come attestato anche dalla presenza di A. pallipes in 6 Zone Speciali di Conservazione (ZSP, previsti dalla Direttiva "Habitat") \cite{Ciutti 2013} \cite{Endrizzi 2013}. Risulta infatti importante la presenza di ombreggiamento e disponibilità di rifugi, forniti dalla presenza di un substrato naturale, caratterizzato da massi e ciottoli, di vegetazione acquatica e riparia, di materiale legnoso e radici sommerse e di detrito vegetale che assicura anche l'elevata disponibilità di nutrienti (tabella \ref{tab_6}). Queste condizioni si riscontrano oggi soprattutto nelle aree collinari e montane che, per l'impervietà del territorio sono state risparmiate dallo sfruttamento antropico (figura \ref{fig_21}). 

\begin{figure}[!h]
  \centering
  \includegraphics[width=.8\columnwidth]{fig_21.jpg}
  \caption{Esempi di habitat naturali di \emph{A. pallipes} (vedi tabella \ref{tab_4}). a) Rio Fornei (bacino Avisio); b) Rio Santa Colomba; c) Rio Carpine; d) Rio Laguna (sorgente); e) Lago Santo di Lamar; f) Torrente Valsorda, foto S. Endrizzi)}
  \label{fig_21}
\end{figure}

L'importanza di questi elementi è testimoniata anche dalla presenza di popolazioni relitte in limitati tratti di corsi d'acqua in ambiente urbano e agricolo che conservano un sufficiente grado di naturalità come osservato nelle rogge di Gardolo (nel tratto urbano), Terlago (bacino dell'Adige), Vezzano e nei rii Rango, Folon e Andogno (bacino dal Sarca) (figura \ref{fig_22}). Nella roggia di Terlago la popolazione di gamberi, oggi estinta, era concentrata nell'unico tratto, a monte delle aree agricole, che ancora conserva un substrato naturale e vegetazione riparia di tipo boschivo almeno su una sponda (figura \ref{fig_22}d). Importante risulta anche il ruolo svolto dai bacini artificiali realizzati nelle aree pedemontane quali habitat rifugio per le popolazioni native, come osservato negli stagni e nei laghetti artificiali di Grigno (bacino del Brenta), Mezzocorona (bacino del Noce), Terlago e Vezzano (bacino dell'Adige) (figura \ref{fig_23}).

I valori estremi rilevati per ciascun parametro delle caratteristiche chimico-fisiche delle acque nei siti del Trentino studiati e caratterizzati dalla presenza di popolazioni di gambero di fiume sono i seguenti: temperatura 2,70 - 22 $^{\circ}$C, conducibilità 118,50 - 616 $\mu S$ cm-1, pH 7,20 - 8,70, ossigeno disciolto 3,51 - 12,70 mg L-1, torbidità 0,16 - 11,50 NTU. Il dettaglio dei valori rilevati in ciascun sito sono riportati in tabella \ref{tab_6}. La variabilità delle caratteristiche chimico-fisiche delle acque registrate nei siti del Trentino indagati non evidenziano influenze significative sulla presenza e l'abbondanza delle popolazioni, così come riscontrato anche in altre regioni \cite{Nardi 2004} \cite{Favaro 2010}. 

\begin{figure}
  \centering
  \includegraphics[width=.8\columnwidth]{fig_22.jpg}
  \caption{Esempi di habitat di \emph{A. pallipes} in ambienti antropizzati (vedi tabella \ref{tab_4}). a) Roggia di Vezzano; b) Rio Andogno; c) Rio Rango; d) Fosso nel Comune di Cavalese; e) Roggia di Terlago; (foto S. Endrizzi)}
  \label{fig_22}
\end{figure}

\begin{figure}
  \centering
  \includegraphics[width=.8\columnwidth]{fig_23.jpg}
  \caption{Esempi di ambienti artificiali caratterizzati dalla presenza di \emph{A. pallipes} (vedi tabella \ref{tab_4}). a) Stagno di Grigno; b) Laghetto da pesca di Terlago (foto S. Endrizzi)}
  \label{fig_23}
\end{figure}


\begin{landscape}
\begin{table}[]
\centering
\resizebox{1\columnwidth}{!}{
\begin{tabular}{@{}lllllllllllllllllll@{}}
\toprule
\rotatebox[origin=c]{90}{\textbf{Sito}} & \rotatebox[origin=c]{90}{\textbf{Data}} & \rotatebox[origin=c]{90}{\textbf{T ($^\circ$C)}} & \rotatebox[origin=c]{90}{\textbf{pH}} & \rotatebox[origin=c]{90}{\textbf{O \%}} & \rotatebox[origin=c]{90}{\textbf{O (mg/l)}} & \rotatebox[origin=c]{90}{\textbf{Cond. (microS/cm)}} & \rotatebox[origin=c]{90}{\textbf{Torbidità (NTU)}} & \rotatebox[origin=c]{90}{\textbf{Prof. (m)}} & \rotatebox[origin=c]{90}{\textbf{Portata}} & \rotatebox[origin=c]{90}{\textbf{Mesohabitat}} & \rotatebox[origin=c]{90}{\textbf{Massi}} & \rotatebox[origin=c]{90}{\textbf{Ciottoli}} & \rotatebox[origin=c]{90}{\textbf{Ghiaia}} & \rotatebox[origin=c]{90}{\textbf{Detrito legnoso}} & \rotatebox[origin=c]{90}{\textbf{Radici}} & \rotatebox[origin=c]{90}{\textbf{Veg. Sommersa}} & \rotatebox[origin=c]{90}{\textbf{Veg. emersa}} & \rotatebox[origin=c]{90}{\textbf{Ombregg.}} \\* \midrule
Laghetto Mezzocorona & 4/27/2013 & 4.4 & 8.7 & 100.4 & 12.65 & 367 & 1.67 & & & & & & & & & & & \\
Lago Costa & 6/28/2011 & 22 & 8.22 & 76 & 6.42 & 349 & 3.02 & 1 & & & & & & \textbullet\ & \textbullet\ & \textbullet\ & \textbullet\ & \textbullet\ \\
Lago Restel & 6/23/2011 & 16.1 & 8.48 & 66.3 & 5.8 & 246 & & & & & & & & & & & & \\
Lago santo di Lamar & 8/30/2011 & 21.9 & 8.38 & 104 & 8.22 & 277 & 11.5 & 0.5 & & & \textbullet\ & \textbullet\ & & \textbullet\ & & \textbullet\ & \textbullet\ & \\
Rio Andogno & 9/13/2012 & 11.8 & 7.94 & 94.3 & 9.44 & 514 & 7.71 & & elevata & Correntino, scivolo, raschio, pozza & \textbullet\ & \textbullet\ & & \textbullet\ & \textbullet\ & & \textbullet\ & \textbullet\ \\
Rio Carpine & 7/6/2011 & 18.2 & & 78.6 & 7.05 & 599 & 7.14 & 0.5 & bassa & Cascata, scivolo, scivolo, pozza & \textbullet\ & \textbullet\ & \textbullet\ & \textbullet\ & \textbullet\ & & \textbullet\ & \textbullet\ \\
Rio Farinella & 8/19/2011 & 16 & 8.37 & 84.3 & 7.99 & 470 & 1.42 & 0.5 & bassa & Cascata, scivolo, scivolo, pozza & \textbullet\ & \textbullet\ & & \textbullet\ & \textbullet\ & & \textbullet\ & \textbullet\ \\
Rio Folon & 7/4/2012 & 13.6 & 7.63 & 60.4 & 5.84 & 360 & 1.48 & & & & & & & & & & & \\
Rio Fornei & 10/4/2011 & 12.6 & 8.11 & 91.8 & 9.03 & 362 & 3.11 & 0.4 & bassa & Scivolo, scivolo, pozza & \textbullet\ & \textbullet\ & & \textbullet\ & \textbullet\ & & & \textbullet\ \\
Rio Laguna & 8/4/2015 & 12.3 & 7,786 & 82.7 & 8.56 & 324 & 3.93 & 0.6 & bassa & Scivolo, pozza & \textbullet\ & \textbullet\ & & \textbullet\ & & \textbullet\ & \textbullet\ & \\
Rio Nero & 7/6/2011 & 15.2 & & 80.7 & 7.53 & 303 & 2.03 & 0.3 & bassa & Cascata, scivolo, scivolo, pozza & \textbullet\ & \textbullet\ & \textbullet\ & \textbullet\ & \textbullet\ & & \textbullet\ & \textbullet\ \\
Rio Rango & 9/13/2012 & 13 & 8.38 & 71.5 & 7.81 & 478 & & & & & & & & & & & & \\
Rio S. Colomba & 8/19/2011 & 16.7 & 8.38 & 90.5 & 8.28 & 538 & 3.97 & 0.3 & bassa & Cascata, scivolo, scivolo, pozza & \textbullet\ & \textbullet\ & & \textbullet\ & \textbullet\ & & \textbullet\ & \textbullet\ \\
Torrente Valsorda & 9/28/2011 & 11.2 & 8.32 & 87.5 & 9.11 & 507 & 1.11 & 0.4 & bassa & Cascata, scivolo, scivolo, pozza & \textbullet\ & \textbullet\ & & \textbullet\ & \textbullet\ & \textbullet\ & \textbullet\ & \textbullet\ \\
Roggia di Gardolo & 7/12/2011 & 19.2 & & 65.2 & 5.79 & 616 & 7.37 & 0.4 & bassa & Cascata, scivolo, scivolo, pozza & \textbullet\ & \textbullet\ & \textbullet\ & \textbullet\ & \textbullet\ & & \textbullet\ & \textbullet\ \\
Roggia di Terlago & 9/27/2011 & 15.1 & 8.06 & 125.2 & 12.7 & 352 & 2.35 & 0.2 & molto bassa & Scivolo, pozza & \textbullet\ & \textbullet\ & \textbullet\ & \textbullet\ & \textbullet\ & & \textbullet\ & \textbullet\ \\
Roggia di Vezzano & 6/21/2012 & 14.1 & 7.92 & 117.9 & 11.36 & 470 & 6.06 & & & & & & & & & & & \\* \bottomrule
\end{tabular}
}
\caption{Valori dei parametri fisico-chimici e ambientali degli habitat delle popolazioni campionate da Endrizzi et al., (2013) e nel corso del progetto LIFE +TEN (da sinistra: temperatura, pH, Ossigeno: percentuale di saturazione e mg/l, conducibilità, torbidità, profondità, portata,.mesohabitat principali, presenza di: massi (25,6 - 40 cm), ciottoli (6,5 - 25,6 cm), ghiaia (6,4 - 0,4 cm), detrito legnoso depositato in alveo, radici sporgenti, vegetazione sommersa ed emersa, ombreggiamento}
\label{tab_6}
\end{table}
\end{landscape}


\section{Fattori di minaccia e di declino delle popolazioni naturali} 
Dalla seconda metà del XX secolo le popolazioni di \emph{A. pallipes}, in molti bacini italiani ed europei, si sono rarefatte o estinte a causa di numerosi fattori legati principalmente alla perdita di habitat e all'introduzione di specie alloctone che hanno causato pressioni dovute a competizione e alla diffusione di parassiti e patogeni \cite{Aquiloni 2010}. Le popolazioni residue di \emph{A. pallipes} si trovano così sempre più spesso confinate in zone a ridotto impatto antropico caratterizzate dall'assenza di gamberi alloctoni, con conseguente possibile isolamento spaziale e genetico delle popolazioni, ponendo ulteriori problemi per la gestione della specie. Le popolazioni di \emph{A. pallipes} in Trentino hanno seguito questo trend negativo; le cause del declino e/o estinzione in molti casi sono riconducibili a impatti documentati, e permettono quindi di fornire importanti indicazioni gestionali.

\subsection{La perdita e la frammentazione di habitat e il degrado ambientale}
Le alterazioni fisiche e idrologiche dei corpi idrici sono considerate tra le maggiori cause di estinzione delle popolazioni di gambero in Italia \cite{Fureder 2003b} \cite{Sint 2007} \cite{Brusconi 2008} \cite{Aquiloni 2010}. Le alterazioni chimiche hanno anch'esse un ruolo importante nei fenomeni di estinzione di popolazioni locali e, sebbene diversi studi abbiano dimostrato la capacità del gambero di fiume di sopravvivere in condizioni di eutrofizzazione e acidificazione delle acque \cite{Demers 2002}, esso non è in grado di sopportare gli effetti causati da sostanze chimiche \cite{Renai 2006}. L'inquinamento da metalli pesanti e pesticidi hanno effetti negativi sui processi riproduttivi e di muta dei gamberi causandone l'estinzione \cite{Nystrom 2002} e il bioaccumulo di inquinanti comporta poi il trasferimento delle sostanze nocive nella catena alimentare fino al consumatore primario. Effetti negativi si sono ad esempio osservati sulla riproduzione di uccelli che si nutrono di gamberi.

Oltre agli effetti più estremi ed immediati che causano l'estinzione locale di intere popolazioni, la riduzione e la frammentazione degli habitat hanno effetti diretti sulla numerosità delle popolazioni e sulla capacità di dispersione degli individui. Crolli demografici e riduzione della connettività e del flusso genico hanno a loro volta effetti sulla diversità genetica delle popolazioni stesse, riducendone conseguentemente il potenziale adattativo in risposta ai continui cambiamenti ambientali, incluso l'arrivo di nuovi parassiti e patogeni.
In Trentino, A. pallipes è quasi del tutto scomparsa dalle aree di fondovalle, dove la sua presenza è limitata a poche popolazioni relitte segregate in limitati tratti di ambienti acquatici che conservano ancora un sufficiente grado di naturalità. E' il caso della popolazione del Rio Laguna, presso Grigno in Bassa Valsugana, che sopravvive nell'area sorgentizia del corso d'acqua e nell'adiacente stagno artificiale realizzato all'interno di un giardinetto pubblico. La popolazione, probabilmente ampiamente diffusa in passato, è del tutto scomparsa nelle aree circostanti influenzate dalle attività agricole e dalle opere di canalizzazione del corso d'acqua (figura \ref{fig_24}) . La completa scomparsa del gambero di fiume dalle aree agricole intensive in Trentino è indice dell'elevata sensibilità della specie all'inquinamento e alle alterazioni dell'habitat connesse a questo tipo di attività. La scomparsa della popolazione di gambero di fiume del Torrente Valsorda, affluente di sinistra del Fiume Adige a Sud di Trento (figura \ref{fig_21}), è stata segnalata il 10 luglio 2016 in seguito all'immissione di sostanze chimiche da parte di ignoti che avrebbe causato anche un'estesa moria di pesci (Adige).

La capacità di resilienza in seguito a eventi estremi è limitata in popolazioni già indebolite a causa delle precarie condizioni ambientali. Nell'estate del 2010 il Bacino del torrente Fersina è stato interessato da un improvviso evento di piena che avrebbe causato, secondo quanto riportato dai locali, una moria di pesci e gamberi anche nel Rio Nero, presso Serso (Pergine Valsugana), determinandone il declino. Le possibilità di ripresa della popolazione in questo corso d'acqua sono scarse in considerazione dei frequenti eventi di secca cui è sottoposto nel periodo estivo, anche a causa della richiesta idrica per la produzione idroelettrica e le attività agricole. 

L'attuazione di interventi di gestione e sistemazione straordinaria di corsi d'acqua e bacini caratterizzati dalla presenza di popolazioni di \emph{A. pallipes} può rappresentare un ulteriore fattore di rischio; purtroppo in Trentino si riscontrano alcuni casi di estinzioni o estreme riduzioni di popolazioni dovute proprio ad interventi antropici di questo tipo: i) le modalità di gestione delle operazioni di pulizia dei laghetti da pesca privati che prevedono spesso il completo svuotamento del bacino (figura \ref{fig_24}), sarebbe stata la causa dell'estinzione della numerosa popolazione della Roggia di Terlago, secondo quanto riportato dai locali. Anche questa popolazione, come quella osservata nel Rio Laguna, aveva trovato rifugio nel laghetto da pesca (figura \ref{fig_23}) e nel primo tratto della roggia (figura \ref{fig_22}) a monte dei terreni agricoli e rappresentava molto probabilmente la parte residuale della popolazione ampiamente diffusa in passato nei laghi di Terlago. Anche nell' adiacente Fossa Maestra di Terlago, che conserva apparentemente un sufficiente grado di naturalità, sono infatti stati rilievati solo pochi segni di presenza del gambero quali un esuvia e un individuo morto; ii) gli interventi di sistemazione in alveo effettuati nella Roggia di Gardolo e nel Rio Carpine (figura \ref{fig_21}) nel corso del 2014, con conseguenti diffusi depositi di limo, potrebbero essere stati la causa della diminuzione di densità se non della scomparsa delle popolazioni osservate gli anni precedenti e non rinvenute nel 2014 \cite{Maiolini 2014}; iii) l'intervento di escavazione di un fosso agricolo privato in Val di Fiemme (figura \ref{fig_22}) nel periodo invernale 2016-2017, avrebbe comportato l'accidentale prelievo, con il sedimento, di numerosi gamberi che sarebbero stati, secondo testimonianze, reimmessi successivamente in acqua. La sopravvivenza di tali individui, tra i quali molte femmine con uova, nel periodo di inattività è certamente messa a rischio. Secondo quanto riportato dai forestali della zona si tratterebbe dell'ultima popolazione ancora presente in Valle di Fiemme. 

\begin{figure}
  \centering
  \includegraphics[width=.8\columnwidth]{fig_24.jpg}
  \caption{Esempi di ambienti artificializzati che impediscono la colonizzazione e la diffusione di \emph{A.pallipes} (a) Rio Laguna (a valle dello stagno); b) Roggia di Terlago (in prossimità del paese) e di gestione di bacini artificiali non sostenibile per la sopravvivenza delle popolazioni (d) Laghetto da pesca di Vezzano durante gli interventi di pulizia)}
  \label{fig_24}
\end{figure}

La presenza di piccole popolazioni in aree che conservano ancora una sufficiente naturalità, come nel caso del Rio Laguna in Valsugana o di alcuni tratti urbani della Roggia di Gardolo, a Nord di Trento, è la testimonianza della possibilità di una convivenza tra il gambero e le attività antropiche, realizzabile sulla base di una gestione sostenibile degli ecosistemi acquatici

\subsection{La diffusione di specie alloctone}
\label{subs_alloct}
In Trentino è attualmente segnalata la presenza di due specie aliene invasive, il gambero americano \emph{O. limosus} e il gambero della Louisiana \emph{P. clarkii} (cfr. \ref{fig_2}), diffuse nei bacini idrografici dei Fiumi Sarca, Brenta e del Torrente Fersina tra i 400 e i 1000 metri di quota. Per quanto riguarda il bacino del Brenta la specie alloctona \emph{O. limosus} è stata segnalata nel Lago di Levico a partire dal 2007 \cite{Paoli 2008}, dove avrebbe sostituito la specie autoctona formando una popolazione molto numerosa e colonizzando, nel corso degli ultimi anni, anche il tratto terminale del Rio Vignola, immissario del lago \cite{Ciutti 2013}. Una popolazione di \emph{O. limosus} è stata segnalata nel 2008 nel Rio Resenzuola in Bassa Valsugana (Maiolini pers comm.), anch'esso occupato in passato da una popolazione di \emph{A. pallipes}, che ora risulta estinta, probabilmente in conseguenza dello stabilirsi della popolazione della specie alloctona. Indagini svolte nell'ambito del progetto LIFE+TEN, nel 2013, hanno però riportato la scomparsa della specie alloctona da questo sito, per cause ignote. Le prime segnalazioni di \emph{O. limosus} nel lago di Caldonazzo risalgono invece al 2010 (Maiolini, pers. comm.). La presenza di piccole popolazioni in limitati tratti del lago fa presupporre una recente colonizzazione da parte della specie. Nel bacino del Torrente Fersina due popolazioni di \emph{O.limosus} a elevata densità sono state rilevate nei laghi di Madrano e Canzolino \cite{Endrizzi 2013}, fortemente eutrofici a causa della passata immissione di sostanze organiche di derivazione civile e agricola. Per quanto riguarda il Bacino del Fiume Sarca, è stata di recente segnalata la presenza della specie invasiva \emph{P. clarkii} nel lago di Lagolo sul Monte Bondone \cite{Cappelletti 2014}. Questa specie, altamente invasiva, è in grado di colonizzare rapidamente nuovi ambienti muovendosi anche al di fuori dell'acqua ed è responsabile di importanti danni agli ecosistemi e alle coltivazioni per via della sua intensa attività di foraggiamento.
Va considerato che dal 1 gennaio 2015 è in vigore nei paesi dell'Unione Europea il Regolamento (UE) n. 1143/2014, recante disposizioni volte a prevenire e gestire l'introduzione e la diffusione delle specie esotiche invasive. Tale regolamento prevede la redazione di un elenco di specie esotiche invasive di rilevanza comunitaria e conseguenti procedure volte a gestire l'invasione Una prima lista (si prevedono aggiornamenti progressivi) di 37 specie esotiche vegetali ed animali di rilevanza comunitaria è stata pubblicata sulla Gazzetta ufficiale dell'Unione europea il 14 luglio 2016; sia \emph{O. limosus} che \emph{P. clarkii} sono elencati tra gli invertebrati alloctoni presenti anche in Italia, e per i quali devono essere previste, nel limite del possibile, misure di ripristino che consistono in interventi fisici, chimici o biologici, letali o non letali, volti all'eradicazione, al controllo numerico o al contenimento delle popolazioni esotiche invasive in questione.

\subsection{La diffusione di patogeni}
\label{subsec_diffusionepat}
Parassiti e patogeni possono svolgere un ruolo importante per il mantenimento dell'equilibrio dell'ecosistema controllando la densità delle popolazioni \cite{Vogt 1999}. In condizioni di degrado degli habitat e in presenza di popolazioni indebolite tale equilibrio può però essere turbato portando anche al verificarsi di morie estese. La diffusione di nuovi patogeni inoltre, causata dall'introduzione di specie alloctone, è oggi considerata tra le maggiori cause di estinzione delle popolazioni native \cite{Pockl 2002} \cite{Fureder 2006} \cite{Brusconi 2008} \cite{Aquiloni 2010}.

L'oomicota Aphanomyces astaci è l'agente eziologico della peste del gambero che rappresenta la principale causa di estinzione delle popolazioni europee di gambero risultando essere estremamente sensibili e presentando in presenza della malattia anche una mortalità del 100\% \cite{Holdich 1999}. Al contrario le specie nord americane, quali \emph{P. clarkii} e \emph{O. limosus}, sono particolarmente resistenti all'infezione e fungono da vettori biologici, in quanto albergano il patogeno a livello di macchie melanizzate della cuticola e non manifestano nessun altro tipo di lesioni \cite{Minghetti 2012b}. In Trentino, il primo caso di moria di \emph{A. pallipes} dovuta ad afanomicosi è stato accertato dall'Istituto Zooprofilattico Sperimentale delle Venezie in una popolazione del Fiume Chiese, in località Ponte S. Lucia, nel 2011 \cite{Quaglio 2011}; successivamente la positività ad \emph{A. astaci} è stata segnalata per la popolazione nativa del laghetto di Mezzocorona \cite{Cappelletti 2016}. Un'indagine svolta nell'estate del 2012 su esemplari di \emph{O. limosus} provenienti dai Laghi di Caldonazzo, Levico e Canzolino ha portato all'isolamento, mediante esami bio-molecolari, di \emph{A. astaci} anche in queste popolazioni \cite{Minghetti 2012b}. La presenza di abbondanti popolazioni alloctone affette da afanomicosi, può essere stata la causa di estinzione di popolazioni native adiacenti. Ne è un esempio la popolazione di \emph{A. pallipes} del Lago Costa segnalata fino al 2010 \cite{Endrizzi 2013} e scomparsa l'anno successivo. La diffusione del fungo può essere avvenuta attraverso il canale intermittente che collega il lago Costa al Lago di Canzolino posto poco più a monte causando l'estinzione della popolazione. Similmente, l'ultima segnalazione di \emph{A. pallipes} nei due laghi maggiori di Caldonazzo e Levico risale al 2004. La sua scomparsa è probabilmente attribuibile all'introduzione di \emph{O. limosus} nel lago di Levico, segnalata per la prima volta nel 2007 \cite{Paoli 2008}. La diffusione della peste del gambero attraverso il movimento di materiali, attrezzature o animali ha probabilmente causato successivamente l'estinzione della popolazione nativa segnalata nel lago di Caldonazzo, dove è attualmente presente una popolazione di \emph{O. limosus} molto ridotta, circoscritta a un tratto molto limitato del lago (spiaggia di Calceranica), osservata dal 2010 (cfr. \ref{subs_alloct}) e quindi originatasi da una colonizzazione molto più recente. Analogamente, la diffusione accidentale del fungo attraverso il movimento di attrezzature da pesca e animali tra un bacino e l'altro può essere stata la causa della scomparsa di \emph{A. pallipes} avvenuta tra il 2004 e il 2007 dal Lago delle Piazze, di Santa Colomba e di Lases (bacino del Torrente Fersina), per i quali l'ultima segnalazione risale rispettivamente al 2004 (Piazze) e al 2007 (Santa Colomba e Lases) \cite{Endrizzi 2013}. Nel bacino del fiume Chiese una popolazione segnalata fino alla fine degli anni ‘80 nel lago di Roncone, poco lontano da quella affetta da \emph{A. astaci} rilevata nel 2011 \cite{Quaglio 2011}, è risultata estinta.

Per quel che riguarda le patologie batteriche potenzialmente trasmissibili a \emph{A. pallipes}, in tempi recenti è stata riportata in \emph{O. limosus} la presenza di \emph{Pseudomonas fluorescens} e \emph{Chryseobacterium indologenes} da esemplari del Lago di Canzolino e di \emph{Aeromonas caviae} da esemplari del Lago di Caldonazzo \cite{Minghetti 2012b}.

Episodi di estinzione possono verificarsi anche nel caso di gravi infestazioni da parassiti che possono causare un'eccessiva mortalità di individui in ambienti degradati \cite{Vogt 1999}. Il verificarsi di queste condizioni non è da escludere per alcune popolazioni di \emph{A. pallipes} in Trentino che dovrebbero essere oggetto di frequenti monitoraggi per la valutazione dello stato di conservazione. Di particolare rilievo è l'infestazione causata dal microsporidio \emph{Thelohania contejeani}, responsabile della malattia della porcellana. Si tratta di un parassita obbligato intracellulare, che si localizza a livello di fibre muscolari, dove si riproduce mediante sporogonia causando la distruzione e la perdita delle funzionalità muscolare, e le cui spore possono sopravvivere al di fuori del corpo del gambero per parecchi mesi \cite{Dieguez Uribeondo 2006}, aumentando così il rischio di trasmissione dell'infezione. In Trentino, le indagini svolte tra il 2010 e il 2012 \cite{Minghetti 2012b} \cite{Endrizzi 2013} hanno rilevato la presenza di soggetti infetti provenienti da diverse località: Roggia di Gardolo, Torrente Valsorda, Lago di Cei (Bacino dell'Adige), Rio S. Colomba (Bacino del Fersina), Fiume Chiese (Bacino del Chiese) con un tasso di infestazione compreso tra 0,74 e 11,1 \%. Valori simili sono stati registrati in popolazioni di \emph{A. pallipes} in Italia e Spagna nell'ambito di altri studi \cite{Dieguez Uribeondo 1997} \cite{Mori 2000} e non rappresentare al momento una minaccia per la conservazione delle popolazioni locali.

Le indagini istologiche eseguite su soggetti di \emph{A. pallipes} provenienti dalla Troticoltura Brenta-Saone da Minghetti \cite{Minghetti 2012b} hanno anche evidenziato lesioni micotiche probabilmente riferibili al genere Saprolegnia. La presenza di branchiobdellidi ectosimbionti sulla superficie del carapace è stata riportata da Minghetti \cite{Minghetti 2012b} in esemplari di \emph{A. pallipes} provenienti da siti monitorati nel bacino del fiume Adige (Prà dell'Albi -Cei), Sarca (Roggia di Vezzano, Troticoltura Brenta-Saone, torbiera Lomasona), Chiese (bacino Morandin, Val di Daone), Brenta (Rio Laguna Grigno). Inoltre, Minghetti \cite{Minghetti 2012b} ha isolato da popolazioni di \emph{A. pallipes} anche ceppi batterici: \emph{Aeromonas spp}., \emph{A. bestiarium}, \emph{A. caviae}, \emph{A. hydrophila}, \emph{A. sobria}, \emph{Enterobacter cloacae}, \emph{Pseudomonas spp}., \emph{Pseudomonas fluorescens}, \emph{Cryseobacterium indologenes}, \emph{Flavobacterium spp}. per le stesse popolazioni e anche nelle popolazioni nei bacini dell'Adige (Prà dell'Albi Cei), Avisio (Rio Masen) e Fersina (Rio Santa Colomba). In alcuni casi alla positività batteriologica hanno fatto riscontro lesioni istologiche a carico di diversi organi, quali branchie, epatopancreas e intestino.

\subsection{Il prelievo illegale}
Il gambero di fiume ha rappresentato in passato una importante fonte alimentare per l'uomo in molte realtà locali del Trentino come è emerso da interviste alla popolazione. Questa usanza è andata perdendosi nel tempo, sia per via del miglioramento delle condizioni di vita degli abitanti, sia per la disponibilità sempre minore di gamberi causata dal degrado degli ambienti acquatici. La scomparsa delle popolazioni del Rio Banale, Squero e Poia in Valle del Sarca può essere stata proprio determinata dall'alterazione degli habitat unita all'intensa attività di pesca attuata in passato in questa valle. Il gambero di fiume è oggi considerato specie minacciata e, come tale, il suo prelievo è regolamentato dalla normativa europea, recepita a livello locale da leggi statali e provinciali. Secondo alcune testimonianze però, la pesca illegale sembra essere ancora praticata e può essere la concausa di eventi estintivi come segnalato per la popolazione del Lago di Lamar in Valle dei Laghi presente fino al 2007 e oggi estinta, e la forte riduzione della popolazione del laghetto da pesca di Mezzocorona osservata tra il 2013 e il 2014. Analogamente il prelievo illegale è considerato una delle cause di riduzione delle popolazioni autoctone anche in altre regioni \cite{Aquiloni 2010}.

\section{Interventi di conservazione in Trentino}
I primi interventi volti alla conservazione del gambero di fiume in Trentino sono stati realizzati nell'ambito dell'Azione C10 "Azione dimostrativa di tutela di specie: salvaguardia delle popolazioni autoctone di gambero di fiume" del Progetto Europeo Life+TEN, che è stato condotto tra il 2012 e il 2016 dalla Provincia Autonoma di Trento e dal MUSE. L'azione ha avuto lo scopo di offrire degli esempi pratici di conservazione di A. pallipes sul territorio provinciale. In particolare, sono stati attuati interventi di riqualificazione di habitat acquatici, utili a ripristinare le condizioni ecologiche ottimali alla diffusione del gambero di fiume, oltre all'allevamento per la reintroduzione di popolazioni di gambero di fiume in aree storicamente caratterizzate dalla presenza della specie. L'azione ha interessato territori caratterizzati da una diversa destinazione d'uso e in particolare un'area agricola della Piana Rotaliana, in Valle dell'Adige, e un'area sottoposta a vincoli di conservazione, la \texttt{ZSC IT3120030} Fontanazzo in Bassa Valsugana, così da offrire esempi concreti d'intervento sia in ambienti naturali che antropizzati. Le attività di riqualificazione e di reintroduzione sono state precedute da una fase preliminare di studio, al fine di individuare gli ambienti più adatti alla vita e alla diffusione del gambero di fiume e le popolazioni di gambero più idonee a fornire individui da avviare all'allevamento e quindi alla reintroduzione. Sono inoltre stati condotti monitoraggi post-intervento utili a valutare successi, criticità ed elaborare eventuali strategie correttive.

\subsection{Monitoraggio e riqualificazione di habitat}
L'individuazione degli ambienti più idonei ad ospitare nuove popolazioni di gambero di fiume è avvenuta attraverso un'attenta analisi di materiale cartografico e bibliografico e di dati rilevati attraverso sopralluoghi e campionamenti appositamente effettuati nel corso del primo anno di svolgimento del progetto. Nell'area agricola della Piana Rotaliana, tra gli abitati di Lavis e Mezzolombardo, sono stati individuati alcuni fossi agricoli particolarmente favorevoli per le condizioni idrologiche, il tipo di substrato, la presenza di vegetazione acquatica e riparia, l'ombreggiamento, la disponibilità di rifugi per i gamberi, la distanza dalle colture e la disponibilità di aree utili alla creazione di fasce di vegetazione riparia. Tra maggio e ottobre 2013 sono quindi state effettuate tre campagne di campionamento per la raccolta di dati sulle caratteristiche fisico-chimiche delle acque e sulla comunità di macroinvertebrati in otto siti selezionati. (figura \ref{fig_25}). Sono state così indagate più nel dettaglio le condizioni ecologiche di questi ambienti attraverso il calcolo dell'Indice Biotico Esteso (IBE) \cite{Ghetti 1997}. 

\begin{figure}
  \centering
  \includegraphics[width=.8\columnwidth]{fig_25.jpg}
  \caption{Rete di fossi della Piana Rotaliana. Area oggetto dei sopralluoghi per il rilievo delle caratteristiche fisiche degli habitat. I numeri si riferiscono alle stazioni di prelievo di campioni di macroinvertebrati e di misurazione dei parametri fisico-chimici delle acque. In verde le aree in cui sono stati svolti gli interventi di scavo per il miglioramento dell'habitat di reintroduzione di \emph{A. pallipes}}
  \label{fig_25}
\end{figure}

I risultati di queste indagini hanno evidenziato un buono stato ecologico dei fossi a monte, con classi di qualità IBE pari a I-II (ambiente non alterato e ambiente con moderati sintomi di alterazione) stabili nel corso delle stagioni, con una tendenza però al peggioramento verso valle, con classi di qualità pari a \cellcolor{Goldenrod}III e \cellcolor{BurntOrange}IV(ambiente alterato e ambiente molto alterato) e in un caso anche a \cellcolor{Red}V(ambiente fortemente degradato) (tabella \ref{tab_7}). 
Il tratto a monte è stato quindi ritenuto idoneo alla reintroduzione del gambero di fiume mentre il miglioramento ecologico dei tratti a valle, importante per garantirne la diffusione, è stato ritenuto possibile attraverso l'attuazione di interventi di riqualificazione. 

In particolare sono stati attuati nel corso del 2014 i seguenti interventi:
\begin{enumerate}
  \item il rimodellamento della morfologia dei canali e la creazione di stagni laterali offrono una disponibilità di habitat diversificati necessari allo svolgimento delle diverse fasi del ciclo vitale di molti organismi e permettono quindi l'insediamento di una maggiore biodiversità. Riportare alla luce fossi interrati è inoltre importante per incrementare l'estensione degli ambienti acquatici disponibili e la connettività ecologica. Le opere di rimodellamento e scavo sono state effettuate nella fossa Noce presso il terreno del Maso delle Part della Fondazione Edmund Mach (Figure \ref{fig_25} e \ref{fig_26});
  \item la presenza e la corretta gestione della vegetazione acquatica e riparia favorisce la disponibilità di microhabitat diversificati e di materia organica oltre a svolgere l'importante funzione di ombreggiamento, necessaria al mantenimento di caratteristiche chimico-fisiche delle acque equilibrate. La vegetazione riparia svolge inoltre una funzione di barriera nei confronti dell'ambiente agricolo limitando l'immissione di inquinanti in alveo. Una gestione sostenibile della vegetazione acquatica e riparia con tagli della vegetazione alternati e limitati ad alcuni periodi dell'anno è stata concordata con gli operatori del Consorzio di Bonifica nel rispetto delle loro attività gestionali. Fasce di vegetazione riparia sono state piantumate in alcuni tratti sottoposti a rimodellamento della fossa Noce (figura \ref{fig_26});
  \item l'allacciamento degli scarichi civili alla rete fognaria elimina l'eccessivo carico di inquinanti organici immesso nei fossi eliminando fenomeni di eutrofia e favorendo l'insediamento di specie sensibili. L'allacciamento al collettore di tutti gli scarichi in Piana Rotaliana è stato completato nel 2014.
\end{enumerate}

\begin{landscape}
\begin{table}[]
\centering
\begin{tabular}{@{}llccccccc@{}}
\toprule
\textbf{}& \textbf{} & \textbf{Classe di qualità  IBE} & \textbf{}& \textbf{} & \textbf{}& \textbf{} & \textbf{}& \textbf{}\\ \midrule
\textbf{Sito} & \textbf{Fosso/Località}  & \textbf{5/27/2013} & \textbf{8/22/2013} & \textbf{10/28/2013} & \textbf{5/13/2015} & \textbf{9/1/2015} & \textbf{11/9/2015} & \textbf{8/25/2016} \\
\midrule
1& Fossa Noce, Maso delle Parti & \cellcolor{SeaGreen}I-II & \cellcolor{LimeGreen}II & \cellcolor{SeaGreen}I-II & \cellcolor{Cyan}I& \cellcolor{Cyan}I &\cellcolor{Cyan} I&\\
2& Fossa Noce, Baron Cles & \cellcolor{LimeGreen}II& \cellcolor{SeaGreen}I-II  & \cellcolor{BrickRed}IV-V & \cellcolor{GreenYellow}II-III& \cellcolor{LimeGreen}II & \cellcolor{LimeGreen}II &\\
3& Fossa Maestra, Quadrifoglio sud & \cellcolor{BurntOrange}IV& \cellcolor{LimeGreen}II & \cellcolor{SeaGreen}I-II & \cellcolor{Cyan}I& \cellcolor{LimeGreen}II & \cellcolor{LimeGreen}II &\\
4& Fossa Sassudelli, Quadrifoglio  & \cellcolor{BurntOrange}IV& \cellcolor{BurntOrange}IV& \cellcolor{Dandelion}III-IV & \cellcolor{LimeGreen}II & \cellcolor{BrickRed}IV-V & \cellcolor{Goldenrod}III&\\
5& Fossa Maestra, Quadrifoglio nord& \cellcolor{Goldenrod}III& \cellcolor{LimeGreen}II & & \cellcolor{LimeGreen}II & \cellcolor{LimeGreen}II & \cellcolor{Cyan}I&\\
6& Fossa Noce, Quadrifoglio & \cellcolor{LimeGreen}II& \cellcolor{Goldenrod}III& & \cellcolor{SeaGreen}I-II  & \cellcolor{LimeGreen}II & \cellcolor{Dandelion}III-IV&\\
7& Fossa Maestra, Maso Rosabel& \cellcolor{Goldenrod}III& \cellcolor{LimeGreen}II & \cellcolor{LimeGreen}II& \cellcolor{Cyan}I& \cellcolor{LimeGreen}II & \cellcolor{LimeGreen}II &\\
8& Fossa Maestra, chiusa & \cellcolor{BurntOrange}IV& \cellcolor{LimeGreen}II & \cellcolor{Red}V& \cellcolor{LimeGreen}II & \cellcolor{BurntOrange}IV& \cellcolor{LimeGreen}II &\\
 & Fosso disotterrato  &&& && && \cellcolor{LimeGreen}II \\
\bottomrule
\end{tabular}%
\caption{Classi di qualità ecologica IBE rilevate negli 8 siti di campionamento (vedi figura 25) prima (2013) e dopo (2015) gli interventi di riqualificazione e nel fosso dissotterrato nella primavera del 2016 e campionato nell'agosto 2016. I (\crule[Cyan]{.2cm}{.2cm}): Ambiente non alterato | I-II(\crule[SeaGreen]{.2cm}{.2cm}): Ambiente con deboli sintomi di alterazione | II(\crule[LimeGreen]{.2cm}{.2cm}):  Ambiente con moderati sintomi di alterazione | III(\crule[Goldenrod]{.2cm}{.2cm}): Ambiente alterato | IV(\crule[BurntOrange]{.2cm}{.2cm}): Ambiente molto alterato | V(\crule[Red]{.2cm}{.2cm}): Ambiente fortemente degradato}
\label{tab_7}
\end{table}
\end{landscape}

\begin{figure}[H]
  \centering
  \includegraphics[width=.8\columnwidth]{fig_26.jpg}
  \caption{Opere di riqualificazione in Piana Rotaliana: immagine della Fossa del Noce (loc. Maso delle Parti) prima (2013 a sinistra) e dopo (2015 a destra) gli interventi di riqualificazione}
  \label{fig_26}
\end{figure}

Per valutare i risultati ottenuti attraverso gli interventi, nel corso del 2015, sono stati ripetuti i campionamenti per il rilievo delle caratteristiche fisiche e biologiche dei fossi, che hanno evidenziato un netto e generale miglioramento dello stato ecologico, con passaggio delle classi di qualità IBE da III- \cellcolor{BurntOrange}IV(ambiente alterato o molto alterato) rilevate nel 2013 a I-II (ambiente non inquinato o con moderati sintomi di alterazione) nel 2015 (tabella \ref{tab_7}). Gli interventi hanno favorito, come atteso, la creazione di una elevata varietà di microhabitat con conseguente insediamento di una comunità di invertebrati più numerosa, diversificata ed equilibrata (figura \ref{fig_27}) e un aumento dell'abbondanza e della diversità di specie sensibili (figura \ref{fig_28}). La creazione di due stagni e il dissotterramento di un tratto di fosso a monte, hanno comportato un ulteriore aumento della diversità di habitat di buona qualità disponibili, importanti per numerose specie legate all'ambiente acquatico. E' stato inoltre ottenuto un generale miglioramento dal punto di vista paesaggistico. 

\begin{figure}
  \centering
  \includegraphics[width=.9\columnwidth]{fig_27.png}
  \caption{Analisi dei dati relativi alla comunità di macroinvertebrati bentonici rilevata nelle otto stazioni (st) monitorate nei fossi della Piana Rotaliana prima (2013, in rosso) e dopo (2015, in blu) gli interventi di riqualificazione oltre che nel fosso dissotterrato (FD) nel marzo 2016 e campionato nell'agosto 2016. A sinistra: abbondanza media di macroinvertebrati rilevata per stazione; a destra: indice di diversità di Shannon-Wiener rilevato per stazione}
  \label{fig_27}
\end{figure} 

Nella \texttt{ZSC IT3120030} Fontanazzo in Bassa Valsugana, i canali alimentati da risorgive sono stati invece ritenuti idonei alla reintroduzione del gambero di fiume senza necessità di riqualificazione. I canali, che scorrono all'interno di un'area boscata, presentano substrato naturale e una elevata disponibilità di rifugi creati anche dall'abbondanza di materiale legnoso e di radici in alveo (figura \ref{fig_29}). Inoltre, la classe di qualità ecologica IBE pari a I, rilevata nell'ambito di precedenti monitoraggi (Servizio Sviluppo Sostenibile e Aree Protette della Provincia Autonoma di Trento) indica l'assenza di alterazioni dell'habitat. 

\begin{figure}[H]
  \centering
  \includegraphics[width=.9\columnwidth]{fig_28.png}
  \caption{Analisi dei dati relativi ai soli taxa sensibili Ephemeroptera, Plecoptera, Trichoptera (EPT) rilevati nelle otto stazioni (st) monitorate nei fossi della Piana Rotaliana prima (2013, in rosso) e dopo (2015, in blu) gli interventi di riqualificazione oltre che nel fosso dissotterrato (FD) nel marzo 2016 e campionato nell'agosto 2016 A sinistra: abbondanza media di EPT rilevata per stazione; a destra: indice di diversità di Shannon-Wiener di EPT rilevato per stazione}
  \label{fig_28}
\end{figure} 


\begin{figure}[H]
  \centering
  \includegraphics[width=.8\columnwidth]{fig_29.jpg}
  \caption{\texttt{ZSC IT3120030} Fontanazzo: canale di risorgiva (foto: S. Endrizzi)}
  \label{fig_29}
\end{figure} 

Oltre alle caratteristiche ambientali è stata poi valutata l'eventuale presenza e densità di popolazioni autoctone e alloctone di gambero nelle aree circostanti e a monte dei siti selezionati per gli interventi, attraverso l'analisi di dati rilevati in precedenti studi e attraverso indagini effettuate appositamente. La presenza della specie aliena invasiva \emph{O. limosus}, nei laghi di Levico e Caldonazzo, ha portato a escludere come siti di reintroduzione il ruscello e il laghetto artificiali presenti all'interno della \texttt{ZSC IT3120030} Fontanazzo. I due bacini infatti, pur presentando buone condizioni ecologiche, sono alimentati dalle acque del Fiume Brenta, che nasce dai due grandi laghi, costituendo un possibile corridoio di trasmissione della peste del gambero verso valle. Per questo motivo gli interventi di reintroduzione nella \texttt{ZSC IT3120030} Fontanazzo sono stati successivamente programmati per i soli fossi alimentati da risorgive (figura \ref{fig_30}). 

\begin{figure}[!h]
  \centering
  \includegraphics[width=.8\columnwidth]{fig_30.jpg}
  \caption{\texttt{ZSC IT3120030} Fontanazzo. In azzurro sono rappresentati i canali naturali alimentati da risorgive mentre in blu gli ambienti acquatici, creati nell'ambito di opere di riqualificazione, alimentati dalle acque del Fiume Brenta (PAT, Aree Protette)}
  \label{fig_30}
\end{figure} 

\subsection{Allevamento e reintroduzione}
\label{sub_allevamento}
In considerazione della variabilità genetica rilevata tra le popolazioni di gambero di fiume nei diversi bacini idrologici del Trentino (cfr. \ref{sec_gene}) si è scelto di utilizzare come possibili popolazioni sorgente per l'attività di reintroduzione, due popolazioni presenti in siti vicini a quelli scelti per l'intervento: il Laghetto di Mezzocorona in Piana Rotaliana e il Rio Laguna presso Grigno in Bassa Valsugana. Le due popolazioni sono state sottoposte a controlli approfonditi per valutare l'idoneità al prelievo di femmine ovigere considerando struttura, densità e stato sanitario. I dati rilevati hanno evidenziato caratteristiche simili per le due popolazioni con una struttura per classi di taglia equilibrata (figura \ref{fig_12_a}), un rapporto sessi bilanciato e una densità, calcolata come CPUE (Catch Per Unit Effort: n gamberi catturati/n operatori/tempo impiegato per la cattura) corrispondente alla fascia medio-elevata rispetto ai valori rilevati nelle popolazioni di gambero studiate in Trentino (figura \ref{fig_11}). Le indagini sanitarie condotte nel 2012 \cite{Minghetti 2012b} non hanno inoltre rilevato infestazioni gravi nelle due popolazioni. Ulteriori controlli visivi effettuati nel 2013 non hanno evidenziato segni di infestazioni ad eccezione per la presenza di branchiobdellidi nella popolazione di Grigno, ossia ectoparassiti simbionti tipici dei crostacei, che non determinano rischi per la sopravvivenza delle popolazioni. Alcuni individui prelevati dalle popolazioni sorgente sono stati inoltre inviati all'istituto Zooprofilattico Sperimentale delle Venezie (dott. Manfrin) e sottoposti ad analisi per la ricerca di virus responsabili della Necrosi Pancreatica Infettiva (NPI), che produce mortalità nei salmonidi e può essere veicolata dai gamberi. Data la provenienza dei gamberi prelevati da bacini indenni da malattie virali dei salmonidi SEV (Setticemia Emorragica Virale) e NEI (Necrosi Pancretica Infettiva), come attestato dal Servizio Veterinario dell'ASL, non è stato ritenuto necessario svolgere analisi in tal senso \cite{Cappelletti 2016}.

In considerazione di questi risultati le due popolazioni sono quindi state ritenute idonee al prelievo di gamberi da destinare all'allevamento. Femmine ovigere sono state prelevate nel corso di due campagne di raccolta notturne nella primavera del 2013 e del 2014 e trasferite in allevamento presso il Centro Ittico della Fondazione E. Mach. L'attività di prelievo si è svolta con il coinvolgimento dell'Associazione Pescatori Dilettanti Trentini, per il sito di Mezzocorona, e dell'Associazione Pescatori Dilettanti della Vasugana, per il sito di Grigno. Per l'allevamento, sono state allestite vasche di ridotte dimensioni (30x85x30) dotate di substrati artificiali, a funzione di rifugio, e gabbie metalliche in acciaio a pareti forate (con fori di dimensioni 1x1) e supporto rialzato in PVC, per favorire la separazione dei nuovi giovani nati dagli individui adulti ed evitare fenomeni di cannibalismo. Le vasche sono state riempite con acqua di pozzo portata a temperatura simile a quella rilevata negli habitat di provenienza in modo tale da permettere una graduale acclimatazione all'ambiente di allevamento. Ciascuna vasca ha ospitato da 1 a 3 femmine ovigere che sono state sottoposte a controlli settimanali. Una volta rilasciati, i piccoli sono stati contati e riuniti in vasche più grandi fornite di acqua corrente, substrati artificiali, sassi e vegetazione acquatica, mentre le madri sono state reimmesse nell'habitat di provenienza. Nel 2013 sono state prelevate in tutto 8 femmine ovigere dal laghetto di Mezzocorona e 12 dal Rio Laguna che hanno portato rispettivamente alla produzione finale di 200 e 197 giovani. Nel corso del 2014 sono state prelevate ulteriori 22 femmine ovigere dal laghetto di Mezzocorona e 3 dal Rio Laguna con una produzione finale di 400 e 55 giovani rispettivamente. La mortalità dei nuovi nati è stata pari al 43\% per la popolazione di Mezzocorona e al 35\% per la popolazione del Rio Laguna \cite{Maiolini 2014} \cite{Cappelletti 2016}.

Nonostante la negatività al parassita \emph{A. astaci} della popolazione di Mezzocorona rilevata nel 2012 tramite analisi molecolari \cite{Minghetti 2012b}, nel corso del 2014 sono risultati evidenti segni di infestazione negli individui in allevamento. Le analisi richieste all'Istituto Zooprofilattico Sperimentale delle Venezie hanno confermato la positività al parassita \cite{Cappelletti 2016}. Il rilascio di questa popolazione nei fossi della Piana Rotaliana non è quindi stato possibile e l'intervento di reintroduzione è stato attuato per il solo sito della \texttt{ZSC IT3120030} Fontanazzo. Un primo rilascio di 132 individui è stato effettuata nella primavera del 2014. L'attività di reintroduzione è stata condotta con il coinvolgimento dell'Associazione Pescatori Dilettanti della Valsugana e il veterinario competente della zona e con la partecipazione del sindaco di Grigno, di un giornalista locale e di due classi delle scuole elementari di Grigno \cite{Maiolini 2014} (figura \ref{fig_31}).

\begin{figure}[!h]
  \centering
  \includegraphics[width=.8\columnwidth]{fig_31.jpg}
  \caption{Reintroduzione di individui di \emph{A. pallipes} in un canale di risorgiva della \texttt{ZSC IT3120030} Fontanazzo (Foto Maiolini B.)}
  \label{fig_31}
\end{figure}

I monitoraggi effettuati nel 2015 e nel 2016 per il controllo della popolazione reintrodotta non hanno permesso di rilevare la presenza di gamberi. La reintroduzione di gamberi di età 1+ nel periodo primaverile, particolarmente delicato per via delle numerose mute affrontate nelle fasi giovanili, può aver reso più difficoltosa la loro sopravvivenza. E' comunque da considerare che il rilievo di pochi esemplari in un ambiente ampio e difficilmente accessibile può risultare difficoltosa. Gli individui introdotti sarebbero infatti in grado di riprodursi, e aumentare la densità della popolazione e la conseguente contattabilità, solo a partire dal terzo o quarto anno di vita. La reintroduzione degli individui allevati a partire dal 2014 sono quindi stati mantenuti per precauzione in allevamento più a lungo e saranno rilasciati nell'autunno del 2017. Secondo l'esperienza maturata attraverso il Progetto CRAINat in Lombardia (\cite{CRAINAT}), il ritrovamento di almeno il 2\% (corrispondente nel nostro caso a n. 3 - 4 individui) dei gamberi rilasciati cosituisce indicatore di successo dell'intervento e, nel caso sia rilevato nel corso dei monitoraggi post-intervento un mancato insediamento o la mancata crescita della popolazione reintrodotta, è consigliata l'attuazione di ulteriori interventi di rinforzo attraverso il rilascio di ulteriori stock, a meno che non vengano rilevati fattori evidenti che interferiscono con la sopravvivenza dei gamberi. La reintroduzione ha rappresentato un prima fase sperimentale di questo tipo di attività in Trentino, e al di là dei primi risultati raggiunti, l'azione dimostrativa ha rappresentato un esempio importante per la messa a punto dei futuri interventi di reintroduzione che si vorranno attuare sul territorio, sia in ambienti naturali che ripristinati.

\subsection{Divulgazione}
L'azione C10 (Azione dimostrativa di tutela di specie: salvaguardia delle popolazioni autoctone di gambero di fiume) del Progetto Life+TEN è stata oggetto di divulgazione pubblica nel corso di  due eventi: l'Expo Caccia, Pesca e Ambiente 2017, tenuto a Riva del Garda (TN) dal 25 al 26 marzo 2017, nel corso del quale è stato allestitito uno banco espositivo sul gambero, all'interno di un'esposizione più ampia sulle specie aliene acquatiche organizzata dal MUSE e dal Servizio Faunistico PAT (figura \ref{fig_32}), e il laboratorio intitolato "L'INVASIONE DEGLI ALIENI" all'interno dell'evento "il Trentino per la BIOdiversità" organizzato dalla Fondazione Edmund Mach il 20 maggio 2017, in occasione della giornata mondiale per la biodiversità (figura \ref{fig_33}). Agli eventi sono stati esposti tre acquari con esemplari vivi delle tre specie di gambero presenti in Trentino, corredate da pannelli esplicativi delle caratteristiche morfologiche che hanno aiutato ad attrarre l'attenzione del pubblico e a illustrare le caratteristiche morfologiche e la biologia delle tre specie. Gli eventi hanno rappresentato un importante momento di confronto con i pescatori del Trentino e delle regioni limitrofi, di divulgazione e sensibilizzazione del pubblico generico, utile anche alla raccolta di alcune nuove segnalazioni. Nel corso della fiera e del laboratorio è inoltre stato distribuito il depliant intitolato "il Gambero di fiume in Trentino" (figura 32) realizzato nell'ambito dell'azione C10 del Progetto Life+TEN e contenente informazioni sulla biologia e distribuzione di \emph{A. pallipes} e le principali minacce a cui è soggetta la specie, alcune buone pratiche per la sua salvaguardia, e una chiave dicotomica illustrata per il riconoscimento delle tre specie presenti in Trentino (\emph{A. pallipes}, \emph{O. limosus}, \emph{P. clarkii}).

\begin{figure}
  \centering
  \includegraphics[width=.7\columnwidth]{fig_32.jpg}
  \caption{Immagine del volantino "il Gambero di fiume in Trentino" (in alto) e dell'allestimento (in basse) presso l'Expo Caccia, Pesca e Ambiente 2017 di Riva del Garda (TN) del 25-26 marzo 2017 (in basso; foto Endrizzi S.)}
  \label{fig_32}
\end{figure}

\begin{figure}
  \centering
  \includegraphics[width=.7\columnwidth]{fig_33.jpg}
  \caption{Allestimento "L'INVASIONE DEGLI ALIENI", evento "il Trentino per la BIOdiversità", Fondazione Edmund Mach 20 maggio 2017, Giornata Mondiale per la Biodiversità (foto Endrizzi S.)}
  \label{fig_33}
\end{figure}

\chapter{Strategie e interventi del piano di gestione}
\DeactivateBG
\AddToShipoutPicture*{\BackgroundPicTwo}

Gli interventi gestionali a favore di popolazioni a rischio di estinzione o minacciate si suddividono in due tipologie, a seconda del luogo in cui si attua la gestione:
\begin{itemize}
  \item strategie di protezione \emph{in situ}: si attuano nell'ambiente naturale delle specie interessate e possono riguardare sia interventi diretti sulle popolazioni sia interventi sugli habitat;
  \item strategie di protezione \emph{ex situ}: la gestione è effettuata all'esterno dell'ambiente naturale della specie oggetto degli interventi, in strutture apposite. Nel caso del gambero, in centri di allevamento e riproduzione utilizzati per l'ottenimento del materiale destinato alla "semina", ovvero alle attività di reintroduzione e di ripopolamento.
\end{itemize}

Pertanto, il presente piano intende offrire proposte gestionali utili a garantire la conservazione del gambero di fiume in Trentino e a favorirne una nuova espansione attraverso: la naturale diffusione delle popolazioni residue, dalle aree rifugio a tutto il corpo idrico mediante l'eliminazione dei fattori che ne hanno determinato l'isolamento e l'attuazione di interventi di reintroduzione e ripopolamento in ambienti in cui il gambero è estinto, utilizzando stock opportunamente selezionati ed allevati. Questi risultati possono essere ottenuti solo attraverso una profonda conoscenza delle popolazioni ancora presenti, della loro genetica, dei loro habitat e dei fattori di minaccia che ne stanno determinando il declino. Di seguito saranno indicate nel dettaglio le azioni e i metodi di applicazione. 

\section{Monitoraggio delle popolazioni e dei loro habitat}
\label{sec_monitoraggio_pop}
I monitoraggi svolti a partire dal 2010 hanno permesso di definire, in modo preliminare, la distribuzione residua e l'ecologia di \emph{A. pallipes} in alcune aree del Trentino. Per ottenere informazioni utili a definire lo stato di conservazione della specie autoctona e l'integrità dei suoi habitat risulta però necessario estendere lo studio a tutto il territorio provinciale. Le informazioni relative alla presenza, struttura e dinamica delle popolazioni e alle caratteristiche dei loro habitat, opportunamente archiviate, organizzate, elaborate e costantemente aggiornate, rappresentano infatti gli strumenti di base per lo sviluppo di politiche di gestione efficaci. 


\subsection{Scelta dei siti da monitorare} 
Per la programmazione dei monitoraggi è necessario innanzitutto individuare i siti da osservare attraverso un'attenta analisi di materiale cartografico, dei dati disponibili sulla presenza/assenza storica e attuale della specie, oltre che sulle caratteristiche ecologiche e idromorfologiche dei corpi idrici e di uso del suolo degli ambienti in cui si inseriscono. Per i corsi d'acqua esistono utili strumenti di supporto che permettono di estrapolare i tratti caratterizzati dalla presenza di habitat potenzialmente idonei per il gambero come il database IFF realizzato dall'APPA, relativamente a una selezione di corsi d'acqua tipizzati della provincia per un totale di 1519 km \cite{APPA 2013}, e la modellazione a meso-scala (MesoHABSIM - Mesohabitat Simulation Model) \cite{Parasiewicz 2007} \cite{Vezza 2014} in fase di realizzazione anche per i corsi d'acqua Trentini, che possono essere integrati da ulteriori informazioni ricavabili dai report periodici dell'APPA sullo stato di corsi d'acqua, laghi e bacini artificiali.

La selezione dei siti, fatta preventivamente su base cartografica, dovrà essere comunque affinata sul campo al momento del primo rilievo considerando l'idoneità in rapporto alla disponibilità di rifugi (vedi tabella \ref{tab_8}) e di altri elementi importanti per il gambero come il tipo di substrato, la presenza di vegetazione sommersa e riparia, l'ombreggiamento e così via (cfr. \ref{subsub_hab}). E' infine importante considerare l'accessibilità al sito, che deve essere agevole anche di notte (momento migliore per effettuare i monitoraggi), in modo da garantire la sicurezza degli operatori durante i campionamenti. 

Il numero e la distribuzione dei siti da indagare deve essere tale da fornire un'indicazione efficace dell'effettiva presenza/assenza e abbondanza delle popolazioni. Per i censimenti (rilievo presenza/assenza e conteggio per calcolo densità cfr. \ref{subsub_cens}) sarà quindi utile individuare siti che coprano complessivamente almeno il 30\% dei tratti di corso d'acqua o di sponda lacustre potenzialmente idonei a ospitare popolazioni di gambero. All'interno del tratto complessivamente censito dovrà essere successivamente selezionato (scegliendo per esempio il tratto in cui la densità di gamberi risulta più elevata) almeno un sito per il campionamento (struttura delle popolazioni, sesso, dati morfometrici, eventuali campioni per analisi genetica, cfr. \ref{subsub_cens}) di 100 - 200 m rappresentativo della popolazione presente.  

I siti selezionati saranno georeferenziati e inseriti nel database predisposto per la gestione dei dati (se necessario i punti di inizio e di fine del sito di campionamento potranno essere segnalati sul campo in modo tale da permetterne la corretta localizzazione) anche nel corso dei monitoraggi successivi.

\begin{table}[]
\centering
\begin{tabular}{@{}p{.33\columnwidth}p{.33\columnwidth}p{.33\columnwidth}@{}}
\toprule
\textbf{Condizioni con livello preferenza massima}     & \textbf{Livello preferenza medio} & \textbf{Livello preferenza minimo (o condizioni evitate)}     \\ \midrule

Massi e sassi (diam \textgreater25 cm)   & Ciottoli (diametro 15-25 cm)  & Ghiaia (diametro 5-15 cm) \\

\rowcolor[HTML]{EFEFEF} Pozze e tratti con flusso laminare  a velocità  ridotta (se con rifugi)   & raschio     & Tratti ad elevata energia quali rapide (tratti evitati)   \\

Velocità  locale 0.1 m s-1 o minori      & Velocità  locale minore di 0.2 m s-1  & Velocità  locale maggiore di 0.2 m s-1 (tratti evitati)   \\

\rowcolor[HTML]{EFEFEF} Massi e sassi in mucchi con piccoli spazi e fessure    & Massi e sassi isolati su substrato fine di ciottoli e ghiaia  & Solo substrato fine di ciottoli e ghiaia    \\

Fessure profonde in roccia & Massi e sassi appiattiti  & Ciottoli rotondeggianti (rotolano facilmente)   \\

\rowcolor[HTML]{EFEFEF} Substrato sottostante di sabbia/ghiaia fine con pochi ciottoli & Ciottoli e ghiaia grossolana  & Limo e argilla \\

Massi liberi &  & Massi incastrati in un letto compatto   \\

\rowcolor[HTML]{EFEFEF} Rifugi sommersi in sponde stabili (es: fessure naturali, sponde con erosione basale e vegetazione sporgente, grosse radici) & Rifugi in zone a flusso ridotto   & Rifugi nelc entro del canale (specialmente se in tratti a raschio o correntino) \\

Zone adiacenti a habitat favorevoli sulla sponda & Zone adiacenti a sponde senza rifugio (es: sponde poco inclinate) & Zone adiacenti a sponde collassanti o in erosione attiva  \\ \bottomrule
\end{tabular}
\caption{Strutture naturali utilizzate da \emph{A. pallipes} come rifugio e grado di preferenza}
\label{tab_8}
\end{table}

\subsection{Monitoraggio degli habitat} 
\ActivateBG
Le procedure di rilevamento degli habitat possono essere standardizzate tramite l'adozione di un protocollo di rilevamento ambientale speditivo e di utilizzo relativamente facile, che prevede una valutazione delle caratteristiche ambientali negli stessi tratti del corpo idrico in cui sono realizzati i censimenti astacicoli. Tale protocollo è stato adattato mediando quanto suggerito dai progetti CRAINat (che prevede l'utilizzo del protocollo Habitat Assessment \cite{Barbour 1999}, RARITY (che propone l'utilizzo delle schede dell'Indice di Funzionalità Fluviale \cite{Siligardi 2007}, e dal progetto Life in UK Rivers \cite{Peay 2003}, tenendo in considerazione le caratteristiche dei corpi idrici trentini in cui è stata già rilevata la presenza di A. pallipes e le caratteristiche ecologiche descritte da Holdich \cite{Holdich 1999}. Le schede sono presentate in Allegato I.

La scheda di rilevamento ambientale è descrittiva, riporta le caratteristiche del tratto esaminato come l'uso del suolo nell'area circostante, la struttura delle sponde, la composizione del substrato, della vegetazione acquatica e riparia, la presenza di potenziali rifugi e di minacce per A. pallipes oltre a prevedere la misurazione di parametri chimico-fisici quali temperatura ($^{\circ}$C), conducibilità ($\frac{\mu S}{cm}$), pH, ossigeno disciolto (O2 ppm, O2 \%) e torbidità. Queste misure vengono agevolmente rilevate in campo mediante l'utilizzo di apposite sonde. La qualità dell'habitat fluviale viene inoltre valutata utilizzando l'Indice Biotico Esteso (IBE). Le classi di qualità ottenute con questo metodo sono integrate nella scheda di valutazione degli habitat (nella sezione "comunità macroinvertebrati"). In caso di variazioni spaziali delle caratteristiche ambientali nei siti di monitoraggio considerati, dovranno essere compilate più schede in modo da ottenere una descrizione dettagliata degli habitat disponibili e poter mettere in relazione eventuali variazioni nella densità e struttura delle popolazioni rilevate alla variabilità ambientale. 

I dati raccolti sul campo dovranno essere integrati con informazioni rilevanti ai fini della gestione delle popolazioni di gambero ottenute attraverso l'analisi cartografica e bibliografica, che permettono il rilievo delle caratteristiche ambientali su scala più ampia rispetto al sito osservato sul campo, come: l'uso del suolo o la presenza di minacce a livello di bacino idrografico, l'estensione, l'origine e il carattere del corso d'acqua, la superficie e la profondità del lago e così via. Tali informazioni saranno inserite nel database nella fase di programmazione dei siti da monitorare. 

Il rilievo dei dati ambientali è fondamentale ai fini della conservazione delle popolazioni in quanto permette di individuare i corpi idrici in cui: 1) intervenire con azioni di riqualificazione per il miglioramento dello stato qualitativo dell'habitat e la conseguente mitigazione dell'isolamento delle popolazioni; 2) attuare misure utili all'eliminazione o alla mitigazione dei fattori di minaccia; 3) effettuare ripopolamenti e reintroduzioni in ambienti particolarmente favorevoli.

\subsection{Monitoraggio delle popolazioni}
\subsubsection{Segnalazioni}
\label{subsub_segn}
Le segnalazioni di presenza delle popolazioni autoctone o alloctone di gambero costituiscono un'importante base di partenza per la scelta delle aree e dei siti da indagare e per avviare procedure urgenti di eradicazione di popolazioni alloctone appena insediate. Per la raccolta di segnalazioni si propone il coinvolgimento di: Corpo Forestale e custodi forestali, associazioni di pescatori e cittadini, informati attraverso le azioni di divulgazione del progetto (cfr. \ref{sec_div}).
Le segnalazioni potranno essere raccolte mediante uno dei seguenti metodi, a seconda del target di rilevatori che si vuole intercettare (cittadini e associazioni o custodi forestali):
\begin{itemize}
  \item l'utilizzo di applicativi per dispositivi mobili che consentono di registrare le osservazioni del mondo naturale. Ad esempio l'applicazione iNaturalist (\url{www.inaturalist.org}) è già predisposta per la creazione di progetti specifici che permetteranno di informare gli utenti relativamente alla ricerca in corso e alla necessità di raccogliere informazioni sulla distribuzione di determinati \emph{taxa}. Le segnalazioni saranno accompagnate da immagine fotografica dell'individuo osservato e validate dagli altri utenti; 
  \item la compilazione di un'apposita scheda di segnalazione (Allegato II) che sarà resa disponibile: in forma cartacea, presso le stazioni forestali e le associazioni di pescatori e dovrà essere restituita agli operatori individuati per la gestione del database generale, e in forma elettronica sul WebGIS dell'Azione A1 del Progetto LIFE+TEN ed eventualmente sul sito predisposto per il progetto (cfr. \ref{sub_dat}).
\end{itemize}

Questi metodi (ad eccezione del sistema cartaceo, il cui utilizzo deve quindi essere limitato) permettono di raccogliere e archiviare le informazioni all'interno di un database per la gestione dei dati (cfr. \ref{subsub_dat}) in modo del tutto automatico e quindi senza consistenti oneri di gestione.
 
\subsubsection{Censimenti e campionamenti}
\label{subsub_cens}
Censimenti e campionamenti devono essere effettuati nel periodo di massima attività dei gamberi, ovvero da tarda primavera a inizio autunno, da personale opportunamente addestrato, utilizzando metodi sperimentati e standardizzati, come indicato per A. pallipes nel manuale ISPRA per il monitoraggio di specie animali di interesse comunitario \cite{Scalici 2016}, che si possono riassumere in diverse modalità a seconda del tipo di ambiente: 1) in corsi d'acqua poco profondi, con velocità di corrente ridotta e buona visibilità del fondale: mediante ricerca notturna, a vista con torcia e/o ricerca diurna, ispezionando i probabili rifugi (tabella \ref{tab_8}), e cattura a mano e con retino da pesca, percorrendo da valle a monte le sub-unità e i siti selezionati; 2) in corsi d'acqua profondi: mediante nasse, reti da lancio, immersioni subacquee; 3) nei laghi: mediante immersioni subacquee (o snorkeling), o ricerca notturna di gamberi attivi lungo le rive; 4) in presenza di fondali ricchi di vegetazione, limosi e poco profondi: mediante una rete per macroinvertebrati.

Per il monitoraggio delle popolazioni in Trentino si propone il censimento/campionamento notturno con ausilio di torcia e retino per il rilievo dei gamberi in attività, sia in laghi che in corsi d'acqua. Il primo passo consiste quindi nel predisporre dei censimenti annuali (nel contesto faunistico, per censimento si intende la rilevazione, mediante specifici piani di campionamento i cui risultati sono trattabili statisticamente, di tutte le popolazioni, in questo caso di A. pallipes, presenti sul territorio indagato) svolti a rotazione, sia nei corpi idrici in cui sono già state rilevate popolazioni di gambero, sia in corpi idrici che potenzialmente potrebbero ospitarli (siti già monitorati in passato in cui \emph{A. pallipes} risulta estinto e siti mai indagati precedentemente). L'obiettivo dei censimenti è infatti quello di ottenere dati costantemente aggiornati sulla presenza/assenza e l'abbondanza (calcolata come Catch Per Unit Effort (CPUE): (numero di gamberi catturati) / (numero operatori) / (tempo impiegato per la cattura) delle popolazioni attraverso metodi speditivi quali: il conteggio dei gamberi osservati in ciascun sito selezionato considerando il tempo impiegato per il rilievo.  
Sulla base dei risultati dei censimenti saranno individuati i siti di campionamento, ossia tratti di corso d'acqua o di sponda di almeno 100 m., ritenuti rappresentativi della popolazione presente ossia dove la densità di gamberi risulta elevata e con presenza di individui di classi di taglia diversa. Il campionamento prevede la cattura dei gamberi nel corso di una o più sessioni tenendo conto del numero di campionatori e del tempo impiegato per percorrere l'intero tratto considerato. I gamberi catturati dovranno essere stoccati temporaneamente in vaschette con acqua e, al termine della sessione di cattura, dovranno essere pesati (microbilancia di precisione digitale con precisione 0.0083 g), misurati (rilievo della lunghezza del cefalotorace con utilizzo di un calibro) e caratterizzati per sesso. Saranno inoltre raccolte eventuali osservazioni su: segni di danneggiamento presenti sul corpo dell'animale, come chele perse o rigenerate, segni di infestazioni da parassiti e patogeni ecc. (vedi Allegato IV). Il campionamento delle popolazioni, eseguito ogni tre anni, permette di ottenere informazioni più dettagliate sullo stato delle popolazioni considerando, oltre alla densità (CPUE), anche la struttura per sesso e taglia, e permettendo di fare delle previsioni sull'incremento/decremento delle popolazioni negli anni successivi e programmare eventuali strategie per sostenere le popolazioni a rischio. Studi più approfonditi delle popolazioni, attraverso il metodo della cattura-marcatura-ricattura, sono inoltre consigliati per valutare lo stato di popolazioni e habitat di particolare interesse, come nel caso di reintroduzioni o interventi di riqualificazione e per l'elaborazione di migliori strategie di gestione delle popolazioni e dei loro habitat. Questo metodo, basato sulla possibilità di riconoscimento individuale dei gamberi attraverso l'apposizione di una marca sull'animale durante la prima occasione di cattura, pur richiedendo uno sforzo maggiore, permette infatti di:
\begin{itemize}
  \item calcolare la densità della popolazione in modo molto preciso, espressa in numero d'individui per unità di superficie;
  \item di ottenere informazioni dettagliate sul movimento degli individui, utili a valutare ad esempio le modalità di utilizzo dei rifugi e dei diversi microhabitat e a stimare la velocità di colonizzazione di nuovi ambienti;
  \item stimare parametri demografici (sopravvivenza, fecondità, dispersione) e analizzare le dinamiche di popolazione e i fattori che la influenzano.
\end{itemize}

L'applicazione di questo metodo richiede il campionamento notturno delle popolazioni e dovrà prevedere più sessioni di cattura (almeno tre) effettuate in giornate consecutive. Per uno studio sul movimento degli individui deve essere annotato il punto esatto in cui avviene la cattura di ciascun gambero apponendo un apposito indicatore sul campo che dovrà essere in seguito georeferenziato. Sebbene Scalici \etal \cite{Scalici 2016} scoraggino l'utilizzo di questo metodo per la sua invasività (in quanto potrebbe essere causa di infezioni) e per la possibile perdita della marcatura a seguito dei numerosi eventi di muta, risultati soddisfacenti sono stati ottenuti nell'ambito del progetto CRAINAT, attraverso l'utilizzo di pit-tags, per monitorare popolazioni reintrodotte e per valutare gli effetti di interventi di riqualificazione di habitat. L'applicazione del metodo, attraverso l'utilizzo di pit-tags per studi di lungo periodo o inchiostri indelebili da applicare sull'esoscheletro per studi a breve termine (si veda AA. VV., progetto CRAINAT \cite{AA. VV. 2014}, par 4.3.3., sulle metodologie di marcatura), è pertanto consigliata se motivati da necessità di studio per la programmazione e la valutazione di interventi di gestione delle popolazioni e degli habitat
.
Infine, nell'ottica di ottimizzazione dei costi e degli sforzi di monitoraggio per il rilievo di presenza/assenza nel maggior numero possibile di siti potenzialmente idonei ad ospitare popolazioni di A. pallipes, potrà essere considerato come strumento esplorativo, in alternativa alla ricerca attiva degli animali, l'analisi del DNA ambientale (eDNA, environmental DNA) (Rees 2014; Goldberg 2016). Tale approccio si basa sull'analisi del DNA estratto da un campione ambientale, nello specifico da un campione di acqua, che contine molecole rilasciate dall'organismo di interesse attraverso saliva, feci, urina o cellule. Attraverso motodiche di barcoding del DNA è quindi possibile evidenziare la presenza recente di un determinato organismo nell'ambiente campionato, senza che l'organismo stesso debba essere direttamente campionato o individuato. In caso di rilievo di presenza della specie si procede comunque con l'individuazione e il censimento/campionamento della popolazione presente al fine di valutarne lo stato di conservazione (tabella \ref{tab_9}). 

\vspace{1cm}
\begin{table}[ht!]
\centering
\begin{tabular}{@{}p{.15\columnwidth}p{.15\columnwidth}p{.15\columnwidth}p{.15\columnwidth}p{.15\columnwidth}@{}}
&&&&& \\
&&&&& \\
\textbf{Sito}     & \rotatebox{90}{\textbf{Censimenti}} & \rotatebox{90}{\parbox{2cm}{\textbf{Monitoraggio \\habitat}}}& \rotatebox{90}{\parbox{2cm}{\textbf{Campionamento \\popolazioni}}}  & \rotatebox{90}{\parbox{2cm}{\textbf{Campioni \\DNA}}}  \\
\midrule
1. Siti con  documentata presenza di gamberi     & Annuale, ripetere in caso di cambiamenti significativi (opere in alveo, inquinamento, ecc.) & Triennale, ripetere in caso di cambiamenti significativi (opere in alveo, inquinamento, ecc.) & Triennale, o qualora si verifichi evidente contrazione  della popolazione (osservata attraverso i censimenti) & Una volta, se nuova popolazione e in caso di contrazione demografica di  popolazione pre-esistente (per  analizzare eventuale perdita di diversità ) \\
\rowcolor[HTML]{EFEFEF}2. Nuovi siti potenzialmente adatti a \emph{A. pallipes} & Annuale, al primo rilievo& Come al punto 1 se rilevata la presenza       & Come al punto 1 se rilevata la presenza    &\\
3. Siti in cui \emph{A. pallipes} è considerato estinto & Triennale, al primo rilievo& Come al punto 4 se rilevata la presenza       & Come al punto 1 se rilevata la presenza    &\\
\rowcolor[HTML]{EFEFEF}4. Siti di reintroduzione di \emph{A. pallipes}& Annuale, ripetere in caso di cambiamenti significativi (opere in alveo, inquinamento, ecc.) & Come al punto 1. & Annuale per i primi cinque anni, poi come punto 1   & Dopo cinque anni dalla reintroduzione o in caso di  significativa contrazione demografica  \\ \bottomrule
\end{tabular}%
\caption{Elenco delle attività previste per il censimento (rilievo di presenza/assenza e densità (CPUE) e il monitoraggio (rilievo di dati per la stima di densità (CPUE) e della struttura (peso, lunghezza, sesso) delle popolazioni di gambero e degli habitat (compilazione delle schede predisposte per il Trentino) e la raccolta di campioni da destinare all'analisi genetica con indicazione della cadenza temporale secondo indicazioni ISPRA: Manuali per il monitoraggio di specie e habitat di interesse comunitario (Direttiva 92/43/CEE) in Italia: specie animali (\cite{Scalici 2016})}
\label{tab_9}
\end{table}

\subsection{Analisi genetica di nuove popolazioni} 
I dati genetici attualmente disponibili necessitano integrazioni. Tutte le popolazioni di gambero precedentemente censite e campionate ma non analizzate da un punto di vista genetico, dovranno essere caratterizzate mediante l'utilizzo delle stesse metodiche di indagine precedentemente adottate, basate sull'analisi di marcatori del DNA mitocondriale e del DNA nucleare. Analogamente, tutte le nuove popolazioni rilevate nel corso delle attività di monitoraggio dovranno essere sottoposte ad analisi genetica. Inoltre, qualora nel corso delle attività di monitoraggio si rilevi, in popolazioni già caratterizzate geneticamente, una significativa contrazione demografica, una nuova analisi genetica degli individui residui sarà in grado di evidenziare se alla riduzione della dimensione di popolazione corrisponda una significativa perdita di diversità genetica.

Al fine di consentire le analisi genetiche, ogni individuo campionato nel corso dei monitoraggi dovrà essere sottoposto ad un prelievo di tessuto attraverso l'asportazione del terzo arto ambulacrale, che potrà essere in seguito rigenerato dall'animale. Il prelievo dovrà essere fatto mediante utilizzo di materiale sterile e il campione dovrà essere conservato in etanolo 90\% e, una volta in laboratorio, in congelatore a -20$^{\circ}$C oppure a -80$^{\circ}$C nel caso di stoccaggio di lungo periodo.

Al fine di consentire una maggiore comprensione e una più esaustiva interpretazione dei dati, il dataset genetico disponibile a livello provinciale dovrebbe essere integrato con quelli disponibili per altre realtà territoriali in cui Austopotamobius sia stato oggetto di indagine (ad es. CRAINat e RARITY). A tal fine dovrà essere favorita la condivisione di campioni da sottoporre ad analisi genetiche secondo le medesime metodiche analitiche, tali da poter produrre dati pienamente confrontabili e pertanto maggiormente informativi.

\subsection{Identificazione dei fattori di minaccia} 
\label{sub_identific_min}
Tutte le informazioni ottenute, relative a presenza, abbondanza e struttura delle popolazioni di gambero, alla loro composizione genetica e allo stato degli habitat, permettono di identificare la presenza e l'intensità dei fattori di minaccia che gravano su popolazioni e habitat e che sono riconducibili alle seguenti categorie:
\begin{description}
  \item[degrado ambientale] l'artificializzazione dell'alveo, la distruzione della fascia riparia, le opere di captazione idrica, le attività di svaso e gli eventi di inquinamento diffuso e puntiforme possono causare l'isolamento delle popolazioni in aree naturali residue, con conseguente indebolimento per interruzione del flusso genico, o nel peggiore dei casi, l'estinzione anche improvvisa delle popolazioni; 
  \item[diffusione di patogeni] la trasmissione di patogeni, e in particolar modo della peste del gambero, direttamente per introduzione e propagazione di specie alloctone, o indirettamente per trasferimento di materiali e attrezzature da pesca da un bacino all'altro causa l'indebolimento e l'estinzione delle popolazioni;
  \item[diffusione di specie alloctone] la competizione con specie alloctone, spesso più produttive e resistenti al degrado ambientale oltre che portatrici di patogeni letali, può causare l'isolamento e l'estinzione delle popolazioni native; 
  \item[prelievo illegale] il prelievo di gamberi anche in misura limitata può causare l'estinzione di popolazioni già indebolite da altri fattori di minaccia.
\end{description}

L'individuazione della presenza e intensità dei fattori di minaccia è di fondamentale importanza per la conservazione delle popolazioni e degli habitat rilevati oltre che per l'individuazione delle aree di rischio, di ripristino e di reintroduzione. Tutti i possibili fattori di minaccia sono rilevati mediante la scheda di rilievo ambientale (in Allegato I) e quindi monitorati in occasione di ciascun censimento/monitoraggio. 

\section{Conservazione e ripristino degli habitat e delle popolazioni}
Gli interventi di gestione di specie minacciate e protette implicano, oltre agli interventi diretti sulle popolazioni, anche azioni sul territorio realizzate su scale geografiche diverse. Per le popolazioni del Trentino l'elaborazione delle informazioni ottenute nella prima fase di monitoraggio e di analisi di laboratorio (distribuzione storica e attuale delle specie di decapodi, status e caratterizzazione genetica delle popolazioni native, condizioni degli habitat e presenza di minacce) permetterà di individuare possibili aree nelle quali potranno essere attuati nel tempo diversi tipi di interventi e in particolare:
\begin{description}
  \item[aree di ripristino] ambienti in cui il miglioramento della funzionalità ecologica degli habitat, attraverso adeguati interventi di riqualificazione (cfr. \ref{sub_riqhab}), può migliorare lo status di popolazioni presenti, favorire l'espansione di popolazioni isolate limitando così la frammentazione e aumentando il flusso genico, e offrire ambienti ottimali a ospitare nuove popolazioni; 
  \item[aree di rischio] ambienti interessati dalla presenza di popolazioni alloctone di gambero, da gravi infestazioni di patogeni, in particolare da \emph{A. astaci}, da degrado ambientale e/o da ripetuti episodi di bracconaggio, che possono determinare un rischio per la conservazione delle popolazioni autoctone e dei loro habitat e devono quindi essere tenute sotto stretto controllo attraverso adeguate misure di intervento (cfr. \ref{sub_prev}); 
  \item[aree rifugio] ambienti che per le caratteristiche ecologiche particolarmente favorevoli risultano idonee alla conservazione di popolazioni abbondanti che possono svolgere anche la funzione di sorgente di individui per interventi di reintroduzione e di ripopolamento (source area, stagni multifunzionali). Possono essere rappresentate da siti isolati, con acqua corrente e/o ferma, che non siano minacciati da gamberi alieni e da peste del gambero e che possano sostenere popolazioni di gambero con un impegno gestionale nullo o minimo; 
  \item[aree di reintroduzione e di ripopolamento] ambienti che per le caratteristiche ecologiche si mostrano particolarmente idonei a sostenere popolazioni stabili e possono essere oggetto di immissione di individui per la creazione di nuclei forti ed equilibrati (cfr. \ref{sub_studio}).
\end{description}

\subsection{Riqualificazione di habitat} 
\label{sub_riqhab}
Il degrado e la scomparsa degli habitat acquatici sono una delle principali cause di estinzione e di isolamento delle popolazioni di \emph{A. pallipes}. Interventi di riqualificazione degli habitat, attraverso metodiche di restoration ecology che consistono nel ripristino delle capacità funzionali dell'ecosistema/habitat della specie, dovrebbero quindi essere implementati nelle aree di sottobacini idrografici e/o corpi idrici che, sulla base dei monitoraggi periodici, risultano:
\begin{enumerate}
  \item aver ospitato popolazioni di \emph{A. pallipes} vitali e di elevata abbondanza che però sono andate declinando in tempi recenti per un peggioramento dell'habitat dovuto a regimazione, e/o artificializzazione delle rive;
  \item non ospitare popolazioni di \emph{A. pallipes} ma che possono rappresentare corridoi ecologici tra aree in cui le popolazioni sono presenti; 
  \item ospitare popolazioni di \emph{A. pallipes} vitali, con diversità genetica relativamente elevata, che potrebbero ulteriormente aumentare se la qualità dell'habitat fosse migliorata.
\end{enumerate}

Considerando le caratteristiche fisiche degli habitat che influenzano la presenza dei gamberi potranno quindi essere attuati gli interventi descritti nei seguenti sottocapitoli, in rapporto alle criticità evidenziate nelle schede di rilevamento ambientale compilate durante i monitoraggi.

\subsubsection{Rinaturalizzazione dei corpi idrici}
La rettifica, l'artificializzazione e l'interramento dei corsi d'acqua comporta la perdita degli habitat essenziali alle diverse fasi del ciclo biologico del gambero. Per favorire la conservazione e l'espansione delle popolazioni è quindi importante individuare i tratti modificati per riportarli alle condizioni naturali attraverso: 
\begin{itemize}
  \item l'eliminazione di condotte sotterranee, riportando in superficie il corso d'acqua e garantendo così la continuità longitudinale dell'ecosistema acquatico;
  \item la rimozione di rivestimenti in cemento, di pavimentazioni del fondo e delle sponde in modo da riportare allo scoperto il substrato naturale sottostante e offrire una elevata disponibilità di microhabitat differenziati;
  \item il rimodellamento dei tratti di corso d'acqua caratterizzati da una morfologia rettilinea, attraverso operazioni di scavo, per il conferimento di una struttura che favorisca la divagazione laterale della corrente e il conseguente aumento della diversità di habitat disponibili;
  \item l'aumento della portata, nel caso di canali o corsi d'acqua interessati da captazioni idriche, per garantire il mantenimento del deflusso minimo vitale e favorire l'asportazione di depositi limosi;
  \item la realizzazione di trappole per sedimenti, nei casi in cui il corso d'acqua/canale sia periodicamente interessato da interventi di asportazione del sedimento depositato in alveo. In questo modo si possono evitare operazioni di scavo lungo l'intero corpo idrico e i danni arrecati da queste pratiche alle specie acquatiche;
  \item la creazione di una fascia di vegetazione riparia, nei tratti in cui è stata eliminata o eccessivamente ridotta, attraverso la piantumazione di specie autoctone tipiche di questi ambienti (come \emph{Alnus glutinosa}, \emph{Alnus incana}, \emph{Populus alba}, \emph{Populus nigra}, \emph{Salix alba}), che svolga funzione di ombreggiamento, apporto di materia organica e di barriera nei confronti di pesticidi ed altri inquinanti provenienti da emissioni diffuse e tampone per l'intercettazione dei nutrienti rilasciati dai terreni agricoli (fosfati e composti azotati responsabili dei fenomeni di eutrofizzazione). 
\end{itemize}
L'applicazione di queste pratiche dovrà essere valutata in rapporto al tipo di corso d'acqua, delle caratteristiche del tratto considerato tenendo conto del rischio idraulico, della proprietà privata e delle esigenze dei diversi fruitori della risorsa idrica. Indicazioni tecniche dettagliate sono indicate in diversi manuali per la riqualificazione e la gestione di corsi d'acqua e canali agricoli \cite{CIRF 2006} \cite{Regione Emilia Romagna 2012} \cite{Veneto Agricoltura 2011}.

\subsubsection{Creazione di ambienti umidi}
La presenza di terreni non sfruttati, lungo le aste fluviali principali di fondovalle, costituisce un'opportunità per il recupero delle estese aree umide scomparse in seguito alle opere di bonifica. La riqualificazione di queste aree, attraverso la creazione di canali laterali e stagni, permette di aumentare la diversità di habitat disponibili e potrà interessare aree caratterizzate dalla presenza di popolazioni di \emph{A. pallipes} per favorirne la diffusione. I canali laterali dovranno presentare andamento meandriforme, substrato naturale duro e diversificato, vegetazione acquatica e/o riparia secondo quanto già indicato nel paragrafo \ref{sub_riqhab}. Per quanto riguarda la realizzazione di stagni, in rapporto alle esigenze ecologiche di \emph{A. pallipes}, questi dovranno presentare: fondo naturale caratterizzato da sassi, ciottoli e ghiaia, in modo tale da offrire rifugi ai gamberi durante tutte le fasi di crescita. Nel caso di elevata deposizione di materiale limoso è necessario creare un fondo duro posizionando di più strati di massi in modo da favorire il deposito di limo negli interstizi al di sotto di questi. La presenza di vegetazione acquatica e riparia permette inoltre di offrire rifugio, ombreggiamento e materia organica creando un ambiente particolarmente favorevole per il gambero. 

\subsubsection{Interventi per l'incremento della connettività ecologica}
La presenza di briglie lungo le aste fluviali costituisce una barriera spesso insormontabili per gran parte delle specie acquatiche determinando la frammentazione e l'isolamento delle popolazioni. L'eliminazione di queste strutture di vecchia concezione e la loro sostituzione con elementi di ingegneria naturalistica quali grossi massi favorirebbe la continuità dell'habitat fluviale evitando l'isolamento delle popolazioni di gamberi e pesci e garantendo comunque dal punto di vista idrologico la stessa funzionalità esercitata dalle briglie. Dove l'eliminazione delle barriere non fosse possibile si può favorire la connettività attraverso la creazione di scale di risalita per i gamberi. Nell'ambito del progetto CRAINat \cite{AA.VV. 2014} si è scelto di optare per strutture di facile ed economica realizzazione, ossia assi in legno che, pur non assicurando una durata a lungo termine possono essere sostituite facilmente e a costi limitati in caso di deterioramento o asportazione in seguito a eventi di piena estremi. In alternativa è possibile progettare scale permanenti come previsto per i pesci.  E' però importante considerare che, sebbene la presenza di barriere comporti da un lato l'isolamento delle popolazioni autoctone, può rappresentare dall'altro un limite alla diffusione di specie alloctone; l'eliminazione di queste strutture deve quindi essere attentamente valutata in termini di rischi e benefici che potrebbero derivarne ai fini conservazionistici.

\subsubsection{Gestione della vegetazione acquatica e riparia}
La vegetazione acquatica e riparia svolge importanti funzioni per l'equilibrio dell'ecosistema richiedendo quindi dove possibile il ripristino e se necessario, una corretta gestione. Interventi periodici sulla vegetazione riparia dovrebbero essere volti all'eliminazione di specie alloctone come \emph{Ailanthus altissima}, \emph{Robinia pseudoacacia}, \emph{Amorpha fruticosa}, \emph{Buddleja davidii}, \emph{Impatiens spp}., \emph{Fallopia japonica} ecc. Per quanto riguarda i canali di fondovalle, gestiti in ambito agricolo e urbano, il taglio della vegetazione dovrebbe essere effettuato con metodi poco invasivi, che permettano di minimizzare la movimentazioni del fondo, e limitato a una sponda, a scacchiera o nella sola porzione centrale del canale evitandone la completa asportazione in modo tale da garantire la presenza di microhabitat diversificati e lo svolgimento dei naturali processi svolti dalla vegetazione. Il materiale tagliato deve essere rimossa entro le 12 ore successive, in modo da evitare fenomeni di eutrofia, e gli interventi dovrebbero essere limitati a periodi dell'anno compatibili con il compimento delle fasi più delicate del ciclo vitale della maggioranza degli organismi (ad esempio due volte l'anno, in inverno e tarda estate). Minori sforzi di gestione possono essere ottenuti attraverso la creazione di una fascia di vegetazione riparia arborea che, svolgendo funzione di ombreggiamento, limita la proliferazione di piante acquatiche all'interno del canale. Indicazioni dettagliate sui metodi di ripristino e di gestione della vegetazione riparia sono disponibili in: "Linee guida per la gestione della vegetazione lungo i corsi d'acqua in Trentino", realizzato nell'ambito dell'azione A7 del progetto Life+TEN \cite{Trentini 2014}, "linee guida per la riqualificazione ambientale dei canali di bonifica in Emilia Romagna" (Regione Emilia Romagna 2012) e "Manuale per la gestione ambientale dei corsi d'acqua a supporto dei Consorzi di bonifica" \cite{Veneto Agricoltura 2011}, \cite{Regione Emilia Romagna 2012}.

\subsubsection{Eliminazione o attenuazione delle fonti inquinanti}
L'abbattimento degli inquinanti da fonti diffuse si può ottenere attraverso il mantenimento e la corretta gestione della zona riparia che in condizioni naturali svolge l'importante funzione di filtro biologico tra l'ambiente terrestre e quello acquatico. La presenza di un ecotone ripario vegetato permette infatti di ridurre la velocità di riscellamento superifiale delle acque e di trattenere il carico solido (che porta con sé reflui zootecnici, fertilizzanti e fitofarmaci) assorbendo nutrienti e inquinanti. Questo fa sì che si inneschino i naturali processi di denitrificazione e di riduzione di fosfati ad opera di batteri e piante evitando così fenomeni di eutrofizzazione delle acque. È inoltre importante identificare ed eliminare eventuali fonti di inquinamento puntuali come scarichi  civili, per i quali sarebbe necessario l'allacciamento alla rete fognaria, industriali o zootecnici, che dovrebbero essere richiamati all'osservanza delle normative sul trattamento dei reflui. Date le difficoltà nell'esercitare forme di controllo sui comportamenti adottati dagli agricoltori sarebbe utile effettuare delle campagne di sensibilizzazione sugli impatti derivanti dall'immissione diretta di pesticidi e diserbanti nelle acque promuovendo comportamenti corretti come l'utilizzo delle apposite aree per la pulizia e il rifornimento idrico dei mezzi agricoli utilizzati per i trattamenti, il taglio della vegetazione da preferire al diserbo, e l'osservanza della distanza minima delle colture dai corpi idrici prevista per legge (pari a 4 m).

\subsection{Prevenzione e mitigazione dei fattori di minaccia}
\label{sub_prev}
Al fine di limitare l'estinzione delle popolazioni di \emph{A. pallipes} ancora presenti in Trentino e favorire l'espansione della specie sul territorio, è necessario individuare le minacce presenti (cfr. \ref{sub_identific_min}) e mettere in atto misure utili a contrastarle. 

\subsubsection{Contenimento della diffusione di specie aliene invasive} 
\label{subsub_conten}
L'individuazione ed il controllo delle popolazioni di decapodi alloctone sul territorio è necessaria al fine di evitarne ulteriori espansioni, che potrebbero determinare pressioni competitive nei confronti della specie nativa oltre alla diffusione di \emph{A. astaci}. L'eradicazione delle popolazioni alloctone risulta spesso impossibile per via dell'elevata densità di individui, a meno che non si tratti di colonizzazioni molto recenti. L'eradicazione si basa sulle seguenti tecniche \cite{Girardet 2012} \cite{AA.VV. 2014}, elencate in ordine decrescente di efficacia:
\begin{enumerate}[label={\arabic*)}]
  \item rimozione fisica: in stagni artificiali, mediante completo prosciugamento per otto mesi e allestimento di barriere alte 50cm intorno agli stessi; pulizia del fondo e trattamento con calce e cloro; riempimento con cemento degli interstizi tra le rocce e le sponde. Questo metodo garantisce la totale eradicazione, ma ha un alto impatto ambientale, causando la scomparsa di tutte le altre specie presenti \cite{Girardet 2012};
  \item rimozione meccanica tramite cattura intensiva degli esemplari utilizzando trappole armate con esche trofiche ad alta attrattività. Le trappole, nel maggior numero possibile, devono essere controllate ogni giorno cambiando l'esca e il trappolaggio intensivo deve essere ripetuto fino a che le catture si riducono di almeno il 60\% rispetto alle iniziali. Protocolli e indicazioni metodologiche sono state fornite dal progetto LIFE RARITY, dedicato all'eradicazione di \emph{P. clarkii} in Friuli Venezia Giulia" (RARITY). Questo approccio ha un basso impatto ambientale ma è efficace solo se ripetuto negli anni e solo in ambienti circoscritti. I costi in termini di uomini e mezzi sono quindi elevati;
  \item uso di biocidi, cioè sostanze chimiche o microorganismi in grado di uccidere le specie bersaglio. Tra queste, il piretro naturale (utilizzato in agricoltura biologica) si è dimostrato efficace \cite{Cecchinelli 2012}. Tuttavia, poichè i biocidi non sono specifici, possono colpire anche specie non bersaglio;
  \item uso di autocidi, quali rilascio di maschi sterili in grado di competere con quelli fertili nell'ambiente naturale. Questa tecnica, ampiamente utilizzata in agricoltura biologica, è altamente specifica e a basso impatto ambientale; l'efficienza per il gambero rosso della Louisiana è di oltre il 40\% \cite{Aquiloni 2009} anche se presenta costi elevati;
  \item lotta biologica: utilizzo di pesci predatori, es. anguille \cite{Aquiloni 2010b}. Questo metodo ha bassa specificità, costi elevati e può causare possibili squilibri ecosistemici.
\end{enumerate}

Nell'ambito del Progetto Life RARITY risultati ottimali, che hanno portato anche all'eradicazione definitiva di una popolazione di \emph{P. clarkii}, sono stati ottenuti attraverso la cattura e la rimozione dei gamberi oltre al rilascio di maschi sterili. La cattura è stata effettuata attraverso l'utilizzo di nasse a doppio inganno di 90 cm di lunghezza e 30 di diametro, costruite con maglia da 11 mm su telaio di filo armonico che permette la chiusura agevolandone il trasporto in grandi quantità. Le nasse vanno innescate con cibo per gatti in vaschette di alluminio o fegato fresco. Il trappolaggio intensivo può avere una durata variabile, da una settimana a un mese, nel periodo primaverile in rapporto alla dimensione della popolazione e permette di valutare l'abbondanza di gamberi prima dell'intervento, selezionare i maschi da sterilizzare (e successivamente rilasciare (maschi adulti di taglia maggiore o uguale a 40mm e con entrambe le chele) e ridurre il numero di riproduttori presenti nella popolazione attraverso l'eliminazione delle femmine e dei maschi di dimensioni inferiori ai 40 mm catturati. I maschi da sterilizzare sono stati trasportati, entro vaschette in plexiglass (40x40; 30 cm di altezza), al reparto di radioterapia dell'ospedale di riferimento per il trattamento con radiazione di 40 Gy per 15 minuti. I gamberi sono stati trattati direttamente all'interno delle vasche utilizzate per il trasporto, immersi in una quantità d'acqua sufficiente a coprire gli animali, chiuse con coperchio. L'effetto del trattamento risulta massimo dopo una settimana. I maschi rilasciati devono essere marcati in modo tale da renderli distinguibili da quelli non ancora sottoposti a trattamento. Maggiore è la densità della popolazione e maggiore sarà il numero di maschi sterili da rilasciare. L'intero procedimento deve essere seguito da personale tecnico-scientifico appositamente individuato. Maggiori informazioni sono disponibili nella pubblicazione finale del Progetto Life RARITY \cite{Zanetti 2014b}.

In Trentino si consiglia l'eradicazione urgente dell'unica popolazione di Procambarus clarkii finora presente, dal lago di Lagolo (bacino del Sarca), data la capacità di espansione di questa specie, la sua potenziale pericolosità come vettore dell'afanomicosi e i gravi impatti ambientali generati. Nei siti di nuova occupazione è infatti importante intervenire tempestivamente per catturare il maggior numero di individui possibile, nel tentativo di eradicare la popolazione prima che si adatti e inizi il processo di colonizzazione \cite{Zanetti 2014}. Qualora l'eradicazione di questa popolazione non risulti efficace è necessario tenerla sotto controllo programmando monitoraggi periodici per verificare la densità ed evitarne l'ulteriore diffusione.
Sarebbe inoltre auspicabile la messa a punto per il Trentino di Protocolli di risposta rapida (EDRR- Early Detection Rapid Response), così come già implementati per il progetto RARITY dall'Università di Firenze \cite{Zanetti 2014c}. I protocolli di risposta rapida (Early Detection and Rapid Response, EDRR) sono, in linea con quanto previsto nell'European Strategy on Invasive Alien Species \cite{Genovesi 2004}, uno degli strumenti più efficaci per la lotta alle specie invasive, in quanto consentono di intervenire in modo tempestivo su nuove popolazioni che si stanno stabilizzando in un territorio. Per il corretto funzionamento del protocollo, va predisposto un appropriato sito web e/o materiale informativo cartaceo \cite{RARITY} in cui sia elencato, nel modo più dettagliato possibile, l'iter da seguire per fare una segnalazione (cfr. \ref{subsub_segn}). E' importante che siano individuati i referenti degli Enti responsabili per le singole attività previste dal protocollo \cite{Zanetti 2014b} che devono attivare le opportune procedure di pronta riposta: 1) verifica della segnalazione; 2) monitoraggio della popolazione; 3) attività di eradicazione/controllo.

Nei casi in cui le popolazioni di specie aliene non siano eradicabili, come è il caso delle popolazioni di \emph{O. limosus} in Trentino, che ha raggiunto densità troppo elevate nei laghi in cui si è insediato per poter efficacemente trappolare e rimuovere tutti gli individui, gli unici strumenti per contrastarne la diffusione sono il monitoraggio periodico delle popolazioni, l'organizzazione di campagne educative e la sorveglianza sull'osservanza delle norme in materia. Il mantenimento di alcune barriere presenti lungo le aste fluviali (quali traverse, briglie, stramazzi) potrebbe essere considerato in alcuni casi utile ad evitare la diffusione delle specie alloctone lungo l'intero bacino. 

Nel caso di \emph{O. limosus}, la rimozione del canale artificiale che collega il lago di Canzolino (che ospita un'abbondante popolazione di \emph{O. limosus} infestata da \emph{Aphanomyces astaci}) dalla \texttt{ZSC IT3120041} Lago Costa in cui A. pallipes si è estinto in tempi recenti, permetterebbe l'isolamento del lago Costa e, il contenimento dell'espansione della peste del gambero. Una più intensa sorveglianza dovrà inoltre essere attuata in bacini idrografici che si estendono parzialmente in territori extraprovinciali e per i quali sia stata segnalata la presenza di specie alloctone invasive. E' ad esempio il caso del bacino del Fiume Chiese per il quale è stata segnalata l'espansione di \emph{P. clarkii} anche nel corso d'acqua principale nel tratto lombardo. 

\subsubsection{Contenimento di patogeni} 
La prevenzione e il controllo di infestazioni da parassiti letali per \emph{A. pallipes} quali \emph{A. astaci} e \emph{Thelohania sp}. può essere effettuata attraverso periodiche analisi istologiche e biomolecolari da effettuare su individui di popolazioni autoctone e alloctone. I corpi idrici caratterizzati dalla presenza di popolazioni infestate dovrebbero essere sottoposti a particolare regolamentazione, soprattutto per quanto riguarda la pesca, in modo tale da evitare la diffusione dei patogeni da un bacino ad un altro. Infatti la trasmissione di molti patogeni, tra cui l'afanomicosi, avviene per contatto diretto tra gambero alieno malato o portatore e gambero autoctono, ma molto più frequentemente per contatto indiretto, ovvero per trasporto di spore da parte dell'uomo con l'attrezzatura da pesca. L'organizzazione di campagne informative sui metodi di disinfezione delle attrezzature e di sensibilizzazione, rivolte soprattutto ai pescatori, potrebbero aiutare a contrastare il problema. É quindi importante organizzare incontri con le associazioni di pescatori dei vari bacini idrografici trentini per una corretta informazione sull'argomento, e per la formazione di personale appartenente a ciascuna associazione incaricato di diffondere l'informazione tra i soci. L'argomento dovrebbe essere trattato nell'ambito del corso obbligatorio previsto per il rilascio della licenza da pesca. Inoltre, materiale divulgativo che spieghi in modo semplice le strategie efficaci di disinfezione del materiale da pesca dovrebbe essere obbligatoriamente fornito a ogni utente che richieda un permesso di pesca. Sarebbe inoltre utile diffondere informazioni sui focolai di A.astaci rilevati attraverso materiale costantemente aggiornato e pubblicato sul sito web del progetto gambero (cfr. \ref{subsub_web}), oltre che sui siti web e le riviste delle associazioni di pescatori e/o attraverso la segnalazione in prossimità del corpo idrico attraverso pannelli informativi.

\subsubsection{Prevenzione di traslocazioni e prelievi illegali}
\label{subsub_prev_trasl}
Le misure per contrastare il bracconaggio comprendono: 1) una legislazione provinciale che sancisce divieti per tali fenomeni e il coinvolgimento di organi di gestione e sorveglianza per un controllo efficace ed effettivo sul territorio che sanzioni i trasgressori; 2) campagne di divulgazione e sensibilizzazione dell'opinione pubblica, con il coinvolgimento di organi di gestione e sorveglianza come pure di organizzazioni scientifiche e associazioni naturalistiche. In Trentino, alcune attività divulgative sono state svolte nell'ambito del progetto LIFE+TEN, ma sarebbe opportuno sviluppare una campagna divulgativa a larga scala, coinvolgendo scuole, organizzazioni scientifiche, associazioni naturalistiche e consorzi di pesca. La mancanza di comunicazione nei confronti della popolazione può comportare il ripetersi di comportamenti dannosi attuati spesso inconsapevolmente con conseguente fallimento degli sforzi messi in atto per la conservazione della specie.

\subsection{Incremento delle popolazioni per gestione diretta}
Tra gli interventi di conservazione in situ, è rilevante la gestione diretta di una popolazione naturale a rischio, che per \emph{A. pallipes} si traduce in:
\begin{itemize}
  \item controllo ed eradicazione di specie alloctone (cfr. \ref{subsub_conten});
  \item limitazione del prelievo (cfr. \ref{subsub_prev_trasl});
  \item reintroduzioni e ripopolamenti (cfr. \ref{subsub_ripop_int}); 
  \item realizzazione di siti rifugio a funzione di \emph{source area} (cfr. \ref{subsub_ripop_int}).
\end{itemize}

\subsubsection{Interventi di ripopolamento o reintroduzione/rinforzo}
\label{subsub_ripop_int}
Tra le strategie di protezione in situ dei gamberi, sono rilevanti gli interventi di "semina", ovvero l'immissione di nuovi individui in un determinato ambiente in cui la specie sia o sia stata naturalmente presente, a partire da materiale reperito in natura o prodotto in centri di allevamento e riproduzione appositi (cfr. \ref{sub_allevamento}). Gli interventi di semina vanno distinti in ripopolamenti e reintroduzioni. Nel primo caso la specie è ancora presente nell'area di rilascio ma a basse densità. Mediante il ripopolamento, vengono quindi aggiunti nuovi individui per portare la popolazione a densità adeguate alla capacità portante dell'ambiente, che favoriscano quindi l'autosostentamento della popolazione stessa. Nel caso delle reintroduzioni, invece, è possibile ristabilire una popolazione in una parte dell'areale in cui la specie era presente in tempi storici o recenti ma in cui è attualmente estinta a causa di diversi fattori come inquinamento, patogeni, competizione con specie aliene ecc. La finalità delle reintroduzioni è quella di stabilire una popolazione isolata "sicura" di A. pallipes, in una parte dell'areale in cui il futuro della specie non sia minacciato; deve essere quindi accertata la cessazione della pressione che ha causato la precedente estinzione. Una ulteriore e differente attività di semina, qualificabile come strategia di protezione \emph{ex situ}, consiste nella traslocazione di una popolazione dal sito di origine (\emph{source}) ad un sito diverso (\emph{sink}). Tale metodo, tuttavia, ha lo svantaggio di non cambiare sostanzialmente l'abbondanza totale di gamberi, dato che se il numero aumenta da una parte (sink area), diminuisce da un'altra (\emph{source}), ma è invece auspicabile in situazioni in cui una popolazione si trova in una situazione di rischio imminente determinato dalla competizione con specie aliene o dalla necessità di esecuzione di lavori di sistemazione dell'alveo \cite{Kemp 2003}, e viene quindi traslocata a un sito più sicuro.

Sia le attività di ripopolamento, che prevedono il prelievo di riproduttori, il loro allevamento e il rilascio dei giovani gamberi nella stessa area di origine, che le attività di ripopolamento, con materiale originario da ambienti differenti da quello di destinazione previsto per la reintroduzione, richiedono un'analisi accurata basata su uno studio di fattibilità \cite{Souty-Grosset 2009} volto a valutare:
\begin{itemize}
  \item idoneità degli habitat (solo per attività di reintroduzione). I siti prescelti devono essere all'interno dell'areale naturale della specie, in cui non siano attualmente presenti popolazioni di \emph{A. pallipes} e una ricolonizzazione risulta improbabile. La priorità va ai corsi d'acqua che ospitavano popolazioni autoctone di gambero fino a tempi recenti; non devono essere presenti specie aliene invasive possibilmente entro 50 km o almeno 20 km di acque adiacenti \cite{Peay 2003} \cite{Souty-Grosset 2009};
  \item rispetto dei requisiti ecologici per l'habitat e la qualità dell'acqua (direttiva quadro UE 2000/60 / CE) tra cui: basso rischio di inquinamento, presenza degli habitat utili ai gamberi nei vari stadi di sviluppo (fessure nel basamento roccioso, massi, ciottoli, ghiaia per gli adulti e piccole radici, briofite, lettiera, sabbia per i piccoli) e per il conseguente raggiungimento dell'autosostentamento della popolazione, e l'equilibrata presenza di predatori (es. pesci). Se necessario, gli habitat possono essere migliorati con tecniche di ingegneria naturalistica (cfr. \ref{sub_riqhab});
  \item idoneità della popolazione donatrice. Per le popolazioni trentine, gli studi genetici svolti (cfr. \ref{sec_gene}) escludono molto chiaramente la possibilità di usare altre popolazioni dell'arco alpino o appenniniche come sorgente per attività di ripopolamento in Trentino. Le popolazioni donatrici devono provenire preferibilmente dallo stesso bacino ed essere selezionate in base alla loro similarità genetica e all'abbondanza. Il prelievo non deve superare il 10\% della popolazione. Le popolazioni donatrici devono infine essere sottoposte a screening sanitario per escludere che siano affette o portatrici di afanomicosi; individui affetti con \emph{Thelohania} devono rappresentare meno del 10\% della popolazione, perché questa possa essere ritenuta idonea come popolazione donatrice;
  \item presenza di potenziali minacce. Nel caso in cui permangano i fattori di minaccia che hanno causato la scomparsa o la riduzione della popolazione nel sito di reintroduzione/ripopolamento, devono essere messe in atto misure per la loro eliminazione o mitigazione. La contaminazione da spore di \emph{Aphanomyces astaci} \cite{Schulz 2002} deve essere accertata nei siti in cui si vuole attuare una semina. Secondo Spink e Frayling \cite{Spink 2000} il test più efficace è l'esposizione in situ di gamberi nativi in una gabbia nel periodo estivo, quando il potenziale di infezione è alto a causa della muta del gambero. Il periodo di esposizione in situ dovrebbe durare da quattro a sei settimane, tuttavia la cattività può portare ad elevata mortalità non collegata alla peste. Tale metodo si raccomanda quindi solo nel caso in cui siano stati accertati in passato eventi di peste o se ne sospetti la presenza nel sito di reintroduzione.
  \item fattori di rischio: tutte le fasi del progetto devono essere condotte senza rischi per la sicurezza e la salute umana;
  \item aspetti finanziari e socio-economici: le risorse economiche devono essere disponibili per tutte le fasi del progetto, incluso il controllo ex-post;
  \item aspetti legislativi: il trasferimento diretto di gamberi da un corpo idrico ad un altro di solito richiede permessi specifici.
\end{itemize}
Le reintroduzioni devono essere attuate osservando regole precise che prevedono: 
\begin{itemize}
  \item scelta degli stock: i rilasci andrebbero effettuati in varie fasi, da agosto a ottobre in un periodo di tre anni, utilizzando individui adulti, o se possibile una popolazione bilanciata con tutte le classi di età. Gli adulti si disperdono velocemente e quindi permettono una ricolonizzaione più veloce e con minore tasso di mortalità dei giovani, ma hanno lo svantaggio di un'attesa di vita minore. I giovani, d'altro canto, hanno una più elevata adattabilità ad ambienti nuovi, un tasso di migrazione minore e sono più facilmente ottenibili in allevamento rispetto agli adulti. Tuttavia, essi sono facilmente predati dai pesci, e una popolazione reintrodotta costituta da giovani richiede un periodo più lungo per raggiungere l'età riproduttiva. Le raccomandazioni \cite{Souty-Grosset 2009} sulle popolazioni da utilizzare per le reintroduzioni sono le seguenti: 1) utilizzare rapporto sessi M:F ottimale (da 1:1 a 1:3) per gli adulti; 2) utilizzare varie classi di taglia (ovvero di età); 3) utilizzare un minimo di 50-100 individui; 4) utilizzare donatori caratterizzati geneticamente;
  \item trasporto: il tempo di traslocazione dovrebbe essere ridotto al minimo, quindi possibilmente pianificando le reimmissioni nello stesso bacino in cui sono presenti le popolazioni donatrici (naturali o allevate); le reintroduzioni dovrebbero essere concluse entro massimo 18 ore dal prelievo. L'aggressione tra individui durante il trasporto va minimizzata fornendo rifugi (muschi, vegetazione acquatica, ecc) e riducendo lo stress termico (mantenere la temperatura dell'acqua costante durante il trasporto e uguale a quella dei punti di prelievo) e chimico (utilizzare acque del luogo di origine), controllando con aeratori l'ossigeno disciolto in acqua durante il trasporto dei gamberi, che a volte può richiedere tempi abbastanza lunghi;
  \item rilascio: prima del rilascio, i gamberi vanno acclimatati da 1 a 4 giorni in gabbie di rete che contengano rifugi adeguati e il rilascio deve essere effettuato in aree con condizioni fisico-chimiche favorevoli e con disponibilità di rifugi (naturali, se possibile, o creati artificialmente ad hoc prima del rilascio). Tutto il materiale e gli stivali degli operatori vanno preventivamente disinfettati per evitare la diffusione della peste del gambero. Infine, va tenuto un registro accurato delle attività di reintroduzione per avere un feedback sia in caso di reintroduzioni postive che negative;
\end{itemize}

Per indicazioni di dettaglio circa la corretta progettazione ed esecuzione di interventi di ripopolamento o reintroduzione si veda IUCN \cite{IUCN 2013b}, AA.VV. \cite{AA.VV. 2007}, Kemp \etal \cite{Kemp 2003} e Souty Grosset e Reynolds \cite{Souty-Grosset 2009}.

La messa in opera di impianti di allevamento presenta costi di progettazione, realizzazione, manutenzione e di impiego di personale molto differenti a seconda delle seguenti due tipologie: 
\begin{enumerate}[label={\arabic*)}]
  \item intensivo in vasche (incubatoi) all'interno di una struttura: richiede costi elevati di costruzione e manutenzione, e personale dedicato in modo continuativo. I costi possono essere ridotti utilizzando, invece di impianti costruiti \emph{ex-novo}, vasche in stabilimenti ittiogenici già esistenti o vasche ed impianti dismessi. In quest'ultimo caso, si ha l'ulteriore vantaggio di eliminare il rischio di trasmissione di patogeni tra gamberi e fauna ittica in allevamento.
  \item estensivo in vasche seminaturali installate all'aperto: richiede costi ridotti di costruzione, di manutenzione e di personale. Tali tipologie di impianti di allevamento hanno il vantaggio di poter essere mantenuti anche al termine di un eventuale finanziamento dedicato, dati i costi nettamente inferiori rispetto all'allevamento intensivo. È anche possibile disattivare le vasche nei periodi in cui i fondi non sono disponibili, e riattivarle in seguito, senza incorrere in spese eccessive di ripristino, data la semplicità degli impianti stessi.  
\end{enumerate}

Per ridurre i costi è pertanto suggerito, per la provincia di Trento, di seguire le procedure di allevamento seminaturali attuate e testate nel progetto CRAINat, descritte nel "Action Plan per la conservazione di \emph{Austropotamobius pallipes} in Italia" \cite{AA.VV. 2014} ed esposte in occasione del Seminario "Il gambero di fiume autoctono (\emph{Austropotamobius pallipes})" tenutosi il 7 aprile 2017 presso il Centro Visitatori del Parco Alto Garda di Prabione di Tignale (BS), nell'ambito del progetto LIFE IP GESTIRE 2020. In occasione del seminario è stata organizzata la visita al Centro di riproduzione ed allevamento del gambero gestito dall'ERSAF (Ente Regionale per i Servizi all'Agricoltura e alle Foreste) Lombardia (figura \ref{fig_34}), al quale hanno assistito operatori MUSE (Sezione di Zoologia dei Vertebrati), FEM (Unità di idrobiologia) e PAT (Servizio Sviluppo Sostenibile e Aree Protette). Sulla base delle indicazioni raccolte durante l'esperienza si consiglia pertanto di identificare anche per il Trentino siti idonei alla realizzazione di allevamenti all'aperto, costituiti da vasche, ricavate attraverso lo scavo di depressioni nel terreno, rivestite da teli di plastica impermeabile e circondate da barriere, per evitare la fuga degli animali e impedire la deposizione di ovature da parte degli anfibi, e coperte da reti per evitare la deposizione delle libellule e la predazione da parte di uccelli. Tali vasche devono essere dotate di rifugi e possono essere alimentate da acque di sorgente, falda o da corsi d'acqua adiacenti. Inoltre, se fossero già disponibili edifici con spazi inutilizzati nelle vicinanze è possibile predisporre piccole vasche per la stabulazione degli adulti in casi di emergenza o per motivi di studio. L'impegno nella gestione di questo tipo di strutture è minima e consiste nel controllo periodico (una o due volte a settimana) per la verifica del corretto funzionamento della circolazione dell'acqua, per l'eliminazione di eventuali individui morti e, se necessario, per l'alimentazione dei gamberi. Al fine di ridurre il rischio di infestazione da \emph{A. astaci}, è preferibile alimentare i centri di ripopolamento con acqua prelevata da falda, invece che da adiacenti corpi idrici. Inoltre, nel caso di prelievo di riproduttori da corpi idrici diversi destinati a un unico allevamento, è necessaria una preventiva quarantena degli animali una volta giunti nel centro di allevamento, mantenendo le popolazioni separate, almeno un paio di settimane, periodo sufficiente a far sì che i possibili agenti eziologici manifestino la loro virulenza.

In Trentino, strutture di questo tipo potrebbero essere costruite e gestite in collaborazione con le associazioni di pescatori prevedendo anche vasche di stabulazione per le emergenze all'interno dei rispettivi incubatoi. Sarebbe utile individuare diverse strutture distinte per bacino idrografico, previa identificazione di opportune popolazioni sorgente (si veda capitolo successivo), in modo da creare allevamenti nel rispetto dell'identità genetica delle popolazioni di ciascun bacino. Per il bacino del Brenta, si suggerisce il coinvolgimento dell'Associazione Pescatori di Grigno (cha ha già manifestato interesse nelle fasi sperimentali di cattura, allevamento e rilascio condotte dalla Fondazione Mach nel corso del Progetto Life+TEN) per l'allevamento di gamberi provenienti dal Rio Laguna e dallo stagno di Grigno, da utilizzare per ripopolamenti nella \texttt{ZSC IT3120030 Fontanazzo} e nel Rio Resenzuola. Per l'Avisio, sarebbe auspicabile creare un centro di allevamento in val di Fiemme, dove le popolazioni sono tutte estinte ad eccezione di una popolazione segnalata in un fosso privato a Cavalese e attualmente a rischio per via degli interventi di manutenzione periodici effettuati in alveo. Tale popolazione, se correttamente allevata, potrebbe rappresentare la sorgente per ripopolamenti in vari corsi d'acqua minori dell'alto corso dell'Avisio. Nel bacino del Sarca, potrebbe essere coinvolta la Rete di Riserve, che ha già manifestato interesse nell'attuazione di interventi a favore del gambero di fiume, per la creazione di stagni multifunzionali o centri di ripopolamento all'interno del Parco fluviale del Sarca. Altre località/popolazioni sorgente potranno essere identificate per gli altri bacini. 


\begin{figure}
  \centering
  \includegraphics[width=.8\columnwidth]{fig_34.jpg}
  \caption{Centro di riproduzione ed allevamento del gambero di fiume autoctono \emph{Austropotamobius pallipes} di Prabione di Tignale (BS). Dall'alto in basso in senso orario: vasche per allevamento e riproduzione all'aperto; vasche per allevamento giovani in interno e stabulazione di emergenza di adulti, areatori e serbatoio per la riserva di acqua da utilizzare in caso di malfunzionamento del sistema di circolazione}
  \label{fig_34}
\end{figure}

\subsubsection{Realizzazione di siti rifugio: il ruolo delle aree sorgente}
La realizzazione di source areas è una tipologia di intervento innovativa e sperimentale, che riproduce, mediante tecniche di ingegneria naturalistica, quanto già presente in natura in alcuni contesti territoriali. Nel caso di \emph{A. pallipes} le aree sorgente si configurano come canalizzazioni sinuose parallele in tratti di corsi d'acqua naturali e in continuità con questi, realizzate in modo tale da rallentare il flusso d'acqua e consentire la formazione di ambienti ottimali per la riproduzione in situ dei gamberi \cite{AA.VV. 2014}. Tali aree rappresentano dei siti-rifugio con una minore pressione predatoria e con caratteristiche ambientali ottimali al gambero, soprattutto nei primi mesi di vita (lento deflusso delle acque, limitata presenza di predatori, abbondanza di rifugi, ecc). Infatti, dopo la schiusa delle uova, le aree sorgente rappresentano l'area di sviluppo per i giovani, permettendone successivamente la migrazione, naturale ed autonoma (perché in continuità ambientale), nei corsi idrici principali. Gli stagni multifunzionali sono strutture naturaliformi che rispondono a più obiettivi particolari, volti tutti alla tutela e conservazione della specie, e dove vengono ospitate in modo più o meno confinato piccole popolazioni di gambero. Le loro caratteristiche richiamano le pozze laterali a flusso lento che si trovano nei torrenti pedemontani, dove l'apporto di ossigeno ed il ricambio di acqua sono assicurati in modo continuo grazie alla vicinanza al flusso principale. Questi stagni vengono realizzati in contesti strategici, ed in prossimità di una fonte d'acqua che li possa alimentare durante tutto l'anno, e si prestano a finalità di conservazione: vengono immessi gamberi in età riproduttiva ed il novellame qui nato può essere prelevato per piccoli interventi di semina. Essi possono inoltre essere utilizzati per attività di educazione ambientale e di sensibilizzazione essendo agevolmente accessibili e se dotati di rifugi artificiali facilmente ispezionabili consentono di osservare i gamberi nelle varie fasi del ciclo biologico. 
In Trentino lo stagno di Grigno ospita una delle popolazioni più abbondanti e strutturate di \emph{A. pallipes} in quanto alimentato dal Rio Laguna (in parte captato) che è di ottima qualità; nello stagno sono presenti rifugi naturali (pietre, muschi, canneto) che permettono la riproduzione della specie. Lo stagno, costruito per finalità ricreative, è facilmente accessibile al pubblico, essendo situato in un parco pubblico adiacente la stazione ferroviaria. Questo stagno merita particolare attenzione e con una gestione accurata potrebbe rappresentare il primo stagno multifunzionale trentino.

\subsection{Studio di fattibilità degli interventi di conservazione}
\label{sub_studio}
Tutti gli interventi proposti per la conservazione delle popolazioni devono essere innanzitutto sottoposti a valutazione per verificare la fattibilità dal punto di vista organizzativo, economico, biologico e sociale.
Nel caso delle reintroduzioni sarà necessario:
\begin{itemize}
  \item individuare gli habitat ottimali per la specie che ricadano all'interno dell'areale storico di distribuzione e che abbiano ospitato popolazioni di gambero in tempi recenti; 
  \item assicurarsi che le cause dell'estinzione (alterazione dell'habitat, captazione idrica, inquinamento e presenza di eventuali specie alloctone, ecc.) siano cessate; 
  \item programmare l'immissione di una popolazione costituita da un numero di individui e con una struttura per sesso ed età adeguati.
\end{itemize}

A tale scopo, l'utilizzo di modelli matematici è essenziale alla programmazione. I modelli di sviluppo demografico delle popolazioni sono estremamente utili per ottimizzare le reintroduzioni sia in termini qualitativi (rapporto sessi e classi di età degli individui adulti traslocati), sia quantitativi (numero d'individui). I modelli per la simulazione della disponibilità di habitat (cfr. \ref{subsub_modellazione}) a seguito dell'applicazione di determinati interventi di riqualificazione e reintroduzione costituiscono un utile strumento di programmazione. 
Per la riuscita degli interventi di conservazione delle popolazioni è necessario che siano soddisfatte due condizioni:
\begin{itemize}
  \item una adeguata disponibilità economica per tutte le fasi dell'operazione;
  \item il consenso della popolazione locale, o meglio ancora il suo coinvolgimento e la sua cooperazione.
\end{itemize}

Per la determinazione dei costi, dopo aver verificato la fattibilità tecnico-scientifica dell'intervento occorre definire l'organigramma del personale da coinvolgere nel progetto. I costi di riqualificazione ambientale vanno accuratamente stimati da personale addetto, al fine di definire una scala di priorità d'intervento in relazione alle esigenze della specie, ai costi previsti e alle risorse (economiche, personale addetto) disponibili. Poiché spesso i progetti sono a lungo termine, è richiesto un impegno finanziario e un appoggio tecnico/amministrativo/politico per lunghi periodi. Si deve considerare l'impatto visivo ed emotivo della specie reintrodotta, soprattutto se tra le cause di estinzione vi è il fattore umano (es. prelievo). Andrebbero inoltre organizzate campagne di sensibilizzazione per il coinvolgimento della popolazione nelle fasi di allevamento e di reintroduzione (attraverso manifestazioni che coinvolgano, ad esempio, le scolaresche con progetti di studio a vario livello di approfondimento, legate alle tematiche della conservazione di specie, della riqualificazione degli habitat e della gestione ecocompatibile di specie sensibili).
La valutazione della fattibilità degli interventi deve anche verificare:
\begin{enumerate}[label={\arabic*)}]
  \item la compatibilità con altri progetti di conservazione che interessano l'area, e gli eventuali vincoli legali/amministrativi/gestionali, al fine di evitare interazioni negative con altri pprogetti già avviati;
  \item i potenziali effetti della reintroduzione e/o riqualificazione su attività antropiche locali di interesse economico quali uso agricolo e zootecnico, e programmare eventuali risarcimenti in caso di danni ad attivita antropiche.
\end{enumerate}

\subsubsection{Modellazione dell'idoneità degli habitat per \emph{A. pallipes} 
(A cura di Paolo Vezza. Dipartimento di Ingegneria dell'Ambiente, del Territorio e delle Infrastrutture, Politecnico di Torino)} 
\label{subsub_modellazione}
Generalmente la valutazione dello stato di conservazione dell'habitat per una data specie viene definito mettendo in relazione lo stato delle popolazioni e le caratteristiche dell'ambiente in cui vivono. Recentemente per i torrenti alpini è stata applicata e validata la modellazione a meso-scala (MesoHABSIM - Mesohabitat Simulation Model) \cite{Parasiewicz 2007} \cite{Vezza 2014} che permette di simulare la variazione dell'habitat disponibile per varie componenti target dell'ecosistema acquatico in funzione della portata. Nei corsi d'acqua naturali, il mesohabitat corrisponde generalmente per estensione all'unità idro-morfologiche, cioè un'area sommersa o emersa (ad es., barre, riffle, piana inondabile) creata da processi di deposizione e/o erosione e situata all'interno dell'alveo o al suo esterno (nella pianura alluvionale). La tecnica consiste nella definizione di un tratto rappresentativo di analisi e, all'interno di esso, nel riconoscimento di meso-habitat o unità morfologiche \cite{Rinaldi 2016}, ossia aree caratterizzate da particolari condizioni idrodinamiche e di geometria dell'alveo, che, unitamente ad altri fattori (quali la presenza di rifugi o il tipo di substrato presente) possono determinare e permettere, o meno, la presenza e lo sviluppo della specie oggetto di studio. 
Considerato che l'habitat fluviale varia con la portata defluente, per descrivere la variazione spazio-temporale delle caratteristiche idro-morfologiche e del mosaico di mesohabitat al variare di questa, devono essere considerati almeno  tre-cinque rilievi del tratto considerato in differenti condizioni di deflusso \cite{Vezza 2014}. Durante i rilievi sul campo, ogni unità morfologica identificata visualmente viene mappata e georefenziata tramite l'utilizzo di un telemetro laser collegato ad un dispositivo palmare con applicazione GIS. Successivamente, per ogni unità idromorfologica vengono rilevate le caratteristiche biotiche ed abiotiche dell'ecosistema che influenzano la distribuzione della specie, ad es. potenziali rifugi per i gamberi (come presenza di massi, rive incavate, vegetazione sommersa, detriti e lettiera), unitamente alla descrizione della distribuzione di profondità, velocità di corrente e substrato presente. 

Il campionamento quantitativo della popolazione nel tratto interessato può essere eseguito al fine di descrivere la relazione che lega l'habitat fluviale e la distribuzione della specie studiata, ovvero per tarare un modello di idoneità di habitat in funzione delle condizioni idro-morfologiche locali. Nel caso del gambero autoctono, è preferibile un campionamento notturno, nel periodo di massima attività (luglio-agosto). Il campionamento quantitativo viene condotto a scala di mesohabitat, ossia in ogni unità morfologica identificata vengono catturati tutti i gamberi presenti. Si rilevano il sesso, l'eventuale presenza di mutilazioni, le misure morfometriche principali (lunghezza totale, lunghezza cefalotoracica) e il peso. Alla fine del campionamento, i gamberi vengono poi rilasciati nel medesimo mesohabitat in cui sono stati catturati. Attraverso tecniche statistiche ad apprendimento automatico (ad es. Random Forest \cite{Breiman 2001}) vengono quindi definiti modelli di idoneità d'habitat che stabiliscono la soglia di probabilità oltre la quale la specie può essere presente o abbondante in funzione delle condizioni ambientali del corso d'acqua. 

Questo approccio permette così di analizzare la presenza e la densità di una specie a livello di mesohabitat, le cui caratteristiche variano al variare del deflusso di un corso d'acqua, e di definire i tratti fluviali dove sono presenti habitat ottimali, o habitat sfavorevoli per la specie target. É possibile inoltre effettuare delle simulazioni pre-opera per valutare l'efficacia di eventuali interventi di riqualificazione di habitat o l'idoneità ambientale per la reintroduzione di popolazioni di \emph{A. pallipes}. Una descrizione completa della metodologia e delle sue applicazioni sul territorio Italiano è possibile trovarla nel Manuale tecnico-operativo per la modellazione e la valutazione dell'integrità dell'habitat fluviale dell'ISPRA \cite{Vezza 2017}.

\section{Buone pratiche gestionali}
Le azioni indicate in questa sezione rientrano in gran parte nelle attività svolte dagli enti e dalle associazioni provinciali, quali ad esempio: la sorveglianza, il rilascio di autorizzazioni, la gestione del territorio (APPA, Guardiapesca, Corpo Forestale, Servizio Bacini Montani, Servizio Agricoltura, Consorzi di bonifica) o che rientrano in progetti già in corso come la riqualificazione di habitat acquatici, la gestione sostenibile della vegetazione riparia, l'eradicazione di specie aliene, etc.

\subsection{Misure di conservazione di popolazioni e di habitat nelle aree protette}
La presenza di popolazioni di \emph{A. pallipes} all'interno di aree protette costituisce un'importante occasione per l'attuazione di una tutela rigorosa della specie e dei suoi habitat non applicabile altrove. Al di fuori delle aree protette è infatti sempre necessario trovare un compromesso tra le esigenze di conservazione ambientale e quelle richieste per il normale svolgimento delle attività antropiche esponendo le popolazioni di gambero ad un certo grado di rischio. In ambienti tutelati dovrebbero quindi essere vietate o strettamente regolamentate tutte le attività che possono comportare un rischio per la conservazione delle popolazioni autoctone di gambero come le attività di pesca (che possono comportare la diffusione di \emph{A. astaci}) e attività connesse (come le semine di salmonidi che comportano un eccessiva pressione predatoria nei confronti dei gamberi), ma anche gli interventi in alveo e le captazioni idriche. Va infatti rimarcato come, in mancanza di adeguate misure, dal 2002 ad oggi si siano estinte 14 popolazioni presenti in Siti della Rete Natura 2000 (tabella \ref{tab_4}).

\subsection{Prevenzione di fenomeni di degrado e di perdita di habitat}
Le attività antropiche che richiedono lo sfruttamento della risorsa idrica come la produzione di energia idroelettrica, l'agricoltura, la pescicoltura e la pesca in laghetti privati possono comportare impatti anche gravi sugli habitat e sulle comunità acquatiche. Per le attività idroelettriche e agricole, il prelievo di acqua è regolamentato e permesso nell'osservanza del deflusso minimo vitale. Tale limite però non è sempre osservato e le comunità acquatiche sono sottoposte a forti pressioni che possono determinarne anche la scomparsa nel periodo secco estivo. La popolazione di gambero rilevata nel Rio Nero (Bacino del Fersina; tabella \ref{tab_4}) in declino, è fortemente influenzata dall'eccessivo sfruttamento della risorsa idrica. La sorveglianza sul rispetto del deflusso minimo vitale sarebbe necessaria per la conservazione di popolazioni di \emph{A. pallipes}. Per quanto riguarda i laghetti da pesca e le pescicolture, i metodi comunemente utilizzati per la manutenzione e la gestione degli impianti, come il completo svuotamento dei laghetti da pesca per le operazioni di pulizia (osservato ad esempio nei laghetti di Vezzano e di Terlago; tabella \ref{tab_4}) possono causare il declino o la scomparsa delle popolazioni di gambero naturalmente presenti all'interno degli impianti e nei loro emissari. La conservazione delle popolazioni di gambero in questi ambienti può essere garantita solo instaurando un rapporto di collaborazione con i proprietari degli impianti con i quali sarebbe opportuno elaborare dei metodi di gestione più sostenibili, sensibilizzandoli e coinvolgendoli nel progetto di conservazione della specie. 
Le operazioni di svaso e le attività di manutenzione in alveo, che comportano la movimentazione di sedimento causano gravi impatti sulle comunità acquatiche per via di danneggiamenti diretti sugli organismi, il soffocamento provocato dall'eccessivo intorbidimento delle acque e la modificazione degli habitat. Ad esempio, l'asportazione di materiale di deposizione da un fosso agricolo in Val di Fiemme ha causato gravi danni all'unica popolazione ancora presente in zona. Si sospetta inoltre che la scomparsa di popolazioni da altri siti (\texttt{ZPS-ZSC IT3120038} Inghiaie, Roggia di Gardolo e Rio Carpine a Gardolo di Mezzo, tabella \ref{tab_4}) possa essere stata causata da ripetute opere di questo tipo. Qualsiasi intervento in alveo che comporti movimentazione del fondo, elevato grado di intorbidimento delle acque, la deposizione di abbondante materiale limoso dovrebbe quindi essere sottoposto a un giudizio di fattibilità da parte degli organi competenti e autorizzato solo in seguito ad accertamenti sull'assenza di specie sensibili, che potrebbero subirne gravi conseguenze, o all'elaborazione di strategie utili a ridurre il rischio per le popolazioni presenti. In presenza di \emph{A. pallipes} può essere considerata la possibilità di stabulare temporaneamente i gamberi in cattività fino a fine lavori e in seguito al ripristino delle condizioni idonee dell'habitat. Nel caso di interventi in alveo ripetuti periodicamente, come l'asportazione di sedimento dai fossi agricoli, sarebbe necessario elaborare strategie per la loro riduzione o eliminazione attraverso, ad esempio, la creazione di trappole per sedimenti o il conferimento al fosso di una struttura più naturale che permetta di aumentare la velocità della corrente ed evitare la deposizione di eccessive quantità di limo. 

\subsection{Prevenzione della diffusione della peste del gambero}
\emph{Aphanomices astaci} è un fungo parassita che causa morie massicce nei gamberi nativi europei (cfr. \ref{sub_parapat}) e poichè le specie aliene nordamericane sono portatrici sane della micosi, \emph{A. astaci} può essere diffuso sia per contatto diretto tra gamberi che indirettamente con il flusso dell'acqua, il movimento di animali acquatici. La sua diffusione nelle popolazioni contagiate è rapida e incontrollabile e da imputare principalmente all'azione umana, soprattutto nel caso di infestazioni in bacini o sottobacini idrografici in cui non sono presenti popolazioni alloctone. È pertanto necessario mettere in opera azioni di prevenzione per evitare la diffusione di patogeni tra diversi corpi idrici:
\begin{itemize} 
  \item evitare la movimentazione di gamberi vivi o morti, potenzialmente infetti, di acqua o attrezzatura contaminata, verso aree indenni ospitanti popolazioni suscettibili.
  \item evitare la movimentazione ed il rilascio di specie ittiche provenienti da aree soggette ad episodi di afanomicosi (le spore rimangono vitali nel muco della pelle e del tratto intestinale dei pesci, pertanto l'introduzione di materiale ittico da ripopolamento può facilitare la diffusione di \emph{A. astaci} anche per la presenza di spore nell'acqua di trasporto. La pulizia e l'eviscerazione dei pesci provenienti da altri corpi idrici è una potenziale fonte di infezione 
\end{itemize}

Applicare una corretta disinfezione di attrezzatura potenzialmente contaminata (guadini, nasse, stivali, vestiario, natanti, veicoli) tra campionamenti effettuati in bacini diversi (ad esempio: asciugamento per 48-72 ore; riscaldmento a 60$^{\circ}$C per pochi minuti; congelamento a -20$^{\circ}$C per 72 ore; sterilizzazione con varechina diluita in acqua 1:3 o con iodofori a concentrazione 100 ppm per 30 minuti.
Attuare campagne informative per la popolazione e soprattutto per i pescatori, indicando gli accorgimenti per decontaminare materiali e attrezzature. 

\subsection{Regolamentazione delle attività antropiche e sorveglianza}
Nonostante \emph{A. pallipes} rientri nella lista delle specie protette in Trentino (Allegato C, Legge Provinciale 23 maggio 2007 n. 11, cfr. \ref{sec_normativa}), il suo prelievo è consentito e regolamentato dalle norme per l'esercizio della pesca nella provincia di Trento (Consiglio della Provincia Autonoma di Trento, 1979), che si limitano ad imporre il prelievo di esemplari con una misura minima di 7 cm. ed un periodo di fermo pesca di 3 mesi, dal 1 aprile al 30 giugno. La cattura è consentita esclusivamente con l'utilizzo della canna da pesca e non è quindi permesso l'utilizzo di nasse, retini o il prelievo manuale. In previsione dell'emanazione della nuova normativa sulla pesca sarebbe opportuno proporre l'eliminazione del gambero autoctono  dalla lista delle specie pescabili.

La sorveglianza da parte di personale addetto è infine necessaria a evitare comportamenti dannosi che possono portare all'alterazione delle popolazioni e degli habitat. In particolare dovrebbero essere previsti controlli volti a evitare:
\begin{itemize}
  \item bracconaggio;
  \item immissione diretta di pesticidi e diserbanti in alveo (ad esempio per l'eccessiva vicinanza delle colture ai corpi idrici) e scarichi diretti illegali in acque superficiali (scarichi civili, industriali, zootecnici);
  \item fenomeni di secca per il mancato rispetto del minimo deflusso vitale e le derivazioni illegali dei corsi d'acqua del reticolo maggiore e minore;
  \item interventi in alveo non autorizzati o effettuati secondo metodi non sostenibili.
\end{itemize}

La sorveglianza deve essere prioritaria per le popolazioni presenti in zone tutelate (riserve naturali e biotopi di interesse provinciale, Zone Speciali di Conservazione della rete Natura 2000, parchi nazionali) e per le popolazioni più abbondanti in zone non tutelate e se possibile estesa a tutte le popolazioni segnalate. Per ottimizzare i controlli sarebbe auspicabile il coinvolgimento delle associazioni di pescatori nella segnalazione di comportamenti scorretti, fenomeni di degrado e morie estese di gamberi previa formazione del personale volontario individuato allo scopo.

\section{Gestione del monitoraggio e delle informazioni}
\subsection{Organizzazione dei monitoraggi delle popolazioni e degli habitat}
I dati di monitoraggio ottenuti attraverso censimenti, campionamenti e il rilievo delle caratteristiche ambientali potranno essere analizzati per ottenere informazioni relative alle  variazioni di distribuzione, abbondanza e struttura delle popolazioni nel tempo che permettono di valutare lo status della specie in provincia e di evidenziare fenomeni di declino o di espansione delle popolazioni permettendo di fare delle previsioni sulle condizioni future. Individuare le popolazioni minacciate e a rischio di estinzione, per le quali potranno essere previsti interventi di conservazione o popolazioni particolarmente abbondanti che potranno svolgere la funzione di sorgente di individui per le attività di allevamento, reintroduzione o di divulgazione. E' inoltre possibile individuare i fattori ambientali che influenzano la presenza/assenza della specie permettendo di rilevare i fattori di minaccia e di programmare interventi per il ripristino degli habitat. Il monitoraggio delle popolazioni e degli habitat è dunque fondamentale ai fini della corretta gestione e della conservazione della specie e delle singole popolazioni in Trentino e deve essere effettuato con cadenze temporali adeguate ad attuare idonee strategie di intervento. In particolare, si raccomanda di effettuare:
\begin{description}
  \item[monitoraggio degli habitat] al primo rilievo di nuovi habitat, successivamente ogni tre anni, in occasione dei campionamenti delle popolazioni, o eccezionalmente in caso di variazioni evidenti delle condizioni ambientali;
  \item[censimenti] annuali;
  \item[campionamenti] triennali o qualora si verifichi evidente contrazione della popolazione rilevata durante i censimenti.
\end{description}

Il comportamento dei gamberi può variare nel corso della stagione in rapporto alla classe di taglia e al sesso influenzando leggermente la struttura e il rapporto sessi rilevato. E' quindi necessario ripetere censimenti e campionamenti di ciascuna popolazione sempre nello stesso periodo (variando al massimo di qualche settimana) nel corso degli anni in modo tale da ottenere dati confrontabili nel tempo. 

I censimenti e i monitoraggi delle popolazioni e dei loro habitat dovranno essere condotti da personale qualificato e formato sulla biologia e l'ecologia del gambero, i rischi e le minacce, sull'identificazione delle specie e sui metodi previsti dal presente piano di gestione per eseguire i censimenti e i campionamenti delle popolazioni (rilievo di dati biometrici, sesso e raccolta di campioni biologici per le analisi genetiche). Per il rilievo degli habitat sono richieste inoltre conoscenze di base di ecologia fluviale e capacità di riconoscimento di taxa di macroinvertebrati acquatici. L'eventuale formazione di personale può essere svolta dagli enti di ricerca della Provincia già coinvolti in progetti per il monitoraggio e la conservazione del gambero di fiume in Trentino (MUSE, FEM). 

Per una corretta standardizzazione delle attività e dei dati raccolti, sono predisposti dei protocolli di campionamento per il personale. Tali protocolli, similmente a quanto proposto da altri progetti LIFE condotti in nord Italia e dedicati ai gamberi \cite{CRAINAT}\cite{RARITY} contengono: 
\begin{itemize}
  \item scheda da campo per il monitoraggio degli habitat: modificata da Rapid Bioassessment Protocols for Use in Streams and Wadable Rivers e predisposta per le condizioni idromorfologiche prevalenti in Trentino (Allegato I);
  \item scheda da campo per il censimento dei decapodi (Allegato III): per la quale si richiede l'indicazione di: bacino idrografico, corpo idrico e località in cui ricade il sito indagato, la data, l'ora di inizio e fine censimento, le specie di gambero presenti, il codice del sito, il codice di rilievo delle coordinate GPS per il punto di inizio e di fine del tratto osservato, la presenza e il numero di gamberi adulti, giovani, morti osservati o le eventuali tracce (esuvie), e l'ID delle fotografie scattate;
  \item scheda da campo per il monitoraggio dei decapodi (Allegato IV): per ogni corpo idrico si richiede l'indicazione di: bacino idrografico, corpo idrico e località in cui ricade il sito indagato, la data, l'ora di inizio e fine monitoraggio, il numero di rilevatori e il tipo di campionamento impiegati, le specie di gambero presenti, il codice del sito, il codice di rilievo delle coordinate GPS per il punto di inizio e di fine del tratto osservato e, per ciascun individuo caratterizzato: il numero progressivo, il sesso, il peso, la lunghezza del cefalotorace, la presenza di chele rigenerate, il codice del campione di tessuto (se prevista l'analisi genetica) ed eventuali note (presenza di uova, lesioni, parassiti, sintomi da patogeni).
\end{itemize}

\subsection{Gestione dei dati e delle informazioni}
\label{sub_dat}
\subsubsection{Organizzazione e gestione del database}
\label{subsub_dat}
La corretta attuazione del piano di gestione richiede una raccolta centralizzata di tutti i dati rilevati attraverso le segnalazioni, i censimenti, i monitoraggi e delle analisi genetiche, sanitarie e cartografiche. Deve quindi essere predisposto un database per la gestione dei dati che, costantemente aggiornato, possa rappresentare lo strumento fondamentale per la consultazione e l'analisi dei dati raccolti, la valutazione delle azioni intraprese e la programmazione di nuove azioni (censimenti, monitoraggi, riqualificazione di habitat, reintroduzione/ripopolamento di nuove popolazioni, controllo delle minacce ecc.) e la gestione delle emergenze (eradicazione/contenimento di popolazioni alloctone, segnalazione focolai di peste del gambero, popolazioni di \emph{A. pallipes} minacciate di estinzione ecc.). 

In particolare il database, dovrà contenere:
\begin{itemize}
  \item le segnalazioni di presenza di specie: dati rilevati attraverso applicazioni per dispositivi mobili (es. iNaturalist) e la scheda di segnalazione (allegato II) pubblicata sul WebGIS creato nell'ambito dell'Azione A1 del Progetto Life+TEN ed eventualmente sul sito web dedicato al progetto di conservazione del gambero di fiume, e fornita in alcuni casi in forma cartacea (ad esempio presso le stazioni forestali);
  \item i dati rilevati durante censimenti e monitoraggi delle popolazioni: registrati tramite le schede di censimento (allegato III) e di monitoraggio (allegato IV) del gambero;
  \item la caratterizzazione genetica delle popolazioni risultante delle analisi genetiche;
  \item lo stato sanitario delle popolazioni ottenuto da analisi istologiche per il rilievo di infestazioni da parassiti e patogeni;
  \item i dati ambientali: registrati tramite la scheda di valutazione degli habitat (allegato I) e le informazioni integrative ottenute da analisi cartografica.
\end{itemize}

Il database deve consentire:
\begin{itemize}
  \item la consultazione dei dati da parte degli operatori coinvolti nel progetto e del pubblico generico con livelli di dettaglio dell'informazione calibrata al tipo di utente;  
  \item l'estrazione dei dati per le analisi per la valutazione dello status delle popolazioni e degli habitat, l'individuazione delle aree di rischio ecc.; 
  \item la visualizzazione su mappa dei dati inseriti. 
\end{itemize}

La creazione e la gestione del database deve essere affidata a un ente/servizio che abbia maturato esperienza in tale settore e che operi a stretto contatto con i soggetti coinvolti nel rilievo e nell'analisi dei dati, nella programmazione delle azioni e nella gestione delle emergenze e che sia in grado di coordinarsi con i servizi gestori delle risorse naturali della Provincia. L'inserimento dei dati deve essere effettuato dagli operatori che si sono occupati dei rilievi o da personale dedicato che lavori comunque a stretto contatto con i rilevatori. Tempi e costi di informatizzazione dei dati possono essere ridotti attraverso l'utilizzo sul campo di applicazioni per dispositivi mobili (come GIS Cloud's Mobile Data Collection) che consentono la registrazione dei dati e il loro automatico trasferimento al database generale. Per l'utilizzo di tali applicazioni dovranno essere predisposte delle schede di rilievo dei dati contenenti campi corrispondenti a quelli del database generale. Per una prima impostazione è quindi opportuno l'intervento dell'operatore addetto alla creazione del database che dovrà anche impostare nell'applicazione la giusta connessione al database e creare nel database gli utenti relativi, con i giusti permessi. 

Le segnalazioni automaticamente inserite nel database, attraverso l'applicazione per dispositivi mobili e la scheda di segnalazione pubblicato sul sito web, dovranno essere validate dal personale formato e già individuato per la raccolta e l'analisi dei dati, e costituiranno (per quanto riguarda le popolazioni non ancora monitorate), assieme ai dati ambientali, le basi per la selezione dei siti che saranno interessati dai successivi monitoraggi. 
I dati integrativi derivanti da analisi cartografica e dalla bibliografia (pubblicazioni e report realizzati nell'ambito di progetti di ricerca o di monitoraggi ambientali) devono essere inseriti nel database dal personale che si occupa della programmazione e dell'individuazione dei siti da monitorare. Le analisi dei dati dovranno essere eseguite da ricercatori o tecnici formati sull'utilizzo di modelli matematici per lo studio delle popolazioni e degli habitat e con un'approfondita conoscenza sullo stato del gambero in Trentino così da garantire una corretta interpretazione dei risultati e l'elaborazione delle migliori strategie da mettere in atto per la conservazione di \emph{A. pallipes}. 

\subsubsection{Organizzazione e gestione di una piattaforma web}
\label{subsub_web}
È auspicabile che le azioni messe in atto con il Piano di gestione siano opportunamente divulgate attraverso la realizzazione di un sito web dedicato che possa rappresentare un supporto anche per: la raccolta di segnalazioni, attraverso la pubblicazione della scheda di segnalazione (Allegato II) e l'invito all'utilizzo, in alternativa, l'applicazione per dispositivi mobili "iNaturalist", e le attività di educazione volte alla sensibilizzazione del pubblico relativamente alla conservazione delle popolazioni native di gambero, ai rischi connessi alla diffusione delle specie alloctone e ai comportamenti da tenere o da evitare per contribuire alla conservazione delle popolazioni native. Il sito deve essere costantemente aggiornato sulle azioni messe in atto, sui risultati ottenuti e sugli eventi divulgativi/educativi in programma. Sarà inoltre collegato al database e permetterà la visualizzazione cartografica sempre aggiornata della distribuzione delle specie di gambero in Trentino (con un livello di dettaglio adeguato alla consultazione da parte del pubblico). Il sito web sarà realizzato dagli operatori coinvolti nella realizzazione del database e gestito da personale coinvolto nelle azioni di divulgazione del piano di gestione. 

\section{Divulgazione}  
\label{sec_div}
La divulgazione delle azioni attuate attraverso il piano di gestione del gambero di fiume in Trentino è fondamentale per il raggiungimento degli obiettivi conservazionistici previsti e può essere rivolta a diverse categorie di soggetti secondo varie modalità.  
I pescatori dilettanti rappresentano un bacino di utenza estremamente ampio in Trentino e le associazioni concessionarie, oltre a gestire i diritti di pesca sul territorio tramite il rilascio dei permessi d'ospite o annuali, hanno anche il compito di salvaguardare la fauna ittica, svolgere una funzione di sorveglianza, e fornire informazioni turistiche e materiale informativo aggiuntivo. Le associazioni possono quindi rappresentare dei centri di distribuzione di materiale specificatamente predisposto per informare i pescatori sui rischi connessi alla diffusione della peste del gambero, le modalità di prevenzione e le altre principali minacce oltre che sulle modalità di identificazione e di segnalazione delle specie (cfr. \ref{subsub_segn}). Tali informazioni potrebbero essere divulgate tramite brochure da distribuire e allegare ai permessi da pesca giornalieri e stagionali. Informazioni più dettagliate che comprendano anche la biologia e l'ecologia dei gamberi possono essere fornite durante eventi informativi o corsi come quello periodicamente organizzato dall'Accademia Ambiente Foreste e Fauna del Trentino (AAFF), Fondazione Edmund Mach, per il rilascio della licenza di pesca o attraverso la pubblicazione di articoli divulgativi sui siti web e sulle riviste e delle associazioni di pescatori (come la rivista "Il Pescatore Trentino", edita dall'Associazione Pescatori Dilettanti Trentini, con tre numeri annuali ognuno con una tiratura di 6.000 copie, che quindi raggiunge una buona utenza locale). Azioni di informazione e sensibilizzazione potrebbero essere programmate attraverso incontri con le principali associazioni di pescatori, quali l'Associazione Pescatori Dilettanti Trentini (che conta quasi 1500 soci e gestisce la pesca nelle acque del Trentino centrale ossia nel fiume Adige e nei tratti inferiori dei suoi affluenti tra cui l'Avisio in val di Cembra e il Noce oltre al territorio dell'altopiano di Pinè), e la Federazione Pescatori Trentini (che gestisce i terriotri di: alto Chiese, Valsugana, Val di Fiemme e di Fassa, Sarca, Valle di Ledro, Val di Sole e Val di Non, Grigno e Levico Terme). Tali incontri costituirebbero un'opportunità per instaurare possibili rapporti di collaborazione nell'attuazione di particolari azioni previste dal piano di gestione.  
Il successo nell'applicazione di un piano di gestione per la conservazione della specie si ottiene anche attraverso il consenso e la collaborazione da parte della popolazione. Comportamenti errati da parte della popolazione, come l'introduzione di specie alloctone, il prelievo e la traslocazione di organismi e la generazione di impatti ambientali, sono infatti spesso dovuti ad una mancanza di informazione e possono annullare gli sforzi messi in atto attraverso le azioni di conservazione. L'organizzazione di campagne informative, relative agli interventi che si vogliono attuare sul territorio, e di sensibilizzazione sulle problematiche legate alla conservazione di \emph{A. pallipes}, è quindi essenziale per il raggiungimento di buoni risultati. I cittadini trentini sono tradizionalmente legati al loro territorio e rappresentano quindi un target per attività di coinvolgimento, educazione e sensibilizzazione relative alla conservazione del gambero di fiume. E' quindi opportuno predisporre materiale informativo sulle azioni del piano di gestione del gambero, che riporti il link al sito web dedicato e le modalità per contribuire al progetto attraverso la segnalazione di gamberi, come brochure e articoli divulgativi da pubblicare su riviste di carattere ambientale e quotidiani locali. Andranno poi predisposti opportuni progetti di educazione ambientale legati alla conservazione del gambero di fiume per le scuole e per il pubblico in generale anche attraverso lo sviluppo di percorsi didattici basati sulla biologia e l'ecologia della specie e sui rapporti con l'uomo e le minacce in senso generale e locale. Esperienze di altri progetti dedicati a \emph{A. pallipes} (ad esempio CRAINat) suggeriscono lo sviluppo di attività didattiche che coinvolgano le ultime classi della Scuola Primaria e Secondaria di primo grado, che comprendano attività da svolgere in aula e sul campo individuando i siti idonei a svolgere tale attività. Azioni di formazione, sensibilizzione e la distribuzione di materiale informativo possono essere intraprese dalle aree protette, dalle reti di riserve e dalle APT.

I contenuti del piano di gestione e i risultati delle azioni potranno essere divulgate alla comunità scientifica durante convegni e workshop anche nell'ottica di un confronto con esperienze maturate nell'ambito di altri progetti da altri gruppi di ricerca e di ottenere dei feedback da parte di altri ricercatori operanti su queste tematiche.

\chapter{Priorità d'intervento e organizzazione delle azioni}
\DeactivateBG
\AddToShipoutPicture*{\BackgroundPicTwo}
Il piano di gestione offre un quadro generale di tutte le azioni che possono essere intraprese per garantire la conservazione delle popolazioni di \emph{A. pallipes} in Trentino. E' quindi necessario individuare a questo punto le priorità d'intervento che permetteranno di distinguere:
\begin{enumerate}[label={\arabic*)}]
  \item \textbf{azioni urgenti (priorità alta}: da attuare in tempi brevi (entro i primi 3 anni dall'adozione del Piano di Gestione) per evitare l'estinzione delle popolazioni ancora presenti sul territorio provinciale. Queste azioni comprendono:
  \begin{itemize}
    \item il monitoraggio delle popolazioni e degli habitat (cfr. \ref{sec_monitoraggio_pop}): per ottenere un quadro preciso sul numero, la distribuzione e lo stato di conservazione delle popolazioni di gambero di fiume ancora presenti in Trentino e per individuare popolazioni alloctone da eradicare;
    \item l'analisi del DNA nucleare delle popolazioni già conosciute per le quali sono già disponibili i campioni (presso il Dipartimento di Biodiversità ed Ecologia Molecolare della FEM) e del DNA mitocondriale e nucleare di eventuali nuove popolazioni che si rileveranno durante i monitoraggi;
    \item interventi su popolazioni e habitat già indagati per i quali sono state osservate gravi criticità (vedi allegato V);
    \item la creazione di un database per l'archiviazione, la consultazione, l'analisi e la divulgazione dei dati e l'impostazione delle applicazioni collegate (es. GIS Cloud's Mobile Data Collection);
    \item la pubblicazione sul WebGIS creato nell'ambito dell'Azione A1 del Progetto Life+T.E.N delle schede di segnalazione e del collegamento all'applicazione per dispositivi mobili per effettuare delle segnalazioni. 
    \item divulgazione presso le Associazioni di pescatori: distribuzione di materiale informativo, pubblicazione di articoli su riviste di settore e siti web e coinvolgimento dei pescatori nelle attività di conservazione; 
    \item realizzazione di un sito web dedicato al progetto di conservazione del gambero;
  \end{itemize}
  \item \textbf{azioni a priorità media}: azioni che permettono nel lungo periodo (entro i primi 6 anni dall'adozione del Piano di Gestione) di migliorare le condizioni delle popolazioni e degli habitat anche attraverso la collaborazione delle associazioni locali e l'informazione del pubblico:
  \begin{itemize}
    \item riqualificazione di habitat ospitanti \emph{A. pallipes} per favorire l'espansione delle popolazioni;
    \item creazione di allevamenti all'aperto e stagni multifunzione: che permettano interventi sulle popolazioni nel lungo periodo (attraverso l'allevamento e il rilascio di un numero contenuto di individui da effettuare nel corso di più anni in modo da non incidere negativamente sulle popolazioni sorgente) a costi limitati e rappresentino anche strumenti utili allo svolgimento di attività didattiche; 
    \item collaborazione con i gestori di laghetti da pesca e di pescicolture che presentano popolazioni naturali di \emph{A. pallipes} nei loro impianti per favorirne la conservazione;
    \item attivazione di collaborazioni con i fruitori e i gestori delle risorse idriche (ad esempio BIM, produttori idroelettrico) che potrebbero realizzare o finanziare particolari interventi;
    \item divulgazione al pubblico generico e divulgazione scientifica;
  \end{itemize}
  \item \textbf{Azioni a priorità bassa} comprendono interventi che potranno essere programmati nel lungo periodo (entro i primi 9 anni dall'adozione del Piano di Gestione), in base alle necessità e alle criticità emerse attraverso le analisi dei dati di monitoraggio:
  \begin{itemize}
    \item studi che permettono di approfondire notevolmente le conoscenze sullo stato delle popolazioni in casi particolari o di valutare gli interventi attuati (studi di cattura-marcatura-ricattura e analisi genetiche finalizzate al rilievo della variabilità della popolazione su popolazioni reintrodotte o in declino);
  \end{itemize}
\end{enumerate}

Azioni a priorità elevata e media, possono essere inserite nella programmazione delle attività e degli interventi attuati dalle Reti di Riserve che, in alcuni casi hanno già avviato monitoraggi o hanno mostrato interesse per l'attuazione di interventi a favore del gambero di fiume come le Reti di Riserve Monte Bondone, Sarca e Fiemme-Destra Avisio. La realizzazione delle azioni a priorità bassa richiede il coordinamento tra gli enti e i dipartimenti interni per la condivisione di progetti, obiettivi e attività in modo tale da elaborare strategie comuni ed ottenere i massimi risultati minimizzando gli sforzi. Il progetto di conservazione del gambero di fiume dovrebbe essere quindi portato a conoscenza di tutti gli operatori degli enti potenzialmente coinvolti in modo che le azioni proposte possano essere tenute presenti nello svolgimento delle attività di ciascuno.
\ActivateBG
In allegato VI la tabella riassuntiva delle singole azioni con indicazione del livello di priorità e degli operatori che dovrebbero essere necessariamente coinvolti. In allegato VII il cronoprogramma di tutte le attività illustrate nel piano, per i prossimi tre trienni (da integrare nel PAF, Prioritised Action Framework del progetto LIFE +TEN), che viene sintetizzato in tabella\ref{tab_10}.

% Please add the following required packages to your document preamble:
% \usepackage{booktabs}
% \usepackage{graphicx}
\begin{table}[]
\centering
\resizebox{\textwidth}{!}{%
\begin{tabular}{@{}c@{}}
\toprule
\textbf{Programma di azioni e interventi}                                                            \\ \midrule
                                                                                                     \\
\textbf{PRIMO TRIENNIO}                                                                              \\
1. Monitoraggio habitat e popolazioni                                                                \\
2. Analisi dati raccolti per la definizione del quadro di conservazione delle specie e degli habitat \\
3. Analisi genetiche                                                                                 \\
4. Definizione azioni concrete (= azioni di riqualificazione)                                        \\
5. Emergenza eradicazioni                                                                            \\
6. Avvio programma di allevamento                                                                    \\
7. Azioni di comunicazione e formazione                                                              \\
\\
\textbf{SECONDO E TERZO TRIENNIO}                                                                    \\
8. Monitoraggio habitat e popolazioni                                                                \\
9. Attuazione azioni concrete sulla base del programma definito in 4                                 \\
10. Emergenza eradicazioni                                                                           \\
11. Allevamenti                                                                                      \\
12. Azioni di comunicazione e formazione                                                             \\ \bottomrule
\end{tabular}%
}
\caption{Programma di azioni e interventi del piano di gestione, per il primo e secondo+terzo triennio di implementazione.}
\label{tab_10}
\end{table}


\begin{thebibliography}{9}
\DeactivateBG
\AddToShipoutPicture*{\BackgroundPicTwo}
\bibitem{APPA 2013} Agenzia Provinciale per la Protezione dell'Ambiente Settore Informazione e Monitoraggi, 2013. Progetto IFF del trentino. \emph{Applicazione dell'IFF 2007 (indice di funzionalità fluviale) sui corsi d'acqua tipizzati del trentino}.

\bibitem{AA.VV. 2007} AA.VV., 2007. \emph{Linee guida per l'immissione di specie faunistiche}. Quad. Cons. Natura, 27, Min. Ambiente - Ist. Naz. Fauna Selvatica.

\bibitem{AA.VV. 2014} AA.VV., 2014. \emph{Action plan per la conservazione di Austropotamobius pallipes in Italia}. Pubblicazione realizzata nell'ambito del progetto LIFE08 NAT/IT/000352 - CRAINat con il contributo finanziario del programma "LIFE-Natura e Biodiversità" della Commissione Europea,  144 pp.

\bibitem{Ackefors 1996} Ackefors H., 1996. \emph{The development of crayfish culture in Sweden during the last decade}. Freshwater Crayfish 11: 627-654.

\bibitem{Albrecht 1982} Albrecht H., 1982. \emph{Das System der europäischen Flußkrebse ( Decapoda, Asticidae) Vorschlag und Begrüdung}. Mitt. Hamb. Zool. Mus. Inst. 79:187‐210.

\bibitem{Alderman 1986} Alderman D.J., Feist S.W., Polglase J.L., 1986. \emph{Possible nocardiosis of crayfish, Austropotamobius pallipes}. J. Fish Dis. 9:345-347.

\bibitem{Aquiloni 2009} Aquiloni L., Becciolini A., Berti R., Porciani S., Trunfio C., Gherardi F., 2009. \emph{Managing invasive crayfish: use of X-ray sterilization of males}. Freshwater Biology 54:1510-1519.

\bibitem{Aquiloni 2010} Aquiloni L., Tricarico E., Gherardi F., 2010. \emph{Crayfish in Italy: distribution, threats and management}. International Aquatic Research (2010) 2:1-14. 

\bibitem{Aquiloni 2010b} Aquiloni L., Brusconi S., Cecchinelli E., Tricarico E., Mazza G., Paglianti A., Gherardi F., 2010b. \emph{Biological control of invasive populations of crayfish: the European eel (Anguilla anguilla) as a predator of Procambarus clarkii}. Biol. Invasions 12:3817- 3824.

\bibitem{Arrignon 1996} Arrignon J., 1996. \emph{L'écrevisse et son élevage, \cellcolor{Goldenrod}IIIed. Paris: Tech. \& Doc}. Ed.

\bibitem{Barbour 1999} Barbour M.T., Gerritsen J., Snyder B.D., Stribling J.B., 1999. \emph{Rapid Bioassessment Protocols for Use in Streams and Wadeable Rivers: Periphyton, Benthic Macroinvertebrates and Fish}. Second Edition. EPA 841-B-99-002. U.S. Environmental Protection Agency; Office of Water; Washington, D.C.

\bibitem{Baric 2005} Baric S., Höllrigl A., Füreder L., Dalla Via J., 2005. \emph{Mitochondrial and microsatellite DNA analysis of Austropotamobius pallipes populations in South Tyrol (Italy) and Tyrol (Austria)}. BFPP - Bull. Francais Peche Prot. Milieux Aquat., pp. 376-377, 599-612.

\bibitem{Bernini 2016} Bernini G., Bellati A., Pellegrino I., Negri A., Ghia D., Fea G., Sacchi R., Nardi P.A., Fasola M., Galeotti P., 2016. \emph{Complexity of biogeographic pattern in the endangered crayfish Austropotamobius italicus in northern Italy: molecular insights of conservation concern}. Conserv. Genet. 17:141. 

\bibitem{Bertocchi 2008} Bertocchi S., Brusconi S., Gherardi F., Buccianti A., Scalici M., 2008. \emph{Morphometrical characterization of the Austropotamobius pallipes species complex}. J. Nat. Hist. 42:2063-77.

\bibitem{Bertucci Maresca 2015} Bertucci Maresca V., 2015. \emph{Genetic characterization of Austropotamobius pallipes (Lereboullet, 1858) complex in Friuli Venezia Giulia for restocking purposes}. Tesi di dottorato di Ricerca in Biologia Ambientale. Università degli Studi di Trieste, Dipartimento di Scienze della Vita.

\bibitem{Bott 1950} Bott R., 1950. Die Flusskrebse Europas. Abh. Senckenb. Natforsch. Ges. 483:1-36.

\bibitem{Breiman 2001} Breiman L., 2001. \emph{Random Forest}. Machine Learning 45:5-32. 

\bibitem{Broasky 1983} Broasky S. Y. \emph{On the systematic of paleartic crayfishes (Crustacea, Astacidae)}. Freshwater Crayfish, pp. 464-70.

\bibitem{Brusconi 2008} Brusconi S., Bertocchi S., Renai B., Scalici M., Souty-Grosset C., Gherardi F., 2008. \emph{Conserving indigenous crayfish: stock assessment and habitat requirements in the threatened Austropotamobius italicus}. Aquat. Conserv., 18: 1227-1239. 

\bibitem{Cappelletti 2014} Cappelletti C., Ciutti F., 2014. \emph{Prima segnalazione di Procambarus clarkii (Girard, 1852) nella provincia di Trento}. XV Congresso Nazionale A.I.I.A.D. - Gorizia, 14-15 novembre 2014. book of abstracts pp. 43-44.

\bibitem{Cappelletti 2016} Cappelletti C., Ciutti F., 2016. \emph{Relazione finale Progetto LIFE+TEN azione C10 - Salvaguardia delle popolazioni autoctone di Austropotamobius pallipes - Allevamento di Austropotamobius pallipes}. 8 pp.

\bibitem{Carral 1994} Carral J.M., Celada J.D., González J., Sáez-Royuela M., Gaudioso V.R., 1994. \emph{Mating and spawning of freshwater crayfish, Austropotamobius pallipes Lereboullet, under laboratory conditions}. Aquaculture and Fisheries Management 25:721-727.

\bibitem{Carral 2000} Carral J. M., J. D. Celada, C. Muñoz, M. Sáez-Royuela \& J. R. Pérez. 2000. \emph{Effects of the presence or absence of males throughout spawning and maternal incubation on the reproductive efficiency of astacid crayfish (Austropotamobius pallipes) under controlled conditions}. Invertebrate Reproduction \& Development. 38:1-5.

\bibitem{Cataudella 2006} Cataudella R., Puillandre N., Grandjean F., 2006. \emph{Genetic analysis for conservation of Austropotamobius italicus populations in Marches region (central Italy)}. Bull. Fr. Peche Piscic. pp. 380-381.

\bibitem{Cecchinelli 2012} Cecchinelli E., Aquiloni L., Maltagliati G., Orioli G., Tricarico E., Gherardi F., 2012. \emph{Use of natural pyrethrum to control the red swamp crayfish Procambarus clarkii in a rural district of Italy}. Pest Management Science, 68:839-844.

\bibitem{Chiesa 2011} Chiesa S., Scalici M., Negrini R., Gibertini G., Nonnis Marzano F., 2011. \emph{Fine-scale genetic structure, phylogeny and systematics of threatened crayfish species complex}. Mol. Phylogenet. Evol. 61:1-11.

\bibitem{CIRF 2006} CIRF, 2006. \emph{La riqualificazione fluviale in Italia. Linee guida, strumenti ed esperienze per gestire i corsi d'acqua e il territorio}. Nardini A., Sansoni G. (curatori) e collaboratori, Mazzanti Editori, Venezia.

\bibitem{Ciutti 2013} Ciutti F., Fin V., Lunelli F., Cappelletti C., 2013. \emph{Il gambero di fiume Austropotamobius Pallipes nelle aree protette della rete natura 2000 della provincia di Trento}. pp.95-105. In DENDRONATURA - ISSN:1121-7782 vol. 34 (2).
\ActivateBG
\bibitem{De Luise 2006} De Luise G., 2006. \emph{I crostacei decapodi d'acqua dolce in Friuli Venezia Giulia. Recenti acquisizioni sul comportamento e sulla distribuzione nelle acque dolci della Regione}. Venti anni di studi e ricerche. Ente Tutela Pesca - Regione Autonoma Friuli Venezia Giulia. pp. 94.

\bibitem{De Luise 2012} De Luise G., 2012. \emph{Protocolli di cattura, allevamento e ripopolamento del gambero di fiume nativo Austropotamobius pallipes in Friuli Venezia Giulia}. In: Didattica per gli operatori. Pubblicazione realizzata con il contributo finanziario della CE, nell'ambito del Progetto RARITY, LIFE10 NAT/IT/000239, editing testi Tiziano Scovacricchi, pp. 77-87.

\bibitem{Demers 2002} Demers A., Reynolds J. D., 2002. \emph{A survey of the white-clawed crayfish, Austropotamobius pallipes (Lereboullet), and of water quality in two catchments of eastern Ireland}. Bull. Fr. Peche Piscic., 367:729-740.

\bibitem{Dieguez Uribeondo 2006} Diéguez-Uribeondo J., Cerenius L., Dyková I., Gelder S., Henntonen P., Jiravanichpaisal P., Lom J. and Söderhäll K., 2006. \emph{Pathogens, parasites and ectocommensals}. In: Souty-Grosset C., Holdich D.M., Noël P.Y., Reynolds J.D. and Haffner P. (eds.), Atlas of Crayfish in European, Muséum national d'Histoire naturelle, Paris, Collection Patrimoines Naturels.

\bibitem{Dieguez Uribeondo 1997} Diéguez-Uribeondo J., Pinedo-Ruíz J., Múzquiz J. L., 1997 - \emph{Thelohania contejeani in the province of Alava, Spain}. Bull. Fr. Peche Piscic., 347:749-752.

\bibitem{Elgar 1992} Elgar M.A., Crespi, B.J., 1992. \emph{Ecology and evolution of cannibalsim. In: Cannibalism Ecology and Evolution among Diverse Taxa}. Oxford University Press, pp. 1-12.

\bibitem{Endrizzi 2013} Endrizzi S., Bruno M. C., Maiolini B., 2013. \emph{Distribution and morphometry of native and alien crayfish in Trentino (Italian Alps)}. Journal of Limnology, 72/2:343-360. 

\bibitem{Endrizzi 2014} Endrizzi S., Bruno M. C., Maiolini B., 2014. \emph{Action plans per la conservazione di specie focali di interesse comunitario ‐ Gambero di fiume}. Azione A8 del Progetto LIFE+TEN (LIFE11/NAT/IT/000187), 23 pp.

\bibitem{Faller 2016} Faller M., Harvey G.L., Henshaw A.J., Bertoldi W., Bruno M.C., England J., 2016. \emph{River bank burrowing by invasive crayfish: Spatial distribution, biophysical controls and biogeomorphic significance}. Science of the Total Environment, 569-570:1190-200. 

\bibitem{Favaro 2010} Favaro L., Tirelli T., Pessani D., 2010. \emph{The role of water chemistry in the distribution of Austropotamobius pallipes (Crustacea Decapoda Astacidae) in Piedmont (Italy)}. C. R. Biol. 333:68-75.

\bibitem{Fratini 2005} Fratini S., Zaccara S., Barbaresi S., Grandjean F., Souty-Grosset C., Crosa G., Gherardi F., 2005. \emph{Phylogeography of the threatened crayfish (genus Austropotamobius) in Italy: implications for its taxonomy and conservation}. Heredity 94, 108-118.

\bibitem{Fureder 2003} Füreder L., Reynolds J.D., 2003. \emph{Is Austropotamobius pallipes a good bioindicator?} Bull. Fr. Peche Piscic. 370-371:157-163.

\bibitem{Fureder 2003b} Füreder L., Oberkofler B., Hanel R., Leiter J., Thaler B., 2003. \emph{The freshwater crayfish Austropotamobius pallipes in south Tyrol: heritage species and bioindicator}. Bull. Fr. Peche Piscic., 370-371: 79-95.

\bibitem{Fureder 2006} Füreder L., Edsman L., Holdich D., Kozák P., Machino Y., Pöckl M., Renai B., Reynolds J., Schulz H., Schulz R., Sint D., Taugbol T., Trouilhé M.C., 2006. \emph{Indigenous crayfish habitat and threats}. In: Souty Grosset C., Holdich D.M., Noel P.Y., Reynolds J.D., Haffner P. (eds.), Atlas of crayfish in Europe. Muséum National d'Histoire Naturelle, Paris: 25-47. 

\bibitem{Fureder 2007} Füreder L., 2007. \emph{Artenschutzprojekt Südtiroler Bachkrebs (Decapoda Astacidae: Austropotamobius pallipes)}. Gredleriana 7: 155-170.

\bibitem{Gandolfi 2015} Gandolfi A., Vernesi C., 2015. \emph{Relazione tecnica a Provincia Autonoma di Trento, Servizio Sviluppo Sostenibile e Aree Protette: ‘Effettuazione di analisi genetiche sulle popolazioni trentine di gambero di fiume (Austropotamobius pallipes) a fini di reintroduzione nel bacino dell'Adige}.

\bibitem{Gelder 1999} Gelder S.R., Delmastro G.B., Rayburn J.N., 1999. \emph{Distribution of native and exotic branchiobdellidans (Annelida: Clitellata) on their respective hosts in northern Italy, with the first record of native Branchiodbella species on an exotic North American crayfish}. Journal of Limnology 58:20-24.

\bibitem{Genovesi 2004} Genovesi P., Shine C., 2004. \emph{European strategy on invasive alien species}. Nature and Environment, n.137, Strasbourg.

\bibitem{Ghetti 1997} Ghetti P. F., 1997. \emph{Indice Biotico Esteso (I.B.E.). I macroinvertebrati nel 
controllo della qualità degli ambienti di acque correnti}, Provincia autonoma di Trento e Agenzia provinciale per la protezione dell'ambiente

\bibitem{Ghia 2011} Ghia D., Fea G., Bernini F., Nardi P.A., 2011. \emph{Reproduction experiment on Austropotamobius pallipes complex under controlled conditions: Can hybrids be hatched?} Knowl. Managt. Aquatic Ecosyst. 401:16.

\bibitem{Girardet 2012} Girardet M.A., Cherix D., Hofmann F., Rubin J.F., 2012. \emph{Eradication of a red swamp crayfish Procambarus clarkii population in Vidy pond and crayfish population status at Lausanne, Switzerland}. Bulletin de la Societe Vaudoise des Sciences Naturelles, 93(1-2): 2-12.

\bibitem{Goldberg 2016} Goldberg, C.S., Turner, C.R., Deiner, K., Klymus, K.E., Thomsen, P.F., Murphy, M.A., Spear, S.F., McKee, A.M., Oyler-McCance, S.J., Cornman, R.S., Laramie, M.B., Mahon, A.R., Lance, R.F., Pilliod, D.S., Strickler, K.M., Waits, L.P., Fremier, A.K., Takahara, T., Herder, J.E., Taberlet, P., 2016. \emph{Critical considerations for the application of environmental DNA methods to detect aquatic species}. Methods in Ecology and Evolution, v. 7, p. 1299-1307

\bibitem{Gouin 2001} Gouin N., Grandjean F., Bouchon D., Reynolds J. D., Souty-Grosset C., 2001. \emph{Population genetic structure of the endangered freshwater crayfish Austropotamobius pallipes, assessed using RAPD markers}. Heredity 87:80-7.

\bibitem{Gouin 2006} Gouin N., Grandjean F., Souty-Grosset C., 2006. \emph{Population genetic structure of the endangered freshwater crayfish Austropotamobius pallipes in France based on microsatellite variation: biogeographical inferences and conservazion implications}. Freshwater Biology 51:1369-1387.

\bibitem{Grandjean 1997} Grandjean F., Romain D., Souty-Grosset C., Mocquard, J. P., 1997. \emph{Size at maturity and morphometric variability in three populations of Austropotamobius pallipes pallipes (Lereboullet, 1858) according to a restocking strategy}. Crustaceana 70, 454-468.

\bibitem{Grandjean 2000} Grandjean F., Harris D. J., Souty-Grosset C., Crandall K.A., 2000. \emph{Systematic of the European endangered crayfish species Austropotamobius pallipes (Decapoda: Astacidae)}. J. Crustac. Biol. 20:522-529.

\bibitem{Grandjean 2000b} Grandjean F., Souty-Grosset C., 2000. \emph{Mitochondrial DNA variation and population genetic structure of the white-clawed crayfish Austropotamobius pallipes pallipes}. Conservation Genetics, 1:309-319.

\bibitem{Grandjean 2002} Grandjean F., Frelon-Raimond M., Souty-Grosset C., 2002. \emph{Compilation of molecular data for the phylogeny of the genus Austropotamobius: one species or several?} Bull. Fr. Peche Pisc. 367:671-680.

\bibitem{Hewitt 1996} Hewitt G. M., 1996. Some genetic consequences of ice ages, and their role in divergence and speciation. Biol J Linn Soc, 58: 247-276Italian Journal Of Zoology Vol. 77 , Iss. 4, 410-418.

\bibitem{Holdich 1999} Holdich D.M., Gydemo R., Rogers W.D., 1999. \emph{A review of possible methods for controlling alien crayfish populations}. Crustacean Issues, 11: 245-270.

\bibitem{Holdich 1999b} Holdich D., 2003. \emph{Ecology of the White-clawed Crayfish}. Conserving Natura 2000 Rivers Ecology Series No. 1. English Nature, Peterborough.

\bibitem{IUCN 2013} IUCN, 2013. \emph{IUCN Red List of Threatened Species}. Version 2013.2. \url{http://iucnredlist.org}.

\bibitem{IUCN 2013b} IUCN/SSC, 2013. \emph{Guidelines for Reintroductions and Other Conservation Translocations}. Version 1.0. Gland, Switzerland: IUCN Species Survival Commission, 57 pp. Documento disponibile nel sito: \url{www.iucnsscrsg.org}

\bibitem{Karaman 1962} Karaman M.S., 1962. \emph{Ein Beitrag zur Systematik der Astacidae (Decapoda)}. Crustaceana 3:173-91.

\bibitem{Keller 1988} Keller M., 1988. \emph{Finding a profitable population density in rearing summerlings of European crayfish, Astacus astacus}. L. Freshwater Crayfish, 7:259-266.

\bibitem{Kemp 2003} Kemp E., Birkinshaw N., Peay S., Hiley P. D., 2003. \emph{Reintroducing the White-clawed Crayfish Austropotamobius pallipes}. Conserving Natura 2000 Rivers Conservation Techniques Series No. 1. English Nature, Peterborough.

\bibitem{Kouba 2014} Kouba A., Petrusek A., Kozák P., 2014. \emph{Continental-wide distribution of crayfish species in Europe: update and maps}. Knowledge and Management of Aquatic Ecosystems. 413, 05 4 DOI: 10.1051/kmae/2014007.

\bibitem{Kouba 2015} Kouba A., Buřič M., Kozák P., 2015. \emph{Crayfish species in Europe}. In: P. Kozák, Z. Ďuriš, A. Petrusek, M. Buřič, I. Horká,.A. Kouba, E. Kozubíková-Balcarová, T. Policar. Crayfish Biology and Culture. University of South Bohemia in in České Budějovice ed., pp 81-163.

\bibitem{Largiader 2000} Largiader C. R., Herger F., Lortscher M., Scholl A., 2000. \emph{Assessment of natural and artificial propagation of the white-clawed crayfish (Austropotamobius pallipes species complex) in the Alpine region with nuclear and mitochondrial markers}. Mol Ecol, 9: 25-37, 10.1046/j.1365-294x.2000.00830.x.  |

\bibitem{Longshaw 2016} Longshaw M., 2016. \emph{Parasites, Commensals, Pathogens and Diseases of Crayfish}. Pp 171-250 in: Longshaw M., Stebbing P., (eds.): \emph{Biology and Ecology of Crayfish}. CRC Press, Boca Raton, FL.

\bibitem{Lortscher 1998} Lörtscher M., Clalüna M., Scholl A., 1998. \emph{Genetic population structure of Austropotamobius pallipes (Lereboullet 1858) (Decapoda: Astacidae) in Switzerland, based on allozyme data}. Aquatic Sciences 60:118-129.

\bibitem{Lowery 1988} Lowery R.S., 1988. \emph{Growth, moulting and reproduction}. In: Holdich DM and Lowery R.S. (eds). Freshwater crayfish: biology, management and exploitation. Croom Helm, London, 83-113.

\bibitem{Maiolini 2014} Maiolini B., Ciutti F., Cappelletti C., Lunelli F., 2014. \emph{Progetto Life+TEN, azione C10 - Salvaguardia delle popolazioni autoctone di A. pallipes. Campagna di cattura di femmine ovigere e stabulazione}. Allegato B, Rapporto Tecnico alla Provincia Autonoma di Trento, Dipartimento Territorio, Agricoltura, Ambiente e Foreste. 5 pp.

\bibitem{Mancini 1986} Mancini A., 1986. Astacicoltura. \emph{Allevamento e pesca dei gamberi d'acqua dolce}. Bologna: Edagricole.

\bibitem{Mason 1974} Mason J. C., 1974. \emph{Crayfish production in a small woodlandstream}. Freshwat. Crayfish, 2: 449 - 479.

\bibitem{Matallanas 2013} Matallanas B., Ochando M. D., Alonso F., Callejas C., 2013. \emph{Phylogeography of the white-clawed crayfish (Austropotamobius italicus) in Spain: inferences from microsatellite markers}. Mol Biol Rep. 40:5327-5338.

\bibitem{Mazzoni 2004} Mazzoni D., Gherardi F., Ferrarini P., 2004. \emph{Guida al riconoscimento dei gamberi d'acqua dolce}. Regione Emilia Romagna, Greentime SpA editrice, 56 pp.

\bibitem{Minghetti 2012} Minghetti G., Cappelletti C., Ciutti F., Bruno M. C., Endrizzi S., Quaglio F., Manfrin A., Pretto T. 2012. \emph{Indagine sullo stato sanitario del gambero americano Orconectes limosus in quattro popolazioni del Trentino}. \cellcolor{BurntOrange}IVConvegno AIIAD -Ittiologia come governance delle acque dolci italiane, Torino, 15‐17 November 2012. Book of abstract pp. 55.

\bibitem{Minghetti 2012b} Minghetti G., 2012. \emph{Indagine sulla diffusione e sullo stato sanitario del gambero di fiume Austropotamobius pallipes in Trentino}. Tesi scuola di specializzazione in Allevamento, Igiene, Patologia delle Specie Acquatiche e Controllo dei Prodotti Derivati. Università degli Studi di Padova, Dipartimento di biomedicina comparata e alimentazione. 

\bibitem{Mori 2000} Mori M., Salvadio S., 2000. \emph{The occurrence of Thelohania contejeani Henneguy, a microsporidian parasite of the crayfish Austropotamobius pallipes (Lereboullet), in Liguria Region (NW Italy)}. Journal of Limnology, 59: 167‐ 169.

\bibitem{Nardi 2004} Nardi P.A., Bernini F., Bo T., Bonardi A., Fea G., Ferrari S., Ghia D., Negri A., Razzetti E., Rossi S., 2004. \emph{Il gambero di fiume nella provincia di Alessandria}. Pavia: PI-ME. 111 pp.

\bibitem{Nonnis Marzano 2009} Nonnis Marzano F, Scalici M, Chiesa S, Gherardi F, Piccinini A, Gibertini G. 2009. \emph{The first record of the marbled crayfish adds further threats to fresh waters in Italy}. Aquat. Invas. 4: 401-404.

\bibitem{Nystrom 2002} Nyström P., 2002. \emph{Ecology and Biology of freshwater crayfish}. Edited by D.M. Holdich. Blackwell Science Ltd., Oxford. pp 192-235.

\bibitem{Oberkofler 2002} Oberkofler B., Quaglio F., Fureder L., Fioravanti M. L., Giannetto S., Morolli C., Minelli G., 2002. \emph{Species of Branchiobdellidae (Annelida) on freshwater crayfish in South Tyrol (Northern Italy)}. Bulletin Francais de la Peche et de la Pisciculture 367:777-784.

\bibitem{Pagotto 1995} Pagotto G., 1995. \emph{Studio e sperimentazione finalizzati alla identificazione delle possibilità di ripopolare con il gambero la parte veneziana del fiume Caomaggiore}. W.W.F. Delegazione Veneto Publ., Vicenza: 130 pp.

\bibitem{Paoli 2008} Paoli F., 2008. \emph{Specie aliene invasive in Alta Valsugana. Distribuzione di Orconectes limosus e Dreissena polymorpha}. Tesi di laurea. Università di Padova. 82 pp.

\bibitem{Parasiewicz 2007} Parasiewicz P., 2007. \emph{The MesoHABSIM model revisited}. River Res. Appl. 23:893-903.

\bibitem{Peay 2003} Peay S., 2003. \emph{Monitoring the White-clawed Crayfish Austropotamobius pallipes}. Conserving Natura 2000 Rivers Monitoring Series No. 1, English Nature, Peterborough.

\bibitem{Pecchioli 2006} Pecchioli E., Vernesi C., Crestanello C., Davoli F., Caramelli D., Bertorelle G., Hauffe H.C. 2006. \emph{Progetto FAUNAGEN: Conservazione e gestione della fauna: un approccio genetico (FAUNAGEN)}. Report n. 35 of the Centro di Ecologia Alpina. 86 pp.

\bibitem{Pockl 2002} Pockl M., Pekny R., 2002. \emph{Interaction between native and alien species of crayfish in Austria: case studies}. Bull. Fr. Peche Piscic. 367:763-776.

\bibitem{Policar 2015} Policar T., Kozák P., 2015. \emph{Production and culture of crayfish}. In: P. Kozák, Z. Ďuriš, A. Petrusek, M. Buřič, I. Horká,.A. Kouba, E. Kozubíková-Balcarová, T. Policar. \emph{Crayfish Biology and Culture}. University of South Bohemia in in České Budějovice ed., pp 295-363.

\bibitem{Pond 1975} Pond C.M., 1975. \emph{The role of the ‘walking legs' in aquatic and terrestrial locomotion of the crayfish Austropotamobius pallipes (Lereboullet)}. J. Exp. Biol., 62:447-454

\bibitem{Pretto 2011} Pretto T., Manfrin A., 2011. \emph{Patologie dei gamberi d'acqua dolce nel contesto del progetto LIFE+ RARITY}. Istituto Zooprofilattico Sperimentale delle Venezie, Laboratorio Nazionale di riferimento per le malattie dei crostacei. Ed. Adria: 9 pp.

\bibitem{Pritchard 2000} Pritchard J.K., Stephens M., Donnelly P., 2000. \emph{Inference of population structure using multilocus genotype data}. Genetics 155:945-959. 

\bibitem{Pritchard 2001} Pritchard J. K., Donnelly P., 2001. \emph{Case-control studies of association in structured or admixed populations}. Theor. Popul. Biol. 60(3): 227-237.

\bibitem{Pritchard 2004} Pritchard J. K., Wen W., 2004. \emph{Documentation for STRUCTURE software: Version 2}. Disponibile all'indirizzo: http://pritch.bsd.uchicago.edu.    

\bibitem{Quaglio 2006} Quaglio F., Morolli C., Galuppi R., Bonoli C., Marcer F. Nobile L., De Luise G.T, Tampieri M.P., 2006. \emph{Preliminary investigations of disease-causing organisms in the white-clawed crayfish Austropotamobius pallipes complex from streams of northern Italy}. Bull. Fr. Peche Piscicult., 380- 381: 1271-1290.

\bibitem{Quaglio 2008} Quaglio F., Galuppi R., Capovilla P., Santoro D., Bonoli C., Tampieri M.P., Fioretto B., 2008. \emph{Infezione da Saprolegniaceae in gamberi di fiume, Austropotamobius pallipes complex, in un allevamento sperimentale del nord Italia}. Ittiopatologia, 5, 1: 19-33.

\bibitem{Quaglio 2011} Quaglio F., Gustinelli A., Manfrin A., 2011. \emph{Patologie dei gamberi d'acqua dolce. Malattie infettive e di origine micotica}. Ittiopatologia, 2011, 8: 5-52.

\bibitem{Quaglio 2011b} Quaglio F., Capovilla P.,. Fioravanti M. L, Marino F., Gaglio G., Florio D., Fioretto B., Gustinelli A., 2011. \emph{Histological analysis of thelohaniasis in white-clawed crayfish Austropotamobius pallipes complex}. Knowl. Managt. Aquatic Ecosyst., 401:27. DOI: http://dx.doi.org/10.1051/kmae/2011045.

\bibitem{Rees 2014} Rees H. C., Maddison B. C., Middleditch D. J., Patmore J. R. M., Gough K. C., 2014. \emph{The detection of aquatic animal species using environmental DNA - a review of eDNA as a survey tool in ecology}. Journal of Applied Ecology 51, 1450-1459.

\bibitem{Regione Emilia Romagna 2012} Regione Emilia Romagna, 2012. \emph{Linee guida per la riqualificazione ambientale dei canali di bonifica in Emilia Romagna}. \url{http://ambiente.regione.emilia-romagna.it/suolo-bacino/sezioni/ pubblicazioni/servizio-difesa-del-suolo-della-costa-e-bonifica/pdf/Linee-Guida-Riqualificazione-Ambientale-Canali-RER_web2-leggera.pdf}.

\bibitem{Renai 2006} Renai B., Bertocchi S., Brusconi S., Gherardi F., Grandjean F., Lebboroni M., Perinet B., Souty Grosset C., Trouihè M.C., 2006. \emph{Ecological characterization of streams in Tuscany (Italy) for the management of the threatened crayfish Austropotamobius pallipes complex}. Bull. Fr. Peche Piscic., 380-381:1095-1114. 

\bibitem{Reynolds 1992} Reynolds J.D., Celada J.D., Carral J.M. and Matthews M.A. 1992. \emph{Reproduction of astacid crayfish in captivity-current developments and implications for culture, with special reference to Ireland and Spain}. Invertebrate Reproduction and Development 22: 253-256.

\bibitem{Rinaldi 2016} Rinaldi M., Belletti B., Comiti F., Nardi L., Mao L., Bussettini M., 2016. \emph{Sistema di rilevamento e classificazione delle Unità Morfologiche dei corsi d'acqua (SUM) - Versione aggiornata 2016}. Roma, Italy: Istituto Superiore per la Ricerca e la Protezione Ambientale (ISPRA). Manuali e Linee Guida 132/2016, 178 pp.

\bibitem{Santucci 1997} Santucci F., Iaconelli M., Andreani P., Cianchi R., Nascetti G., Bullini L., 1997. \emph{Allozyme diversity of European freshwater crayfish of the genus Austropotamobius}. Bull. Fr. Peche Piscic. 347:663-76.

\bibitem{Scalici 2008} Scalici M., Belluscio A., Gibertini G., 2008. \emph{Understanding population structure and dynamics in threatened crayfish}. J. Zool. 275:160-171.

\bibitem{Scalici 2009} Scalici M., Chiesa S., Gherardi F., Ruffini M., Gibertini G., Nonnis Marzano F., 2009. \emph{The new alien threat for the Italian continental waters from the crayfish gang: the turn of the invasive 
yabby Cherax destructor Clark, 1936}. Hydrobiologia. 632:341-345.

\bibitem{Scalici 2010} Scalici M., Di Giulio A., Gibertini G., 2010. \emph{Biological and morphological aspects of Branchiobdella italica (Annelida: Clitellata) in a native crayfish population of central Italy}. Italian Journal of Zoology, 77:4, 410-418.

\bibitem{Scalici 2012} Scalici M., Bravi R., 2012. \emph{Solving alpha-diversity by morphological markers contributes to arranging the systematic status of a crayfish species complex (Crustacea, Decapoda)}. J Zool Syst Evol Res. 50(2): 89-98.

\bibitem{Scalici 2016} Scalici M., Rovelli V., Zapparoli M. A., 2016. \emph{Austropotamobius pallipes (Lereboullet, 1858) sensu lato (gambero di fiume) A. torrentium (Schrank, 1803)}. In: Stoch F., Genovesi P. (ed.), Manuali per il monitoraggio di specie e habitat di interesse comunitario (Direttiva 92/43/CEE) in Italia: specie animali. ISPRA, Serie Manuali e linee guida, 141/2016.

\bibitem{Schonswetter 2004} Schönswetter P, Tribsch A, Stehlik I, Niklfeld H. 2004. \emph{Glacial history of high alpine Ranunculus glacialis (Ranunculaceae) in the European Alps in a comparative phylogeographical context}. Biological Journal of the Linnean Society 81: 183-195. 

\bibitem{Schulz 2002} Schulz R., Stucki T. and Souty-Grosset C., 2002. \emph{Roundtable Session 4A. Management: reintroductionsand restocking}. Bull. Fr. Peche Piscic. 367:917-922.

\bibitem{Siligardi 2007} Siligardi M., Avolio F., Baldaccini G., Bernabei S., Bucci M.S., Cappelletti C., Chierici E., Ciutti F., Floris B., Franceschini A., Mancini L., Minciardi M.R., Monauni C., Negri P., Pineschi G., Pozzi S., Rossi G.L., Sansoni G., Spaggiari R., Tamburro C., Zanetti M., 2007. IFF 2007. \emph{Indice di Funzionalità Fluviale. Nuova versione del metodo revisionata e aggiornata}. - APAT Serie Manuali.

\bibitem{Sint 2007} Sint D., Dalla Via J., Füreder L., 2007. \emph{Phenotipical characterization of indigenous freshwater crayfish populations}. J. Zool. 273: 210-219.

\bibitem{Smith 1996} Smith G. R. T., Learner M. A., Slater F. M., Foster J., 1996. \emph{Habitat features important for the conservation of the native crayfish Austropotamobius pallipes in Britain}. Biol. Conserv. 75:239-246.

\bibitem{Souty-Grosset 2006} Souty-Grosset C., Holdich D. M., Noël P. Y., Reynolds J. D., Haffner P., 2006. \emph{Atlas of Crayfish in Europe}. In. Paris: Muséum national d'Histoire naturelle 187.

\bibitem{Souty-Grosset 2009} Souty-Grosset C., Reynolds J.D, 2009. \emph{Current ideas on methodological approaches in European crayfish conservation and restocking procedures}. Knowl. Managt. Aquatic Ecosyst., 394-395. DOI: https://doi.org/10.1051/kmae/2009021.

\bibitem{Spink 2000} Spink J., FraylingM., 2000. \emph{An assessment of post-plague reintroduced native white-clawed crayfish Austropotamobius pallipes in the Sherston Avon and Tetbury Avon, Wiltshire}. Freshwater Forum, 14:59-69.

\bibitem{Stefani 2011} Stefani F., Zaccara S., Delmastro G. B., Buscarino M., 2011. \emph{The endangered white- clawed crayfish Austropotamobius pallipes (Decapoda, Astacidae) east and west of the Maritime Alps: a result of human translocation?} Conserv. Genet. 12: 51- 60.

\bibitem{Svoboda 2017} Svoboda J., Mrugała A., Kozubíková-Balcarová E., Petrusek, A., 2017. \emph{Hosts and transmission of the crayfish plague pathogen Aphanomyces astaci: a review}. J Fish Dis, 40: 127-140. doi:10.1111/jfd.12472

\bibitem{Tomasi 2014} Tomasi M., Odasso M., 2014. \emph{Linee guida per la gestione degli habitat di interesse comunitario presenti in Trentino con allegata codifica di un set standardizzato di misure di gestione e definizione dei relativi costi unitari}. Azione A6 del Progetto LIFE+TEN(LIFE11/NAT/IT000187).

\bibitem{Trentini 2014} Trentini G.,  Fossi G., 2014. \emph{Linee guida per la gestione della vegetazione lungo i corsi d'acqua in Trentino}. Azione A7 del Progetto LIFE+TEN (LIFE11/NAT/IT/000187).

\bibitem{Trontelj 2005} Trontelj P., Machino Y., Sket B., 2005. \emph{Phylogenetic and phylogeographic relationships in the crayfish genus Austropotamobius inferred from mitochondrial COI gene sequences}. Mol. Phylogenet. Evol. 34:212-226.

\bibitem{Trouilhe 2007} Trouilhè M. C., Souty-Grosset C., Grandjean F., Parinet B., 2007. \emph{Physical and chemical water requirements of the whiteclawed crayfish (Austropotamobius pallipes) in western France}. Aquat. Conserv. 17:520-538. 

\bibitem{Veneto Agricoltura 2011} Veneto Agricoltura, 2011. \emph{Manuale per la gestione ambientale dei corsi d'acqua a supporto dei Consorzi di bonifica}. \url{http://www.venetoagricoltura.org/upload/pubblicazioni/Manuale% 20Gestione% 20Ambientale%20E418/Manuale%20Gestione%20Acque%20Web.pdf}.

\bibitem{Vernesi 2016} Vernesi V., Hoban S.M., Pecchioli E., Crestanello B., Bertorelle G., Rosà R., Hauffe H.C. 2016. \emph{Ecology, environment and evolutionary history influence genetic structure in five mammal species from the Italian Alps}. Biological Journal of the Linnean Society 117:428-446. 

\bibitem{Vezza 2014} Vezza P., Parasiewicz P., Spairani M., Comoglio C., 2014. \emph{Habitat modelling in high gradient streams: the meso-scale approach and application}. Ecological Applications 24:844-861.

\bibitem{Vezza 2017} Vezza P., Zanin A. , Parasiewicz P., 2017. \emph{Manuale tecnico-operativo per la modellazione e la valutazione dell'integrità dell'habitat fluviale}. ISPRA, Serie Manuali e linee guida, 154/2017.

\bibitem{Vogt 1999} Vogt G., 1999. \emph{Diseases of European freshwater crayfish, with particular emphasis on interspecific transmission of pathogens}. In: Crayfish in Europe as Alien Species: How to Make the Best of a Bad Situation, GHERARDI F. and HOLDICH D.M. (Eds.), 87-103. A.A. Balkema Publishers, Rotterdam.

\bibitem{Vogt 2012} Vogt G., 2012. \emph{Ageing and longevity in the Decapoda (Crustacea): a review}. Zool. Anz. 251:1-25.

\bibitem{Zaccara 2004} Zaccara S., Stefani F., Galli P., Nardi P. A., Crosa G., 2004. \emph{Taxonomic implications in conservation management of whiteclawed crayfish (Austropotamobius pallipes) (Decapoda, Astacidae) in Northern Italy}. Biol. Conserv. 120:1-10.

\bibitem{Zanetti 2014} Zanetti M., 2014a. \emph{Gestione delle specie di gamberi alloctone invasive in Friuli Venezia Giulia. In: Manuale per le pubbliche amministrazioni. La gestione consapevole dei gamberi di fiume in Friuli Venezia Giulia. Pubblicazione realizzata con il contributo finanziario della CE, nell'ambito del progetto RARITY, LIFE10 NAT/IT/000239}. Editing testi Tiziano Scovacricchi, pp. 9-16.

\bibitem{Zanetti 2014b} Zanetti M., A. Rucli F., 2014b. \emph{Contrasto alla diffusione del gambero rosso della Louisiana Procambarus clarkii. In: Manuale per le pubbliche amministrazioni. La gestione consapevole dei gamberi di fiume in Friuli Venezia Giulia. Pubblicazione realizzata con il contributo finanziario della CE, nell'ambito del progetto RARITY, LIFE10 NAT/IT/000239}. Editing testi Tiziano Scovacricchi, pp 29-32.

\bibitem{Zanetti 2014c} Zanetti M., Rucli A., Aquiloni L., 2014. \emph{Protocolli di risposta rapida (EDRR- Early Detection Rapid Response). In: "RARITY. Eradicazione del gambero rosso della Louisiana e protezione dei gamberi di fiume del Friuli Venezia Giulia"}. Pubblicazione realizzata con il contributo finanziario della CE, nell'ambito del progetto RARITY, LIFE10 NAT/IT/000239, pp 55-57.

\bibitem{Zanetti 2014 d} Zanetti M., Aquiloni L., 2014. \emph{Approccio integrato di trappolaggio intensivo ed SMRT per il controllo di Procambarus clarkii nel sito di Casette. In: "RARITY. Eradicazione del gambero rosso della Louisiana e protezione dei gamberi di fiume del Friuli Venezia Giulia"}. Pubblicazione realizzata con il contributo finanziario della CE, nell'ambito del progetto RARITY, LIFE10 NAT/IT/000239, pp. 144.

\item[]\hspace{-\labelwidth}\hspace{-\labelsep}\textbf{Sitografia}

\bibitem{Adige} \url{www.ladige.it/territori/trento/2016/08/01/insetticida-rio-valsorda-moria-gamberi-acque-vigolo-vattaro-mattar}
\bibitem{CRAINAT} \url{www.lifecrainat.it/progetto.html}
\bibitem{Formulari Natura 2000} \url{www.areeprotette.provincia.tn.it}
\bibitem{RARITY} \url{www.life-rarity.eu}

\end{thebibliography}


\newpage
\pagestyle{empty}
\DeactivateBG
\chapter{Allegati}
\newpage
\invisiblesection{IA - Scheda di campo per caratterizzazione ambienti lentici}
\includepdf[pages=-]{all/Allegato_IA.pdf}
\newpage
\invisiblesection{IB - Scheda di campo per caratterizzazione ambienti lotici}
\includepdf[pages=-]{all/Allegato_IB.pdf}
\newpage
\invisiblesection{II - Scheda di segnalazione}
\includepdf[pages=-]{all/Allegato_II.pdf}
\newpages
\invisiblesection{III - Scheda di censimento}
\includepdf[pages=-,angle=90]{all/Allegato_III.pdf}
\newpage
\invisiblesection{IV - Scheda di monitoraggio}
\includepdf[pages=-,angle=90]{all/Allegato_IV.pdf}
\newpage
\section{V - Misure urgenti in siti Natura 2000 e/o altri siti identificati come prioritari per la conservazione di \emph{A. pallipes}}
L’analisi incrociata dei dati di distribuzione e habitat presentati nel piano di gestione e dell’inventario delle azioni di tutela attiva e di ricostruzione della connettività negli Ambiti Territoriali Omogenei in cui viene suddiviso il territorio provinciale (Azione C2 progetto LIFE + T.E.N.) ha permesso di identificare, per ogni ATO e/o bacino idrologico, gli interventi strutturali e operativi da mettere in atto parallelamente alle attività di monitoraggio indicate dal Piano, a fini conservazionistici. Tali attività verranno di seguito illustrate divise per bacino idrologico e indicando l’ATO di appartenenza.

\subsection{ATO Valsugana}
\subsubsection{Bacino Brenta}
Con asterisco sono indicati i siti inclusi nelle attività previste dall’azione C2 del progetto LIFE +T.E.N., rappresentate da  monitoraggi e reintroduzioni. Il costo totale dell’azione si esplica quindi in 20.000 € per il monitoraggio e in 20.000 € per progetti di reintroduzione, utilizzando come fonte di finaziamento il Piano di Sviluppo Rurale 2014-2020.\\

\begin{description}\itemsep0pt
\item[Lago Levico] Tratto terminale in prossimità dell'affluente del Rio Vignola e tratto iniziale del ramo del Brenta di Levico (area protetta inclusa nel sito: *SIC -IT3120039 Canneto di Levico). Presente la specie \emph{Orconectes limosus}.

\item[Lago di Caldonazzo] e tratto iniziale del ramo Brenta di Caldonazzo (area protetta inclusa nel sito: *Biotopo non isituito Caldonazzo-Brenta, *SIC IT3120042 Canneto di San Cristoforo). Presente la specie \emph{Orconectes limosus}.

\textbf{AZIONI PROPOSTE}\\

\begin{itemize}\itemsep0pt 
  \item contenimento diffusione \emph{O. limosus} su rami Brenta Caldonazzo e Brenta Levico, e su tributari (Rio Vignola, Rio San Cristoforo);
  \item prevenzione traslocazioni e diffusione \emph{A. astaci}: installlazione di pannelli informativi sul campo per pescatori e visitatori (contenenti informazioni su gamberi alloctoni, \emph{A. astaci}, importanza di prevenire tralsocazioni di animali e diffusione della peste del gambero, metodi di disinfezione e attrezzatura)
\end{itemize}
\item[Stagno artificiale nel parco della stazione ferroviaria di Grigno] Presente \emph{A. pallipes} con una popolazione abbondante.

\textbf{AZIONI PROPOSTE}\\
Utilizzo come stagno multifunzionale / \emph{source area} per allevamenti in incubatoio di valle.

\item[Rio Laguna] Presente \emph{A. pallipes}, con una popolazione in contrazione.

\textbf{AZIONI PROPOSTE}\\

Ripopolamento naturale dallo stagno di Grigno. Rinaturalizzazione tratto iniziale (in uscita dallo stagno) con creazione di rifugi.

\item[Fontanazzo] (area protetta inclusa nel sito: *ZPS-IT3120030 Fontanazzo). \emph{A. pallipes} estinto dal sito in tempi recenti, ma ottima qualità dell' habitat.

\textbf{AZIONI PROPOSTE}\\

Reintroduzione di \emph{A. pallipes} dalle popolazioni dello stagno di Grigno allevate in incubatoi di valle. Limitazione o contingentamento dell'immissione di specie ittiche predatrici.

\item[Torrente Resenzuola e Rio Vena-Inghiaie] (area protetta inclusa nel sito: *ZPS- ZSC IT3120038 Inghiaie; * ZSC- IT3120029 Sorgente del Resenzuola). \emph{A. pallipes} estinto in tempi recenti, ma ottima qualità dell'habitat.

\textbf{AZIONI PROPOSTE}\\

Reintroduzione di \emph{A. pallipes} dalla popolazione dello stagno di Grigno allevata in incubatoi di valle.
\end{description}
\subsubsection{Bacino Fersina}
\begin{description}\itemsep0pt
  \item [SIC \texttt{IT3120041} - Lago Costa] \emph{A. pallipes} estinto in tempi recenti. 

  \textbf{AZIONI PROPOSTE}\\

  Contenimento dell'ingresso dell' \emph{O. limosus} dai laghi di Canzolino e Madrano. 

  \item[Laghi di Canzolino Madrano]. Presente l'\emph{O. limosus}. 

  \textbf{AZIONI PROPOSTE}

  \begin{itemize}\itemsep0pt
    \item contenimento diffusione dell' \emph{O. limosus} al Lago Costa
    \item prevenzione delle traslocazioni e della diffusione di \emph{A. astaci} tramite installlazione di pannelli informativi sul campo per pescatori e visitatori contenenti informazioni su gamberi alloctoni, \emph{A. astaci}, importanza della prevenzione delle traslocazioni di animali e diffusione della peste del gambero, metodi di disinfezione dell'attrezzatura.
  \end{itemize}
  
  \item[Rio Nero] \emph{A. pallipes} presente ma in forte contrazione.

  \textbf{AZIONI PROPOSTE}\\
  
  Garantire il Deflusso Minimo Vitale, gestione migliore degli svasi dal bacino artificiale (floitare il sedimento con più portata). Qualora la specie recuperasse in densità, da utilizzare per allevamenti/reintroduzioni sia nel Fersina che in sinistra Adige (si veda par. 3.3.4).
  
  \item[Rio Farinella]. Da utilizzare per allevamenti/reintroduzioni sia nel Fersina che in sinistra Adige (si veda par. 3.3.4).

  \item[Altri siti sorgenti per reintroduzioni] Lago Santo di Cembra, Rio Fornei, Rio Carpine, Roggia di Gardolo (si veda par. 3.3.4).
\end{description}

\subsection{ATO Fiume Noce}
\subsubsection{Bacino Noce} 
Con asterisco sono indicati i siti inclusi nelle attività previste dall’azione C2 del progetto LIFE TEN+, rappresentate da monitoraggi. Costo totale dell’azione: 4000 € + IVA. Fonte di finaziamento: fondi provinciali.

\begin{description}\itemsep0pt
  \item[*ZSC- \texttt{IT3120060} - Forra di Santa Giustina] Presente \emph{A. pallipes}.

  \textbf{AZIONI PROPOSTE}\\

  Utilizzo della popolazione presente come sorgente di adulti da avviare ad allevamenti per successivi  ripopolamenti del tratto a valle, inclusa la *ZPS \texttt{IT3120061} - La Rocchetta. Necessaria preventiva indagine genetica e sanitaria, in quanto l’unica altra popolazione esistente a valle (Mezzolombardo, si veda sito seguente)  è risultata affetta da una forma non letale di \emph{A. astaci}. Mantenere il DMV dalla diga di Santa Giustina e la qualità ambientale e diversificazione substrati ad oggi presente (si veda la relazione FEM riguardo al SIC).

  \item[Laghetto di Mezzocorona] Presente \emph{A. pallipes}. 

  \textbf{AZIONI PROPOSTE}\\

  Gestione naturalistica del laghetto tramite introduzione di fasce di vegetazione sulle sponde, come rifugio, prevenzione del bracconaggio e monitoraggio della possibile presenza di \emph{A. astaci}. Inserimento di pannnello informativo sulla peste gambero, divieto traslocazioni e disinfezione attrezzatura.
\end{description}

\subsection{ATO Fiume Sarca} 
\subsubsection{Bacino Sarca}
Con asterisco è indicato il sito incluso nelle attività previste dall’azione C2 del progetto LIFE + T.E.N., rappresentate da Riqualificazione ambientale della Riserva Locale “Rio Folon”. Il costo totale dell’azione sarà di 40000 €, utilizzando come fonte di finaziamento il Piano di Sviluppo Rurale 2014-2020. Da considerare che l’azione C2 prevede l'eradicazione di gamebri alieni dal Rio Folon e successive reintroduzioni di popolazioni autoctone, ma nel Rio Folon NON sono presenti specie aliene, bensì esclusivamente una popolazione di \emph{A. pallipes}.

\begin{description}
  \item[* Riserva locale Rio Folon] Presente \emph{A. pallipes}.

  \textbf{AZIONI PROPOSTE}\\

  Riqualificazione.

  \item[Roggia di Vezzano] Presente \emph{A. pallipes}.

  \textbf{AZIONI PROPOSTE}\\

  \begin{itemize}\itemsep0pt
    \item Gestione compatibile del laghetto di pesca (Laghetto di Vezzano) a monte (prevenire svuotamenti, che rappresentano la pratica abituale)
    \item riqualificazione roggia nel  tratto monte e valle del laghetto, rimuovendo corazzamenti artificiali dell’alveo e ripristinando rifugi e creando zona riparia per ombreggiatura.
  \end{itemize}

  \item[Torrente Dal] Impianto ittico Scalfi, Torrente Ponale - troticoltura, Rio Cioc - Troticoltura, Rio El Pison - troticoltura. Presente \emph{A. pallipes}

  \textbf{AZIONI PROPOSTE} \\

  Indicazioni gestione compatibile, possibile utilizzo per allevamenti/reintroduzioni previa analisi genetica e patologica.

  \item[Piscicoltura Noselli] Il proprietario + disponibile per l'allevamento di gamberi autoctoni e per la creazione di un centro educativo.

  \item[Lago Lagolo] Presente \emph{Procambarus clarkii}.

  \textbf{AZIONI PROPOSTE}\\

  Eradicazione (da svolgere in collaborazione tra Rete Riserve Sarca e rete Bondone, utilizzando i protocolli LIFE RARITY)
\end{description}
\subsection{ATO Val di Fassa}
\subsubsection{Bacino Avisio}

\begin{description}
  \item[Rosta Cavazal - Cavalese] Presente \emph{A. pallipes}.

  \textbf{AZIONI PROPOSTE}\\

  Gestione e riqualificazione fosso (rischio interramento per cause antropiche), o traslocazione della popolazione.
  Siti da sfruttare come sorgenti per le reintroduzioni possono essere il Lago Santo di Cembra, il Rio Fornei, il Rio Farinella, il Rio Carpine e la Roggia di Gardolo (si veda 3.3.4).
\end{description}

\newpage
\section{VI - Sunto Azioni}
% Please add the following required packages to your document preamble:
% \usepackage{booktabs}
% \usepackage{longtable}
% Note: It may be necessary to compile the document several times to get a multi-page table to line up properly
\begin{longtable}[c]{@{}p{.2\columnwidth}p{.25\columnwidth}p{.1\columnwidth}p{.2\columnwidth}p{.2\columnwidth}@{}}
\toprule
{\cellcolor{white}}\textbf{Azioni} & \textbf{Attività} & \textbf{Priorità} & \textbf{Frequenza} & \textbf{Operatore / competenze} \\* \midrule
\endfirsthead
\multicolumn{5}{l}{\footnotesize Continua dalla pagina precedente}\\
\toprule
\textbf{Azioni} & \textbf{Attività} & \textbf{Priorità} & \textbf{Frequenza} & \textbf{Operatore / competenze} \\* \midrule
\endhead
1. Segnalazione di presenza di gamberi. & Avvistamento di gamberi e segnalazione tramite: 1) applicazione per dispositivi; 2) scheda di segnalazione. & Alta & Continua & Corpo Forestale, custodi forestali, associazioni pescatori (guardiapesca), cittadini. \\
\rowcolor[HTML]{EFEFEF}2. Censimento delle specie presenti in Trentino (presenza/assenza) (cfr tab. 9, colonna “censimenti”) & 1) Scelta dei siti da monitorare (da analisi cartografica) 2) rilievo di dati di presenza , scheda di censimento & Alta & 1 censimento per sito ogni 1 anno (si veda tab. 9) & 1 tecnico formato sui metodi di censimento previsti dal piano di gestione  + 1 assistente da campo. \\
3. Campionamento delle popolazioni per valutazione dello stato di conservazione (densità e struttura) (cfr colonna “campionamento delle popolazioni” in tab. 9). & 1) Campionamento delle popolazioni secondo metodo CMR o CPUE con cadenza triennale. Rilievo di dati secondo quanto previsto dalle scheda di monitoraggio e in caso di popolazioni nuove, reintrodotte o in declino, raccolta di campioni biologici per l'analisi genetica & Alta & 1 campionamento per sito ogni 3 (1) anni & 1 tecnico formato sui metodi di campionamento previsti dal piano di gestione  + 1 assistente da campo. \\
\rowcolor[HTML]{EFEFEF}4. Monitoraggio degli habitat per individuazione di: caratteristiche preferenziali per il gambero, aree rifugio, di rischio, corridoi di diffusione, siti da riqualificare (cfr colonna “monitoraggio degli habitat” in tab. 9). & Rilievo di dati ambientali nei siti di censimento e campionamento delle popolazioni (scheda di valutazione habitat lentici e lotici) & Alta & 1 monitoraggio per sito ogni 3 (1) anni , in occasione dei campionamenti & 1 tecnico formato sui metodi di rilievo degli habitat previsti dal piano di gestione  + 1 assistente da campo. \\
5. Studi di cattura-marcatura-ricattura a lungo termine: valutazione di parametri demografici in popolazioni selezionate & Solo per popolazioni selezionate: sessioni di cattura, marcatura tramite pit-tags e identificazione con lettore ottico. & Bassa & Più sessioni di cattura nel corso della stagione e di anni consecutivi. & 1 ricercatore + 1 tecnico formato sui metodi di campionamento. \\
\rowcolor[HTML]{EFEFEF}6. Contenimento della diffusione di specie aliene invasive & Eradicazioni delle popolazioni non stabilizzate & Alta & quando necessario & 1 tecnico formato sui metodi di eradicazione + 1 assistente da campo. \\
Controllo della diffusione delle popolazioni stabilizzate & Alta & annuale & 1 tecnico formato sui metodi di monitoraggio e di contenimento  + 1 assistente da campo. &  \\* \bottomrule
\end{longtable}
\newpage
\section{VII - Azioni nei vari trienni}
\footnoteslsize
\begin{landscape}
\begin{longtable}[c]{@{}p{.1\columnwidth}p{.008\columnwidth}p{.008\columnwidth}p{.008\columnwidth}p{.008\columnwidth}p{.008\columnwidth}p{.008\columnwidth}p{.008\columnwidth}p{.008\columnwidth}p{.008\columnwidth}p{.008\columnwidth}p{.008\columnwidth}p{.008\columnwidth}|p{.008\columnwidth}p{.008\columnwidth}p{.008\columnwidth}p{.008\columnwidth}p{.008\columnwidth}p{.008\columnwidth}p{.008\columnwidth}p{.008\columnwidth}p{.008\columnwidth}p{.008\columnwidth}p{.008\columnwidth}p{.008\columnwidth}|p{.008\columnwidth}p{.008\columnwidth}p{.008\columnwidth}p{.008\columnwidth}p{.008\columnwidth}p{.008\columnwidth}p{.008\columnwidth}p{.008\columnwidth}p{.008\columnwidth}p{.008\columnwidth}p{.008\columnwidth}p{.008\columnwidth}p{.008\columnwidth}p{.008\columnwidth}@{}}
\caption{Primo triennio} \\
\toprule
{\cellcolor{white}}\textbf{Azioni} & \multicolumn{12}{c}{\textbf{2018}} & \multicolumn{12}{c}{\textbf{2019}} & \multicolumn{11}{c}{\textbf{2020}} &  &  &  \\* \midrule
\endfirsthead
\multicolumn{37}{l}{\footnotesize Continua dalla pagina precedente}\\
\toprule
\textbf{Azioni} & \multicolumn{12}{c}{\textbf{2018}} & \multicolumn{12}{c}{\textbf{2019}} & \multicolumn{11}{c}{\textbf{2020}} &  &  &  \\* \midrule
\midrule
\endhead
 & I & II & III & IV & V & VI & VII & VIII & IX & X & XI & XII & I & II & III & IV & V & VI & VII & VIII & IX & X & XI & XII & I & II & III & IV & V & VI & VII & VIII & IX & X & XI & XII &  &  \\
\rowcolor[HTML]{EFEFEF}  Censimento &  &  &  &  &  & $\bullet$ & $\bullet$ & $\bullet$ & $\bullet$ &  &  &  &  &  &  &  &  & $\bullet$ & $\bullet$ & $\bullet$ & $\bullet$ &  &  &  &  &  &  &  &  & $\bullet$ & $\bullet$ & $\bullet$ & $\bullet$ &  &  &  &  &  \\
Campionamento popolazioni &  &  &  &  &  & $\bullet$ & $\bullet$ & $\bullet$ & $\bullet$ &  &  &  &  &  &  &  &  & $\circ$ & $\circ$ & $\circ$ & $\circ$ &  &  &  &  &  &  &  &  & $\circ$ & $\circ$ & $\circ$ & $\circ$ &  &  &  &  &  \\
\rowcolor[HTML]{EFEFEF} Monitoraggio degli habitat &  &  &  &  &  & $\bullet$ & $\bullet$ & $\bullet$ & $\bullet$ &  &  &  &  &  &  &  &  & $\bullet$ & $\bullet$ & $\bullet$ & $\bullet$ &  &  &  &  &  &  &  &  & $\bullet$ & $\bullet$ & $\bullet$ & $\bullet$ &  &  &  &  &  \\
Studi di cattura-marcatura-ricattura: valutazione di parametri demografici in popolazioni selezionate &  &  &  &  &  &  &  &  &  &  &  &  &  &  &  &  &  & $\circ$ & $\circ$ & $\circ$ & $\circ$ &  &  &  &  &  &  &  &  & $\circ$ & $\circ$ & $\circ$ & $\circ$ &  &  &  &  &  \\
\rowcolor[HTML]{EFEFEF} Controllo diffusione alloctone &  &  &  &  &  & $\bullet$ & $\bullet$ & $\bullet$ & $\bullet$ &  &  &  &  &  &  &  &  & $\bullet$ & $\bullet$ & $\bullet$ & $\bullet$ &  &  &  &  &  &  &  &  & $\bullet$ & $\bullet$ & $\bullet$ & $\bullet$ &  &  &  &  &  \\
Eradicazione popolazioni alloctone &  &  &  &  &  & $\bullet$ & $\bullet$ & $\bullet$ & $\bullet$ &  &  &  &  &  &  &  &  & $\circ$ & $\circ$ & $\circ$ & $\circ$ &  &  &  &  &  &  &  &  & $\circ$ & $\circ$ & $\circ$ & $\circ$ &  &  &  &  &  \\
\rowcolor[HTML]{EFEFEF} Caratterizzazione genetica: DNA nucleare campioni già disponibili &  & $\bullet$ & $\bullet$ & $\bullet$ & $\bullet$ &  &  &  &  &  &  &  &  &  &  &  &  &  &  &  &  &  &  &  &  &  &  &  &  &  &  &  &  &  &  &  &  &  \\
Caratterizzazione genetica di nuove popolazioni rilevate &  &  &  &  &  &  &  &  &  &  &  &  &  & $\circ$ & $\circ$ & $\circ$ & $\circ$ &  &  &  &  &  &  &  &  & $\circ$ & $\circ$ & $\circ$ & $\circ$ &  &  &  &  &  &  &  &  &  \\
\rowcolor[HTML]{EFEFEF} Analisi dei dati &  &  &  &  &  &  &  &  &  & $\bullet$ & $\bullet$ & $\bullet$ &  &  &  &  &  &  &  &  &  & $\bullet$ & $\bullet$ & $\bullet$ &  &  &  &  &  &  &  &  &  & $\bullet$ & $\bullet$ & $\bullet$ &  &  \\
Creazione database & $\bullet$ & $\bullet$ &  &  &  &  &  &  &  &  &  &  &  &  &  &  &  &  &  &  &  &  &  &  &  &  &  &  &  &  &  &  &  &  &  &  &  &  \\
\rowcolor[HTML]{EFEFEF} Creazione sito web e sistemi segnalazione & $\bullet$ & $\bullet$ &  &  &  &  &  &  &  &  &  &  &  &  &  &  &  &  &  &  &  &  &  &  &  &  &  &  &  &  &  &  &  &  &  &  &  &  \\
Gestione database e sito web &  &  &  &  &  &  &  &  &  & $\diamond$ & $\diamond$ & $\diamond$ & $\diamond$ & $\diamond$ &  &  &  &  &  &  &  & $\diamond$ & $\diamond$ & $\diamond$ & $\diamond$ & $\diamond$ &  &  &  &  &  &  &  & $\diamond$ & $\diamond$ & $\diamond$ &  &  \\
\rowcolor[HTML]{EFEFEF} Formazione e sensibilizzazione di pescatori, gestori impianti ittici, fruitori risorsa idrica & $\diamond$ & $\diamond$ & $\diamond$ & $\diamond$ & $\diamond$ & $\diamond$ & $\diamond$ & $\diamond$ & $\diamond$ & $\diamond$ & $\diamond$ & $\diamond$ & $\diamond$ & $\diamond$ & $\diamond$ & $\diamond$ & $\diamond$ & $\diamond$ & $\diamond$ & $\diamond$ & $\diamond$ & $\diamond$ & $\diamond$ & $\diamond$ & $\diamond$ & $\diamond$ & $\diamond$ & $\diamond$ & $\diamond$ & $\diamond$ & $\diamond$ & $\diamond$ & $\diamond$ & $\diamond$ & $\diamond$ & $\diamond$ &  &  \\
Divulgazione al pubblico generico &  &  &  &  &  & $\circ$ & $\circ$ & $\circ$ & $\circ$ &  &  &  &  &  &  &  &  & $\circ$ & $\circ$ & $\circ$ & $\circ$ &  &  &  &  &  &  &  &  & $\circ$ & $\circ$ & $\circ$ & $\circ$ &  &  &  &  &  \\
\rowcolor[HTML]{EFEFEF} Divulgazione scientifica & $\diamond$ & $\diamond$ & $\diamond$ & $\diamond$ & $\diamond$ & $\diamond$ & $\diamond$ & $\diamond$ & $\diamond$ & $\diamond$ & $\diamond$ & $\diamond$ & $\diamond$ & $\diamond$ & $\diamond$ & $\diamond$ & $\diamond$ & $\diamond$ & $\diamond$ & $\diamond$ & $\diamond$ & $\diamond$ & $\diamond$ & $\diamond$ & $\diamond$ & $\diamond$ & $\diamond$ & $\diamond$ & $\diamond$ & $\diamond$ & $\diamond$ & $\diamond$ & $\diamond$ & $\diamond$ & $\diamond$ & $\diamond$ &  &  \\
Contenimento di patogeni: preparazione testi e posizionamento pannelli & $\bullet$ & $\bullet$ & $\bullet$ & $\bullet$ & $\bullet$ &  &  &  &  &  &  &  &  &  &  &  &  &  &  &  &  &  &  &  &  &  &  &  &  &  &  &  &  &  &  &  &  &  \\
\rowcolor[HTML]{EFEFEF} Allevamenti: individuazione siti e creazione impianti &  &  &  &  &  &  &  &  &  & $\bullet$ & $\bullet$ & $\bullet$ & $\bullet$ & $\bullet$ & $\bullet$ &  &  &  &  &  &  &  &  &  &  &  &  &  &  &  &  &  &  &  &  &  &  &  \\
Allevamenti: raccolta riproduttori &  &  &  &  &  &  &  &  &  &  &  &  &  &  &  &  &  &  &  & $\bullet$ & $\bullet$ & $\bullet$ &  &  &  &  &  &  &  &  &  & $\bullet$ & $\bullet$ & $\bullet$ &  &  &  &  \\
\rowcolor[HTML]{EFEFEF} Ripopolamenti e reintroduzioni &  &  &  &  &  &  &  &  &  &  &  &  &  &  &  &  &  &  &  &  & $\bullet$ & $\bullet$ &  &  &  &  &  &  &  &  &  &  & $\bullet$ & $\bullet$ &  &  &  &  \\
Rinaturalizzazione dei corpi idrici effettivamente occupati o potenzialmente idonei alla diffusione di \emph{A. pallipes}: Interventi caso-specifici di rinaturalizzazione, ripristino della connettività longitudinale, creazione ambienti umidi, creazione di aree rifugio in ambienti acquatici laterali. & $\diamond$ & $\diamond$ & $\diamond$ & $\diamond$ & $\diamond$ & $\diamond$ & $\diamond$ & $\diamond$ & $\diamond$ & $\diamond$ & $\diamond$ & $\diamond$ & $\diamond$ & $\diamond$ & $\diamond$ & $\diamond$ & $\diamond$ & $\diamond$ & $\diamond$ & $\diamond$ & $\diamond$ & $\diamond$ & $\diamond$ & $\diamond$ & $\diamond$ & $\diamond$ & $\diamond$ & $\diamond$ & $\diamond$ & $\diamond$ & $\diamond$ & $\diamond$ & $\diamond$ & $\diamond$ & $\diamond$ & $\diamond$ \\
\rowcolor[HTML]{EFEFEF} Gestione sostenibile della vegetazione acquatica e riparia negli ambienti in cui sono previsti tagli periodici & $\diamond$ & $\diamond$ & $\diamond$ & $\diamond$ & $\diamond$ & $\diamond$ & $\diamond$ & $\diamond$ & $\diamond$ & $\diamond$ & $\diamond$ & $\diamond$ & $\diamond$ & $\diamond$ & $\diamond$ & $\diamond$ & $\diamond$ & $\diamond$ & $\diamond$ & $\diamond$ & $\diamond$ & $\diamond$ & $\diamond$ & $\diamond$ & $\diamond$ & $\diamond$ & $\diamond$ & $\diamond$ & $\diamond$ & $\diamond$ & $\diamond$ & $\diamond$ & $\diamond$ & $\diamond$ & $\diamond$ & $\diamond$ &  &  \\
Mitigazione di inquinamento diffuso: creazione di una zona filtro tra corpi idrici e ambiente acquatico. & $\diamond$ & $\diamond$ & $\diamond$ & $\diamond$ & $\diamond$ & $\diamond$ & $\diamond$ & $\diamond$ & $\diamond$ & $\diamond$ & $\diamond$ & $\diamond$ & $\diamond$ & $\diamond$ & $\diamond$ & $\diamond$ & $\diamond$ & $\diamond$ & $\diamond$ & $\diamond$ & $\diamond$ & $\diamond$ & $\diamond$ & $\diamond$ & $\diamond$ & $\diamond$ & $\diamond$ & $\diamond$ & $\diamond$ & $\diamond$ & $\diamond$ & $\diamond$ & $\diamond$ & $\diamond$ & $\diamond$ & $\diamond$ &  &  \\
\rowcolor[HTML]{EFEFEF} Eliminazione di fonti d'inquinamento puntuali. & $\diamond$ & $\diamond$ & $\diamond$ & $\diamond$ & $\diamond$ & $\diamond$ & $\diamond$ & $\diamond$ & $\diamond$ & $\diamond$ & $\diamond$ & $\diamond$ & $\diamond$ & $\diamond$ & $\diamond$ & $\diamond$ & $\diamond$ & $\diamond$ & $\diamond$ & $\diamond$ & $\diamond$ & $\diamond$ & $\diamond$ & $\diamond$ & $\diamond$ & $\diamond$ & $\diamond$ & $\diamond$ & $\diamond$ & $\diamond$ & $\diamond$ & $\diamond$ & $\diamond$ & $\diamond$ & $\diamond$ & $\diamond$ &  &  \\* \bottomrule
\end{longtable}
\end{landscape}
% Please add the following required packages to your document preamble:
% \usepackage{booktabs}
\begin{landscape}
\begin{longtable}[c]{@{}p{.1\columnwidth}p{.008\columnwidth}p{.008\columnwidth}p{.008\columnwidth}p{.008\columnwidth}p{.008\columnwidth}p{.008\columnwidth}p{.008\columnwidth}p{.008\columnwidth}p{.008\columnwidth}p{.008\columnwidth}p{.008\columnwidth}p{.008\columnwidth}|p{.008\columnwidth}p{.008\columnwidth}p{.008\columnwidth}p{.008\columnwidth}p{.008\columnwidth}p{.008\columnwidth}p{.008\columnwidth}p{.008\columnwidth}p{.008\columnwidth}p{.008\columnwidth}p{.008\columnwidth}p{.008\columnwidth}|p{.008\columnwidth}p{.008\columnwidth}p{.008\columnwidth}p{.008\columnwidth}p{.008\columnwidth}p{.008\columnwidth}p{.008\columnwidth}p{.008\columnwidth}p{.008\columnwidth}p{.008\columnwidth}p{.008\columnwidth}p{.008\columnwidth}p{.008\columnwidth}p{.008\columnwidth}@{}}
\caption{Secondo triennio} \\
\toprule
\textbf{Azioni} & \multicolumn{12}{c}{\textbf{2024}} & \multicolumn{12}{c}{\textbf{2025}} & \multicolumn{12}{c}{\textbf{2026}} & \textbf{} & \textbf{} \\ \midrule
\endfirsthead
\multicolumn{37}{l}{\footnotesize Continua dalla pagina precedente}\\
\toprule
\textbf{Azioni} & \multicolumn{12}{c}{\textbf{2024}} & \multicolumn{12}{c}{\textbf{2025}} & \multicolumn{12}{c}{\textbf{2026}} & \textbf{} & \textbf{} \\ \midrule
\endhead
\textbf{} & \textbf{I} & \textbf{II} & \textbf{III} & \textbf{IV} & \textbf{V} & \textbf{VI} & \textbf{VII} & \textbf{VIII} & \textbf{IX} & \textbf{X} & \textbf{XI} & \textbf{XII} & \textbf{I} & \textbf{II} & \textbf{III} & \textbf{IV} & \textbf{V} & \textbf{VI} & \textbf{VII} & \textbf{VIII} & \textbf{IX} & \textbf{X} & \textbf{XI} & \textbf{XII} & \textbf{I} & \textbf{II} & \textbf{III} & \textbf{IV} & \textbf{V} & \textbf{VI} & \textbf{VII} & \textbf{VIII} & \textbf{IX} & \textbf{X} & \textbf{XI} & \textbf{XII} & \textbf{} & \textbf{} \\
\rowcolor[HTML]{EFEFEF} Censimento &  &  &  &  &  & $\bullet$ & $\bullet$ & $\bullet$ & $\bullet$ &  &  &  &  &  &  &  &  & $\bullet$ & $\bullet$ & $\bullet$ & $\bullet$ &  &  &  &  &  &  &  &  & $\bullet$ & $\bullet$ & $\bullet$ & $\bullet$ &  &  &  &  &  \\
Campionamento popolazioni &  &  &  &  &  & $\bullet$ & $\bullet$ & $\bullet$ & $\bullet$ &  &  &  &  &  &  &  &  & $\circ$ & $\circ$ & $\circ$ & $\circ$ &  &  &  &  &  &  &  &  & $\circ$ & $\circ$ & $\circ$ & $\circ$ &  &  &  &  &  \\
\rowcolor[HTML]{EFEFEF} Monitoraggio degli habitat &  &  &  &  &  & $\bullet$ & $\bullet$ & $\bullet$ & $\bullet$ &  &  &  &  &  &  &  &  & $\bullet$ & $\bullet$ & $\bullet$ & $\bullet$ &  &  &  &  &  &  &  &  & $\bullet$ & $\bullet$ & $\bullet$ & $\bullet$ &  &  &  &  &  \\
Eradicazione popolazioni alloctone &  &  &  &  &  & $\circ$ & $\circ$ & $\circ$ & $\circ$ &  &  &  &  &  &  &  &  & $\circ$ & $\circ$ & $\circ$ & $\circ$ &  &  &  &  &  &  &  &  & $\circ$ & $\circ$ & $\circ$ & $\circ$ &  &  &  &  &  \\
\rowcolor[HTML]{EFEFEF} Controllo diffusione alloctone &  &  &  &  &  & $\bullet$ & $\bullet$ & $\bullet$ & $\bullet$ &  &  &  &  &  &  &  &  & $\circ$ & $\circ$ & $\circ$ & $\circ$ &  &  &  &  &  &  &  &  & $\circ$ & $\circ$ & $\circ$ & $\circ$ &  &  &  &  &  \\
Creazione e gestione database & $\bullet$ & $\bullet$ & $\bullet$ & $\bullet$ & $\bullet$ &  &  &  &  &  &  &  &  &  &  &  &  &  &  &  &  &  &  &  &  &  &  &  &  &  &  &  &  &  &  &  &  &  \\
\rowcolor[HTML]{EFEFEF} Analisi dei dati &  &  &  &  &  &  &  &  &  & $\bullet$ & $\bullet$ & $\bullet$ &  &  &  &  &  &  &  &  &  & $\bullet$ & $\bullet$ & $\bullet$ &  &  &  &  &  &  &  &  &  & $\bullet$ & $\bullet$ & $\bullet$ &  &  \\
Caratterizzazione genetica di nuove popolazioni rilevate &  &  &  &  &  &  &  &  &  &  &  &  &  & $\circ$ & $\circ$ & $\circ$ & $\circ$ &  &  &  &  &  &  &  &  & $\circ$ & $\circ$ & $\circ$ & $\circ$ &  &  &  &  &  &  &  &  &  \\
\rowcolor[HTML]{EFEFEF} Gestione database e sito web &  &  &  &  &  &  &  &  &  & $\diamond$ & $\diamond$ & $\diamond$ & $\diamond$ & $\diamond$ &  &  &  &  &  &  &  & $\diamond$ & $\diamond$ & $\diamond$ & $\diamond$ & $\diamond$ &  &  &  &  &  &  &  & $\diamond$ & $\diamond$ & $\diamond$ &  &  \\
Formazione e sensibilizzazione di pescatori & $\diamond$ & $\diamond$ & $\diamond$ & $\diamond$ & $\diamond$ & $\diamond$ & $\diamond$ & $\diamond$ & $\diamond$ & $\diamond$ & $\diamond$ & $\diamond$ & $\diamond$ & $\diamond$ & $\diamond$ & $\diamond$ & $\diamond$ & $\diamond$ & $\diamond$ & $\diamond$ & $\diamond$ & $\diamond$ & $\diamond$ & $\diamond$ & $\diamond$ & $\diamond$ & $\diamond$ & $\diamond$ & $\diamond$ & $\diamond$ & $\diamond$ & $\diamond$ & $\diamond$ & $\diamond$ & $\diamond$ & $\diamond$ &  &  \\
\rowcolor[HTML]{EFEFEF} Collaborazione con gestori di laghetti da pesca e pescicolture, soprattutto se ospitano popolazioni di \emph{A. pallipes} & $\diamond$ & $\diamond$ & $\diamond$ & $\diamond$ & $\diamond$ & $\diamond$ & $\diamond$ & $\diamond$ & $\diamond$ & $\diamond$ & $\diamond$ & $\diamond$ & $\diamond$ & $\diamond$ & $\diamond$ & $\diamond$ & $\diamond$ & $\diamond$ & $\diamond$ & $\diamond$ & $\diamond$ & $\diamond$ & $\diamond$ & $\diamond$ & $\diamond$ & $\diamond$ & $\diamond$ & $\diamond$ & $\diamond$ & $\diamond$ & $\diamond$ & $\diamond$ & $\diamond$ & $\diamond$ & $\diamond$ & $\diamond$ &  &  \\
Sensibilizzazione di gestori e fruitori della risorsa idrica & $\diamond$ & $\diamond$ & $\diamond$ & $\diamond$ & $\diamond$ & $\diamond$ & $\diamond$ & $\diamond$ & $\diamond$ & $\diamond$ & $\diamond$ & $\diamond$ & $\diamond$ & $\diamond$ & $\diamond$ & $\diamond$ & $\diamond$ & $\diamond$ & $\diamond$ & $\diamond$ & $\diamond$ & $\diamond$ & $\diamond$ & $\diamond$ & $\diamond$ & $\diamond$ & $\diamond$ & $\diamond$ & $\diamond$ & $\diamond$ & $\diamond$ & $\diamond$ & $\diamond$ & $\diamond$ & $\diamond$ & $\diamond$ &  &  \\
Divulgazione al pubblico generico &  &  &  &  &  & $\bullet$ & $\bullet$ & $\bullet$ & $\bullet$ &  &  &  &  &  &  &  &  & $\bullet$ & $\bullet$ & $\bullet$ & $\bullet$ &  &  &  &  &  &  &  &  & $\bullet$ & $\bullet$ & $\bullet$ & $\bullet$ &  &  &  &  &  \\
\rowcolor[HTML]{EFEFEF} Divulgazione scientifica & $\diamond$ & $\diamond$ & $\diamond$ & $\diamond$ & $\diamond$ & $\diamond$ & $\diamond$ & $\diamond$ & $\diamond$ & $\diamond$ & $\diamond$ & $\diamond$ & $\diamond$ & $\diamond$ & $\diamond$ & $\diamond$ & $\diamond$ & $\diamond$ & $\diamond$ & $\diamond$ & $\diamond$ & $\diamond$ & $\diamond$ & $\diamond$ & $\diamond$ & $\diamond$ & $\diamond$ & $\diamond$ & $\diamond$ & $\diamond$ & $\diamond$ & $\diamond$ & $\diamond$ & $\diamond$ & $\diamond$ & $\diamond$ &  &  \\
Contenimento di patogeni & $\diamond$ & $\diamond$ & $\diamond$ & $\diamond$ & $\diamond$ & $\diamond$ & $\diamond$ & $\diamond$ & $\diamond$ & $\diamond$ & $\diamond$ & $\diamond$ & $\diamond$ & $\diamond$ & $\diamond$ & $\diamond$ & $\diamond$ & $\diamond$ & $\diamond$ & $\diamond$ & $\diamond$ & $\diamond$ & $\diamond$ & $\diamond$ & $\diamond$ & $\diamond$ & $\diamond$ & $\diamond$ & $\diamond$ & $\diamond$ & $\diamond$ & $\diamond$ & $\diamond$ & $\diamond$ & $\diamond$ & $\diamond$ &  &  \\
\rowcolor[HTML]{EFEFEF} Allevamenti: individuazione siti e creazione impianti &  &  &  &  &  &  &  &  &  & $\bullet$ & $\bullet$ & $\bullet$ & $\bullet$ & $\bullet$ & $\bullet$ &  &  &  &  &  &  &  &  &  &  &  &  &  &  &  &  &  &  &  &  &  &  &  \\
Allevamenti: raccolta riproduttori &  &  &  &  &  &  &  &  &  &  &  &  &  &  &  &  &  &  &  & $\bullet$ & $\bullet$ & $\bullet$ &  &  &  &  &  &  &  &  &  & $\bullet$ & $\bullet$ & $\bullet$ &  &  &  &  \\
\rowcolor[HTML]{EFEFEF} Ripopolamenti e reintroduzioni &  &  &  &  &  &  &  &  &  &  &  &  &  &  &  &  &  &  &  &  & $\bullet$ & $\bullet$ &  &  &  &  &  &  &  &  &  &  & $\bullet$ & $\bullet$ &  &  &  &  \\
Rinaturalizzazione dei corpi idrici effettivamente occupati o potenzialmente idonei alla diffusione di \emph{A. pallipes} : Interventi caso-specifici di rinaturalizzazione ripristino della connettività longitudinale creazione ambienti umidi, creazione di aree rifugio in ambienti acquatici laterali. & $\diamond$ & $\diamond$ & $\diamond$ & $\diamond$ & $\diamond$ & $\diamond$ & $\diamond$ & $\diamond$ & $\diamond$ & $\diamond$ & $\diamond$ & $\diamond$ & $\diamond$ & $\diamond$ & $\diamond$ & $\diamond$ & $\diamond$ & $\diamond$ & $\diamond$ & $\diamond$ & $\diamond$ & $\diamond$ & $\diamond$ & $\diamond$ & $\diamond$ & $\diamond$ & $\diamond$ & $\diamond$ & $\diamond$ & $\diamond$ & $\diamond$ & $\diamond$ & $\diamond$ & $\diamond$ & $\diamond$ & $\diamond$ \\
\rowcolor[HTML]{EFEFEF} Gestione sostenibile della vegetazione acquatica e riparia negli ambienti in cui sono previsti tagli periodici & $\diamond$ & $\diamond$ & $\diamond$ & $\diamond$ & $\diamond$ & $\diamond$ & $\diamond$ & $\diamond$ & $\diamond$ & $\diamond$ & $\diamond$ & $\diamond$ & $\diamond$ & $\diamond$ & $\diamond$ & $\diamond$ & $\diamond$ & $\diamond$ & $\diamond$ & $\diamond$ & $\diamond$ & $\diamond$ & $\diamond$ & $\diamond$ & $\diamond$ & $\diamond$ & $\diamond$ & $\diamond$ & $\diamond$ & $\diamond$ & $\diamond$ & $\diamond$ & $\diamond$ & $\diamond$ & $\diamond$ & $\diamond$ &  &  \\
Mitigazione di inquinamento diffuso: creazione di una zona filtro tra corpi idrici e ambiente acquatico & $\diamond$ & $\diamond$ & $\diamond$ & $\diamond$ & $\diamond$ & $\diamond$ & $\diamond$ & $\diamond$ & $\diamond$ & $\diamond$ & $\diamond$ & $\diamond$ & $\diamond$ & $\diamond$ & $\diamond$ & $\diamond$ & $\diamond$ & $\diamond$ & $\diamond$ & $\diamond$ & $\diamond$ & $\diamond$ & $\diamond$ & $\diamond$ & $\diamond$ & $\diamond$ & $\diamond$ & $\diamond$ & $\diamond$ & $\diamond$ & $\diamond$ & $\diamond$ & $\diamond$ & $\diamond$ & $\diamond$ & $\diamond$ &  &  \\
\rowcolor[HTML]{EFEFEF} Eliminazione di fonti d'inquinamento puntuali & $\diamond$ & $\diamond$ & $\diamond$ & $\diamond$ & $\diamond$ & $\diamond$ & $\diamond$ & $\diamond$ & $\diamond$ & $\diamond$ & $\diamond$ & $\diamond$ & $\diamond$ & $\diamond$ & $\diamond$ & $\diamond$ & $\diamond$ & $\diamond$ & $\diamond$ & $\diamond$ & $\diamond$ & $\diamond$ & $\diamond$ & $\diamond$ & $\diamond$ & $\diamond$ & $\diamond$ & $\diamond$ & $\diamond$ & $\diamond$ & $\diamond$ & $\diamond$ & $\diamond$ & $\diamond$ & $\diamond$ & $\diamond$ &  &  \\
Studi di cattura-marcatura-ricattura: valutazione di parametri demografici in popolazioni selezionate & $\diamond$ & $\diamond$ & $\diamond$ & $\diamond$ & $\diamond$ & $\diamond$ & $\diamond$ & $\diamond$ & $\diamond$ & $\diamond$ & $\diamond$ & $\diamond$ & $\diamond$ & $\diamond$ & $\diamond$ & $\diamond$ & $\diamond$ & $\diamond$ & $\diamond$ & $\diamond$ & $\diamond$ & $\diamond$ & $\diamond$ & $\diamond$ & $\diamond$ & $\diamond$ & $\diamond$ & $\diamond$ & $\diamond$ & $\diamond$ & $\diamond$ & $\diamond$ & $\diamond$ & $\diamond$ & $\diamond$ & $\diamond$ &  &  \\ \bottomrule
\end{longtable}
\end{landscape}

\begin{landscape}
\begin{longtable}[c]{@{}p{.1\columnwidth}p{.008\columnwidth}p{.008\columnwidth}p{.008\columnwidth}p{.008\columnwidth}p{.008\columnwidth}p{.008\columnwidth}p{.008\columnwidth}p{.008\columnwidth}p{.008\columnwidth}p{.008\columnwidth}p{.008\columnwidth}p{.008\columnwidth}|p{.008\columnwidth}p{.008\columnwidth}p{.008\columnwidth}p{.008\columnwidth}p{.008\columnwidth}p{.008\columnwidth}p{.008\columnwidth}p{.008\columnwidth}p{.008\columnwidth}p{.008\columnwidth}p{.008\columnwidth}p{.008\columnwidth}|p{.008\columnwidth}p{.008\columnwidth}p{.008\columnwidth}p{.008\columnwidth}p{.008\columnwidth}p{.008\columnwidth}p{.008\columnwidth}p{.008\columnwidth}p{.008\columnwidth}p{.008\columnwidth}p{.008\columnwidth}p{.008\columnwidth}p{.008\columnwidth}p{.008\columnwidth}@{}}
\caption{Terzo triennio} \\
\toprule
\textbf{Azioni} & \multicolumn{12}{c}{\textbf{2024}} & \multicolumn{12}{c}{\textbf{2025}} & \multicolumn{12}{c}{\textbf{2026}} & \textbf{} & \textbf{} \\ \midrule
\endfirsthead
\multicolumn{37}{l}{\footnotesize Continua dalla pagina precedente}\\
\toprule
\textbf{Azioni} & \multicolumn{12}{c}{\textbf{2024}} & \multicolumn{12}{c}{\textbf{2025}} & \multicolumn{12}{c}{\textbf{2026}} & \textbf{} & \textbf{} \\ \midrule
\endhead
\textbf{} & \textbf{I} & \textbf{II} & \textbf{III} & \textbf{IV} & \textbf{V} & \textbf{VI} & \textbf{VII} & \textbf{VIII} & \textbf{IX} & \textbf{X} & \textbf{XI} & \textbf{XII} & \textbf{I} & \textbf{II} & \textbf{III} & \textbf{IV} & \textbf{V} & \textbf{VI} & \textbf{VII} & \textbf{VIII} & \textbf{IX} & \textbf{X} & \textbf{XI} & \textbf{XII} & \textbf{I} & \textbf{II} & \textbf{III} & \textbf{IV} & \textbf{V} & \textbf{VI} & \textbf{VII} & \textbf{VIII} & \textbf{IX} & \textbf{X} & \textbf{XI} & \textbf{XII} & \textbf{} & \textbf{} \\
\rowcolor[HTML]{EFEFEF} Censimento &  &  &  &  &  & $\bullet$ & $\bullet$ & $\bullet$ & $\bullet$ &  &  &  &  &  &  &  &  & $\bullet$ & $\bullet$ & $\bullet$ & $\bullet$ &  &  &  &  &  &  &  &  & $\bullet$ & $\bullet$ & $\bullet$ & $\bullet$ &  &  &  &  &  \\
Campionamento popolazioni &  &  &  &  &  & $\bullet$ & $\bullet$ & $\bullet$ & $\bullet$ &  &  &  &  &  &  &  &  & $\circ$ & $\circ$ & $\circ$ & $\circ$ &  &  &  &  &  &  &  &  & $\circ$ & $\circ$ & $\circ$ & $\circ$ &  &  &  &  &  \\
\rowcolor[HTML]{EFEFEF} Monitoraggio degli habitat &  &  &  &  &  & $\bullet$ & $\bullet$ & $\bullet$ & $\bullet$ &  &  &  &  &  &  &  &  & $\bullet$ & $\bullet$ & $\bullet$ & $\bullet$ &  &  &  &  &  &  &  &  & $\bullet$ & $\bullet$ & $\bullet$ & $\bullet$ &  &  &  &  &  \\
Caratterizzazione genetica &  &  &  &  &  &  &  &  &  &  &  &  &  &  &  &  &  &  &  &  &  &  &  &  &  &  &  &  &  &  &  &  &  &  &  &  &  &  \\
Eradicazione popolazioni alloctone &  &  &  &  &  & $\circ$ & $\circ$ & $\circ$ & $\circ$ &  &  &  &  &  &  &  &  & $\circ$ & $\circ$ & $\circ$ & $\circ$ &  &  &  &  &  &  &  &  & $\circ$ & $\circ$ & $\circ$ & $\circ$ &  &  &  &  &  \\
\rowcolor[HTML]{EFEFEF} Controllo diffusione alloctone &  &  &  &  &  & $\bullet$ & $\bullet$ & $\bullet$ & $\bullet$ &  &  &  &  &  &  &  &  & $\circ$ & $\circ$ & $\circ$ & $\circ$ &  &  &  &  &  &  &  &  & $\circ$ & $\circ$ & $\circ$ & $\circ$ &  &  &  &  &  \\
Caratterizzazione genetica di nuove popolazioni rilevate &  &  &  &  &  &  &  &  &  &  &  &  &  & $\circ$ & $\circ$ & $\circ$ & $\circ$ &  &  &  &  &  &  &  &  & $\circ$ & $\circ$ & $\circ$ & $\circ$ &  &  &  &  &  &  &  &  &  \\
\rowcolor[HTML]{EFEFEF} Gestione database e sito web &  &  &  &  &  &  &  &  &  & $\diamond$ & $\diamond$ & $\diamond$ & $\diamond$ & $\diamond$ &  &  &  &  &  &  &  & $\diamond$ & $\diamond$ & $\diamond$ & $\diamond$ & $\diamond$ &  &  &  &  &  &  &  & $\diamond$ & $\diamond$ & $\diamond$ &  &  \\
Analisi dei dati &  &  &  &  &  &  &  &  &  & $\bullet$ & $\bullet$ & $\bullet$ &  &  &  &  &  &  &  &  &  & $\bullet$ & $\bullet$ & $\bullet$ &  &  &  &  &  &  &  &  &  & $\bullet$ & $\bullet$ & $\bullet$ &  &  \\
\rowcolor[HTML]{EFEFEF} Formazione e sensibilizzazione di pescatori & $\diamond$ & $\diamond$ & $\diamond$ & $\diamond$ & $\diamond$ & $\diamond$ & $\diamond$ & $\diamond$ & $\diamond$ & $\diamond$ & $\diamond$ & $\diamond$ & $\diamond$ & $\diamond$ & $\diamond$ & $\diamond$ & $\diamond$ & $\diamond$ & $\diamond$ & $\diamond$ & $\diamond$ & $\diamond$ & $\diamond$ & $\diamond$ & $\diamond$ & $\diamond$ & $\diamond$ & $\diamond$ & $\diamond$ & $\diamond$ & $\diamond$ & $\diamond$ & $\diamond$ & $\diamond$ & $\diamond$ & $\diamond$ &  &  \\
Collaborazione con gestori di laghetti da pesca e pescicolture, soprattutto se ospitano popolazioni di \emph{A. pallipes} & $\diamond$ & $\diamond$ & $\diamond$ & $\diamond$ & $\diamond$ & $\diamond$ & $\diamond$ & $\diamond$ & $\diamond$ & $\diamond$ & $\diamond$ & $\diamond$ & $\diamond$ & $\diamond$ & $\diamond$ & $\diamond$ & $\diamond$ & $\diamond$ & $\diamond$ & $\diamond$ & $\diamond$ & $\diamond$ & $\diamond$ & $\diamond$ & $\diamond$ & $\diamond$ & $\diamond$ & $\diamond$ & $\diamond$ & $\diamond$ & $\diamond$ & $\diamond$ & $\diamond$ & $\diamond$ & $\diamond$ & $\diamond$ &  &  \\
\rowcolor[HTML]{EFEFEF} Sensibilizzazione di gestori e  fruitori della risorsa idrica & $\diamond$ & $\diamond$ & $\diamond$ & $\diamond$ & $\diamond$ & $\diamond$ & $\diamond$ & $\diamond$ & $\diamond$ & $\diamond$ & $\diamond$ & $\diamond$ & $\diamond$ & $\diamond$ & $\diamond$ & $\diamond$ & $\diamond$ & $\diamond$ & $\diamond$ & $\diamond$ & $\diamond$ & $\diamond$ & $\diamond$ & $\diamond$ & $\diamond$ & $\diamond$ & $\diamond$ & $\diamond$ & $\diamond$ & $\diamond$ & $\diamond$ & $\diamond$ & $\diamond$ & $\diamond$ & $\diamond$ & $\diamond$ &  &  \\
Divulgazione al pubblico generico &  &  &  &  &  & $\bullet$ & $\bullet$ & $\bullet$ & $\bullet$ &  &  &  &  &  &  &  &  & $\bullet$ & $\bullet$ & $\bullet$ & $\bullet$ &  &  &  &  &  &  &  &  & $\bullet$ & $\bullet$ & $\bullet$ & $\bullet$ &  &  &  &  &  \\
\rowcolor[HTML]{EFEFEF} Divulgazione scientifica & $\diamond$ & $\diamond$ & $\diamond$ & $\diamond$ & $\diamond$ & $\diamond$ & $\diamond$ & $\diamond$ & $\diamond$ & $\diamond$ & $\diamond$ & $\diamond$ & $\diamond$ & $\diamond$ & $\diamond$ & $\diamond$ & $\diamond$ & $\diamond$ & $\diamond$ & $\diamond$ & $\diamond$ & $\diamond$ & $\diamond$ & $\diamond$ & $\diamond$ & $\diamond$ & $\diamond$ & $\diamond$ & $\diamond$ & $\diamond$ & $\diamond$ & $\diamond$ & $\diamond$ & $\diamond$ & $\diamond$ & $\diamond$ &  &  \\
Contenimento di patogeni & $\diamond$ & $\diamond$ & $\diamond$ & $\diamond$ & $\diamond$ & $\diamond$ & $\diamond$ & $\diamond$ & $\diamond$ & $\diamond$ & $\diamond$ & $\diamond$ & $\diamond$ & $\diamond$ & $\diamond$ & $\diamond$ & $\diamond$ & $\diamond$ & $\diamond$ & $\diamond$ & $\diamond$ & $\diamond$ & $\diamond$ & $\diamond$ & $\diamond$ & $\diamond$ & $\diamond$ & $\diamond$ & $\diamond$ & $\diamond$ & $\diamond$ & $\diamond$ & $\diamond$ & $\diamond$ & $\diamond$ & $\diamond$ &  &  \\
\rowcolor[HTML]{EFEFEF} Allevamenti: individuazione siti e creazione impianti &  &  &  &  &  &  &  &  &  & $\bullet$ & $\bullet$ & $\bullet$ & $\bullet$ & $\bullet$ & $\bullet$ &  &  &  &  &  &  &  &  &  &  &  &  &  &  &  &  &  &  &  &  &  &  &  \\
Allevamenti: raccolta riproduttori &  &  &  &  &  &  &  &  &  &  &  &  &  &  &  &  &  &  &  & $\bullet$ & $\bullet$ & $\bullet$ &  &  &  &  &  &  &  &  &  & $\bullet$ & $\bullet$ & $\bullet$ &  &  &  &  \\
\rowcolor[HTML]{EFEFEF} Ripopolamenti e reintroduzioni &  &  &  &  &  &  &  &  &  &  &  &  &  &  &  &  &  &  &  &  & $\bullet$ & $\bullet$ &  &  &  &  &  &  &  &  &  &  & $\bullet$ & $\bullet$ &  &  &  &  \\
Rinaturalizzazione dei corpi idrici effettivamente occupati o potenzialmente idonei alla diffusione di \emph{A. pallipes}: interventi caso-specifici di rinaturalizzazione, ripristino della connettività longitudinale ,creazione ambienti umidi, creazione di aree rifugio in ambienti acquaici laterali. & $\diamond$ & $\diamond$ & $\diamond$ & $\diamond$ & $\diamond$ & $\diamond$ & $\diamond$ & $\diamond$ & $\diamond$ & $\diamond$ & $\diamond$ & $\diamond$ & $\diamond$ & $\diamond$ & $\diamond$ & $\diamond$ & $\diamond$ & $\diamond$ & $\diamond$ & $\diamond$ & $\diamond$ & $\diamond$ & $\diamond$ & $\diamond$ & $\diamond$ & $\diamond$ & $\diamond$ & $\diamond$ & $\diamond$ & $\diamond$ & $\diamond$ & $\diamond$ & $\diamond$ & $\diamond$ & $\diamond$ & $\diamond$ \\
\rowcolor[HTML]{EFEFEF} Gestione sostenibile della vegetazione acquatica e riparia negli ambienti in cui sono previsti tagli periodici & $\diamond$ & $\diamond$ & $\diamond$ & $\diamond$ & $\diamond$ & $\diamond$ & $\diamond$ & $\diamond$ & $\diamond$ & $\diamond$ & $\diamond$ & $\diamond$ & $\diamond$ & $\diamond$ & $\diamond$ & $\diamond$ & $\diamond$ & $\diamond$ & $\diamond$ & $\diamond$ & $\diamond$ & $\diamond$ & $\diamond$ & $\diamond$ & $\diamond$ & $\diamond$ & $\diamond$ & $\diamond$ & $\diamond$ & $\diamond$ & $\diamond$ & $\diamond$ & $\diamond$ & $\diamond$ & $\diamond$ & $\diamond$ &  &  \\
Mitigazione di inquinamento diffuso: creazione di una zona filtro tra corpi idrici e ambiente acquatico. & $\diamond$ & $\diamond$ & $\diamond$ & $\diamond$ & $\diamond$ & $\diamond$ & $\diamond$ & $\diamond$ & $\diamond$ & $\diamond$ & $\diamond$ & $\diamond$ & $\diamond$ & $\diamond$ & $\diamond$ & $\diamond$ & $\diamond$ & $\diamond$ & $\diamond$ & $\diamond$ & $\diamond$ & $\diamond$ & $\diamond$ & $\diamond$ & $\diamond$ & $\diamond$ & $\diamond$ & $\diamond$ & $\diamond$ & $\diamond$ & $\diamond$ & $\diamond$ & $\diamond$ & $\diamond$ & $\diamond$ & $\diamond$ &  &  \\
\rowcolor[HTML]{EFEFEF} Eliminazione di fonti d'inquinamento puntuali. & $\diamond$ & $\diamond$ & $\diamond$ & $\diamond$ & $\diamond$ & $\diamond$ & $\diamond$ & $\diamond$ & $\diamond$ & $\diamond$ & $\diamond$ & $\diamond$ & $\diamond$ & $\diamond$ & $\diamond$ & $\diamond$ & $\diamond$ & $\diamond$ & $\diamond$ & $\diamond$ & $\diamond$ & $\diamond$ & $\diamond$ & $\diamond$ & $\diamond$ & $\diamond$ & $\diamond$ & $\diamond$ & $\diamond$ & $\diamond$ & $\diamond$ & $\diamond$ & $\diamond$ & $\diamond$ & $\diamond$ & $\diamond$ &  &  \\
Studi di cattura-marcatura-ricattura: valutazione di parametri demografici in popolazioni selezionate & $\diamond$ & $\diamond$ & $\diamond$ & $\diamond$ & $\diamond$ & $\diamond$ & $\diamond$ & $\diamond$ & $\diamond$ & $\diamond$ & $\diamond$ & $\diamond$ & $\diamond$ & $\diamond$ & $\diamond$ & $\diamond$ & $\diamond$ & $\diamond$ & $\diamond$ & $\diamond$ & $\diamond$ & $\diamond$ & $\diamond$ & $\diamond$ & $\diamond$ & $\diamond$ & $\diamond$ & $\diamond$ & $\diamond$ & $\diamond$ & $\diamond$ & $\diamond$ & $\diamond$ & $\diamond$ & $\diamond$ & $\diamond$ &  &  \\ \bottomrule
\end{longtable}
\end{landscape}

\end{document}